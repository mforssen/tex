% \iffalse meta-comment
%
% File: luatex.dtx
% Version: 2010/03/09 v0.4
% Info: LuaTeX basic definition package
%
% Copyright (C) 2007, 2009, 2010 by
%    Heiko Oberdiek <heiko.oberdiek at googlemail.com>
%
% This work may be distributed and/or modified under the
% conditions of the LaTeX Project Public License, either
% version 1.3c of this license or (at your option) any later
% version. This version of this license is in
%    http://www.latex-project.org/lppl/lppl-1-3c.txt
% and the latest version of this license is in
%    http://www.latex-project.org/lppl.txt
% and version 1.3 or later is part of all distributions of
% LaTeX version 2005/12/01 or later.
%
% This work has the LPPL maintenance status "maintained".
%
% This Current Maintainer of this work is Heiko Oberdiek.
%
% The Base Interpreter refers to any `TeX-Format',
% because some files are installed in TDS:tex/generic//.
%
% This work consists of the main source file luatex.dtx
% and the derived files
%    luatex.sty, luatex.pdf, luatex.ins, luatex.drv, luatex-loader.sty,
%    luatex-test1.tex, luatex-test2.tex, luatex-test3.tex,
%    luatex-test4.tex, luatex-test5.tex, oberdiek.luatex.lua.
%
% Distribution:
%    CTAN:macros/latex/contrib/oberdiek/luatex.dtx
%    CTAN:macros/latex/contrib/oberdiek/luatex.pdf
%
% Unpacking:
%    (a) If luatex.ins is present:
%           tex luatex.ins
%    (b) Without luatex.ins:
%           tex luatex.dtx
%    (c) If you insist on using LaTeX
%           latex \let\install=y% \iffalse meta-comment
%
% File: luatex.dtx
% Version: 2010/03/09 v0.4
% Info: LuaTeX basic definition package
%
% Copyright (C) 2007, 2009, 2010 by
%    Heiko Oberdiek <heiko.oberdiek at googlemail.com>
%
% This work may be distributed and/or modified under the
% conditions of the LaTeX Project Public License, either
% version 1.3c of this license or (at your option) any later
% version. This version of this license is in
%    http://www.latex-project.org/lppl/lppl-1-3c.txt
% and the latest version of this license is in
%    http://www.latex-project.org/lppl.txt
% and version 1.3 or later is part of all distributions of
% LaTeX version 2005/12/01 or later.
%
% This work has the LPPL maintenance status "maintained".
%
% This Current Maintainer of this work is Heiko Oberdiek.
%
% The Base Interpreter refers to any `TeX-Format',
% because some files are installed in TDS:tex/generic//.
%
% This work consists of the main source file luatex.dtx
% and the derived files
%    luatex.sty, luatex.pdf, luatex.ins, luatex.drv, luatex-loader.sty,
%    luatex-test1.tex, luatex-test2.tex, luatex-test3.tex,
%    luatex-test4.tex, luatex-test5.tex, oberdiek.luatex.lua.
%
% Distribution:
%    CTAN:macros/latex/contrib/oberdiek/luatex.dtx
%    CTAN:macros/latex/contrib/oberdiek/luatex.pdf
%
% Unpacking:
%    (a) If luatex.ins is present:
%           tex luatex.ins
%    (b) Without luatex.ins:
%           tex luatex.dtx
%    (c) If you insist on using LaTeX
%           latex \let\install=y% \iffalse meta-comment
%
% File: luatex.dtx
% Version: 2010/03/09 v0.4
% Info: LuaTeX basic definition package
%
% Copyright (C) 2007, 2009, 2010 by
%    Heiko Oberdiek <heiko.oberdiek at googlemail.com>
%
% This work may be distributed and/or modified under the
% conditions of the LaTeX Project Public License, either
% version 1.3c of this license or (at your option) any later
% version. This version of this license is in
%    http://www.latex-project.org/lppl/lppl-1-3c.txt
% and the latest version of this license is in
%    http://www.latex-project.org/lppl.txt
% and version 1.3 or later is part of all distributions of
% LaTeX version 2005/12/01 or later.
%
% This work has the LPPL maintenance status "maintained".
%
% This Current Maintainer of this work is Heiko Oberdiek.
%
% The Base Interpreter refers to any `TeX-Format',
% because some files are installed in TDS:tex/generic//.
%
% This work consists of the main source file luatex.dtx
% and the derived files
%    luatex.sty, luatex.pdf, luatex.ins, luatex.drv, luatex-loader.sty,
%    luatex-test1.tex, luatex-test2.tex, luatex-test3.tex,
%    luatex-test4.tex, luatex-test5.tex, oberdiek.luatex.lua.
%
% Distribution:
%    CTAN:macros/latex/contrib/oberdiek/luatex.dtx
%    CTAN:macros/latex/contrib/oberdiek/luatex.pdf
%
% Unpacking:
%    (a) If luatex.ins is present:
%           tex luatex.ins
%    (b) Without luatex.ins:
%           tex luatex.dtx
%    (c) If you insist on using LaTeX
%           latex \let\install=y% \iffalse meta-comment
%
% File: luatex.dtx
% Version: 2010/03/09 v0.4
% Info: LuaTeX basic definition package
%
% Copyright (C) 2007, 2009, 2010 by
%    Heiko Oberdiek <heiko.oberdiek at googlemail.com>
%
% This work may be distributed and/or modified under the
% conditions of the LaTeX Project Public License, either
% version 1.3c of this license or (at your option) any later
% version. This version of this license is in
%    http://www.latex-project.org/lppl/lppl-1-3c.txt
% and the latest version of this license is in
%    http://www.latex-project.org/lppl.txt
% and version 1.3 or later is part of all distributions of
% LaTeX version 2005/12/01 or later.
%
% This work has the LPPL maintenance status "maintained".
%
% This Current Maintainer of this work is Heiko Oberdiek.
%
% The Base Interpreter refers to any `TeX-Format',
% because some files are installed in TDS:tex/generic//.
%
% This work consists of the main source file luatex.dtx
% and the derived files
%    luatex.sty, luatex.pdf, luatex.ins, luatex.drv, luatex-loader.sty,
%    luatex-test1.tex, luatex-test2.tex, luatex-test3.tex,
%    luatex-test4.tex, luatex-test5.tex, oberdiek.luatex.lua.
%
% Distribution:
%    CTAN:macros/latex/contrib/oberdiek/luatex.dtx
%    CTAN:macros/latex/contrib/oberdiek/luatex.pdf
%
% Unpacking:
%    (a) If luatex.ins is present:
%           tex luatex.ins
%    (b) Without luatex.ins:
%           tex luatex.dtx
%    (c) If you insist on using LaTeX
%           latex \let\install=y\input{luatex.dtx}
%        (quote the arguments according to the demands of your shell)
%
% Documentation:
%    (a) If luatex.drv is present:
%           latex luatex.drv
%    (b) Without luatex.drv:
%           latex luatex.dtx; ...
%    The class ltxdoc loads the configuration file ltxdoc.cfg
%    if available. Here you can specify further options, e.g.
%    use A4 as paper format:
%       \PassOptionsToClass{a4paper}{article}
%
%    Programm calls to get the documentation (example):
%       pdflatex luatex.dtx
%       makeindex -s gind.ist luatex.idx
%       pdflatex luatex.dtx
%       makeindex -s gind.ist luatex.idx
%       pdflatex luatex.dtx
%
% Installation:
%    TDS:tex/generic/oberdiek/luatex.sty
%    TDS:tex/generic/oberdiek/luatex-loader.sty
%    TDS:scripts/oberdiek/oberdiek.luatex.lua
%    TDS:doc/latex/oberdiek/luatex.pdf
%    TDS:doc/latex/oberdiek/test/luatex-test1.tex
%    TDS:doc/latex/oberdiek/test/luatex-test2.tex
%    TDS:doc/latex/oberdiek/test/luatex-test3.tex
%    TDS:doc/latex/oberdiek/test/luatex-test4.tex
%    TDS:doc/latex/oberdiek/test/luatex-test5.tex
%    TDS:source/latex/oberdiek/luatex.dtx
%
%<*ignore>
\begingroup
  \catcode123=1 %
  \catcode125=2 %
  \def\x{LaTeX2e}%
\expandafter\endgroup
\ifcase 0\ifx\install y1\fi\expandafter
         \ifx\csname processbatchFile\endcsname\relax\else1\fi
         \ifx\fmtname\x\else 1\fi\relax
\else\csname fi\endcsname
%</ignore>
%<*install>
\input docstrip.tex
\Msg{************************************************************************}
\Msg{* Installation}
\Msg{* Package: luatex 2010/03/09 v0.4 LuaTeX basic definition package (HO)}
\Msg{************************************************************************}

\keepsilent
\askforoverwritefalse

\let\MetaPrefix\relax
\preamble

This is a generated file.

Project: luatex
Version: 2010/03/09 v0.4

Copyright (C) 2007, 2009, 2010 by
   Heiko Oberdiek <heiko.oberdiek at googlemail.com>

This work may be distributed and/or modified under the
conditions of the LaTeX Project Public License, either
version 1.3c of this license or (at your option) any later
version. This version of this license is in
   http://www.latex-project.org/lppl/lppl-1-3c.txt
and the latest version of this license is in
   http://www.latex-project.org/lppl.txt
and version 1.3 or later is part of all distributions of
LaTeX version 2005/12/01 or later.

This work has the LPPL maintenance status "maintained".

This Current Maintainer of this work is Heiko Oberdiek.

The Base Interpreter refers to any `TeX-Format',
because some files are installed in TDS:tex/generic//.

This work consists of the main source file luatex.dtx
and the derived files
   luatex.sty, luatex.pdf, luatex.ins, luatex.drv, luatex-loader.sty,
   luatex-test1.tex, luatex-test2.tex, luatex-test3.tex,
   luatex-test4.tex, luatex-test5.tex, oberdiek.luatex.lua.

\endpreamble
\let\MetaPrefix\DoubleperCent

\generate{%
  \file{luatex.ins}{\from{luatex.dtx}{install}}%
  \file{luatex.drv}{\from{luatex.dtx}{driver}}%
  \usedir{tex/generic/oberdiek}%
  \file{luatex.sty}{\from{luatex.dtx}{package}}%
  \file{luatex-loader.sty}{\from{luatex.dtx}{loader}}%
  \usedir{doc/latex/oberdiek/test}%
  \file{luatex-test1.tex}{\from{luatex.dtx}{test1}}%
  \file{luatex-test2.tex}{\from{luatex.dtx}{test1,test2}}%
  \file{luatex-test3.tex}{\from{luatex.dtx}{test1,test3}}%
  \file{luatex-test4.tex}{\from{luatex.dtx}{test4}}%
  \file{luatex-test5.tex}{\from{luatex.dtx}{test5}}%
  \nopreamble
  \nopostamble
  \usedir{source/latex/oberdiek/catalogue}%
  \file{luatex.xml}{\from{luatex.dtx}{catalogue}}%
}
\def\MetaPrefix{-- }
\def\defaultpostamble{%
  \MetaPrefix^^J%
  \MetaPrefix\space End of File `\outFileName'.%
}
\def\currentpostamble{\defaultpostamble}%
\generate{%
  \usedir{scripts/oberdiek}%
  \file{oberdiek.luatex.lua}{\from{luatex.dtx}{lua}}%
}

\catcode32=13\relax% active space
\let =\space%
\Msg{************************************************************************}
\Msg{*}
\Msg{* To finish the installation you have to move the following}
\Msg{* files into a directory searched by TeX:}
\Msg{*}
\Msg{*     luatex.sty, luatex-loader.sty}
\Msg{*}
\Msg{* And install the following script file:}
\Msg{*}
\Msg{*     oberdiek.luatex.lua}
\Msg{*}
\Msg{* To produce the documentation run the file `luatex.drv'}
\Msg{* through LaTeX.}
\Msg{*}
\Msg{* Happy TeXing!}
\Msg{*}
\Msg{************************************************************************}

\endbatchfile
%</install>
%<*ignore>
\fi
%</ignore>
%<*driver>
\NeedsTeXFormat{LaTeX2e}
\ProvidesFile{luatex.drv}%
  [2010/03/09 v0.4 LuaTeX basic definition package (HO)]%
\documentclass{ltxdoc}
\usepackage{holtxdoc}[2011/11/22]
\usepackage{paralist}
\begin{document}
  \DocInput{luatex.dtx}%
\end{document}
%</driver>
% \fi
%
% \CheckSum{1429}
%
% \CharacterTable
%  {Upper-case    \A\B\C\D\E\F\G\H\I\J\K\L\M\N\O\P\Q\R\S\T\U\V\W\X\Y\Z
%   Lower-case    \a\b\c\d\e\f\g\h\i\j\k\l\m\n\o\p\q\r\s\t\u\v\w\x\y\z
%   Digits        \0\1\2\3\4\5\6\7\8\9
%   Exclamation   \!     Double quote  \"     Hash (number) \#
%   Dollar        \$     Percent       \%     Ampersand     \&
%   Acute accent  \'     Left paren    \(     Right paren   \)
%   Asterisk      \*     Plus          \+     Comma         \,
%   Minus         \-     Point         \.     Solidus       \/
%   Colon         \:     Semicolon     \;     Less than     \<
%   Equals        \=     Greater than  \>     Question mark \?
%   Commercial at \@     Left bracket  \[     Backslash     \\
%   Right bracket \]     Circumflex    \^     Underscore    \_
%   Grave accent  \`     Left brace    \{     Vertical bar  \|
%   Right brace   \}     Tilde         \~}
%
% \GetFileInfo{luatex.drv}
%
% \title{The \xpackage{luatex} package}
% \date{2010/03/09 v0.4}
% \author{Heiko Oberdiek\\\xemail{heiko.oberdiek at googlemail.com}}
%
% \maketitle
%
% \begin{abstract}
% This package manages the new and extended features and resources
% that \LuaTeX\ provides. Examples are attributes and catcode
% tables.
% \end{abstract}
%
% \tableofcontents
%
% \section{Documentation}
%
% \subsection{Introduction}
%
% \TeX\ provides global resources such as registers.
% But it does not provide an interface for managing these resources.
% For example, two packages want to use a counter register.
% If they take the same register
% number, then the use of both packages will conflict and they cannot
% be used together.
% Therefore formats such as \plainTeX\ or \LaTeX\ implement
% an allocation scheme for registers. A package reserves with
% \cs{newcount} an unused register number for its own exclusive use.
%
% Nowadays \TeX\ is not alone anymore: \eTeX, \pdfTeX\ and other
% compilers for \TeX\ are developed that extend and add new
% features and resources.
%
% Now \LuaTeX\ has reached beta state. It inherits most of
% \pdfTeX's features including \eTeX. Also it implements
% new concepts such as attributes or catcode tables.
%
% \subsubsection{\LaTeX}
%
% \LaTeXe\ is frozen and therefore refuses to even notice
% the new \TeX\ variants. Not even the old \eTeX\ is supported
% by its kernel. At least there is a third party package \xpackage{etex}
% that manages the new \eTeX\ resources.
%
% This package tries to do the same for \LuaTeX\ and starts to
% support at least a few of the new features.
%
% \subsubsection{\plainTeX}
%
% \LaTeX\ has inherited its resource handling from \plainTeX.
% The interface is basically the same: \cs{newcount}, \dots
% Therefore this package tries to follow this tradition by
% providing compatibility to \plainTeX. It can be loaded
% with \plainTeX\ and defines at least some of the
% features that this packages provides for \LaTeX.
%
% \subsection{Register allocation}
%
% \subsubsection{Register with 16 bit}
%
% Because \LuaTeX\ is a super set of \eTeX\ regarding registers,
% the register allocation scheme should not conflict with
% package \xpackage{etex}. Therefore this package is loaded
% to inherit its allocation scheme. The only change is currently
% that the limit is increased to 65536 registers for the following
% register classes:
% \begin{compactitem}
% \item count
% \item dimen
% \item skip
% \item muskip
% \item marks
% \item toks
% \item box
% \end{compactitem}
% This affects the number of global and local registers.
% Because it is done in a package and not in the kernel,
% it is possible that someone loads package \xpackage{etex}
% before uses the local allocation variants. This will prevent
% the extension for this register class. If more registers are
% needed, just load package \xpackage{luatex} earlier.
%
% \subsubsection{Insertions}
%
% Insertions need four registers \cs{count}, \cs{dimen}, \cs{skip},
% and \cs{box} with the same number. Usually they are allocated
% downwards from 254, 253, \dots Also \cs{newcount}, \cs{newdimen}, \dots
% fill up these register numbers from below before switching to
% higher register numbers by package \xpackage{etex}.
% When this occurs, no insertions can be allocated anymore.
%
% Therefore \cs{newcount}, \cs{newdimen}, \cs{newskip}, and
% \cs{newbox} are replaced by their global variants (\cs{globcount}, \dots)
% that use the higher numbers immediately, leaving the room for insertions.
% There should not be an efficiency penalty because \LuaTeX\ stores
% the registers of a class in the same Lua table unlike \eTeX, where
% registers below 256 are stored in an array and higher numbers
% are put in a tree structure.
%
% \subsection{Attributes}
%
% Nodes can have custom attributes in \LuaTeX. These attributes
% are organized by a new register class. As the other registers
% up to $2^{16}$ attributes are supported. An attribute value
% can be negative that means the attribute is not set. Otherwise
% \TeX's range of non-negative integers up to $2^31$ are available.
%
% \begin{declcs}{newattribute} \M{cmd}
% \end{declcs}
% Macro \cs{newattribute} defines command \meta{cmd} using \cs{attributedef}
% using an new attribute number. The new attribute is initially unset.
%
% \begin{declcs}{setattribute} \M{cmd} \M{value}
% \end{declcs}
% Macro \cs{setattribute} locally sets attribute command \meta{cmd}
% to the number \meta{value}. Valid values range from $-1$ until $2^31$
% (the upper limit is the same as for other \TeX\ integer numbers).
%
% \begin{declcs}{unsetattribute} \M{cmd}
% \end{declcs}
% Macro \cs{unsettattribute} clears the attribute command \meta{cmd}.
%
% \subsection{Catcode tables}
%
% \LuaTeX\ introduces catcode tables as new feature, see documentation.
% There is need for discussion, how to deal best:
% \begin{itemize}
% \item \cs{initcatcodetable} and \cs{setcatcodetable} act globally.
% \item \cs{catcodetable} causes an error if used with an uninitialized
%   catcode table.
% \item Large catcode table numbers should be avoided because of
%   performance breakdown.
% \item Use case \LaTeX\ package:
%   The package must not be surprised by changed catcodes and
%   must not surprise by changing catcodes accidently.
%   Catcode tables could offer a solution. At the begin
%   a catcode regime with standard catcodes is established
%   and the old one is restored afterwards.
% \item Use case: \LuaTeX's |tex.print| might be used with
%   a catcode table number, for example a table where all
%   entries have catcode ``other''.
% \item Readonly catcode tables.
% \item Is there is a need for local allocations?
%   (Package \xpackage{etex}'s \cs{loc} variants are not used in
%   \TeX\ Live 2007.)
% \end{itemize}
%
% \subsubsection{Interface proposal}
%
% The idea: \cs{newcatcodetable} allocates odd numbered catcode
% tables. Even numbered tables are managed as stack.
% Also some catcode tables are defined. These must not be
% changed.
%
% \begin{declcs}{newcatcodetable} \M{cmd}
% \end{declcs}
% Macro \cs{newcatcodetable} reserves a new catcode table and
% remembers its number in \meta{cmd}. The catcode table is
% initialized with \iniTeX's catcodes.
%
% \begin{declcs}{CatcodeTableIniTeX}\\
%   \cs{CatcodeTableString}\\
%   \cs{CatcodeTableOther}\\
%   \cs{CatcodeTableLaTeX}\\
% \end{declcs}
% These are catcode tables and must not be changed.
% \cs{CatcodeTableIniTeX} contains the catcode settings of \iniTeX.
% \cs{CatcodeTableString} follows \TeX's convention of
% \cs{string}, \cs{meaning} and friends. The space gets catcode
% 10 (space), the other characters have catcode 12 (other).
% In \cs{CatcodeTableOther} all entries have catcode 12 (other).
% \cs{CatcodeTableLaTeX} contains the setting of a pure \LaTeX\ format
% (`at' is other).
%
% \begin{declcs}{CatcodeTableStack}\\
%   \cs{IncCatcodeTableStack}\\
%   \cs{DecCatcodeTableStack}
% \end{declcs}
% \cs{CatcodeTableStack} is the stack pointer. Initially
% it is catcode table zero. \cs{IncCatcodeTableStack} and
% \cs{DecCatcodeTableStack} increments and decrements the
% stack pointer. Currently \cs{IncCatcodeTableStack} does not
% initialize a new catcode table. Both increment and decrement
% operations do not set a catcode table.
%
% \begin{declcs}{PushCatcodeTableNumStack}\\
%   \cs{PopCatcodeTableNumStack}
% \end{declcs}
% It can be handy to have a global stack for catcode table
% numbers to deal with the global assignment property of
% \cs{initcatcodetable} and \cs{savecatcodetable}.
% \cs{PushCatcodeTableNumStack} pushes the current catcode table
% on the stack.
% \cs{PopCatcodeTableNumStack} pops the topmost number off the number
% stack to set the current catcode table. Catcode table zero is
% used in case of an empty stack.
%
% \begin{declcs}{BeginCatcodeRegime} \M{catcodetable}\\
%   \cs{EndCatcodeRegime}
% \end{declcs}
% \cs{BeginCatcodeRegime} remembers the current catcode table number.
% Then it creates and uses a fresh catcode table on the stack
% that is initialized by \meta{catcodetable}:
% \begin{quote}\ttfamily
%   \cs{PushCatcodeTableNumStack}\\
%   \cs{catcodetable}\meta{catcodetable}||
%   \cs{IncCatcodeTableStack}\\
%   \cs{savecatcodetable}\cs{CatcodeTableStack}\\
%   \cs{catcodetable}\cs{CatcodeTableStack}
% \end{quote}
% \cs{EndCatcodeRegime} drops the catcode table, created by
% \cs{BeginCatcodeRegime} and sets the catcode table that was
% active before:
% \begin{quote}
%   \cs{DecCatcodeTableStack}\\
%   \cs{PopCatcodeTableNumStack}
% \end{quote}
% These macros solve the use case, described earlier for a \LaTeX\ package:
% \begin{quote}
%   |% package foobar.sty|\\
%   |\BeginCatcodeRegime\CatcodeTableLaTeX|\\
%   |\makeatletter|\\
%   |% ... package contents ...|\\
%   |\EndCatcodeRegime|\\
%   |% end of package|
% \end{quote}
% If the package wants to change catcodes after its loading,
% \cs{AtBeginDocument} or \cs{AtEndOfPackage} can be used.
%
% \begin{declcs}{SetCatcodeRange} \M{from} \M{to} \M{catcode}
% \end{declcs}
% The catcodes of characters in range from \meta{from}
% to inclusive \meta{to} are set to \meta{catcode}.
%
% \subsection{Lua module loading}
%
% Currently \LuaTeX\ (version 0.20) does not support Lua script
% files inside \xfile{TDS:scripts//}, because Lua's mechanism
% for module loading does not use the \textsf{kpathsea} library.
% Therefore this packages appends a kpse loader to the list of
% Lua's module loaders. It finds the module \meta{module} by
% \begin{quote}
%   \ttfamily |kpse.find_file("|\meta{module}|.lua", "texmfscripts")|
% \end{quote}
% Unhappily \textsf{kpathsea} does not support directory components
% in a file name. Therefore the Lua convention is not followed to
% replace dots in the module name by the directory separator.
%
% Example:
% A Lua script of a package \xpackage{foobar} wants the following
% modules:
% \begin{quote}
%   |require("foobar.hello.world")|\\
%   |require("org.somewhere.xyz")|
% \end{quote}
% Then they can be find in:
% \begin{quote}
%   |TDS:scripts/foobar/foobar.hello.world.lua|\\
%   |TDS:scripts/foobar/org.somewhere.xyz.lua|
% \end{quote}
% I would have preferred the following locations, following lua conventions,
% e.\,g.:
% \begin{quote}
%   |TDS:scripts/foobar/hello/world.lua|\\
%   |TDS:scripts/foobar/org/somewhere/xyz.lua|
% \end{quote}
% But I do not know, how to achieve this in a reliable way using
% \textsf{kpathsea}.
%
% \subsubsection{Package \xpackage{luatex-loader}}
%
% If someone do not need or want package \xpackage{luatex} but
% it's extension for module loading, then he can use
% package \xpackage{luatex-loader}. Both \plainTeX\ and \LaTeX\ are
% supported.
%
% \StopEventually{
% }
%
% \section{Implementation}
%
%    \begin{macrocode}
%<*package>
%    \end{macrocode}
%
% \subsection{Reload check and package identification}
%    Reload check, especially if the package is not used with \LaTeX.
%    \begin{macrocode}
\begingroup\catcode61\catcode48\catcode32=10\relax%
  \catcode13=5 % ^^M
  \endlinechar=13 %
  \catcode35=6 % #
  \catcode39=12 % '
  \catcode44=12 % ,
  \catcode45=12 % -
  \catcode46=12 % .
  \catcode58=12 % :
  \catcode64=11 % @
  \catcode123=1 % {
  \catcode125=2 % }
  \expandafter\let\expandafter\x\csname ver@luatex.sty\endcsname
  \ifx\x\relax % plain-TeX, first loading
  \else
    \def\empty{}%
    \ifx\x\empty % LaTeX, first loading,
      % variable is initialized, but \ProvidesPackage not yet seen
    \else
      \expandafter\ifx\csname PackageInfo\endcsname\relax
        \def\x#1#2{%
          \immediate\write-1{Package #1 Info: #2.}%
        }%
      \else
        \def\x#1#2{\PackageInfo{#1}{#2, stopped}}%
      \fi
      \x{luatex}{The package is already loaded}%
      \aftergroup\endinput
    \fi
  \fi
\endgroup%
%    \end{macrocode}
%    Package identification:
%    \begin{macrocode}
\begingroup\catcode61\catcode48\catcode32=10\relax%
  \catcode13=5 % ^^M
  \endlinechar=13 %
  \catcode35=6 % #
  \catcode39=12 % '
  \catcode40=12 % (
  \catcode41=12 % )
  \catcode44=12 % ,
  \catcode45=12 % -
  \catcode46=12 % .
  \catcode47=12 % /
  \catcode58=12 % :
  \catcode64=11 % @
  \catcode91=12 % [
  \catcode93=12 % ]
  \catcode123=1 % {
  \catcode125=2 % }
  \expandafter\ifx\csname ProvidesPackage\endcsname\relax
    \def\x#1#2#3[#4]{\endgroup
      \immediate\write-1{Package: #3 #4}%
      \xdef#1{#4}%
    }%
  \else
    \def\x#1#2[#3]{\endgroup
      #2[{#3}]%
      \ifx#1\@undefined
        \xdef#1{#3}%
      \fi
      \ifx#1\relax
        \xdef#1{#3}%
      \fi
    }%
  \fi
\expandafter\x\csname ver@luatex.sty\endcsname
\ProvidesPackage{luatex}%
  [2010/03/09 v0.4 LuaTeX basic definition package (HO)]%
%    \end{macrocode}
%
% \subsection{Catcodes}
%
%    \begin{macrocode}
\begingroup\catcode61\catcode48\catcode32=10\relax%
  \catcode13=5 % ^^M
  \endlinechar=13 %
  \catcode123=1 % {
  \catcode125=2 % }
  \catcode64=11 % @
  \def\x{\endgroup
    \expandafter\edef\csname LuT@AtEnd\endcsname{%
      \endlinechar=\the\endlinechar\relax
      \catcode13=\the\catcode13\relax
      \catcode32=\the\catcode32\relax
      \catcode35=\the\catcode35\relax
      \catcode61=\the\catcode61\relax
      \catcode64=\the\catcode64\relax
      \catcode123=\the\catcode123\relax
      \catcode125=\the\catcode125\relax
    }%
  }%
\x\catcode61\catcode48\catcode32=10\relax%
\catcode13=5 % ^^M
\endlinechar=13 %
\catcode35=6 % #
\catcode64=11 % @
\catcode123=1 % {
\catcode125=2 % }
\def\TMP@EnsureCode#1#2{%
  \edef\LuT@AtEnd{%
    \LuT@AtEnd
    \catcode#1=\the\catcode#1\relax
  }%
  \catcode#1=#2\relax
}
\TMP@EnsureCode{10}{12}% ^^J
\TMP@EnsureCode{34}{12}% "
\TMP@EnsureCode{36}{3}% $
\TMP@EnsureCode{39}{12}% '
\TMP@EnsureCode{40}{12}% (
\TMP@EnsureCode{41}{12}% )
\TMP@EnsureCode{42}{12}% *
\TMP@EnsureCode{43}{12}% +
\TMP@EnsureCode{44}{12}% ,
\TMP@EnsureCode{45}{12}% -
\TMP@EnsureCode{46}{12}% .
\TMP@EnsureCode{47}{12}% /
\TMP@EnsureCode{60}{12}% <
\TMP@EnsureCode{62}{12}% >
\TMP@EnsureCode{91}{12}% [
\TMP@EnsureCode{93}{12}% ]
\TMP@EnsureCode{95}{12}% _ (other!)
\TMP@EnsureCode{96}{12}% `
\edef\LuT@AtEnd{\LuT@AtEnd\noexpand\endinput}
%    \end{macrocode}
%
% \subsection{Check for \LuaTeX}
%
%    Without \LuaTeX\ there is no point in using this package.
%    \begin{macrocode}
\begingroup\expandafter\expandafter\expandafter\endgroup
\expandafter\ifx\csname RequirePackage\endcsname\relax
  \input infwarerr.sty\relax
  \input ifluatex.sty\relax
\else
  \RequirePackage{infwarerr}[2007/09/09]%
  \RequirePackage{ifluatex}[2009/04/10]%
\fi
%    \end{macrocode}
%    \begin{macrocode}
\ifluatex
\else
  \@PackageError{luatex}{%
    This package may only be run using LuaTeX%
  }\@ehc
  \expandafter\LuT@AtEnd
\fi%
%    \end{macrocode}
%
% \subsection{Provide Lua\TeX\ primitives}
%
%    \begin{macrocode}
\ifnum\luatexversion<36 %
  \def\LuT@MakePrimitive#1{%
    \expandafter\let\csname luatex#1\expandafter\endcsname
    \csname #1\endcsname
  }%
\else
  \def\LuT@MakeLuatexPrimitive#1{%
    \begingroup\expandafter\expandafter\expandafter\endgroup
    \expandafter\ifx\csname luatex#1\endcsname\relax
      \begingroup\expandafter\expandafter\expandafter\endgroup
      \expandafter\ifx\csname #1\endcsname\relax
      \else
        \expandafter\let
        \csname luatex#1\expandafter\endcsname
        \csname #1\endcsname
      \fi
    \fi
    \begingroup\expandafter\expandafter\expandafter\endgroup
    \expandafter\ifx\csname luatex#1\endcsname\relax
      \begingroup
        \expandafter\let\csname luatex#1\endcsname\@undefined
        \ifnum0%
          \directlua{%
            if tex.enableprimitives then %
              tex.enableprimitives('luatex',{'#1'})%
              tex.print('1')%
            end%
          }%
          \expandafter\ifx\csname luatex#1\endcsname\relax\else1\fi
        =11 %
          \global\expandafter\let
          \csname luatex#1\expandafter\endcsname
          \csname luatex#1\endcsname
        \else
          \@PackageError{luatex}{%
            tex.enableprimitives failed for `#1'%
          }\@ehc
        \fi
      \endgroup
    \fi
  }%
  \def\LuT@MakePrimitive#1{%
    \begingroup\expandafter\expandafter\expandafter\endgroup
    \expandafter\ifx\csname#1\endcsname\relax
      \begingroup
        \expandafter\let\csname#1\endcsname\@undefined
        \ifnum0%
          \directlua{%
            if tex.enableprimitives then %
              tex.enableprimitives('',{'#1'})%
              tex.print('1')%
            end%
          }%
          \expandafter\ifx\csname#1\endcsname\relax\else1\fi
        =11 %
          \global\expandafter\let
          \csname#1\expandafter\endcsname
          \csname#1\endcsname
        \else
          \@PackageError{luatex}{%
            tex.enableprimitives failed for `#1'%
          }\@ehc
        \fi
      \endgroup
    \fi
  }%
\fi
\LuT@MakeLuatexPrimitive{attribute}
\LuT@MakeLuatexPrimitive{attributedef}
\LuT@MakeLuatexPrimitive{catcodetable}
\LuT@MakeLuatexPrimitive{initcatcodetable}
\LuT@MakeLuatexPrimitive{luaescapestring}
\LuT@MakeLuatexPrimitive{savecatcodetable}
\LuT@MakePrimitive{numexpr}
%    \end{macrocode}
%
% \subsection{Inherit support for \eTeX}
%
%    Package \xpackage{etex} is not compatible for \plainTeX.
%    But it could be present if a format is used that is based
%    on \xfile{etex.src}. Therefore we only load the package
%    in case of \LaTeX\ and tests its presence independently
%    of the format by looking for \cs{et@xins}.
%    \begin{macrocode}
\begingroup\expandafter\expandafter\expandafter\endgroup
\expandafter\ifx\csname RequirePackage\endcsname\relax
\else
  \RequirePackage{etex}[1998/03/26]%
\fi
%    \end{macrocode}
%
% \subsection{Adaption of \eTeX's register allocation}
%
%    \eTeX has increased the number of \TeX\ registers
%    from $2^8$ (256) to $2^{15}$ (32768) for a register class.
%    \LuaTeX\ extends the limit further to $2^16$ (65536).
%    The allocation scheme of package \xpackage{etex} is not
%    changed. But this can be subject for discussion.
%
%    If a register class hasn't registered any local registers
%    yet, then the limit can safely be pushed to 65536.
%    \begin{macrocode}
\begingroup\expandafter\expandafter\expandafter\endgroup
\expandafter\ifx\csname et@xins\endcsname\relax
  \@PackageWarningNoLine{luatex}{%
    Support for eTeX is not loaded (etex.src)%
  }%
\else
  \def\LuT@temp#1{%
    \ifnum\count27#1=32768 %
      \count27#1=65536 %
    \fi
  }%
  \LuT@temp0%
  \LuT@temp1%
  \LuT@temp2%
  \LuT@temp3%
  \LuT@temp4%
  \LuT@temp5%
  \LuT@temp6%
%    \end{macrocode}
%
% \eTeX\ uses an array for the first 256 registers and then a tree
% structure. \LuaTeX\ stores all registers of a class in one Lua table.
% There shouldn't be large performance differences.
% This allows starting immediately in the extended
% area, leaving room for insertions.
%    \begin{macrocode}
  \let\newcount\globcount
  \let\newdimen\globdimen
  \let\newskip\globskip
  \let\newbox\globbox
\fi
%    \end{macrocode}
%
% \subsection{\plainTeX\ compatibility}
%
%    \begin{macro}{\@empty}
%    \begin{macrocode}
\expandafter\ifx\csname @empty\endcsname\relax
  \def\@empty{}%
\fi
%    \end{macrocode}
%    \end{macro}
%    \begin{macro}{\@gobble}
%    \begin{macrocode}
\expandafter\ifx\csname @gobble\endcsname\relax
  \long\def\@gobble#1{}%
\fi
%    \end{macrocode}
%    \end{macro}
%    \begin{macro}{\@firstofone}
%    \begin{macrocode}
\expandafter\ifx\csname @firstofone\endcsname\relax
  \long\def\@firstofone#1{#1}%
\fi
%    \end{macrocode}
%    \end{macro}
%    \begin{macro}{\@firstoftwo}
%    \begin{macrocode}
\expandafter\ifx\csname @firstoftwo\endcsname\relax
  \long\def\@firstoftwo#1#2{#1}%
\fi
%    \end{macrocode}
%    \end{macro}
%    \begin{macro}{\@car}
%    \begin{macrocode}
\expandafter\ifx\csname @car\endcsname\relax
  \def\@car#1#2\@nil{#1}%
\fi
%    \end{macrocode}
%    \end{macro}
%    \begin{macro}{\@cdr}
%    \begin{macrocode}
\expandafter\ifx\csname @cdr\endcsname\relax
  \def\@cdr#1#2\@nil{#2}%
\fi
%    \end{macrocode}
%    \end{macro}
%    \begin{macro}{\@ifstar}
%    \begin{macrocode}
\expandafter\ifx\csname @ifstar\endcsname\relax
  \def\@ifstar#1{%
    \@ifnextchar*{\@firstoftwo{#1}}%
  }%
%    \end{macrocode}
%    \end{macro}
%    \begin{macro}{\@ifnextchar}
%    \begin{macrocode}
  \long\def\@ifnextchar#1#2#3{%
    \let\reserved@d=#1%
    \def\reserved@a{#2}%
    \def\reserved@b{#3}%
    \futurelet\@let@token\@ifnch
  }%
%    \end{macrocode}
%    \end{macro}
%    \begin{macro}{\@ifnch}
%    \begin{macrocode}
  \def\@ifnch{%
    \ifx\@let@token\@sptoken
      \let\reserved@c\@xifnch
    \else
      \ifx\@let@token\reserved@d
        \let\reserved@c\reserved@a
      \else
        \let\reserved@c\reserved@b
      \fi
    \fi
    \reserved@c
  }%
%    \end{macrocode}
%    \end{macro}
%    \begin{macro}{\@sptoken}
%    \begin{macrocode}
  \let\LuT@temp\:%
  \def\:{\let\@sptoken= }%
  \: % explicit space
%    \end{macrocode}
%    \end{macro}
%    \begin{macro}{\@xifnch}
%    \begin{macrocode}
  \def\:{\@xifnch}%
  \expandafter\def\: {%
    \futurelet\@let@token\@ifnch
  }%
  \let\:\LuT@temp
\fi
%    \end{macrocode}
%    \end{macro}
%
%    \begin{macro}{\@tempcnta}
%    \begin{macrocode}
\expandafter\ifx\csname @tempcnta\endcsname\relax
  \csname newcount\endcsname\@tempcnta
\fi
%    \end{macrocode}
%    \end{macro}
%
%    \begin{macro}{\@tempcntb}
%    \begin{macrocode}
\expandafter\ifx\csname @tempcntb\endcsname\relax
  \csname newcount\endcsname\@tempcntb
\fi
%    \end{macrocode}
%    \end{macro}
%
%    \begin{macro}{\LuT@newcommand}
%    \begin{macrocode}
\begingroup\expandafter\expandafter\expandafter\endgroup
\expandafter\ifx\csname newcommand\endcsname\relax
  \def\LuT@newcommand#1[#2]#3{%
    \ifx#1\@undefined
      \let#1\relax
    \else
      \ifx#1\relax
      \else
        \@PackageError{luatex}{%
          \string#1 is already defined.\MessageBreak
          Redefinition is skipped%
        }\@ehc
      \fi
    \fi
    \ifx#1\relax
      \ifcase#2 %
        \def#1{#3}%
      \or
        \def#1##1{#3}%
      \or
        \def#1##1##2{#3}%
      \or
        \def#1##1##2##3{#3}%
      \or
        \@INTERNAL@ERROR
      \fi
    \fi
  }%
\else
  \def\LuT@newcommand{\newcommand*}%
\fi
%    \end{macrocode}
%    \end{macro}
%
% \subsection{Attributes}
%
% \subsubsection{Allocation}
%
%    \begin{macro}{\LuT@AllocAttribute}
%    \begin{macrocode}
\newcount\LuT@AllocAttribute
\LuT@AllocAttribute=\m@ne
%    \end{macrocode}
%    \end{macro}
%    \begin{macro}{\newattribute}
%    \begin{macrocode}
\LuT@newcommand\newattribute[1]{%
  \ifnum\LuT@AllocAttribute<65535 %
    \global\advance\LuT@AllocAttribute\@ne
    \allocationnumber\LuT@AllocAttribute
    \global\luatexattributedef#1=\allocationnumber
    \unsetattribute{#1}%
    \wlog{\string#1=\string\attribute\the\allocationnumber}%
  \else
    \errmessage{No room for a new \string\attribute}%
  \fi
}
%    \end{macrocode}
%    \end{macro}
%
% \subsubsection{Interface}
%
%    \begin{macro}{\setattribute}
%    \begin{macrocode}
\LuT@newcommand\setattribute[2]{%
  #1=\numexpr#2\relax
}
%    \end{macrocode}
%    \end{macro}
%    \begin{macro}{\unsetattribute}
%    \begin{macrocode}
\ifnum\luatexversion<37
  \LuT@newcommand\LuT@UnsetAttributeValue[0]{}%
  \let\LuT@UnsetAttributeValue\m@ne
\else
  \LuT@newcommand\LuT@UnsetAttributeValue[0]{-2147483647 }%
\fi
\LuT@newcommand\unsetattribute[1]{%
  #1=\LuT@UnsetAttributeValue
}
%    \end{macrocode}
%    \end{macro}
%
% \subsection{Catcode tables}
%
% \subsubsection{Allocation}
%
%    \begin{macro}{\LuT@AllocCatcodeTable}
%    \begin{macrocode}
\newcount\LuT@AllocCatcodeTable
\LuT@AllocCatcodeTable=\m@ne
\newcount\CatcodeTableStack
\CatcodeTableStack=\z@
%    \end{macrocode}
%    \end{macro}
%    \begin{macro}{\newcatcodetable}
%    \begin{macrocode}
\LuT@newcommand\newcatcodetable[1]{%
  \ifnum\LuT@AllocCatcodeTable<1114110 % 0x10FFFF is maximal \chardef
    % or < 268435455 % 2^28 - 1
    \global\advance\LuT@AllocCatcodeTable by\tw@
    \allocationnumber=\LuT@AllocCatcodeTable
    \global\chardef#1=\allocationnumber
    \wlog{%
      \string#1=\string\catcodetable\the\allocationnumber
    }%
  \else
    \errmessage{No room for a new \string\catcodetable}%
  \fi
}%
%    \end{macrocode}
%    \end{macro}
%
%    \begin{macro}{\IncCatcodeTableStack}
%    \begin{macrocode}
\LuT@newcommand\IncCatcodeTableStack[0]{%
  \ifnum\CatcodeTableStack<268435454 %
    \global\advance\CatcodeTableStack by\tw@
  \else
    \@PackageError{luatex}{%
      Catcode table stack overflow%
    }\@ehd
  \fi
}
%    \end{macrocode}
%    \end{macro}
%
%    \begin{macro}{\DecCatcodeTableStack}
%    \begin{macrocode}
\LuT@newcommand\DecCatcodeTableStack[0]{%
  \ifnum\CatcodeTableStack>\z@
    \global\advance\CatcodeTableStack by-2 %
  \else
    \@PackageError{luatex}{%
      Catcode table stack is empty%
    }\@ehd
  \fi
}
%    \end{macrocode}
%    \end{macro}
%
% \subsubsection{\cs{SetCatcodeRange}}
%
%    \begin{macro}{\SetCatcodeRange}
%    \begin{macrocode}
\LuT@newcommand\SetCatcodeRange[3]{%
  \edef\LuT@temp{%
    \noexpand\@tempcnta=\the\@tempcnta
    \noexpand\@tempcntb=\the\@tempcntb
    \noexpand\count@=\the\count@
    \relax
  }%
  \@tempcnta=\numexpr#1\relax
  \@tempcntb=\numexpr#2\relax
  \count@=\numexpr#3\relax
  \loop
    \unless\ifnum\@tempcnta>\@tempcntb
    \catcode\@tempcnta=\count@
    \advance\@tempcnta by \@ne
  \repeat
  \LuT@temp
}
%    \end{macrocode}
%    \end{macro}
%
% \subsubsection{Predefined catcode tables}
%
%    \begin{macrocode}
\newcatcodetable\CatcodeTableIniTeX
\newcatcodetable\CatcodeTableString
\newcatcodetable\CatcodeTableOther
\newcatcodetable\CatcodeTableLaTeX
%    \end{macrocode}
%
%    \begin{macrocode}
\luatexinitcatcodetable\CatcodeTableIniTeX
\begingroup
  \def\@makeother#1{\catcode#1=12\relax}%
  \@firstofone{%
    \luatexcatcodetable\CatcodeTableIniTeX
    \begingroup
      \SetCatcodeRange{0}{8}{15}%
      \catcode9=10 % tab
      \catcode11=15 %
      \catcode12=13 % form feed
      \SetCatcodeRange{14}{31}{15}%
      \catcode35=6 % hash
      \catcode36=3 % dollar
      \catcode38=4 % ampersand
      \catcode94=7 % circumflex
      \catcode95=8 % underscore
      \catcode123=1 % brace left
      \catcode125=2 % brace right
      \catcode126=13 % tilde
      \catcode127=15 %
      \luatexsavecatcodetable\CatcodeTableLaTeX
    \endgroup
    \@makeother{0}% nul
    \@makeother{13}% carriage return
    \@makeother{37}% percent
    \@makeother{92}% backslash
    \@makeother{127}%
    \SetCatcodeRange{65}{90}{12}% A-Z
    \SetCatcodeRange{97}{122}{12}% a-z
    \luatexsavecatcodetable\CatcodeTableString
    \@makeother{32}% space
    \luatexsavecatcodetable\CatcodeTableOther
  \endgroup
}%
%    \end{macrocode}
%
% \subsubsection{Number stack}
%
%    \begin{macro}{\LuT@NumStackEmpty}
%    A special empty stack value because of \cs{@cdr}'s brace removal.
%    \begin{macrocode}
\def\LuT@NumStackEmpty{0}
%    \end{macrocode}
%    \end{macro}
%    \begin{macro}{\LuT@NumStack}
%    \begin{macrocode}
\let\LuT@NumStack\LuT@NumStackEmpty
%    \end{macrocode}
%    \end{macro}
%    \begin{macro}{\PushCatcodeTableNumStack}
%    \begin{macrocode}
\LuT@newcommand\PushCatcodeTableNumStack[0]{%
  \xdef\LuT@NumStack{%
    {\the\luatexcatcodetable}\LuT@NumStack
  }%
}
%    \end{macrocode}
%    \end{macro}
%    \begin{macro}{\PopCatcodeTableNumStack}
%    \begin{macrocode}
\LuT@newcommand\PopCatcodeTableNumStack[0]{%
  \ifx\LuT@NumStack\LuT@NumStackEmpty
    \@PackageWarning{luatex}{Empty catcode table number stack}%
    \luatexcatcodetable\z@
  \else
    \luatexcatcodetable=\expandafter\@car\LuT@NumStack\@nil\relax
    \xdef\LuT@NumStack{%
      \expandafter\@cdr\LuT@NumStack\@nil
    }%
  \fi
}
%    \end{macrocode}
%    \end{macro}
%
% \subsubsection{Catcode regime macros}
%
%    \begin{macro}{\BeginCatcodeRegime}
%    \begin{macrocode}
\LuT@newcommand\BeginCatcodeRegime[1]{%
  \PushCatcodeTableNumStack
  \luatexcatcodetable=\numexpr#1\relax
  \IncCatcodeTableStack
  \luatexsavecatcodetable\CatcodeTableStack
  \luatexcatcodetable\CatcodeTableStack
}
%    \end{macrocode}
%    \end{macro}
%
%    \begin{macro}{\EndCatcodeRegime}
%    \begin{macrocode}
\LuT@newcommand\EndCatcodeRegime[0]{%
  \DecCatcodeTableStack
  \PopCatcodeTableNumStack
}
%    \end{macrocode}
%    \end{macro}
%
% \subsection{Lua module loader}
%
%    \begin{macrocode}
\begingroup\expandafter\expandafter\expandafter\endgroup
\expandafter\ifx\csname RequirePackage\endcsname\relax
  \input luatex-loader.sty\relax
\else
  \RequirePackage{luatex-loader}[2010/03/09]%
\fi
%    \end{macrocode}
%    \begin{macrocode}
\LuT@AtEnd%
%</package>
%    \end{macrocode}
%
%    \begin{macrocode}
%<*loader>
%    \end{macrocode}
%
%    Reload check, especially if the package is not used with \LaTeX.
%    \begin{macrocode}
\begingroup\catcode61\catcode48\catcode32=10\relax%
  \catcode13=5 % ^^M
  \endlinechar=13 %
  \catcode35=6 % #
  \catcode39=12 % '
  \catcode44=12 % ,
  \catcode45=12 % -
  \catcode46=12 % .
  \catcode58=12 % :
  \catcode64=11 % @
  \catcode123=1 % {
  \catcode125=2 % }
  \expandafter\let\expandafter\x\csname ver@luatex-loader.sty\endcsname
  \ifx\x\relax % plain-TeX, first loading
  \else
    \def\empty{}%
    \ifx\x\empty % LaTeX, first loading,
      % variable is initialized, but \ProvidesPackage not yet seen
    \else
      \expandafter\ifx\csname PackageInfo\endcsname\relax
        \def\x#1#2{%
          \immediate\write-1{Package #1 Info: #2.}%
        }%
      \else
        \def\x#1#2{\PackageInfo{#1}{#2, stopped}}%
      \fi
      \x{luatex-loader}{The package is already loaded}%
      \aftergroup\endinput
    \fi
  \fi
\endgroup%
%    \end{macrocode}
%    Package identification:
%    \begin{macrocode}
\begingroup\catcode61\catcode48\catcode32=10\relax%
  \catcode13=5 % ^^M
  \endlinechar=13 %
  \catcode35=6 % #
  \catcode39=12 % '
  \catcode40=12 % (
  \catcode41=12 % )
  \catcode44=12 % ,
  \catcode45=12 % -
  \catcode46=12 % .
  \catcode47=12 % /
  \catcode58=12 % :
  \catcode64=11 % @
  \catcode91=12 % [
  \catcode93=12 % ]
  \catcode123=1 % {
  \catcode125=2 % }
  \expandafter\ifx\csname ProvidesPackage\endcsname\relax
    \def\x#1#2#3[#4]{\endgroup
      \immediate\write-1{Package: #3 #4}%
      \xdef#1{#4}%
    }%
  \else
    \def\x#1#2[#3]{\endgroup
      #2[{#3}]%
      \ifx#1\@undefined
        \xdef#1{#3}%
      \fi
      \ifx#1\relax
        \xdef#1{#3}%
      \fi
    }%
  \fi
\expandafter\x\csname ver@luatex-loader.sty\endcsname
\ProvidesPackage{luatex-loader}%
  [2010/03/09 v0.4 Lua module loader (HO)]%
%    \end{macrocode}
%    \begin{macrocode}
\begingroup\catcode61\catcode48\catcode32=10\relax%
  \catcode13=5\endlinechar=13\relax%
  \catcode10=12 % ^^J
  \catcode34=12 % "
  \catcode39=12 % '
  \catcode40=12 % (
  \catcode41=12 % )
  \catcode44=12 % ,
  \catcode46=12 % .
  \catcode60=12 % <
  \catcode61=12 % =
  \catcode95=12 % _ (other!)
  \catcode96=12 % `
  \catcode123=1 % {
  \catcode125=2 % }
  \endlinechar=10 %
  \ifnum\luatexversion<36 %
    \directlua0%
  \else %
    \expandafter\directlua %
  \fi %
  {%
     do
      local script = "oberdiek.luatex.lua"
      local file = kpse.find_file(script, "texmfscripts")
      if file then
        texio.write_nl("(" .. file .. ")")
        dofile(file)
      else
        error("File `" .. script .. "' not found")
      end
    end
  }%
\endgroup\endinput%
%    \end{macrocode}
%
%    \begin{macrocode}
%</loader>
%    \end{macrocode}
%
% \subsection{Lua script}
%
%    Currently \LuaTeX\ does not use KPSE when searching for module
%    files. The following Lua script implements a workaround.
%    It extends |package.loader| by another search method.
%    Modules are found by the module name with extension \xext{lua}
%    similar to
%    \begin{quote}
%      \ttfamily |kpsewhich --format=texmfscripts| \meta{module}|.lua|
%    \end{quote}
%    Unhappily \textsf{kpsewhich} does not support directory components
%    in the file name. Therefore a module |a.b.c| cannot be installed
%    as \xfile{a/b/c.lua}. The script must be named \xfile{a.b.c.lua}.
%    \begin{macrocode}
%<*lua>
%    \end{macrocode}
%    \begin{macrocode}
module("oberdiek.luatex", package.seeall)
%    \end{macrocode}
%    \begin{macrocode}
function kpse_module_loader(module)
  local script = module .. ".lua"
  local file = kpse.find_file(script, "texmfscripts")
  if file then
    local loader, error = loadfile(file)
    if loader then
      texio.write_nl("(" .. file .. ")")
      return loader
    end
    return "\n\t[oberdiek.luatex.kpse_module_loader] Loading error:\n\t"
           .. error
  end
  return "\n\t[oberdiek.luatex.kpse_module_loader] Search failed"
end
table.insert(package.loaders, kpse_module_loader)
%    \end{macrocode}
%    \begin{macrocode}
%</lua>
%    \end{macrocode}
%
% \section{Test}
%
%    \begin{macrocode}
%<*test2>
\documentclass{article}
\def\LoadCommand{%
  \RequirePackage{luatex}[2010/03/09]%
}
%</test2>
%    \end{macrocode}
%    \begin{macrocode}
%<*test3>
\documentclass{article}
\def\LoadCommand{%
  \RequirePackage{luatex-loader}[2010/03/09]%
}
%</test3>
%    \end{macrocode}
% \subsection{Catcode checks for loading}
%
%    \begin{macrocode}
%<*test1>
%    \end{macrocode}
%    \begin{macrocode}
\catcode`\{=1 %
\catcode`\}=2 %
\catcode`\#=6 %
\catcode`\@=11 %
\expandafter\ifx\csname count@\endcsname\relax
  \countdef\count@=255 %
\fi
\expandafter\ifx\csname @gobble\endcsname\relax
  \long\def\@gobble#1{}%
\fi
\expandafter\ifx\csname @firstofone\endcsname\relax
  \long\def\@firstofone#1{#1}%
\fi
\expandafter\ifx\csname loop\endcsname\relax
  \expandafter\@firstofone
\else
  \expandafter\@gobble
\fi
{%
  \def\loop#1\repeat{%
    \def\body{#1}%
    \iterate
  }%
  \def\iterate{%
    \body
      \let\next\iterate
    \else
      \let\next\relax
    \fi
    \next
  }%
  \let\repeat=\fi
}%
\def\RestoreCatcodes{}
\count@=0 %
\loop
  \edef\RestoreCatcodes{%
    \RestoreCatcodes
    \catcode\the\count@=\the\catcode\count@\relax
  }%
\ifnum\count@<255 %
  \advance\count@ 1 %
\repeat

\def\RangeCatcodeInvalid#1#2{%
  \count@=#1\relax
  \loop
    \catcode\count@=15 %
  \ifnum\count@<#2\relax
    \advance\count@ 1 %
  \repeat
}
\def\RangeCatcodeCheck#1#2#3{%
  \count@=#1\relax
  \loop
    \ifnum#3=\catcode\count@
    \else
      \errmessage{%
        Character \the\count@\space
        with wrong catcode \the\catcode\count@\space
        instead of \number#3%
      }%
    \fi
  \ifnum\count@<#2\relax
    \advance\count@ 1 %
  \repeat
}
\def\space{ }
\expandafter\ifx\csname LoadCommand\endcsname\relax
  \def\LoadCommand{\input luatex.sty\relax}%
\fi
\def\Test{%
  \RangeCatcodeInvalid{0}{47}%
  \RangeCatcodeInvalid{58}{64}%
  \RangeCatcodeInvalid{91}{96}%
  \RangeCatcodeInvalid{123}{255}%
  \catcode`\@=12 %
  \catcode`\\=0 %
  \catcode`\%=14 %
  \LoadCommand
  \RangeCatcodeCheck{0}{36}{15}%
  \RangeCatcodeCheck{37}{37}{14}%
  \RangeCatcodeCheck{38}{47}{15}%
  \RangeCatcodeCheck{48}{57}{12}%
  \RangeCatcodeCheck{58}{63}{15}%
  \RangeCatcodeCheck{64}{64}{12}%
  \RangeCatcodeCheck{65}{90}{11}%
  \RangeCatcodeCheck{91}{91}{15}%
  \RangeCatcodeCheck{92}{92}{0}%
  \RangeCatcodeCheck{93}{96}{15}%
  \RangeCatcodeCheck{97}{122}{11}%
  \RangeCatcodeCheck{123}{255}{15}%
  \RestoreCatcodes
}
\Test
\csname @@end\endcsname
\end
%    \end{macrocode}
%    \begin{macrocode}
%</test1>
%    \end{macrocode}
%
% \subsection{Catcode tables}
%
% \subsubsection{Predefined catcode tables}
%
%    \begin{macrocode}
%<*test4>
\NeedsTeXFormat{LaTeX2e}
%    \end{macrocode}
%    Remember \LaTeX's initial catcodes in count registers
%    starting at \cs{TestLaTeX}.
%    \begin{macrocode}
\count0=0 %
\chardef\TestLaTeX=1000 %
\chardef\TestMax=300 %
\loop
  \count\numexpr\TestLaTeX+\count0\relax=\catcode\count0 %
\ifnum\count0<\TestMax
  \advance\count0 by 1 %
\repeat
\documentclass{minimal}
\usepackage{luatex}[2010/03/09]
\usepackage{qstest}
\IncludeTests{*}
\LogTests{log}{*}{*}
\makeatletter
\def\Check#1{%
  \Expect*{\the\count@=\the\catcode\count@}%
         *{\the\count@=#1}%
}
\newcount\scratch
\def\Test#1#2{%
  \begin{qstest}{CatcodeTable#1}{CatcodeTable#1}%
    \luatexcatcodetable\csname CatcodeTable#1\endcsname
    \count@=\z@
    \loop
      \scratch=#2\relax
      \Expect*{\the\count@=\the\catcode\count@}%
             *{\the\count@=\the\scratch}%
    \ifnum\count@<\TestMax
      \advance\count@\@ne
    \repeat
  \end{qstest}%
}
\begingroup
  % luatex-unicode-letters.tex makes some slots to letters
  \def\TestMax{169}%
  \Test{LaTeX}{\the\count\numexpr\TestLaTeX+\count@}%
\endgroup
\Test{String}{\ifnum\count@=32 10\else 12\fi}
\Test{Other}{12}
\luatexinitcatcodetable99 %
\Test{IniTeX}{%
  0\relax
  \begingroup
    \luatexcatcodetable99 %
    \global\scratch=\the\catcode\count@
  \endgroup
}
%    \end{macrocode}
%
% \subsubsection{Catcode table number stack}
%
%    \begin{macrocode}
\begin{qstest}{CatcodeTableNumStack}{CatcodeTableNumStack}
  \def\TestStack#1{%
    \Expect*{\LuT@NumStack}{#1}%
  }%
  \TestStack{0}%
  \PushCatcodeTableNumStack
  \TestStack{{0}0}%
  \@firstofone{%
    \begingroup
      \luatexinitcatcodetable12 %
      \luatexcatcodetable12 %
      \PushCatcodeTableNumStack
      \TestStack{{12}{0}0}%
      \PopCatcodeTableNumStack
      \TestStack{{0}0}%
      \PopCatcodeTableNumStack
      \TestStack{0}%
      \def\TestWarning{Missing empty stack warning}%
      \def\@PackageWarning#1#2{\def\TestWarning{empty stack}}%
      \PopCatcodeTableNumStack
      \TestStack{0}%
      \Expect*{\TestWarning}{empty stack}%
    \endgroup
  }%
\end{qstest}
%    \end{macrocode}
%
% \subsubsection{Catcode table stack}
%
%    \begin{macrocode}
\begin{qstest}{CatcodeTableStack}{CatcodeTableStack}
  \def\TestStack#1{%
    \Expect*{\the\CatcodeTableStack}{#1}%
  }%
  \TestStack{0}%
  \IncCatcodeTableStack
  \TestStack{2}%
  \IncCatcodeTableStack
  \TestStack{4}%
  \begingroup
    \IncCatcodeTableStack
    \TestStack{6}%
  \endgroup
  \TestStack{6}%
  \begingroup
    \DecCatcodeTableStack
    \TestStack{4}%
  \endgroup
  \TestStack{4}%
  \DecCatcodeTableStack
  \TestStack{2}%
  \DecCatcodeTableStack
  \TestStack{0}%
  \begingroup
    \def\TestError{Missing error}%
    \def\@PackageError#1#2#3{%
      \def\TestError{Empty stack}%
    }%
    \DecCatcodeTableStack
    \TestStack{0}%
    \Expect*{\TestError}{Empty stack}%
  \endgroup
\end{qstest}
%    \end{macrocode}
%
% \subsubsection{Catcode regime macros}
%
%    \begin{macrocode}
\begin{qstest}{CatcodeRegime}{CatcodeRegime}
  \def\TestStacks#1#2#3{%
    \Expect*{\the\luatexcatcodetable}{#1}%
    \Expect*{\the\CatcodeTableStack}{#2}%
    \Expect*{\LuT@NumStack}{#3}%
  }%
  \TestStacks{0}{0}{0}%
  \catcode`\|=7 %
  \BeginCatcodeRegime\CatcodeTableLaTeX
    \TestStacks{2}{2}{{0}0}%
    \Expect*{\the\catcode`\|}{12}%
  \EndCatcodeRegime
  \TestStacks{0}{0}{0}%
  \Expect*{\the\catcode`\|}{7}%
\end{qstest}
%    \end{macrocode}
%
% \subsection{Attribute allocation}
%
%    \begin{macrocode}
\begin{qstest}{Attributes}{Attributes}
  \newattribute\TestAttr
  \Expect*{\meaning\TestAttr}%
         *{\string\attribute\number\allocationnumber}%
  \Expect*{\the\allocationnumber}{0}%
  \begingroup
    \newattribute\TestAttr
    \Expect*{\the\allocationnumber}{1}%
  \endgroup
  \Expect*{\the\allocationnumber}{0}%
  \Expect*{\meaning\TestAttr}*{\string\attribute1}%
  \Expect*{\the\TestAttr}*{\number\LuT@UnsetAttributeValue}%
  \def\Test#1{%
    \setattribute\TestAttr{#1}%
    \Expect*{\the\TestAttr}{#1}%
  }%
  \Test{0}%
  \Test{1}%
  \Test{-1}%
  \Test{123}%
  \unsetattribute\TestAttr
  \Expect*{\the\TestAttr}*{\number\LuT@UnsetAttributeValue}%
  \begingroup
    \Expect*{\the\TestAttr}*{\number\LuT@UnsetAttributeValue}%
    \Test{1234}%
  \endgroup
  \Expect*{\the\TestAttr}*{\number\LuT@UnsetAttributeValue}%
\end{qstest}
%    \end{macrocode}
%
%    \begin{macrocode}
\@@end
%</test4>
%    \end{macrocode}
%
% \subsection{Short test for \plainTeX}
%
%    \begin{macrocode}
%<*test5>
\input luatex.sty\relax
\newattribute\TestAttr
\setattribute\TestAttr{10}
\unsetattribute\TestAttr
\newcatcodetable\TestCTa
\begingroup
  \SetCatcodeRange{`A}{`Z}{12}%
\endgroup
\BeginCatcodeRegime\CatcodeTableLaTeX
\EndCatcodeRegime
\end
%</test5>
%    \end{macrocode}
%
% \section{Installation}
%
% \subsection{Download}
%
% \paragraph{Package.} This package is available on
% CTAN\footnote{\url{ftp://ftp.ctan.org/tex-archive/}}:
% \begin{description}
% \item[\CTAN{macros/latex/contrib/oberdiek/luatex.dtx}] The source file.
% \item[\CTAN{macros/latex/contrib/oberdiek/luatex.pdf}] Documentation.
% \end{description}
%
%
% \paragraph{Bundle.} All the packages of the bundle `oberdiek'
% are also available in a TDS compliant ZIP archive. There
% the packages are already unpacked and the documentation files
% are generated. The files and directories obey the TDS standard.
% \begin{description}
% \item[\CTAN{install/macros/latex/contrib/oberdiek.tds.zip}]
% \end{description}
% \emph{TDS} refers to the standard ``A Directory Structure
% for \TeX\ Files'' (\CTAN{tds/tds.pdf}). Directories
% with \xfile{texmf} in their name are usually organized this way.
%
% \subsection{Bundle installation}
%
% \paragraph{Unpacking.} Unpack the \xfile{oberdiek.tds.zip} in the
% TDS tree (also known as \xfile{texmf} tree) of your choice.
% Example (linux):
% \begin{quote}
%   |unzip oberdiek.tds.zip -d ~/texmf|
% \end{quote}
%
% \paragraph{Script installation.}
% Check the directory \xfile{TDS:scripts/oberdiek/} for
% scripts that need further installation steps.
% Package \xpackage{attachfile2} comes with the Perl script
% \xfile{pdfatfi.pl} that should be installed in such a way
% that it can be called as \texttt{pdfatfi}.
% Example (linux):
% \begin{quote}
%   |chmod +x scripts/oberdiek/pdfatfi.pl|\\
%   |cp scripts/oberdiek/pdfatfi.pl /usr/local/bin/|
% \end{quote}
%
% \subsection{Package installation}
%
% \paragraph{Unpacking.} The \xfile{.dtx} file is a self-extracting
% \docstrip\ archive. The files are extracted by running the
% \xfile{.dtx} through \plainTeX:
% \begin{quote}
%   \verb|tex luatex.dtx|
% \end{quote}
%
% \paragraph{TDS.} Now the different files must be moved into
% the different directories in your installation TDS tree
% (also known as \xfile{texmf} tree):
% \begin{quote}
% \def\t{^^A
% \begin{tabular}{@{}>{\ttfamily}l@{ $\rightarrow$ }>{\ttfamily}l@{}}
%   luatex.sty & tex/generic/oberdiek/luatex.sty\\
%   luatex-loader.sty & tex/generic/oberdiek/luatex-loader.sty\\
%   oberdiek.luatex.lua & scripts/oberdiek/oberdiek.luatex.lua\\
%   luatex.pdf & doc/latex/oberdiek/luatex.pdf\\
%   test/luatex-test1.tex & doc/latex/oberdiek/test/luatex-test1.tex\\
%   test/luatex-test2.tex & doc/latex/oberdiek/test/luatex-test2.tex\\
%   test/luatex-test3.tex & doc/latex/oberdiek/test/luatex-test3.tex\\
%   test/luatex-test4.tex & doc/latex/oberdiek/test/luatex-test4.tex\\
%   test/luatex-test5.tex & doc/latex/oberdiek/test/luatex-test5.tex\\
%   luatex.dtx & source/latex/oberdiek/luatex.dtx\\
% \end{tabular}^^A
% }^^A
% \sbox0{\t}^^A
% \ifdim\wd0>\linewidth
%   \begingroup
%     \advance\linewidth by\leftmargin
%     \advance\linewidth by\rightmargin
%   \edef\x{\endgroup
%     \def\noexpand\lw{\the\linewidth}^^A
%   }\x
%   \def\lwbox{^^A
%     \leavevmode
%     \hbox to \linewidth{^^A
%       \kern-\leftmargin\relax
%       \hss
%       \usebox0
%       \hss
%       \kern-\rightmargin\relax
%     }^^A
%   }^^A
%   \ifdim\wd0>\lw
%     \sbox0{\small\t}^^A
%     \ifdim\wd0>\linewidth
%       \ifdim\wd0>\lw
%         \sbox0{\footnotesize\t}^^A
%         \ifdim\wd0>\linewidth
%           \ifdim\wd0>\lw
%             \sbox0{\scriptsize\t}^^A
%             \ifdim\wd0>\linewidth
%               \ifdim\wd0>\lw
%                 \sbox0{\tiny\t}^^A
%                 \ifdim\wd0>\linewidth
%                   \lwbox
%                 \else
%                   \usebox0
%                 \fi
%               \else
%                 \lwbox
%               \fi
%             \else
%               \usebox0
%             \fi
%           \else
%             \lwbox
%           \fi
%         \else
%           \usebox0
%         \fi
%       \else
%         \lwbox
%       \fi
%     \else
%       \usebox0
%     \fi
%   \else
%     \lwbox
%   \fi
% \else
%   \usebox0
% \fi
% \end{quote}
% If you have a \xfile{docstrip.cfg} that configures and enables \docstrip's
% TDS installing feature, then some files can already be in the right
% place, see the documentation of \docstrip.
%
% \subsection{Refresh file name databases}
%
% If your \TeX~distribution
% (\teTeX, \mikTeX, \dots) relies on file name databases, you must refresh
% these. For example, \teTeX\ users run \verb|texhash| or
% \verb|mktexlsr|.
%
% \subsection{Some details for the interested}
%
% \paragraph{Attached source.}
%
% The PDF documentation on CTAN also includes the
% \xfile{.dtx} source file. It can be extracted by
% AcrobatReader 6 or higher. Another option is \textsf{pdftk},
% e.g. unpack the file into the current directory:
% \begin{quote}
%   \verb|pdftk luatex.pdf unpack_files output .|
% \end{quote}
%
% \paragraph{Unpacking with \LaTeX.}
% The \xfile{.dtx} chooses its action depending on the format:
% \begin{description}
% \item[\plainTeX:] Run \docstrip\ and extract the files.
% \item[\LaTeX:] Generate the documentation.
% \end{description}
% If you insist on using \LaTeX\ for \docstrip\ (really,
% \docstrip\ does not need \LaTeX), then inform the autodetect routine
% about your intention:
% \begin{quote}
%   \verb|latex \let\install=y\input{luatex.dtx}|
% \end{quote}
% Do not forget to quote the argument according to the demands
% of your shell.
%
% \paragraph{Generating the documentation.}
% You can use both the \xfile{.dtx} or the \xfile{.drv} to generate
% the documentation. The process can be configured by the
% configuration file \xfile{ltxdoc.cfg}. For instance, put this
% line into this file, if you want to have A4 as paper format:
% \begin{quote}
%   \verb|\PassOptionsToClass{a4paper}{article}|
% \end{quote}
% An example follows how to generate the
% documentation with pdf\LaTeX:
% \begin{quote}
%\begin{verbatim}
%pdflatex luatex.dtx
%makeindex -s gind.ist luatex.idx
%pdflatex luatex.dtx
%makeindex -s gind.ist luatex.idx
%pdflatex luatex.dtx
%\end{verbatim}
% \end{quote}
%
% \section{Catalogue}
%
% The following XML file can be used as source for the
% \href{http://mirror.ctan.org/help/Catalogue/catalogue.html}{\TeX\ Catalogue}.
% The elements \texttt{caption} and \texttt{description} are imported
% from the original XML file from the Catalogue.
% The name of the XML file in the Catalogue is \xfile{luatex.xml}.
%    \begin{macrocode}
%<*catalogue>
<?xml version='1.0' encoding='us-ascii'?>
<!DOCTYPE entry SYSTEM 'catalogue.dtd'>
<entry datestamp='$Date$' modifier='$Author$' id='luatex'>
  <name>luatex</name>
  <caption>The LuaTeX engine.</caption>
  <authorref id='auth:oberdiek'/>
  <copyright owner='Heiko Oberdiek' year='2007,2009,2010'/>
  <license type='lppl1.3'/>
  <version number='0.4'/>
  <description>
    LuaTeX is an extended version of pdfTeX using Lua as an embedded
    scripting language.  The LuaTeX project&#x2019;s main objective
    is to provide an open and configurable variant of TeX while at the
    same time offering downward compatibility.
    <p/>
    LuaTeX uses Unicode (as UTF-8) as its default input encoding, and
    is able to use modern (OpenType) fonts (for both text and mathematics).
    <p/>
    It should be noted that LuaTeX is still under development; its
    specification has been declared stable, but absolute stability
    may not in practice be assumed.
    <p/>
    The package is part of the <xref refid='oberdiek'>oberdiek</xref> bundle.
  </description>
  <documentation details='Package documentation'
      href='ctan:/macros/latex/contrib/oberdiek/luatex.pdf'/>
  <ctan file='true' path='/macros/latex/contrib/oberdiek/luatex.dtx'/>
  <miktex location='oberdiek'/>
  <texlive location='oberdiek'/>
  <install path='/macros/latex/contrib/oberdiek/oberdiek.tds.zip'/>
</entry>
%</catalogue>
%    \end{macrocode}
%
% \begin{History}
%   \begin{Version}{2007/12/12 v0.1}
%   \item
%     First public version.
%   \end{Version}
%   \begin{Version}{2009/04/10 v0.2}
%   \item
%     Requires package \xpackage{ifluatex} in version 2.0 to ensure
%     \cs{luatexversion}.
%   \item
%     Updates the call of \cs{directlua}, the syntax has changed
%     in \LuaTeX\ 0.36.
%   \end{Version}
%   \begin{Version}{2009/12/02 v0.3}
%   \item
%     Unsetting of attributes updated for \LuaTeX\ 0.37.
%   \end{Version}
%   \begin{Version}{2010/03/09 v0.4}
%   \item
%     Support for lua states removed.
%   \item
%     Calling tex.enableprimitives for used primitives.
%   \end{Version}
% \end{History}
%
% \PrintIndex
%
% \Finale
\endinput

%        (quote the arguments according to the demands of your shell)
%
% Documentation:
%    (a) If luatex.drv is present:
%           latex luatex.drv
%    (b) Without luatex.drv:
%           latex luatex.dtx; ...
%    The class ltxdoc loads the configuration file ltxdoc.cfg
%    if available. Here you can specify further options, e.g.
%    use A4 as paper format:
%       \PassOptionsToClass{a4paper}{article}
%
%    Programm calls to get the documentation (example):
%       pdflatex luatex.dtx
%       makeindex -s gind.ist luatex.idx
%       pdflatex luatex.dtx
%       makeindex -s gind.ist luatex.idx
%       pdflatex luatex.dtx
%
% Installation:
%    TDS:tex/generic/oberdiek/luatex.sty
%    TDS:tex/generic/oberdiek/luatex-loader.sty
%    TDS:scripts/oberdiek/oberdiek.luatex.lua
%    TDS:doc/latex/oberdiek/luatex.pdf
%    TDS:doc/latex/oberdiek/test/luatex-test1.tex
%    TDS:doc/latex/oberdiek/test/luatex-test2.tex
%    TDS:doc/latex/oberdiek/test/luatex-test3.tex
%    TDS:doc/latex/oberdiek/test/luatex-test4.tex
%    TDS:doc/latex/oberdiek/test/luatex-test5.tex
%    TDS:source/latex/oberdiek/luatex.dtx
%
%<*ignore>
\begingroup
  \catcode123=1 %
  \catcode125=2 %
  \def\x{LaTeX2e}%
\expandafter\endgroup
\ifcase 0\ifx\install y1\fi\expandafter
         \ifx\csname processbatchFile\endcsname\relax\else1\fi
         \ifx\fmtname\x\else 1\fi\relax
\else\csname fi\endcsname
%</ignore>
%<*install>
\input docstrip.tex
\Msg{************************************************************************}
\Msg{* Installation}
\Msg{* Package: luatex 2010/03/09 v0.4 LuaTeX basic definition package (HO)}
\Msg{************************************************************************}

\keepsilent
\askforoverwritefalse

\let\MetaPrefix\relax
\preamble

This is a generated file.

Project: luatex
Version: 2010/03/09 v0.4

Copyright (C) 2007, 2009, 2010 by
   Heiko Oberdiek <heiko.oberdiek at googlemail.com>

This work may be distributed and/or modified under the
conditions of the LaTeX Project Public License, either
version 1.3c of this license or (at your option) any later
version. This version of this license is in
   http://www.latex-project.org/lppl/lppl-1-3c.txt
and the latest version of this license is in
   http://www.latex-project.org/lppl.txt
and version 1.3 or later is part of all distributions of
LaTeX version 2005/12/01 or later.

This work has the LPPL maintenance status "maintained".

This Current Maintainer of this work is Heiko Oberdiek.

The Base Interpreter refers to any `TeX-Format',
because some files are installed in TDS:tex/generic//.

This work consists of the main source file luatex.dtx
and the derived files
   luatex.sty, luatex.pdf, luatex.ins, luatex.drv, luatex-loader.sty,
   luatex-test1.tex, luatex-test2.tex, luatex-test3.tex,
   luatex-test4.tex, luatex-test5.tex, oberdiek.luatex.lua.

\endpreamble
\let\MetaPrefix\DoubleperCent

\generate{%
  \file{luatex.ins}{\from{luatex.dtx}{install}}%
  \file{luatex.drv}{\from{luatex.dtx}{driver}}%
  \usedir{tex/generic/oberdiek}%
  \file{luatex.sty}{\from{luatex.dtx}{package}}%
  \file{luatex-loader.sty}{\from{luatex.dtx}{loader}}%
  \usedir{doc/latex/oberdiek/test}%
  \file{luatex-test1.tex}{\from{luatex.dtx}{test1}}%
  \file{luatex-test2.tex}{\from{luatex.dtx}{test1,test2}}%
  \file{luatex-test3.tex}{\from{luatex.dtx}{test1,test3}}%
  \file{luatex-test4.tex}{\from{luatex.dtx}{test4}}%
  \file{luatex-test5.tex}{\from{luatex.dtx}{test5}}%
  \nopreamble
  \nopostamble
  \usedir{source/latex/oberdiek/catalogue}%
  \file{luatex.xml}{\from{luatex.dtx}{catalogue}}%
}
\def\MetaPrefix{-- }
\def\defaultpostamble{%
  \MetaPrefix^^J%
  \MetaPrefix\space End of File `\outFileName'.%
}
\def\currentpostamble{\defaultpostamble}%
\generate{%
  \usedir{scripts/oberdiek}%
  \file{oberdiek.luatex.lua}{\from{luatex.dtx}{lua}}%
}

\catcode32=13\relax% active space
\let =\space%
\Msg{************************************************************************}
\Msg{*}
\Msg{* To finish the installation you have to move the following}
\Msg{* files into a directory searched by TeX:}
\Msg{*}
\Msg{*     luatex.sty, luatex-loader.sty}
\Msg{*}
\Msg{* And install the following script file:}
\Msg{*}
\Msg{*     oberdiek.luatex.lua}
\Msg{*}
\Msg{* To produce the documentation run the file `luatex.drv'}
\Msg{* through LaTeX.}
\Msg{*}
\Msg{* Happy TeXing!}
\Msg{*}
\Msg{************************************************************************}

\endbatchfile
%</install>
%<*ignore>
\fi
%</ignore>
%<*driver>
\NeedsTeXFormat{LaTeX2e}
\ProvidesFile{luatex.drv}%
  [2010/03/09 v0.4 LuaTeX basic definition package (HO)]%
\documentclass{ltxdoc}
\usepackage{holtxdoc}[2011/11/22]
\usepackage{paralist}
\begin{document}
  \DocInput{luatex.dtx}%
\end{document}
%</driver>
% \fi
%
% \CheckSum{1429}
%
% \CharacterTable
%  {Upper-case    \A\B\C\D\E\F\G\H\I\J\K\L\M\N\O\P\Q\R\S\T\U\V\W\X\Y\Z
%   Lower-case    \a\b\c\d\e\f\g\h\i\j\k\l\m\n\o\p\q\r\s\t\u\v\w\x\y\z
%   Digits        \0\1\2\3\4\5\6\7\8\9
%   Exclamation   \!     Double quote  \"     Hash (number) \#
%   Dollar        \$     Percent       \%     Ampersand     \&
%   Acute accent  \'     Left paren    \(     Right paren   \)
%   Asterisk      \*     Plus          \+     Comma         \,
%   Minus         \-     Point         \.     Solidus       \/
%   Colon         \:     Semicolon     \;     Less than     \<
%   Equals        \=     Greater than  \>     Question mark \?
%   Commercial at \@     Left bracket  \[     Backslash     \\
%   Right bracket \]     Circumflex    \^     Underscore    \_
%   Grave accent  \`     Left brace    \{     Vertical bar  \|
%   Right brace   \}     Tilde         \~}
%
% \GetFileInfo{luatex.drv}
%
% \title{The \xpackage{luatex} package}
% \date{2010/03/09 v0.4}
% \author{Heiko Oberdiek\\\xemail{heiko.oberdiek at googlemail.com}}
%
% \maketitle
%
% \begin{abstract}
% This package manages the new and extended features and resources
% that \LuaTeX\ provides. Examples are attributes and catcode
% tables.
% \end{abstract}
%
% \tableofcontents
%
% \section{Documentation}
%
% \subsection{Introduction}
%
% \TeX\ provides global resources such as registers.
% But it does not provide an interface for managing these resources.
% For example, two packages want to use a counter register.
% If they take the same register
% number, then the use of both packages will conflict and they cannot
% be used together.
% Therefore formats such as \plainTeX\ or \LaTeX\ implement
% an allocation scheme for registers. A package reserves with
% \cs{newcount} an unused register number for its own exclusive use.
%
% Nowadays \TeX\ is not alone anymore: \eTeX, \pdfTeX\ and other
% compilers for \TeX\ are developed that extend and add new
% features and resources.
%
% Now \LuaTeX\ has reached beta state. It inherits most of
% \pdfTeX's features including \eTeX. Also it implements
% new concepts such as attributes or catcode tables.
%
% \subsubsection{\LaTeX}
%
% \LaTeXe\ is frozen and therefore refuses to even notice
% the new \TeX\ variants. Not even the old \eTeX\ is supported
% by its kernel. At least there is a third party package \xpackage{etex}
% that manages the new \eTeX\ resources.
%
% This package tries to do the same for \LuaTeX\ and starts to
% support at least a few of the new features.
%
% \subsubsection{\plainTeX}
%
% \LaTeX\ has inherited its resource handling from \plainTeX.
% The interface is basically the same: \cs{newcount}, \dots
% Therefore this package tries to follow this tradition by
% providing compatibility to \plainTeX. It can be loaded
% with \plainTeX\ and defines at least some of the
% features that this packages provides for \LaTeX.
%
% \subsection{Register allocation}
%
% \subsubsection{Register with 16 bit}
%
% Because \LuaTeX\ is a super set of \eTeX\ regarding registers,
% the register allocation scheme should not conflict with
% package \xpackage{etex}. Therefore this package is loaded
% to inherit its allocation scheme. The only change is currently
% that the limit is increased to 65536 registers for the following
% register classes:
% \begin{compactitem}
% \item count
% \item dimen
% \item skip
% \item muskip
% \item marks
% \item toks
% \item box
% \end{compactitem}
% This affects the number of global and local registers.
% Because it is done in a package and not in the kernel,
% it is possible that someone loads package \xpackage{etex}
% before uses the local allocation variants. This will prevent
% the extension for this register class. If more registers are
% needed, just load package \xpackage{luatex} earlier.
%
% \subsubsection{Insertions}
%
% Insertions need four registers \cs{count}, \cs{dimen}, \cs{skip},
% and \cs{box} with the same number. Usually they are allocated
% downwards from 254, 253, \dots Also \cs{newcount}, \cs{newdimen}, \dots
% fill up these register numbers from below before switching to
% higher register numbers by package \xpackage{etex}.
% When this occurs, no insertions can be allocated anymore.
%
% Therefore \cs{newcount}, \cs{newdimen}, \cs{newskip}, and
% \cs{newbox} are replaced by their global variants (\cs{globcount}, \dots)
% that use the higher numbers immediately, leaving the room for insertions.
% There should not be an efficiency penalty because \LuaTeX\ stores
% the registers of a class in the same Lua table unlike \eTeX, where
% registers below 256 are stored in an array and higher numbers
% are put in a tree structure.
%
% \subsection{Attributes}
%
% Nodes can have custom attributes in \LuaTeX. These attributes
% are organized by a new register class. As the other registers
% up to $2^{16}$ attributes are supported. An attribute value
% can be negative that means the attribute is not set. Otherwise
% \TeX's range of non-negative integers up to $2^31$ are available.
%
% \begin{declcs}{newattribute} \M{cmd}
% \end{declcs}
% Macro \cs{newattribute} defines command \meta{cmd} using \cs{attributedef}
% using an new attribute number. The new attribute is initially unset.
%
% \begin{declcs}{setattribute} \M{cmd} \M{value}
% \end{declcs}
% Macro \cs{setattribute} locally sets attribute command \meta{cmd}
% to the number \meta{value}. Valid values range from $-1$ until $2^31$
% (the upper limit is the same as for other \TeX\ integer numbers).
%
% \begin{declcs}{unsetattribute} \M{cmd}
% \end{declcs}
% Macro \cs{unsettattribute} clears the attribute command \meta{cmd}.
%
% \subsection{Catcode tables}
%
% \LuaTeX\ introduces catcode tables as new feature, see documentation.
% There is need for discussion, how to deal best:
% \begin{itemize}
% \item \cs{initcatcodetable} and \cs{setcatcodetable} act globally.
% \item \cs{catcodetable} causes an error if used with an uninitialized
%   catcode table.
% \item Large catcode table numbers should be avoided because of
%   performance breakdown.
% \item Use case \LaTeX\ package:
%   The package must not be surprised by changed catcodes and
%   must not surprise by changing catcodes accidently.
%   Catcode tables could offer a solution. At the begin
%   a catcode regime with standard catcodes is established
%   and the old one is restored afterwards.
% \item Use case: \LuaTeX's |tex.print| might be used with
%   a catcode table number, for example a table where all
%   entries have catcode ``other''.
% \item Readonly catcode tables.
% \item Is there is a need for local allocations?
%   (Package \xpackage{etex}'s \cs{loc} variants are not used in
%   \TeX\ Live 2007.)
% \end{itemize}
%
% \subsubsection{Interface proposal}
%
% The idea: \cs{newcatcodetable} allocates odd numbered catcode
% tables. Even numbered tables are managed as stack.
% Also some catcode tables are defined. These must not be
% changed.
%
% \begin{declcs}{newcatcodetable} \M{cmd}
% \end{declcs}
% Macro \cs{newcatcodetable} reserves a new catcode table and
% remembers its number in \meta{cmd}. The catcode table is
% initialized with \iniTeX's catcodes.
%
% \begin{declcs}{CatcodeTableIniTeX}\\
%   \cs{CatcodeTableString}\\
%   \cs{CatcodeTableOther}\\
%   \cs{CatcodeTableLaTeX}\\
% \end{declcs}
% These are catcode tables and must not be changed.
% \cs{CatcodeTableIniTeX} contains the catcode settings of \iniTeX.
% \cs{CatcodeTableString} follows \TeX's convention of
% \cs{string}, \cs{meaning} and friends. The space gets catcode
% 10 (space), the other characters have catcode 12 (other).
% In \cs{CatcodeTableOther} all entries have catcode 12 (other).
% \cs{CatcodeTableLaTeX} contains the setting of a pure \LaTeX\ format
% (`at' is other).
%
% \begin{declcs}{CatcodeTableStack}\\
%   \cs{IncCatcodeTableStack}\\
%   \cs{DecCatcodeTableStack}
% \end{declcs}
% \cs{CatcodeTableStack} is the stack pointer. Initially
% it is catcode table zero. \cs{IncCatcodeTableStack} and
% \cs{DecCatcodeTableStack} increments and decrements the
% stack pointer. Currently \cs{IncCatcodeTableStack} does not
% initialize a new catcode table. Both increment and decrement
% operations do not set a catcode table.
%
% \begin{declcs}{PushCatcodeTableNumStack}\\
%   \cs{PopCatcodeTableNumStack}
% \end{declcs}
% It can be handy to have a global stack for catcode table
% numbers to deal with the global assignment property of
% \cs{initcatcodetable} and \cs{savecatcodetable}.
% \cs{PushCatcodeTableNumStack} pushes the current catcode table
% on the stack.
% \cs{PopCatcodeTableNumStack} pops the topmost number off the number
% stack to set the current catcode table. Catcode table zero is
% used in case of an empty stack.
%
% \begin{declcs}{BeginCatcodeRegime} \M{catcodetable}\\
%   \cs{EndCatcodeRegime}
% \end{declcs}
% \cs{BeginCatcodeRegime} remembers the current catcode table number.
% Then it creates and uses a fresh catcode table on the stack
% that is initialized by \meta{catcodetable}:
% \begin{quote}\ttfamily
%   \cs{PushCatcodeTableNumStack}\\
%   \cs{catcodetable}\meta{catcodetable}||
%   \cs{IncCatcodeTableStack}\\
%   \cs{savecatcodetable}\cs{CatcodeTableStack}\\
%   \cs{catcodetable}\cs{CatcodeTableStack}
% \end{quote}
% \cs{EndCatcodeRegime} drops the catcode table, created by
% \cs{BeginCatcodeRegime} and sets the catcode table that was
% active before:
% \begin{quote}
%   \cs{DecCatcodeTableStack}\\
%   \cs{PopCatcodeTableNumStack}
% \end{quote}
% These macros solve the use case, described earlier for a \LaTeX\ package:
% \begin{quote}
%   |% package foobar.sty|\\
%   |\BeginCatcodeRegime\CatcodeTableLaTeX|\\
%   |\makeatletter|\\
%   |% ... package contents ...|\\
%   |\EndCatcodeRegime|\\
%   |% end of package|
% \end{quote}
% If the package wants to change catcodes after its loading,
% \cs{AtBeginDocument} or \cs{AtEndOfPackage} can be used.
%
% \begin{declcs}{SetCatcodeRange} \M{from} \M{to} \M{catcode}
% \end{declcs}
% The catcodes of characters in range from \meta{from}
% to inclusive \meta{to} are set to \meta{catcode}.
%
% \subsection{Lua module loading}
%
% Currently \LuaTeX\ (version 0.20) does not support Lua script
% files inside \xfile{TDS:scripts//}, because Lua's mechanism
% for module loading does not use the \textsf{kpathsea} library.
% Therefore this packages appends a kpse loader to the list of
% Lua's module loaders. It finds the module \meta{module} by
% \begin{quote}
%   \ttfamily |kpse.find_file("|\meta{module}|.lua", "texmfscripts")|
% \end{quote}
% Unhappily \textsf{kpathsea} does not support directory components
% in a file name. Therefore the Lua convention is not followed to
% replace dots in the module name by the directory separator.
%
% Example:
% A Lua script of a package \xpackage{foobar} wants the following
% modules:
% \begin{quote}
%   |require("foobar.hello.world")|\\
%   |require("org.somewhere.xyz")|
% \end{quote}
% Then they can be find in:
% \begin{quote}
%   |TDS:scripts/foobar/foobar.hello.world.lua|\\
%   |TDS:scripts/foobar/org.somewhere.xyz.lua|
% \end{quote}
% I would have preferred the following locations, following lua conventions,
% e.\,g.:
% \begin{quote}
%   |TDS:scripts/foobar/hello/world.lua|\\
%   |TDS:scripts/foobar/org/somewhere/xyz.lua|
% \end{quote}
% But I do not know, how to achieve this in a reliable way using
% \textsf{kpathsea}.
%
% \subsubsection{Package \xpackage{luatex-loader}}
%
% If someone do not need or want package \xpackage{luatex} but
% it's extension for module loading, then he can use
% package \xpackage{luatex-loader}. Both \plainTeX\ and \LaTeX\ are
% supported.
%
% \StopEventually{
% }
%
% \section{Implementation}
%
%    \begin{macrocode}
%<*package>
%    \end{macrocode}
%
% \subsection{Reload check and package identification}
%    Reload check, especially if the package is not used with \LaTeX.
%    \begin{macrocode}
\begingroup\catcode61\catcode48\catcode32=10\relax%
  \catcode13=5 % ^^M
  \endlinechar=13 %
  \catcode35=6 % #
  \catcode39=12 % '
  \catcode44=12 % ,
  \catcode45=12 % -
  \catcode46=12 % .
  \catcode58=12 % :
  \catcode64=11 % @
  \catcode123=1 % {
  \catcode125=2 % }
  \expandafter\let\expandafter\x\csname ver@luatex.sty\endcsname
  \ifx\x\relax % plain-TeX, first loading
  \else
    \def\empty{}%
    \ifx\x\empty % LaTeX, first loading,
      % variable is initialized, but \ProvidesPackage not yet seen
    \else
      \expandafter\ifx\csname PackageInfo\endcsname\relax
        \def\x#1#2{%
          \immediate\write-1{Package #1 Info: #2.}%
        }%
      \else
        \def\x#1#2{\PackageInfo{#1}{#2, stopped}}%
      \fi
      \x{luatex}{The package is already loaded}%
      \aftergroup\endinput
    \fi
  \fi
\endgroup%
%    \end{macrocode}
%    Package identification:
%    \begin{macrocode}
\begingroup\catcode61\catcode48\catcode32=10\relax%
  \catcode13=5 % ^^M
  \endlinechar=13 %
  \catcode35=6 % #
  \catcode39=12 % '
  \catcode40=12 % (
  \catcode41=12 % )
  \catcode44=12 % ,
  \catcode45=12 % -
  \catcode46=12 % .
  \catcode47=12 % /
  \catcode58=12 % :
  \catcode64=11 % @
  \catcode91=12 % [
  \catcode93=12 % ]
  \catcode123=1 % {
  \catcode125=2 % }
  \expandafter\ifx\csname ProvidesPackage\endcsname\relax
    \def\x#1#2#3[#4]{\endgroup
      \immediate\write-1{Package: #3 #4}%
      \xdef#1{#4}%
    }%
  \else
    \def\x#1#2[#3]{\endgroup
      #2[{#3}]%
      \ifx#1\@undefined
        \xdef#1{#3}%
      \fi
      \ifx#1\relax
        \xdef#1{#3}%
      \fi
    }%
  \fi
\expandafter\x\csname ver@luatex.sty\endcsname
\ProvidesPackage{luatex}%
  [2010/03/09 v0.4 LuaTeX basic definition package (HO)]%
%    \end{macrocode}
%
% \subsection{Catcodes}
%
%    \begin{macrocode}
\begingroup\catcode61\catcode48\catcode32=10\relax%
  \catcode13=5 % ^^M
  \endlinechar=13 %
  \catcode123=1 % {
  \catcode125=2 % }
  \catcode64=11 % @
  \def\x{\endgroup
    \expandafter\edef\csname LuT@AtEnd\endcsname{%
      \endlinechar=\the\endlinechar\relax
      \catcode13=\the\catcode13\relax
      \catcode32=\the\catcode32\relax
      \catcode35=\the\catcode35\relax
      \catcode61=\the\catcode61\relax
      \catcode64=\the\catcode64\relax
      \catcode123=\the\catcode123\relax
      \catcode125=\the\catcode125\relax
    }%
  }%
\x\catcode61\catcode48\catcode32=10\relax%
\catcode13=5 % ^^M
\endlinechar=13 %
\catcode35=6 % #
\catcode64=11 % @
\catcode123=1 % {
\catcode125=2 % }
\def\TMP@EnsureCode#1#2{%
  \edef\LuT@AtEnd{%
    \LuT@AtEnd
    \catcode#1=\the\catcode#1\relax
  }%
  \catcode#1=#2\relax
}
\TMP@EnsureCode{10}{12}% ^^J
\TMP@EnsureCode{34}{12}% "
\TMP@EnsureCode{36}{3}% $
\TMP@EnsureCode{39}{12}% '
\TMP@EnsureCode{40}{12}% (
\TMP@EnsureCode{41}{12}% )
\TMP@EnsureCode{42}{12}% *
\TMP@EnsureCode{43}{12}% +
\TMP@EnsureCode{44}{12}% ,
\TMP@EnsureCode{45}{12}% -
\TMP@EnsureCode{46}{12}% .
\TMP@EnsureCode{47}{12}% /
\TMP@EnsureCode{60}{12}% <
\TMP@EnsureCode{62}{12}% >
\TMP@EnsureCode{91}{12}% [
\TMP@EnsureCode{93}{12}% ]
\TMP@EnsureCode{95}{12}% _ (other!)
\TMP@EnsureCode{96}{12}% `
\edef\LuT@AtEnd{\LuT@AtEnd\noexpand\endinput}
%    \end{macrocode}
%
% \subsection{Check for \LuaTeX}
%
%    Without \LuaTeX\ there is no point in using this package.
%    \begin{macrocode}
\begingroup\expandafter\expandafter\expandafter\endgroup
\expandafter\ifx\csname RequirePackage\endcsname\relax
  \input infwarerr.sty\relax
  \input ifluatex.sty\relax
\else
  \RequirePackage{infwarerr}[2007/09/09]%
  \RequirePackage{ifluatex}[2009/04/10]%
\fi
%    \end{macrocode}
%    \begin{macrocode}
\ifluatex
\else
  \@PackageError{luatex}{%
    This package may only be run using LuaTeX%
  }\@ehc
  \expandafter\LuT@AtEnd
\fi%
%    \end{macrocode}
%
% \subsection{Provide Lua\TeX\ primitives}
%
%    \begin{macrocode}
\ifnum\luatexversion<36 %
  \def\LuT@MakePrimitive#1{%
    \expandafter\let\csname luatex#1\expandafter\endcsname
    \csname #1\endcsname
  }%
\else
  \def\LuT@MakeLuatexPrimitive#1{%
    \begingroup\expandafter\expandafter\expandafter\endgroup
    \expandafter\ifx\csname luatex#1\endcsname\relax
      \begingroup\expandafter\expandafter\expandafter\endgroup
      \expandafter\ifx\csname #1\endcsname\relax
      \else
        \expandafter\let
        \csname luatex#1\expandafter\endcsname
        \csname #1\endcsname
      \fi
    \fi
    \begingroup\expandafter\expandafter\expandafter\endgroup
    \expandafter\ifx\csname luatex#1\endcsname\relax
      \begingroup
        \expandafter\let\csname luatex#1\endcsname\@undefined
        \ifnum0%
          \directlua{%
            if tex.enableprimitives then %
              tex.enableprimitives('luatex',{'#1'})%
              tex.print('1')%
            end%
          }%
          \expandafter\ifx\csname luatex#1\endcsname\relax\else1\fi
        =11 %
          \global\expandafter\let
          \csname luatex#1\expandafter\endcsname
          \csname luatex#1\endcsname
        \else
          \@PackageError{luatex}{%
            tex.enableprimitives failed for `#1'%
          }\@ehc
        \fi
      \endgroup
    \fi
  }%
  \def\LuT@MakePrimitive#1{%
    \begingroup\expandafter\expandafter\expandafter\endgroup
    \expandafter\ifx\csname#1\endcsname\relax
      \begingroup
        \expandafter\let\csname#1\endcsname\@undefined
        \ifnum0%
          \directlua{%
            if tex.enableprimitives then %
              tex.enableprimitives('',{'#1'})%
              tex.print('1')%
            end%
          }%
          \expandafter\ifx\csname#1\endcsname\relax\else1\fi
        =11 %
          \global\expandafter\let
          \csname#1\expandafter\endcsname
          \csname#1\endcsname
        \else
          \@PackageError{luatex}{%
            tex.enableprimitives failed for `#1'%
          }\@ehc
        \fi
      \endgroup
    \fi
  }%
\fi
\LuT@MakeLuatexPrimitive{attribute}
\LuT@MakeLuatexPrimitive{attributedef}
\LuT@MakeLuatexPrimitive{catcodetable}
\LuT@MakeLuatexPrimitive{initcatcodetable}
\LuT@MakeLuatexPrimitive{luaescapestring}
\LuT@MakeLuatexPrimitive{savecatcodetable}
\LuT@MakePrimitive{numexpr}
%    \end{macrocode}
%
% \subsection{Inherit support for \eTeX}
%
%    Package \xpackage{etex} is not compatible for \plainTeX.
%    But it could be present if a format is used that is based
%    on \xfile{etex.src}. Therefore we only load the package
%    in case of \LaTeX\ and tests its presence independently
%    of the format by looking for \cs{et@xins}.
%    \begin{macrocode}
\begingroup\expandafter\expandafter\expandafter\endgroup
\expandafter\ifx\csname RequirePackage\endcsname\relax
\else
  \RequirePackage{etex}[1998/03/26]%
\fi
%    \end{macrocode}
%
% \subsection{Adaption of \eTeX's register allocation}
%
%    \eTeX has increased the number of \TeX\ registers
%    from $2^8$ (256) to $2^{15}$ (32768) for a register class.
%    \LuaTeX\ extends the limit further to $2^16$ (65536).
%    The allocation scheme of package \xpackage{etex} is not
%    changed. But this can be subject for discussion.
%
%    If a register class hasn't registered any local registers
%    yet, then the limit can safely be pushed to 65536.
%    \begin{macrocode}
\begingroup\expandafter\expandafter\expandafter\endgroup
\expandafter\ifx\csname et@xins\endcsname\relax
  \@PackageWarningNoLine{luatex}{%
    Support for eTeX is not loaded (etex.src)%
  }%
\else
  \def\LuT@temp#1{%
    \ifnum\count27#1=32768 %
      \count27#1=65536 %
    \fi
  }%
  \LuT@temp0%
  \LuT@temp1%
  \LuT@temp2%
  \LuT@temp3%
  \LuT@temp4%
  \LuT@temp5%
  \LuT@temp6%
%    \end{macrocode}
%
% \eTeX\ uses an array for the first 256 registers and then a tree
% structure. \LuaTeX\ stores all registers of a class in one Lua table.
% There shouldn't be large performance differences.
% This allows starting immediately in the extended
% area, leaving room for insertions.
%    \begin{macrocode}
  \let\newcount\globcount
  \let\newdimen\globdimen
  \let\newskip\globskip
  \let\newbox\globbox
\fi
%    \end{macrocode}
%
% \subsection{\plainTeX\ compatibility}
%
%    \begin{macro}{\@empty}
%    \begin{macrocode}
\expandafter\ifx\csname @empty\endcsname\relax
  \def\@empty{}%
\fi
%    \end{macrocode}
%    \end{macro}
%    \begin{macro}{\@gobble}
%    \begin{macrocode}
\expandafter\ifx\csname @gobble\endcsname\relax
  \long\def\@gobble#1{}%
\fi
%    \end{macrocode}
%    \end{macro}
%    \begin{macro}{\@firstofone}
%    \begin{macrocode}
\expandafter\ifx\csname @firstofone\endcsname\relax
  \long\def\@firstofone#1{#1}%
\fi
%    \end{macrocode}
%    \end{macro}
%    \begin{macro}{\@firstoftwo}
%    \begin{macrocode}
\expandafter\ifx\csname @firstoftwo\endcsname\relax
  \long\def\@firstoftwo#1#2{#1}%
\fi
%    \end{macrocode}
%    \end{macro}
%    \begin{macro}{\@car}
%    \begin{macrocode}
\expandafter\ifx\csname @car\endcsname\relax
  \def\@car#1#2\@nil{#1}%
\fi
%    \end{macrocode}
%    \end{macro}
%    \begin{macro}{\@cdr}
%    \begin{macrocode}
\expandafter\ifx\csname @cdr\endcsname\relax
  \def\@cdr#1#2\@nil{#2}%
\fi
%    \end{macrocode}
%    \end{macro}
%    \begin{macro}{\@ifstar}
%    \begin{macrocode}
\expandafter\ifx\csname @ifstar\endcsname\relax
  \def\@ifstar#1{%
    \@ifnextchar*{\@firstoftwo{#1}}%
  }%
%    \end{macrocode}
%    \end{macro}
%    \begin{macro}{\@ifnextchar}
%    \begin{macrocode}
  \long\def\@ifnextchar#1#2#3{%
    \let\reserved@d=#1%
    \def\reserved@a{#2}%
    \def\reserved@b{#3}%
    \futurelet\@let@token\@ifnch
  }%
%    \end{macrocode}
%    \end{macro}
%    \begin{macro}{\@ifnch}
%    \begin{macrocode}
  \def\@ifnch{%
    \ifx\@let@token\@sptoken
      \let\reserved@c\@xifnch
    \else
      \ifx\@let@token\reserved@d
        \let\reserved@c\reserved@a
      \else
        \let\reserved@c\reserved@b
      \fi
    \fi
    \reserved@c
  }%
%    \end{macrocode}
%    \end{macro}
%    \begin{macro}{\@sptoken}
%    \begin{macrocode}
  \let\LuT@temp\:%
  \def\:{\let\@sptoken= }%
  \: % explicit space
%    \end{macrocode}
%    \end{macro}
%    \begin{macro}{\@xifnch}
%    \begin{macrocode}
  \def\:{\@xifnch}%
  \expandafter\def\: {%
    \futurelet\@let@token\@ifnch
  }%
  \let\:\LuT@temp
\fi
%    \end{macrocode}
%    \end{macro}
%
%    \begin{macro}{\@tempcnta}
%    \begin{macrocode}
\expandafter\ifx\csname @tempcnta\endcsname\relax
  \csname newcount\endcsname\@tempcnta
\fi
%    \end{macrocode}
%    \end{macro}
%
%    \begin{macro}{\@tempcntb}
%    \begin{macrocode}
\expandafter\ifx\csname @tempcntb\endcsname\relax
  \csname newcount\endcsname\@tempcntb
\fi
%    \end{macrocode}
%    \end{macro}
%
%    \begin{macro}{\LuT@newcommand}
%    \begin{macrocode}
\begingroup\expandafter\expandafter\expandafter\endgroup
\expandafter\ifx\csname newcommand\endcsname\relax
  \def\LuT@newcommand#1[#2]#3{%
    \ifx#1\@undefined
      \let#1\relax
    \else
      \ifx#1\relax
      \else
        \@PackageError{luatex}{%
          \string#1 is already defined.\MessageBreak
          Redefinition is skipped%
        }\@ehc
      \fi
    \fi
    \ifx#1\relax
      \ifcase#2 %
        \def#1{#3}%
      \or
        \def#1##1{#3}%
      \or
        \def#1##1##2{#3}%
      \or
        \def#1##1##2##3{#3}%
      \or
        \@INTERNAL@ERROR
      \fi
    \fi
  }%
\else
  \def\LuT@newcommand{\newcommand*}%
\fi
%    \end{macrocode}
%    \end{macro}
%
% \subsection{Attributes}
%
% \subsubsection{Allocation}
%
%    \begin{macro}{\LuT@AllocAttribute}
%    \begin{macrocode}
\newcount\LuT@AllocAttribute
\LuT@AllocAttribute=\m@ne
%    \end{macrocode}
%    \end{macro}
%    \begin{macro}{\newattribute}
%    \begin{macrocode}
\LuT@newcommand\newattribute[1]{%
  \ifnum\LuT@AllocAttribute<65535 %
    \global\advance\LuT@AllocAttribute\@ne
    \allocationnumber\LuT@AllocAttribute
    \global\luatexattributedef#1=\allocationnumber
    \unsetattribute{#1}%
    \wlog{\string#1=\string\attribute\the\allocationnumber}%
  \else
    \errmessage{No room for a new \string\attribute}%
  \fi
}
%    \end{macrocode}
%    \end{macro}
%
% \subsubsection{Interface}
%
%    \begin{macro}{\setattribute}
%    \begin{macrocode}
\LuT@newcommand\setattribute[2]{%
  #1=\numexpr#2\relax
}
%    \end{macrocode}
%    \end{macro}
%    \begin{macro}{\unsetattribute}
%    \begin{macrocode}
\ifnum\luatexversion<37
  \LuT@newcommand\LuT@UnsetAttributeValue[0]{}%
  \let\LuT@UnsetAttributeValue\m@ne
\else
  \LuT@newcommand\LuT@UnsetAttributeValue[0]{-2147483647 }%
\fi
\LuT@newcommand\unsetattribute[1]{%
  #1=\LuT@UnsetAttributeValue
}
%    \end{macrocode}
%    \end{macro}
%
% \subsection{Catcode tables}
%
% \subsubsection{Allocation}
%
%    \begin{macro}{\LuT@AllocCatcodeTable}
%    \begin{macrocode}
\newcount\LuT@AllocCatcodeTable
\LuT@AllocCatcodeTable=\m@ne
\newcount\CatcodeTableStack
\CatcodeTableStack=\z@
%    \end{macrocode}
%    \end{macro}
%    \begin{macro}{\newcatcodetable}
%    \begin{macrocode}
\LuT@newcommand\newcatcodetable[1]{%
  \ifnum\LuT@AllocCatcodeTable<1114110 % 0x10FFFF is maximal \chardef
    % or < 268435455 % 2^28 - 1
    \global\advance\LuT@AllocCatcodeTable by\tw@
    \allocationnumber=\LuT@AllocCatcodeTable
    \global\chardef#1=\allocationnumber
    \wlog{%
      \string#1=\string\catcodetable\the\allocationnumber
    }%
  \else
    \errmessage{No room for a new \string\catcodetable}%
  \fi
}%
%    \end{macrocode}
%    \end{macro}
%
%    \begin{macro}{\IncCatcodeTableStack}
%    \begin{macrocode}
\LuT@newcommand\IncCatcodeTableStack[0]{%
  \ifnum\CatcodeTableStack<268435454 %
    \global\advance\CatcodeTableStack by\tw@
  \else
    \@PackageError{luatex}{%
      Catcode table stack overflow%
    }\@ehd
  \fi
}
%    \end{macrocode}
%    \end{macro}
%
%    \begin{macro}{\DecCatcodeTableStack}
%    \begin{macrocode}
\LuT@newcommand\DecCatcodeTableStack[0]{%
  \ifnum\CatcodeTableStack>\z@
    \global\advance\CatcodeTableStack by-2 %
  \else
    \@PackageError{luatex}{%
      Catcode table stack is empty%
    }\@ehd
  \fi
}
%    \end{macrocode}
%    \end{macro}
%
% \subsubsection{\cs{SetCatcodeRange}}
%
%    \begin{macro}{\SetCatcodeRange}
%    \begin{macrocode}
\LuT@newcommand\SetCatcodeRange[3]{%
  \edef\LuT@temp{%
    \noexpand\@tempcnta=\the\@tempcnta
    \noexpand\@tempcntb=\the\@tempcntb
    \noexpand\count@=\the\count@
    \relax
  }%
  \@tempcnta=\numexpr#1\relax
  \@tempcntb=\numexpr#2\relax
  \count@=\numexpr#3\relax
  \loop
    \unless\ifnum\@tempcnta>\@tempcntb
    \catcode\@tempcnta=\count@
    \advance\@tempcnta by \@ne
  \repeat
  \LuT@temp
}
%    \end{macrocode}
%    \end{macro}
%
% \subsubsection{Predefined catcode tables}
%
%    \begin{macrocode}
\newcatcodetable\CatcodeTableIniTeX
\newcatcodetable\CatcodeTableString
\newcatcodetable\CatcodeTableOther
\newcatcodetable\CatcodeTableLaTeX
%    \end{macrocode}
%
%    \begin{macrocode}
\luatexinitcatcodetable\CatcodeTableIniTeX
\begingroup
  \def\@makeother#1{\catcode#1=12\relax}%
  \@firstofone{%
    \luatexcatcodetable\CatcodeTableIniTeX
    \begingroup
      \SetCatcodeRange{0}{8}{15}%
      \catcode9=10 % tab
      \catcode11=15 %
      \catcode12=13 % form feed
      \SetCatcodeRange{14}{31}{15}%
      \catcode35=6 % hash
      \catcode36=3 % dollar
      \catcode38=4 % ampersand
      \catcode94=7 % circumflex
      \catcode95=8 % underscore
      \catcode123=1 % brace left
      \catcode125=2 % brace right
      \catcode126=13 % tilde
      \catcode127=15 %
      \luatexsavecatcodetable\CatcodeTableLaTeX
    \endgroup
    \@makeother{0}% nul
    \@makeother{13}% carriage return
    \@makeother{37}% percent
    \@makeother{92}% backslash
    \@makeother{127}%
    \SetCatcodeRange{65}{90}{12}% A-Z
    \SetCatcodeRange{97}{122}{12}% a-z
    \luatexsavecatcodetable\CatcodeTableString
    \@makeother{32}% space
    \luatexsavecatcodetable\CatcodeTableOther
  \endgroup
}%
%    \end{macrocode}
%
% \subsubsection{Number stack}
%
%    \begin{macro}{\LuT@NumStackEmpty}
%    A special empty stack value because of \cs{@cdr}'s brace removal.
%    \begin{macrocode}
\def\LuT@NumStackEmpty{0}
%    \end{macrocode}
%    \end{macro}
%    \begin{macro}{\LuT@NumStack}
%    \begin{macrocode}
\let\LuT@NumStack\LuT@NumStackEmpty
%    \end{macrocode}
%    \end{macro}
%    \begin{macro}{\PushCatcodeTableNumStack}
%    \begin{macrocode}
\LuT@newcommand\PushCatcodeTableNumStack[0]{%
  \xdef\LuT@NumStack{%
    {\the\luatexcatcodetable}\LuT@NumStack
  }%
}
%    \end{macrocode}
%    \end{macro}
%    \begin{macro}{\PopCatcodeTableNumStack}
%    \begin{macrocode}
\LuT@newcommand\PopCatcodeTableNumStack[0]{%
  \ifx\LuT@NumStack\LuT@NumStackEmpty
    \@PackageWarning{luatex}{Empty catcode table number stack}%
    \luatexcatcodetable\z@
  \else
    \luatexcatcodetable=\expandafter\@car\LuT@NumStack\@nil\relax
    \xdef\LuT@NumStack{%
      \expandafter\@cdr\LuT@NumStack\@nil
    }%
  \fi
}
%    \end{macrocode}
%    \end{macro}
%
% \subsubsection{Catcode regime macros}
%
%    \begin{macro}{\BeginCatcodeRegime}
%    \begin{macrocode}
\LuT@newcommand\BeginCatcodeRegime[1]{%
  \PushCatcodeTableNumStack
  \luatexcatcodetable=\numexpr#1\relax
  \IncCatcodeTableStack
  \luatexsavecatcodetable\CatcodeTableStack
  \luatexcatcodetable\CatcodeTableStack
}
%    \end{macrocode}
%    \end{macro}
%
%    \begin{macro}{\EndCatcodeRegime}
%    \begin{macrocode}
\LuT@newcommand\EndCatcodeRegime[0]{%
  \DecCatcodeTableStack
  \PopCatcodeTableNumStack
}
%    \end{macrocode}
%    \end{macro}
%
% \subsection{Lua module loader}
%
%    \begin{macrocode}
\begingroup\expandafter\expandafter\expandafter\endgroup
\expandafter\ifx\csname RequirePackage\endcsname\relax
  \input luatex-loader.sty\relax
\else
  \RequirePackage{luatex-loader}[2010/03/09]%
\fi
%    \end{macrocode}
%    \begin{macrocode}
\LuT@AtEnd%
%</package>
%    \end{macrocode}
%
%    \begin{macrocode}
%<*loader>
%    \end{macrocode}
%
%    Reload check, especially if the package is not used with \LaTeX.
%    \begin{macrocode}
\begingroup\catcode61\catcode48\catcode32=10\relax%
  \catcode13=5 % ^^M
  \endlinechar=13 %
  \catcode35=6 % #
  \catcode39=12 % '
  \catcode44=12 % ,
  \catcode45=12 % -
  \catcode46=12 % .
  \catcode58=12 % :
  \catcode64=11 % @
  \catcode123=1 % {
  \catcode125=2 % }
  \expandafter\let\expandafter\x\csname ver@luatex-loader.sty\endcsname
  \ifx\x\relax % plain-TeX, first loading
  \else
    \def\empty{}%
    \ifx\x\empty % LaTeX, first loading,
      % variable is initialized, but \ProvidesPackage not yet seen
    \else
      \expandafter\ifx\csname PackageInfo\endcsname\relax
        \def\x#1#2{%
          \immediate\write-1{Package #1 Info: #2.}%
        }%
      \else
        \def\x#1#2{\PackageInfo{#1}{#2, stopped}}%
      \fi
      \x{luatex-loader}{The package is already loaded}%
      \aftergroup\endinput
    \fi
  \fi
\endgroup%
%    \end{macrocode}
%    Package identification:
%    \begin{macrocode}
\begingroup\catcode61\catcode48\catcode32=10\relax%
  \catcode13=5 % ^^M
  \endlinechar=13 %
  \catcode35=6 % #
  \catcode39=12 % '
  \catcode40=12 % (
  \catcode41=12 % )
  \catcode44=12 % ,
  \catcode45=12 % -
  \catcode46=12 % .
  \catcode47=12 % /
  \catcode58=12 % :
  \catcode64=11 % @
  \catcode91=12 % [
  \catcode93=12 % ]
  \catcode123=1 % {
  \catcode125=2 % }
  \expandafter\ifx\csname ProvidesPackage\endcsname\relax
    \def\x#1#2#3[#4]{\endgroup
      \immediate\write-1{Package: #3 #4}%
      \xdef#1{#4}%
    }%
  \else
    \def\x#1#2[#3]{\endgroup
      #2[{#3}]%
      \ifx#1\@undefined
        \xdef#1{#3}%
      \fi
      \ifx#1\relax
        \xdef#1{#3}%
      \fi
    }%
  \fi
\expandafter\x\csname ver@luatex-loader.sty\endcsname
\ProvidesPackage{luatex-loader}%
  [2010/03/09 v0.4 Lua module loader (HO)]%
%    \end{macrocode}
%    \begin{macrocode}
\begingroup\catcode61\catcode48\catcode32=10\relax%
  \catcode13=5\endlinechar=13\relax%
  \catcode10=12 % ^^J
  \catcode34=12 % "
  \catcode39=12 % '
  \catcode40=12 % (
  \catcode41=12 % )
  \catcode44=12 % ,
  \catcode46=12 % .
  \catcode60=12 % <
  \catcode61=12 % =
  \catcode95=12 % _ (other!)
  \catcode96=12 % `
  \catcode123=1 % {
  \catcode125=2 % }
  \endlinechar=10 %
  \ifnum\luatexversion<36 %
    \directlua0%
  \else %
    \expandafter\directlua %
  \fi %
  {%
     do
      local script = "oberdiek.luatex.lua"
      local file = kpse.find_file(script, "texmfscripts")
      if file then
        texio.write_nl("(" .. file .. ")")
        dofile(file)
      else
        error("File `" .. script .. "' not found")
      end
    end
  }%
\endgroup\endinput%
%    \end{macrocode}
%
%    \begin{macrocode}
%</loader>
%    \end{macrocode}
%
% \subsection{Lua script}
%
%    Currently \LuaTeX\ does not use KPSE when searching for module
%    files. The following Lua script implements a workaround.
%    It extends |package.loader| by another search method.
%    Modules are found by the module name with extension \xext{lua}
%    similar to
%    \begin{quote}
%      \ttfamily |kpsewhich --format=texmfscripts| \meta{module}|.lua|
%    \end{quote}
%    Unhappily \textsf{kpsewhich} does not support directory components
%    in the file name. Therefore a module |a.b.c| cannot be installed
%    as \xfile{a/b/c.lua}. The script must be named \xfile{a.b.c.lua}.
%    \begin{macrocode}
%<*lua>
%    \end{macrocode}
%    \begin{macrocode}
module("oberdiek.luatex", package.seeall)
%    \end{macrocode}
%    \begin{macrocode}
function kpse_module_loader(module)
  local script = module .. ".lua"
  local file = kpse.find_file(script, "texmfscripts")
  if file then
    local loader, error = loadfile(file)
    if loader then
      texio.write_nl("(" .. file .. ")")
      return loader
    end
    return "\n\t[oberdiek.luatex.kpse_module_loader] Loading error:\n\t"
           .. error
  end
  return "\n\t[oberdiek.luatex.kpse_module_loader] Search failed"
end
table.insert(package.loaders, kpse_module_loader)
%    \end{macrocode}
%    \begin{macrocode}
%</lua>
%    \end{macrocode}
%
% \section{Test}
%
%    \begin{macrocode}
%<*test2>
\documentclass{article}
\def\LoadCommand{%
  \RequirePackage{luatex}[2010/03/09]%
}
%</test2>
%    \end{macrocode}
%    \begin{macrocode}
%<*test3>
\documentclass{article}
\def\LoadCommand{%
  \RequirePackage{luatex-loader}[2010/03/09]%
}
%</test3>
%    \end{macrocode}
% \subsection{Catcode checks for loading}
%
%    \begin{macrocode}
%<*test1>
%    \end{macrocode}
%    \begin{macrocode}
\catcode`\{=1 %
\catcode`\}=2 %
\catcode`\#=6 %
\catcode`\@=11 %
\expandafter\ifx\csname count@\endcsname\relax
  \countdef\count@=255 %
\fi
\expandafter\ifx\csname @gobble\endcsname\relax
  \long\def\@gobble#1{}%
\fi
\expandafter\ifx\csname @firstofone\endcsname\relax
  \long\def\@firstofone#1{#1}%
\fi
\expandafter\ifx\csname loop\endcsname\relax
  \expandafter\@firstofone
\else
  \expandafter\@gobble
\fi
{%
  \def\loop#1\repeat{%
    \def\body{#1}%
    \iterate
  }%
  \def\iterate{%
    \body
      \let\next\iterate
    \else
      \let\next\relax
    \fi
    \next
  }%
  \let\repeat=\fi
}%
\def\RestoreCatcodes{}
\count@=0 %
\loop
  \edef\RestoreCatcodes{%
    \RestoreCatcodes
    \catcode\the\count@=\the\catcode\count@\relax
  }%
\ifnum\count@<255 %
  \advance\count@ 1 %
\repeat

\def\RangeCatcodeInvalid#1#2{%
  \count@=#1\relax
  \loop
    \catcode\count@=15 %
  \ifnum\count@<#2\relax
    \advance\count@ 1 %
  \repeat
}
\def\RangeCatcodeCheck#1#2#3{%
  \count@=#1\relax
  \loop
    \ifnum#3=\catcode\count@
    \else
      \errmessage{%
        Character \the\count@\space
        with wrong catcode \the\catcode\count@\space
        instead of \number#3%
      }%
    \fi
  \ifnum\count@<#2\relax
    \advance\count@ 1 %
  \repeat
}
\def\space{ }
\expandafter\ifx\csname LoadCommand\endcsname\relax
  \def\LoadCommand{\input luatex.sty\relax}%
\fi
\def\Test{%
  \RangeCatcodeInvalid{0}{47}%
  \RangeCatcodeInvalid{58}{64}%
  \RangeCatcodeInvalid{91}{96}%
  \RangeCatcodeInvalid{123}{255}%
  \catcode`\@=12 %
  \catcode`\\=0 %
  \catcode`\%=14 %
  \LoadCommand
  \RangeCatcodeCheck{0}{36}{15}%
  \RangeCatcodeCheck{37}{37}{14}%
  \RangeCatcodeCheck{38}{47}{15}%
  \RangeCatcodeCheck{48}{57}{12}%
  \RangeCatcodeCheck{58}{63}{15}%
  \RangeCatcodeCheck{64}{64}{12}%
  \RangeCatcodeCheck{65}{90}{11}%
  \RangeCatcodeCheck{91}{91}{15}%
  \RangeCatcodeCheck{92}{92}{0}%
  \RangeCatcodeCheck{93}{96}{15}%
  \RangeCatcodeCheck{97}{122}{11}%
  \RangeCatcodeCheck{123}{255}{15}%
  \RestoreCatcodes
}
\Test
\csname @@end\endcsname
\end
%    \end{macrocode}
%    \begin{macrocode}
%</test1>
%    \end{macrocode}
%
% \subsection{Catcode tables}
%
% \subsubsection{Predefined catcode tables}
%
%    \begin{macrocode}
%<*test4>
\NeedsTeXFormat{LaTeX2e}
%    \end{macrocode}
%    Remember \LaTeX's initial catcodes in count registers
%    starting at \cs{TestLaTeX}.
%    \begin{macrocode}
\count0=0 %
\chardef\TestLaTeX=1000 %
\chardef\TestMax=300 %
\loop
  \count\numexpr\TestLaTeX+\count0\relax=\catcode\count0 %
\ifnum\count0<\TestMax
  \advance\count0 by 1 %
\repeat
\documentclass{minimal}
\usepackage{luatex}[2010/03/09]
\usepackage{qstest}
\IncludeTests{*}
\LogTests{log}{*}{*}
\makeatletter
\def\Check#1{%
  \Expect*{\the\count@=\the\catcode\count@}%
         *{\the\count@=#1}%
}
\newcount\scratch
\def\Test#1#2{%
  \begin{qstest}{CatcodeTable#1}{CatcodeTable#1}%
    \luatexcatcodetable\csname CatcodeTable#1\endcsname
    \count@=\z@
    \loop
      \scratch=#2\relax
      \Expect*{\the\count@=\the\catcode\count@}%
             *{\the\count@=\the\scratch}%
    \ifnum\count@<\TestMax
      \advance\count@\@ne
    \repeat
  \end{qstest}%
}
\begingroup
  % luatex-unicode-letters.tex makes some slots to letters
  \def\TestMax{169}%
  \Test{LaTeX}{\the\count\numexpr\TestLaTeX+\count@}%
\endgroup
\Test{String}{\ifnum\count@=32 10\else 12\fi}
\Test{Other}{12}
\luatexinitcatcodetable99 %
\Test{IniTeX}{%
  0\relax
  \begingroup
    \luatexcatcodetable99 %
    \global\scratch=\the\catcode\count@
  \endgroup
}
%    \end{macrocode}
%
% \subsubsection{Catcode table number stack}
%
%    \begin{macrocode}
\begin{qstest}{CatcodeTableNumStack}{CatcodeTableNumStack}
  \def\TestStack#1{%
    \Expect*{\LuT@NumStack}{#1}%
  }%
  \TestStack{0}%
  \PushCatcodeTableNumStack
  \TestStack{{0}0}%
  \@firstofone{%
    \begingroup
      \luatexinitcatcodetable12 %
      \luatexcatcodetable12 %
      \PushCatcodeTableNumStack
      \TestStack{{12}{0}0}%
      \PopCatcodeTableNumStack
      \TestStack{{0}0}%
      \PopCatcodeTableNumStack
      \TestStack{0}%
      \def\TestWarning{Missing empty stack warning}%
      \def\@PackageWarning#1#2{\def\TestWarning{empty stack}}%
      \PopCatcodeTableNumStack
      \TestStack{0}%
      \Expect*{\TestWarning}{empty stack}%
    \endgroup
  }%
\end{qstest}
%    \end{macrocode}
%
% \subsubsection{Catcode table stack}
%
%    \begin{macrocode}
\begin{qstest}{CatcodeTableStack}{CatcodeTableStack}
  \def\TestStack#1{%
    \Expect*{\the\CatcodeTableStack}{#1}%
  }%
  \TestStack{0}%
  \IncCatcodeTableStack
  \TestStack{2}%
  \IncCatcodeTableStack
  \TestStack{4}%
  \begingroup
    \IncCatcodeTableStack
    \TestStack{6}%
  \endgroup
  \TestStack{6}%
  \begingroup
    \DecCatcodeTableStack
    \TestStack{4}%
  \endgroup
  \TestStack{4}%
  \DecCatcodeTableStack
  \TestStack{2}%
  \DecCatcodeTableStack
  \TestStack{0}%
  \begingroup
    \def\TestError{Missing error}%
    \def\@PackageError#1#2#3{%
      \def\TestError{Empty stack}%
    }%
    \DecCatcodeTableStack
    \TestStack{0}%
    \Expect*{\TestError}{Empty stack}%
  \endgroup
\end{qstest}
%    \end{macrocode}
%
% \subsubsection{Catcode regime macros}
%
%    \begin{macrocode}
\begin{qstest}{CatcodeRegime}{CatcodeRegime}
  \def\TestStacks#1#2#3{%
    \Expect*{\the\luatexcatcodetable}{#1}%
    \Expect*{\the\CatcodeTableStack}{#2}%
    \Expect*{\LuT@NumStack}{#3}%
  }%
  \TestStacks{0}{0}{0}%
  \catcode`\|=7 %
  \BeginCatcodeRegime\CatcodeTableLaTeX
    \TestStacks{2}{2}{{0}0}%
    \Expect*{\the\catcode`\|}{12}%
  \EndCatcodeRegime
  \TestStacks{0}{0}{0}%
  \Expect*{\the\catcode`\|}{7}%
\end{qstest}
%    \end{macrocode}
%
% \subsection{Attribute allocation}
%
%    \begin{macrocode}
\begin{qstest}{Attributes}{Attributes}
  \newattribute\TestAttr
  \Expect*{\meaning\TestAttr}%
         *{\string\attribute\number\allocationnumber}%
  \Expect*{\the\allocationnumber}{0}%
  \begingroup
    \newattribute\TestAttr
    \Expect*{\the\allocationnumber}{1}%
  \endgroup
  \Expect*{\the\allocationnumber}{0}%
  \Expect*{\meaning\TestAttr}*{\string\attribute1}%
  \Expect*{\the\TestAttr}*{\number\LuT@UnsetAttributeValue}%
  \def\Test#1{%
    \setattribute\TestAttr{#1}%
    \Expect*{\the\TestAttr}{#1}%
  }%
  \Test{0}%
  \Test{1}%
  \Test{-1}%
  \Test{123}%
  \unsetattribute\TestAttr
  \Expect*{\the\TestAttr}*{\number\LuT@UnsetAttributeValue}%
  \begingroup
    \Expect*{\the\TestAttr}*{\number\LuT@UnsetAttributeValue}%
    \Test{1234}%
  \endgroup
  \Expect*{\the\TestAttr}*{\number\LuT@UnsetAttributeValue}%
\end{qstest}
%    \end{macrocode}
%
%    \begin{macrocode}
\@@end
%</test4>
%    \end{macrocode}
%
% \subsection{Short test for \plainTeX}
%
%    \begin{macrocode}
%<*test5>
\input luatex.sty\relax
\newattribute\TestAttr
\setattribute\TestAttr{10}
\unsetattribute\TestAttr
\newcatcodetable\TestCTa
\begingroup
  \SetCatcodeRange{`A}{`Z}{12}%
\endgroup
\BeginCatcodeRegime\CatcodeTableLaTeX
\EndCatcodeRegime
\end
%</test5>
%    \end{macrocode}
%
% \section{Installation}
%
% \subsection{Download}
%
% \paragraph{Package.} This package is available on
% CTAN\footnote{\url{ftp://ftp.ctan.org/tex-archive/}}:
% \begin{description}
% \item[\CTAN{macros/latex/contrib/oberdiek/luatex.dtx}] The source file.
% \item[\CTAN{macros/latex/contrib/oberdiek/luatex.pdf}] Documentation.
% \end{description}
%
%
% \paragraph{Bundle.} All the packages of the bundle `oberdiek'
% are also available in a TDS compliant ZIP archive. There
% the packages are already unpacked and the documentation files
% are generated. The files and directories obey the TDS standard.
% \begin{description}
% \item[\CTAN{install/macros/latex/contrib/oberdiek.tds.zip}]
% \end{description}
% \emph{TDS} refers to the standard ``A Directory Structure
% for \TeX\ Files'' (\CTAN{tds/tds.pdf}). Directories
% with \xfile{texmf} in their name are usually organized this way.
%
% \subsection{Bundle installation}
%
% \paragraph{Unpacking.} Unpack the \xfile{oberdiek.tds.zip} in the
% TDS tree (also known as \xfile{texmf} tree) of your choice.
% Example (linux):
% \begin{quote}
%   |unzip oberdiek.tds.zip -d ~/texmf|
% \end{quote}
%
% \paragraph{Script installation.}
% Check the directory \xfile{TDS:scripts/oberdiek/} for
% scripts that need further installation steps.
% Package \xpackage{attachfile2} comes with the Perl script
% \xfile{pdfatfi.pl} that should be installed in such a way
% that it can be called as \texttt{pdfatfi}.
% Example (linux):
% \begin{quote}
%   |chmod +x scripts/oberdiek/pdfatfi.pl|\\
%   |cp scripts/oberdiek/pdfatfi.pl /usr/local/bin/|
% \end{quote}
%
% \subsection{Package installation}
%
% \paragraph{Unpacking.} The \xfile{.dtx} file is a self-extracting
% \docstrip\ archive. The files are extracted by running the
% \xfile{.dtx} through \plainTeX:
% \begin{quote}
%   \verb|tex luatex.dtx|
% \end{quote}
%
% \paragraph{TDS.} Now the different files must be moved into
% the different directories in your installation TDS tree
% (also known as \xfile{texmf} tree):
% \begin{quote}
% \def\t{^^A
% \begin{tabular}{@{}>{\ttfamily}l@{ $\rightarrow$ }>{\ttfamily}l@{}}
%   luatex.sty & tex/generic/oberdiek/luatex.sty\\
%   luatex-loader.sty & tex/generic/oberdiek/luatex-loader.sty\\
%   oberdiek.luatex.lua & scripts/oberdiek/oberdiek.luatex.lua\\
%   luatex.pdf & doc/latex/oberdiek/luatex.pdf\\
%   test/luatex-test1.tex & doc/latex/oberdiek/test/luatex-test1.tex\\
%   test/luatex-test2.tex & doc/latex/oberdiek/test/luatex-test2.tex\\
%   test/luatex-test3.tex & doc/latex/oberdiek/test/luatex-test3.tex\\
%   test/luatex-test4.tex & doc/latex/oberdiek/test/luatex-test4.tex\\
%   test/luatex-test5.tex & doc/latex/oberdiek/test/luatex-test5.tex\\
%   luatex.dtx & source/latex/oberdiek/luatex.dtx\\
% \end{tabular}^^A
% }^^A
% \sbox0{\t}^^A
% \ifdim\wd0>\linewidth
%   \begingroup
%     \advance\linewidth by\leftmargin
%     \advance\linewidth by\rightmargin
%   \edef\x{\endgroup
%     \def\noexpand\lw{\the\linewidth}^^A
%   }\x
%   \def\lwbox{^^A
%     \leavevmode
%     \hbox to \linewidth{^^A
%       \kern-\leftmargin\relax
%       \hss
%       \usebox0
%       \hss
%       \kern-\rightmargin\relax
%     }^^A
%   }^^A
%   \ifdim\wd0>\lw
%     \sbox0{\small\t}^^A
%     \ifdim\wd0>\linewidth
%       \ifdim\wd0>\lw
%         \sbox0{\footnotesize\t}^^A
%         \ifdim\wd0>\linewidth
%           \ifdim\wd0>\lw
%             \sbox0{\scriptsize\t}^^A
%             \ifdim\wd0>\linewidth
%               \ifdim\wd0>\lw
%                 \sbox0{\tiny\t}^^A
%                 \ifdim\wd0>\linewidth
%                   \lwbox
%                 \else
%                   \usebox0
%                 \fi
%               \else
%                 \lwbox
%               \fi
%             \else
%               \usebox0
%             \fi
%           \else
%             \lwbox
%           \fi
%         \else
%           \usebox0
%         \fi
%       \else
%         \lwbox
%       \fi
%     \else
%       \usebox0
%     \fi
%   \else
%     \lwbox
%   \fi
% \else
%   \usebox0
% \fi
% \end{quote}
% If you have a \xfile{docstrip.cfg} that configures and enables \docstrip's
% TDS installing feature, then some files can already be in the right
% place, see the documentation of \docstrip.
%
% \subsection{Refresh file name databases}
%
% If your \TeX~distribution
% (\teTeX, \mikTeX, \dots) relies on file name databases, you must refresh
% these. For example, \teTeX\ users run \verb|texhash| or
% \verb|mktexlsr|.
%
% \subsection{Some details for the interested}
%
% \paragraph{Attached source.}
%
% The PDF documentation on CTAN also includes the
% \xfile{.dtx} source file. It can be extracted by
% AcrobatReader 6 or higher. Another option is \textsf{pdftk},
% e.g. unpack the file into the current directory:
% \begin{quote}
%   \verb|pdftk luatex.pdf unpack_files output .|
% \end{quote}
%
% \paragraph{Unpacking with \LaTeX.}
% The \xfile{.dtx} chooses its action depending on the format:
% \begin{description}
% \item[\plainTeX:] Run \docstrip\ and extract the files.
% \item[\LaTeX:] Generate the documentation.
% \end{description}
% If you insist on using \LaTeX\ for \docstrip\ (really,
% \docstrip\ does not need \LaTeX), then inform the autodetect routine
% about your intention:
% \begin{quote}
%   \verb|latex \let\install=y% \iffalse meta-comment
%
% File: luatex.dtx
% Version: 2010/03/09 v0.4
% Info: LuaTeX basic definition package
%
% Copyright (C) 2007, 2009, 2010 by
%    Heiko Oberdiek <heiko.oberdiek at googlemail.com>
%
% This work may be distributed and/or modified under the
% conditions of the LaTeX Project Public License, either
% version 1.3c of this license or (at your option) any later
% version. This version of this license is in
%    http://www.latex-project.org/lppl/lppl-1-3c.txt
% and the latest version of this license is in
%    http://www.latex-project.org/lppl.txt
% and version 1.3 or later is part of all distributions of
% LaTeX version 2005/12/01 or later.
%
% This work has the LPPL maintenance status "maintained".
%
% This Current Maintainer of this work is Heiko Oberdiek.
%
% The Base Interpreter refers to any `TeX-Format',
% because some files are installed in TDS:tex/generic//.
%
% This work consists of the main source file luatex.dtx
% and the derived files
%    luatex.sty, luatex.pdf, luatex.ins, luatex.drv, luatex-loader.sty,
%    luatex-test1.tex, luatex-test2.tex, luatex-test3.tex,
%    luatex-test4.tex, luatex-test5.tex, oberdiek.luatex.lua.
%
% Distribution:
%    CTAN:macros/latex/contrib/oberdiek/luatex.dtx
%    CTAN:macros/latex/contrib/oberdiek/luatex.pdf
%
% Unpacking:
%    (a) If luatex.ins is present:
%           tex luatex.ins
%    (b) Without luatex.ins:
%           tex luatex.dtx
%    (c) If you insist on using LaTeX
%           latex \let\install=y\input{luatex.dtx}
%        (quote the arguments according to the demands of your shell)
%
% Documentation:
%    (a) If luatex.drv is present:
%           latex luatex.drv
%    (b) Without luatex.drv:
%           latex luatex.dtx; ...
%    The class ltxdoc loads the configuration file ltxdoc.cfg
%    if available. Here you can specify further options, e.g.
%    use A4 as paper format:
%       \PassOptionsToClass{a4paper}{article}
%
%    Programm calls to get the documentation (example):
%       pdflatex luatex.dtx
%       makeindex -s gind.ist luatex.idx
%       pdflatex luatex.dtx
%       makeindex -s gind.ist luatex.idx
%       pdflatex luatex.dtx
%
% Installation:
%    TDS:tex/generic/oberdiek/luatex.sty
%    TDS:tex/generic/oberdiek/luatex-loader.sty
%    TDS:scripts/oberdiek/oberdiek.luatex.lua
%    TDS:doc/latex/oberdiek/luatex.pdf
%    TDS:doc/latex/oberdiek/test/luatex-test1.tex
%    TDS:doc/latex/oberdiek/test/luatex-test2.tex
%    TDS:doc/latex/oberdiek/test/luatex-test3.tex
%    TDS:doc/latex/oberdiek/test/luatex-test4.tex
%    TDS:doc/latex/oberdiek/test/luatex-test5.tex
%    TDS:source/latex/oberdiek/luatex.dtx
%
%<*ignore>
\begingroup
  \catcode123=1 %
  \catcode125=2 %
  \def\x{LaTeX2e}%
\expandafter\endgroup
\ifcase 0\ifx\install y1\fi\expandafter
         \ifx\csname processbatchFile\endcsname\relax\else1\fi
         \ifx\fmtname\x\else 1\fi\relax
\else\csname fi\endcsname
%</ignore>
%<*install>
\input docstrip.tex
\Msg{************************************************************************}
\Msg{* Installation}
\Msg{* Package: luatex 2010/03/09 v0.4 LuaTeX basic definition package (HO)}
\Msg{************************************************************************}

\keepsilent
\askforoverwritefalse

\let\MetaPrefix\relax
\preamble

This is a generated file.

Project: luatex
Version: 2010/03/09 v0.4

Copyright (C) 2007, 2009, 2010 by
   Heiko Oberdiek <heiko.oberdiek at googlemail.com>

This work may be distributed and/or modified under the
conditions of the LaTeX Project Public License, either
version 1.3c of this license or (at your option) any later
version. This version of this license is in
   http://www.latex-project.org/lppl/lppl-1-3c.txt
and the latest version of this license is in
   http://www.latex-project.org/lppl.txt
and version 1.3 or later is part of all distributions of
LaTeX version 2005/12/01 or later.

This work has the LPPL maintenance status "maintained".

This Current Maintainer of this work is Heiko Oberdiek.

The Base Interpreter refers to any `TeX-Format',
because some files are installed in TDS:tex/generic//.

This work consists of the main source file luatex.dtx
and the derived files
   luatex.sty, luatex.pdf, luatex.ins, luatex.drv, luatex-loader.sty,
   luatex-test1.tex, luatex-test2.tex, luatex-test3.tex,
   luatex-test4.tex, luatex-test5.tex, oberdiek.luatex.lua.

\endpreamble
\let\MetaPrefix\DoubleperCent

\generate{%
  \file{luatex.ins}{\from{luatex.dtx}{install}}%
  \file{luatex.drv}{\from{luatex.dtx}{driver}}%
  \usedir{tex/generic/oberdiek}%
  \file{luatex.sty}{\from{luatex.dtx}{package}}%
  \file{luatex-loader.sty}{\from{luatex.dtx}{loader}}%
  \usedir{doc/latex/oberdiek/test}%
  \file{luatex-test1.tex}{\from{luatex.dtx}{test1}}%
  \file{luatex-test2.tex}{\from{luatex.dtx}{test1,test2}}%
  \file{luatex-test3.tex}{\from{luatex.dtx}{test1,test3}}%
  \file{luatex-test4.tex}{\from{luatex.dtx}{test4}}%
  \file{luatex-test5.tex}{\from{luatex.dtx}{test5}}%
  \nopreamble
  \nopostamble
  \usedir{source/latex/oberdiek/catalogue}%
  \file{luatex.xml}{\from{luatex.dtx}{catalogue}}%
}
\def\MetaPrefix{-- }
\def\defaultpostamble{%
  \MetaPrefix^^J%
  \MetaPrefix\space End of File `\outFileName'.%
}
\def\currentpostamble{\defaultpostamble}%
\generate{%
  \usedir{scripts/oberdiek}%
  \file{oberdiek.luatex.lua}{\from{luatex.dtx}{lua}}%
}

\catcode32=13\relax% active space
\let =\space%
\Msg{************************************************************************}
\Msg{*}
\Msg{* To finish the installation you have to move the following}
\Msg{* files into a directory searched by TeX:}
\Msg{*}
\Msg{*     luatex.sty, luatex-loader.sty}
\Msg{*}
\Msg{* And install the following script file:}
\Msg{*}
\Msg{*     oberdiek.luatex.lua}
\Msg{*}
\Msg{* To produce the documentation run the file `luatex.drv'}
\Msg{* through LaTeX.}
\Msg{*}
\Msg{* Happy TeXing!}
\Msg{*}
\Msg{************************************************************************}

\endbatchfile
%</install>
%<*ignore>
\fi
%</ignore>
%<*driver>
\NeedsTeXFormat{LaTeX2e}
\ProvidesFile{luatex.drv}%
  [2010/03/09 v0.4 LuaTeX basic definition package (HO)]%
\documentclass{ltxdoc}
\usepackage{holtxdoc}[2011/11/22]
\usepackage{paralist}
\begin{document}
  \DocInput{luatex.dtx}%
\end{document}
%</driver>
% \fi
%
% \CheckSum{1429}
%
% \CharacterTable
%  {Upper-case    \A\B\C\D\E\F\G\H\I\J\K\L\M\N\O\P\Q\R\S\T\U\V\W\X\Y\Z
%   Lower-case    \a\b\c\d\e\f\g\h\i\j\k\l\m\n\o\p\q\r\s\t\u\v\w\x\y\z
%   Digits        \0\1\2\3\4\5\6\7\8\9
%   Exclamation   \!     Double quote  \"     Hash (number) \#
%   Dollar        \$     Percent       \%     Ampersand     \&
%   Acute accent  \'     Left paren    \(     Right paren   \)
%   Asterisk      \*     Plus          \+     Comma         \,
%   Minus         \-     Point         \.     Solidus       \/
%   Colon         \:     Semicolon     \;     Less than     \<
%   Equals        \=     Greater than  \>     Question mark \?
%   Commercial at \@     Left bracket  \[     Backslash     \\
%   Right bracket \]     Circumflex    \^     Underscore    \_
%   Grave accent  \`     Left brace    \{     Vertical bar  \|
%   Right brace   \}     Tilde         \~}
%
% \GetFileInfo{luatex.drv}
%
% \title{The \xpackage{luatex} package}
% \date{2010/03/09 v0.4}
% \author{Heiko Oberdiek\\\xemail{heiko.oberdiek at googlemail.com}}
%
% \maketitle
%
% \begin{abstract}
% This package manages the new and extended features and resources
% that \LuaTeX\ provides. Examples are attributes and catcode
% tables.
% \end{abstract}
%
% \tableofcontents
%
% \section{Documentation}
%
% \subsection{Introduction}
%
% \TeX\ provides global resources such as registers.
% But it does not provide an interface for managing these resources.
% For example, two packages want to use a counter register.
% If they take the same register
% number, then the use of both packages will conflict and they cannot
% be used together.
% Therefore formats such as \plainTeX\ or \LaTeX\ implement
% an allocation scheme for registers. A package reserves with
% \cs{newcount} an unused register number for its own exclusive use.
%
% Nowadays \TeX\ is not alone anymore: \eTeX, \pdfTeX\ and other
% compilers for \TeX\ are developed that extend and add new
% features and resources.
%
% Now \LuaTeX\ has reached beta state. It inherits most of
% \pdfTeX's features including \eTeX. Also it implements
% new concepts such as attributes or catcode tables.
%
% \subsubsection{\LaTeX}
%
% \LaTeXe\ is frozen and therefore refuses to even notice
% the new \TeX\ variants. Not even the old \eTeX\ is supported
% by its kernel. At least there is a third party package \xpackage{etex}
% that manages the new \eTeX\ resources.
%
% This package tries to do the same for \LuaTeX\ and starts to
% support at least a few of the new features.
%
% \subsubsection{\plainTeX}
%
% \LaTeX\ has inherited its resource handling from \plainTeX.
% The interface is basically the same: \cs{newcount}, \dots
% Therefore this package tries to follow this tradition by
% providing compatibility to \plainTeX. It can be loaded
% with \plainTeX\ and defines at least some of the
% features that this packages provides for \LaTeX.
%
% \subsection{Register allocation}
%
% \subsubsection{Register with 16 bit}
%
% Because \LuaTeX\ is a super set of \eTeX\ regarding registers,
% the register allocation scheme should not conflict with
% package \xpackage{etex}. Therefore this package is loaded
% to inherit its allocation scheme. The only change is currently
% that the limit is increased to 65536 registers for the following
% register classes:
% \begin{compactitem}
% \item count
% \item dimen
% \item skip
% \item muskip
% \item marks
% \item toks
% \item box
% \end{compactitem}
% This affects the number of global and local registers.
% Because it is done in a package and not in the kernel,
% it is possible that someone loads package \xpackage{etex}
% before uses the local allocation variants. This will prevent
% the extension for this register class. If more registers are
% needed, just load package \xpackage{luatex} earlier.
%
% \subsubsection{Insertions}
%
% Insertions need four registers \cs{count}, \cs{dimen}, \cs{skip},
% and \cs{box} with the same number. Usually they are allocated
% downwards from 254, 253, \dots Also \cs{newcount}, \cs{newdimen}, \dots
% fill up these register numbers from below before switching to
% higher register numbers by package \xpackage{etex}.
% When this occurs, no insertions can be allocated anymore.
%
% Therefore \cs{newcount}, \cs{newdimen}, \cs{newskip}, and
% \cs{newbox} are replaced by their global variants (\cs{globcount}, \dots)
% that use the higher numbers immediately, leaving the room for insertions.
% There should not be an efficiency penalty because \LuaTeX\ stores
% the registers of a class in the same Lua table unlike \eTeX, where
% registers below 256 are stored in an array and higher numbers
% are put in a tree structure.
%
% \subsection{Attributes}
%
% Nodes can have custom attributes in \LuaTeX. These attributes
% are organized by a new register class. As the other registers
% up to $2^{16}$ attributes are supported. An attribute value
% can be negative that means the attribute is not set. Otherwise
% \TeX's range of non-negative integers up to $2^31$ are available.
%
% \begin{declcs}{newattribute} \M{cmd}
% \end{declcs}
% Macro \cs{newattribute} defines command \meta{cmd} using \cs{attributedef}
% using an new attribute number. The new attribute is initially unset.
%
% \begin{declcs}{setattribute} \M{cmd} \M{value}
% \end{declcs}
% Macro \cs{setattribute} locally sets attribute command \meta{cmd}
% to the number \meta{value}. Valid values range from $-1$ until $2^31$
% (the upper limit is the same as for other \TeX\ integer numbers).
%
% \begin{declcs}{unsetattribute} \M{cmd}
% \end{declcs}
% Macro \cs{unsettattribute} clears the attribute command \meta{cmd}.
%
% \subsection{Catcode tables}
%
% \LuaTeX\ introduces catcode tables as new feature, see documentation.
% There is need for discussion, how to deal best:
% \begin{itemize}
% \item \cs{initcatcodetable} and \cs{setcatcodetable} act globally.
% \item \cs{catcodetable} causes an error if used with an uninitialized
%   catcode table.
% \item Large catcode table numbers should be avoided because of
%   performance breakdown.
% \item Use case \LaTeX\ package:
%   The package must not be surprised by changed catcodes and
%   must not surprise by changing catcodes accidently.
%   Catcode tables could offer a solution. At the begin
%   a catcode regime with standard catcodes is established
%   and the old one is restored afterwards.
% \item Use case: \LuaTeX's |tex.print| might be used with
%   a catcode table number, for example a table where all
%   entries have catcode ``other''.
% \item Readonly catcode tables.
% \item Is there is a need for local allocations?
%   (Package \xpackage{etex}'s \cs{loc} variants are not used in
%   \TeX\ Live 2007.)
% \end{itemize}
%
% \subsubsection{Interface proposal}
%
% The idea: \cs{newcatcodetable} allocates odd numbered catcode
% tables. Even numbered tables are managed as stack.
% Also some catcode tables are defined. These must not be
% changed.
%
% \begin{declcs}{newcatcodetable} \M{cmd}
% \end{declcs}
% Macro \cs{newcatcodetable} reserves a new catcode table and
% remembers its number in \meta{cmd}. The catcode table is
% initialized with \iniTeX's catcodes.
%
% \begin{declcs}{CatcodeTableIniTeX}\\
%   \cs{CatcodeTableString}\\
%   \cs{CatcodeTableOther}\\
%   \cs{CatcodeTableLaTeX}\\
% \end{declcs}
% These are catcode tables and must not be changed.
% \cs{CatcodeTableIniTeX} contains the catcode settings of \iniTeX.
% \cs{CatcodeTableString} follows \TeX's convention of
% \cs{string}, \cs{meaning} and friends. The space gets catcode
% 10 (space), the other characters have catcode 12 (other).
% In \cs{CatcodeTableOther} all entries have catcode 12 (other).
% \cs{CatcodeTableLaTeX} contains the setting of a pure \LaTeX\ format
% (`at' is other).
%
% \begin{declcs}{CatcodeTableStack}\\
%   \cs{IncCatcodeTableStack}\\
%   \cs{DecCatcodeTableStack}
% \end{declcs}
% \cs{CatcodeTableStack} is the stack pointer. Initially
% it is catcode table zero. \cs{IncCatcodeTableStack} and
% \cs{DecCatcodeTableStack} increments and decrements the
% stack pointer. Currently \cs{IncCatcodeTableStack} does not
% initialize a new catcode table. Both increment and decrement
% operations do not set a catcode table.
%
% \begin{declcs}{PushCatcodeTableNumStack}\\
%   \cs{PopCatcodeTableNumStack}
% \end{declcs}
% It can be handy to have a global stack for catcode table
% numbers to deal with the global assignment property of
% \cs{initcatcodetable} and \cs{savecatcodetable}.
% \cs{PushCatcodeTableNumStack} pushes the current catcode table
% on the stack.
% \cs{PopCatcodeTableNumStack} pops the topmost number off the number
% stack to set the current catcode table. Catcode table zero is
% used in case of an empty stack.
%
% \begin{declcs}{BeginCatcodeRegime} \M{catcodetable}\\
%   \cs{EndCatcodeRegime}
% \end{declcs}
% \cs{BeginCatcodeRegime} remembers the current catcode table number.
% Then it creates and uses a fresh catcode table on the stack
% that is initialized by \meta{catcodetable}:
% \begin{quote}\ttfamily
%   \cs{PushCatcodeTableNumStack}\\
%   \cs{catcodetable}\meta{catcodetable}||
%   \cs{IncCatcodeTableStack}\\
%   \cs{savecatcodetable}\cs{CatcodeTableStack}\\
%   \cs{catcodetable}\cs{CatcodeTableStack}
% \end{quote}
% \cs{EndCatcodeRegime} drops the catcode table, created by
% \cs{BeginCatcodeRegime} and sets the catcode table that was
% active before:
% \begin{quote}
%   \cs{DecCatcodeTableStack}\\
%   \cs{PopCatcodeTableNumStack}
% \end{quote}
% These macros solve the use case, described earlier for a \LaTeX\ package:
% \begin{quote}
%   |% package foobar.sty|\\
%   |\BeginCatcodeRegime\CatcodeTableLaTeX|\\
%   |\makeatletter|\\
%   |% ... package contents ...|\\
%   |\EndCatcodeRegime|\\
%   |% end of package|
% \end{quote}
% If the package wants to change catcodes after its loading,
% \cs{AtBeginDocument} or \cs{AtEndOfPackage} can be used.
%
% \begin{declcs}{SetCatcodeRange} \M{from} \M{to} \M{catcode}
% \end{declcs}
% The catcodes of characters in range from \meta{from}
% to inclusive \meta{to} are set to \meta{catcode}.
%
% \subsection{Lua module loading}
%
% Currently \LuaTeX\ (version 0.20) does not support Lua script
% files inside \xfile{TDS:scripts//}, because Lua's mechanism
% for module loading does not use the \textsf{kpathsea} library.
% Therefore this packages appends a kpse loader to the list of
% Lua's module loaders. It finds the module \meta{module} by
% \begin{quote}
%   \ttfamily |kpse.find_file("|\meta{module}|.lua", "texmfscripts")|
% \end{quote}
% Unhappily \textsf{kpathsea} does not support directory components
% in a file name. Therefore the Lua convention is not followed to
% replace dots in the module name by the directory separator.
%
% Example:
% A Lua script of a package \xpackage{foobar} wants the following
% modules:
% \begin{quote}
%   |require("foobar.hello.world")|\\
%   |require("org.somewhere.xyz")|
% \end{quote}
% Then they can be find in:
% \begin{quote}
%   |TDS:scripts/foobar/foobar.hello.world.lua|\\
%   |TDS:scripts/foobar/org.somewhere.xyz.lua|
% \end{quote}
% I would have preferred the following locations, following lua conventions,
% e.\,g.:
% \begin{quote}
%   |TDS:scripts/foobar/hello/world.lua|\\
%   |TDS:scripts/foobar/org/somewhere/xyz.lua|
% \end{quote}
% But I do not know, how to achieve this in a reliable way using
% \textsf{kpathsea}.
%
% \subsubsection{Package \xpackage{luatex-loader}}
%
% If someone do not need or want package \xpackage{luatex} but
% it's extension for module loading, then he can use
% package \xpackage{luatex-loader}. Both \plainTeX\ and \LaTeX\ are
% supported.
%
% \StopEventually{
% }
%
% \section{Implementation}
%
%    \begin{macrocode}
%<*package>
%    \end{macrocode}
%
% \subsection{Reload check and package identification}
%    Reload check, especially if the package is not used with \LaTeX.
%    \begin{macrocode}
\begingroup\catcode61\catcode48\catcode32=10\relax%
  \catcode13=5 % ^^M
  \endlinechar=13 %
  \catcode35=6 % #
  \catcode39=12 % '
  \catcode44=12 % ,
  \catcode45=12 % -
  \catcode46=12 % .
  \catcode58=12 % :
  \catcode64=11 % @
  \catcode123=1 % {
  \catcode125=2 % }
  \expandafter\let\expandafter\x\csname ver@luatex.sty\endcsname
  \ifx\x\relax % plain-TeX, first loading
  \else
    \def\empty{}%
    \ifx\x\empty % LaTeX, first loading,
      % variable is initialized, but \ProvidesPackage not yet seen
    \else
      \expandafter\ifx\csname PackageInfo\endcsname\relax
        \def\x#1#2{%
          \immediate\write-1{Package #1 Info: #2.}%
        }%
      \else
        \def\x#1#2{\PackageInfo{#1}{#2, stopped}}%
      \fi
      \x{luatex}{The package is already loaded}%
      \aftergroup\endinput
    \fi
  \fi
\endgroup%
%    \end{macrocode}
%    Package identification:
%    \begin{macrocode}
\begingroup\catcode61\catcode48\catcode32=10\relax%
  \catcode13=5 % ^^M
  \endlinechar=13 %
  \catcode35=6 % #
  \catcode39=12 % '
  \catcode40=12 % (
  \catcode41=12 % )
  \catcode44=12 % ,
  \catcode45=12 % -
  \catcode46=12 % .
  \catcode47=12 % /
  \catcode58=12 % :
  \catcode64=11 % @
  \catcode91=12 % [
  \catcode93=12 % ]
  \catcode123=1 % {
  \catcode125=2 % }
  \expandafter\ifx\csname ProvidesPackage\endcsname\relax
    \def\x#1#2#3[#4]{\endgroup
      \immediate\write-1{Package: #3 #4}%
      \xdef#1{#4}%
    }%
  \else
    \def\x#1#2[#3]{\endgroup
      #2[{#3}]%
      \ifx#1\@undefined
        \xdef#1{#3}%
      \fi
      \ifx#1\relax
        \xdef#1{#3}%
      \fi
    }%
  \fi
\expandafter\x\csname ver@luatex.sty\endcsname
\ProvidesPackage{luatex}%
  [2010/03/09 v0.4 LuaTeX basic definition package (HO)]%
%    \end{macrocode}
%
% \subsection{Catcodes}
%
%    \begin{macrocode}
\begingroup\catcode61\catcode48\catcode32=10\relax%
  \catcode13=5 % ^^M
  \endlinechar=13 %
  \catcode123=1 % {
  \catcode125=2 % }
  \catcode64=11 % @
  \def\x{\endgroup
    \expandafter\edef\csname LuT@AtEnd\endcsname{%
      \endlinechar=\the\endlinechar\relax
      \catcode13=\the\catcode13\relax
      \catcode32=\the\catcode32\relax
      \catcode35=\the\catcode35\relax
      \catcode61=\the\catcode61\relax
      \catcode64=\the\catcode64\relax
      \catcode123=\the\catcode123\relax
      \catcode125=\the\catcode125\relax
    }%
  }%
\x\catcode61\catcode48\catcode32=10\relax%
\catcode13=5 % ^^M
\endlinechar=13 %
\catcode35=6 % #
\catcode64=11 % @
\catcode123=1 % {
\catcode125=2 % }
\def\TMP@EnsureCode#1#2{%
  \edef\LuT@AtEnd{%
    \LuT@AtEnd
    \catcode#1=\the\catcode#1\relax
  }%
  \catcode#1=#2\relax
}
\TMP@EnsureCode{10}{12}% ^^J
\TMP@EnsureCode{34}{12}% "
\TMP@EnsureCode{36}{3}% $
\TMP@EnsureCode{39}{12}% '
\TMP@EnsureCode{40}{12}% (
\TMP@EnsureCode{41}{12}% )
\TMP@EnsureCode{42}{12}% *
\TMP@EnsureCode{43}{12}% +
\TMP@EnsureCode{44}{12}% ,
\TMP@EnsureCode{45}{12}% -
\TMP@EnsureCode{46}{12}% .
\TMP@EnsureCode{47}{12}% /
\TMP@EnsureCode{60}{12}% <
\TMP@EnsureCode{62}{12}% >
\TMP@EnsureCode{91}{12}% [
\TMP@EnsureCode{93}{12}% ]
\TMP@EnsureCode{95}{12}% _ (other!)
\TMP@EnsureCode{96}{12}% `
\edef\LuT@AtEnd{\LuT@AtEnd\noexpand\endinput}
%    \end{macrocode}
%
% \subsection{Check for \LuaTeX}
%
%    Without \LuaTeX\ there is no point in using this package.
%    \begin{macrocode}
\begingroup\expandafter\expandafter\expandafter\endgroup
\expandafter\ifx\csname RequirePackage\endcsname\relax
  \input infwarerr.sty\relax
  \input ifluatex.sty\relax
\else
  \RequirePackage{infwarerr}[2007/09/09]%
  \RequirePackage{ifluatex}[2009/04/10]%
\fi
%    \end{macrocode}
%    \begin{macrocode}
\ifluatex
\else
  \@PackageError{luatex}{%
    This package may only be run using LuaTeX%
  }\@ehc
  \expandafter\LuT@AtEnd
\fi%
%    \end{macrocode}
%
% \subsection{Provide Lua\TeX\ primitives}
%
%    \begin{macrocode}
\ifnum\luatexversion<36 %
  \def\LuT@MakePrimitive#1{%
    \expandafter\let\csname luatex#1\expandafter\endcsname
    \csname #1\endcsname
  }%
\else
  \def\LuT@MakeLuatexPrimitive#1{%
    \begingroup\expandafter\expandafter\expandafter\endgroup
    \expandafter\ifx\csname luatex#1\endcsname\relax
      \begingroup\expandafter\expandafter\expandafter\endgroup
      \expandafter\ifx\csname #1\endcsname\relax
      \else
        \expandafter\let
        \csname luatex#1\expandafter\endcsname
        \csname #1\endcsname
      \fi
    \fi
    \begingroup\expandafter\expandafter\expandafter\endgroup
    \expandafter\ifx\csname luatex#1\endcsname\relax
      \begingroup
        \expandafter\let\csname luatex#1\endcsname\@undefined
        \ifnum0%
          \directlua{%
            if tex.enableprimitives then %
              tex.enableprimitives('luatex',{'#1'})%
              tex.print('1')%
            end%
          }%
          \expandafter\ifx\csname luatex#1\endcsname\relax\else1\fi
        =11 %
          \global\expandafter\let
          \csname luatex#1\expandafter\endcsname
          \csname luatex#1\endcsname
        \else
          \@PackageError{luatex}{%
            tex.enableprimitives failed for `#1'%
          }\@ehc
        \fi
      \endgroup
    \fi
  }%
  \def\LuT@MakePrimitive#1{%
    \begingroup\expandafter\expandafter\expandafter\endgroup
    \expandafter\ifx\csname#1\endcsname\relax
      \begingroup
        \expandafter\let\csname#1\endcsname\@undefined
        \ifnum0%
          \directlua{%
            if tex.enableprimitives then %
              tex.enableprimitives('',{'#1'})%
              tex.print('1')%
            end%
          }%
          \expandafter\ifx\csname#1\endcsname\relax\else1\fi
        =11 %
          \global\expandafter\let
          \csname#1\expandafter\endcsname
          \csname#1\endcsname
        \else
          \@PackageError{luatex}{%
            tex.enableprimitives failed for `#1'%
          }\@ehc
        \fi
      \endgroup
    \fi
  }%
\fi
\LuT@MakeLuatexPrimitive{attribute}
\LuT@MakeLuatexPrimitive{attributedef}
\LuT@MakeLuatexPrimitive{catcodetable}
\LuT@MakeLuatexPrimitive{initcatcodetable}
\LuT@MakeLuatexPrimitive{luaescapestring}
\LuT@MakeLuatexPrimitive{savecatcodetable}
\LuT@MakePrimitive{numexpr}
%    \end{macrocode}
%
% \subsection{Inherit support for \eTeX}
%
%    Package \xpackage{etex} is not compatible for \plainTeX.
%    But it could be present if a format is used that is based
%    on \xfile{etex.src}. Therefore we only load the package
%    in case of \LaTeX\ and tests its presence independently
%    of the format by looking for \cs{et@xins}.
%    \begin{macrocode}
\begingroup\expandafter\expandafter\expandafter\endgroup
\expandafter\ifx\csname RequirePackage\endcsname\relax
\else
  \RequirePackage{etex}[1998/03/26]%
\fi
%    \end{macrocode}
%
% \subsection{Adaption of \eTeX's register allocation}
%
%    \eTeX has increased the number of \TeX\ registers
%    from $2^8$ (256) to $2^{15}$ (32768) for a register class.
%    \LuaTeX\ extends the limit further to $2^16$ (65536).
%    The allocation scheme of package \xpackage{etex} is not
%    changed. But this can be subject for discussion.
%
%    If a register class hasn't registered any local registers
%    yet, then the limit can safely be pushed to 65536.
%    \begin{macrocode}
\begingroup\expandafter\expandafter\expandafter\endgroup
\expandafter\ifx\csname et@xins\endcsname\relax
  \@PackageWarningNoLine{luatex}{%
    Support for eTeX is not loaded (etex.src)%
  }%
\else
  \def\LuT@temp#1{%
    \ifnum\count27#1=32768 %
      \count27#1=65536 %
    \fi
  }%
  \LuT@temp0%
  \LuT@temp1%
  \LuT@temp2%
  \LuT@temp3%
  \LuT@temp4%
  \LuT@temp5%
  \LuT@temp6%
%    \end{macrocode}
%
% \eTeX\ uses an array for the first 256 registers and then a tree
% structure. \LuaTeX\ stores all registers of a class in one Lua table.
% There shouldn't be large performance differences.
% This allows starting immediately in the extended
% area, leaving room for insertions.
%    \begin{macrocode}
  \let\newcount\globcount
  \let\newdimen\globdimen
  \let\newskip\globskip
  \let\newbox\globbox
\fi
%    \end{macrocode}
%
% \subsection{\plainTeX\ compatibility}
%
%    \begin{macro}{\@empty}
%    \begin{macrocode}
\expandafter\ifx\csname @empty\endcsname\relax
  \def\@empty{}%
\fi
%    \end{macrocode}
%    \end{macro}
%    \begin{macro}{\@gobble}
%    \begin{macrocode}
\expandafter\ifx\csname @gobble\endcsname\relax
  \long\def\@gobble#1{}%
\fi
%    \end{macrocode}
%    \end{macro}
%    \begin{macro}{\@firstofone}
%    \begin{macrocode}
\expandafter\ifx\csname @firstofone\endcsname\relax
  \long\def\@firstofone#1{#1}%
\fi
%    \end{macrocode}
%    \end{macro}
%    \begin{macro}{\@firstoftwo}
%    \begin{macrocode}
\expandafter\ifx\csname @firstoftwo\endcsname\relax
  \long\def\@firstoftwo#1#2{#1}%
\fi
%    \end{macrocode}
%    \end{macro}
%    \begin{macro}{\@car}
%    \begin{macrocode}
\expandafter\ifx\csname @car\endcsname\relax
  \def\@car#1#2\@nil{#1}%
\fi
%    \end{macrocode}
%    \end{macro}
%    \begin{macro}{\@cdr}
%    \begin{macrocode}
\expandafter\ifx\csname @cdr\endcsname\relax
  \def\@cdr#1#2\@nil{#2}%
\fi
%    \end{macrocode}
%    \end{macro}
%    \begin{macro}{\@ifstar}
%    \begin{macrocode}
\expandafter\ifx\csname @ifstar\endcsname\relax
  \def\@ifstar#1{%
    \@ifnextchar*{\@firstoftwo{#1}}%
  }%
%    \end{macrocode}
%    \end{macro}
%    \begin{macro}{\@ifnextchar}
%    \begin{macrocode}
  \long\def\@ifnextchar#1#2#3{%
    \let\reserved@d=#1%
    \def\reserved@a{#2}%
    \def\reserved@b{#3}%
    \futurelet\@let@token\@ifnch
  }%
%    \end{macrocode}
%    \end{macro}
%    \begin{macro}{\@ifnch}
%    \begin{macrocode}
  \def\@ifnch{%
    \ifx\@let@token\@sptoken
      \let\reserved@c\@xifnch
    \else
      \ifx\@let@token\reserved@d
        \let\reserved@c\reserved@a
      \else
        \let\reserved@c\reserved@b
      \fi
    \fi
    \reserved@c
  }%
%    \end{macrocode}
%    \end{macro}
%    \begin{macro}{\@sptoken}
%    \begin{macrocode}
  \let\LuT@temp\:%
  \def\:{\let\@sptoken= }%
  \: % explicit space
%    \end{macrocode}
%    \end{macro}
%    \begin{macro}{\@xifnch}
%    \begin{macrocode}
  \def\:{\@xifnch}%
  \expandafter\def\: {%
    \futurelet\@let@token\@ifnch
  }%
  \let\:\LuT@temp
\fi
%    \end{macrocode}
%    \end{macro}
%
%    \begin{macro}{\@tempcnta}
%    \begin{macrocode}
\expandafter\ifx\csname @tempcnta\endcsname\relax
  \csname newcount\endcsname\@tempcnta
\fi
%    \end{macrocode}
%    \end{macro}
%
%    \begin{macro}{\@tempcntb}
%    \begin{macrocode}
\expandafter\ifx\csname @tempcntb\endcsname\relax
  \csname newcount\endcsname\@tempcntb
\fi
%    \end{macrocode}
%    \end{macro}
%
%    \begin{macro}{\LuT@newcommand}
%    \begin{macrocode}
\begingroup\expandafter\expandafter\expandafter\endgroup
\expandafter\ifx\csname newcommand\endcsname\relax
  \def\LuT@newcommand#1[#2]#3{%
    \ifx#1\@undefined
      \let#1\relax
    \else
      \ifx#1\relax
      \else
        \@PackageError{luatex}{%
          \string#1 is already defined.\MessageBreak
          Redefinition is skipped%
        }\@ehc
      \fi
    \fi
    \ifx#1\relax
      \ifcase#2 %
        \def#1{#3}%
      \or
        \def#1##1{#3}%
      \or
        \def#1##1##2{#3}%
      \or
        \def#1##1##2##3{#3}%
      \or
        \@INTERNAL@ERROR
      \fi
    \fi
  }%
\else
  \def\LuT@newcommand{\newcommand*}%
\fi
%    \end{macrocode}
%    \end{macro}
%
% \subsection{Attributes}
%
% \subsubsection{Allocation}
%
%    \begin{macro}{\LuT@AllocAttribute}
%    \begin{macrocode}
\newcount\LuT@AllocAttribute
\LuT@AllocAttribute=\m@ne
%    \end{macrocode}
%    \end{macro}
%    \begin{macro}{\newattribute}
%    \begin{macrocode}
\LuT@newcommand\newattribute[1]{%
  \ifnum\LuT@AllocAttribute<65535 %
    \global\advance\LuT@AllocAttribute\@ne
    \allocationnumber\LuT@AllocAttribute
    \global\luatexattributedef#1=\allocationnumber
    \unsetattribute{#1}%
    \wlog{\string#1=\string\attribute\the\allocationnumber}%
  \else
    \errmessage{No room for a new \string\attribute}%
  \fi
}
%    \end{macrocode}
%    \end{macro}
%
% \subsubsection{Interface}
%
%    \begin{macro}{\setattribute}
%    \begin{macrocode}
\LuT@newcommand\setattribute[2]{%
  #1=\numexpr#2\relax
}
%    \end{macrocode}
%    \end{macro}
%    \begin{macro}{\unsetattribute}
%    \begin{macrocode}
\ifnum\luatexversion<37
  \LuT@newcommand\LuT@UnsetAttributeValue[0]{}%
  \let\LuT@UnsetAttributeValue\m@ne
\else
  \LuT@newcommand\LuT@UnsetAttributeValue[0]{-2147483647 }%
\fi
\LuT@newcommand\unsetattribute[1]{%
  #1=\LuT@UnsetAttributeValue
}
%    \end{macrocode}
%    \end{macro}
%
% \subsection{Catcode tables}
%
% \subsubsection{Allocation}
%
%    \begin{macro}{\LuT@AllocCatcodeTable}
%    \begin{macrocode}
\newcount\LuT@AllocCatcodeTable
\LuT@AllocCatcodeTable=\m@ne
\newcount\CatcodeTableStack
\CatcodeTableStack=\z@
%    \end{macrocode}
%    \end{macro}
%    \begin{macro}{\newcatcodetable}
%    \begin{macrocode}
\LuT@newcommand\newcatcodetable[1]{%
  \ifnum\LuT@AllocCatcodeTable<1114110 % 0x10FFFF is maximal \chardef
    % or < 268435455 % 2^28 - 1
    \global\advance\LuT@AllocCatcodeTable by\tw@
    \allocationnumber=\LuT@AllocCatcodeTable
    \global\chardef#1=\allocationnumber
    \wlog{%
      \string#1=\string\catcodetable\the\allocationnumber
    }%
  \else
    \errmessage{No room for a new \string\catcodetable}%
  \fi
}%
%    \end{macrocode}
%    \end{macro}
%
%    \begin{macro}{\IncCatcodeTableStack}
%    \begin{macrocode}
\LuT@newcommand\IncCatcodeTableStack[0]{%
  \ifnum\CatcodeTableStack<268435454 %
    \global\advance\CatcodeTableStack by\tw@
  \else
    \@PackageError{luatex}{%
      Catcode table stack overflow%
    }\@ehd
  \fi
}
%    \end{macrocode}
%    \end{macro}
%
%    \begin{macro}{\DecCatcodeTableStack}
%    \begin{macrocode}
\LuT@newcommand\DecCatcodeTableStack[0]{%
  \ifnum\CatcodeTableStack>\z@
    \global\advance\CatcodeTableStack by-2 %
  \else
    \@PackageError{luatex}{%
      Catcode table stack is empty%
    }\@ehd
  \fi
}
%    \end{macrocode}
%    \end{macro}
%
% \subsubsection{\cs{SetCatcodeRange}}
%
%    \begin{macro}{\SetCatcodeRange}
%    \begin{macrocode}
\LuT@newcommand\SetCatcodeRange[3]{%
  \edef\LuT@temp{%
    \noexpand\@tempcnta=\the\@tempcnta
    \noexpand\@tempcntb=\the\@tempcntb
    \noexpand\count@=\the\count@
    \relax
  }%
  \@tempcnta=\numexpr#1\relax
  \@tempcntb=\numexpr#2\relax
  \count@=\numexpr#3\relax
  \loop
    \unless\ifnum\@tempcnta>\@tempcntb
    \catcode\@tempcnta=\count@
    \advance\@tempcnta by \@ne
  \repeat
  \LuT@temp
}
%    \end{macrocode}
%    \end{macro}
%
% \subsubsection{Predefined catcode tables}
%
%    \begin{macrocode}
\newcatcodetable\CatcodeTableIniTeX
\newcatcodetable\CatcodeTableString
\newcatcodetable\CatcodeTableOther
\newcatcodetable\CatcodeTableLaTeX
%    \end{macrocode}
%
%    \begin{macrocode}
\luatexinitcatcodetable\CatcodeTableIniTeX
\begingroup
  \def\@makeother#1{\catcode#1=12\relax}%
  \@firstofone{%
    \luatexcatcodetable\CatcodeTableIniTeX
    \begingroup
      \SetCatcodeRange{0}{8}{15}%
      \catcode9=10 % tab
      \catcode11=15 %
      \catcode12=13 % form feed
      \SetCatcodeRange{14}{31}{15}%
      \catcode35=6 % hash
      \catcode36=3 % dollar
      \catcode38=4 % ampersand
      \catcode94=7 % circumflex
      \catcode95=8 % underscore
      \catcode123=1 % brace left
      \catcode125=2 % brace right
      \catcode126=13 % tilde
      \catcode127=15 %
      \luatexsavecatcodetable\CatcodeTableLaTeX
    \endgroup
    \@makeother{0}% nul
    \@makeother{13}% carriage return
    \@makeother{37}% percent
    \@makeother{92}% backslash
    \@makeother{127}%
    \SetCatcodeRange{65}{90}{12}% A-Z
    \SetCatcodeRange{97}{122}{12}% a-z
    \luatexsavecatcodetable\CatcodeTableString
    \@makeother{32}% space
    \luatexsavecatcodetable\CatcodeTableOther
  \endgroup
}%
%    \end{macrocode}
%
% \subsubsection{Number stack}
%
%    \begin{macro}{\LuT@NumStackEmpty}
%    A special empty stack value because of \cs{@cdr}'s brace removal.
%    \begin{macrocode}
\def\LuT@NumStackEmpty{0}
%    \end{macrocode}
%    \end{macro}
%    \begin{macro}{\LuT@NumStack}
%    \begin{macrocode}
\let\LuT@NumStack\LuT@NumStackEmpty
%    \end{macrocode}
%    \end{macro}
%    \begin{macro}{\PushCatcodeTableNumStack}
%    \begin{macrocode}
\LuT@newcommand\PushCatcodeTableNumStack[0]{%
  \xdef\LuT@NumStack{%
    {\the\luatexcatcodetable}\LuT@NumStack
  }%
}
%    \end{macrocode}
%    \end{macro}
%    \begin{macro}{\PopCatcodeTableNumStack}
%    \begin{macrocode}
\LuT@newcommand\PopCatcodeTableNumStack[0]{%
  \ifx\LuT@NumStack\LuT@NumStackEmpty
    \@PackageWarning{luatex}{Empty catcode table number stack}%
    \luatexcatcodetable\z@
  \else
    \luatexcatcodetable=\expandafter\@car\LuT@NumStack\@nil\relax
    \xdef\LuT@NumStack{%
      \expandafter\@cdr\LuT@NumStack\@nil
    }%
  \fi
}
%    \end{macrocode}
%    \end{macro}
%
% \subsubsection{Catcode regime macros}
%
%    \begin{macro}{\BeginCatcodeRegime}
%    \begin{macrocode}
\LuT@newcommand\BeginCatcodeRegime[1]{%
  \PushCatcodeTableNumStack
  \luatexcatcodetable=\numexpr#1\relax
  \IncCatcodeTableStack
  \luatexsavecatcodetable\CatcodeTableStack
  \luatexcatcodetable\CatcodeTableStack
}
%    \end{macrocode}
%    \end{macro}
%
%    \begin{macro}{\EndCatcodeRegime}
%    \begin{macrocode}
\LuT@newcommand\EndCatcodeRegime[0]{%
  \DecCatcodeTableStack
  \PopCatcodeTableNumStack
}
%    \end{macrocode}
%    \end{macro}
%
% \subsection{Lua module loader}
%
%    \begin{macrocode}
\begingroup\expandafter\expandafter\expandafter\endgroup
\expandafter\ifx\csname RequirePackage\endcsname\relax
  \input luatex-loader.sty\relax
\else
  \RequirePackage{luatex-loader}[2010/03/09]%
\fi
%    \end{macrocode}
%    \begin{macrocode}
\LuT@AtEnd%
%</package>
%    \end{macrocode}
%
%    \begin{macrocode}
%<*loader>
%    \end{macrocode}
%
%    Reload check, especially if the package is not used with \LaTeX.
%    \begin{macrocode}
\begingroup\catcode61\catcode48\catcode32=10\relax%
  \catcode13=5 % ^^M
  \endlinechar=13 %
  \catcode35=6 % #
  \catcode39=12 % '
  \catcode44=12 % ,
  \catcode45=12 % -
  \catcode46=12 % .
  \catcode58=12 % :
  \catcode64=11 % @
  \catcode123=1 % {
  \catcode125=2 % }
  \expandafter\let\expandafter\x\csname ver@luatex-loader.sty\endcsname
  \ifx\x\relax % plain-TeX, first loading
  \else
    \def\empty{}%
    \ifx\x\empty % LaTeX, first loading,
      % variable is initialized, but \ProvidesPackage not yet seen
    \else
      \expandafter\ifx\csname PackageInfo\endcsname\relax
        \def\x#1#2{%
          \immediate\write-1{Package #1 Info: #2.}%
        }%
      \else
        \def\x#1#2{\PackageInfo{#1}{#2, stopped}}%
      \fi
      \x{luatex-loader}{The package is already loaded}%
      \aftergroup\endinput
    \fi
  \fi
\endgroup%
%    \end{macrocode}
%    Package identification:
%    \begin{macrocode}
\begingroup\catcode61\catcode48\catcode32=10\relax%
  \catcode13=5 % ^^M
  \endlinechar=13 %
  \catcode35=6 % #
  \catcode39=12 % '
  \catcode40=12 % (
  \catcode41=12 % )
  \catcode44=12 % ,
  \catcode45=12 % -
  \catcode46=12 % .
  \catcode47=12 % /
  \catcode58=12 % :
  \catcode64=11 % @
  \catcode91=12 % [
  \catcode93=12 % ]
  \catcode123=1 % {
  \catcode125=2 % }
  \expandafter\ifx\csname ProvidesPackage\endcsname\relax
    \def\x#1#2#3[#4]{\endgroup
      \immediate\write-1{Package: #3 #4}%
      \xdef#1{#4}%
    }%
  \else
    \def\x#1#2[#3]{\endgroup
      #2[{#3}]%
      \ifx#1\@undefined
        \xdef#1{#3}%
      \fi
      \ifx#1\relax
        \xdef#1{#3}%
      \fi
    }%
  \fi
\expandafter\x\csname ver@luatex-loader.sty\endcsname
\ProvidesPackage{luatex-loader}%
  [2010/03/09 v0.4 Lua module loader (HO)]%
%    \end{macrocode}
%    \begin{macrocode}
\begingroup\catcode61\catcode48\catcode32=10\relax%
  \catcode13=5\endlinechar=13\relax%
  \catcode10=12 % ^^J
  \catcode34=12 % "
  \catcode39=12 % '
  \catcode40=12 % (
  \catcode41=12 % )
  \catcode44=12 % ,
  \catcode46=12 % .
  \catcode60=12 % <
  \catcode61=12 % =
  \catcode95=12 % _ (other!)
  \catcode96=12 % `
  \catcode123=1 % {
  \catcode125=2 % }
  \endlinechar=10 %
  \ifnum\luatexversion<36 %
    \directlua0%
  \else %
    \expandafter\directlua %
  \fi %
  {%
     do
      local script = "oberdiek.luatex.lua"
      local file = kpse.find_file(script, "texmfscripts")
      if file then
        texio.write_nl("(" .. file .. ")")
        dofile(file)
      else
        error("File `" .. script .. "' not found")
      end
    end
  }%
\endgroup\endinput%
%    \end{macrocode}
%
%    \begin{macrocode}
%</loader>
%    \end{macrocode}
%
% \subsection{Lua script}
%
%    Currently \LuaTeX\ does not use KPSE when searching for module
%    files. The following Lua script implements a workaround.
%    It extends |package.loader| by another search method.
%    Modules are found by the module name with extension \xext{lua}
%    similar to
%    \begin{quote}
%      \ttfamily |kpsewhich --format=texmfscripts| \meta{module}|.lua|
%    \end{quote}
%    Unhappily \textsf{kpsewhich} does not support directory components
%    in the file name. Therefore a module |a.b.c| cannot be installed
%    as \xfile{a/b/c.lua}. The script must be named \xfile{a.b.c.lua}.
%    \begin{macrocode}
%<*lua>
%    \end{macrocode}
%    \begin{macrocode}
module("oberdiek.luatex", package.seeall)
%    \end{macrocode}
%    \begin{macrocode}
function kpse_module_loader(module)
  local script = module .. ".lua"
  local file = kpse.find_file(script, "texmfscripts")
  if file then
    local loader, error = loadfile(file)
    if loader then
      texio.write_nl("(" .. file .. ")")
      return loader
    end
    return "\n\t[oberdiek.luatex.kpse_module_loader] Loading error:\n\t"
           .. error
  end
  return "\n\t[oberdiek.luatex.kpse_module_loader] Search failed"
end
table.insert(package.loaders, kpse_module_loader)
%    \end{macrocode}
%    \begin{macrocode}
%</lua>
%    \end{macrocode}
%
% \section{Test}
%
%    \begin{macrocode}
%<*test2>
\documentclass{article}
\def\LoadCommand{%
  \RequirePackage{luatex}[2010/03/09]%
}
%</test2>
%    \end{macrocode}
%    \begin{macrocode}
%<*test3>
\documentclass{article}
\def\LoadCommand{%
  \RequirePackage{luatex-loader}[2010/03/09]%
}
%</test3>
%    \end{macrocode}
% \subsection{Catcode checks for loading}
%
%    \begin{macrocode}
%<*test1>
%    \end{macrocode}
%    \begin{macrocode}
\catcode`\{=1 %
\catcode`\}=2 %
\catcode`\#=6 %
\catcode`\@=11 %
\expandafter\ifx\csname count@\endcsname\relax
  \countdef\count@=255 %
\fi
\expandafter\ifx\csname @gobble\endcsname\relax
  \long\def\@gobble#1{}%
\fi
\expandafter\ifx\csname @firstofone\endcsname\relax
  \long\def\@firstofone#1{#1}%
\fi
\expandafter\ifx\csname loop\endcsname\relax
  \expandafter\@firstofone
\else
  \expandafter\@gobble
\fi
{%
  \def\loop#1\repeat{%
    \def\body{#1}%
    \iterate
  }%
  \def\iterate{%
    \body
      \let\next\iterate
    \else
      \let\next\relax
    \fi
    \next
  }%
  \let\repeat=\fi
}%
\def\RestoreCatcodes{}
\count@=0 %
\loop
  \edef\RestoreCatcodes{%
    \RestoreCatcodes
    \catcode\the\count@=\the\catcode\count@\relax
  }%
\ifnum\count@<255 %
  \advance\count@ 1 %
\repeat

\def\RangeCatcodeInvalid#1#2{%
  \count@=#1\relax
  \loop
    \catcode\count@=15 %
  \ifnum\count@<#2\relax
    \advance\count@ 1 %
  \repeat
}
\def\RangeCatcodeCheck#1#2#3{%
  \count@=#1\relax
  \loop
    \ifnum#3=\catcode\count@
    \else
      \errmessage{%
        Character \the\count@\space
        with wrong catcode \the\catcode\count@\space
        instead of \number#3%
      }%
    \fi
  \ifnum\count@<#2\relax
    \advance\count@ 1 %
  \repeat
}
\def\space{ }
\expandafter\ifx\csname LoadCommand\endcsname\relax
  \def\LoadCommand{\input luatex.sty\relax}%
\fi
\def\Test{%
  \RangeCatcodeInvalid{0}{47}%
  \RangeCatcodeInvalid{58}{64}%
  \RangeCatcodeInvalid{91}{96}%
  \RangeCatcodeInvalid{123}{255}%
  \catcode`\@=12 %
  \catcode`\\=0 %
  \catcode`\%=14 %
  \LoadCommand
  \RangeCatcodeCheck{0}{36}{15}%
  \RangeCatcodeCheck{37}{37}{14}%
  \RangeCatcodeCheck{38}{47}{15}%
  \RangeCatcodeCheck{48}{57}{12}%
  \RangeCatcodeCheck{58}{63}{15}%
  \RangeCatcodeCheck{64}{64}{12}%
  \RangeCatcodeCheck{65}{90}{11}%
  \RangeCatcodeCheck{91}{91}{15}%
  \RangeCatcodeCheck{92}{92}{0}%
  \RangeCatcodeCheck{93}{96}{15}%
  \RangeCatcodeCheck{97}{122}{11}%
  \RangeCatcodeCheck{123}{255}{15}%
  \RestoreCatcodes
}
\Test
\csname @@end\endcsname
\end
%    \end{macrocode}
%    \begin{macrocode}
%</test1>
%    \end{macrocode}
%
% \subsection{Catcode tables}
%
% \subsubsection{Predefined catcode tables}
%
%    \begin{macrocode}
%<*test4>
\NeedsTeXFormat{LaTeX2e}
%    \end{macrocode}
%    Remember \LaTeX's initial catcodes in count registers
%    starting at \cs{TestLaTeX}.
%    \begin{macrocode}
\count0=0 %
\chardef\TestLaTeX=1000 %
\chardef\TestMax=300 %
\loop
  \count\numexpr\TestLaTeX+\count0\relax=\catcode\count0 %
\ifnum\count0<\TestMax
  \advance\count0 by 1 %
\repeat
\documentclass{minimal}
\usepackage{luatex}[2010/03/09]
\usepackage{qstest}
\IncludeTests{*}
\LogTests{log}{*}{*}
\makeatletter
\def\Check#1{%
  \Expect*{\the\count@=\the\catcode\count@}%
         *{\the\count@=#1}%
}
\newcount\scratch
\def\Test#1#2{%
  \begin{qstest}{CatcodeTable#1}{CatcodeTable#1}%
    \luatexcatcodetable\csname CatcodeTable#1\endcsname
    \count@=\z@
    \loop
      \scratch=#2\relax
      \Expect*{\the\count@=\the\catcode\count@}%
             *{\the\count@=\the\scratch}%
    \ifnum\count@<\TestMax
      \advance\count@\@ne
    \repeat
  \end{qstest}%
}
\begingroup
  % luatex-unicode-letters.tex makes some slots to letters
  \def\TestMax{169}%
  \Test{LaTeX}{\the\count\numexpr\TestLaTeX+\count@}%
\endgroup
\Test{String}{\ifnum\count@=32 10\else 12\fi}
\Test{Other}{12}
\luatexinitcatcodetable99 %
\Test{IniTeX}{%
  0\relax
  \begingroup
    \luatexcatcodetable99 %
    \global\scratch=\the\catcode\count@
  \endgroup
}
%    \end{macrocode}
%
% \subsubsection{Catcode table number stack}
%
%    \begin{macrocode}
\begin{qstest}{CatcodeTableNumStack}{CatcodeTableNumStack}
  \def\TestStack#1{%
    \Expect*{\LuT@NumStack}{#1}%
  }%
  \TestStack{0}%
  \PushCatcodeTableNumStack
  \TestStack{{0}0}%
  \@firstofone{%
    \begingroup
      \luatexinitcatcodetable12 %
      \luatexcatcodetable12 %
      \PushCatcodeTableNumStack
      \TestStack{{12}{0}0}%
      \PopCatcodeTableNumStack
      \TestStack{{0}0}%
      \PopCatcodeTableNumStack
      \TestStack{0}%
      \def\TestWarning{Missing empty stack warning}%
      \def\@PackageWarning#1#2{\def\TestWarning{empty stack}}%
      \PopCatcodeTableNumStack
      \TestStack{0}%
      \Expect*{\TestWarning}{empty stack}%
    \endgroup
  }%
\end{qstest}
%    \end{macrocode}
%
% \subsubsection{Catcode table stack}
%
%    \begin{macrocode}
\begin{qstest}{CatcodeTableStack}{CatcodeTableStack}
  \def\TestStack#1{%
    \Expect*{\the\CatcodeTableStack}{#1}%
  }%
  \TestStack{0}%
  \IncCatcodeTableStack
  \TestStack{2}%
  \IncCatcodeTableStack
  \TestStack{4}%
  \begingroup
    \IncCatcodeTableStack
    \TestStack{6}%
  \endgroup
  \TestStack{6}%
  \begingroup
    \DecCatcodeTableStack
    \TestStack{4}%
  \endgroup
  \TestStack{4}%
  \DecCatcodeTableStack
  \TestStack{2}%
  \DecCatcodeTableStack
  \TestStack{0}%
  \begingroup
    \def\TestError{Missing error}%
    \def\@PackageError#1#2#3{%
      \def\TestError{Empty stack}%
    }%
    \DecCatcodeTableStack
    \TestStack{0}%
    \Expect*{\TestError}{Empty stack}%
  \endgroup
\end{qstest}
%    \end{macrocode}
%
% \subsubsection{Catcode regime macros}
%
%    \begin{macrocode}
\begin{qstest}{CatcodeRegime}{CatcodeRegime}
  \def\TestStacks#1#2#3{%
    \Expect*{\the\luatexcatcodetable}{#1}%
    \Expect*{\the\CatcodeTableStack}{#2}%
    \Expect*{\LuT@NumStack}{#3}%
  }%
  \TestStacks{0}{0}{0}%
  \catcode`\|=7 %
  \BeginCatcodeRegime\CatcodeTableLaTeX
    \TestStacks{2}{2}{{0}0}%
    \Expect*{\the\catcode`\|}{12}%
  \EndCatcodeRegime
  \TestStacks{0}{0}{0}%
  \Expect*{\the\catcode`\|}{7}%
\end{qstest}
%    \end{macrocode}
%
% \subsection{Attribute allocation}
%
%    \begin{macrocode}
\begin{qstest}{Attributes}{Attributes}
  \newattribute\TestAttr
  \Expect*{\meaning\TestAttr}%
         *{\string\attribute\number\allocationnumber}%
  \Expect*{\the\allocationnumber}{0}%
  \begingroup
    \newattribute\TestAttr
    \Expect*{\the\allocationnumber}{1}%
  \endgroup
  \Expect*{\the\allocationnumber}{0}%
  \Expect*{\meaning\TestAttr}*{\string\attribute1}%
  \Expect*{\the\TestAttr}*{\number\LuT@UnsetAttributeValue}%
  \def\Test#1{%
    \setattribute\TestAttr{#1}%
    \Expect*{\the\TestAttr}{#1}%
  }%
  \Test{0}%
  \Test{1}%
  \Test{-1}%
  \Test{123}%
  \unsetattribute\TestAttr
  \Expect*{\the\TestAttr}*{\number\LuT@UnsetAttributeValue}%
  \begingroup
    \Expect*{\the\TestAttr}*{\number\LuT@UnsetAttributeValue}%
    \Test{1234}%
  \endgroup
  \Expect*{\the\TestAttr}*{\number\LuT@UnsetAttributeValue}%
\end{qstest}
%    \end{macrocode}
%
%    \begin{macrocode}
\@@end
%</test4>
%    \end{macrocode}
%
% \subsection{Short test for \plainTeX}
%
%    \begin{macrocode}
%<*test5>
\input luatex.sty\relax
\newattribute\TestAttr
\setattribute\TestAttr{10}
\unsetattribute\TestAttr
\newcatcodetable\TestCTa
\begingroup
  \SetCatcodeRange{`A}{`Z}{12}%
\endgroup
\BeginCatcodeRegime\CatcodeTableLaTeX
\EndCatcodeRegime
\end
%</test5>
%    \end{macrocode}
%
% \section{Installation}
%
% \subsection{Download}
%
% \paragraph{Package.} This package is available on
% CTAN\footnote{\url{ftp://ftp.ctan.org/tex-archive/}}:
% \begin{description}
% \item[\CTAN{macros/latex/contrib/oberdiek/luatex.dtx}] The source file.
% \item[\CTAN{macros/latex/contrib/oberdiek/luatex.pdf}] Documentation.
% \end{description}
%
%
% \paragraph{Bundle.} All the packages of the bundle `oberdiek'
% are also available in a TDS compliant ZIP archive. There
% the packages are already unpacked and the documentation files
% are generated. The files and directories obey the TDS standard.
% \begin{description}
% \item[\CTAN{install/macros/latex/contrib/oberdiek.tds.zip}]
% \end{description}
% \emph{TDS} refers to the standard ``A Directory Structure
% for \TeX\ Files'' (\CTAN{tds/tds.pdf}). Directories
% with \xfile{texmf} in their name are usually organized this way.
%
% \subsection{Bundle installation}
%
% \paragraph{Unpacking.} Unpack the \xfile{oberdiek.tds.zip} in the
% TDS tree (also known as \xfile{texmf} tree) of your choice.
% Example (linux):
% \begin{quote}
%   |unzip oberdiek.tds.zip -d ~/texmf|
% \end{quote}
%
% \paragraph{Script installation.}
% Check the directory \xfile{TDS:scripts/oberdiek/} for
% scripts that need further installation steps.
% Package \xpackage{attachfile2} comes with the Perl script
% \xfile{pdfatfi.pl} that should be installed in such a way
% that it can be called as \texttt{pdfatfi}.
% Example (linux):
% \begin{quote}
%   |chmod +x scripts/oberdiek/pdfatfi.pl|\\
%   |cp scripts/oberdiek/pdfatfi.pl /usr/local/bin/|
% \end{quote}
%
% \subsection{Package installation}
%
% \paragraph{Unpacking.} The \xfile{.dtx} file is a self-extracting
% \docstrip\ archive. The files are extracted by running the
% \xfile{.dtx} through \plainTeX:
% \begin{quote}
%   \verb|tex luatex.dtx|
% \end{quote}
%
% \paragraph{TDS.} Now the different files must be moved into
% the different directories in your installation TDS tree
% (also known as \xfile{texmf} tree):
% \begin{quote}
% \def\t{^^A
% \begin{tabular}{@{}>{\ttfamily}l@{ $\rightarrow$ }>{\ttfamily}l@{}}
%   luatex.sty & tex/generic/oberdiek/luatex.sty\\
%   luatex-loader.sty & tex/generic/oberdiek/luatex-loader.sty\\
%   oberdiek.luatex.lua & scripts/oberdiek/oberdiek.luatex.lua\\
%   luatex.pdf & doc/latex/oberdiek/luatex.pdf\\
%   test/luatex-test1.tex & doc/latex/oberdiek/test/luatex-test1.tex\\
%   test/luatex-test2.tex & doc/latex/oberdiek/test/luatex-test2.tex\\
%   test/luatex-test3.tex & doc/latex/oberdiek/test/luatex-test3.tex\\
%   test/luatex-test4.tex & doc/latex/oberdiek/test/luatex-test4.tex\\
%   test/luatex-test5.tex & doc/latex/oberdiek/test/luatex-test5.tex\\
%   luatex.dtx & source/latex/oberdiek/luatex.dtx\\
% \end{tabular}^^A
% }^^A
% \sbox0{\t}^^A
% \ifdim\wd0>\linewidth
%   \begingroup
%     \advance\linewidth by\leftmargin
%     \advance\linewidth by\rightmargin
%   \edef\x{\endgroup
%     \def\noexpand\lw{\the\linewidth}^^A
%   }\x
%   \def\lwbox{^^A
%     \leavevmode
%     \hbox to \linewidth{^^A
%       \kern-\leftmargin\relax
%       \hss
%       \usebox0
%       \hss
%       \kern-\rightmargin\relax
%     }^^A
%   }^^A
%   \ifdim\wd0>\lw
%     \sbox0{\small\t}^^A
%     \ifdim\wd0>\linewidth
%       \ifdim\wd0>\lw
%         \sbox0{\footnotesize\t}^^A
%         \ifdim\wd0>\linewidth
%           \ifdim\wd0>\lw
%             \sbox0{\scriptsize\t}^^A
%             \ifdim\wd0>\linewidth
%               \ifdim\wd0>\lw
%                 \sbox0{\tiny\t}^^A
%                 \ifdim\wd0>\linewidth
%                   \lwbox
%                 \else
%                   \usebox0
%                 \fi
%               \else
%                 \lwbox
%               \fi
%             \else
%               \usebox0
%             \fi
%           \else
%             \lwbox
%           \fi
%         \else
%           \usebox0
%         \fi
%       \else
%         \lwbox
%       \fi
%     \else
%       \usebox0
%     \fi
%   \else
%     \lwbox
%   \fi
% \else
%   \usebox0
% \fi
% \end{quote}
% If you have a \xfile{docstrip.cfg} that configures and enables \docstrip's
% TDS installing feature, then some files can already be in the right
% place, see the documentation of \docstrip.
%
% \subsection{Refresh file name databases}
%
% If your \TeX~distribution
% (\teTeX, \mikTeX, \dots) relies on file name databases, you must refresh
% these. For example, \teTeX\ users run \verb|texhash| or
% \verb|mktexlsr|.
%
% \subsection{Some details for the interested}
%
% \paragraph{Attached source.}
%
% The PDF documentation on CTAN also includes the
% \xfile{.dtx} source file. It can be extracted by
% AcrobatReader 6 or higher. Another option is \textsf{pdftk},
% e.g. unpack the file into the current directory:
% \begin{quote}
%   \verb|pdftk luatex.pdf unpack_files output .|
% \end{quote}
%
% \paragraph{Unpacking with \LaTeX.}
% The \xfile{.dtx} chooses its action depending on the format:
% \begin{description}
% \item[\plainTeX:] Run \docstrip\ and extract the files.
% \item[\LaTeX:] Generate the documentation.
% \end{description}
% If you insist on using \LaTeX\ for \docstrip\ (really,
% \docstrip\ does not need \LaTeX), then inform the autodetect routine
% about your intention:
% \begin{quote}
%   \verb|latex \let\install=y\input{luatex.dtx}|
% \end{quote}
% Do not forget to quote the argument according to the demands
% of your shell.
%
% \paragraph{Generating the documentation.}
% You can use both the \xfile{.dtx} or the \xfile{.drv} to generate
% the documentation. The process can be configured by the
% configuration file \xfile{ltxdoc.cfg}. For instance, put this
% line into this file, if you want to have A4 as paper format:
% \begin{quote}
%   \verb|\PassOptionsToClass{a4paper}{article}|
% \end{quote}
% An example follows how to generate the
% documentation with pdf\LaTeX:
% \begin{quote}
%\begin{verbatim}
%pdflatex luatex.dtx
%makeindex -s gind.ist luatex.idx
%pdflatex luatex.dtx
%makeindex -s gind.ist luatex.idx
%pdflatex luatex.dtx
%\end{verbatim}
% \end{quote}
%
% \section{Catalogue}
%
% The following XML file can be used as source for the
% \href{http://mirror.ctan.org/help/Catalogue/catalogue.html}{\TeX\ Catalogue}.
% The elements \texttt{caption} and \texttt{description} are imported
% from the original XML file from the Catalogue.
% The name of the XML file in the Catalogue is \xfile{luatex.xml}.
%    \begin{macrocode}
%<*catalogue>
<?xml version='1.0' encoding='us-ascii'?>
<!DOCTYPE entry SYSTEM 'catalogue.dtd'>
<entry datestamp='$Date$' modifier='$Author$' id='luatex'>
  <name>luatex</name>
  <caption>The LuaTeX engine.</caption>
  <authorref id='auth:oberdiek'/>
  <copyright owner='Heiko Oberdiek' year='2007,2009,2010'/>
  <license type='lppl1.3'/>
  <version number='0.4'/>
  <description>
    LuaTeX is an extended version of pdfTeX using Lua as an embedded
    scripting language.  The LuaTeX project&#x2019;s main objective
    is to provide an open and configurable variant of TeX while at the
    same time offering downward compatibility.
    <p/>
    LuaTeX uses Unicode (as UTF-8) as its default input encoding, and
    is able to use modern (OpenType) fonts (for both text and mathematics).
    <p/>
    It should be noted that LuaTeX is still under development; its
    specification has been declared stable, but absolute stability
    may not in practice be assumed.
    <p/>
    The package is part of the <xref refid='oberdiek'>oberdiek</xref> bundle.
  </description>
  <documentation details='Package documentation'
      href='ctan:/macros/latex/contrib/oberdiek/luatex.pdf'/>
  <ctan file='true' path='/macros/latex/contrib/oberdiek/luatex.dtx'/>
  <miktex location='oberdiek'/>
  <texlive location='oberdiek'/>
  <install path='/macros/latex/contrib/oberdiek/oberdiek.tds.zip'/>
</entry>
%</catalogue>
%    \end{macrocode}
%
% \begin{History}
%   \begin{Version}{2007/12/12 v0.1}
%   \item
%     First public version.
%   \end{Version}
%   \begin{Version}{2009/04/10 v0.2}
%   \item
%     Requires package \xpackage{ifluatex} in version 2.0 to ensure
%     \cs{luatexversion}.
%   \item
%     Updates the call of \cs{directlua}, the syntax has changed
%     in \LuaTeX\ 0.36.
%   \end{Version}
%   \begin{Version}{2009/12/02 v0.3}
%   \item
%     Unsetting of attributes updated for \LuaTeX\ 0.37.
%   \end{Version}
%   \begin{Version}{2010/03/09 v0.4}
%   \item
%     Support for lua states removed.
%   \item
%     Calling tex.enableprimitives for used primitives.
%   \end{Version}
% \end{History}
%
% \PrintIndex
%
% \Finale
\endinput
|
% \end{quote}
% Do not forget to quote the argument according to the demands
% of your shell.
%
% \paragraph{Generating the documentation.}
% You can use both the \xfile{.dtx} or the \xfile{.drv} to generate
% the documentation. The process can be configured by the
% configuration file \xfile{ltxdoc.cfg}. For instance, put this
% line into this file, if you want to have A4 as paper format:
% \begin{quote}
%   \verb|\PassOptionsToClass{a4paper}{article}|
% \end{quote}
% An example follows how to generate the
% documentation with pdf\LaTeX:
% \begin{quote}
%\begin{verbatim}
%pdflatex luatex.dtx
%makeindex -s gind.ist luatex.idx
%pdflatex luatex.dtx
%makeindex -s gind.ist luatex.idx
%pdflatex luatex.dtx
%\end{verbatim}
% \end{quote}
%
% \section{Catalogue}
%
% The following XML file can be used as source for the
% \href{http://mirror.ctan.org/help/Catalogue/catalogue.html}{\TeX\ Catalogue}.
% The elements \texttt{caption} and \texttt{description} are imported
% from the original XML file from the Catalogue.
% The name of the XML file in the Catalogue is \xfile{luatex.xml}.
%    \begin{macrocode}
%<*catalogue>
<?xml version='1.0' encoding='us-ascii'?>
<!DOCTYPE entry SYSTEM 'catalogue.dtd'>
<entry datestamp='$Date$' modifier='$Author$' id='luatex'>
  <name>luatex</name>
  <caption>The LuaTeX engine.</caption>
  <authorref id='auth:oberdiek'/>
  <copyright owner='Heiko Oberdiek' year='2007,2009,2010'/>
  <license type='lppl1.3'/>
  <version number='0.4'/>
  <description>
    LuaTeX is an extended version of pdfTeX using Lua as an embedded
    scripting language.  The LuaTeX project&#x2019;s main objective
    is to provide an open and configurable variant of TeX while at the
    same time offering downward compatibility.
    <p/>
    LuaTeX uses Unicode (as UTF-8) as its default input encoding, and
    is able to use modern (OpenType) fonts (for both text and mathematics).
    <p/>
    It should be noted that LuaTeX is still under development; its
    specification has been declared stable, but absolute stability
    may not in practice be assumed.
    <p/>
    The package is part of the <xref refid='oberdiek'>oberdiek</xref> bundle.
  </description>
  <documentation details='Package documentation'
      href='ctan:/macros/latex/contrib/oberdiek/luatex.pdf'/>
  <ctan file='true' path='/macros/latex/contrib/oberdiek/luatex.dtx'/>
  <miktex location='oberdiek'/>
  <texlive location='oberdiek'/>
  <install path='/macros/latex/contrib/oberdiek/oberdiek.tds.zip'/>
</entry>
%</catalogue>
%    \end{macrocode}
%
% \begin{History}
%   \begin{Version}{2007/12/12 v0.1}
%   \item
%     First public version.
%   \end{Version}
%   \begin{Version}{2009/04/10 v0.2}
%   \item
%     Requires package \xpackage{ifluatex} in version 2.0 to ensure
%     \cs{luatexversion}.
%   \item
%     Updates the call of \cs{directlua}, the syntax has changed
%     in \LuaTeX\ 0.36.
%   \end{Version}
%   \begin{Version}{2009/12/02 v0.3}
%   \item
%     Unsetting of attributes updated for \LuaTeX\ 0.37.
%   \end{Version}
%   \begin{Version}{2010/03/09 v0.4}
%   \item
%     Support for lua states removed.
%   \item
%     Calling tex.enableprimitives for used primitives.
%   \end{Version}
% \end{History}
%
% \PrintIndex
%
% \Finale
\endinput

%        (quote the arguments according to the demands of your shell)
%
% Documentation:
%    (a) If luatex.drv is present:
%           latex luatex.drv
%    (b) Without luatex.drv:
%           latex luatex.dtx; ...
%    The class ltxdoc loads the configuration file ltxdoc.cfg
%    if available. Here you can specify further options, e.g.
%    use A4 as paper format:
%       \PassOptionsToClass{a4paper}{article}
%
%    Programm calls to get the documentation (example):
%       pdflatex luatex.dtx
%       makeindex -s gind.ist luatex.idx
%       pdflatex luatex.dtx
%       makeindex -s gind.ist luatex.idx
%       pdflatex luatex.dtx
%
% Installation:
%    TDS:tex/generic/oberdiek/luatex.sty
%    TDS:tex/generic/oberdiek/luatex-loader.sty
%    TDS:scripts/oberdiek/oberdiek.luatex.lua
%    TDS:doc/latex/oberdiek/luatex.pdf
%    TDS:doc/latex/oberdiek/test/luatex-test1.tex
%    TDS:doc/latex/oberdiek/test/luatex-test2.tex
%    TDS:doc/latex/oberdiek/test/luatex-test3.tex
%    TDS:doc/latex/oberdiek/test/luatex-test4.tex
%    TDS:doc/latex/oberdiek/test/luatex-test5.tex
%    TDS:source/latex/oberdiek/luatex.dtx
%
%<*ignore>
\begingroup
  \catcode123=1 %
  \catcode125=2 %
  \def\x{LaTeX2e}%
\expandafter\endgroup
\ifcase 0\ifx\install y1\fi\expandafter
         \ifx\csname processbatchFile\endcsname\relax\else1\fi
         \ifx\fmtname\x\else 1\fi\relax
\else\csname fi\endcsname
%</ignore>
%<*install>
\input docstrip.tex
\Msg{************************************************************************}
\Msg{* Installation}
\Msg{* Package: luatex 2010/03/09 v0.4 LuaTeX basic definition package (HO)}
\Msg{************************************************************************}

\keepsilent
\askforoverwritefalse

\let\MetaPrefix\relax
\preamble

This is a generated file.

Project: luatex
Version: 2010/03/09 v0.4

Copyright (C) 2007, 2009, 2010 by
   Heiko Oberdiek <heiko.oberdiek at googlemail.com>

This work may be distributed and/or modified under the
conditions of the LaTeX Project Public License, either
version 1.3c of this license or (at your option) any later
version. This version of this license is in
   http://www.latex-project.org/lppl/lppl-1-3c.txt
and the latest version of this license is in
   http://www.latex-project.org/lppl.txt
and version 1.3 or later is part of all distributions of
LaTeX version 2005/12/01 or later.

This work has the LPPL maintenance status "maintained".

This Current Maintainer of this work is Heiko Oberdiek.

The Base Interpreter refers to any `TeX-Format',
because some files are installed in TDS:tex/generic//.

This work consists of the main source file luatex.dtx
and the derived files
   luatex.sty, luatex.pdf, luatex.ins, luatex.drv, luatex-loader.sty,
   luatex-test1.tex, luatex-test2.tex, luatex-test3.tex,
   luatex-test4.tex, luatex-test5.tex, oberdiek.luatex.lua.

\endpreamble
\let\MetaPrefix\DoubleperCent

\generate{%
  \file{luatex.ins}{\from{luatex.dtx}{install}}%
  \file{luatex.drv}{\from{luatex.dtx}{driver}}%
  \usedir{tex/generic/oberdiek}%
  \file{luatex.sty}{\from{luatex.dtx}{package}}%
  \file{luatex-loader.sty}{\from{luatex.dtx}{loader}}%
  \usedir{doc/latex/oberdiek/test}%
  \file{luatex-test1.tex}{\from{luatex.dtx}{test1}}%
  \file{luatex-test2.tex}{\from{luatex.dtx}{test1,test2}}%
  \file{luatex-test3.tex}{\from{luatex.dtx}{test1,test3}}%
  \file{luatex-test4.tex}{\from{luatex.dtx}{test4}}%
  \file{luatex-test5.tex}{\from{luatex.dtx}{test5}}%
  \nopreamble
  \nopostamble
  \usedir{source/latex/oberdiek/catalogue}%
  \file{luatex.xml}{\from{luatex.dtx}{catalogue}}%
}
\def\MetaPrefix{-- }
\def\defaultpostamble{%
  \MetaPrefix^^J%
  \MetaPrefix\space End of File `\outFileName'.%
}
\def\currentpostamble{\defaultpostamble}%
\generate{%
  \usedir{scripts/oberdiek}%
  \file{oberdiek.luatex.lua}{\from{luatex.dtx}{lua}}%
}

\catcode32=13\relax% active space
\let =\space%
\Msg{************************************************************************}
\Msg{*}
\Msg{* To finish the installation you have to move the following}
\Msg{* files into a directory searched by TeX:}
\Msg{*}
\Msg{*     luatex.sty, luatex-loader.sty}
\Msg{*}
\Msg{* And install the following script file:}
\Msg{*}
\Msg{*     oberdiek.luatex.lua}
\Msg{*}
\Msg{* To produce the documentation run the file `luatex.drv'}
\Msg{* through LaTeX.}
\Msg{*}
\Msg{* Happy TeXing!}
\Msg{*}
\Msg{************************************************************************}

\endbatchfile
%</install>
%<*ignore>
\fi
%</ignore>
%<*driver>
\NeedsTeXFormat{LaTeX2e}
\ProvidesFile{luatex.drv}%
  [2010/03/09 v0.4 LuaTeX basic definition package (HO)]%
\documentclass{ltxdoc}
\usepackage{holtxdoc}[2011/11/22]
\usepackage{paralist}
\begin{document}
  \DocInput{luatex.dtx}%
\end{document}
%</driver>
% \fi
%
% \CheckSum{1429}
%
% \CharacterTable
%  {Upper-case    \A\B\C\D\E\F\G\H\I\J\K\L\M\N\O\P\Q\R\S\T\U\V\W\X\Y\Z
%   Lower-case    \a\b\c\d\e\f\g\h\i\j\k\l\m\n\o\p\q\r\s\t\u\v\w\x\y\z
%   Digits        \0\1\2\3\4\5\6\7\8\9
%   Exclamation   \!     Double quote  \"     Hash (number) \#
%   Dollar        \$     Percent       \%     Ampersand     \&
%   Acute accent  \'     Left paren    \(     Right paren   \)
%   Asterisk      \*     Plus          \+     Comma         \,
%   Minus         \-     Point         \.     Solidus       \/
%   Colon         \:     Semicolon     \;     Less than     \<
%   Equals        \=     Greater than  \>     Question mark \?
%   Commercial at \@     Left bracket  \[     Backslash     \\
%   Right bracket \]     Circumflex    \^     Underscore    \_
%   Grave accent  \`     Left brace    \{     Vertical bar  \|
%   Right brace   \}     Tilde         \~}
%
% \GetFileInfo{luatex.drv}
%
% \title{The \xpackage{luatex} package}
% \date{2010/03/09 v0.4}
% \author{Heiko Oberdiek\\\xemail{heiko.oberdiek at googlemail.com}}
%
% \maketitle
%
% \begin{abstract}
% This package manages the new and extended features and resources
% that \LuaTeX\ provides. Examples are attributes and catcode
% tables.
% \end{abstract}
%
% \tableofcontents
%
% \section{Documentation}
%
% \subsection{Introduction}
%
% \TeX\ provides global resources such as registers.
% But it does not provide an interface for managing these resources.
% For example, two packages want to use a counter register.
% If they take the same register
% number, then the use of both packages will conflict and they cannot
% be used together.
% Therefore formats such as \plainTeX\ or \LaTeX\ implement
% an allocation scheme for registers. A package reserves with
% \cs{newcount} an unused register number for its own exclusive use.
%
% Nowadays \TeX\ is not alone anymore: \eTeX, \pdfTeX\ and other
% compilers for \TeX\ are developed that extend and add new
% features and resources.
%
% Now \LuaTeX\ has reached beta state. It inherits most of
% \pdfTeX's features including \eTeX. Also it implements
% new concepts such as attributes or catcode tables.
%
% \subsubsection{\LaTeX}
%
% \LaTeXe\ is frozen and therefore refuses to even notice
% the new \TeX\ variants. Not even the old \eTeX\ is supported
% by its kernel. At least there is a third party package \xpackage{etex}
% that manages the new \eTeX\ resources.
%
% This package tries to do the same for \LuaTeX\ and starts to
% support at least a few of the new features.
%
% \subsubsection{\plainTeX}
%
% \LaTeX\ has inherited its resource handling from \plainTeX.
% The interface is basically the same: \cs{newcount}, \dots
% Therefore this package tries to follow this tradition by
% providing compatibility to \plainTeX. It can be loaded
% with \plainTeX\ and defines at least some of the
% features that this packages provides for \LaTeX.
%
% \subsection{Register allocation}
%
% \subsubsection{Register with 16 bit}
%
% Because \LuaTeX\ is a super set of \eTeX\ regarding registers,
% the register allocation scheme should not conflict with
% package \xpackage{etex}. Therefore this package is loaded
% to inherit its allocation scheme. The only change is currently
% that the limit is increased to 65536 registers for the following
% register classes:
% \begin{compactitem}
% \item count
% \item dimen
% \item skip
% \item muskip
% \item marks
% \item toks
% \item box
% \end{compactitem}
% This affects the number of global and local registers.
% Because it is done in a package and not in the kernel,
% it is possible that someone loads package \xpackage{etex}
% before uses the local allocation variants. This will prevent
% the extension for this register class. If more registers are
% needed, just load package \xpackage{luatex} earlier.
%
% \subsubsection{Insertions}
%
% Insertions need four registers \cs{count}, \cs{dimen}, \cs{skip},
% and \cs{box} with the same number. Usually they are allocated
% downwards from 254, 253, \dots Also \cs{newcount}, \cs{newdimen}, \dots
% fill up these register numbers from below before switching to
% higher register numbers by package \xpackage{etex}.
% When this occurs, no insertions can be allocated anymore.
%
% Therefore \cs{newcount}, \cs{newdimen}, \cs{newskip}, and
% \cs{newbox} are replaced by their global variants (\cs{globcount}, \dots)
% that use the higher numbers immediately, leaving the room for insertions.
% There should not be an efficiency penalty because \LuaTeX\ stores
% the registers of a class in the same Lua table unlike \eTeX, where
% registers below 256 are stored in an array and higher numbers
% are put in a tree structure.
%
% \subsection{Attributes}
%
% Nodes can have custom attributes in \LuaTeX. These attributes
% are organized by a new register class. As the other registers
% up to $2^{16}$ attributes are supported. An attribute value
% can be negative that means the attribute is not set. Otherwise
% \TeX's range of non-negative integers up to $2^31$ are available.
%
% \begin{declcs}{newattribute} \M{cmd}
% \end{declcs}
% Macro \cs{newattribute} defines command \meta{cmd} using \cs{attributedef}
% using an new attribute number. The new attribute is initially unset.
%
% \begin{declcs}{setattribute} \M{cmd} \M{value}
% \end{declcs}
% Macro \cs{setattribute} locally sets attribute command \meta{cmd}
% to the number \meta{value}. Valid values range from $-1$ until $2^31$
% (the upper limit is the same as for other \TeX\ integer numbers).
%
% \begin{declcs}{unsetattribute} \M{cmd}
% \end{declcs}
% Macro \cs{unsettattribute} clears the attribute command \meta{cmd}.
%
% \subsection{Catcode tables}
%
% \LuaTeX\ introduces catcode tables as new feature, see documentation.
% There is need for discussion, how to deal best:
% \begin{itemize}
% \item \cs{initcatcodetable} and \cs{setcatcodetable} act globally.
% \item \cs{catcodetable} causes an error if used with an uninitialized
%   catcode table.
% \item Large catcode table numbers should be avoided because of
%   performance breakdown.
% \item Use case \LaTeX\ package:
%   The package must not be surprised by changed catcodes and
%   must not surprise by changing catcodes accidently.
%   Catcode tables could offer a solution. At the begin
%   a catcode regime with standard catcodes is established
%   and the old one is restored afterwards.
% \item Use case: \LuaTeX's |tex.print| might be used with
%   a catcode table number, for example a table where all
%   entries have catcode ``other''.
% \item Readonly catcode tables.
% \item Is there is a need for local allocations?
%   (Package \xpackage{etex}'s \cs{loc} variants are not used in
%   \TeX\ Live 2007.)
% \end{itemize}
%
% \subsubsection{Interface proposal}
%
% The idea: \cs{newcatcodetable} allocates odd numbered catcode
% tables. Even numbered tables are managed as stack.
% Also some catcode tables are defined. These must not be
% changed.
%
% \begin{declcs}{newcatcodetable} \M{cmd}
% \end{declcs}
% Macro \cs{newcatcodetable} reserves a new catcode table and
% remembers its number in \meta{cmd}. The catcode table is
% initialized with \iniTeX's catcodes.
%
% \begin{declcs}{CatcodeTableIniTeX}\\
%   \cs{CatcodeTableString}\\
%   \cs{CatcodeTableOther}\\
%   \cs{CatcodeTableLaTeX}\\
% \end{declcs}
% These are catcode tables and must not be changed.
% \cs{CatcodeTableIniTeX} contains the catcode settings of \iniTeX.
% \cs{CatcodeTableString} follows \TeX's convention of
% \cs{string}, \cs{meaning} and friends. The space gets catcode
% 10 (space), the other characters have catcode 12 (other).
% In \cs{CatcodeTableOther} all entries have catcode 12 (other).
% \cs{CatcodeTableLaTeX} contains the setting of a pure \LaTeX\ format
% (`at' is other).
%
% \begin{declcs}{CatcodeTableStack}\\
%   \cs{IncCatcodeTableStack}\\
%   \cs{DecCatcodeTableStack}
% \end{declcs}
% \cs{CatcodeTableStack} is the stack pointer. Initially
% it is catcode table zero. \cs{IncCatcodeTableStack} and
% \cs{DecCatcodeTableStack} increments and decrements the
% stack pointer. Currently \cs{IncCatcodeTableStack} does not
% initialize a new catcode table. Both increment and decrement
% operations do not set a catcode table.
%
% \begin{declcs}{PushCatcodeTableNumStack}\\
%   \cs{PopCatcodeTableNumStack}
% \end{declcs}
% It can be handy to have a global stack for catcode table
% numbers to deal with the global assignment property of
% \cs{initcatcodetable} and \cs{savecatcodetable}.
% \cs{PushCatcodeTableNumStack} pushes the current catcode table
% on the stack.
% \cs{PopCatcodeTableNumStack} pops the topmost number off the number
% stack to set the current catcode table. Catcode table zero is
% used in case of an empty stack.
%
% \begin{declcs}{BeginCatcodeRegime} \M{catcodetable}\\
%   \cs{EndCatcodeRegime}
% \end{declcs}
% \cs{BeginCatcodeRegime} remembers the current catcode table number.
% Then it creates and uses a fresh catcode table on the stack
% that is initialized by \meta{catcodetable}:
% \begin{quote}\ttfamily
%   \cs{PushCatcodeTableNumStack}\\
%   \cs{catcodetable}\meta{catcodetable}||
%   \cs{IncCatcodeTableStack}\\
%   \cs{savecatcodetable}\cs{CatcodeTableStack}\\
%   \cs{catcodetable}\cs{CatcodeTableStack}
% \end{quote}
% \cs{EndCatcodeRegime} drops the catcode table, created by
% \cs{BeginCatcodeRegime} and sets the catcode table that was
% active before:
% \begin{quote}
%   \cs{DecCatcodeTableStack}\\
%   \cs{PopCatcodeTableNumStack}
% \end{quote}
% These macros solve the use case, described earlier for a \LaTeX\ package:
% \begin{quote}
%   |% package foobar.sty|\\
%   |\BeginCatcodeRegime\CatcodeTableLaTeX|\\
%   |\makeatletter|\\
%   |% ... package contents ...|\\
%   |\EndCatcodeRegime|\\
%   |% end of package|
% \end{quote}
% If the package wants to change catcodes after its loading,
% \cs{AtBeginDocument} or \cs{AtEndOfPackage} can be used.
%
% \begin{declcs}{SetCatcodeRange} \M{from} \M{to} \M{catcode}
% \end{declcs}
% The catcodes of characters in range from \meta{from}
% to inclusive \meta{to} are set to \meta{catcode}.
%
% \subsection{Lua module loading}
%
% Currently \LuaTeX\ (version 0.20) does not support Lua script
% files inside \xfile{TDS:scripts//}, because Lua's mechanism
% for module loading does not use the \textsf{kpathsea} library.
% Therefore this packages appends a kpse loader to the list of
% Lua's module loaders. It finds the module \meta{module} by
% \begin{quote}
%   \ttfamily |kpse.find_file("|\meta{module}|.lua", "texmfscripts")|
% \end{quote}
% Unhappily \textsf{kpathsea} does not support directory components
% in a file name. Therefore the Lua convention is not followed to
% replace dots in the module name by the directory separator.
%
% Example:
% A Lua script of a package \xpackage{foobar} wants the following
% modules:
% \begin{quote}
%   |require("foobar.hello.world")|\\
%   |require("org.somewhere.xyz")|
% \end{quote}
% Then they can be find in:
% \begin{quote}
%   |TDS:scripts/foobar/foobar.hello.world.lua|\\
%   |TDS:scripts/foobar/org.somewhere.xyz.lua|
% \end{quote}
% I would have preferred the following locations, following lua conventions,
% e.\,g.:
% \begin{quote}
%   |TDS:scripts/foobar/hello/world.lua|\\
%   |TDS:scripts/foobar/org/somewhere/xyz.lua|
% \end{quote}
% But I do not know, how to achieve this in a reliable way using
% \textsf{kpathsea}.
%
% \subsubsection{Package \xpackage{luatex-loader}}
%
% If someone do not need or want package \xpackage{luatex} but
% it's extension for module loading, then he can use
% package \xpackage{luatex-loader}. Both \plainTeX\ and \LaTeX\ are
% supported.
%
% \StopEventually{
% }
%
% \section{Implementation}
%
%    \begin{macrocode}
%<*package>
%    \end{macrocode}
%
% \subsection{Reload check and package identification}
%    Reload check, especially if the package is not used with \LaTeX.
%    \begin{macrocode}
\begingroup\catcode61\catcode48\catcode32=10\relax%
  \catcode13=5 % ^^M
  \endlinechar=13 %
  \catcode35=6 % #
  \catcode39=12 % '
  \catcode44=12 % ,
  \catcode45=12 % -
  \catcode46=12 % .
  \catcode58=12 % :
  \catcode64=11 % @
  \catcode123=1 % {
  \catcode125=2 % }
  \expandafter\let\expandafter\x\csname ver@luatex.sty\endcsname
  \ifx\x\relax % plain-TeX, first loading
  \else
    \def\empty{}%
    \ifx\x\empty % LaTeX, first loading,
      % variable is initialized, but \ProvidesPackage not yet seen
    \else
      \expandafter\ifx\csname PackageInfo\endcsname\relax
        \def\x#1#2{%
          \immediate\write-1{Package #1 Info: #2.}%
        }%
      \else
        \def\x#1#2{\PackageInfo{#1}{#2, stopped}}%
      \fi
      \x{luatex}{The package is already loaded}%
      \aftergroup\endinput
    \fi
  \fi
\endgroup%
%    \end{macrocode}
%    Package identification:
%    \begin{macrocode}
\begingroup\catcode61\catcode48\catcode32=10\relax%
  \catcode13=5 % ^^M
  \endlinechar=13 %
  \catcode35=6 % #
  \catcode39=12 % '
  \catcode40=12 % (
  \catcode41=12 % )
  \catcode44=12 % ,
  \catcode45=12 % -
  \catcode46=12 % .
  \catcode47=12 % /
  \catcode58=12 % :
  \catcode64=11 % @
  \catcode91=12 % [
  \catcode93=12 % ]
  \catcode123=1 % {
  \catcode125=2 % }
  \expandafter\ifx\csname ProvidesPackage\endcsname\relax
    \def\x#1#2#3[#4]{\endgroup
      \immediate\write-1{Package: #3 #4}%
      \xdef#1{#4}%
    }%
  \else
    \def\x#1#2[#3]{\endgroup
      #2[{#3}]%
      \ifx#1\@undefined
        \xdef#1{#3}%
      \fi
      \ifx#1\relax
        \xdef#1{#3}%
      \fi
    }%
  \fi
\expandafter\x\csname ver@luatex.sty\endcsname
\ProvidesPackage{luatex}%
  [2010/03/09 v0.4 LuaTeX basic definition package (HO)]%
%    \end{macrocode}
%
% \subsection{Catcodes}
%
%    \begin{macrocode}
\begingroup\catcode61\catcode48\catcode32=10\relax%
  \catcode13=5 % ^^M
  \endlinechar=13 %
  \catcode123=1 % {
  \catcode125=2 % }
  \catcode64=11 % @
  \def\x{\endgroup
    \expandafter\edef\csname LuT@AtEnd\endcsname{%
      \endlinechar=\the\endlinechar\relax
      \catcode13=\the\catcode13\relax
      \catcode32=\the\catcode32\relax
      \catcode35=\the\catcode35\relax
      \catcode61=\the\catcode61\relax
      \catcode64=\the\catcode64\relax
      \catcode123=\the\catcode123\relax
      \catcode125=\the\catcode125\relax
    }%
  }%
\x\catcode61\catcode48\catcode32=10\relax%
\catcode13=5 % ^^M
\endlinechar=13 %
\catcode35=6 % #
\catcode64=11 % @
\catcode123=1 % {
\catcode125=2 % }
\def\TMP@EnsureCode#1#2{%
  \edef\LuT@AtEnd{%
    \LuT@AtEnd
    \catcode#1=\the\catcode#1\relax
  }%
  \catcode#1=#2\relax
}
\TMP@EnsureCode{10}{12}% ^^J
\TMP@EnsureCode{34}{12}% "
\TMP@EnsureCode{36}{3}% $
\TMP@EnsureCode{39}{12}% '
\TMP@EnsureCode{40}{12}% (
\TMP@EnsureCode{41}{12}% )
\TMP@EnsureCode{42}{12}% *
\TMP@EnsureCode{43}{12}% +
\TMP@EnsureCode{44}{12}% ,
\TMP@EnsureCode{45}{12}% -
\TMP@EnsureCode{46}{12}% .
\TMP@EnsureCode{47}{12}% /
\TMP@EnsureCode{60}{12}% <
\TMP@EnsureCode{62}{12}% >
\TMP@EnsureCode{91}{12}% [
\TMP@EnsureCode{93}{12}% ]
\TMP@EnsureCode{95}{12}% _ (other!)
\TMP@EnsureCode{96}{12}% `
\edef\LuT@AtEnd{\LuT@AtEnd\noexpand\endinput}
%    \end{macrocode}
%
% \subsection{Check for \LuaTeX}
%
%    Without \LuaTeX\ there is no point in using this package.
%    \begin{macrocode}
\begingroup\expandafter\expandafter\expandafter\endgroup
\expandafter\ifx\csname RequirePackage\endcsname\relax
  \input infwarerr.sty\relax
  \input ifluatex.sty\relax
\else
  \RequirePackage{infwarerr}[2007/09/09]%
  \RequirePackage{ifluatex}[2009/04/10]%
\fi
%    \end{macrocode}
%    \begin{macrocode}
\ifluatex
\else
  \@PackageError{luatex}{%
    This package may only be run using LuaTeX%
  }\@ehc
  \expandafter\LuT@AtEnd
\fi%
%    \end{macrocode}
%
% \subsection{Provide Lua\TeX\ primitives}
%
%    \begin{macrocode}
\ifnum\luatexversion<36 %
  \def\LuT@MakePrimitive#1{%
    \expandafter\let\csname luatex#1\expandafter\endcsname
    \csname #1\endcsname
  }%
\else
  \def\LuT@MakeLuatexPrimitive#1{%
    \begingroup\expandafter\expandafter\expandafter\endgroup
    \expandafter\ifx\csname luatex#1\endcsname\relax
      \begingroup\expandafter\expandafter\expandafter\endgroup
      \expandafter\ifx\csname #1\endcsname\relax
      \else
        \expandafter\let
        \csname luatex#1\expandafter\endcsname
        \csname #1\endcsname
      \fi
    \fi
    \begingroup\expandafter\expandafter\expandafter\endgroup
    \expandafter\ifx\csname luatex#1\endcsname\relax
      \begingroup
        \expandafter\let\csname luatex#1\endcsname\@undefined
        \ifnum0%
          \directlua{%
            if tex.enableprimitives then %
              tex.enableprimitives('luatex',{'#1'})%
              tex.print('1')%
            end%
          }%
          \expandafter\ifx\csname luatex#1\endcsname\relax\else1\fi
        =11 %
          \global\expandafter\let
          \csname luatex#1\expandafter\endcsname
          \csname luatex#1\endcsname
        \else
          \@PackageError{luatex}{%
            tex.enableprimitives failed for `#1'%
          }\@ehc
        \fi
      \endgroup
    \fi
  }%
  \def\LuT@MakePrimitive#1{%
    \begingroup\expandafter\expandafter\expandafter\endgroup
    \expandafter\ifx\csname#1\endcsname\relax
      \begingroup
        \expandafter\let\csname#1\endcsname\@undefined
        \ifnum0%
          \directlua{%
            if tex.enableprimitives then %
              tex.enableprimitives('',{'#1'})%
              tex.print('1')%
            end%
          }%
          \expandafter\ifx\csname#1\endcsname\relax\else1\fi
        =11 %
          \global\expandafter\let
          \csname#1\expandafter\endcsname
          \csname#1\endcsname
        \else
          \@PackageError{luatex}{%
            tex.enableprimitives failed for `#1'%
          }\@ehc
        \fi
      \endgroup
    \fi
  }%
\fi
\LuT@MakeLuatexPrimitive{attribute}
\LuT@MakeLuatexPrimitive{attributedef}
\LuT@MakeLuatexPrimitive{catcodetable}
\LuT@MakeLuatexPrimitive{initcatcodetable}
\LuT@MakeLuatexPrimitive{luaescapestring}
\LuT@MakeLuatexPrimitive{savecatcodetable}
\LuT@MakePrimitive{numexpr}
%    \end{macrocode}
%
% \subsection{Inherit support for \eTeX}
%
%    Package \xpackage{etex} is not compatible for \plainTeX.
%    But it could be present if a format is used that is based
%    on \xfile{etex.src}. Therefore we only load the package
%    in case of \LaTeX\ and tests its presence independently
%    of the format by looking for \cs{et@xins}.
%    \begin{macrocode}
\begingroup\expandafter\expandafter\expandafter\endgroup
\expandafter\ifx\csname RequirePackage\endcsname\relax
\else
  \RequirePackage{etex}[1998/03/26]%
\fi
%    \end{macrocode}
%
% \subsection{Adaption of \eTeX's register allocation}
%
%    \eTeX has increased the number of \TeX\ registers
%    from $2^8$ (256) to $2^{15}$ (32768) for a register class.
%    \LuaTeX\ extends the limit further to $2^16$ (65536).
%    The allocation scheme of package \xpackage{etex} is not
%    changed. But this can be subject for discussion.
%
%    If a register class hasn't registered any local registers
%    yet, then the limit can safely be pushed to 65536.
%    \begin{macrocode}
\begingroup\expandafter\expandafter\expandafter\endgroup
\expandafter\ifx\csname et@xins\endcsname\relax
  \@PackageWarningNoLine{luatex}{%
    Support for eTeX is not loaded (etex.src)%
  }%
\else
  \def\LuT@temp#1{%
    \ifnum\count27#1=32768 %
      \count27#1=65536 %
    \fi
  }%
  \LuT@temp0%
  \LuT@temp1%
  \LuT@temp2%
  \LuT@temp3%
  \LuT@temp4%
  \LuT@temp5%
  \LuT@temp6%
%    \end{macrocode}
%
% \eTeX\ uses an array for the first 256 registers and then a tree
% structure. \LuaTeX\ stores all registers of a class in one Lua table.
% There shouldn't be large performance differences.
% This allows starting immediately in the extended
% area, leaving room for insertions.
%    \begin{macrocode}
  \let\newcount\globcount
  \let\newdimen\globdimen
  \let\newskip\globskip
  \let\newbox\globbox
\fi
%    \end{macrocode}
%
% \subsection{\plainTeX\ compatibility}
%
%    \begin{macro}{\@empty}
%    \begin{macrocode}
\expandafter\ifx\csname @empty\endcsname\relax
  \def\@empty{}%
\fi
%    \end{macrocode}
%    \end{macro}
%    \begin{macro}{\@gobble}
%    \begin{macrocode}
\expandafter\ifx\csname @gobble\endcsname\relax
  \long\def\@gobble#1{}%
\fi
%    \end{macrocode}
%    \end{macro}
%    \begin{macro}{\@firstofone}
%    \begin{macrocode}
\expandafter\ifx\csname @firstofone\endcsname\relax
  \long\def\@firstofone#1{#1}%
\fi
%    \end{macrocode}
%    \end{macro}
%    \begin{macro}{\@firstoftwo}
%    \begin{macrocode}
\expandafter\ifx\csname @firstoftwo\endcsname\relax
  \long\def\@firstoftwo#1#2{#1}%
\fi
%    \end{macrocode}
%    \end{macro}
%    \begin{macro}{\@car}
%    \begin{macrocode}
\expandafter\ifx\csname @car\endcsname\relax
  \def\@car#1#2\@nil{#1}%
\fi
%    \end{macrocode}
%    \end{macro}
%    \begin{macro}{\@cdr}
%    \begin{macrocode}
\expandafter\ifx\csname @cdr\endcsname\relax
  \def\@cdr#1#2\@nil{#2}%
\fi
%    \end{macrocode}
%    \end{macro}
%    \begin{macro}{\@ifstar}
%    \begin{macrocode}
\expandafter\ifx\csname @ifstar\endcsname\relax
  \def\@ifstar#1{%
    \@ifnextchar*{\@firstoftwo{#1}}%
  }%
%    \end{macrocode}
%    \end{macro}
%    \begin{macro}{\@ifnextchar}
%    \begin{macrocode}
  \long\def\@ifnextchar#1#2#3{%
    \let\reserved@d=#1%
    \def\reserved@a{#2}%
    \def\reserved@b{#3}%
    \futurelet\@let@token\@ifnch
  }%
%    \end{macrocode}
%    \end{macro}
%    \begin{macro}{\@ifnch}
%    \begin{macrocode}
  \def\@ifnch{%
    \ifx\@let@token\@sptoken
      \let\reserved@c\@xifnch
    \else
      \ifx\@let@token\reserved@d
        \let\reserved@c\reserved@a
      \else
        \let\reserved@c\reserved@b
      \fi
    \fi
    \reserved@c
  }%
%    \end{macrocode}
%    \end{macro}
%    \begin{macro}{\@sptoken}
%    \begin{macrocode}
  \let\LuT@temp\:%
  \def\:{\let\@sptoken= }%
  \: % explicit space
%    \end{macrocode}
%    \end{macro}
%    \begin{macro}{\@xifnch}
%    \begin{macrocode}
  \def\:{\@xifnch}%
  \expandafter\def\: {%
    \futurelet\@let@token\@ifnch
  }%
  \let\:\LuT@temp
\fi
%    \end{macrocode}
%    \end{macro}
%
%    \begin{macro}{\@tempcnta}
%    \begin{macrocode}
\expandafter\ifx\csname @tempcnta\endcsname\relax
  \csname newcount\endcsname\@tempcnta
\fi
%    \end{macrocode}
%    \end{macro}
%
%    \begin{macro}{\@tempcntb}
%    \begin{macrocode}
\expandafter\ifx\csname @tempcntb\endcsname\relax
  \csname newcount\endcsname\@tempcntb
\fi
%    \end{macrocode}
%    \end{macro}
%
%    \begin{macro}{\LuT@newcommand}
%    \begin{macrocode}
\begingroup\expandafter\expandafter\expandafter\endgroup
\expandafter\ifx\csname newcommand\endcsname\relax
  \def\LuT@newcommand#1[#2]#3{%
    \ifx#1\@undefined
      \let#1\relax
    \else
      \ifx#1\relax
      \else
        \@PackageError{luatex}{%
          \string#1 is already defined.\MessageBreak
          Redefinition is skipped%
        }\@ehc
      \fi
    \fi
    \ifx#1\relax
      \ifcase#2 %
        \def#1{#3}%
      \or
        \def#1##1{#3}%
      \or
        \def#1##1##2{#3}%
      \or
        \def#1##1##2##3{#3}%
      \or
        \@INTERNAL@ERROR
      \fi
    \fi
  }%
\else
  \def\LuT@newcommand{\newcommand*}%
\fi
%    \end{macrocode}
%    \end{macro}
%
% \subsection{Attributes}
%
% \subsubsection{Allocation}
%
%    \begin{macro}{\LuT@AllocAttribute}
%    \begin{macrocode}
\newcount\LuT@AllocAttribute
\LuT@AllocAttribute=\m@ne
%    \end{macrocode}
%    \end{macro}
%    \begin{macro}{\newattribute}
%    \begin{macrocode}
\LuT@newcommand\newattribute[1]{%
  \ifnum\LuT@AllocAttribute<65535 %
    \global\advance\LuT@AllocAttribute\@ne
    \allocationnumber\LuT@AllocAttribute
    \global\luatexattributedef#1=\allocationnumber
    \unsetattribute{#1}%
    \wlog{\string#1=\string\attribute\the\allocationnumber}%
  \else
    \errmessage{No room for a new \string\attribute}%
  \fi
}
%    \end{macrocode}
%    \end{macro}
%
% \subsubsection{Interface}
%
%    \begin{macro}{\setattribute}
%    \begin{macrocode}
\LuT@newcommand\setattribute[2]{%
  #1=\numexpr#2\relax
}
%    \end{macrocode}
%    \end{macro}
%    \begin{macro}{\unsetattribute}
%    \begin{macrocode}
\ifnum\luatexversion<37
  \LuT@newcommand\LuT@UnsetAttributeValue[0]{}%
  \let\LuT@UnsetAttributeValue\m@ne
\else
  \LuT@newcommand\LuT@UnsetAttributeValue[0]{-2147483647 }%
\fi
\LuT@newcommand\unsetattribute[1]{%
  #1=\LuT@UnsetAttributeValue
}
%    \end{macrocode}
%    \end{macro}
%
% \subsection{Catcode tables}
%
% \subsubsection{Allocation}
%
%    \begin{macro}{\LuT@AllocCatcodeTable}
%    \begin{macrocode}
\newcount\LuT@AllocCatcodeTable
\LuT@AllocCatcodeTable=\m@ne
\newcount\CatcodeTableStack
\CatcodeTableStack=\z@
%    \end{macrocode}
%    \end{macro}
%    \begin{macro}{\newcatcodetable}
%    \begin{macrocode}
\LuT@newcommand\newcatcodetable[1]{%
  \ifnum\LuT@AllocCatcodeTable<1114110 % 0x10FFFF is maximal \chardef
    % or < 268435455 % 2^28 - 1
    \global\advance\LuT@AllocCatcodeTable by\tw@
    \allocationnumber=\LuT@AllocCatcodeTable
    \global\chardef#1=\allocationnumber
    \wlog{%
      \string#1=\string\catcodetable\the\allocationnumber
    }%
  \else
    \errmessage{No room for a new \string\catcodetable}%
  \fi
}%
%    \end{macrocode}
%    \end{macro}
%
%    \begin{macro}{\IncCatcodeTableStack}
%    \begin{macrocode}
\LuT@newcommand\IncCatcodeTableStack[0]{%
  \ifnum\CatcodeTableStack<268435454 %
    \global\advance\CatcodeTableStack by\tw@
  \else
    \@PackageError{luatex}{%
      Catcode table stack overflow%
    }\@ehd
  \fi
}
%    \end{macrocode}
%    \end{macro}
%
%    \begin{macro}{\DecCatcodeTableStack}
%    \begin{macrocode}
\LuT@newcommand\DecCatcodeTableStack[0]{%
  \ifnum\CatcodeTableStack>\z@
    \global\advance\CatcodeTableStack by-2 %
  \else
    \@PackageError{luatex}{%
      Catcode table stack is empty%
    }\@ehd
  \fi
}
%    \end{macrocode}
%    \end{macro}
%
% \subsubsection{\cs{SetCatcodeRange}}
%
%    \begin{macro}{\SetCatcodeRange}
%    \begin{macrocode}
\LuT@newcommand\SetCatcodeRange[3]{%
  \edef\LuT@temp{%
    \noexpand\@tempcnta=\the\@tempcnta
    \noexpand\@tempcntb=\the\@tempcntb
    \noexpand\count@=\the\count@
    \relax
  }%
  \@tempcnta=\numexpr#1\relax
  \@tempcntb=\numexpr#2\relax
  \count@=\numexpr#3\relax
  \loop
    \unless\ifnum\@tempcnta>\@tempcntb
    \catcode\@tempcnta=\count@
    \advance\@tempcnta by \@ne
  \repeat
  \LuT@temp
}
%    \end{macrocode}
%    \end{macro}
%
% \subsubsection{Predefined catcode tables}
%
%    \begin{macrocode}
\newcatcodetable\CatcodeTableIniTeX
\newcatcodetable\CatcodeTableString
\newcatcodetable\CatcodeTableOther
\newcatcodetable\CatcodeTableLaTeX
%    \end{macrocode}
%
%    \begin{macrocode}
\luatexinitcatcodetable\CatcodeTableIniTeX
\begingroup
  \def\@makeother#1{\catcode#1=12\relax}%
  \@firstofone{%
    \luatexcatcodetable\CatcodeTableIniTeX
    \begingroup
      \SetCatcodeRange{0}{8}{15}%
      \catcode9=10 % tab
      \catcode11=15 %
      \catcode12=13 % form feed
      \SetCatcodeRange{14}{31}{15}%
      \catcode35=6 % hash
      \catcode36=3 % dollar
      \catcode38=4 % ampersand
      \catcode94=7 % circumflex
      \catcode95=8 % underscore
      \catcode123=1 % brace left
      \catcode125=2 % brace right
      \catcode126=13 % tilde
      \catcode127=15 %
      \luatexsavecatcodetable\CatcodeTableLaTeX
    \endgroup
    \@makeother{0}% nul
    \@makeother{13}% carriage return
    \@makeother{37}% percent
    \@makeother{92}% backslash
    \@makeother{127}%
    \SetCatcodeRange{65}{90}{12}% A-Z
    \SetCatcodeRange{97}{122}{12}% a-z
    \luatexsavecatcodetable\CatcodeTableString
    \@makeother{32}% space
    \luatexsavecatcodetable\CatcodeTableOther
  \endgroup
}%
%    \end{macrocode}
%
% \subsubsection{Number stack}
%
%    \begin{macro}{\LuT@NumStackEmpty}
%    A special empty stack value because of \cs{@cdr}'s brace removal.
%    \begin{macrocode}
\def\LuT@NumStackEmpty{0}
%    \end{macrocode}
%    \end{macro}
%    \begin{macro}{\LuT@NumStack}
%    \begin{macrocode}
\let\LuT@NumStack\LuT@NumStackEmpty
%    \end{macrocode}
%    \end{macro}
%    \begin{macro}{\PushCatcodeTableNumStack}
%    \begin{macrocode}
\LuT@newcommand\PushCatcodeTableNumStack[0]{%
  \xdef\LuT@NumStack{%
    {\the\luatexcatcodetable}\LuT@NumStack
  }%
}
%    \end{macrocode}
%    \end{macro}
%    \begin{macro}{\PopCatcodeTableNumStack}
%    \begin{macrocode}
\LuT@newcommand\PopCatcodeTableNumStack[0]{%
  \ifx\LuT@NumStack\LuT@NumStackEmpty
    \@PackageWarning{luatex}{Empty catcode table number stack}%
    \luatexcatcodetable\z@
  \else
    \luatexcatcodetable=\expandafter\@car\LuT@NumStack\@nil\relax
    \xdef\LuT@NumStack{%
      \expandafter\@cdr\LuT@NumStack\@nil
    }%
  \fi
}
%    \end{macrocode}
%    \end{macro}
%
% \subsubsection{Catcode regime macros}
%
%    \begin{macro}{\BeginCatcodeRegime}
%    \begin{macrocode}
\LuT@newcommand\BeginCatcodeRegime[1]{%
  \PushCatcodeTableNumStack
  \luatexcatcodetable=\numexpr#1\relax
  \IncCatcodeTableStack
  \luatexsavecatcodetable\CatcodeTableStack
  \luatexcatcodetable\CatcodeTableStack
}
%    \end{macrocode}
%    \end{macro}
%
%    \begin{macro}{\EndCatcodeRegime}
%    \begin{macrocode}
\LuT@newcommand\EndCatcodeRegime[0]{%
  \DecCatcodeTableStack
  \PopCatcodeTableNumStack
}
%    \end{macrocode}
%    \end{macro}
%
% \subsection{Lua module loader}
%
%    \begin{macrocode}
\begingroup\expandafter\expandafter\expandafter\endgroup
\expandafter\ifx\csname RequirePackage\endcsname\relax
  \input luatex-loader.sty\relax
\else
  \RequirePackage{luatex-loader}[2010/03/09]%
\fi
%    \end{macrocode}
%    \begin{macrocode}
\LuT@AtEnd%
%</package>
%    \end{macrocode}
%
%    \begin{macrocode}
%<*loader>
%    \end{macrocode}
%
%    Reload check, especially if the package is not used with \LaTeX.
%    \begin{macrocode}
\begingroup\catcode61\catcode48\catcode32=10\relax%
  \catcode13=5 % ^^M
  \endlinechar=13 %
  \catcode35=6 % #
  \catcode39=12 % '
  \catcode44=12 % ,
  \catcode45=12 % -
  \catcode46=12 % .
  \catcode58=12 % :
  \catcode64=11 % @
  \catcode123=1 % {
  \catcode125=2 % }
  \expandafter\let\expandafter\x\csname ver@luatex-loader.sty\endcsname
  \ifx\x\relax % plain-TeX, first loading
  \else
    \def\empty{}%
    \ifx\x\empty % LaTeX, first loading,
      % variable is initialized, but \ProvidesPackage not yet seen
    \else
      \expandafter\ifx\csname PackageInfo\endcsname\relax
        \def\x#1#2{%
          \immediate\write-1{Package #1 Info: #2.}%
        }%
      \else
        \def\x#1#2{\PackageInfo{#1}{#2, stopped}}%
      \fi
      \x{luatex-loader}{The package is already loaded}%
      \aftergroup\endinput
    \fi
  \fi
\endgroup%
%    \end{macrocode}
%    Package identification:
%    \begin{macrocode}
\begingroup\catcode61\catcode48\catcode32=10\relax%
  \catcode13=5 % ^^M
  \endlinechar=13 %
  \catcode35=6 % #
  \catcode39=12 % '
  \catcode40=12 % (
  \catcode41=12 % )
  \catcode44=12 % ,
  \catcode45=12 % -
  \catcode46=12 % .
  \catcode47=12 % /
  \catcode58=12 % :
  \catcode64=11 % @
  \catcode91=12 % [
  \catcode93=12 % ]
  \catcode123=1 % {
  \catcode125=2 % }
  \expandafter\ifx\csname ProvidesPackage\endcsname\relax
    \def\x#1#2#3[#4]{\endgroup
      \immediate\write-1{Package: #3 #4}%
      \xdef#1{#4}%
    }%
  \else
    \def\x#1#2[#3]{\endgroup
      #2[{#3}]%
      \ifx#1\@undefined
        \xdef#1{#3}%
      \fi
      \ifx#1\relax
        \xdef#1{#3}%
      \fi
    }%
  \fi
\expandafter\x\csname ver@luatex-loader.sty\endcsname
\ProvidesPackage{luatex-loader}%
  [2010/03/09 v0.4 Lua module loader (HO)]%
%    \end{macrocode}
%    \begin{macrocode}
\begingroup\catcode61\catcode48\catcode32=10\relax%
  \catcode13=5\endlinechar=13\relax%
  \catcode10=12 % ^^J
  \catcode34=12 % "
  \catcode39=12 % '
  \catcode40=12 % (
  \catcode41=12 % )
  \catcode44=12 % ,
  \catcode46=12 % .
  \catcode60=12 % <
  \catcode61=12 % =
  \catcode95=12 % _ (other!)
  \catcode96=12 % `
  \catcode123=1 % {
  \catcode125=2 % }
  \endlinechar=10 %
  \ifnum\luatexversion<36 %
    \directlua0%
  \else %
    \expandafter\directlua %
  \fi %
  {%
     do
      local script = "oberdiek.luatex.lua"
      local file = kpse.find_file(script, "texmfscripts")
      if file then
        texio.write_nl("(" .. file .. ")")
        dofile(file)
      else
        error("File `" .. script .. "' not found")
      end
    end
  }%
\endgroup\endinput%
%    \end{macrocode}
%
%    \begin{macrocode}
%</loader>
%    \end{macrocode}
%
% \subsection{Lua script}
%
%    Currently \LuaTeX\ does not use KPSE when searching for module
%    files. The following Lua script implements a workaround.
%    It extends |package.loader| by another search method.
%    Modules are found by the module name with extension \xext{lua}
%    similar to
%    \begin{quote}
%      \ttfamily |kpsewhich --format=texmfscripts| \meta{module}|.lua|
%    \end{quote}
%    Unhappily \textsf{kpsewhich} does not support directory components
%    in the file name. Therefore a module |a.b.c| cannot be installed
%    as \xfile{a/b/c.lua}. The script must be named \xfile{a.b.c.lua}.
%    \begin{macrocode}
%<*lua>
%    \end{macrocode}
%    \begin{macrocode}
module("oberdiek.luatex", package.seeall)
%    \end{macrocode}
%    \begin{macrocode}
function kpse_module_loader(module)
  local script = module .. ".lua"
  local file = kpse.find_file(script, "texmfscripts")
  if file then
    local loader, error = loadfile(file)
    if loader then
      texio.write_nl("(" .. file .. ")")
      return loader
    end
    return "\n\t[oberdiek.luatex.kpse_module_loader] Loading error:\n\t"
           .. error
  end
  return "\n\t[oberdiek.luatex.kpse_module_loader] Search failed"
end
table.insert(package.loaders, kpse_module_loader)
%    \end{macrocode}
%    \begin{macrocode}
%</lua>
%    \end{macrocode}
%
% \section{Test}
%
%    \begin{macrocode}
%<*test2>
\documentclass{article}
\def\LoadCommand{%
  \RequirePackage{luatex}[2010/03/09]%
}
%</test2>
%    \end{macrocode}
%    \begin{macrocode}
%<*test3>
\documentclass{article}
\def\LoadCommand{%
  \RequirePackage{luatex-loader}[2010/03/09]%
}
%</test3>
%    \end{macrocode}
% \subsection{Catcode checks for loading}
%
%    \begin{macrocode}
%<*test1>
%    \end{macrocode}
%    \begin{macrocode}
\catcode`\{=1 %
\catcode`\}=2 %
\catcode`\#=6 %
\catcode`\@=11 %
\expandafter\ifx\csname count@\endcsname\relax
  \countdef\count@=255 %
\fi
\expandafter\ifx\csname @gobble\endcsname\relax
  \long\def\@gobble#1{}%
\fi
\expandafter\ifx\csname @firstofone\endcsname\relax
  \long\def\@firstofone#1{#1}%
\fi
\expandafter\ifx\csname loop\endcsname\relax
  \expandafter\@firstofone
\else
  \expandafter\@gobble
\fi
{%
  \def\loop#1\repeat{%
    \def\body{#1}%
    \iterate
  }%
  \def\iterate{%
    \body
      \let\next\iterate
    \else
      \let\next\relax
    \fi
    \next
  }%
  \let\repeat=\fi
}%
\def\RestoreCatcodes{}
\count@=0 %
\loop
  \edef\RestoreCatcodes{%
    \RestoreCatcodes
    \catcode\the\count@=\the\catcode\count@\relax
  }%
\ifnum\count@<255 %
  \advance\count@ 1 %
\repeat

\def\RangeCatcodeInvalid#1#2{%
  \count@=#1\relax
  \loop
    \catcode\count@=15 %
  \ifnum\count@<#2\relax
    \advance\count@ 1 %
  \repeat
}
\def\RangeCatcodeCheck#1#2#3{%
  \count@=#1\relax
  \loop
    \ifnum#3=\catcode\count@
    \else
      \errmessage{%
        Character \the\count@\space
        with wrong catcode \the\catcode\count@\space
        instead of \number#3%
      }%
    \fi
  \ifnum\count@<#2\relax
    \advance\count@ 1 %
  \repeat
}
\def\space{ }
\expandafter\ifx\csname LoadCommand\endcsname\relax
  \def\LoadCommand{\input luatex.sty\relax}%
\fi
\def\Test{%
  \RangeCatcodeInvalid{0}{47}%
  \RangeCatcodeInvalid{58}{64}%
  \RangeCatcodeInvalid{91}{96}%
  \RangeCatcodeInvalid{123}{255}%
  \catcode`\@=12 %
  \catcode`\\=0 %
  \catcode`\%=14 %
  \LoadCommand
  \RangeCatcodeCheck{0}{36}{15}%
  \RangeCatcodeCheck{37}{37}{14}%
  \RangeCatcodeCheck{38}{47}{15}%
  \RangeCatcodeCheck{48}{57}{12}%
  \RangeCatcodeCheck{58}{63}{15}%
  \RangeCatcodeCheck{64}{64}{12}%
  \RangeCatcodeCheck{65}{90}{11}%
  \RangeCatcodeCheck{91}{91}{15}%
  \RangeCatcodeCheck{92}{92}{0}%
  \RangeCatcodeCheck{93}{96}{15}%
  \RangeCatcodeCheck{97}{122}{11}%
  \RangeCatcodeCheck{123}{255}{15}%
  \RestoreCatcodes
}
\Test
\csname @@end\endcsname
\end
%    \end{macrocode}
%    \begin{macrocode}
%</test1>
%    \end{macrocode}
%
% \subsection{Catcode tables}
%
% \subsubsection{Predefined catcode tables}
%
%    \begin{macrocode}
%<*test4>
\NeedsTeXFormat{LaTeX2e}
%    \end{macrocode}
%    Remember \LaTeX's initial catcodes in count registers
%    starting at \cs{TestLaTeX}.
%    \begin{macrocode}
\count0=0 %
\chardef\TestLaTeX=1000 %
\chardef\TestMax=300 %
\loop
  \count\numexpr\TestLaTeX+\count0\relax=\catcode\count0 %
\ifnum\count0<\TestMax
  \advance\count0 by 1 %
\repeat
\documentclass{minimal}
\usepackage{luatex}[2010/03/09]
\usepackage{qstest}
\IncludeTests{*}
\LogTests{log}{*}{*}
\makeatletter
\def\Check#1{%
  \Expect*{\the\count@=\the\catcode\count@}%
         *{\the\count@=#1}%
}
\newcount\scratch
\def\Test#1#2{%
  \begin{qstest}{CatcodeTable#1}{CatcodeTable#1}%
    \luatexcatcodetable\csname CatcodeTable#1\endcsname
    \count@=\z@
    \loop
      \scratch=#2\relax
      \Expect*{\the\count@=\the\catcode\count@}%
             *{\the\count@=\the\scratch}%
    \ifnum\count@<\TestMax
      \advance\count@\@ne
    \repeat
  \end{qstest}%
}
\begingroup
  % luatex-unicode-letters.tex makes some slots to letters
  \def\TestMax{169}%
  \Test{LaTeX}{\the\count\numexpr\TestLaTeX+\count@}%
\endgroup
\Test{String}{\ifnum\count@=32 10\else 12\fi}
\Test{Other}{12}
\luatexinitcatcodetable99 %
\Test{IniTeX}{%
  0\relax
  \begingroup
    \luatexcatcodetable99 %
    \global\scratch=\the\catcode\count@
  \endgroup
}
%    \end{macrocode}
%
% \subsubsection{Catcode table number stack}
%
%    \begin{macrocode}
\begin{qstest}{CatcodeTableNumStack}{CatcodeTableNumStack}
  \def\TestStack#1{%
    \Expect*{\LuT@NumStack}{#1}%
  }%
  \TestStack{0}%
  \PushCatcodeTableNumStack
  \TestStack{{0}0}%
  \@firstofone{%
    \begingroup
      \luatexinitcatcodetable12 %
      \luatexcatcodetable12 %
      \PushCatcodeTableNumStack
      \TestStack{{12}{0}0}%
      \PopCatcodeTableNumStack
      \TestStack{{0}0}%
      \PopCatcodeTableNumStack
      \TestStack{0}%
      \def\TestWarning{Missing empty stack warning}%
      \def\@PackageWarning#1#2{\def\TestWarning{empty stack}}%
      \PopCatcodeTableNumStack
      \TestStack{0}%
      \Expect*{\TestWarning}{empty stack}%
    \endgroup
  }%
\end{qstest}
%    \end{macrocode}
%
% \subsubsection{Catcode table stack}
%
%    \begin{macrocode}
\begin{qstest}{CatcodeTableStack}{CatcodeTableStack}
  \def\TestStack#1{%
    \Expect*{\the\CatcodeTableStack}{#1}%
  }%
  \TestStack{0}%
  \IncCatcodeTableStack
  \TestStack{2}%
  \IncCatcodeTableStack
  \TestStack{4}%
  \begingroup
    \IncCatcodeTableStack
    \TestStack{6}%
  \endgroup
  \TestStack{6}%
  \begingroup
    \DecCatcodeTableStack
    \TestStack{4}%
  \endgroup
  \TestStack{4}%
  \DecCatcodeTableStack
  \TestStack{2}%
  \DecCatcodeTableStack
  \TestStack{0}%
  \begingroup
    \def\TestError{Missing error}%
    \def\@PackageError#1#2#3{%
      \def\TestError{Empty stack}%
    }%
    \DecCatcodeTableStack
    \TestStack{0}%
    \Expect*{\TestError}{Empty stack}%
  \endgroup
\end{qstest}
%    \end{macrocode}
%
% \subsubsection{Catcode regime macros}
%
%    \begin{macrocode}
\begin{qstest}{CatcodeRegime}{CatcodeRegime}
  \def\TestStacks#1#2#3{%
    \Expect*{\the\luatexcatcodetable}{#1}%
    \Expect*{\the\CatcodeTableStack}{#2}%
    \Expect*{\LuT@NumStack}{#3}%
  }%
  \TestStacks{0}{0}{0}%
  \catcode`\|=7 %
  \BeginCatcodeRegime\CatcodeTableLaTeX
    \TestStacks{2}{2}{{0}0}%
    \Expect*{\the\catcode`\|}{12}%
  \EndCatcodeRegime
  \TestStacks{0}{0}{0}%
  \Expect*{\the\catcode`\|}{7}%
\end{qstest}
%    \end{macrocode}
%
% \subsection{Attribute allocation}
%
%    \begin{macrocode}
\begin{qstest}{Attributes}{Attributes}
  \newattribute\TestAttr
  \Expect*{\meaning\TestAttr}%
         *{\string\attribute\number\allocationnumber}%
  \Expect*{\the\allocationnumber}{0}%
  \begingroup
    \newattribute\TestAttr
    \Expect*{\the\allocationnumber}{1}%
  \endgroup
  \Expect*{\the\allocationnumber}{0}%
  \Expect*{\meaning\TestAttr}*{\string\attribute1}%
  \Expect*{\the\TestAttr}*{\number\LuT@UnsetAttributeValue}%
  \def\Test#1{%
    \setattribute\TestAttr{#1}%
    \Expect*{\the\TestAttr}{#1}%
  }%
  \Test{0}%
  \Test{1}%
  \Test{-1}%
  \Test{123}%
  \unsetattribute\TestAttr
  \Expect*{\the\TestAttr}*{\number\LuT@UnsetAttributeValue}%
  \begingroup
    \Expect*{\the\TestAttr}*{\number\LuT@UnsetAttributeValue}%
    \Test{1234}%
  \endgroup
  \Expect*{\the\TestAttr}*{\number\LuT@UnsetAttributeValue}%
\end{qstest}
%    \end{macrocode}
%
%    \begin{macrocode}
\@@end
%</test4>
%    \end{macrocode}
%
% \subsection{Short test for \plainTeX}
%
%    \begin{macrocode}
%<*test5>
\input luatex.sty\relax
\newattribute\TestAttr
\setattribute\TestAttr{10}
\unsetattribute\TestAttr
\newcatcodetable\TestCTa
\begingroup
  \SetCatcodeRange{`A}{`Z}{12}%
\endgroup
\BeginCatcodeRegime\CatcodeTableLaTeX
\EndCatcodeRegime
\end
%</test5>
%    \end{macrocode}
%
% \section{Installation}
%
% \subsection{Download}
%
% \paragraph{Package.} This package is available on
% CTAN\footnote{\url{ftp://ftp.ctan.org/tex-archive/}}:
% \begin{description}
% \item[\CTAN{macros/latex/contrib/oberdiek/luatex.dtx}] The source file.
% \item[\CTAN{macros/latex/contrib/oberdiek/luatex.pdf}] Documentation.
% \end{description}
%
%
% \paragraph{Bundle.} All the packages of the bundle `oberdiek'
% are also available in a TDS compliant ZIP archive. There
% the packages are already unpacked and the documentation files
% are generated. The files and directories obey the TDS standard.
% \begin{description}
% \item[\CTAN{install/macros/latex/contrib/oberdiek.tds.zip}]
% \end{description}
% \emph{TDS} refers to the standard ``A Directory Structure
% for \TeX\ Files'' (\CTAN{tds/tds.pdf}). Directories
% with \xfile{texmf} in their name are usually organized this way.
%
% \subsection{Bundle installation}
%
% \paragraph{Unpacking.} Unpack the \xfile{oberdiek.tds.zip} in the
% TDS tree (also known as \xfile{texmf} tree) of your choice.
% Example (linux):
% \begin{quote}
%   |unzip oberdiek.tds.zip -d ~/texmf|
% \end{quote}
%
% \paragraph{Script installation.}
% Check the directory \xfile{TDS:scripts/oberdiek/} for
% scripts that need further installation steps.
% Package \xpackage{attachfile2} comes with the Perl script
% \xfile{pdfatfi.pl} that should be installed in such a way
% that it can be called as \texttt{pdfatfi}.
% Example (linux):
% \begin{quote}
%   |chmod +x scripts/oberdiek/pdfatfi.pl|\\
%   |cp scripts/oberdiek/pdfatfi.pl /usr/local/bin/|
% \end{quote}
%
% \subsection{Package installation}
%
% \paragraph{Unpacking.} The \xfile{.dtx} file is a self-extracting
% \docstrip\ archive. The files are extracted by running the
% \xfile{.dtx} through \plainTeX:
% \begin{quote}
%   \verb|tex luatex.dtx|
% \end{quote}
%
% \paragraph{TDS.} Now the different files must be moved into
% the different directories in your installation TDS tree
% (also known as \xfile{texmf} tree):
% \begin{quote}
% \def\t{^^A
% \begin{tabular}{@{}>{\ttfamily}l@{ $\rightarrow$ }>{\ttfamily}l@{}}
%   luatex.sty & tex/generic/oberdiek/luatex.sty\\
%   luatex-loader.sty & tex/generic/oberdiek/luatex-loader.sty\\
%   oberdiek.luatex.lua & scripts/oberdiek/oberdiek.luatex.lua\\
%   luatex.pdf & doc/latex/oberdiek/luatex.pdf\\
%   test/luatex-test1.tex & doc/latex/oberdiek/test/luatex-test1.tex\\
%   test/luatex-test2.tex & doc/latex/oberdiek/test/luatex-test2.tex\\
%   test/luatex-test3.tex & doc/latex/oberdiek/test/luatex-test3.tex\\
%   test/luatex-test4.tex & doc/latex/oberdiek/test/luatex-test4.tex\\
%   test/luatex-test5.tex & doc/latex/oberdiek/test/luatex-test5.tex\\
%   luatex.dtx & source/latex/oberdiek/luatex.dtx\\
% \end{tabular}^^A
% }^^A
% \sbox0{\t}^^A
% \ifdim\wd0>\linewidth
%   \begingroup
%     \advance\linewidth by\leftmargin
%     \advance\linewidth by\rightmargin
%   \edef\x{\endgroup
%     \def\noexpand\lw{\the\linewidth}^^A
%   }\x
%   \def\lwbox{^^A
%     \leavevmode
%     \hbox to \linewidth{^^A
%       \kern-\leftmargin\relax
%       \hss
%       \usebox0
%       \hss
%       \kern-\rightmargin\relax
%     }^^A
%   }^^A
%   \ifdim\wd0>\lw
%     \sbox0{\small\t}^^A
%     \ifdim\wd0>\linewidth
%       \ifdim\wd0>\lw
%         \sbox0{\footnotesize\t}^^A
%         \ifdim\wd0>\linewidth
%           \ifdim\wd0>\lw
%             \sbox0{\scriptsize\t}^^A
%             \ifdim\wd0>\linewidth
%               \ifdim\wd0>\lw
%                 \sbox0{\tiny\t}^^A
%                 \ifdim\wd0>\linewidth
%                   \lwbox
%                 \else
%                   \usebox0
%                 \fi
%               \else
%                 \lwbox
%               \fi
%             \else
%               \usebox0
%             \fi
%           \else
%             \lwbox
%           \fi
%         \else
%           \usebox0
%         \fi
%       \else
%         \lwbox
%       \fi
%     \else
%       \usebox0
%     \fi
%   \else
%     \lwbox
%   \fi
% \else
%   \usebox0
% \fi
% \end{quote}
% If you have a \xfile{docstrip.cfg} that configures and enables \docstrip's
% TDS installing feature, then some files can already be in the right
% place, see the documentation of \docstrip.
%
% \subsection{Refresh file name databases}
%
% If your \TeX~distribution
% (\teTeX, \mikTeX, \dots) relies on file name databases, you must refresh
% these. For example, \teTeX\ users run \verb|texhash| or
% \verb|mktexlsr|.
%
% \subsection{Some details for the interested}
%
% \paragraph{Attached source.}
%
% The PDF documentation on CTAN also includes the
% \xfile{.dtx} source file. It can be extracted by
% AcrobatReader 6 or higher. Another option is \textsf{pdftk},
% e.g. unpack the file into the current directory:
% \begin{quote}
%   \verb|pdftk luatex.pdf unpack_files output .|
% \end{quote}
%
% \paragraph{Unpacking with \LaTeX.}
% The \xfile{.dtx} chooses its action depending on the format:
% \begin{description}
% \item[\plainTeX:] Run \docstrip\ and extract the files.
% \item[\LaTeX:] Generate the documentation.
% \end{description}
% If you insist on using \LaTeX\ for \docstrip\ (really,
% \docstrip\ does not need \LaTeX), then inform the autodetect routine
% about your intention:
% \begin{quote}
%   \verb|latex \let\install=y% \iffalse meta-comment
%
% File: luatex.dtx
% Version: 2010/03/09 v0.4
% Info: LuaTeX basic definition package
%
% Copyright (C) 2007, 2009, 2010 by
%    Heiko Oberdiek <heiko.oberdiek at googlemail.com>
%
% This work may be distributed and/or modified under the
% conditions of the LaTeX Project Public License, either
% version 1.3c of this license or (at your option) any later
% version. This version of this license is in
%    http://www.latex-project.org/lppl/lppl-1-3c.txt
% and the latest version of this license is in
%    http://www.latex-project.org/lppl.txt
% and version 1.3 or later is part of all distributions of
% LaTeX version 2005/12/01 or later.
%
% This work has the LPPL maintenance status "maintained".
%
% This Current Maintainer of this work is Heiko Oberdiek.
%
% The Base Interpreter refers to any `TeX-Format',
% because some files are installed in TDS:tex/generic//.
%
% This work consists of the main source file luatex.dtx
% and the derived files
%    luatex.sty, luatex.pdf, luatex.ins, luatex.drv, luatex-loader.sty,
%    luatex-test1.tex, luatex-test2.tex, luatex-test3.tex,
%    luatex-test4.tex, luatex-test5.tex, oberdiek.luatex.lua.
%
% Distribution:
%    CTAN:macros/latex/contrib/oberdiek/luatex.dtx
%    CTAN:macros/latex/contrib/oberdiek/luatex.pdf
%
% Unpacking:
%    (a) If luatex.ins is present:
%           tex luatex.ins
%    (b) Without luatex.ins:
%           tex luatex.dtx
%    (c) If you insist on using LaTeX
%           latex \let\install=y% \iffalse meta-comment
%
% File: luatex.dtx
% Version: 2010/03/09 v0.4
% Info: LuaTeX basic definition package
%
% Copyright (C) 2007, 2009, 2010 by
%    Heiko Oberdiek <heiko.oberdiek at googlemail.com>
%
% This work may be distributed and/or modified under the
% conditions of the LaTeX Project Public License, either
% version 1.3c of this license or (at your option) any later
% version. This version of this license is in
%    http://www.latex-project.org/lppl/lppl-1-3c.txt
% and the latest version of this license is in
%    http://www.latex-project.org/lppl.txt
% and version 1.3 or later is part of all distributions of
% LaTeX version 2005/12/01 or later.
%
% This work has the LPPL maintenance status "maintained".
%
% This Current Maintainer of this work is Heiko Oberdiek.
%
% The Base Interpreter refers to any `TeX-Format',
% because some files are installed in TDS:tex/generic//.
%
% This work consists of the main source file luatex.dtx
% and the derived files
%    luatex.sty, luatex.pdf, luatex.ins, luatex.drv, luatex-loader.sty,
%    luatex-test1.tex, luatex-test2.tex, luatex-test3.tex,
%    luatex-test4.tex, luatex-test5.tex, oberdiek.luatex.lua.
%
% Distribution:
%    CTAN:macros/latex/contrib/oberdiek/luatex.dtx
%    CTAN:macros/latex/contrib/oberdiek/luatex.pdf
%
% Unpacking:
%    (a) If luatex.ins is present:
%           tex luatex.ins
%    (b) Without luatex.ins:
%           tex luatex.dtx
%    (c) If you insist on using LaTeX
%           latex \let\install=y\input{luatex.dtx}
%        (quote the arguments according to the demands of your shell)
%
% Documentation:
%    (a) If luatex.drv is present:
%           latex luatex.drv
%    (b) Without luatex.drv:
%           latex luatex.dtx; ...
%    The class ltxdoc loads the configuration file ltxdoc.cfg
%    if available. Here you can specify further options, e.g.
%    use A4 as paper format:
%       \PassOptionsToClass{a4paper}{article}
%
%    Programm calls to get the documentation (example):
%       pdflatex luatex.dtx
%       makeindex -s gind.ist luatex.idx
%       pdflatex luatex.dtx
%       makeindex -s gind.ist luatex.idx
%       pdflatex luatex.dtx
%
% Installation:
%    TDS:tex/generic/oberdiek/luatex.sty
%    TDS:tex/generic/oberdiek/luatex-loader.sty
%    TDS:scripts/oberdiek/oberdiek.luatex.lua
%    TDS:doc/latex/oberdiek/luatex.pdf
%    TDS:doc/latex/oberdiek/test/luatex-test1.tex
%    TDS:doc/latex/oberdiek/test/luatex-test2.tex
%    TDS:doc/latex/oberdiek/test/luatex-test3.tex
%    TDS:doc/latex/oberdiek/test/luatex-test4.tex
%    TDS:doc/latex/oberdiek/test/luatex-test5.tex
%    TDS:source/latex/oberdiek/luatex.dtx
%
%<*ignore>
\begingroup
  \catcode123=1 %
  \catcode125=2 %
  \def\x{LaTeX2e}%
\expandafter\endgroup
\ifcase 0\ifx\install y1\fi\expandafter
         \ifx\csname processbatchFile\endcsname\relax\else1\fi
         \ifx\fmtname\x\else 1\fi\relax
\else\csname fi\endcsname
%</ignore>
%<*install>
\input docstrip.tex
\Msg{************************************************************************}
\Msg{* Installation}
\Msg{* Package: luatex 2010/03/09 v0.4 LuaTeX basic definition package (HO)}
\Msg{************************************************************************}

\keepsilent
\askforoverwritefalse

\let\MetaPrefix\relax
\preamble

This is a generated file.

Project: luatex
Version: 2010/03/09 v0.4

Copyright (C) 2007, 2009, 2010 by
   Heiko Oberdiek <heiko.oberdiek at googlemail.com>

This work may be distributed and/or modified under the
conditions of the LaTeX Project Public License, either
version 1.3c of this license or (at your option) any later
version. This version of this license is in
   http://www.latex-project.org/lppl/lppl-1-3c.txt
and the latest version of this license is in
   http://www.latex-project.org/lppl.txt
and version 1.3 or later is part of all distributions of
LaTeX version 2005/12/01 or later.

This work has the LPPL maintenance status "maintained".

This Current Maintainer of this work is Heiko Oberdiek.

The Base Interpreter refers to any `TeX-Format',
because some files are installed in TDS:tex/generic//.

This work consists of the main source file luatex.dtx
and the derived files
   luatex.sty, luatex.pdf, luatex.ins, luatex.drv, luatex-loader.sty,
   luatex-test1.tex, luatex-test2.tex, luatex-test3.tex,
   luatex-test4.tex, luatex-test5.tex, oberdiek.luatex.lua.

\endpreamble
\let\MetaPrefix\DoubleperCent

\generate{%
  \file{luatex.ins}{\from{luatex.dtx}{install}}%
  \file{luatex.drv}{\from{luatex.dtx}{driver}}%
  \usedir{tex/generic/oberdiek}%
  \file{luatex.sty}{\from{luatex.dtx}{package}}%
  \file{luatex-loader.sty}{\from{luatex.dtx}{loader}}%
  \usedir{doc/latex/oberdiek/test}%
  \file{luatex-test1.tex}{\from{luatex.dtx}{test1}}%
  \file{luatex-test2.tex}{\from{luatex.dtx}{test1,test2}}%
  \file{luatex-test3.tex}{\from{luatex.dtx}{test1,test3}}%
  \file{luatex-test4.tex}{\from{luatex.dtx}{test4}}%
  \file{luatex-test5.tex}{\from{luatex.dtx}{test5}}%
  \nopreamble
  \nopostamble
  \usedir{source/latex/oberdiek/catalogue}%
  \file{luatex.xml}{\from{luatex.dtx}{catalogue}}%
}
\def\MetaPrefix{-- }
\def\defaultpostamble{%
  \MetaPrefix^^J%
  \MetaPrefix\space End of File `\outFileName'.%
}
\def\currentpostamble{\defaultpostamble}%
\generate{%
  \usedir{scripts/oberdiek}%
  \file{oberdiek.luatex.lua}{\from{luatex.dtx}{lua}}%
}

\catcode32=13\relax% active space
\let =\space%
\Msg{************************************************************************}
\Msg{*}
\Msg{* To finish the installation you have to move the following}
\Msg{* files into a directory searched by TeX:}
\Msg{*}
\Msg{*     luatex.sty, luatex-loader.sty}
\Msg{*}
\Msg{* And install the following script file:}
\Msg{*}
\Msg{*     oberdiek.luatex.lua}
\Msg{*}
\Msg{* To produce the documentation run the file `luatex.drv'}
\Msg{* through LaTeX.}
\Msg{*}
\Msg{* Happy TeXing!}
\Msg{*}
\Msg{************************************************************************}

\endbatchfile
%</install>
%<*ignore>
\fi
%</ignore>
%<*driver>
\NeedsTeXFormat{LaTeX2e}
\ProvidesFile{luatex.drv}%
  [2010/03/09 v0.4 LuaTeX basic definition package (HO)]%
\documentclass{ltxdoc}
\usepackage{holtxdoc}[2011/11/22]
\usepackage{paralist}
\begin{document}
  \DocInput{luatex.dtx}%
\end{document}
%</driver>
% \fi
%
% \CheckSum{1429}
%
% \CharacterTable
%  {Upper-case    \A\B\C\D\E\F\G\H\I\J\K\L\M\N\O\P\Q\R\S\T\U\V\W\X\Y\Z
%   Lower-case    \a\b\c\d\e\f\g\h\i\j\k\l\m\n\o\p\q\r\s\t\u\v\w\x\y\z
%   Digits        \0\1\2\3\4\5\6\7\8\9
%   Exclamation   \!     Double quote  \"     Hash (number) \#
%   Dollar        \$     Percent       \%     Ampersand     \&
%   Acute accent  \'     Left paren    \(     Right paren   \)
%   Asterisk      \*     Plus          \+     Comma         \,
%   Minus         \-     Point         \.     Solidus       \/
%   Colon         \:     Semicolon     \;     Less than     \<
%   Equals        \=     Greater than  \>     Question mark \?
%   Commercial at \@     Left bracket  \[     Backslash     \\
%   Right bracket \]     Circumflex    \^     Underscore    \_
%   Grave accent  \`     Left brace    \{     Vertical bar  \|
%   Right brace   \}     Tilde         \~}
%
% \GetFileInfo{luatex.drv}
%
% \title{The \xpackage{luatex} package}
% \date{2010/03/09 v0.4}
% \author{Heiko Oberdiek\\\xemail{heiko.oberdiek at googlemail.com}}
%
% \maketitle
%
% \begin{abstract}
% This package manages the new and extended features and resources
% that \LuaTeX\ provides. Examples are attributes and catcode
% tables.
% \end{abstract}
%
% \tableofcontents
%
% \section{Documentation}
%
% \subsection{Introduction}
%
% \TeX\ provides global resources such as registers.
% But it does not provide an interface for managing these resources.
% For example, two packages want to use a counter register.
% If they take the same register
% number, then the use of both packages will conflict and they cannot
% be used together.
% Therefore formats such as \plainTeX\ or \LaTeX\ implement
% an allocation scheme for registers. A package reserves with
% \cs{newcount} an unused register number for its own exclusive use.
%
% Nowadays \TeX\ is not alone anymore: \eTeX, \pdfTeX\ and other
% compilers for \TeX\ are developed that extend and add new
% features and resources.
%
% Now \LuaTeX\ has reached beta state. It inherits most of
% \pdfTeX's features including \eTeX. Also it implements
% new concepts such as attributes or catcode tables.
%
% \subsubsection{\LaTeX}
%
% \LaTeXe\ is frozen and therefore refuses to even notice
% the new \TeX\ variants. Not even the old \eTeX\ is supported
% by its kernel. At least there is a third party package \xpackage{etex}
% that manages the new \eTeX\ resources.
%
% This package tries to do the same for \LuaTeX\ and starts to
% support at least a few of the new features.
%
% \subsubsection{\plainTeX}
%
% \LaTeX\ has inherited its resource handling from \plainTeX.
% The interface is basically the same: \cs{newcount}, \dots
% Therefore this package tries to follow this tradition by
% providing compatibility to \plainTeX. It can be loaded
% with \plainTeX\ and defines at least some of the
% features that this packages provides for \LaTeX.
%
% \subsection{Register allocation}
%
% \subsubsection{Register with 16 bit}
%
% Because \LuaTeX\ is a super set of \eTeX\ regarding registers,
% the register allocation scheme should not conflict with
% package \xpackage{etex}. Therefore this package is loaded
% to inherit its allocation scheme. The only change is currently
% that the limit is increased to 65536 registers for the following
% register classes:
% \begin{compactitem}
% \item count
% \item dimen
% \item skip
% \item muskip
% \item marks
% \item toks
% \item box
% \end{compactitem}
% This affects the number of global and local registers.
% Because it is done in a package and not in the kernel,
% it is possible that someone loads package \xpackage{etex}
% before uses the local allocation variants. This will prevent
% the extension for this register class. If more registers are
% needed, just load package \xpackage{luatex} earlier.
%
% \subsubsection{Insertions}
%
% Insertions need four registers \cs{count}, \cs{dimen}, \cs{skip},
% and \cs{box} with the same number. Usually they are allocated
% downwards from 254, 253, \dots Also \cs{newcount}, \cs{newdimen}, \dots
% fill up these register numbers from below before switching to
% higher register numbers by package \xpackage{etex}.
% When this occurs, no insertions can be allocated anymore.
%
% Therefore \cs{newcount}, \cs{newdimen}, \cs{newskip}, and
% \cs{newbox} are replaced by their global variants (\cs{globcount}, \dots)
% that use the higher numbers immediately, leaving the room for insertions.
% There should not be an efficiency penalty because \LuaTeX\ stores
% the registers of a class in the same Lua table unlike \eTeX, where
% registers below 256 are stored in an array and higher numbers
% are put in a tree structure.
%
% \subsection{Attributes}
%
% Nodes can have custom attributes in \LuaTeX. These attributes
% are organized by a new register class. As the other registers
% up to $2^{16}$ attributes are supported. An attribute value
% can be negative that means the attribute is not set. Otherwise
% \TeX's range of non-negative integers up to $2^31$ are available.
%
% \begin{declcs}{newattribute} \M{cmd}
% \end{declcs}
% Macro \cs{newattribute} defines command \meta{cmd} using \cs{attributedef}
% using an new attribute number. The new attribute is initially unset.
%
% \begin{declcs}{setattribute} \M{cmd} \M{value}
% \end{declcs}
% Macro \cs{setattribute} locally sets attribute command \meta{cmd}
% to the number \meta{value}. Valid values range from $-1$ until $2^31$
% (the upper limit is the same as for other \TeX\ integer numbers).
%
% \begin{declcs}{unsetattribute} \M{cmd}
% \end{declcs}
% Macro \cs{unsettattribute} clears the attribute command \meta{cmd}.
%
% \subsection{Catcode tables}
%
% \LuaTeX\ introduces catcode tables as new feature, see documentation.
% There is need for discussion, how to deal best:
% \begin{itemize}
% \item \cs{initcatcodetable} and \cs{setcatcodetable} act globally.
% \item \cs{catcodetable} causes an error if used with an uninitialized
%   catcode table.
% \item Large catcode table numbers should be avoided because of
%   performance breakdown.
% \item Use case \LaTeX\ package:
%   The package must not be surprised by changed catcodes and
%   must not surprise by changing catcodes accidently.
%   Catcode tables could offer a solution. At the begin
%   a catcode regime with standard catcodes is established
%   and the old one is restored afterwards.
% \item Use case: \LuaTeX's |tex.print| might be used with
%   a catcode table number, for example a table where all
%   entries have catcode ``other''.
% \item Readonly catcode tables.
% \item Is there is a need for local allocations?
%   (Package \xpackage{etex}'s \cs{loc} variants are not used in
%   \TeX\ Live 2007.)
% \end{itemize}
%
% \subsubsection{Interface proposal}
%
% The idea: \cs{newcatcodetable} allocates odd numbered catcode
% tables. Even numbered tables are managed as stack.
% Also some catcode tables are defined. These must not be
% changed.
%
% \begin{declcs}{newcatcodetable} \M{cmd}
% \end{declcs}
% Macro \cs{newcatcodetable} reserves a new catcode table and
% remembers its number in \meta{cmd}. The catcode table is
% initialized with \iniTeX's catcodes.
%
% \begin{declcs}{CatcodeTableIniTeX}\\
%   \cs{CatcodeTableString}\\
%   \cs{CatcodeTableOther}\\
%   \cs{CatcodeTableLaTeX}\\
% \end{declcs}
% These are catcode tables and must not be changed.
% \cs{CatcodeTableIniTeX} contains the catcode settings of \iniTeX.
% \cs{CatcodeTableString} follows \TeX's convention of
% \cs{string}, \cs{meaning} and friends. The space gets catcode
% 10 (space), the other characters have catcode 12 (other).
% In \cs{CatcodeTableOther} all entries have catcode 12 (other).
% \cs{CatcodeTableLaTeX} contains the setting of a pure \LaTeX\ format
% (`at' is other).
%
% \begin{declcs}{CatcodeTableStack}\\
%   \cs{IncCatcodeTableStack}\\
%   \cs{DecCatcodeTableStack}
% \end{declcs}
% \cs{CatcodeTableStack} is the stack pointer. Initially
% it is catcode table zero. \cs{IncCatcodeTableStack} and
% \cs{DecCatcodeTableStack} increments and decrements the
% stack pointer. Currently \cs{IncCatcodeTableStack} does not
% initialize a new catcode table. Both increment and decrement
% operations do not set a catcode table.
%
% \begin{declcs}{PushCatcodeTableNumStack}\\
%   \cs{PopCatcodeTableNumStack}
% \end{declcs}
% It can be handy to have a global stack for catcode table
% numbers to deal with the global assignment property of
% \cs{initcatcodetable} and \cs{savecatcodetable}.
% \cs{PushCatcodeTableNumStack} pushes the current catcode table
% on the stack.
% \cs{PopCatcodeTableNumStack} pops the topmost number off the number
% stack to set the current catcode table. Catcode table zero is
% used in case of an empty stack.
%
% \begin{declcs}{BeginCatcodeRegime} \M{catcodetable}\\
%   \cs{EndCatcodeRegime}
% \end{declcs}
% \cs{BeginCatcodeRegime} remembers the current catcode table number.
% Then it creates and uses a fresh catcode table on the stack
% that is initialized by \meta{catcodetable}:
% \begin{quote}\ttfamily
%   \cs{PushCatcodeTableNumStack}\\
%   \cs{catcodetable}\meta{catcodetable}||
%   \cs{IncCatcodeTableStack}\\
%   \cs{savecatcodetable}\cs{CatcodeTableStack}\\
%   \cs{catcodetable}\cs{CatcodeTableStack}
% \end{quote}
% \cs{EndCatcodeRegime} drops the catcode table, created by
% \cs{BeginCatcodeRegime} and sets the catcode table that was
% active before:
% \begin{quote}
%   \cs{DecCatcodeTableStack}\\
%   \cs{PopCatcodeTableNumStack}
% \end{quote}
% These macros solve the use case, described earlier for a \LaTeX\ package:
% \begin{quote}
%   |% package foobar.sty|\\
%   |\BeginCatcodeRegime\CatcodeTableLaTeX|\\
%   |\makeatletter|\\
%   |% ... package contents ...|\\
%   |\EndCatcodeRegime|\\
%   |% end of package|
% \end{quote}
% If the package wants to change catcodes after its loading,
% \cs{AtBeginDocument} or \cs{AtEndOfPackage} can be used.
%
% \begin{declcs}{SetCatcodeRange} \M{from} \M{to} \M{catcode}
% \end{declcs}
% The catcodes of characters in range from \meta{from}
% to inclusive \meta{to} are set to \meta{catcode}.
%
% \subsection{Lua module loading}
%
% Currently \LuaTeX\ (version 0.20) does not support Lua script
% files inside \xfile{TDS:scripts//}, because Lua's mechanism
% for module loading does not use the \textsf{kpathsea} library.
% Therefore this packages appends a kpse loader to the list of
% Lua's module loaders. It finds the module \meta{module} by
% \begin{quote}
%   \ttfamily |kpse.find_file("|\meta{module}|.lua", "texmfscripts")|
% \end{quote}
% Unhappily \textsf{kpathsea} does not support directory components
% in a file name. Therefore the Lua convention is not followed to
% replace dots in the module name by the directory separator.
%
% Example:
% A Lua script of a package \xpackage{foobar} wants the following
% modules:
% \begin{quote}
%   |require("foobar.hello.world")|\\
%   |require("org.somewhere.xyz")|
% \end{quote}
% Then they can be find in:
% \begin{quote}
%   |TDS:scripts/foobar/foobar.hello.world.lua|\\
%   |TDS:scripts/foobar/org.somewhere.xyz.lua|
% \end{quote}
% I would have preferred the following locations, following lua conventions,
% e.\,g.:
% \begin{quote}
%   |TDS:scripts/foobar/hello/world.lua|\\
%   |TDS:scripts/foobar/org/somewhere/xyz.lua|
% \end{quote}
% But I do not know, how to achieve this in a reliable way using
% \textsf{kpathsea}.
%
% \subsubsection{Package \xpackage{luatex-loader}}
%
% If someone do not need or want package \xpackage{luatex} but
% it's extension for module loading, then he can use
% package \xpackage{luatex-loader}. Both \plainTeX\ and \LaTeX\ are
% supported.
%
% \StopEventually{
% }
%
% \section{Implementation}
%
%    \begin{macrocode}
%<*package>
%    \end{macrocode}
%
% \subsection{Reload check and package identification}
%    Reload check, especially if the package is not used with \LaTeX.
%    \begin{macrocode}
\begingroup\catcode61\catcode48\catcode32=10\relax%
  \catcode13=5 % ^^M
  \endlinechar=13 %
  \catcode35=6 % #
  \catcode39=12 % '
  \catcode44=12 % ,
  \catcode45=12 % -
  \catcode46=12 % .
  \catcode58=12 % :
  \catcode64=11 % @
  \catcode123=1 % {
  \catcode125=2 % }
  \expandafter\let\expandafter\x\csname ver@luatex.sty\endcsname
  \ifx\x\relax % plain-TeX, first loading
  \else
    \def\empty{}%
    \ifx\x\empty % LaTeX, first loading,
      % variable is initialized, but \ProvidesPackage not yet seen
    \else
      \expandafter\ifx\csname PackageInfo\endcsname\relax
        \def\x#1#2{%
          \immediate\write-1{Package #1 Info: #2.}%
        }%
      \else
        \def\x#1#2{\PackageInfo{#1}{#2, stopped}}%
      \fi
      \x{luatex}{The package is already loaded}%
      \aftergroup\endinput
    \fi
  \fi
\endgroup%
%    \end{macrocode}
%    Package identification:
%    \begin{macrocode}
\begingroup\catcode61\catcode48\catcode32=10\relax%
  \catcode13=5 % ^^M
  \endlinechar=13 %
  \catcode35=6 % #
  \catcode39=12 % '
  \catcode40=12 % (
  \catcode41=12 % )
  \catcode44=12 % ,
  \catcode45=12 % -
  \catcode46=12 % .
  \catcode47=12 % /
  \catcode58=12 % :
  \catcode64=11 % @
  \catcode91=12 % [
  \catcode93=12 % ]
  \catcode123=1 % {
  \catcode125=2 % }
  \expandafter\ifx\csname ProvidesPackage\endcsname\relax
    \def\x#1#2#3[#4]{\endgroup
      \immediate\write-1{Package: #3 #4}%
      \xdef#1{#4}%
    }%
  \else
    \def\x#1#2[#3]{\endgroup
      #2[{#3}]%
      \ifx#1\@undefined
        \xdef#1{#3}%
      \fi
      \ifx#1\relax
        \xdef#1{#3}%
      \fi
    }%
  \fi
\expandafter\x\csname ver@luatex.sty\endcsname
\ProvidesPackage{luatex}%
  [2010/03/09 v0.4 LuaTeX basic definition package (HO)]%
%    \end{macrocode}
%
% \subsection{Catcodes}
%
%    \begin{macrocode}
\begingroup\catcode61\catcode48\catcode32=10\relax%
  \catcode13=5 % ^^M
  \endlinechar=13 %
  \catcode123=1 % {
  \catcode125=2 % }
  \catcode64=11 % @
  \def\x{\endgroup
    \expandafter\edef\csname LuT@AtEnd\endcsname{%
      \endlinechar=\the\endlinechar\relax
      \catcode13=\the\catcode13\relax
      \catcode32=\the\catcode32\relax
      \catcode35=\the\catcode35\relax
      \catcode61=\the\catcode61\relax
      \catcode64=\the\catcode64\relax
      \catcode123=\the\catcode123\relax
      \catcode125=\the\catcode125\relax
    }%
  }%
\x\catcode61\catcode48\catcode32=10\relax%
\catcode13=5 % ^^M
\endlinechar=13 %
\catcode35=6 % #
\catcode64=11 % @
\catcode123=1 % {
\catcode125=2 % }
\def\TMP@EnsureCode#1#2{%
  \edef\LuT@AtEnd{%
    \LuT@AtEnd
    \catcode#1=\the\catcode#1\relax
  }%
  \catcode#1=#2\relax
}
\TMP@EnsureCode{10}{12}% ^^J
\TMP@EnsureCode{34}{12}% "
\TMP@EnsureCode{36}{3}% $
\TMP@EnsureCode{39}{12}% '
\TMP@EnsureCode{40}{12}% (
\TMP@EnsureCode{41}{12}% )
\TMP@EnsureCode{42}{12}% *
\TMP@EnsureCode{43}{12}% +
\TMP@EnsureCode{44}{12}% ,
\TMP@EnsureCode{45}{12}% -
\TMP@EnsureCode{46}{12}% .
\TMP@EnsureCode{47}{12}% /
\TMP@EnsureCode{60}{12}% <
\TMP@EnsureCode{62}{12}% >
\TMP@EnsureCode{91}{12}% [
\TMP@EnsureCode{93}{12}% ]
\TMP@EnsureCode{95}{12}% _ (other!)
\TMP@EnsureCode{96}{12}% `
\edef\LuT@AtEnd{\LuT@AtEnd\noexpand\endinput}
%    \end{macrocode}
%
% \subsection{Check for \LuaTeX}
%
%    Without \LuaTeX\ there is no point in using this package.
%    \begin{macrocode}
\begingroup\expandafter\expandafter\expandafter\endgroup
\expandafter\ifx\csname RequirePackage\endcsname\relax
  \input infwarerr.sty\relax
  \input ifluatex.sty\relax
\else
  \RequirePackage{infwarerr}[2007/09/09]%
  \RequirePackage{ifluatex}[2009/04/10]%
\fi
%    \end{macrocode}
%    \begin{macrocode}
\ifluatex
\else
  \@PackageError{luatex}{%
    This package may only be run using LuaTeX%
  }\@ehc
  \expandafter\LuT@AtEnd
\fi%
%    \end{macrocode}
%
% \subsection{Provide Lua\TeX\ primitives}
%
%    \begin{macrocode}
\ifnum\luatexversion<36 %
  \def\LuT@MakePrimitive#1{%
    \expandafter\let\csname luatex#1\expandafter\endcsname
    \csname #1\endcsname
  }%
\else
  \def\LuT@MakeLuatexPrimitive#1{%
    \begingroup\expandafter\expandafter\expandafter\endgroup
    \expandafter\ifx\csname luatex#1\endcsname\relax
      \begingroup\expandafter\expandafter\expandafter\endgroup
      \expandafter\ifx\csname #1\endcsname\relax
      \else
        \expandafter\let
        \csname luatex#1\expandafter\endcsname
        \csname #1\endcsname
      \fi
    \fi
    \begingroup\expandafter\expandafter\expandafter\endgroup
    \expandafter\ifx\csname luatex#1\endcsname\relax
      \begingroup
        \expandafter\let\csname luatex#1\endcsname\@undefined
        \ifnum0%
          \directlua{%
            if tex.enableprimitives then %
              tex.enableprimitives('luatex',{'#1'})%
              tex.print('1')%
            end%
          }%
          \expandafter\ifx\csname luatex#1\endcsname\relax\else1\fi
        =11 %
          \global\expandafter\let
          \csname luatex#1\expandafter\endcsname
          \csname luatex#1\endcsname
        \else
          \@PackageError{luatex}{%
            tex.enableprimitives failed for `#1'%
          }\@ehc
        \fi
      \endgroup
    \fi
  }%
  \def\LuT@MakePrimitive#1{%
    \begingroup\expandafter\expandafter\expandafter\endgroup
    \expandafter\ifx\csname#1\endcsname\relax
      \begingroup
        \expandafter\let\csname#1\endcsname\@undefined
        \ifnum0%
          \directlua{%
            if tex.enableprimitives then %
              tex.enableprimitives('',{'#1'})%
              tex.print('1')%
            end%
          }%
          \expandafter\ifx\csname#1\endcsname\relax\else1\fi
        =11 %
          \global\expandafter\let
          \csname#1\expandafter\endcsname
          \csname#1\endcsname
        \else
          \@PackageError{luatex}{%
            tex.enableprimitives failed for `#1'%
          }\@ehc
        \fi
      \endgroup
    \fi
  }%
\fi
\LuT@MakeLuatexPrimitive{attribute}
\LuT@MakeLuatexPrimitive{attributedef}
\LuT@MakeLuatexPrimitive{catcodetable}
\LuT@MakeLuatexPrimitive{initcatcodetable}
\LuT@MakeLuatexPrimitive{luaescapestring}
\LuT@MakeLuatexPrimitive{savecatcodetable}
\LuT@MakePrimitive{numexpr}
%    \end{macrocode}
%
% \subsection{Inherit support for \eTeX}
%
%    Package \xpackage{etex} is not compatible for \plainTeX.
%    But it could be present if a format is used that is based
%    on \xfile{etex.src}. Therefore we only load the package
%    in case of \LaTeX\ and tests its presence independently
%    of the format by looking for \cs{et@xins}.
%    \begin{macrocode}
\begingroup\expandafter\expandafter\expandafter\endgroup
\expandafter\ifx\csname RequirePackage\endcsname\relax
\else
  \RequirePackage{etex}[1998/03/26]%
\fi
%    \end{macrocode}
%
% \subsection{Adaption of \eTeX's register allocation}
%
%    \eTeX has increased the number of \TeX\ registers
%    from $2^8$ (256) to $2^{15}$ (32768) for a register class.
%    \LuaTeX\ extends the limit further to $2^16$ (65536).
%    The allocation scheme of package \xpackage{etex} is not
%    changed. But this can be subject for discussion.
%
%    If a register class hasn't registered any local registers
%    yet, then the limit can safely be pushed to 65536.
%    \begin{macrocode}
\begingroup\expandafter\expandafter\expandafter\endgroup
\expandafter\ifx\csname et@xins\endcsname\relax
  \@PackageWarningNoLine{luatex}{%
    Support for eTeX is not loaded (etex.src)%
  }%
\else
  \def\LuT@temp#1{%
    \ifnum\count27#1=32768 %
      \count27#1=65536 %
    \fi
  }%
  \LuT@temp0%
  \LuT@temp1%
  \LuT@temp2%
  \LuT@temp3%
  \LuT@temp4%
  \LuT@temp5%
  \LuT@temp6%
%    \end{macrocode}
%
% \eTeX\ uses an array for the first 256 registers and then a tree
% structure. \LuaTeX\ stores all registers of a class in one Lua table.
% There shouldn't be large performance differences.
% This allows starting immediately in the extended
% area, leaving room for insertions.
%    \begin{macrocode}
  \let\newcount\globcount
  \let\newdimen\globdimen
  \let\newskip\globskip
  \let\newbox\globbox
\fi
%    \end{macrocode}
%
% \subsection{\plainTeX\ compatibility}
%
%    \begin{macro}{\@empty}
%    \begin{macrocode}
\expandafter\ifx\csname @empty\endcsname\relax
  \def\@empty{}%
\fi
%    \end{macrocode}
%    \end{macro}
%    \begin{macro}{\@gobble}
%    \begin{macrocode}
\expandafter\ifx\csname @gobble\endcsname\relax
  \long\def\@gobble#1{}%
\fi
%    \end{macrocode}
%    \end{macro}
%    \begin{macro}{\@firstofone}
%    \begin{macrocode}
\expandafter\ifx\csname @firstofone\endcsname\relax
  \long\def\@firstofone#1{#1}%
\fi
%    \end{macrocode}
%    \end{macro}
%    \begin{macro}{\@firstoftwo}
%    \begin{macrocode}
\expandafter\ifx\csname @firstoftwo\endcsname\relax
  \long\def\@firstoftwo#1#2{#1}%
\fi
%    \end{macrocode}
%    \end{macro}
%    \begin{macro}{\@car}
%    \begin{macrocode}
\expandafter\ifx\csname @car\endcsname\relax
  \def\@car#1#2\@nil{#1}%
\fi
%    \end{macrocode}
%    \end{macro}
%    \begin{macro}{\@cdr}
%    \begin{macrocode}
\expandafter\ifx\csname @cdr\endcsname\relax
  \def\@cdr#1#2\@nil{#2}%
\fi
%    \end{macrocode}
%    \end{macro}
%    \begin{macro}{\@ifstar}
%    \begin{macrocode}
\expandafter\ifx\csname @ifstar\endcsname\relax
  \def\@ifstar#1{%
    \@ifnextchar*{\@firstoftwo{#1}}%
  }%
%    \end{macrocode}
%    \end{macro}
%    \begin{macro}{\@ifnextchar}
%    \begin{macrocode}
  \long\def\@ifnextchar#1#2#3{%
    \let\reserved@d=#1%
    \def\reserved@a{#2}%
    \def\reserved@b{#3}%
    \futurelet\@let@token\@ifnch
  }%
%    \end{macrocode}
%    \end{macro}
%    \begin{macro}{\@ifnch}
%    \begin{macrocode}
  \def\@ifnch{%
    \ifx\@let@token\@sptoken
      \let\reserved@c\@xifnch
    \else
      \ifx\@let@token\reserved@d
        \let\reserved@c\reserved@a
      \else
        \let\reserved@c\reserved@b
      \fi
    \fi
    \reserved@c
  }%
%    \end{macrocode}
%    \end{macro}
%    \begin{macro}{\@sptoken}
%    \begin{macrocode}
  \let\LuT@temp\:%
  \def\:{\let\@sptoken= }%
  \: % explicit space
%    \end{macrocode}
%    \end{macro}
%    \begin{macro}{\@xifnch}
%    \begin{macrocode}
  \def\:{\@xifnch}%
  \expandafter\def\: {%
    \futurelet\@let@token\@ifnch
  }%
  \let\:\LuT@temp
\fi
%    \end{macrocode}
%    \end{macro}
%
%    \begin{macro}{\@tempcnta}
%    \begin{macrocode}
\expandafter\ifx\csname @tempcnta\endcsname\relax
  \csname newcount\endcsname\@tempcnta
\fi
%    \end{macrocode}
%    \end{macro}
%
%    \begin{macro}{\@tempcntb}
%    \begin{macrocode}
\expandafter\ifx\csname @tempcntb\endcsname\relax
  \csname newcount\endcsname\@tempcntb
\fi
%    \end{macrocode}
%    \end{macro}
%
%    \begin{macro}{\LuT@newcommand}
%    \begin{macrocode}
\begingroup\expandafter\expandafter\expandafter\endgroup
\expandafter\ifx\csname newcommand\endcsname\relax
  \def\LuT@newcommand#1[#2]#3{%
    \ifx#1\@undefined
      \let#1\relax
    \else
      \ifx#1\relax
      \else
        \@PackageError{luatex}{%
          \string#1 is already defined.\MessageBreak
          Redefinition is skipped%
        }\@ehc
      \fi
    \fi
    \ifx#1\relax
      \ifcase#2 %
        \def#1{#3}%
      \or
        \def#1##1{#3}%
      \or
        \def#1##1##2{#3}%
      \or
        \def#1##1##2##3{#3}%
      \or
        \@INTERNAL@ERROR
      \fi
    \fi
  }%
\else
  \def\LuT@newcommand{\newcommand*}%
\fi
%    \end{macrocode}
%    \end{macro}
%
% \subsection{Attributes}
%
% \subsubsection{Allocation}
%
%    \begin{macro}{\LuT@AllocAttribute}
%    \begin{macrocode}
\newcount\LuT@AllocAttribute
\LuT@AllocAttribute=\m@ne
%    \end{macrocode}
%    \end{macro}
%    \begin{macro}{\newattribute}
%    \begin{macrocode}
\LuT@newcommand\newattribute[1]{%
  \ifnum\LuT@AllocAttribute<65535 %
    \global\advance\LuT@AllocAttribute\@ne
    \allocationnumber\LuT@AllocAttribute
    \global\luatexattributedef#1=\allocationnumber
    \unsetattribute{#1}%
    \wlog{\string#1=\string\attribute\the\allocationnumber}%
  \else
    \errmessage{No room for a new \string\attribute}%
  \fi
}
%    \end{macrocode}
%    \end{macro}
%
% \subsubsection{Interface}
%
%    \begin{macro}{\setattribute}
%    \begin{macrocode}
\LuT@newcommand\setattribute[2]{%
  #1=\numexpr#2\relax
}
%    \end{macrocode}
%    \end{macro}
%    \begin{macro}{\unsetattribute}
%    \begin{macrocode}
\ifnum\luatexversion<37
  \LuT@newcommand\LuT@UnsetAttributeValue[0]{}%
  \let\LuT@UnsetAttributeValue\m@ne
\else
  \LuT@newcommand\LuT@UnsetAttributeValue[0]{-2147483647 }%
\fi
\LuT@newcommand\unsetattribute[1]{%
  #1=\LuT@UnsetAttributeValue
}
%    \end{macrocode}
%    \end{macro}
%
% \subsection{Catcode tables}
%
% \subsubsection{Allocation}
%
%    \begin{macro}{\LuT@AllocCatcodeTable}
%    \begin{macrocode}
\newcount\LuT@AllocCatcodeTable
\LuT@AllocCatcodeTable=\m@ne
\newcount\CatcodeTableStack
\CatcodeTableStack=\z@
%    \end{macrocode}
%    \end{macro}
%    \begin{macro}{\newcatcodetable}
%    \begin{macrocode}
\LuT@newcommand\newcatcodetable[1]{%
  \ifnum\LuT@AllocCatcodeTable<1114110 % 0x10FFFF is maximal \chardef
    % or < 268435455 % 2^28 - 1
    \global\advance\LuT@AllocCatcodeTable by\tw@
    \allocationnumber=\LuT@AllocCatcodeTable
    \global\chardef#1=\allocationnumber
    \wlog{%
      \string#1=\string\catcodetable\the\allocationnumber
    }%
  \else
    \errmessage{No room for a new \string\catcodetable}%
  \fi
}%
%    \end{macrocode}
%    \end{macro}
%
%    \begin{macro}{\IncCatcodeTableStack}
%    \begin{macrocode}
\LuT@newcommand\IncCatcodeTableStack[0]{%
  \ifnum\CatcodeTableStack<268435454 %
    \global\advance\CatcodeTableStack by\tw@
  \else
    \@PackageError{luatex}{%
      Catcode table stack overflow%
    }\@ehd
  \fi
}
%    \end{macrocode}
%    \end{macro}
%
%    \begin{macro}{\DecCatcodeTableStack}
%    \begin{macrocode}
\LuT@newcommand\DecCatcodeTableStack[0]{%
  \ifnum\CatcodeTableStack>\z@
    \global\advance\CatcodeTableStack by-2 %
  \else
    \@PackageError{luatex}{%
      Catcode table stack is empty%
    }\@ehd
  \fi
}
%    \end{macrocode}
%    \end{macro}
%
% \subsubsection{\cs{SetCatcodeRange}}
%
%    \begin{macro}{\SetCatcodeRange}
%    \begin{macrocode}
\LuT@newcommand\SetCatcodeRange[3]{%
  \edef\LuT@temp{%
    \noexpand\@tempcnta=\the\@tempcnta
    \noexpand\@tempcntb=\the\@tempcntb
    \noexpand\count@=\the\count@
    \relax
  }%
  \@tempcnta=\numexpr#1\relax
  \@tempcntb=\numexpr#2\relax
  \count@=\numexpr#3\relax
  \loop
    \unless\ifnum\@tempcnta>\@tempcntb
    \catcode\@tempcnta=\count@
    \advance\@tempcnta by \@ne
  \repeat
  \LuT@temp
}
%    \end{macrocode}
%    \end{macro}
%
% \subsubsection{Predefined catcode tables}
%
%    \begin{macrocode}
\newcatcodetable\CatcodeTableIniTeX
\newcatcodetable\CatcodeTableString
\newcatcodetable\CatcodeTableOther
\newcatcodetable\CatcodeTableLaTeX
%    \end{macrocode}
%
%    \begin{macrocode}
\luatexinitcatcodetable\CatcodeTableIniTeX
\begingroup
  \def\@makeother#1{\catcode#1=12\relax}%
  \@firstofone{%
    \luatexcatcodetable\CatcodeTableIniTeX
    \begingroup
      \SetCatcodeRange{0}{8}{15}%
      \catcode9=10 % tab
      \catcode11=15 %
      \catcode12=13 % form feed
      \SetCatcodeRange{14}{31}{15}%
      \catcode35=6 % hash
      \catcode36=3 % dollar
      \catcode38=4 % ampersand
      \catcode94=7 % circumflex
      \catcode95=8 % underscore
      \catcode123=1 % brace left
      \catcode125=2 % brace right
      \catcode126=13 % tilde
      \catcode127=15 %
      \luatexsavecatcodetable\CatcodeTableLaTeX
    \endgroup
    \@makeother{0}% nul
    \@makeother{13}% carriage return
    \@makeother{37}% percent
    \@makeother{92}% backslash
    \@makeother{127}%
    \SetCatcodeRange{65}{90}{12}% A-Z
    \SetCatcodeRange{97}{122}{12}% a-z
    \luatexsavecatcodetable\CatcodeTableString
    \@makeother{32}% space
    \luatexsavecatcodetable\CatcodeTableOther
  \endgroup
}%
%    \end{macrocode}
%
% \subsubsection{Number stack}
%
%    \begin{macro}{\LuT@NumStackEmpty}
%    A special empty stack value because of \cs{@cdr}'s brace removal.
%    \begin{macrocode}
\def\LuT@NumStackEmpty{0}
%    \end{macrocode}
%    \end{macro}
%    \begin{macro}{\LuT@NumStack}
%    \begin{macrocode}
\let\LuT@NumStack\LuT@NumStackEmpty
%    \end{macrocode}
%    \end{macro}
%    \begin{macro}{\PushCatcodeTableNumStack}
%    \begin{macrocode}
\LuT@newcommand\PushCatcodeTableNumStack[0]{%
  \xdef\LuT@NumStack{%
    {\the\luatexcatcodetable}\LuT@NumStack
  }%
}
%    \end{macrocode}
%    \end{macro}
%    \begin{macro}{\PopCatcodeTableNumStack}
%    \begin{macrocode}
\LuT@newcommand\PopCatcodeTableNumStack[0]{%
  \ifx\LuT@NumStack\LuT@NumStackEmpty
    \@PackageWarning{luatex}{Empty catcode table number stack}%
    \luatexcatcodetable\z@
  \else
    \luatexcatcodetable=\expandafter\@car\LuT@NumStack\@nil\relax
    \xdef\LuT@NumStack{%
      \expandafter\@cdr\LuT@NumStack\@nil
    }%
  \fi
}
%    \end{macrocode}
%    \end{macro}
%
% \subsubsection{Catcode regime macros}
%
%    \begin{macro}{\BeginCatcodeRegime}
%    \begin{macrocode}
\LuT@newcommand\BeginCatcodeRegime[1]{%
  \PushCatcodeTableNumStack
  \luatexcatcodetable=\numexpr#1\relax
  \IncCatcodeTableStack
  \luatexsavecatcodetable\CatcodeTableStack
  \luatexcatcodetable\CatcodeTableStack
}
%    \end{macrocode}
%    \end{macro}
%
%    \begin{macro}{\EndCatcodeRegime}
%    \begin{macrocode}
\LuT@newcommand\EndCatcodeRegime[0]{%
  \DecCatcodeTableStack
  \PopCatcodeTableNumStack
}
%    \end{macrocode}
%    \end{macro}
%
% \subsection{Lua module loader}
%
%    \begin{macrocode}
\begingroup\expandafter\expandafter\expandafter\endgroup
\expandafter\ifx\csname RequirePackage\endcsname\relax
  \input luatex-loader.sty\relax
\else
  \RequirePackage{luatex-loader}[2010/03/09]%
\fi
%    \end{macrocode}
%    \begin{macrocode}
\LuT@AtEnd%
%</package>
%    \end{macrocode}
%
%    \begin{macrocode}
%<*loader>
%    \end{macrocode}
%
%    Reload check, especially if the package is not used with \LaTeX.
%    \begin{macrocode}
\begingroup\catcode61\catcode48\catcode32=10\relax%
  \catcode13=5 % ^^M
  \endlinechar=13 %
  \catcode35=6 % #
  \catcode39=12 % '
  \catcode44=12 % ,
  \catcode45=12 % -
  \catcode46=12 % .
  \catcode58=12 % :
  \catcode64=11 % @
  \catcode123=1 % {
  \catcode125=2 % }
  \expandafter\let\expandafter\x\csname ver@luatex-loader.sty\endcsname
  \ifx\x\relax % plain-TeX, first loading
  \else
    \def\empty{}%
    \ifx\x\empty % LaTeX, first loading,
      % variable is initialized, but \ProvidesPackage not yet seen
    \else
      \expandafter\ifx\csname PackageInfo\endcsname\relax
        \def\x#1#2{%
          \immediate\write-1{Package #1 Info: #2.}%
        }%
      \else
        \def\x#1#2{\PackageInfo{#1}{#2, stopped}}%
      \fi
      \x{luatex-loader}{The package is already loaded}%
      \aftergroup\endinput
    \fi
  \fi
\endgroup%
%    \end{macrocode}
%    Package identification:
%    \begin{macrocode}
\begingroup\catcode61\catcode48\catcode32=10\relax%
  \catcode13=5 % ^^M
  \endlinechar=13 %
  \catcode35=6 % #
  \catcode39=12 % '
  \catcode40=12 % (
  \catcode41=12 % )
  \catcode44=12 % ,
  \catcode45=12 % -
  \catcode46=12 % .
  \catcode47=12 % /
  \catcode58=12 % :
  \catcode64=11 % @
  \catcode91=12 % [
  \catcode93=12 % ]
  \catcode123=1 % {
  \catcode125=2 % }
  \expandafter\ifx\csname ProvidesPackage\endcsname\relax
    \def\x#1#2#3[#4]{\endgroup
      \immediate\write-1{Package: #3 #4}%
      \xdef#1{#4}%
    }%
  \else
    \def\x#1#2[#3]{\endgroup
      #2[{#3}]%
      \ifx#1\@undefined
        \xdef#1{#3}%
      \fi
      \ifx#1\relax
        \xdef#1{#3}%
      \fi
    }%
  \fi
\expandafter\x\csname ver@luatex-loader.sty\endcsname
\ProvidesPackage{luatex-loader}%
  [2010/03/09 v0.4 Lua module loader (HO)]%
%    \end{macrocode}
%    \begin{macrocode}
\begingroup\catcode61\catcode48\catcode32=10\relax%
  \catcode13=5\endlinechar=13\relax%
  \catcode10=12 % ^^J
  \catcode34=12 % "
  \catcode39=12 % '
  \catcode40=12 % (
  \catcode41=12 % )
  \catcode44=12 % ,
  \catcode46=12 % .
  \catcode60=12 % <
  \catcode61=12 % =
  \catcode95=12 % _ (other!)
  \catcode96=12 % `
  \catcode123=1 % {
  \catcode125=2 % }
  \endlinechar=10 %
  \ifnum\luatexversion<36 %
    \directlua0%
  \else %
    \expandafter\directlua %
  \fi %
  {%
     do
      local script = "oberdiek.luatex.lua"
      local file = kpse.find_file(script, "texmfscripts")
      if file then
        texio.write_nl("(" .. file .. ")")
        dofile(file)
      else
        error("File `" .. script .. "' not found")
      end
    end
  }%
\endgroup\endinput%
%    \end{macrocode}
%
%    \begin{macrocode}
%</loader>
%    \end{macrocode}
%
% \subsection{Lua script}
%
%    Currently \LuaTeX\ does not use KPSE when searching for module
%    files. The following Lua script implements a workaround.
%    It extends |package.loader| by another search method.
%    Modules are found by the module name with extension \xext{lua}
%    similar to
%    \begin{quote}
%      \ttfamily |kpsewhich --format=texmfscripts| \meta{module}|.lua|
%    \end{quote}
%    Unhappily \textsf{kpsewhich} does not support directory components
%    in the file name. Therefore a module |a.b.c| cannot be installed
%    as \xfile{a/b/c.lua}. The script must be named \xfile{a.b.c.lua}.
%    \begin{macrocode}
%<*lua>
%    \end{macrocode}
%    \begin{macrocode}
module("oberdiek.luatex", package.seeall)
%    \end{macrocode}
%    \begin{macrocode}
function kpse_module_loader(module)
  local script = module .. ".lua"
  local file = kpse.find_file(script, "texmfscripts")
  if file then
    local loader, error = loadfile(file)
    if loader then
      texio.write_nl("(" .. file .. ")")
      return loader
    end
    return "\n\t[oberdiek.luatex.kpse_module_loader] Loading error:\n\t"
           .. error
  end
  return "\n\t[oberdiek.luatex.kpse_module_loader] Search failed"
end
table.insert(package.loaders, kpse_module_loader)
%    \end{macrocode}
%    \begin{macrocode}
%</lua>
%    \end{macrocode}
%
% \section{Test}
%
%    \begin{macrocode}
%<*test2>
\documentclass{article}
\def\LoadCommand{%
  \RequirePackage{luatex}[2010/03/09]%
}
%</test2>
%    \end{macrocode}
%    \begin{macrocode}
%<*test3>
\documentclass{article}
\def\LoadCommand{%
  \RequirePackage{luatex-loader}[2010/03/09]%
}
%</test3>
%    \end{macrocode}
% \subsection{Catcode checks for loading}
%
%    \begin{macrocode}
%<*test1>
%    \end{macrocode}
%    \begin{macrocode}
\catcode`\{=1 %
\catcode`\}=2 %
\catcode`\#=6 %
\catcode`\@=11 %
\expandafter\ifx\csname count@\endcsname\relax
  \countdef\count@=255 %
\fi
\expandafter\ifx\csname @gobble\endcsname\relax
  \long\def\@gobble#1{}%
\fi
\expandafter\ifx\csname @firstofone\endcsname\relax
  \long\def\@firstofone#1{#1}%
\fi
\expandafter\ifx\csname loop\endcsname\relax
  \expandafter\@firstofone
\else
  \expandafter\@gobble
\fi
{%
  \def\loop#1\repeat{%
    \def\body{#1}%
    \iterate
  }%
  \def\iterate{%
    \body
      \let\next\iterate
    \else
      \let\next\relax
    \fi
    \next
  }%
  \let\repeat=\fi
}%
\def\RestoreCatcodes{}
\count@=0 %
\loop
  \edef\RestoreCatcodes{%
    \RestoreCatcodes
    \catcode\the\count@=\the\catcode\count@\relax
  }%
\ifnum\count@<255 %
  \advance\count@ 1 %
\repeat

\def\RangeCatcodeInvalid#1#2{%
  \count@=#1\relax
  \loop
    \catcode\count@=15 %
  \ifnum\count@<#2\relax
    \advance\count@ 1 %
  \repeat
}
\def\RangeCatcodeCheck#1#2#3{%
  \count@=#1\relax
  \loop
    \ifnum#3=\catcode\count@
    \else
      \errmessage{%
        Character \the\count@\space
        with wrong catcode \the\catcode\count@\space
        instead of \number#3%
      }%
    \fi
  \ifnum\count@<#2\relax
    \advance\count@ 1 %
  \repeat
}
\def\space{ }
\expandafter\ifx\csname LoadCommand\endcsname\relax
  \def\LoadCommand{\input luatex.sty\relax}%
\fi
\def\Test{%
  \RangeCatcodeInvalid{0}{47}%
  \RangeCatcodeInvalid{58}{64}%
  \RangeCatcodeInvalid{91}{96}%
  \RangeCatcodeInvalid{123}{255}%
  \catcode`\@=12 %
  \catcode`\\=0 %
  \catcode`\%=14 %
  \LoadCommand
  \RangeCatcodeCheck{0}{36}{15}%
  \RangeCatcodeCheck{37}{37}{14}%
  \RangeCatcodeCheck{38}{47}{15}%
  \RangeCatcodeCheck{48}{57}{12}%
  \RangeCatcodeCheck{58}{63}{15}%
  \RangeCatcodeCheck{64}{64}{12}%
  \RangeCatcodeCheck{65}{90}{11}%
  \RangeCatcodeCheck{91}{91}{15}%
  \RangeCatcodeCheck{92}{92}{0}%
  \RangeCatcodeCheck{93}{96}{15}%
  \RangeCatcodeCheck{97}{122}{11}%
  \RangeCatcodeCheck{123}{255}{15}%
  \RestoreCatcodes
}
\Test
\csname @@end\endcsname
\end
%    \end{macrocode}
%    \begin{macrocode}
%</test1>
%    \end{macrocode}
%
% \subsection{Catcode tables}
%
% \subsubsection{Predefined catcode tables}
%
%    \begin{macrocode}
%<*test4>
\NeedsTeXFormat{LaTeX2e}
%    \end{macrocode}
%    Remember \LaTeX's initial catcodes in count registers
%    starting at \cs{TestLaTeX}.
%    \begin{macrocode}
\count0=0 %
\chardef\TestLaTeX=1000 %
\chardef\TestMax=300 %
\loop
  \count\numexpr\TestLaTeX+\count0\relax=\catcode\count0 %
\ifnum\count0<\TestMax
  \advance\count0 by 1 %
\repeat
\documentclass{minimal}
\usepackage{luatex}[2010/03/09]
\usepackage{qstest}
\IncludeTests{*}
\LogTests{log}{*}{*}
\makeatletter
\def\Check#1{%
  \Expect*{\the\count@=\the\catcode\count@}%
         *{\the\count@=#1}%
}
\newcount\scratch
\def\Test#1#2{%
  \begin{qstest}{CatcodeTable#1}{CatcodeTable#1}%
    \luatexcatcodetable\csname CatcodeTable#1\endcsname
    \count@=\z@
    \loop
      \scratch=#2\relax
      \Expect*{\the\count@=\the\catcode\count@}%
             *{\the\count@=\the\scratch}%
    \ifnum\count@<\TestMax
      \advance\count@\@ne
    \repeat
  \end{qstest}%
}
\begingroup
  % luatex-unicode-letters.tex makes some slots to letters
  \def\TestMax{169}%
  \Test{LaTeX}{\the\count\numexpr\TestLaTeX+\count@}%
\endgroup
\Test{String}{\ifnum\count@=32 10\else 12\fi}
\Test{Other}{12}
\luatexinitcatcodetable99 %
\Test{IniTeX}{%
  0\relax
  \begingroup
    \luatexcatcodetable99 %
    \global\scratch=\the\catcode\count@
  \endgroup
}
%    \end{macrocode}
%
% \subsubsection{Catcode table number stack}
%
%    \begin{macrocode}
\begin{qstest}{CatcodeTableNumStack}{CatcodeTableNumStack}
  \def\TestStack#1{%
    \Expect*{\LuT@NumStack}{#1}%
  }%
  \TestStack{0}%
  \PushCatcodeTableNumStack
  \TestStack{{0}0}%
  \@firstofone{%
    \begingroup
      \luatexinitcatcodetable12 %
      \luatexcatcodetable12 %
      \PushCatcodeTableNumStack
      \TestStack{{12}{0}0}%
      \PopCatcodeTableNumStack
      \TestStack{{0}0}%
      \PopCatcodeTableNumStack
      \TestStack{0}%
      \def\TestWarning{Missing empty stack warning}%
      \def\@PackageWarning#1#2{\def\TestWarning{empty stack}}%
      \PopCatcodeTableNumStack
      \TestStack{0}%
      \Expect*{\TestWarning}{empty stack}%
    \endgroup
  }%
\end{qstest}
%    \end{macrocode}
%
% \subsubsection{Catcode table stack}
%
%    \begin{macrocode}
\begin{qstest}{CatcodeTableStack}{CatcodeTableStack}
  \def\TestStack#1{%
    \Expect*{\the\CatcodeTableStack}{#1}%
  }%
  \TestStack{0}%
  \IncCatcodeTableStack
  \TestStack{2}%
  \IncCatcodeTableStack
  \TestStack{4}%
  \begingroup
    \IncCatcodeTableStack
    \TestStack{6}%
  \endgroup
  \TestStack{6}%
  \begingroup
    \DecCatcodeTableStack
    \TestStack{4}%
  \endgroup
  \TestStack{4}%
  \DecCatcodeTableStack
  \TestStack{2}%
  \DecCatcodeTableStack
  \TestStack{0}%
  \begingroup
    \def\TestError{Missing error}%
    \def\@PackageError#1#2#3{%
      \def\TestError{Empty stack}%
    }%
    \DecCatcodeTableStack
    \TestStack{0}%
    \Expect*{\TestError}{Empty stack}%
  \endgroup
\end{qstest}
%    \end{macrocode}
%
% \subsubsection{Catcode regime macros}
%
%    \begin{macrocode}
\begin{qstest}{CatcodeRegime}{CatcodeRegime}
  \def\TestStacks#1#2#3{%
    \Expect*{\the\luatexcatcodetable}{#1}%
    \Expect*{\the\CatcodeTableStack}{#2}%
    \Expect*{\LuT@NumStack}{#3}%
  }%
  \TestStacks{0}{0}{0}%
  \catcode`\|=7 %
  \BeginCatcodeRegime\CatcodeTableLaTeX
    \TestStacks{2}{2}{{0}0}%
    \Expect*{\the\catcode`\|}{12}%
  \EndCatcodeRegime
  \TestStacks{0}{0}{0}%
  \Expect*{\the\catcode`\|}{7}%
\end{qstest}
%    \end{macrocode}
%
% \subsection{Attribute allocation}
%
%    \begin{macrocode}
\begin{qstest}{Attributes}{Attributes}
  \newattribute\TestAttr
  \Expect*{\meaning\TestAttr}%
         *{\string\attribute\number\allocationnumber}%
  \Expect*{\the\allocationnumber}{0}%
  \begingroup
    \newattribute\TestAttr
    \Expect*{\the\allocationnumber}{1}%
  \endgroup
  \Expect*{\the\allocationnumber}{0}%
  \Expect*{\meaning\TestAttr}*{\string\attribute1}%
  \Expect*{\the\TestAttr}*{\number\LuT@UnsetAttributeValue}%
  \def\Test#1{%
    \setattribute\TestAttr{#1}%
    \Expect*{\the\TestAttr}{#1}%
  }%
  \Test{0}%
  \Test{1}%
  \Test{-1}%
  \Test{123}%
  \unsetattribute\TestAttr
  \Expect*{\the\TestAttr}*{\number\LuT@UnsetAttributeValue}%
  \begingroup
    \Expect*{\the\TestAttr}*{\number\LuT@UnsetAttributeValue}%
    \Test{1234}%
  \endgroup
  \Expect*{\the\TestAttr}*{\number\LuT@UnsetAttributeValue}%
\end{qstest}
%    \end{macrocode}
%
%    \begin{macrocode}
\@@end
%</test4>
%    \end{macrocode}
%
% \subsection{Short test for \plainTeX}
%
%    \begin{macrocode}
%<*test5>
\input luatex.sty\relax
\newattribute\TestAttr
\setattribute\TestAttr{10}
\unsetattribute\TestAttr
\newcatcodetable\TestCTa
\begingroup
  \SetCatcodeRange{`A}{`Z}{12}%
\endgroup
\BeginCatcodeRegime\CatcodeTableLaTeX
\EndCatcodeRegime
\end
%</test5>
%    \end{macrocode}
%
% \section{Installation}
%
% \subsection{Download}
%
% \paragraph{Package.} This package is available on
% CTAN\footnote{\url{ftp://ftp.ctan.org/tex-archive/}}:
% \begin{description}
% \item[\CTAN{macros/latex/contrib/oberdiek/luatex.dtx}] The source file.
% \item[\CTAN{macros/latex/contrib/oberdiek/luatex.pdf}] Documentation.
% \end{description}
%
%
% \paragraph{Bundle.} All the packages of the bundle `oberdiek'
% are also available in a TDS compliant ZIP archive. There
% the packages are already unpacked and the documentation files
% are generated. The files and directories obey the TDS standard.
% \begin{description}
% \item[\CTAN{install/macros/latex/contrib/oberdiek.tds.zip}]
% \end{description}
% \emph{TDS} refers to the standard ``A Directory Structure
% for \TeX\ Files'' (\CTAN{tds/tds.pdf}). Directories
% with \xfile{texmf} in their name are usually organized this way.
%
% \subsection{Bundle installation}
%
% \paragraph{Unpacking.} Unpack the \xfile{oberdiek.tds.zip} in the
% TDS tree (also known as \xfile{texmf} tree) of your choice.
% Example (linux):
% \begin{quote}
%   |unzip oberdiek.tds.zip -d ~/texmf|
% \end{quote}
%
% \paragraph{Script installation.}
% Check the directory \xfile{TDS:scripts/oberdiek/} for
% scripts that need further installation steps.
% Package \xpackage{attachfile2} comes with the Perl script
% \xfile{pdfatfi.pl} that should be installed in such a way
% that it can be called as \texttt{pdfatfi}.
% Example (linux):
% \begin{quote}
%   |chmod +x scripts/oberdiek/pdfatfi.pl|\\
%   |cp scripts/oberdiek/pdfatfi.pl /usr/local/bin/|
% \end{quote}
%
% \subsection{Package installation}
%
% \paragraph{Unpacking.} The \xfile{.dtx} file is a self-extracting
% \docstrip\ archive. The files are extracted by running the
% \xfile{.dtx} through \plainTeX:
% \begin{quote}
%   \verb|tex luatex.dtx|
% \end{quote}
%
% \paragraph{TDS.} Now the different files must be moved into
% the different directories in your installation TDS tree
% (also known as \xfile{texmf} tree):
% \begin{quote}
% \def\t{^^A
% \begin{tabular}{@{}>{\ttfamily}l@{ $\rightarrow$ }>{\ttfamily}l@{}}
%   luatex.sty & tex/generic/oberdiek/luatex.sty\\
%   luatex-loader.sty & tex/generic/oberdiek/luatex-loader.sty\\
%   oberdiek.luatex.lua & scripts/oberdiek/oberdiek.luatex.lua\\
%   luatex.pdf & doc/latex/oberdiek/luatex.pdf\\
%   test/luatex-test1.tex & doc/latex/oberdiek/test/luatex-test1.tex\\
%   test/luatex-test2.tex & doc/latex/oberdiek/test/luatex-test2.tex\\
%   test/luatex-test3.tex & doc/latex/oberdiek/test/luatex-test3.tex\\
%   test/luatex-test4.tex & doc/latex/oberdiek/test/luatex-test4.tex\\
%   test/luatex-test5.tex & doc/latex/oberdiek/test/luatex-test5.tex\\
%   luatex.dtx & source/latex/oberdiek/luatex.dtx\\
% \end{tabular}^^A
% }^^A
% \sbox0{\t}^^A
% \ifdim\wd0>\linewidth
%   \begingroup
%     \advance\linewidth by\leftmargin
%     \advance\linewidth by\rightmargin
%   \edef\x{\endgroup
%     \def\noexpand\lw{\the\linewidth}^^A
%   }\x
%   \def\lwbox{^^A
%     \leavevmode
%     \hbox to \linewidth{^^A
%       \kern-\leftmargin\relax
%       \hss
%       \usebox0
%       \hss
%       \kern-\rightmargin\relax
%     }^^A
%   }^^A
%   \ifdim\wd0>\lw
%     \sbox0{\small\t}^^A
%     \ifdim\wd0>\linewidth
%       \ifdim\wd0>\lw
%         \sbox0{\footnotesize\t}^^A
%         \ifdim\wd0>\linewidth
%           \ifdim\wd0>\lw
%             \sbox0{\scriptsize\t}^^A
%             \ifdim\wd0>\linewidth
%               \ifdim\wd0>\lw
%                 \sbox0{\tiny\t}^^A
%                 \ifdim\wd0>\linewidth
%                   \lwbox
%                 \else
%                   \usebox0
%                 \fi
%               \else
%                 \lwbox
%               \fi
%             \else
%               \usebox0
%             \fi
%           \else
%             \lwbox
%           \fi
%         \else
%           \usebox0
%         \fi
%       \else
%         \lwbox
%       \fi
%     \else
%       \usebox0
%     \fi
%   \else
%     \lwbox
%   \fi
% \else
%   \usebox0
% \fi
% \end{quote}
% If you have a \xfile{docstrip.cfg} that configures and enables \docstrip's
% TDS installing feature, then some files can already be in the right
% place, see the documentation of \docstrip.
%
% \subsection{Refresh file name databases}
%
% If your \TeX~distribution
% (\teTeX, \mikTeX, \dots) relies on file name databases, you must refresh
% these. For example, \teTeX\ users run \verb|texhash| or
% \verb|mktexlsr|.
%
% \subsection{Some details for the interested}
%
% \paragraph{Attached source.}
%
% The PDF documentation on CTAN also includes the
% \xfile{.dtx} source file. It can be extracted by
% AcrobatReader 6 or higher. Another option is \textsf{pdftk},
% e.g. unpack the file into the current directory:
% \begin{quote}
%   \verb|pdftk luatex.pdf unpack_files output .|
% \end{quote}
%
% \paragraph{Unpacking with \LaTeX.}
% The \xfile{.dtx} chooses its action depending on the format:
% \begin{description}
% \item[\plainTeX:] Run \docstrip\ and extract the files.
% \item[\LaTeX:] Generate the documentation.
% \end{description}
% If you insist on using \LaTeX\ for \docstrip\ (really,
% \docstrip\ does not need \LaTeX), then inform the autodetect routine
% about your intention:
% \begin{quote}
%   \verb|latex \let\install=y\input{luatex.dtx}|
% \end{quote}
% Do not forget to quote the argument according to the demands
% of your shell.
%
% \paragraph{Generating the documentation.}
% You can use both the \xfile{.dtx} or the \xfile{.drv} to generate
% the documentation. The process can be configured by the
% configuration file \xfile{ltxdoc.cfg}. For instance, put this
% line into this file, if you want to have A4 as paper format:
% \begin{quote}
%   \verb|\PassOptionsToClass{a4paper}{article}|
% \end{quote}
% An example follows how to generate the
% documentation with pdf\LaTeX:
% \begin{quote}
%\begin{verbatim}
%pdflatex luatex.dtx
%makeindex -s gind.ist luatex.idx
%pdflatex luatex.dtx
%makeindex -s gind.ist luatex.idx
%pdflatex luatex.dtx
%\end{verbatim}
% \end{quote}
%
% \section{Catalogue}
%
% The following XML file can be used as source for the
% \href{http://mirror.ctan.org/help/Catalogue/catalogue.html}{\TeX\ Catalogue}.
% The elements \texttt{caption} and \texttt{description} are imported
% from the original XML file from the Catalogue.
% The name of the XML file in the Catalogue is \xfile{luatex.xml}.
%    \begin{macrocode}
%<*catalogue>
<?xml version='1.0' encoding='us-ascii'?>
<!DOCTYPE entry SYSTEM 'catalogue.dtd'>
<entry datestamp='$Date$' modifier='$Author$' id='luatex'>
  <name>luatex</name>
  <caption>The LuaTeX engine.</caption>
  <authorref id='auth:oberdiek'/>
  <copyright owner='Heiko Oberdiek' year='2007,2009,2010'/>
  <license type='lppl1.3'/>
  <version number='0.4'/>
  <description>
    LuaTeX is an extended version of pdfTeX using Lua as an embedded
    scripting language.  The LuaTeX project&#x2019;s main objective
    is to provide an open and configurable variant of TeX while at the
    same time offering downward compatibility.
    <p/>
    LuaTeX uses Unicode (as UTF-8) as its default input encoding, and
    is able to use modern (OpenType) fonts (for both text and mathematics).
    <p/>
    It should be noted that LuaTeX is still under development; its
    specification has been declared stable, but absolute stability
    may not in practice be assumed.
    <p/>
    The package is part of the <xref refid='oberdiek'>oberdiek</xref> bundle.
  </description>
  <documentation details='Package documentation'
      href='ctan:/macros/latex/contrib/oberdiek/luatex.pdf'/>
  <ctan file='true' path='/macros/latex/contrib/oberdiek/luatex.dtx'/>
  <miktex location='oberdiek'/>
  <texlive location='oberdiek'/>
  <install path='/macros/latex/contrib/oberdiek/oberdiek.tds.zip'/>
</entry>
%</catalogue>
%    \end{macrocode}
%
% \begin{History}
%   \begin{Version}{2007/12/12 v0.1}
%   \item
%     First public version.
%   \end{Version}
%   \begin{Version}{2009/04/10 v0.2}
%   \item
%     Requires package \xpackage{ifluatex} in version 2.0 to ensure
%     \cs{luatexversion}.
%   \item
%     Updates the call of \cs{directlua}, the syntax has changed
%     in \LuaTeX\ 0.36.
%   \end{Version}
%   \begin{Version}{2009/12/02 v0.3}
%   \item
%     Unsetting of attributes updated for \LuaTeX\ 0.37.
%   \end{Version}
%   \begin{Version}{2010/03/09 v0.4}
%   \item
%     Support for lua states removed.
%   \item
%     Calling tex.enableprimitives for used primitives.
%   \end{Version}
% \end{History}
%
% \PrintIndex
%
% \Finale
\endinput

%        (quote the arguments according to the demands of your shell)
%
% Documentation:
%    (a) If luatex.drv is present:
%           latex luatex.drv
%    (b) Without luatex.drv:
%           latex luatex.dtx; ...
%    The class ltxdoc loads the configuration file ltxdoc.cfg
%    if available. Here you can specify further options, e.g.
%    use A4 as paper format:
%       \PassOptionsToClass{a4paper}{article}
%
%    Programm calls to get the documentation (example):
%       pdflatex luatex.dtx
%       makeindex -s gind.ist luatex.idx
%       pdflatex luatex.dtx
%       makeindex -s gind.ist luatex.idx
%       pdflatex luatex.dtx
%
% Installation:
%    TDS:tex/generic/oberdiek/luatex.sty
%    TDS:tex/generic/oberdiek/luatex-loader.sty
%    TDS:scripts/oberdiek/oberdiek.luatex.lua
%    TDS:doc/latex/oberdiek/luatex.pdf
%    TDS:doc/latex/oberdiek/test/luatex-test1.tex
%    TDS:doc/latex/oberdiek/test/luatex-test2.tex
%    TDS:doc/latex/oberdiek/test/luatex-test3.tex
%    TDS:doc/latex/oberdiek/test/luatex-test4.tex
%    TDS:doc/latex/oberdiek/test/luatex-test5.tex
%    TDS:source/latex/oberdiek/luatex.dtx
%
%<*ignore>
\begingroup
  \catcode123=1 %
  \catcode125=2 %
  \def\x{LaTeX2e}%
\expandafter\endgroup
\ifcase 0\ifx\install y1\fi\expandafter
         \ifx\csname processbatchFile\endcsname\relax\else1\fi
         \ifx\fmtname\x\else 1\fi\relax
\else\csname fi\endcsname
%</ignore>
%<*install>
\input docstrip.tex
\Msg{************************************************************************}
\Msg{* Installation}
\Msg{* Package: luatex 2010/03/09 v0.4 LuaTeX basic definition package (HO)}
\Msg{************************************************************************}

\keepsilent
\askforoverwritefalse

\let\MetaPrefix\relax
\preamble

This is a generated file.

Project: luatex
Version: 2010/03/09 v0.4

Copyright (C) 2007, 2009, 2010 by
   Heiko Oberdiek <heiko.oberdiek at googlemail.com>

This work may be distributed and/or modified under the
conditions of the LaTeX Project Public License, either
version 1.3c of this license or (at your option) any later
version. This version of this license is in
   http://www.latex-project.org/lppl/lppl-1-3c.txt
and the latest version of this license is in
   http://www.latex-project.org/lppl.txt
and version 1.3 or later is part of all distributions of
LaTeX version 2005/12/01 or later.

This work has the LPPL maintenance status "maintained".

This Current Maintainer of this work is Heiko Oberdiek.

The Base Interpreter refers to any `TeX-Format',
because some files are installed in TDS:tex/generic//.

This work consists of the main source file luatex.dtx
and the derived files
   luatex.sty, luatex.pdf, luatex.ins, luatex.drv, luatex-loader.sty,
   luatex-test1.tex, luatex-test2.tex, luatex-test3.tex,
   luatex-test4.tex, luatex-test5.tex, oberdiek.luatex.lua.

\endpreamble
\let\MetaPrefix\DoubleperCent

\generate{%
  \file{luatex.ins}{\from{luatex.dtx}{install}}%
  \file{luatex.drv}{\from{luatex.dtx}{driver}}%
  \usedir{tex/generic/oberdiek}%
  \file{luatex.sty}{\from{luatex.dtx}{package}}%
  \file{luatex-loader.sty}{\from{luatex.dtx}{loader}}%
  \usedir{doc/latex/oberdiek/test}%
  \file{luatex-test1.tex}{\from{luatex.dtx}{test1}}%
  \file{luatex-test2.tex}{\from{luatex.dtx}{test1,test2}}%
  \file{luatex-test3.tex}{\from{luatex.dtx}{test1,test3}}%
  \file{luatex-test4.tex}{\from{luatex.dtx}{test4}}%
  \file{luatex-test5.tex}{\from{luatex.dtx}{test5}}%
  \nopreamble
  \nopostamble
  \usedir{source/latex/oberdiek/catalogue}%
  \file{luatex.xml}{\from{luatex.dtx}{catalogue}}%
}
\def\MetaPrefix{-- }
\def\defaultpostamble{%
  \MetaPrefix^^J%
  \MetaPrefix\space End of File `\outFileName'.%
}
\def\currentpostamble{\defaultpostamble}%
\generate{%
  \usedir{scripts/oberdiek}%
  \file{oberdiek.luatex.lua}{\from{luatex.dtx}{lua}}%
}

\catcode32=13\relax% active space
\let =\space%
\Msg{************************************************************************}
\Msg{*}
\Msg{* To finish the installation you have to move the following}
\Msg{* files into a directory searched by TeX:}
\Msg{*}
\Msg{*     luatex.sty, luatex-loader.sty}
\Msg{*}
\Msg{* And install the following script file:}
\Msg{*}
\Msg{*     oberdiek.luatex.lua}
\Msg{*}
\Msg{* To produce the documentation run the file `luatex.drv'}
\Msg{* through LaTeX.}
\Msg{*}
\Msg{* Happy TeXing!}
\Msg{*}
\Msg{************************************************************************}

\endbatchfile
%</install>
%<*ignore>
\fi
%</ignore>
%<*driver>
\NeedsTeXFormat{LaTeX2e}
\ProvidesFile{luatex.drv}%
  [2010/03/09 v0.4 LuaTeX basic definition package (HO)]%
\documentclass{ltxdoc}
\usepackage{holtxdoc}[2011/11/22]
\usepackage{paralist}
\begin{document}
  \DocInput{luatex.dtx}%
\end{document}
%</driver>
% \fi
%
% \CheckSum{1429}
%
% \CharacterTable
%  {Upper-case    \A\B\C\D\E\F\G\H\I\J\K\L\M\N\O\P\Q\R\S\T\U\V\W\X\Y\Z
%   Lower-case    \a\b\c\d\e\f\g\h\i\j\k\l\m\n\o\p\q\r\s\t\u\v\w\x\y\z
%   Digits        \0\1\2\3\4\5\6\7\8\9
%   Exclamation   \!     Double quote  \"     Hash (number) \#
%   Dollar        \$     Percent       \%     Ampersand     \&
%   Acute accent  \'     Left paren    \(     Right paren   \)
%   Asterisk      \*     Plus          \+     Comma         \,
%   Minus         \-     Point         \.     Solidus       \/
%   Colon         \:     Semicolon     \;     Less than     \<
%   Equals        \=     Greater than  \>     Question mark \?
%   Commercial at \@     Left bracket  \[     Backslash     \\
%   Right bracket \]     Circumflex    \^     Underscore    \_
%   Grave accent  \`     Left brace    \{     Vertical bar  \|
%   Right brace   \}     Tilde         \~}
%
% \GetFileInfo{luatex.drv}
%
% \title{The \xpackage{luatex} package}
% \date{2010/03/09 v0.4}
% \author{Heiko Oberdiek\\\xemail{heiko.oberdiek at googlemail.com}}
%
% \maketitle
%
% \begin{abstract}
% This package manages the new and extended features and resources
% that \LuaTeX\ provides. Examples are attributes and catcode
% tables.
% \end{abstract}
%
% \tableofcontents
%
% \section{Documentation}
%
% \subsection{Introduction}
%
% \TeX\ provides global resources such as registers.
% But it does not provide an interface for managing these resources.
% For example, two packages want to use a counter register.
% If they take the same register
% number, then the use of both packages will conflict and they cannot
% be used together.
% Therefore formats such as \plainTeX\ or \LaTeX\ implement
% an allocation scheme for registers. A package reserves with
% \cs{newcount} an unused register number for its own exclusive use.
%
% Nowadays \TeX\ is not alone anymore: \eTeX, \pdfTeX\ and other
% compilers for \TeX\ are developed that extend and add new
% features and resources.
%
% Now \LuaTeX\ has reached beta state. It inherits most of
% \pdfTeX's features including \eTeX. Also it implements
% new concepts such as attributes or catcode tables.
%
% \subsubsection{\LaTeX}
%
% \LaTeXe\ is frozen and therefore refuses to even notice
% the new \TeX\ variants. Not even the old \eTeX\ is supported
% by its kernel. At least there is a third party package \xpackage{etex}
% that manages the new \eTeX\ resources.
%
% This package tries to do the same for \LuaTeX\ and starts to
% support at least a few of the new features.
%
% \subsubsection{\plainTeX}
%
% \LaTeX\ has inherited its resource handling from \plainTeX.
% The interface is basically the same: \cs{newcount}, \dots
% Therefore this package tries to follow this tradition by
% providing compatibility to \plainTeX. It can be loaded
% with \plainTeX\ and defines at least some of the
% features that this packages provides for \LaTeX.
%
% \subsection{Register allocation}
%
% \subsubsection{Register with 16 bit}
%
% Because \LuaTeX\ is a super set of \eTeX\ regarding registers,
% the register allocation scheme should not conflict with
% package \xpackage{etex}. Therefore this package is loaded
% to inherit its allocation scheme. The only change is currently
% that the limit is increased to 65536 registers for the following
% register classes:
% \begin{compactitem}
% \item count
% \item dimen
% \item skip
% \item muskip
% \item marks
% \item toks
% \item box
% \end{compactitem}
% This affects the number of global and local registers.
% Because it is done in a package and not in the kernel,
% it is possible that someone loads package \xpackage{etex}
% before uses the local allocation variants. This will prevent
% the extension for this register class. If more registers are
% needed, just load package \xpackage{luatex} earlier.
%
% \subsubsection{Insertions}
%
% Insertions need four registers \cs{count}, \cs{dimen}, \cs{skip},
% and \cs{box} with the same number. Usually they are allocated
% downwards from 254, 253, \dots Also \cs{newcount}, \cs{newdimen}, \dots
% fill up these register numbers from below before switching to
% higher register numbers by package \xpackage{etex}.
% When this occurs, no insertions can be allocated anymore.
%
% Therefore \cs{newcount}, \cs{newdimen}, \cs{newskip}, and
% \cs{newbox} are replaced by their global variants (\cs{globcount}, \dots)
% that use the higher numbers immediately, leaving the room for insertions.
% There should not be an efficiency penalty because \LuaTeX\ stores
% the registers of a class in the same Lua table unlike \eTeX, where
% registers below 256 are stored in an array and higher numbers
% are put in a tree structure.
%
% \subsection{Attributes}
%
% Nodes can have custom attributes in \LuaTeX. These attributes
% are organized by a new register class. As the other registers
% up to $2^{16}$ attributes are supported. An attribute value
% can be negative that means the attribute is not set. Otherwise
% \TeX's range of non-negative integers up to $2^31$ are available.
%
% \begin{declcs}{newattribute} \M{cmd}
% \end{declcs}
% Macro \cs{newattribute} defines command \meta{cmd} using \cs{attributedef}
% using an new attribute number. The new attribute is initially unset.
%
% \begin{declcs}{setattribute} \M{cmd} \M{value}
% \end{declcs}
% Macro \cs{setattribute} locally sets attribute command \meta{cmd}
% to the number \meta{value}. Valid values range from $-1$ until $2^31$
% (the upper limit is the same as for other \TeX\ integer numbers).
%
% \begin{declcs}{unsetattribute} \M{cmd}
% \end{declcs}
% Macro \cs{unsettattribute} clears the attribute command \meta{cmd}.
%
% \subsection{Catcode tables}
%
% \LuaTeX\ introduces catcode tables as new feature, see documentation.
% There is need for discussion, how to deal best:
% \begin{itemize}
% \item \cs{initcatcodetable} and \cs{setcatcodetable} act globally.
% \item \cs{catcodetable} causes an error if used with an uninitialized
%   catcode table.
% \item Large catcode table numbers should be avoided because of
%   performance breakdown.
% \item Use case \LaTeX\ package:
%   The package must not be surprised by changed catcodes and
%   must not surprise by changing catcodes accidently.
%   Catcode tables could offer a solution. At the begin
%   a catcode regime with standard catcodes is established
%   and the old one is restored afterwards.
% \item Use case: \LuaTeX's |tex.print| might be used with
%   a catcode table number, for example a table where all
%   entries have catcode ``other''.
% \item Readonly catcode tables.
% \item Is there is a need for local allocations?
%   (Package \xpackage{etex}'s \cs{loc} variants are not used in
%   \TeX\ Live 2007.)
% \end{itemize}
%
% \subsubsection{Interface proposal}
%
% The idea: \cs{newcatcodetable} allocates odd numbered catcode
% tables. Even numbered tables are managed as stack.
% Also some catcode tables are defined. These must not be
% changed.
%
% \begin{declcs}{newcatcodetable} \M{cmd}
% \end{declcs}
% Macro \cs{newcatcodetable} reserves a new catcode table and
% remembers its number in \meta{cmd}. The catcode table is
% initialized with \iniTeX's catcodes.
%
% \begin{declcs}{CatcodeTableIniTeX}\\
%   \cs{CatcodeTableString}\\
%   \cs{CatcodeTableOther}\\
%   \cs{CatcodeTableLaTeX}\\
% \end{declcs}
% These are catcode tables and must not be changed.
% \cs{CatcodeTableIniTeX} contains the catcode settings of \iniTeX.
% \cs{CatcodeTableString} follows \TeX's convention of
% \cs{string}, \cs{meaning} and friends. The space gets catcode
% 10 (space), the other characters have catcode 12 (other).
% In \cs{CatcodeTableOther} all entries have catcode 12 (other).
% \cs{CatcodeTableLaTeX} contains the setting of a pure \LaTeX\ format
% (`at' is other).
%
% \begin{declcs}{CatcodeTableStack}\\
%   \cs{IncCatcodeTableStack}\\
%   \cs{DecCatcodeTableStack}
% \end{declcs}
% \cs{CatcodeTableStack} is the stack pointer. Initially
% it is catcode table zero. \cs{IncCatcodeTableStack} and
% \cs{DecCatcodeTableStack} increments and decrements the
% stack pointer. Currently \cs{IncCatcodeTableStack} does not
% initialize a new catcode table. Both increment and decrement
% operations do not set a catcode table.
%
% \begin{declcs}{PushCatcodeTableNumStack}\\
%   \cs{PopCatcodeTableNumStack}
% \end{declcs}
% It can be handy to have a global stack for catcode table
% numbers to deal with the global assignment property of
% \cs{initcatcodetable} and \cs{savecatcodetable}.
% \cs{PushCatcodeTableNumStack} pushes the current catcode table
% on the stack.
% \cs{PopCatcodeTableNumStack} pops the topmost number off the number
% stack to set the current catcode table. Catcode table zero is
% used in case of an empty stack.
%
% \begin{declcs}{BeginCatcodeRegime} \M{catcodetable}\\
%   \cs{EndCatcodeRegime}
% \end{declcs}
% \cs{BeginCatcodeRegime} remembers the current catcode table number.
% Then it creates and uses a fresh catcode table on the stack
% that is initialized by \meta{catcodetable}:
% \begin{quote}\ttfamily
%   \cs{PushCatcodeTableNumStack}\\
%   \cs{catcodetable}\meta{catcodetable}||
%   \cs{IncCatcodeTableStack}\\
%   \cs{savecatcodetable}\cs{CatcodeTableStack}\\
%   \cs{catcodetable}\cs{CatcodeTableStack}
% \end{quote}
% \cs{EndCatcodeRegime} drops the catcode table, created by
% \cs{BeginCatcodeRegime} and sets the catcode table that was
% active before:
% \begin{quote}
%   \cs{DecCatcodeTableStack}\\
%   \cs{PopCatcodeTableNumStack}
% \end{quote}
% These macros solve the use case, described earlier for a \LaTeX\ package:
% \begin{quote}
%   |% package foobar.sty|\\
%   |\BeginCatcodeRegime\CatcodeTableLaTeX|\\
%   |\makeatletter|\\
%   |% ... package contents ...|\\
%   |\EndCatcodeRegime|\\
%   |% end of package|
% \end{quote}
% If the package wants to change catcodes after its loading,
% \cs{AtBeginDocument} or \cs{AtEndOfPackage} can be used.
%
% \begin{declcs}{SetCatcodeRange} \M{from} \M{to} \M{catcode}
% \end{declcs}
% The catcodes of characters in range from \meta{from}
% to inclusive \meta{to} are set to \meta{catcode}.
%
% \subsection{Lua module loading}
%
% Currently \LuaTeX\ (version 0.20) does not support Lua script
% files inside \xfile{TDS:scripts//}, because Lua's mechanism
% for module loading does not use the \textsf{kpathsea} library.
% Therefore this packages appends a kpse loader to the list of
% Lua's module loaders. It finds the module \meta{module} by
% \begin{quote}
%   \ttfamily |kpse.find_file("|\meta{module}|.lua", "texmfscripts")|
% \end{quote}
% Unhappily \textsf{kpathsea} does not support directory components
% in a file name. Therefore the Lua convention is not followed to
% replace dots in the module name by the directory separator.
%
% Example:
% A Lua script of a package \xpackage{foobar} wants the following
% modules:
% \begin{quote}
%   |require("foobar.hello.world")|\\
%   |require("org.somewhere.xyz")|
% \end{quote}
% Then they can be find in:
% \begin{quote}
%   |TDS:scripts/foobar/foobar.hello.world.lua|\\
%   |TDS:scripts/foobar/org.somewhere.xyz.lua|
% \end{quote}
% I would have preferred the following locations, following lua conventions,
% e.\,g.:
% \begin{quote}
%   |TDS:scripts/foobar/hello/world.lua|\\
%   |TDS:scripts/foobar/org/somewhere/xyz.lua|
% \end{quote}
% But I do not know, how to achieve this in a reliable way using
% \textsf{kpathsea}.
%
% \subsubsection{Package \xpackage{luatex-loader}}
%
% If someone do not need or want package \xpackage{luatex} but
% it's extension for module loading, then he can use
% package \xpackage{luatex-loader}. Both \plainTeX\ and \LaTeX\ are
% supported.
%
% \StopEventually{
% }
%
% \section{Implementation}
%
%    \begin{macrocode}
%<*package>
%    \end{macrocode}
%
% \subsection{Reload check and package identification}
%    Reload check, especially if the package is not used with \LaTeX.
%    \begin{macrocode}
\begingroup\catcode61\catcode48\catcode32=10\relax%
  \catcode13=5 % ^^M
  \endlinechar=13 %
  \catcode35=6 % #
  \catcode39=12 % '
  \catcode44=12 % ,
  \catcode45=12 % -
  \catcode46=12 % .
  \catcode58=12 % :
  \catcode64=11 % @
  \catcode123=1 % {
  \catcode125=2 % }
  \expandafter\let\expandafter\x\csname ver@luatex.sty\endcsname
  \ifx\x\relax % plain-TeX, first loading
  \else
    \def\empty{}%
    \ifx\x\empty % LaTeX, first loading,
      % variable is initialized, but \ProvidesPackage not yet seen
    \else
      \expandafter\ifx\csname PackageInfo\endcsname\relax
        \def\x#1#2{%
          \immediate\write-1{Package #1 Info: #2.}%
        }%
      \else
        \def\x#1#2{\PackageInfo{#1}{#2, stopped}}%
      \fi
      \x{luatex}{The package is already loaded}%
      \aftergroup\endinput
    \fi
  \fi
\endgroup%
%    \end{macrocode}
%    Package identification:
%    \begin{macrocode}
\begingroup\catcode61\catcode48\catcode32=10\relax%
  \catcode13=5 % ^^M
  \endlinechar=13 %
  \catcode35=6 % #
  \catcode39=12 % '
  \catcode40=12 % (
  \catcode41=12 % )
  \catcode44=12 % ,
  \catcode45=12 % -
  \catcode46=12 % .
  \catcode47=12 % /
  \catcode58=12 % :
  \catcode64=11 % @
  \catcode91=12 % [
  \catcode93=12 % ]
  \catcode123=1 % {
  \catcode125=2 % }
  \expandafter\ifx\csname ProvidesPackage\endcsname\relax
    \def\x#1#2#3[#4]{\endgroup
      \immediate\write-1{Package: #3 #4}%
      \xdef#1{#4}%
    }%
  \else
    \def\x#1#2[#3]{\endgroup
      #2[{#3}]%
      \ifx#1\@undefined
        \xdef#1{#3}%
      \fi
      \ifx#1\relax
        \xdef#1{#3}%
      \fi
    }%
  \fi
\expandafter\x\csname ver@luatex.sty\endcsname
\ProvidesPackage{luatex}%
  [2010/03/09 v0.4 LuaTeX basic definition package (HO)]%
%    \end{macrocode}
%
% \subsection{Catcodes}
%
%    \begin{macrocode}
\begingroup\catcode61\catcode48\catcode32=10\relax%
  \catcode13=5 % ^^M
  \endlinechar=13 %
  \catcode123=1 % {
  \catcode125=2 % }
  \catcode64=11 % @
  \def\x{\endgroup
    \expandafter\edef\csname LuT@AtEnd\endcsname{%
      \endlinechar=\the\endlinechar\relax
      \catcode13=\the\catcode13\relax
      \catcode32=\the\catcode32\relax
      \catcode35=\the\catcode35\relax
      \catcode61=\the\catcode61\relax
      \catcode64=\the\catcode64\relax
      \catcode123=\the\catcode123\relax
      \catcode125=\the\catcode125\relax
    }%
  }%
\x\catcode61\catcode48\catcode32=10\relax%
\catcode13=5 % ^^M
\endlinechar=13 %
\catcode35=6 % #
\catcode64=11 % @
\catcode123=1 % {
\catcode125=2 % }
\def\TMP@EnsureCode#1#2{%
  \edef\LuT@AtEnd{%
    \LuT@AtEnd
    \catcode#1=\the\catcode#1\relax
  }%
  \catcode#1=#2\relax
}
\TMP@EnsureCode{10}{12}% ^^J
\TMP@EnsureCode{34}{12}% "
\TMP@EnsureCode{36}{3}% $
\TMP@EnsureCode{39}{12}% '
\TMP@EnsureCode{40}{12}% (
\TMP@EnsureCode{41}{12}% )
\TMP@EnsureCode{42}{12}% *
\TMP@EnsureCode{43}{12}% +
\TMP@EnsureCode{44}{12}% ,
\TMP@EnsureCode{45}{12}% -
\TMP@EnsureCode{46}{12}% .
\TMP@EnsureCode{47}{12}% /
\TMP@EnsureCode{60}{12}% <
\TMP@EnsureCode{62}{12}% >
\TMP@EnsureCode{91}{12}% [
\TMP@EnsureCode{93}{12}% ]
\TMP@EnsureCode{95}{12}% _ (other!)
\TMP@EnsureCode{96}{12}% `
\edef\LuT@AtEnd{\LuT@AtEnd\noexpand\endinput}
%    \end{macrocode}
%
% \subsection{Check for \LuaTeX}
%
%    Without \LuaTeX\ there is no point in using this package.
%    \begin{macrocode}
\begingroup\expandafter\expandafter\expandafter\endgroup
\expandafter\ifx\csname RequirePackage\endcsname\relax
  \input infwarerr.sty\relax
  \input ifluatex.sty\relax
\else
  \RequirePackage{infwarerr}[2007/09/09]%
  \RequirePackage{ifluatex}[2009/04/10]%
\fi
%    \end{macrocode}
%    \begin{macrocode}
\ifluatex
\else
  \@PackageError{luatex}{%
    This package may only be run using LuaTeX%
  }\@ehc
  \expandafter\LuT@AtEnd
\fi%
%    \end{macrocode}
%
% \subsection{Provide Lua\TeX\ primitives}
%
%    \begin{macrocode}
\ifnum\luatexversion<36 %
  \def\LuT@MakePrimitive#1{%
    \expandafter\let\csname luatex#1\expandafter\endcsname
    \csname #1\endcsname
  }%
\else
  \def\LuT@MakeLuatexPrimitive#1{%
    \begingroup\expandafter\expandafter\expandafter\endgroup
    \expandafter\ifx\csname luatex#1\endcsname\relax
      \begingroup\expandafter\expandafter\expandafter\endgroup
      \expandafter\ifx\csname #1\endcsname\relax
      \else
        \expandafter\let
        \csname luatex#1\expandafter\endcsname
        \csname #1\endcsname
      \fi
    \fi
    \begingroup\expandafter\expandafter\expandafter\endgroup
    \expandafter\ifx\csname luatex#1\endcsname\relax
      \begingroup
        \expandafter\let\csname luatex#1\endcsname\@undefined
        \ifnum0%
          \directlua{%
            if tex.enableprimitives then %
              tex.enableprimitives('luatex',{'#1'})%
              tex.print('1')%
            end%
          }%
          \expandafter\ifx\csname luatex#1\endcsname\relax\else1\fi
        =11 %
          \global\expandafter\let
          \csname luatex#1\expandafter\endcsname
          \csname luatex#1\endcsname
        \else
          \@PackageError{luatex}{%
            tex.enableprimitives failed for `#1'%
          }\@ehc
        \fi
      \endgroup
    \fi
  }%
  \def\LuT@MakePrimitive#1{%
    \begingroup\expandafter\expandafter\expandafter\endgroup
    \expandafter\ifx\csname#1\endcsname\relax
      \begingroup
        \expandafter\let\csname#1\endcsname\@undefined
        \ifnum0%
          \directlua{%
            if tex.enableprimitives then %
              tex.enableprimitives('',{'#1'})%
              tex.print('1')%
            end%
          }%
          \expandafter\ifx\csname#1\endcsname\relax\else1\fi
        =11 %
          \global\expandafter\let
          \csname#1\expandafter\endcsname
          \csname#1\endcsname
        \else
          \@PackageError{luatex}{%
            tex.enableprimitives failed for `#1'%
          }\@ehc
        \fi
      \endgroup
    \fi
  }%
\fi
\LuT@MakeLuatexPrimitive{attribute}
\LuT@MakeLuatexPrimitive{attributedef}
\LuT@MakeLuatexPrimitive{catcodetable}
\LuT@MakeLuatexPrimitive{initcatcodetable}
\LuT@MakeLuatexPrimitive{luaescapestring}
\LuT@MakeLuatexPrimitive{savecatcodetable}
\LuT@MakePrimitive{numexpr}
%    \end{macrocode}
%
% \subsection{Inherit support for \eTeX}
%
%    Package \xpackage{etex} is not compatible for \plainTeX.
%    But it could be present if a format is used that is based
%    on \xfile{etex.src}. Therefore we only load the package
%    in case of \LaTeX\ and tests its presence independently
%    of the format by looking for \cs{et@xins}.
%    \begin{macrocode}
\begingroup\expandafter\expandafter\expandafter\endgroup
\expandafter\ifx\csname RequirePackage\endcsname\relax
\else
  \RequirePackage{etex}[1998/03/26]%
\fi
%    \end{macrocode}
%
% \subsection{Adaption of \eTeX's register allocation}
%
%    \eTeX has increased the number of \TeX\ registers
%    from $2^8$ (256) to $2^{15}$ (32768) for a register class.
%    \LuaTeX\ extends the limit further to $2^16$ (65536).
%    The allocation scheme of package \xpackage{etex} is not
%    changed. But this can be subject for discussion.
%
%    If a register class hasn't registered any local registers
%    yet, then the limit can safely be pushed to 65536.
%    \begin{macrocode}
\begingroup\expandafter\expandafter\expandafter\endgroup
\expandafter\ifx\csname et@xins\endcsname\relax
  \@PackageWarningNoLine{luatex}{%
    Support for eTeX is not loaded (etex.src)%
  }%
\else
  \def\LuT@temp#1{%
    \ifnum\count27#1=32768 %
      \count27#1=65536 %
    \fi
  }%
  \LuT@temp0%
  \LuT@temp1%
  \LuT@temp2%
  \LuT@temp3%
  \LuT@temp4%
  \LuT@temp5%
  \LuT@temp6%
%    \end{macrocode}
%
% \eTeX\ uses an array for the first 256 registers and then a tree
% structure. \LuaTeX\ stores all registers of a class in one Lua table.
% There shouldn't be large performance differences.
% This allows starting immediately in the extended
% area, leaving room for insertions.
%    \begin{macrocode}
  \let\newcount\globcount
  \let\newdimen\globdimen
  \let\newskip\globskip
  \let\newbox\globbox
\fi
%    \end{macrocode}
%
% \subsection{\plainTeX\ compatibility}
%
%    \begin{macro}{\@empty}
%    \begin{macrocode}
\expandafter\ifx\csname @empty\endcsname\relax
  \def\@empty{}%
\fi
%    \end{macrocode}
%    \end{macro}
%    \begin{macro}{\@gobble}
%    \begin{macrocode}
\expandafter\ifx\csname @gobble\endcsname\relax
  \long\def\@gobble#1{}%
\fi
%    \end{macrocode}
%    \end{macro}
%    \begin{macro}{\@firstofone}
%    \begin{macrocode}
\expandafter\ifx\csname @firstofone\endcsname\relax
  \long\def\@firstofone#1{#1}%
\fi
%    \end{macrocode}
%    \end{macro}
%    \begin{macro}{\@firstoftwo}
%    \begin{macrocode}
\expandafter\ifx\csname @firstoftwo\endcsname\relax
  \long\def\@firstoftwo#1#2{#1}%
\fi
%    \end{macrocode}
%    \end{macro}
%    \begin{macro}{\@car}
%    \begin{macrocode}
\expandafter\ifx\csname @car\endcsname\relax
  \def\@car#1#2\@nil{#1}%
\fi
%    \end{macrocode}
%    \end{macro}
%    \begin{macro}{\@cdr}
%    \begin{macrocode}
\expandafter\ifx\csname @cdr\endcsname\relax
  \def\@cdr#1#2\@nil{#2}%
\fi
%    \end{macrocode}
%    \end{macro}
%    \begin{macro}{\@ifstar}
%    \begin{macrocode}
\expandafter\ifx\csname @ifstar\endcsname\relax
  \def\@ifstar#1{%
    \@ifnextchar*{\@firstoftwo{#1}}%
  }%
%    \end{macrocode}
%    \end{macro}
%    \begin{macro}{\@ifnextchar}
%    \begin{macrocode}
  \long\def\@ifnextchar#1#2#3{%
    \let\reserved@d=#1%
    \def\reserved@a{#2}%
    \def\reserved@b{#3}%
    \futurelet\@let@token\@ifnch
  }%
%    \end{macrocode}
%    \end{macro}
%    \begin{macro}{\@ifnch}
%    \begin{macrocode}
  \def\@ifnch{%
    \ifx\@let@token\@sptoken
      \let\reserved@c\@xifnch
    \else
      \ifx\@let@token\reserved@d
        \let\reserved@c\reserved@a
      \else
        \let\reserved@c\reserved@b
      \fi
    \fi
    \reserved@c
  }%
%    \end{macrocode}
%    \end{macro}
%    \begin{macro}{\@sptoken}
%    \begin{macrocode}
  \let\LuT@temp\:%
  \def\:{\let\@sptoken= }%
  \: % explicit space
%    \end{macrocode}
%    \end{macro}
%    \begin{macro}{\@xifnch}
%    \begin{macrocode}
  \def\:{\@xifnch}%
  \expandafter\def\: {%
    \futurelet\@let@token\@ifnch
  }%
  \let\:\LuT@temp
\fi
%    \end{macrocode}
%    \end{macro}
%
%    \begin{macro}{\@tempcnta}
%    \begin{macrocode}
\expandafter\ifx\csname @tempcnta\endcsname\relax
  \csname newcount\endcsname\@tempcnta
\fi
%    \end{macrocode}
%    \end{macro}
%
%    \begin{macro}{\@tempcntb}
%    \begin{macrocode}
\expandafter\ifx\csname @tempcntb\endcsname\relax
  \csname newcount\endcsname\@tempcntb
\fi
%    \end{macrocode}
%    \end{macro}
%
%    \begin{macro}{\LuT@newcommand}
%    \begin{macrocode}
\begingroup\expandafter\expandafter\expandafter\endgroup
\expandafter\ifx\csname newcommand\endcsname\relax
  \def\LuT@newcommand#1[#2]#3{%
    \ifx#1\@undefined
      \let#1\relax
    \else
      \ifx#1\relax
      \else
        \@PackageError{luatex}{%
          \string#1 is already defined.\MessageBreak
          Redefinition is skipped%
        }\@ehc
      \fi
    \fi
    \ifx#1\relax
      \ifcase#2 %
        \def#1{#3}%
      \or
        \def#1##1{#3}%
      \or
        \def#1##1##2{#3}%
      \or
        \def#1##1##2##3{#3}%
      \or
        \@INTERNAL@ERROR
      \fi
    \fi
  }%
\else
  \def\LuT@newcommand{\newcommand*}%
\fi
%    \end{macrocode}
%    \end{macro}
%
% \subsection{Attributes}
%
% \subsubsection{Allocation}
%
%    \begin{macro}{\LuT@AllocAttribute}
%    \begin{macrocode}
\newcount\LuT@AllocAttribute
\LuT@AllocAttribute=\m@ne
%    \end{macrocode}
%    \end{macro}
%    \begin{macro}{\newattribute}
%    \begin{macrocode}
\LuT@newcommand\newattribute[1]{%
  \ifnum\LuT@AllocAttribute<65535 %
    \global\advance\LuT@AllocAttribute\@ne
    \allocationnumber\LuT@AllocAttribute
    \global\luatexattributedef#1=\allocationnumber
    \unsetattribute{#1}%
    \wlog{\string#1=\string\attribute\the\allocationnumber}%
  \else
    \errmessage{No room for a new \string\attribute}%
  \fi
}
%    \end{macrocode}
%    \end{macro}
%
% \subsubsection{Interface}
%
%    \begin{macro}{\setattribute}
%    \begin{macrocode}
\LuT@newcommand\setattribute[2]{%
  #1=\numexpr#2\relax
}
%    \end{macrocode}
%    \end{macro}
%    \begin{macro}{\unsetattribute}
%    \begin{macrocode}
\ifnum\luatexversion<37
  \LuT@newcommand\LuT@UnsetAttributeValue[0]{}%
  \let\LuT@UnsetAttributeValue\m@ne
\else
  \LuT@newcommand\LuT@UnsetAttributeValue[0]{-2147483647 }%
\fi
\LuT@newcommand\unsetattribute[1]{%
  #1=\LuT@UnsetAttributeValue
}
%    \end{macrocode}
%    \end{macro}
%
% \subsection{Catcode tables}
%
% \subsubsection{Allocation}
%
%    \begin{macro}{\LuT@AllocCatcodeTable}
%    \begin{macrocode}
\newcount\LuT@AllocCatcodeTable
\LuT@AllocCatcodeTable=\m@ne
\newcount\CatcodeTableStack
\CatcodeTableStack=\z@
%    \end{macrocode}
%    \end{macro}
%    \begin{macro}{\newcatcodetable}
%    \begin{macrocode}
\LuT@newcommand\newcatcodetable[1]{%
  \ifnum\LuT@AllocCatcodeTable<1114110 % 0x10FFFF is maximal \chardef
    % or < 268435455 % 2^28 - 1
    \global\advance\LuT@AllocCatcodeTable by\tw@
    \allocationnumber=\LuT@AllocCatcodeTable
    \global\chardef#1=\allocationnumber
    \wlog{%
      \string#1=\string\catcodetable\the\allocationnumber
    }%
  \else
    \errmessage{No room for a new \string\catcodetable}%
  \fi
}%
%    \end{macrocode}
%    \end{macro}
%
%    \begin{macro}{\IncCatcodeTableStack}
%    \begin{macrocode}
\LuT@newcommand\IncCatcodeTableStack[0]{%
  \ifnum\CatcodeTableStack<268435454 %
    \global\advance\CatcodeTableStack by\tw@
  \else
    \@PackageError{luatex}{%
      Catcode table stack overflow%
    }\@ehd
  \fi
}
%    \end{macrocode}
%    \end{macro}
%
%    \begin{macro}{\DecCatcodeTableStack}
%    \begin{macrocode}
\LuT@newcommand\DecCatcodeTableStack[0]{%
  \ifnum\CatcodeTableStack>\z@
    \global\advance\CatcodeTableStack by-2 %
  \else
    \@PackageError{luatex}{%
      Catcode table stack is empty%
    }\@ehd
  \fi
}
%    \end{macrocode}
%    \end{macro}
%
% \subsubsection{\cs{SetCatcodeRange}}
%
%    \begin{macro}{\SetCatcodeRange}
%    \begin{macrocode}
\LuT@newcommand\SetCatcodeRange[3]{%
  \edef\LuT@temp{%
    \noexpand\@tempcnta=\the\@tempcnta
    \noexpand\@tempcntb=\the\@tempcntb
    \noexpand\count@=\the\count@
    \relax
  }%
  \@tempcnta=\numexpr#1\relax
  \@tempcntb=\numexpr#2\relax
  \count@=\numexpr#3\relax
  \loop
    \unless\ifnum\@tempcnta>\@tempcntb
    \catcode\@tempcnta=\count@
    \advance\@tempcnta by \@ne
  \repeat
  \LuT@temp
}
%    \end{macrocode}
%    \end{macro}
%
% \subsubsection{Predefined catcode tables}
%
%    \begin{macrocode}
\newcatcodetable\CatcodeTableIniTeX
\newcatcodetable\CatcodeTableString
\newcatcodetable\CatcodeTableOther
\newcatcodetable\CatcodeTableLaTeX
%    \end{macrocode}
%
%    \begin{macrocode}
\luatexinitcatcodetable\CatcodeTableIniTeX
\begingroup
  \def\@makeother#1{\catcode#1=12\relax}%
  \@firstofone{%
    \luatexcatcodetable\CatcodeTableIniTeX
    \begingroup
      \SetCatcodeRange{0}{8}{15}%
      \catcode9=10 % tab
      \catcode11=15 %
      \catcode12=13 % form feed
      \SetCatcodeRange{14}{31}{15}%
      \catcode35=6 % hash
      \catcode36=3 % dollar
      \catcode38=4 % ampersand
      \catcode94=7 % circumflex
      \catcode95=8 % underscore
      \catcode123=1 % brace left
      \catcode125=2 % brace right
      \catcode126=13 % tilde
      \catcode127=15 %
      \luatexsavecatcodetable\CatcodeTableLaTeX
    \endgroup
    \@makeother{0}% nul
    \@makeother{13}% carriage return
    \@makeother{37}% percent
    \@makeother{92}% backslash
    \@makeother{127}%
    \SetCatcodeRange{65}{90}{12}% A-Z
    \SetCatcodeRange{97}{122}{12}% a-z
    \luatexsavecatcodetable\CatcodeTableString
    \@makeother{32}% space
    \luatexsavecatcodetable\CatcodeTableOther
  \endgroup
}%
%    \end{macrocode}
%
% \subsubsection{Number stack}
%
%    \begin{macro}{\LuT@NumStackEmpty}
%    A special empty stack value because of \cs{@cdr}'s brace removal.
%    \begin{macrocode}
\def\LuT@NumStackEmpty{0}
%    \end{macrocode}
%    \end{macro}
%    \begin{macro}{\LuT@NumStack}
%    \begin{macrocode}
\let\LuT@NumStack\LuT@NumStackEmpty
%    \end{macrocode}
%    \end{macro}
%    \begin{macro}{\PushCatcodeTableNumStack}
%    \begin{macrocode}
\LuT@newcommand\PushCatcodeTableNumStack[0]{%
  \xdef\LuT@NumStack{%
    {\the\luatexcatcodetable}\LuT@NumStack
  }%
}
%    \end{macrocode}
%    \end{macro}
%    \begin{macro}{\PopCatcodeTableNumStack}
%    \begin{macrocode}
\LuT@newcommand\PopCatcodeTableNumStack[0]{%
  \ifx\LuT@NumStack\LuT@NumStackEmpty
    \@PackageWarning{luatex}{Empty catcode table number stack}%
    \luatexcatcodetable\z@
  \else
    \luatexcatcodetable=\expandafter\@car\LuT@NumStack\@nil\relax
    \xdef\LuT@NumStack{%
      \expandafter\@cdr\LuT@NumStack\@nil
    }%
  \fi
}
%    \end{macrocode}
%    \end{macro}
%
% \subsubsection{Catcode regime macros}
%
%    \begin{macro}{\BeginCatcodeRegime}
%    \begin{macrocode}
\LuT@newcommand\BeginCatcodeRegime[1]{%
  \PushCatcodeTableNumStack
  \luatexcatcodetable=\numexpr#1\relax
  \IncCatcodeTableStack
  \luatexsavecatcodetable\CatcodeTableStack
  \luatexcatcodetable\CatcodeTableStack
}
%    \end{macrocode}
%    \end{macro}
%
%    \begin{macro}{\EndCatcodeRegime}
%    \begin{macrocode}
\LuT@newcommand\EndCatcodeRegime[0]{%
  \DecCatcodeTableStack
  \PopCatcodeTableNumStack
}
%    \end{macrocode}
%    \end{macro}
%
% \subsection{Lua module loader}
%
%    \begin{macrocode}
\begingroup\expandafter\expandafter\expandafter\endgroup
\expandafter\ifx\csname RequirePackage\endcsname\relax
  \input luatex-loader.sty\relax
\else
  \RequirePackage{luatex-loader}[2010/03/09]%
\fi
%    \end{macrocode}
%    \begin{macrocode}
\LuT@AtEnd%
%</package>
%    \end{macrocode}
%
%    \begin{macrocode}
%<*loader>
%    \end{macrocode}
%
%    Reload check, especially if the package is not used with \LaTeX.
%    \begin{macrocode}
\begingroup\catcode61\catcode48\catcode32=10\relax%
  \catcode13=5 % ^^M
  \endlinechar=13 %
  \catcode35=6 % #
  \catcode39=12 % '
  \catcode44=12 % ,
  \catcode45=12 % -
  \catcode46=12 % .
  \catcode58=12 % :
  \catcode64=11 % @
  \catcode123=1 % {
  \catcode125=2 % }
  \expandafter\let\expandafter\x\csname ver@luatex-loader.sty\endcsname
  \ifx\x\relax % plain-TeX, first loading
  \else
    \def\empty{}%
    \ifx\x\empty % LaTeX, first loading,
      % variable is initialized, but \ProvidesPackage not yet seen
    \else
      \expandafter\ifx\csname PackageInfo\endcsname\relax
        \def\x#1#2{%
          \immediate\write-1{Package #1 Info: #2.}%
        }%
      \else
        \def\x#1#2{\PackageInfo{#1}{#2, stopped}}%
      \fi
      \x{luatex-loader}{The package is already loaded}%
      \aftergroup\endinput
    \fi
  \fi
\endgroup%
%    \end{macrocode}
%    Package identification:
%    \begin{macrocode}
\begingroup\catcode61\catcode48\catcode32=10\relax%
  \catcode13=5 % ^^M
  \endlinechar=13 %
  \catcode35=6 % #
  \catcode39=12 % '
  \catcode40=12 % (
  \catcode41=12 % )
  \catcode44=12 % ,
  \catcode45=12 % -
  \catcode46=12 % .
  \catcode47=12 % /
  \catcode58=12 % :
  \catcode64=11 % @
  \catcode91=12 % [
  \catcode93=12 % ]
  \catcode123=1 % {
  \catcode125=2 % }
  \expandafter\ifx\csname ProvidesPackage\endcsname\relax
    \def\x#1#2#3[#4]{\endgroup
      \immediate\write-1{Package: #3 #4}%
      \xdef#1{#4}%
    }%
  \else
    \def\x#1#2[#3]{\endgroup
      #2[{#3}]%
      \ifx#1\@undefined
        \xdef#1{#3}%
      \fi
      \ifx#1\relax
        \xdef#1{#3}%
      \fi
    }%
  \fi
\expandafter\x\csname ver@luatex-loader.sty\endcsname
\ProvidesPackage{luatex-loader}%
  [2010/03/09 v0.4 Lua module loader (HO)]%
%    \end{macrocode}
%    \begin{macrocode}
\begingroup\catcode61\catcode48\catcode32=10\relax%
  \catcode13=5\endlinechar=13\relax%
  \catcode10=12 % ^^J
  \catcode34=12 % "
  \catcode39=12 % '
  \catcode40=12 % (
  \catcode41=12 % )
  \catcode44=12 % ,
  \catcode46=12 % .
  \catcode60=12 % <
  \catcode61=12 % =
  \catcode95=12 % _ (other!)
  \catcode96=12 % `
  \catcode123=1 % {
  \catcode125=2 % }
  \endlinechar=10 %
  \ifnum\luatexversion<36 %
    \directlua0%
  \else %
    \expandafter\directlua %
  \fi %
  {%
     do
      local script = "oberdiek.luatex.lua"
      local file = kpse.find_file(script, "texmfscripts")
      if file then
        texio.write_nl("(" .. file .. ")")
        dofile(file)
      else
        error("File `" .. script .. "' not found")
      end
    end
  }%
\endgroup\endinput%
%    \end{macrocode}
%
%    \begin{macrocode}
%</loader>
%    \end{macrocode}
%
% \subsection{Lua script}
%
%    Currently \LuaTeX\ does not use KPSE when searching for module
%    files. The following Lua script implements a workaround.
%    It extends |package.loader| by another search method.
%    Modules are found by the module name with extension \xext{lua}
%    similar to
%    \begin{quote}
%      \ttfamily |kpsewhich --format=texmfscripts| \meta{module}|.lua|
%    \end{quote}
%    Unhappily \textsf{kpsewhich} does not support directory components
%    in the file name. Therefore a module |a.b.c| cannot be installed
%    as \xfile{a/b/c.lua}. The script must be named \xfile{a.b.c.lua}.
%    \begin{macrocode}
%<*lua>
%    \end{macrocode}
%    \begin{macrocode}
module("oberdiek.luatex", package.seeall)
%    \end{macrocode}
%    \begin{macrocode}
function kpse_module_loader(module)
  local script = module .. ".lua"
  local file = kpse.find_file(script, "texmfscripts")
  if file then
    local loader, error = loadfile(file)
    if loader then
      texio.write_nl("(" .. file .. ")")
      return loader
    end
    return "\n\t[oberdiek.luatex.kpse_module_loader] Loading error:\n\t"
           .. error
  end
  return "\n\t[oberdiek.luatex.kpse_module_loader] Search failed"
end
table.insert(package.loaders, kpse_module_loader)
%    \end{macrocode}
%    \begin{macrocode}
%</lua>
%    \end{macrocode}
%
% \section{Test}
%
%    \begin{macrocode}
%<*test2>
\documentclass{article}
\def\LoadCommand{%
  \RequirePackage{luatex}[2010/03/09]%
}
%</test2>
%    \end{macrocode}
%    \begin{macrocode}
%<*test3>
\documentclass{article}
\def\LoadCommand{%
  \RequirePackage{luatex-loader}[2010/03/09]%
}
%</test3>
%    \end{macrocode}
% \subsection{Catcode checks for loading}
%
%    \begin{macrocode}
%<*test1>
%    \end{macrocode}
%    \begin{macrocode}
\catcode`\{=1 %
\catcode`\}=2 %
\catcode`\#=6 %
\catcode`\@=11 %
\expandafter\ifx\csname count@\endcsname\relax
  \countdef\count@=255 %
\fi
\expandafter\ifx\csname @gobble\endcsname\relax
  \long\def\@gobble#1{}%
\fi
\expandafter\ifx\csname @firstofone\endcsname\relax
  \long\def\@firstofone#1{#1}%
\fi
\expandafter\ifx\csname loop\endcsname\relax
  \expandafter\@firstofone
\else
  \expandafter\@gobble
\fi
{%
  \def\loop#1\repeat{%
    \def\body{#1}%
    \iterate
  }%
  \def\iterate{%
    \body
      \let\next\iterate
    \else
      \let\next\relax
    \fi
    \next
  }%
  \let\repeat=\fi
}%
\def\RestoreCatcodes{}
\count@=0 %
\loop
  \edef\RestoreCatcodes{%
    \RestoreCatcodes
    \catcode\the\count@=\the\catcode\count@\relax
  }%
\ifnum\count@<255 %
  \advance\count@ 1 %
\repeat

\def\RangeCatcodeInvalid#1#2{%
  \count@=#1\relax
  \loop
    \catcode\count@=15 %
  \ifnum\count@<#2\relax
    \advance\count@ 1 %
  \repeat
}
\def\RangeCatcodeCheck#1#2#3{%
  \count@=#1\relax
  \loop
    \ifnum#3=\catcode\count@
    \else
      \errmessage{%
        Character \the\count@\space
        with wrong catcode \the\catcode\count@\space
        instead of \number#3%
      }%
    \fi
  \ifnum\count@<#2\relax
    \advance\count@ 1 %
  \repeat
}
\def\space{ }
\expandafter\ifx\csname LoadCommand\endcsname\relax
  \def\LoadCommand{\input luatex.sty\relax}%
\fi
\def\Test{%
  \RangeCatcodeInvalid{0}{47}%
  \RangeCatcodeInvalid{58}{64}%
  \RangeCatcodeInvalid{91}{96}%
  \RangeCatcodeInvalid{123}{255}%
  \catcode`\@=12 %
  \catcode`\\=0 %
  \catcode`\%=14 %
  \LoadCommand
  \RangeCatcodeCheck{0}{36}{15}%
  \RangeCatcodeCheck{37}{37}{14}%
  \RangeCatcodeCheck{38}{47}{15}%
  \RangeCatcodeCheck{48}{57}{12}%
  \RangeCatcodeCheck{58}{63}{15}%
  \RangeCatcodeCheck{64}{64}{12}%
  \RangeCatcodeCheck{65}{90}{11}%
  \RangeCatcodeCheck{91}{91}{15}%
  \RangeCatcodeCheck{92}{92}{0}%
  \RangeCatcodeCheck{93}{96}{15}%
  \RangeCatcodeCheck{97}{122}{11}%
  \RangeCatcodeCheck{123}{255}{15}%
  \RestoreCatcodes
}
\Test
\csname @@end\endcsname
\end
%    \end{macrocode}
%    \begin{macrocode}
%</test1>
%    \end{macrocode}
%
% \subsection{Catcode tables}
%
% \subsubsection{Predefined catcode tables}
%
%    \begin{macrocode}
%<*test4>
\NeedsTeXFormat{LaTeX2e}
%    \end{macrocode}
%    Remember \LaTeX's initial catcodes in count registers
%    starting at \cs{TestLaTeX}.
%    \begin{macrocode}
\count0=0 %
\chardef\TestLaTeX=1000 %
\chardef\TestMax=300 %
\loop
  \count\numexpr\TestLaTeX+\count0\relax=\catcode\count0 %
\ifnum\count0<\TestMax
  \advance\count0 by 1 %
\repeat
\documentclass{minimal}
\usepackage{luatex}[2010/03/09]
\usepackage{qstest}
\IncludeTests{*}
\LogTests{log}{*}{*}
\makeatletter
\def\Check#1{%
  \Expect*{\the\count@=\the\catcode\count@}%
         *{\the\count@=#1}%
}
\newcount\scratch
\def\Test#1#2{%
  \begin{qstest}{CatcodeTable#1}{CatcodeTable#1}%
    \luatexcatcodetable\csname CatcodeTable#1\endcsname
    \count@=\z@
    \loop
      \scratch=#2\relax
      \Expect*{\the\count@=\the\catcode\count@}%
             *{\the\count@=\the\scratch}%
    \ifnum\count@<\TestMax
      \advance\count@\@ne
    \repeat
  \end{qstest}%
}
\begingroup
  % luatex-unicode-letters.tex makes some slots to letters
  \def\TestMax{169}%
  \Test{LaTeX}{\the\count\numexpr\TestLaTeX+\count@}%
\endgroup
\Test{String}{\ifnum\count@=32 10\else 12\fi}
\Test{Other}{12}
\luatexinitcatcodetable99 %
\Test{IniTeX}{%
  0\relax
  \begingroup
    \luatexcatcodetable99 %
    \global\scratch=\the\catcode\count@
  \endgroup
}
%    \end{macrocode}
%
% \subsubsection{Catcode table number stack}
%
%    \begin{macrocode}
\begin{qstest}{CatcodeTableNumStack}{CatcodeTableNumStack}
  \def\TestStack#1{%
    \Expect*{\LuT@NumStack}{#1}%
  }%
  \TestStack{0}%
  \PushCatcodeTableNumStack
  \TestStack{{0}0}%
  \@firstofone{%
    \begingroup
      \luatexinitcatcodetable12 %
      \luatexcatcodetable12 %
      \PushCatcodeTableNumStack
      \TestStack{{12}{0}0}%
      \PopCatcodeTableNumStack
      \TestStack{{0}0}%
      \PopCatcodeTableNumStack
      \TestStack{0}%
      \def\TestWarning{Missing empty stack warning}%
      \def\@PackageWarning#1#2{\def\TestWarning{empty stack}}%
      \PopCatcodeTableNumStack
      \TestStack{0}%
      \Expect*{\TestWarning}{empty stack}%
    \endgroup
  }%
\end{qstest}
%    \end{macrocode}
%
% \subsubsection{Catcode table stack}
%
%    \begin{macrocode}
\begin{qstest}{CatcodeTableStack}{CatcodeTableStack}
  \def\TestStack#1{%
    \Expect*{\the\CatcodeTableStack}{#1}%
  }%
  \TestStack{0}%
  \IncCatcodeTableStack
  \TestStack{2}%
  \IncCatcodeTableStack
  \TestStack{4}%
  \begingroup
    \IncCatcodeTableStack
    \TestStack{6}%
  \endgroup
  \TestStack{6}%
  \begingroup
    \DecCatcodeTableStack
    \TestStack{4}%
  \endgroup
  \TestStack{4}%
  \DecCatcodeTableStack
  \TestStack{2}%
  \DecCatcodeTableStack
  \TestStack{0}%
  \begingroup
    \def\TestError{Missing error}%
    \def\@PackageError#1#2#3{%
      \def\TestError{Empty stack}%
    }%
    \DecCatcodeTableStack
    \TestStack{0}%
    \Expect*{\TestError}{Empty stack}%
  \endgroup
\end{qstest}
%    \end{macrocode}
%
% \subsubsection{Catcode regime macros}
%
%    \begin{macrocode}
\begin{qstest}{CatcodeRegime}{CatcodeRegime}
  \def\TestStacks#1#2#3{%
    \Expect*{\the\luatexcatcodetable}{#1}%
    \Expect*{\the\CatcodeTableStack}{#2}%
    \Expect*{\LuT@NumStack}{#3}%
  }%
  \TestStacks{0}{0}{0}%
  \catcode`\|=7 %
  \BeginCatcodeRegime\CatcodeTableLaTeX
    \TestStacks{2}{2}{{0}0}%
    \Expect*{\the\catcode`\|}{12}%
  \EndCatcodeRegime
  \TestStacks{0}{0}{0}%
  \Expect*{\the\catcode`\|}{7}%
\end{qstest}
%    \end{macrocode}
%
% \subsection{Attribute allocation}
%
%    \begin{macrocode}
\begin{qstest}{Attributes}{Attributes}
  \newattribute\TestAttr
  \Expect*{\meaning\TestAttr}%
         *{\string\attribute\number\allocationnumber}%
  \Expect*{\the\allocationnumber}{0}%
  \begingroup
    \newattribute\TestAttr
    \Expect*{\the\allocationnumber}{1}%
  \endgroup
  \Expect*{\the\allocationnumber}{0}%
  \Expect*{\meaning\TestAttr}*{\string\attribute1}%
  \Expect*{\the\TestAttr}*{\number\LuT@UnsetAttributeValue}%
  \def\Test#1{%
    \setattribute\TestAttr{#1}%
    \Expect*{\the\TestAttr}{#1}%
  }%
  \Test{0}%
  \Test{1}%
  \Test{-1}%
  \Test{123}%
  \unsetattribute\TestAttr
  \Expect*{\the\TestAttr}*{\number\LuT@UnsetAttributeValue}%
  \begingroup
    \Expect*{\the\TestAttr}*{\number\LuT@UnsetAttributeValue}%
    \Test{1234}%
  \endgroup
  \Expect*{\the\TestAttr}*{\number\LuT@UnsetAttributeValue}%
\end{qstest}
%    \end{macrocode}
%
%    \begin{macrocode}
\@@end
%</test4>
%    \end{macrocode}
%
% \subsection{Short test for \plainTeX}
%
%    \begin{macrocode}
%<*test5>
\input luatex.sty\relax
\newattribute\TestAttr
\setattribute\TestAttr{10}
\unsetattribute\TestAttr
\newcatcodetable\TestCTa
\begingroup
  \SetCatcodeRange{`A}{`Z}{12}%
\endgroup
\BeginCatcodeRegime\CatcodeTableLaTeX
\EndCatcodeRegime
\end
%</test5>
%    \end{macrocode}
%
% \section{Installation}
%
% \subsection{Download}
%
% \paragraph{Package.} This package is available on
% CTAN\footnote{\url{ftp://ftp.ctan.org/tex-archive/}}:
% \begin{description}
% \item[\CTAN{macros/latex/contrib/oberdiek/luatex.dtx}] The source file.
% \item[\CTAN{macros/latex/contrib/oberdiek/luatex.pdf}] Documentation.
% \end{description}
%
%
% \paragraph{Bundle.} All the packages of the bundle `oberdiek'
% are also available in a TDS compliant ZIP archive. There
% the packages are already unpacked and the documentation files
% are generated. The files and directories obey the TDS standard.
% \begin{description}
% \item[\CTAN{install/macros/latex/contrib/oberdiek.tds.zip}]
% \end{description}
% \emph{TDS} refers to the standard ``A Directory Structure
% for \TeX\ Files'' (\CTAN{tds/tds.pdf}). Directories
% with \xfile{texmf} in their name are usually organized this way.
%
% \subsection{Bundle installation}
%
% \paragraph{Unpacking.} Unpack the \xfile{oberdiek.tds.zip} in the
% TDS tree (also known as \xfile{texmf} tree) of your choice.
% Example (linux):
% \begin{quote}
%   |unzip oberdiek.tds.zip -d ~/texmf|
% \end{quote}
%
% \paragraph{Script installation.}
% Check the directory \xfile{TDS:scripts/oberdiek/} for
% scripts that need further installation steps.
% Package \xpackage{attachfile2} comes with the Perl script
% \xfile{pdfatfi.pl} that should be installed in such a way
% that it can be called as \texttt{pdfatfi}.
% Example (linux):
% \begin{quote}
%   |chmod +x scripts/oberdiek/pdfatfi.pl|\\
%   |cp scripts/oberdiek/pdfatfi.pl /usr/local/bin/|
% \end{quote}
%
% \subsection{Package installation}
%
% \paragraph{Unpacking.} The \xfile{.dtx} file is a self-extracting
% \docstrip\ archive. The files are extracted by running the
% \xfile{.dtx} through \plainTeX:
% \begin{quote}
%   \verb|tex luatex.dtx|
% \end{quote}
%
% \paragraph{TDS.} Now the different files must be moved into
% the different directories in your installation TDS tree
% (also known as \xfile{texmf} tree):
% \begin{quote}
% \def\t{^^A
% \begin{tabular}{@{}>{\ttfamily}l@{ $\rightarrow$ }>{\ttfamily}l@{}}
%   luatex.sty & tex/generic/oberdiek/luatex.sty\\
%   luatex-loader.sty & tex/generic/oberdiek/luatex-loader.sty\\
%   oberdiek.luatex.lua & scripts/oberdiek/oberdiek.luatex.lua\\
%   luatex.pdf & doc/latex/oberdiek/luatex.pdf\\
%   test/luatex-test1.tex & doc/latex/oberdiek/test/luatex-test1.tex\\
%   test/luatex-test2.tex & doc/latex/oberdiek/test/luatex-test2.tex\\
%   test/luatex-test3.tex & doc/latex/oberdiek/test/luatex-test3.tex\\
%   test/luatex-test4.tex & doc/latex/oberdiek/test/luatex-test4.tex\\
%   test/luatex-test5.tex & doc/latex/oberdiek/test/luatex-test5.tex\\
%   luatex.dtx & source/latex/oberdiek/luatex.dtx\\
% \end{tabular}^^A
% }^^A
% \sbox0{\t}^^A
% \ifdim\wd0>\linewidth
%   \begingroup
%     \advance\linewidth by\leftmargin
%     \advance\linewidth by\rightmargin
%   \edef\x{\endgroup
%     \def\noexpand\lw{\the\linewidth}^^A
%   }\x
%   \def\lwbox{^^A
%     \leavevmode
%     \hbox to \linewidth{^^A
%       \kern-\leftmargin\relax
%       \hss
%       \usebox0
%       \hss
%       \kern-\rightmargin\relax
%     }^^A
%   }^^A
%   \ifdim\wd0>\lw
%     \sbox0{\small\t}^^A
%     \ifdim\wd0>\linewidth
%       \ifdim\wd0>\lw
%         \sbox0{\footnotesize\t}^^A
%         \ifdim\wd0>\linewidth
%           \ifdim\wd0>\lw
%             \sbox0{\scriptsize\t}^^A
%             \ifdim\wd0>\linewidth
%               \ifdim\wd0>\lw
%                 \sbox0{\tiny\t}^^A
%                 \ifdim\wd0>\linewidth
%                   \lwbox
%                 \else
%                   \usebox0
%                 \fi
%               \else
%                 \lwbox
%               \fi
%             \else
%               \usebox0
%             \fi
%           \else
%             \lwbox
%           \fi
%         \else
%           \usebox0
%         \fi
%       \else
%         \lwbox
%       \fi
%     \else
%       \usebox0
%     \fi
%   \else
%     \lwbox
%   \fi
% \else
%   \usebox0
% \fi
% \end{quote}
% If you have a \xfile{docstrip.cfg} that configures and enables \docstrip's
% TDS installing feature, then some files can already be in the right
% place, see the documentation of \docstrip.
%
% \subsection{Refresh file name databases}
%
% If your \TeX~distribution
% (\teTeX, \mikTeX, \dots) relies on file name databases, you must refresh
% these. For example, \teTeX\ users run \verb|texhash| or
% \verb|mktexlsr|.
%
% \subsection{Some details for the interested}
%
% \paragraph{Attached source.}
%
% The PDF documentation on CTAN also includes the
% \xfile{.dtx} source file. It can be extracted by
% AcrobatReader 6 or higher. Another option is \textsf{pdftk},
% e.g. unpack the file into the current directory:
% \begin{quote}
%   \verb|pdftk luatex.pdf unpack_files output .|
% \end{quote}
%
% \paragraph{Unpacking with \LaTeX.}
% The \xfile{.dtx} chooses its action depending on the format:
% \begin{description}
% \item[\plainTeX:] Run \docstrip\ and extract the files.
% \item[\LaTeX:] Generate the documentation.
% \end{description}
% If you insist on using \LaTeX\ for \docstrip\ (really,
% \docstrip\ does not need \LaTeX), then inform the autodetect routine
% about your intention:
% \begin{quote}
%   \verb|latex \let\install=y% \iffalse meta-comment
%
% File: luatex.dtx
% Version: 2010/03/09 v0.4
% Info: LuaTeX basic definition package
%
% Copyright (C) 2007, 2009, 2010 by
%    Heiko Oberdiek <heiko.oberdiek at googlemail.com>
%
% This work may be distributed and/or modified under the
% conditions of the LaTeX Project Public License, either
% version 1.3c of this license or (at your option) any later
% version. This version of this license is in
%    http://www.latex-project.org/lppl/lppl-1-3c.txt
% and the latest version of this license is in
%    http://www.latex-project.org/lppl.txt
% and version 1.3 or later is part of all distributions of
% LaTeX version 2005/12/01 or later.
%
% This work has the LPPL maintenance status "maintained".
%
% This Current Maintainer of this work is Heiko Oberdiek.
%
% The Base Interpreter refers to any `TeX-Format',
% because some files are installed in TDS:tex/generic//.
%
% This work consists of the main source file luatex.dtx
% and the derived files
%    luatex.sty, luatex.pdf, luatex.ins, luatex.drv, luatex-loader.sty,
%    luatex-test1.tex, luatex-test2.tex, luatex-test3.tex,
%    luatex-test4.tex, luatex-test5.tex, oberdiek.luatex.lua.
%
% Distribution:
%    CTAN:macros/latex/contrib/oberdiek/luatex.dtx
%    CTAN:macros/latex/contrib/oberdiek/luatex.pdf
%
% Unpacking:
%    (a) If luatex.ins is present:
%           tex luatex.ins
%    (b) Without luatex.ins:
%           tex luatex.dtx
%    (c) If you insist on using LaTeX
%           latex \let\install=y\input{luatex.dtx}
%        (quote the arguments according to the demands of your shell)
%
% Documentation:
%    (a) If luatex.drv is present:
%           latex luatex.drv
%    (b) Without luatex.drv:
%           latex luatex.dtx; ...
%    The class ltxdoc loads the configuration file ltxdoc.cfg
%    if available. Here you can specify further options, e.g.
%    use A4 as paper format:
%       \PassOptionsToClass{a4paper}{article}
%
%    Programm calls to get the documentation (example):
%       pdflatex luatex.dtx
%       makeindex -s gind.ist luatex.idx
%       pdflatex luatex.dtx
%       makeindex -s gind.ist luatex.idx
%       pdflatex luatex.dtx
%
% Installation:
%    TDS:tex/generic/oberdiek/luatex.sty
%    TDS:tex/generic/oberdiek/luatex-loader.sty
%    TDS:scripts/oberdiek/oberdiek.luatex.lua
%    TDS:doc/latex/oberdiek/luatex.pdf
%    TDS:doc/latex/oberdiek/test/luatex-test1.tex
%    TDS:doc/latex/oberdiek/test/luatex-test2.tex
%    TDS:doc/latex/oberdiek/test/luatex-test3.tex
%    TDS:doc/latex/oberdiek/test/luatex-test4.tex
%    TDS:doc/latex/oberdiek/test/luatex-test5.tex
%    TDS:source/latex/oberdiek/luatex.dtx
%
%<*ignore>
\begingroup
  \catcode123=1 %
  \catcode125=2 %
  \def\x{LaTeX2e}%
\expandafter\endgroup
\ifcase 0\ifx\install y1\fi\expandafter
         \ifx\csname processbatchFile\endcsname\relax\else1\fi
         \ifx\fmtname\x\else 1\fi\relax
\else\csname fi\endcsname
%</ignore>
%<*install>
\input docstrip.tex
\Msg{************************************************************************}
\Msg{* Installation}
\Msg{* Package: luatex 2010/03/09 v0.4 LuaTeX basic definition package (HO)}
\Msg{************************************************************************}

\keepsilent
\askforoverwritefalse

\let\MetaPrefix\relax
\preamble

This is a generated file.

Project: luatex
Version: 2010/03/09 v0.4

Copyright (C) 2007, 2009, 2010 by
   Heiko Oberdiek <heiko.oberdiek at googlemail.com>

This work may be distributed and/or modified under the
conditions of the LaTeX Project Public License, either
version 1.3c of this license or (at your option) any later
version. This version of this license is in
   http://www.latex-project.org/lppl/lppl-1-3c.txt
and the latest version of this license is in
   http://www.latex-project.org/lppl.txt
and version 1.3 or later is part of all distributions of
LaTeX version 2005/12/01 or later.

This work has the LPPL maintenance status "maintained".

This Current Maintainer of this work is Heiko Oberdiek.

The Base Interpreter refers to any `TeX-Format',
because some files are installed in TDS:tex/generic//.

This work consists of the main source file luatex.dtx
and the derived files
   luatex.sty, luatex.pdf, luatex.ins, luatex.drv, luatex-loader.sty,
   luatex-test1.tex, luatex-test2.tex, luatex-test3.tex,
   luatex-test4.tex, luatex-test5.tex, oberdiek.luatex.lua.

\endpreamble
\let\MetaPrefix\DoubleperCent

\generate{%
  \file{luatex.ins}{\from{luatex.dtx}{install}}%
  \file{luatex.drv}{\from{luatex.dtx}{driver}}%
  \usedir{tex/generic/oberdiek}%
  \file{luatex.sty}{\from{luatex.dtx}{package}}%
  \file{luatex-loader.sty}{\from{luatex.dtx}{loader}}%
  \usedir{doc/latex/oberdiek/test}%
  \file{luatex-test1.tex}{\from{luatex.dtx}{test1}}%
  \file{luatex-test2.tex}{\from{luatex.dtx}{test1,test2}}%
  \file{luatex-test3.tex}{\from{luatex.dtx}{test1,test3}}%
  \file{luatex-test4.tex}{\from{luatex.dtx}{test4}}%
  \file{luatex-test5.tex}{\from{luatex.dtx}{test5}}%
  \nopreamble
  \nopostamble
  \usedir{source/latex/oberdiek/catalogue}%
  \file{luatex.xml}{\from{luatex.dtx}{catalogue}}%
}
\def\MetaPrefix{-- }
\def\defaultpostamble{%
  \MetaPrefix^^J%
  \MetaPrefix\space End of File `\outFileName'.%
}
\def\currentpostamble{\defaultpostamble}%
\generate{%
  \usedir{scripts/oberdiek}%
  \file{oberdiek.luatex.lua}{\from{luatex.dtx}{lua}}%
}

\catcode32=13\relax% active space
\let =\space%
\Msg{************************************************************************}
\Msg{*}
\Msg{* To finish the installation you have to move the following}
\Msg{* files into a directory searched by TeX:}
\Msg{*}
\Msg{*     luatex.sty, luatex-loader.sty}
\Msg{*}
\Msg{* And install the following script file:}
\Msg{*}
\Msg{*     oberdiek.luatex.lua}
\Msg{*}
\Msg{* To produce the documentation run the file `luatex.drv'}
\Msg{* through LaTeX.}
\Msg{*}
\Msg{* Happy TeXing!}
\Msg{*}
\Msg{************************************************************************}

\endbatchfile
%</install>
%<*ignore>
\fi
%</ignore>
%<*driver>
\NeedsTeXFormat{LaTeX2e}
\ProvidesFile{luatex.drv}%
  [2010/03/09 v0.4 LuaTeX basic definition package (HO)]%
\documentclass{ltxdoc}
\usepackage{holtxdoc}[2011/11/22]
\usepackage{paralist}
\begin{document}
  \DocInput{luatex.dtx}%
\end{document}
%</driver>
% \fi
%
% \CheckSum{1429}
%
% \CharacterTable
%  {Upper-case    \A\B\C\D\E\F\G\H\I\J\K\L\M\N\O\P\Q\R\S\T\U\V\W\X\Y\Z
%   Lower-case    \a\b\c\d\e\f\g\h\i\j\k\l\m\n\o\p\q\r\s\t\u\v\w\x\y\z
%   Digits        \0\1\2\3\4\5\6\7\8\9
%   Exclamation   \!     Double quote  \"     Hash (number) \#
%   Dollar        \$     Percent       \%     Ampersand     \&
%   Acute accent  \'     Left paren    \(     Right paren   \)
%   Asterisk      \*     Plus          \+     Comma         \,
%   Minus         \-     Point         \.     Solidus       \/
%   Colon         \:     Semicolon     \;     Less than     \<
%   Equals        \=     Greater than  \>     Question mark \?
%   Commercial at \@     Left bracket  \[     Backslash     \\
%   Right bracket \]     Circumflex    \^     Underscore    \_
%   Grave accent  \`     Left brace    \{     Vertical bar  \|
%   Right brace   \}     Tilde         \~}
%
% \GetFileInfo{luatex.drv}
%
% \title{The \xpackage{luatex} package}
% \date{2010/03/09 v0.4}
% \author{Heiko Oberdiek\\\xemail{heiko.oberdiek at googlemail.com}}
%
% \maketitle
%
% \begin{abstract}
% This package manages the new and extended features and resources
% that \LuaTeX\ provides. Examples are attributes and catcode
% tables.
% \end{abstract}
%
% \tableofcontents
%
% \section{Documentation}
%
% \subsection{Introduction}
%
% \TeX\ provides global resources such as registers.
% But it does not provide an interface for managing these resources.
% For example, two packages want to use a counter register.
% If they take the same register
% number, then the use of both packages will conflict and they cannot
% be used together.
% Therefore formats such as \plainTeX\ or \LaTeX\ implement
% an allocation scheme for registers. A package reserves with
% \cs{newcount} an unused register number for its own exclusive use.
%
% Nowadays \TeX\ is not alone anymore: \eTeX, \pdfTeX\ and other
% compilers for \TeX\ are developed that extend and add new
% features and resources.
%
% Now \LuaTeX\ has reached beta state. It inherits most of
% \pdfTeX's features including \eTeX. Also it implements
% new concepts such as attributes or catcode tables.
%
% \subsubsection{\LaTeX}
%
% \LaTeXe\ is frozen and therefore refuses to even notice
% the new \TeX\ variants. Not even the old \eTeX\ is supported
% by its kernel. At least there is a third party package \xpackage{etex}
% that manages the new \eTeX\ resources.
%
% This package tries to do the same for \LuaTeX\ and starts to
% support at least a few of the new features.
%
% \subsubsection{\plainTeX}
%
% \LaTeX\ has inherited its resource handling from \plainTeX.
% The interface is basically the same: \cs{newcount}, \dots
% Therefore this package tries to follow this tradition by
% providing compatibility to \plainTeX. It can be loaded
% with \plainTeX\ and defines at least some of the
% features that this packages provides for \LaTeX.
%
% \subsection{Register allocation}
%
% \subsubsection{Register with 16 bit}
%
% Because \LuaTeX\ is a super set of \eTeX\ regarding registers,
% the register allocation scheme should not conflict with
% package \xpackage{etex}. Therefore this package is loaded
% to inherit its allocation scheme. The only change is currently
% that the limit is increased to 65536 registers for the following
% register classes:
% \begin{compactitem}
% \item count
% \item dimen
% \item skip
% \item muskip
% \item marks
% \item toks
% \item box
% \end{compactitem}
% This affects the number of global and local registers.
% Because it is done in a package and not in the kernel,
% it is possible that someone loads package \xpackage{etex}
% before uses the local allocation variants. This will prevent
% the extension for this register class. If more registers are
% needed, just load package \xpackage{luatex} earlier.
%
% \subsubsection{Insertions}
%
% Insertions need four registers \cs{count}, \cs{dimen}, \cs{skip},
% and \cs{box} with the same number. Usually they are allocated
% downwards from 254, 253, \dots Also \cs{newcount}, \cs{newdimen}, \dots
% fill up these register numbers from below before switching to
% higher register numbers by package \xpackage{etex}.
% When this occurs, no insertions can be allocated anymore.
%
% Therefore \cs{newcount}, \cs{newdimen}, \cs{newskip}, and
% \cs{newbox} are replaced by their global variants (\cs{globcount}, \dots)
% that use the higher numbers immediately, leaving the room for insertions.
% There should not be an efficiency penalty because \LuaTeX\ stores
% the registers of a class in the same Lua table unlike \eTeX, where
% registers below 256 are stored in an array and higher numbers
% are put in a tree structure.
%
% \subsection{Attributes}
%
% Nodes can have custom attributes in \LuaTeX. These attributes
% are organized by a new register class. As the other registers
% up to $2^{16}$ attributes are supported. An attribute value
% can be negative that means the attribute is not set. Otherwise
% \TeX's range of non-negative integers up to $2^31$ are available.
%
% \begin{declcs}{newattribute} \M{cmd}
% \end{declcs}
% Macro \cs{newattribute} defines command \meta{cmd} using \cs{attributedef}
% using an new attribute number. The new attribute is initially unset.
%
% \begin{declcs}{setattribute} \M{cmd} \M{value}
% \end{declcs}
% Macro \cs{setattribute} locally sets attribute command \meta{cmd}
% to the number \meta{value}. Valid values range from $-1$ until $2^31$
% (the upper limit is the same as for other \TeX\ integer numbers).
%
% \begin{declcs}{unsetattribute} \M{cmd}
% \end{declcs}
% Macro \cs{unsettattribute} clears the attribute command \meta{cmd}.
%
% \subsection{Catcode tables}
%
% \LuaTeX\ introduces catcode tables as new feature, see documentation.
% There is need for discussion, how to deal best:
% \begin{itemize}
% \item \cs{initcatcodetable} and \cs{setcatcodetable} act globally.
% \item \cs{catcodetable} causes an error if used with an uninitialized
%   catcode table.
% \item Large catcode table numbers should be avoided because of
%   performance breakdown.
% \item Use case \LaTeX\ package:
%   The package must not be surprised by changed catcodes and
%   must not surprise by changing catcodes accidently.
%   Catcode tables could offer a solution. At the begin
%   a catcode regime with standard catcodes is established
%   and the old one is restored afterwards.
% \item Use case: \LuaTeX's |tex.print| might be used with
%   a catcode table number, for example a table where all
%   entries have catcode ``other''.
% \item Readonly catcode tables.
% \item Is there is a need for local allocations?
%   (Package \xpackage{etex}'s \cs{loc} variants are not used in
%   \TeX\ Live 2007.)
% \end{itemize}
%
% \subsubsection{Interface proposal}
%
% The idea: \cs{newcatcodetable} allocates odd numbered catcode
% tables. Even numbered tables are managed as stack.
% Also some catcode tables are defined. These must not be
% changed.
%
% \begin{declcs}{newcatcodetable} \M{cmd}
% \end{declcs}
% Macro \cs{newcatcodetable} reserves a new catcode table and
% remembers its number in \meta{cmd}. The catcode table is
% initialized with \iniTeX's catcodes.
%
% \begin{declcs}{CatcodeTableIniTeX}\\
%   \cs{CatcodeTableString}\\
%   \cs{CatcodeTableOther}\\
%   \cs{CatcodeTableLaTeX}\\
% \end{declcs}
% These are catcode tables and must not be changed.
% \cs{CatcodeTableIniTeX} contains the catcode settings of \iniTeX.
% \cs{CatcodeTableString} follows \TeX's convention of
% \cs{string}, \cs{meaning} and friends. The space gets catcode
% 10 (space), the other characters have catcode 12 (other).
% In \cs{CatcodeTableOther} all entries have catcode 12 (other).
% \cs{CatcodeTableLaTeX} contains the setting of a pure \LaTeX\ format
% (`at' is other).
%
% \begin{declcs}{CatcodeTableStack}\\
%   \cs{IncCatcodeTableStack}\\
%   \cs{DecCatcodeTableStack}
% \end{declcs}
% \cs{CatcodeTableStack} is the stack pointer. Initially
% it is catcode table zero. \cs{IncCatcodeTableStack} and
% \cs{DecCatcodeTableStack} increments and decrements the
% stack pointer. Currently \cs{IncCatcodeTableStack} does not
% initialize a new catcode table. Both increment and decrement
% operations do not set a catcode table.
%
% \begin{declcs}{PushCatcodeTableNumStack}\\
%   \cs{PopCatcodeTableNumStack}
% \end{declcs}
% It can be handy to have a global stack for catcode table
% numbers to deal with the global assignment property of
% \cs{initcatcodetable} and \cs{savecatcodetable}.
% \cs{PushCatcodeTableNumStack} pushes the current catcode table
% on the stack.
% \cs{PopCatcodeTableNumStack} pops the topmost number off the number
% stack to set the current catcode table. Catcode table zero is
% used in case of an empty stack.
%
% \begin{declcs}{BeginCatcodeRegime} \M{catcodetable}\\
%   \cs{EndCatcodeRegime}
% \end{declcs}
% \cs{BeginCatcodeRegime} remembers the current catcode table number.
% Then it creates and uses a fresh catcode table on the stack
% that is initialized by \meta{catcodetable}:
% \begin{quote}\ttfamily
%   \cs{PushCatcodeTableNumStack}\\
%   \cs{catcodetable}\meta{catcodetable}||
%   \cs{IncCatcodeTableStack}\\
%   \cs{savecatcodetable}\cs{CatcodeTableStack}\\
%   \cs{catcodetable}\cs{CatcodeTableStack}
% \end{quote}
% \cs{EndCatcodeRegime} drops the catcode table, created by
% \cs{BeginCatcodeRegime} and sets the catcode table that was
% active before:
% \begin{quote}
%   \cs{DecCatcodeTableStack}\\
%   \cs{PopCatcodeTableNumStack}
% \end{quote}
% These macros solve the use case, described earlier for a \LaTeX\ package:
% \begin{quote}
%   |% package foobar.sty|\\
%   |\BeginCatcodeRegime\CatcodeTableLaTeX|\\
%   |\makeatletter|\\
%   |% ... package contents ...|\\
%   |\EndCatcodeRegime|\\
%   |% end of package|
% \end{quote}
% If the package wants to change catcodes after its loading,
% \cs{AtBeginDocument} or \cs{AtEndOfPackage} can be used.
%
% \begin{declcs}{SetCatcodeRange} \M{from} \M{to} \M{catcode}
% \end{declcs}
% The catcodes of characters in range from \meta{from}
% to inclusive \meta{to} are set to \meta{catcode}.
%
% \subsection{Lua module loading}
%
% Currently \LuaTeX\ (version 0.20) does not support Lua script
% files inside \xfile{TDS:scripts//}, because Lua's mechanism
% for module loading does not use the \textsf{kpathsea} library.
% Therefore this packages appends a kpse loader to the list of
% Lua's module loaders. It finds the module \meta{module} by
% \begin{quote}
%   \ttfamily |kpse.find_file("|\meta{module}|.lua", "texmfscripts")|
% \end{quote}
% Unhappily \textsf{kpathsea} does not support directory components
% in a file name. Therefore the Lua convention is not followed to
% replace dots in the module name by the directory separator.
%
% Example:
% A Lua script of a package \xpackage{foobar} wants the following
% modules:
% \begin{quote}
%   |require("foobar.hello.world")|\\
%   |require("org.somewhere.xyz")|
% \end{quote}
% Then they can be find in:
% \begin{quote}
%   |TDS:scripts/foobar/foobar.hello.world.lua|\\
%   |TDS:scripts/foobar/org.somewhere.xyz.lua|
% \end{quote}
% I would have preferred the following locations, following lua conventions,
% e.\,g.:
% \begin{quote}
%   |TDS:scripts/foobar/hello/world.lua|\\
%   |TDS:scripts/foobar/org/somewhere/xyz.lua|
% \end{quote}
% But I do not know, how to achieve this in a reliable way using
% \textsf{kpathsea}.
%
% \subsubsection{Package \xpackage{luatex-loader}}
%
% If someone do not need or want package \xpackage{luatex} but
% it's extension for module loading, then he can use
% package \xpackage{luatex-loader}. Both \plainTeX\ and \LaTeX\ are
% supported.
%
% \StopEventually{
% }
%
% \section{Implementation}
%
%    \begin{macrocode}
%<*package>
%    \end{macrocode}
%
% \subsection{Reload check and package identification}
%    Reload check, especially if the package is not used with \LaTeX.
%    \begin{macrocode}
\begingroup\catcode61\catcode48\catcode32=10\relax%
  \catcode13=5 % ^^M
  \endlinechar=13 %
  \catcode35=6 % #
  \catcode39=12 % '
  \catcode44=12 % ,
  \catcode45=12 % -
  \catcode46=12 % .
  \catcode58=12 % :
  \catcode64=11 % @
  \catcode123=1 % {
  \catcode125=2 % }
  \expandafter\let\expandafter\x\csname ver@luatex.sty\endcsname
  \ifx\x\relax % plain-TeX, first loading
  \else
    \def\empty{}%
    \ifx\x\empty % LaTeX, first loading,
      % variable is initialized, but \ProvidesPackage not yet seen
    \else
      \expandafter\ifx\csname PackageInfo\endcsname\relax
        \def\x#1#2{%
          \immediate\write-1{Package #1 Info: #2.}%
        }%
      \else
        \def\x#1#2{\PackageInfo{#1}{#2, stopped}}%
      \fi
      \x{luatex}{The package is already loaded}%
      \aftergroup\endinput
    \fi
  \fi
\endgroup%
%    \end{macrocode}
%    Package identification:
%    \begin{macrocode}
\begingroup\catcode61\catcode48\catcode32=10\relax%
  \catcode13=5 % ^^M
  \endlinechar=13 %
  \catcode35=6 % #
  \catcode39=12 % '
  \catcode40=12 % (
  \catcode41=12 % )
  \catcode44=12 % ,
  \catcode45=12 % -
  \catcode46=12 % .
  \catcode47=12 % /
  \catcode58=12 % :
  \catcode64=11 % @
  \catcode91=12 % [
  \catcode93=12 % ]
  \catcode123=1 % {
  \catcode125=2 % }
  \expandafter\ifx\csname ProvidesPackage\endcsname\relax
    \def\x#1#2#3[#4]{\endgroup
      \immediate\write-1{Package: #3 #4}%
      \xdef#1{#4}%
    }%
  \else
    \def\x#1#2[#3]{\endgroup
      #2[{#3}]%
      \ifx#1\@undefined
        \xdef#1{#3}%
      \fi
      \ifx#1\relax
        \xdef#1{#3}%
      \fi
    }%
  \fi
\expandafter\x\csname ver@luatex.sty\endcsname
\ProvidesPackage{luatex}%
  [2010/03/09 v0.4 LuaTeX basic definition package (HO)]%
%    \end{macrocode}
%
% \subsection{Catcodes}
%
%    \begin{macrocode}
\begingroup\catcode61\catcode48\catcode32=10\relax%
  \catcode13=5 % ^^M
  \endlinechar=13 %
  \catcode123=1 % {
  \catcode125=2 % }
  \catcode64=11 % @
  \def\x{\endgroup
    \expandafter\edef\csname LuT@AtEnd\endcsname{%
      \endlinechar=\the\endlinechar\relax
      \catcode13=\the\catcode13\relax
      \catcode32=\the\catcode32\relax
      \catcode35=\the\catcode35\relax
      \catcode61=\the\catcode61\relax
      \catcode64=\the\catcode64\relax
      \catcode123=\the\catcode123\relax
      \catcode125=\the\catcode125\relax
    }%
  }%
\x\catcode61\catcode48\catcode32=10\relax%
\catcode13=5 % ^^M
\endlinechar=13 %
\catcode35=6 % #
\catcode64=11 % @
\catcode123=1 % {
\catcode125=2 % }
\def\TMP@EnsureCode#1#2{%
  \edef\LuT@AtEnd{%
    \LuT@AtEnd
    \catcode#1=\the\catcode#1\relax
  }%
  \catcode#1=#2\relax
}
\TMP@EnsureCode{10}{12}% ^^J
\TMP@EnsureCode{34}{12}% "
\TMP@EnsureCode{36}{3}% $
\TMP@EnsureCode{39}{12}% '
\TMP@EnsureCode{40}{12}% (
\TMP@EnsureCode{41}{12}% )
\TMP@EnsureCode{42}{12}% *
\TMP@EnsureCode{43}{12}% +
\TMP@EnsureCode{44}{12}% ,
\TMP@EnsureCode{45}{12}% -
\TMP@EnsureCode{46}{12}% .
\TMP@EnsureCode{47}{12}% /
\TMP@EnsureCode{60}{12}% <
\TMP@EnsureCode{62}{12}% >
\TMP@EnsureCode{91}{12}% [
\TMP@EnsureCode{93}{12}% ]
\TMP@EnsureCode{95}{12}% _ (other!)
\TMP@EnsureCode{96}{12}% `
\edef\LuT@AtEnd{\LuT@AtEnd\noexpand\endinput}
%    \end{macrocode}
%
% \subsection{Check for \LuaTeX}
%
%    Without \LuaTeX\ there is no point in using this package.
%    \begin{macrocode}
\begingroup\expandafter\expandafter\expandafter\endgroup
\expandafter\ifx\csname RequirePackage\endcsname\relax
  \input infwarerr.sty\relax
  \input ifluatex.sty\relax
\else
  \RequirePackage{infwarerr}[2007/09/09]%
  \RequirePackage{ifluatex}[2009/04/10]%
\fi
%    \end{macrocode}
%    \begin{macrocode}
\ifluatex
\else
  \@PackageError{luatex}{%
    This package may only be run using LuaTeX%
  }\@ehc
  \expandafter\LuT@AtEnd
\fi%
%    \end{macrocode}
%
% \subsection{Provide Lua\TeX\ primitives}
%
%    \begin{macrocode}
\ifnum\luatexversion<36 %
  \def\LuT@MakePrimitive#1{%
    \expandafter\let\csname luatex#1\expandafter\endcsname
    \csname #1\endcsname
  }%
\else
  \def\LuT@MakeLuatexPrimitive#1{%
    \begingroup\expandafter\expandafter\expandafter\endgroup
    \expandafter\ifx\csname luatex#1\endcsname\relax
      \begingroup\expandafter\expandafter\expandafter\endgroup
      \expandafter\ifx\csname #1\endcsname\relax
      \else
        \expandafter\let
        \csname luatex#1\expandafter\endcsname
        \csname #1\endcsname
      \fi
    \fi
    \begingroup\expandafter\expandafter\expandafter\endgroup
    \expandafter\ifx\csname luatex#1\endcsname\relax
      \begingroup
        \expandafter\let\csname luatex#1\endcsname\@undefined
        \ifnum0%
          \directlua{%
            if tex.enableprimitives then %
              tex.enableprimitives('luatex',{'#1'})%
              tex.print('1')%
            end%
          }%
          \expandafter\ifx\csname luatex#1\endcsname\relax\else1\fi
        =11 %
          \global\expandafter\let
          \csname luatex#1\expandafter\endcsname
          \csname luatex#1\endcsname
        \else
          \@PackageError{luatex}{%
            tex.enableprimitives failed for `#1'%
          }\@ehc
        \fi
      \endgroup
    \fi
  }%
  \def\LuT@MakePrimitive#1{%
    \begingroup\expandafter\expandafter\expandafter\endgroup
    \expandafter\ifx\csname#1\endcsname\relax
      \begingroup
        \expandafter\let\csname#1\endcsname\@undefined
        \ifnum0%
          \directlua{%
            if tex.enableprimitives then %
              tex.enableprimitives('',{'#1'})%
              tex.print('1')%
            end%
          }%
          \expandafter\ifx\csname#1\endcsname\relax\else1\fi
        =11 %
          \global\expandafter\let
          \csname#1\expandafter\endcsname
          \csname#1\endcsname
        \else
          \@PackageError{luatex}{%
            tex.enableprimitives failed for `#1'%
          }\@ehc
        \fi
      \endgroup
    \fi
  }%
\fi
\LuT@MakeLuatexPrimitive{attribute}
\LuT@MakeLuatexPrimitive{attributedef}
\LuT@MakeLuatexPrimitive{catcodetable}
\LuT@MakeLuatexPrimitive{initcatcodetable}
\LuT@MakeLuatexPrimitive{luaescapestring}
\LuT@MakeLuatexPrimitive{savecatcodetable}
\LuT@MakePrimitive{numexpr}
%    \end{macrocode}
%
% \subsection{Inherit support for \eTeX}
%
%    Package \xpackage{etex} is not compatible for \plainTeX.
%    But it could be present if a format is used that is based
%    on \xfile{etex.src}. Therefore we only load the package
%    in case of \LaTeX\ and tests its presence independently
%    of the format by looking for \cs{et@xins}.
%    \begin{macrocode}
\begingroup\expandafter\expandafter\expandafter\endgroup
\expandafter\ifx\csname RequirePackage\endcsname\relax
\else
  \RequirePackage{etex}[1998/03/26]%
\fi
%    \end{macrocode}
%
% \subsection{Adaption of \eTeX's register allocation}
%
%    \eTeX has increased the number of \TeX\ registers
%    from $2^8$ (256) to $2^{15}$ (32768) for a register class.
%    \LuaTeX\ extends the limit further to $2^16$ (65536).
%    The allocation scheme of package \xpackage{etex} is not
%    changed. But this can be subject for discussion.
%
%    If a register class hasn't registered any local registers
%    yet, then the limit can safely be pushed to 65536.
%    \begin{macrocode}
\begingroup\expandafter\expandafter\expandafter\endgroup
\expandafter\ifx\csname et@xins\endcsname\relax
  \@PackageWarningNoLine{luatex}{%
    Support for eTeX is not loaded (etex.src)%
  }%
\else
  \def\LuT@temp#1{%
    \ifnum\count27#1=32768 %
      \count27#1=65536 %
    \fi
  }%
  \LuT@temp0%
  \LuT@temp1%
  \LuT@temp2%
  \LuT@temp3%
  \LuT@temp4%
  \LuT@temp5%
  \LuT@temp6%
%    \end{macrocode}
%
% \eTeX\ uses an array for the first 256 registers and then a tree
% structure. \LuaTeX\ stores all registers of a class in one Lua table.
% There shouldn't be large performance differences.
% This allows starting immediately in the extended
% area, leaving room for insertions.
%    \begin{macrocode}
  \let\newcount\globcount
  \let\newdimen\globdimen
  \let\newskip\globskip
  \let\newbox\globbox
\fi
%    \end{macrocode}
%
% \subsection{\plainTeX\ compatibility}
%
%    \begin{macro}{\@empty}
%    \begin{macrocode}
\expandafter\ifx\csname @empty\endcsname\relax
  \def\@empty{}%
\fi
%    \end{macrocode}
%    \end{macro}
%    \begin{macro}{\@gobble}
%    \begin{macrocode}
\expandafter\ifx\csname @gobble\endcsname\relax
  \long\def\@gobble#1{}%
\fi
%    \end{macrocode}
%    \end{macro}
%    \begin{macro}{\@firstofone}
%    \begin{macrocode}
\expandafter\ifx\csname @firstofone\endcsname\relax
  \long\def\@firstofone#1{#1}%
\fi
%    \end{macrocode}
%    \end{macro}
%    \begin{macro}{\@firstoftwo}
%    \begin{macrocode}
\expandafter\ifx\csname @firstoftwo\endcsname\relax
  \long\def\@firstoftwo#1#2{#1}%
\fi
%    \end{macrocode}
%    \end{macro}
%    \begin{macro}{\@car}
%    \begin{macrocode}
\expandafter\ifx\csname @car\endcsname\relax
  \def\@car#1#2\@nil{#1}%
\fi
%    \end{macrocode}
%    \end{macro}
%    \begin{macro}{\@cdr}
%    \begin{macrocode}
\expandafter\ifx\csname @cdr\endcsname\relax
  \def\@cdr#1#2\@nil{#2}%
\fi
%    \end{macrocode}
%    \end{macro}
%    \begin{macro}{\@ifstar}
%    \begin{macrocode}
\expandafter\ifx\csname @ifstar\endcsname\relax
  \def\@ifstar#1{%
    \@ifnextchar*{\@firstoftwo{#1}}%
  }%
%    \end{macrocode}
%    \end{macro}
%    \begin{macro}{\@ifnextchar}
%    \begin{macrocode}
  \long\def\@ifnextchar#1#2#3{%
    \let\reserved@d=#1%
    \def\reserved@a{#2}%
    \def\reserved@b{#3}%
    \futurelet\@let@token\@ifnch
  }%
%    \end{macrocode}
%    \end{macro}
%    \begin{macro}{\@ifnch}
%    \begin{macrocode}
  \def\@ifnch{%
    \ifx\@let@token\@sptoken
      \let\reserved@c\@xifnch
    \else
      \ifx\@let@token\reserved@d
        \let\reserved@c\reserved@a
      \else
        \let\reserved@c\reserved@b
      \fi
    \fi
    \reserved@c
  }%
%    \end{macrocode}
%    \end{macro}
%    \begin{macro}{\@sptoken}
%    \begin{macrocode}
  \let\LuT@temp\:%
  \def\:{\let\@sptoken= }%
  \: % explicit space
%    \end{macrocode}
%    \end{macro}
%    \begin{macro}{\@xifnch}
%    \begin{macrocode}
  \def\:{\@xifnch}%
  \expandafter\def\: {%
    \futurelet\@let@token\@ifnch
  }%
  \let\:\LuT@temp
\fi
%    \end{macrocode}
%    \end{macro}
%
%    \begin{macro}{\@tempcnta}
%    \begin{macrocode}
\expandafter\ifx\csname @tempcnta\endcsname\relax
  \csname newcount\endcsname\@tempcnta
\fi
%    \end{macrocode}
%    \end{macro}
%
%    \begin{macro}{\@tempcntb}
%    \begin{macrocode}
\expandafter\ifx\csname @tempcntb\endcsname\relax
  \csname newcount\endcsname\@tempcntb
\fi
%    \end{macrocode}
%    \end{macro}
%
%    \begin{macro}{\LuT@newcommand}
%    \begin{macrocode}
\begingroup\expandafter\expandafter\expandafter\endgroup
\expandafter\ifx\csname newcommand\endcsname\relax
  \def\LuT@newcommand#1[#2]#3{%
    \ifx#1\@undefined
      \let#1\relax
    \else
      \ifx#1\relax
      \else
        \@PackageError{luatex}{%
          \string#1 is already defined.\MessageBreak
          Redefinition is skipped%
        }\@ehc
      \fi
    \fi
    \ifx#1\relax
      \ifcase#2 %
        \def#1{#3}%
      \or
        \def#1##1{#3}%
      \or
        \def#1##1##2{#3}%
      \or
        \def#1##1##2##3{#3}%
      \or
        \@INTERNAL@ERROR
      \fi
    \fi
  }%
\else
  \def\LuT@newcommand{\newcommand*}%
\fi
%    \end{macrocode}
%    \end{macro}
%
% \subsection{Attributes}
%
% \subsubsection{Allocation}
%
%    \begin{macro}{\LuT@AllocAttribute}
%    \begin{macrocode}
\newcount\LuT@AllocAttribute
\LuT@AllocAttribute=\m@ne
%    \end{macrocode}
%    \end{macro}
%    \begin{macro}{\newattribute}
%    \begin{macrocode}
\LuT@newcommand\newattribute[1]{%
  \ifnum\LuT@AllocAttribute<65535 %
    \global\advance\LuT@AllocAttribute\@ne
    \allocationnumber\LuT@AllocAttribute
    \global\luatexattributedef#1=\allocationnumber
    \unsetattribute{#1}%
    \wlog{\string#1=\string\attribute\the\allocationnumber}%
  \else
    \errmessage{No room for a new \string\attribute}%
  \fi
}
%    \end{macrocode}
%    \end{macro}
%
% \subsubsection{Interface}
%
%    \begin{macro}{\setattribute}
%    \begin{macrocode}
\LuT@newcommand\setattribute[2]{%
  #1=\numexpr#2\relax
}
%    \end{macrocode}
%    \end{macro}
%    \begin{macro}{\unsetattribute}
%    \begin{macrocode}
\ifnum\luatexversion<37
  \LuT@newcommand\LuT@UnsetAttributeValue[0]{}%
  \let\LuT@UnsetAttributeValue\m@ne
\else
  \LuT@newcommand\LuT@UnsetAttributeValue[0]{-2147483647 }%
\fi
\LuT@newcommand\unsetattribute[1]{%
  #1=\LuT@UnsetAttributeValue
}
%    \end{macrocode}
%    \end{macro}
%
% \subsection{Catcode tables}
%
% \subsubsection{Allocation}
%
%    \begin{macro}{\LuT@AllocCatcodeTable}
%    \begin{macrocode}
\newcount\LuT@AllocCatcodeTable
\LuT@AllocCatcodeTable=\m@ne
\newcount\CatcodeTableStack
\CatcodeTableStack=\z@
%    \end{macrocode}
%    \end{macro}
%    \begin{macro}{\newcatcodetable}
%    \begin{macrocode}
\LuT@newcommand\newcatcodetable[1]{%
  \ifnum\LuT@AllocCatcodeTable<1114110 % 0x10FFFF is maximal \chardef
    % or < 268435455 % 2^28 - 1
    \global\advance\LuT@AllocCatcodeTable by\tw@
    \allocationnumber=\LuT@AllocCatcodeTable
    \global\chardef#1=\allocationnumber
    \wlog{%
      \string#1=\string\catcodetable\the\allocationnumber
    }%
  \else
    \errmessage{No room for a new \string\catcodetable}%
  \fi
}%
%    \end{macrocode}
%    \end{macro}
%
%    \begin{macro}{\IncCatcodeTableStack}
%    \begin{macrocode}
\LuT@newcommand\IncCatcodeTableStack[0]{%
  \ifnum\CatcodeTableStack<268435454 %
    \global\advance\CatcodeTableStack by\tw@
  \else
    \@PackageError{luatex}{%
      Catcode table stack overflow%
    }\@ehd
  \fi
}
%    \end{macrocode}
%    \end{macro}
%
%    \begin{macro}{\DecCatcodeTableStack}
%    \begin{macrocode}
\LuT@newcommand\DecCatcodeTableStack[0]{%
  \ifnum\CatcodeTableStack>\z@
    \global\advance\CatcodeTableStack by-2 %
  \else
    \@PackageError{luatex}{%
      Catcode table stack is empty%
    }\@ehd
  \fi
}
%    \end{macrocode}
%    \end{macro}
%
% \subsubsection{\cs{SetCatcodeRange}}
%
%    \begin{macro}{\SetCatcodeRange}
%    \begin{macrocode}
\LuT@newcommand\SetCatcodeRange[3]{%
  \edef\LuT@temp{%
    \noexpand\@tempcnta=\the\@tempcnta
    \noexpand\@tempcntb=\the\@tempcntb
    \noexpand\count@=\the\count@
    \relax
  }%
  \@tempcnta=\numexpr#1\relax
  \@tempcntb=\numexpr#2\relax
  \count@=\numexpr#3\relax
  \loop
    \unless\ifnum\@tempcnta>\@tempcntb
    \catcode\@tempcnta=\count@
    \advance\@tempcnta by \@ne
  \repeat
  \LuT@temp
}
%    \end{macrocode}
%    \end{macro}
%
% \subsubsection{Predefined catcode tables}
%
%    \begin{macrocode}
\newcatcodetable\CatcodeTableIniTeX
\newcatcodetable\CatcodeTableString
\newcatcodetable\CatcodeTableOther
\newcatcodetable\CatcodeTableLaTeX
%    \end{macrocode}
%
%    \begin{macrocode}
\luatexinitcatcodetable\CatcodeTableIniTeX
\begingroup
  \def\@makeother#1{\catcode#1=12\relax}%
  \@firstofone{%
    \luatexcatcodetable\CatcodeTableIniTeX
    \begingroup
      \SetCatcodeRange{0}{8}{15}%
      \catcode9=10 % tab
      \catcode11=15 %
      \catcode12=13 % form feed
      \SetCatcodeRange{14}{31}{15}%
      \catcode35=6 % hash
      \catcode36=3 % dollar
      \catcode38=4 % ampersand
      \catcode94=7 % circumflex
      \catcode95=8 % underscore
      \catcode123=1 % brace left
      \catcode125=2 % brace right
      \catcode126=13 % tilde
      \catcode127=15 %
      \luatexsavecatcodetable\CatcodeTableLaTeX
    \endgroup
    \@makeother{0}% nul
    \@makeother{13}% carriage return
    \@makeother{37}% percent
    \@makeother{92}% backslash
    \@makeother{127}%
    \SetCatcodeRange{65}{90}{12}% A-Z
    \SetCatcodeRange{97}{122}{12}% a-z
    \luatexsavecatcodetable\CatcodeTableString
    \@makeother{32}% space
    \luatexsavecatcodetable\CatcodeTableOther
  \endgroup
}%
%    \end{macrocode}
%
% \subsubsection{Number stack}
%
%    \begin{macro}{\LuT@NumStackEmpty}
%    A special empty stack value because of \cs{@cdr}'s brace removal.
%    \begin{macrocode}
\def\LuT@NumStackEmpty{0}
%    \end{macrocode}
%    \end{macro}
%    \begin{macro}{\LuT@NumStack}
%    \begin{macrocode}
\let\LuT@NumStack\LuT@NumStackEmpty
%    \end{macrocode}
%    \end{macro}
%    \begin{macro}{\PushCatcodeTableNumStack}
%    \begin{macrocode}
\LuT@newcommand\PushCatcodeTableNumStack[0]{%
  \xdef\LuT@NumStack{%
    {\the\luatexcatcodetable}\LuT@NumStack
  }%
}
%    \end{macrocode}
%    \end{macro}
%    \begin{macro}{\PopCatcodeTableNumStack}
%    \begin{macrocode}
\LuT@newcommand\PopCatcodeTableNumStack[0]{%
  \ifx\LuT@NumStack\LuT@NumStackEmpty
    \@PackageWarning{luatex}{Empty catcode table number stack}%
    \luatexcatcodetable\z@
  \else
    \luatexcatcodetable=\expandafter\@car\LuT@NumStack\@nil\relax
    \xdef\LuT@NumStack{%
      \expandafter\@cdr\LuT@NumStack\@nil
    }%
  \fi
}
%    \end{macrocode}
%    \end{macro}
%
% \subsubsection{Catcode regime macros}
%
%    \begin{macro}{\BeginCatcodeRegime}
%    \begin{macrocode}
\LuT@newcommand\BeginCatcodeRegime[1]{%
  \PushCatcodeTableNumStack
  \luatexcatcodetable=\numexpr#1\relax
  \IncCatcodeTableStack
  \luatexsavecatcodetable\CatcodeTableStack
  \luatexcatcodetable\CatcodeTableStack
}
%    \end{macrocode}
%    \end{macro}
%
%    \begin{macro}{\EndCatcodeRegime}
%    \begin{macrocode}
\LuT@newcommand\EndCatcodeRegime[0]{%
  \DecCatcodeTableStack
  \PopCatcodeTableNumStack
}
%    \end{macrocode}
%    \end{macro}
%
% \subsection{Lua module loader}
%
%    \begin{macrocode}
\begingroup\expandafter\expandafter\expandafter\endgroup
\expandafter\ifx\csname RequirePackage\endcsname\relax
  \input luatex-loader.sty\relax
\else
  \RequirePackage{luatex-loader}[2010/03/09]%
\fi
%    \end{macrocode}
%    \begin{macrocode}
\LuT@AtEnd%
%</package>
%    \end{macrocode}
%
%    \begin{macrocode}
%<*loader>
%    \end{macrocode}
%
%    Reload check, especially if the package is not used with \LaTeX.
%    \begin{macrocode}
\begingroup\catcode61\catcode48\catcode32=10\relax%
  \catcode13=5 % ^^M
  \endlinechar=13 %
  \catcode35=6 % #
  \catcode39=12 % '
  \catcode44=12 % ,
  \catcode45=12 % -
  \catcode46=12 % .
  \catcode58=12 % :
  \catcode64=11 % @
  \catcode123=1 % {
  \catcode125=2 % }
  \expandafter\let\expandafter\x\csname ver@luatex-loader.sty\endcsname
  \ifx\x\relax % plain-TeX, first loading
  \else
    \def\empty{}%
    \ifx\x\empty % LaTeX, first loading,
      % variable is initialized, but \ProvidesPackage not yet seen
    \else
      \expandafter\ifx\csname PackageInfo\endcsname\relax
        \def\x#1#2{%
          \immediate\write-1{Package #1 Info: #2.}%
        }%
      \else
        \def\x#1#2{\PackageInfo{#1}{#2, stopped}}%
      \fi
      \x{luatex-loader}{The package is already loaded}%
      \aftergroup\endinput
    \fi
  \fi
\endgroup%
%    \end{macrocode}
%    Package identification:
%    \begin{macrocode}
\begingroup\catcode61\catcode48\catcode32=10\relax%
  \catcode13=5 % ^^M
  \endlinechar=13 %
  \catcode35=6 % #
  \catcode39=12 % '
  \catcode40=12 % (
  \catcode41=12 % )
  \catcode44=12 % ,
  \catcode45=12 % -
  \catcode46=12 % .
  \catcode47=12 % /
  \catcode58=12 % :
  \catcode64=11 % @
  \catcode91=12 % [
  \catcode93=12 % ]
  \catcode123=1 % {
  \catcode125=2 % }
  \expandafter\ifx\csname ProvidesPackage\endcsname\relax
    \def\x#1#2#3[#4]{\endgroup
      \immediate\write-1{Package: #3 #4}%
      \xdef#1{#4}%
    }%
  \else
    \def\x#1#2[#3]{\endgroup
      #2[{#3}]%
      \ifx#1\@undefined
        \xdef#1{#3}%
      \fi
      \ifx#1\relax
        \xdef#1{#3}%
      \fi
    }%
  \fi
\expandafter\x\csname ver@luatex-loader.sty\endcsname
\ProvidesPackage{luatex-loader}%
  [2010/03/09 v0.4 Lua module loader (HO)]%
%    \end{macrocode}
%    \begin{macrocode}
\begingroup\catcode61\catcode48\catcode32=10\relax%
  \catcode13=5\endlinechar=13\relax%
  \catcode10=12 % ^^J
  \catcode34=12 % "
  \catcode39=12 % '
  \catcode40=12 % (
  \catcode41=12 % )
  \catcode44=12 % ,
  \catcode46=12 % .
  \catcode60=12 % <
  \catcode61=12 % =
  \catcode95=12 % _ (other!)
  \catcode96=12 % `
  \catcode123=1 % {
  \catcode125=2 % }
  \endlinechar=10 %
  \ifnum\luatexversion<36 %
    \directlua0%
  \else %
    \expandafter\directlua %
  \fi %
  {%
     do
      local script = "oberdiek.luatex.lua"
      local file = kpse.find_file(script, "texmfscripts")
      if file then
        texio.write_nl("(" .. file .. ")")
        dofile(file)
      else
        error("File `" .. script .. "' not found")
      end
    end
  }%
\endgroup\endinput%
%    \end{macrocode}
%
%    \begin{macrocode}
%</loader>
%    \end{macrocode}
%
% \subsection{Lua script}
%
%    Currently \LuaTeX\ does not use KPSE when searching for module
%    files. The following Lua script implements a workaround.
%    It extends |package.loader| by another search method.
%    Modules are found by the module name with extension \xext{lua}
%    similar to
%    \begin{quote}
%      \ttfamily |kpsewhich --format=texmfscripts| \meta{module}|.lua|
%    \end{quote}
%    Unhappily \textsf{kpsewhich} does not support directory components
%    in the file name. Therefore a module |a.b.c| cannot be installed
%    as \xfile{a/b/c.lua}. The script must be named \xfile{a.b.c.lua}.
%    \begin{macrocode}
%<*lua>
%    \end{macrocode}
%    \begin{macrocode}
module("oberdiek.luatex", package.seeall)
%    \end{macrocode}
%    \begin{macrocode}
function kpse_module_loader(module)
  local script = module .. ".lua"
  local file = kpse.find_file(script, "texmfscripts")
  if file then
    local loader, error = loadfile(file)
    if loader then
      texio.write_nl("(" .. file .. ")")
      return loader
    end
    return "\n\t[oberdiek.luatex.kpse_module_loader] Loading error:\n\t"
           .. error
  end
  return "\n\t[oberdiek.luatex.kpse_module_loader] Search failed"
end
table.insert(package.loaders, kpse_module_loader)
%    \end{macrocode}
%    \begin{macrocode}
%</lua>
%    \end{macrocode}
%
% \section{Test}
%
%    \begin{macrocode}
%<*test2>
\documentclass{article}
\def\LoadCommand{%
  \RequirePackage{luatex}[2010/03/09]%
}
%</test2>
%    \end{macrocode}
%    \begin{macrocode}
%<*test3>
\documentclass{article}
\def\LoadCommand{%
  \RequirePackage{luatex-loader}[2010/03/09]%
}
%</test3>
%    \end{macrocode}
% \subsection{Catcode checks for loading}
%
%    \begin{macrocode}
%<*test1>
%    \end{macrocode}
%    \begin{macrocode}
\catcode`\{=1 %
\catcode`\}=2 %
\catcode`\#=6 %
\catcode`\@=11 %
\expandafter\ifx\csname count@\endcsname\relax
  \countdef\count@=255 %
\fi
\expandafter\ifx\csname @gobble\endcsname\relax
  \long\def\@gobble#1{}%
\fi
\expandafter\ifx\csname @firstofone\endcsname\relax
  \long\def\@firstofone#1{#1}%
\fi
\expandafter\ifx\csname loop\endcsname\relax
  \expandafter\@firstofone
\else
  \expandafter\@gobble
\fi
{%
  \def\loop#1\repeat{%
    \def\body{#1}%
    \iterate
  }%
  \def\iterate{%
    \body
      \let\next\iterate
    \else
      \let\next\relax
    \fi
    \next
  }%
  \let\repeat=\fi
}%
\def\RestoreCatcodes{}
\count@=0 %
\loop
  \edef\RestoreCatcodes{%
    \RestoreCatcodes
    \catcode\the\count@=\the\catcode\count@\relax
  }%
\ifnum\count@<255 %
  \advance\count@ 1 %
\repeat

\def\RangeCatcodeInvalid#1#2{%
  \count@=#1\relax
  \loop
    \catcode\count@=15 %
  \ifnum\count@<#2\relax
    \advance\count@ 1 %
  \repeat
}
\def\RangeCatcodeCheck#1#2#3{%
  \count@=#1\relax
  \loop
    \ifnum#3=\catcode\count@
    \else
      \errmessage{%
        Character \the\count@\space
        with wrong catcode \the\catcode\count@\space
        instead of \number#3%
      }%
    \fi
  \ifnum\count@<#2\relax
    \advance\count@ 1 %
  \repeat
}
\def\space{ }
\expandafter\ifx\csname LoadCommand\endcsname\relax
  \def\LoadCommand{\input luatex.sty\relax}%
\fi
\def\Test{%
  \RangeCatcodeInvalid{0}{47}%
  \RangeCatcodeInvalid{58}{64}%
  \RangeCatcodeInvalid{91}{96}%
  \RangeCatcodeInvalid{123}{255}%
  \catcode`\@=12 %
  \catcode`\\=0 %
  \catcode`\%=14 %
  \LoadCommand
  \RangeCatcodeCheck{0}{36}{15}%
  \RangeCatcodeCheck{37}{37}{14}%
  \RangeCatcodeCheck{38}{47}{15}%
  \RangeCatcodeCheck{48}{57}{12}%
  \RangeCatcodeCheck{58}{63}{15}%
  \RangeCatcodeCheck{64}{64}{12}%
  \RangeCatcodeCheck{65}{90}{11}%
  \RangeCatcodeCheck{91}{91}{15}%
  \RangeCatcodeCheck{92}{92}{0}%
  \RangeCatcodeCheck{93}{96}{15}%
  \RangeCatcodeCheck{97}{122}{11}%
  \RangeCatcodeCheck{123}{255}{15}%
  \RestoreCatcodes
}
\Test
\csname @@end\endcsname
\end
%    \end{macrocode}
%    \begin{macrocode}
%</test1>
%    \end{macrocode}
%
% \subsection{Catcode tables}
%
% \subsubsection{Predefined catcode tables}
%
%    \begin{macrocode}
%<*test4>
\NeedsTeXFormat{LaTeX2e}
%    \end{macrocode}
%    Remember \LaTeX's initial catcodes in count registers
%    starting at \cs{TestLaTeX}.
%    \begin{macrocode}
\count0=0 %
\chardef\TestLaTeX=1000 %
\chardef\TestMax=300 %
\loop
  \count\numexpr\TestLaTeX+\count0\relax=\catcode\count0 %
\ifnum\count0<\TestMax
  \advance\count0 by 1 %
\repeat
\documentclass{minimal}
\usepackage{luatex}[2010/03/09]
\usepackage{qstest}
\IncludeTests{*}
\LogTests{log}{*}{*}
\makeatletter
\def\Check#1{%
  \Expect*{\the\count@=\the\catcode\count@}%
         *{\the\count@=#1}%
}
\newcount\scratch
\def\Test#1#2{%
  \begin{qstest}{CatcodeTable#1}{CatcodeTable#1}%
    \luatexcatcodetable\csname CatcodeTable#1\endcsname
    \count@=\z@
    \loop
      \scratch=#2\relax
      \Expect*{\the\count@=\the\catcode\count@}%
             *{\the\count@=\the\scratch}%
    \ifnum\count@<\TestMax
      \advance\count@\@ne
    \repeat
  \end{qstest}%
}
\begingroup
  % luatex-unicode-letters.tex makes some slots to letters
  \def\TestMax{169}%
  \Test{LaTeX}{\the\count\numexpr\TestLaTeX+\count@}%
\endgroup
\Test{String}{\ifnum\count@=32 10\else 12\fi}
\Test{Other}{12}
\luatexinitcatcodetable99 %
\Test{IniTeX}{%
  0\relax
  \begingroup
    \luatexcatcodetable99 %
    \global\scratch=\the\catcode\count@
  \endgroup
}
%    \end{macrocode}
%
% \subsubsection{Catcode table number stack}
%
%    \begin{macrocode}
\begin{qstest}{CatcodeTableNumStack}{CatcodeTableNumStack}
  \def\TestStack#1{%
    \Expect*{\LuT@NumStack}{#1}%
  }%
  \TestStack{0}%
  \PushCatcodeTableNumStack
  \TestStack{{0}0}%
  \@firstofone{%
    \begingroup
      \luatexinitcatcodetable12 %
      \luatexcatcodetable12 %
      \PushCatcodeTableNumStack
      \TestStack{{12}{0}0}%
      \PopCatcodeTableNumStack
      \TestStack{{0}0}%
      \PopCatcodeTableNumStack
      \TestStack{0}%
      \def\TestWarning{Missing empty stack warning}%
      \def\@PackageWarning#1#2{\def\TestWarning{empty stack}}%
      \PopCatcodeTableNumStack
      \TestStack{0}%
      \Expect*{\TestWarning}{empty stack}%
    \endgroup
  }%
\end{qstest}
%    \end{macrocode}
%
% \subsubsection{Catcode table stack}
%
%    \begin{macrocode}
\begin{qstest}{CatcodeTableStack}{CatcodeTableStack}
  \def\TestStack#1{%
    \Expect*{\the\CatcodeTableStack}{#1}%
  }%
  \TestStack{0}%
  \IncCatcodeTableStack
  \TestStack{2}%
  \IncCatcodeTableStack
  \TestStack{4}%
  \begingroup
    \IncCatcodeTableStack
    \TestStack{6}%
  \endgroup
  \TestStack{6}%
  \begingroup
    \DecCatcodeTableStack
    \TestStack{4}%
  \endgroup
  \TestStack{4}%
  \DecCatcodeTableStack
  \TestStack{2}%
  \DecCatcodeTableStack
  \TestStack{0}%
  \begingroup
    \def\TestError{Missing error}%
    \def\@PackageError#1#2#3{%
      \def\TestError{Empty stack}%
    }%
    \DecCatcodeTableStack
    \TestStack{0}%
    \Expect*{\TestError}{Empty stack}%
  \endgroup
\end{qstest}
%    \end{macrocode}
%
% \subsubsection{Catcode regime macros}
%
%    \begin{macrocode}
\begin{qstest}{CatcodeRegime}{CatcodeRegime}
  \def\TestStacks#1#2#3{%
    \Expect*{\the\luatexcatcodetable}{#1}%
    \Expect*{\the\CatcodeTableStack}{#2}%
    \Expect*{\LuT@NumStack}{#3}%
  }%
  \TestStacks{0}{0}{0}%
  \catcode`\|=7 %
  \BeginCatcodeRegime\CatcodeTableLaTeX
    \TestStacks{2}{2}{{0}0}%
    \Expect*{\the\catcode`\|}{12}%
  \EndCatcodeRegime
  \TestStacks{0}{0}{0}%
  \Expect*{\the\catcode`\|}{7}%
\end{qstest}
%    \end{macrocode}
%
% \subsection{Attribute allocation}
%
%    \begin{macrocode}
\begin{qstest}{Attributes}{Attributes}
  \newattribute\TestAttr
  \Expect*{\meaning\TestAttr}%
         *{\string\attribute\number\allocationnumber}%
  \Expect*{\the\allocationnumber}{0}%
  \begingroup
    \newattribute\TestAttr
    \Expect*{\the\allocationnumber}{1}%
  \endgroup
  \Expect*{\the\allocationnumber}{0}%
  \Expect*{\meaning\TestAttr}*{\string\attribute1}%
  \Expect*{\the\TestAttr}*{\number\LuT@UnsetAttributeValue}%
  \def\Test#1{%
    \setattribute\TestAttr{#1}%
    \Expect*{\the\TestAttr}{#1}%
  }%
  \Test{0}%
  \Test{1}%
  \Test{-1}%
  \Test{123}%
  \unsetattribute\TestAttr
  \Expect*{\the\TestAttr}*{\number\LuT@UnsetAttributeValue}%
  \begingroup
    \Expect*{\the\TestAttr}*{\number\LuT@UnsetAttributeValue}%
    \Test{1234}%
  \endgroup
  \Expect*{\the\TestAttr}*{\number\LuT@UnsetAttributeValue}%
\end{qstest}
%    \end{macrocode}
%
%    \begin{macrocode}
\@@end
%</test4>
%    \end{macrocode}
%
% \subsection{Short test for \plainTeX}
%
%    \begin{macrocode}
%<*test5>
\input luatex.sty\relax
\newattribute\TestAttr
\setattribute\TestAttr{10}
\unsetattribute\TestAttr
\newcatcodetable\TestCTa
\begingroup
  \SetCatcodeRange{`A}{`Z}{12}%
\endgroup
\BeginCatcodeRegime\CatcodeTableLaTeX
\EndCatcodeRegime
\end
%</test5>
%    \end{macrocode}
%
% \section{Installation}
%
% \subsection{Download}
%
% \paragraph{Package.} This package is available on
% CTAN\footnote{\url{ftp://ftp.ctan.org/tex-archive/}}:
% \begin{description}
% \item[\CTAN{macros/latex/contrib/oberdiek/luatex.dtx}] The source file.
% \item[\CTAN{macros/latex/contrib/oberdiek/luatex.pdf}] Documentation.
% \end{description}
%
%
% \paragraph{Bundle.} All the packages of the bundle `oberdiek'
% are also available in a TDS compliant ZIP archive. There
% the packages are already unpacked and the documentation files
% are generated. The files and directories obey the TDS standard.
% \begin{description}
% \item[\CTAN{install/macros/latex/contrib/oberdiek.tds.zip}]
% \end{description}
% \emph{TDS} refers to the standard ``A Directory Structure
% for \TeX\ Files'' (\CTAN{tds/tds.pdf}). Directories
% with \xfile{texmf} in their name are usually organized this way.
%
% \subsection{Bundle installation}
%
% \paragraph{Unpacking.} Unpack the \xfile{oberdiek.tds.zip} in the
% TDS tree (also known as \xfile{texmf} tree) of your choice.
% Example (linux):
% \begin{quote}
%   |unzip oberdiek.tds.zip -d ~/texmf|
% \end{quote}
%
% \paragraph{Script installation.}
% Check the directory \xfile{TDS:scripts/oberdiek/} for
% scripts that need further installation steps.
% Package \xpackage{attachfile2} comes with the Perl script
% \xfile{pdfatfi.pl} that should be installed in such a way
% that it can be called as \texttt{pdfatfi}.
% Example (linux):
% \begin{quote}
%   |chmod +x scripts/oberdiek/pdfatfi.pl|\\
%   |cp scripts/oberdiek/pdfatfi.pl /usr/local/bin/|
% \end{quote}
%
% \subsection{Package installation}
%
% \paragraph{Unpacking.} The \xfile{.dtx} file is a self-extracting
% \docstrip\ archive. The files are extracted by running the
% \xfile{.dtx} through \plainTeX:
% \begin{quote}
%   \verb|tex luatex.dtx|
% \end{quote}
%
% \paragraph{TDS.} Now the different files must be moved into
% the different directories in your installation TDS tree
% (also known as \xfile{texmf} tree):
% \begin{quote}
% \def\t{^^A
% \begin{tabular}{@{}>{\ttfamily}l@{ $\rightarrow$ }>{\ttfamily}l@{}}
%   luatex.sty & tex/generic/oberdiek/luatex.sty\\
%   luatex-loader.sty & tex/generic/oberdiek/luatex-loader.sty\\
%   oberdiek.luatex.lua & scripts/oberdiek/oberdiek.luatex.lua\\
%   luatex.pdf & doc/latex/oberdiek/luatex.pdf\\
%   test/luatex-test1.tex & doc/latex/oberdiek/test/luatex-test1.tex\\
%   test/luatex-test2.tex & doc/latex/oberdiek/test/luatex-test2.tex\\
%   test/luatex-test3.tex & doc/latex/oberdiek/test/luatex-test3.tex\\
%   test/luatex-test4.tex & doc/latex/oberdiek/test/luatex-test4.tex\\
%   test/luatex-test5.tex & doc/latex/oberdiek/test/luatex-test5.tex\\
%   luatex.dtx & source/latex/oberdiek/luatex.dtx\\
% \end{tabular}^^A
% }^^A
% \sbox0{\t}^^A
% \ifdim\wd0>\linewidth
%   \begingroup
%     \advance\linewidth by\leftmargin
%     \advance\linewidth by\rightmargin
%   \edef\x{\endgroup
%     \def\noexpand\lw{\the\linewidth}^^A
%   }\x
%   \def\lwbox{^^A
%     \leavevmode
%     \hbox to \linewidth{^^A
%       \kern-\leftmargin\relax
%       \hss
%       \usebox0
%       \hss
%       \kern-\rightmargin\relax
%     }^^A
%   }^^A
%   \ifdim\wd0>\lw
%     \sbox0{\small\t}^^A
%     \ifdim\wd0>\linewidth
%       \ifdim\wd0>\lw
%         \sbox0{\footnotesize\t}^^A
%         \ifdim\wd0>\linewidth
%           \ifdim\wd0>\lw
%             \sbox0{\scriptsize\t}^^A
%             \ifdim\wd0>\linewidth
%               \ifdim\wd0>\lw
%                 \sbox0{\tiny\t}^^A
%                 \ifdim\wd0>\linewidth
%                   \lwbox
%                 \else
%                   \usebox0
%                 \fi
%               \else
%                 \lwbox
%               \fi
%             \else
%               \usebox0
%             \fi
%           \else
%             \lwbox
%           \fi
%         \else
%           \usebox0
%         \fi
%       \else
%         \lwbox
%       \fi
%     \else
%       \usebox0
%     \fi
%   \else
%     \lwbox
%   \fi
% \else
%   \usebox0
% \fi
% \end{quote}
% If you have a \xfile{docstrip.cfg} that configures and enables \docstrip's
% TDS installing feature, then some files can already be in the right
% place, see the documentation of \docstrip.
%
% \subsection{Refresh file name databases}
%
% If your \TeX~distribution
% (\teTeX, \mikTeX, \dots) relies on file name databases, you must refresh
% these. For example, \teTeX\ users run \verb|texhash| or
% \verb|mktexlsr|.
%
% \subsection{Some details for the interested}
%
% \paragraph{Attached source.}
%
% The PDF documentation on CTAN also includes the
% \xfile{.dtx} source file. It can be extracted by
% AcrobatReader 6 or higher. Another option is \textsf{pdftk},
% e.g. unpack the file into the current directory:
% \begin{quote}
%   \verb|pdftk luatex.pdf unpack_files output .|
% \end{quote}
%
% \paragraph{Unpacking with \LaTeX.}
% The \xfile{.dtx} chooses its action depending on the format:
% \begin{description}
% \item[\plainTeX:] Run \docstrip\ and extract the files.
% \item[\LaTeX:] Generate the documentation.
% \end{description}
% If you insist on using \LaTeX\ for \docstrip\ (really,
% \docstrip\ does not need \LaTeX), then inform the autodetect routine
% about your intention:
% \begin{quote}
%   \verb|latex \let\install=y\input{luatex.dtx}|
% \end{quote}
% Do not forget to quote the argument according to the demands
% of your shell.
%
% \paragraph{Generating the documentation.}
% You can use both the \xfile{.dtx} or the \xfile{.drv} to generate
% the documentation. The process can be configured by the
% configuration file \xfile{ltxdoc.cfg}. For instance, put this
% line into this file, if you want to have A4 as paper format:
% \begin{quote}
%   \verb|\PassOptionsToClass{a4paper}{article}|
% \end{quote}
% An example follows how to generate the
% documentation with pdf\LaTeX:
% \begin{quote}
%\begin{verbatim}
%pdflatex luatex.dtx
%makeindex -s gind.ist luatex.idx
%pdflatex luatex.dtx
%makeindex -s gind.ist luatex.idx
%pdflatex luatex.dtx
%\end{verbatim}
% \end{quote}
%
% \section{Catalogue}
%
% The following XML file can be used as source for the
% \href{http://mirror.ctan.org/help/Catalogue/catalogue.html}{\TeX\ Catalogue}.
% The elements \texttt{caption} and \texttt{description} are imported
% from the original XML file from the Catalogue.
% The name of the XML file in the Catalogue is \xfile{luatex.xml}.
%    \begin{macrocode}
%<*catalogue>
<?xml version='1.0' encoding='us-ascii'?>
<!DOCTYPE entry SYSTEM 'catalogue.dtd'>
<entry datestamp='$Date$' modifier='$Author$' id='luatex'>
  <name>luatex</name>
  <caption>The LuaTeX engine.</caption>
  <authorref id='auth:oberdiek'/>
  <copyright owner='Heiko Oberdiek' year='2007,2009,2010'/>
  <license type='lppl1.3'/>
  <version number='0.4'/>
  <description>
    LuaTeX is an extended version of pdfTeX using Lua as an embedded
    scripting language.  The LuaTeX project&#x2019;s main objective
    is to provide an open and configurable variant of TeX while at the
    same time offering downward compatibility.
    <p/>
    LuaTeX uses Unicode (as UTF-8) as its default input encoding, and
    is able to use modern (OpenType) fonts (for both text and mathematics).
    <p/>
    It should be noted that LuaTeX is still under development; its
    specification has been declared stable, but absolute stability
    may not in practice be assumed.
    <p/>
    The package is part of the <xref refid='oberdiek'>oberdiek</xref> bundle.
  </description>
  <documentation details='Package documentation'
      href='ctan:/macros/latex/contrib/oberdiek/luatex.pdf'/>
  <ctan file='true' path='/macros/latex/contrib/oberdiek/luatex.dtx'/>
  <miktex location='oberdiek'/>
  <texlive location='oberdiek'/>
  <install path='/macros/latex/contrib/oberdiek/oberdiek.tds.zip'/>
</entry>
%</catalogue>
%    \end{macrocode}
%
% \begin{History}
%   \begin{Version}{2007/12/12 v0.1}
%   \item
%     First public version.
%   \end{Version}
%   \begin{Version}{2009/04/10 v0.2}
%   \item
%     Requires package \xpackage{ifluatex} in version 2.0 to ensure
%     \cs{luatexversion}.
%   \item
%     Updates the call of \cs{directlua}, the syntax has changed
%     in \LuaTeX\ 0.36.
%   \end{Version}
%   \begin{Version}{2009/12/02 v0.3}
%   \item
%     Unsetting of attributes updated for \LuaTeX\ 0.37.
%   \end{Version}
%   \begin{Version}{2010/03/09 v0.4}
%   \item
%     Support for lua states removed.
%   \item
%     Calling tex.enableprimitives for used primitives.
%   \end{Version}
% \end{History}
%
% \PrintIndex
%
% \Finale
\endinput
|
% \end{quote}
% Do not forget to quote the argument according to the demands
% of your shell.
%
% \paragraph{Generating the documentation.}
% You can use both the \xfile{.dtx} or the \xfile{.drv} to generate
% the documentation. The process can be configured by the
% configuration file \xfile{ltxdoc.cfg}. For instance, put this
% line into this file, if you want to have A4 as paper format:
% \begin{quote}
%   \verb|\PassOptionsToClass{a4paper}{article}|
% \end{quote}
% An example follows how to generate the
% documentation with pdf\LaTeX:
% \begin{quote}
%\begin{verbatim}
%pdflatex luatex.dtx
%makeindex -s gind.ist luatex.idx
%pdflatex luatex.dtx
%makeindex -s gind.ist luatex.idx
%pdflatex luatex.dtx
%\end{verbatim}
% \end{quote}
%
% \section{Catalogue}
%
% The following XML file can be used as source for the
% \href{http://mirror.ctan.org/help/Catalogue/catalogue.html}{\TeX\ Catalogue}.
% The elements \texttt{caption} and \texttt{description} are imported
% from the original XML file from the Catalogue.
% The name of the XML file in the Catalogue is \xfile{luatex.xml}.
%    \begin{macrocode}
%<*catalogue>
<?xml version='1.0' encoding='us-ascii'?>
<!DOCTYPE entry SYSTEM 'catalogue.dtd'>
<entry datestamp='$Date$' modifier='$Author$' id='luatex'>
  <name>luatex</name>
  <caption>The LuaTeX engine.</caption>
  <authorref id='auth:oberdiek'/>
  <copyright owner='Heiko Oberdiek' year='2007,2009,2010'/>
  <license type='lppl1.3'/>
  <version number='0.4'/>
  <description>
    LuaTeX is an extended version of pdfTeX using Lua as an embedded
    scripting language.  The LuaTeX project&#x2019;s main objective
    is to provide an open and configurable variant of TeX while at the
    same time offering downward compatibility.
    <p/>
    LuaTeX uses Unicode (as UTF-8) as its default input encoding, and
    is able to use modern (OpenType) fonts (for both text and mathematics).
    <p/>
    It should be noted that LuaTeX is still under development; its
    specification has been declared stable, but absolute stability
    may not in practice be assumed.
    <p/>
    The package is part of the <xref refid='oberdiek'>oberdiek</xref> bundle.
  </description>
  <documentation details='Package documentation'
      href='ctan:/macros/latex/contrib/oberdiek/luatex.pdf'/>
  <ctan file='true' path='/macros/latex/contrib/oberdiek/luatex.dtx'/>
  <miktex location='oberdiek'/>
  <texlive location='oberdiek'/>
  <install path='/macros/latex/contrib/oberdiek/oberdiek.tds.zip'/>
</entry>
%</catalogue>
%    \end{macrocode}
%
% \begin{History}
%   \begin{Version}{2007/12/12 v0.1}
%   \item
%     First public version.
%   \end{Version}
%   \begin{Version}{2009/04/10 v0.2}
%   \item
%     Requires package \xpackage{ifluatex} in version 2.0 to ensure
%     \cs{luatexversion}.
%   \item
%     Updates the call of \cs{directlua}, the syntax has changed
%     in \LuaTeX\ 0.36.
%   \end{Version}
%   \begin{Version}{2009/12/02 v0.3}
%   \item
%     Unsetting of attributes updated for \LuaTeX\ 0.37.
%   \end{Version}
%   \begin{Version}{2010/03/09 v0.4}
%   \item
%     Support for lua states removed.
%   \item
%     Calling tex.enableprimitives for used primitives.
%   \end{Version}
% \end{History}
%
% \PrintIndex
%
% \Finale
\endinput
|
% \end{quote}
% Do not forget to quote the argument according to the demands
% of your shell.
%
% \paragraph{Generating the documentation.}
% You can use both the \xfile{.dtx} or the \xfile{.drv} to generate
% the documentation. The process can be configured by the
% configuration file \xfile{ltxdoc.cfg}. For instance, put this
% line into this file, if you want to have A4 as paper format:
% \begin{quote}
%   \verb|\PassOptionsToClass{a4paper}{article}|
% \end{quote}
% An example follows how to generate the
% documentation with pdf\LaTeX:
% \begin{quote}
%\begin{verbatim}
%pdflatex luatex.dtx
%makeindex -s gind.ist luatex.idx
%pdflatex luatex.dtx
%makeindex -s gind.ist luatex.idx
%pdflatex luatex.dtx
%\end{verbatim}
% \end{quote}
%
% \section{Catalogue}
%
% The following XML file can be used as source for the
% \href{http://mirror.ctan.org/help/Catalogue/catalogue.html}{\TeX\ Catalogue}.
% The elements \texttt{caption} and \texttt{description} are imported
% from the original XML file from the Catalogue.
% The name of the XML file in the Catalogue is \xfile{luatex.xml}.
%    \begin{macrocode}
%<*catalogue>
<?xml version='1.0' encoding='us-ascii'?>
<!DOCTYPE entry SYSTEM 'catalogue.dtd'>
<entry datestamp='$Date$' modifier='$Author$' id='luatex'>
  <name>luatex</name>
  <caption>The LuaTeX engine.</caption>
  <authorref id='auth:oberdiek'/>
  <copyright owner='Heiko Oberdiek' year='2007,2009,2010'/>
  <license type='lppl1.3'/>
  <version number='0.4'/>
  <description>
    LuaTeX is an extended version of pdfTeX using Lua as an embedded
    scripting language.  The LuaTeX project&#x2019;s main objective
    is to provide an open and configurable variant of TeX while at the
    same time offering downward compatibility.
    <p/>
    LuaTeX uses Unicode (as UTF-8) as its default input encoding, and
    is able to use modern (OpenType) fonts (for both text and mathematics).
    <p/>
    It should be noted that LuaTeX is still under development; its
    specification has been declared stable, but absolute stability
    may not in practice be assumed.
    <p/>
    The package is part of the <xref refid='oberdiek'>oberdiek</xref> bundle.
  </description>
  <documentation details='Package documentation'
      href='ctan:/macros/latex/contrib/oberdiek/luatex.pdf'/>
  <ctan file='true' path='/macros/latex/contrib/oberdiek/luatex.dtx'/>
  <miktex location='oberdiek'/>
  <texlive location='oberdiek'/>
  <install path='/macros/latex/contrib/oberdiek/oberdiek.tds.zip'/>
</entry>
%</catalogue>
%    \end{macrocode}
%
% \begin{History}
%   \begin{Version}{2007/12/12 v0.1}
%   \item
%     First public version.
%   \end{Version}
%   \begin{Version}{2009/04/10 v0.2}
%   \item
%     Requires package \xpackage{ifluatex} in version 2.0 to ensure
%     \cs{luatexversion}.
%   \item
%     Updates the call of \cs{directlua}, the syntax has changed
%     in \LuaTeX\ 0.36.
%   \end{Version}
%   \begin{Version}{2009/12/02 v0.3}
%   \item
%     Unsetting of attributes updated for \LuaTeX\ 0.37.
%   \end{Version}
%   \begin{Version}{2010/03/09 v0.4}
%   \item
%     Support for lua states removed.
%   \item
%     Calling tex.enableprimitives for used primitives.
%   \end{Version}
% \end{History}
%
% \PrintIndex
%
% \Finale
\endinput

%        (quote the arguments according to the demands of your shell)
%
% Documentation:
%    (a) If luatex.drv is present:
%           latex luatex.drv
%    (b) Without luatex.drv:
%           latex luatex.dtx; ...
%    The class ltxdoc loads the configuration file ltxdoc.cfg
%    if available. Here you can specify further options, e.g.
%    use A4 as paper format:
%       \PassOptionsToClass{a4paper}{article}
%
%    Programm calls to get the documentation (example):
%       pdflatex luatex.dtx
%       makeindex -s gind.ist luatex.idx
%       pdflatex luatex.dtx
%       makeindex -s gind.ist luatex.idx
%       pdflatex luatex.dtx
%
% Installation:
%    TDS:tex/generic/oberdiek/luatex.sty
%    TDS:tex/generic/oberdiek/luatex-loader.sty
%    TDS:scripts/oberdiek/oberdiek.luatex.lua
%    TDS:doc/latex/oberdiek/luatex.pdf
%    TDS:doc/latex/oberdiek/test/luatex-test1.tex
%    TDS:doc/latex/oberdiek/test/luatex-test2.tex
%    TDS:doc/latex/oberdiek/test/luatex-test3.tex
%    TDS:doc/latex/oberdiek/test/luatex-test4.tex
%    TDS:doc/latex/oberdiek/test/luatex-test5.tex
%    TDS:source/latex/oberdiek/luatex.dtx
%
%<*ignore>
\begingroup
  \catcode123=1 %
  \catcode125=2 %
  \def\x{LaTeX2e}%
\expandafter\endgroup
\ifcase 0\ifx\install y1\fi\expandafter
         \ifx\csname processbatchFile\endcsname\relax\else1\fi
         \ifx\fmtname\x\else 1\fi\relax
\else\csname fi\endcsname
%</ignore>
%<*install>
\input docstrip.tex
\Msg{************************************************************************}
\Msg{* Installation}
\Msg{* Package: luatex 2010/03/09 v0.4 LuaTeX basic definition package (HO)}
\Msg{************************************************************************}

\keepsilent
\askforoverwritefalse

\let\MetaPrefix\relax
\preamble

This is a generated file.

Project: luatex
Version: 2010/03/09 v0.4

Copyright (C) 2007, 2009, 2010 by
   Heiko Oberdiek <heiko.oberdiek at googlemail.com>

This work may be distributed and/or modified under the
conditions of the LaTeX Project Public License, either
version 1.3c of this license or (at your option) any later
version. This version of this license is in
   http://www.latex-project.org/lppl/lppl-1-3c.txt
and the latest version of this license is in
   http://www.latex-project.org/lppl.txt
and version 1.3 or later is part of all distributions of
LaTeX version 2005/12/01 or later.

This work has the LPPL maintenance status "maintained".

This Current Maintainer of this work is Heiko Oberdiek.

The Base Interpreter refers to any `TeX-Format',
because some files are installed in TDS:tex/generic//.

This work consists of the main source file luatex.dtx
and the derived files
   luatex.sty, luatex.pdf, luatex.ins, luatex.drv, luatex-loader.sty,
   luatex-test1.tex, luatex-test2.tex, luatex-test3.tex,
   luatex-test4.tex, luatex-test5.tex, oberdiek.luatex.lua.

\endpreamble
\let\MetaPrefix\DoubleperCent

\generate{%
  \file{luatex.ins}{\from{luatex.dtx}{install}}%
  \file{luatex.drv}{\from{luatex.dtx}{driver}}%
  \usedir{tex/generic/oberdiek}%
  \file{luatex.sty}{\from{luatex.dtx}{package}}%
  \file{luatex-loader.sty}{\from{luatex.dtx}{loader}}%
  \usedir{doc/latex/oberdiek/test}%
  \file{luatex-test1.tex}{\from{luatex.dtx}{test1}}%
  \file{luatex-test2.tex}{\from{luatex.dtx}{test1,test2}}%
  \file{luatex-test3.tex}{\from{luatex.dtx}{test1,test3}}%
  \file{luatex-test4.tex}{\from{luatex.dtx}{test4}}%
  \file{luatex-test5.tex}{\from{luatex.dtx}{test5}}%
  \nopreamble
  \nopostamble
  \usedir{source/latex/oberdiek/catalogue}%
  \file{luatex.xml}{\from{luatex.dtx}{catalogue}}%
}
\def\MetaPrefix{-- }
\def\defaultpostamble{%
  \MetaPrefix^^J%
  \MetaPrefix\space End of File `\outFileName'.%
}
\def\currentpostamble{\defaultpostamble}%
\generate{%
  \usedir{scripts/oberdiek}%
  \file{oberdiek.luatex.lua}{\from{luatex.dtx}{lua}}%
}

\catcode32=13\relax% active space
\let =\space%
\Msg{************************************************************************}
\Msg{*}
\Msg{* To finish the installation you have to move the following}
\Msg{* files into a directory searched by TeX:}
\Msg{*}
\Msg{*     luatex.sty, luatex-loader.sty}
\Msg{*}
\Msg{* And install the following script file:}
\Msg{*}
\Msg{*     oberdiek.luatex.lua}
\Msg{*}
\Msg{* To produce the documentation run the file `luatex.drv'}
\Msg{* through LaTeX.}
\Msg{*}
\Msg{* Happy TeXing!}
\Msg{*}
\Msg{************************************************************************}

\endbatchfile
%</install>
%<*ignore>
\fi
%</ignore>
%<*driver>
\NeedsTeXFormat{LaTeX2e}
\ProvidesFile{luatex.drv}%
  [2010/03/09 v0.4 LuaTeX basic definition package (HO)]%
\documentclass{ltxdoc}
\usepackage{holtxdoc}[2011/11/22]
\usepackage{paralist}
\begin{document}
  \DocInput{luatex.dtx}%
\end{document}
%</driver>
% \fi
%
% \CheckSum{1429}
%
% \CharacterTable
%  {Upper-case    \A\B\C\D\E\F\G\H\I\J\K\L\M\N\O\P\Q\R\S\T\U\V\W\X\Y\Z
%   Lower-case    \a\b\c\d\e\f\g\h\i\j\k\l\m\n\o\p\q\r\s\t\u\v\w\x\y\z
%   Digits        \0\1\2\3\4\5\6\7\8\9
%   Exclamation   \!     Double quote  \"     Hash (number) \#
%   Dollar        \$     Percent       \%     Ampersand     \&
%   Acute accent  \'     Left paren    \(     Right paren   \)
%   Asterisk      \*     Plus          \+     Comma         \,
%   Minus         \-     Point         \.     Solidus       \/
%   Colon         \:     Semicolon     \;     Less than     \<
%   Equals        \=     Greater than  \>     Question mark \?
%   Commercial at \@     Left bracket  \[     Backslash     \\
%   Right bracket \]     Circumflex    \^     Underscore    \_
%   Grave accent  \`     Left brace    \{     Vertical bar  \|
%   Right brace   \}     Tilde         \~}
%
% \GetFileInfo{luatex.drv}
%
% \title{The \xpackage{luatex} package}
% \date{2010/03/09 v0.4}
% \author{Heiko Oberdiek\\\xemail{heiko.oberdiek at googlemail.com}}
%
% \maketitle
%
% \begin{abstract}
% This package manages the new and extended features and resources
% that \LuaTeX\ provides. Examples are attributes and catcode
% tables.
% \end{abstract}
%
% \tableofcontents
%
% \section{Documentation}
%
% \subsection{Introduction}
%
% \TeX\ provides global resources such as registers.
% But it does not provide an interface for managing these resources.
% For example, two packages want to use a counter register.
% If they take the same register
% number, then the use of both packages will conflict and they cannot
% be used together.
% Therefore formats such as \plainTeX\ or \LaTeX\ implement
% an allocation scheme for registers. A package reserves with
% \cs{newcount} an unused register number for its own exclusive use.
%
% Nowadays \TeX\ is not alone anymore: \eTeX, \pdfTeX\ and other
% compilers for \TeX\ are developed that extend and add new
% features and resources.
%
% Now \LuaTeX\ has reached beta state. It inherits most of
% \pdfTeX's features including \eTeX. Also it implements
% new concepts such as attributes or catcode tables.
%
% \subsubsection{\LaTeX}
%
% \LaTeXe\ is frozen and therefore refuses to even notice
% the new \TeX\ variants. Not even the old \eTeX\ is supported
% by its kernel. At least there is a third party package \xpackage{etex}
% that manages the new \eTeX\ resources.
%
% This package tries to do the same for \LuaTeX\ and starts to
% support at least a few of the new features.
%
% \subsubsection{\plainTeX}
%
% \LaTeX\ has inherited its resource handling from \plainTeX.
% The interface is basically the same: \cs{newcount}, \dots
% Therefore this package tries to follow this tradition by
% providing compatibility to \plainTeX. It can be loaded
% with \plainTeX\ and defines at least some of the
% features that this packages provides for \LaTeX.
%
% \subsection{Register allocation}
%
% \subsubsection{Register with 16 bit}
%
% Because \LuaTeX\ is a super set of \eTeX\ regarding registers,
% the register allocation scheme should not conflict with
% package \xpackage{etex}. Therefore this package is loaded
% to inherit its allocation scheme. The only change is currently
% that the limit is increased to 65536 registers for the following
% register classes:
% \begin{compactitem}
% \item count
% \item dimen
% \item skip
% \item muskip
% \item marks
% \item toks
% \item box
% \end{compactitem}
% This affects the number of global and local registers.
% Because it is done in a package and not in the kernel,
% it is possible that someone loads package \xpackage{etex}
% before uses the local allocation variants. This will prevent
% the extension for this register class. If more registers are
% needed, just load package \xpackage{luatex} earlier.
%
% \subsubsection{Insertions}
%
% Insertions need four registers \cs{count}, \cs{dimen}, \cs{skip},
% and \cs{box} with the same number. Usually they are allocated
% downwards from 254, 253, \dots Also \cs{newcount}, \cs{newdimen}, \dots
% fill up these register numbers from below before switching to
% higher register numbers by package \xpackage{etex}.
% When this occurs, no insertions can be allocated anymore.
%
% Therefore \cs{newcount}, \cs{newdimen}, \cs{newskip}, and
% \cs{newbox} are replaced by their global variants (\cs{globcount}, \dots)
% that use the higher numbers immediately, leaving the room for insertions.
% There should not be an efficiency penalty because \LuaTeX\ stores
% the registers of a class in the same Lua table unlike \eTeX, where
% registers below 256 are stored in an array and higher numbers
% are put in a tree structure.
%
% \subsection{Attributes}
%
% Nodes can have custom attributes in \LuaTeX. These attributes
% are organized by a new register class. As the other registers
% up to $2^{16}$ attributes are supported. An attribute value
% can be negative that means the attribute is not set. Otherwise
% \TeX's range of non-negative integers up to $2^31$ are available.
%
% \begin{declcs}{newattribute} \M{cmd}
% \end{declcs}
% Macro \cs{newattribute} defines command \meta{cmd} using \cs{attributedef}
% using an new attribute number. The new attribute is initially unset.
%
% \begin{declcs}{setattribute} \M{cmd} \M{value}
% \end{declcs}
% Macro \cs{setattribute} locally sets attribute command \meta{cmd}
% to the number \meta{value}. Valid values range from $-1$ until $2^31$
% (the upper limit is the same as for other \TeX\ integer numbers).
%
% \begin{declcs}{unsetattribute} \M{cmd}
% \end{declcs}
% Macro \cs{unsettattribute} clears the attribute command \meta{cmd}.
%
% \subsection{Catcode tables}
%
% \LuaTeX\ introduces catcode tables as new feature, see documentation.
% There is need for discussion, how to deal best:
% \begin{itemize}
% \item \cs{initcatcodetable} and \cs{setcatcodetable} act globally.
% \item \cs{catcodetable} causes an error if used with an uninitialized
%   catcode table.
% \item Large catcode table numbers should be avoided because of
%   performance breakdown.
% \item Use case \LaTeX\ package:
%   The package must not be surprised by changed catcodes and
%   must not surprise by changing catcodes accidently.
%   Catcode tables could offer a solution. At the begin
%   a catcode regime with standard catcodes is established
%   and the old one is restored afterwards.
% \item Use case: \LuaTeX's |tex.print| might be used with
%   a catcode table number, for example a table where all
%   entries have catcode ``other''.
% \item Readonly catcode tables.
% \item Is there is a need for local allocations?
%   (Package \xpackage{etex}'s \cs{loc} variants are not used in
%   \TeX\ Live 2007.)
% \end{itemize}
%
% \subsubsection{Interface proposal}
%
% The idea: \cs{newcatcodetable} allocates odd numbered catcode
% tables. Even numbered tables are managed as stack.
% Also some catcode tables are defined. These must not be
% changed.
%
% \begin{declcs}{newcatcodetable} \M{cmd}
% \end{declcs}
% Macro \cs{newcatcodetable} reserves a new catcode table and
% remembers its number in \meta{cmd}. The catcode table is
% initialized with \iniTeX's catcodes.
%
% \begin{declcs}{CatcodeTableIniTeX}\\
%   \cs{CatcodeTableString}\\
%   \cs{CatcodeTableOther}\\
%   \cs{CatcodeTableLaTeX}\\
% \end{declcs}
% These are catcode tables and must not be changed.
% \cs{CatcodeTableIniTeX} contains the catcode settings of \iniTeX.
% \cs{CatcodeTableString} follows \TeX's convention of
% \cs{string}, \cs{meaning} and friends. The space gets catcode
% 10 (space), the other characters have catcode 12 (other).
% In \cs{CatcodeTableOther} all entries have catcode 12 (other).
% \cs{CatcodeTableLaTeX} contains the setting of a pure \LaTeX\ format
% (`at' is other).
%
% \begin{declcs}{CatcodeTableStack}\\
%   \cs{IncCatcodeTableStack}\\
%   \cs{DecCatcodeTableStack}
% \end{declcs}
% \cs{CatcodeTableStack} is the stack pointer. Initially
% it is catcode table zero. \cs{IncCatcodeTableStack} and
% \cs{DecCatcodeTableStack} increments and decrements the
% stack pointer. Currently \cs{IncCatcodeTableStack} does not
% initialize a new catcode table. Both increment and decrement
% operations do not set a catcode table.
%
% \begin{declcs}{PushCatcodeTableNumStack}\\
%   \cs{PopCatcodeTableNumStack}
% \end{declcs}
% It can be handy to have a global stack for catcode table
% numbers to deal with the global assignment property of
% \cs{initcatcodetable} and \cs{savecatcodetable}.
% \cs{PushCatcodeTableNumStack} pushes the current catcode table
% on the stack.
% \cs{PopCatcodeTableNumStack} pops the topmost number off the number
% stack to set the current catcode table. Catcode table zero is
% used in case of an empty stack.
%
% \begin{declcs}{BeginCatcodeRegime} \M{catcodetable}\\
%   \cs{EndCatcodeRegime}
% \end{declcs}
% \cs{BeginCatcodeRegime} remembers the current catcode table number.
% Then it creates and uses a fresh catcode table on the stack
% that is initialized by \meta{catcodetable}:
% \begin{quote}\ttfamily
%   \cs{PushCatcodeTableNumStack}\\
%   \cs{catcodetable}\meta{catcodetable}||
%   \cs{IncCatcodeTableStack}\\
%   \cs{savecatcodetable}\cs{CatcodeTableStack}\\
%   \cs{catcodetable}\cs{CatcodeTableStack}
% \end{quote}
% \cs{EndCatcodeRegime} drops the catcode table, created by
% \cs{BeginCatcodeRegime} and sets the catcode table that was
% active before:
% \begin{quote}
%   \cs{DecCatcodeTableStack}\\
%   \cs{PopCatcodeTableNumStack}
% \end{quote}
% These macros solve the use case, described earlier for a \LaTeX\ package:
% \begin{quote}
%   |% package foobar.sty|\\
%   |\BeginCatcodeRegime\CatcodeTableLaTeX|\\
%   |\makeatletter|\\
%   |% ... package contents ...|\\
%   |\EndCatcodeRegime|\\
%   |% end of package|
% \end{quote}
% If the package wants to change catcodes after its loading,
% \cs{AtBeginDocument} or \cs{AtEndOfPackage} can be used.
%
% \begin{declcs}{SetCatcodeRange} \M{from} \M{to} \M{catcode}
% \end{declcs}
% The catcodes of characters in range from \meta{from}
% to inclusive \meta{to} are set to \meta{catcode}.
%
% \subsection{Lua module loading}
%
% Currently \LuaTeX\ (version 0.20) does not support Lua script
% files inside \xfile{TDS:scripts//}, because Lua's mechanism
% for module loading does not use the \textsf{kpathsea} library.
% Therefore this packages appends a kpse loader to the list of
% Lua's module loaders. It finds the module \meta{module} by
% \begin{quote}
%   \ttfamily |kpse.find_file("|\meta{module}|.lua", "texmfscripts")|
% \end{quote}
% Unhappily \textsf{kpathsea} does not support directory components
% in a file name. Therefore the Lua convention is not followed to
% replace dots in the module name by the directory separator.
%
% Example:
% A Lua script of a package \xpackage{foobar} wants the following
% modules:
% \begin{quote}
%   |require("foobar.hello.world")|\\
%   |require("org.somewhere.xyz")|
% \end{quote}
% Then they can be find in:
% \begin{quote}
%   |TDS:scripts/foobar/foobar.hello.world.lua|\\
%   |TDS:scripts/foobar/org.somewhere.xyz.lua|
% \end{quote}
% I would have preferred the following locations, following lua conventions,
% e.\,g.:
% \begin{quote}
%   |TDS:scripts/foobar/hello/world.lua|\\
%   |TDS:scripts/foobar/org/somewhere/xyz.lua|
% \end{quote}
% But I do not know, how to achieve this in a reliable way using
% \textsf{kpathsea}.
%
% \subsubsection{Package \xpackage{luatex-loader}}
%
% If someone do not need or want package \xpackage{luatex} but
% it's extension for module loading, then he can use
% package \xpackage{luatex-loader}. Both \plainTeX\ and \LaTeX\ are
% supported.
%
% \StopEventually{
% }
%
% \section{Implementation}
%
%    \begin{macrocode}
%<*package>
%    \end{macrocode}
%
% \subsection{Reload check and package identification}
%    Reload check, especially if the package is not used with \LaTeX.
%    \begin{macrocode}
\begingroup\catcode61\catcode48\catcode32=10\relax%
  \catcode13=5 % ^^M
  \endlinechar=13 %
  \catcode35=6 % #
  \catcode39=12 % '
  \catcode44=12 % ,
  \catcode45=12 % -
  \catcode46=12 % .
  \catcode58=12 % :
  \catcode64=11 % @
  \catcode123=1 % {
  \catcode125=2 % }
  \expandafter\let\expandafter\x\csname ver@luatex.sty\endcsname
  \ifx\x\relax % plain-TeX, first loading
  \else
    \def\empty{}%
    \ifx\x\empty % LaTeX, first loading,
      % variable is initialized, but \ProvidesPackage not yet seen
    \else
      \expandafter\ifx\csname PackageInfo\endcsname\relax
        \def\x#1#2{%
          \immediate\write-1{Package #1 Info: #2.}%
        }%
      \else
        \def\x#1#2{\PackageInfo{#1}{#2, stopped}}%
      \fi
      \x{luatex}{The package is already loaded}%
      \aftergroup\endinput
    \fi
  \fi
\endgroup%
%    \end{macrocode}
%    Package identification:
%    \begin{macrocode}
\begingroup\catcode61\catcode48\catcode32=10\relax%
  \catcode13=5 % ^^M
  \endlinechar=13 %
  \catcode35=6 % #
  \catcode39=12 % '
  \catcode40=12 % (
  \catcode41=12 % )
  \catcode44=12 % ,
  \catcode45=12 % -
  \catcode46=12 % .
  \catcode47=12 % /
  \catcode58=12 % :
  \catcode64=11 % @
  \catcode91=12 % [
  \catcode93=12 % ]
  \catcode123=1 % {
  \catcode125=2 % }
  \expandafter\ifx\csname ProvidesPackage\endcsname\relax
    \def\x#1#2#3[#4]{\endgroup
      \immediate\write-1{Package: #3 #4}%
      \xdef#1{#4}%
    }%
  \else
    \def\x#1#2[#3]{\endgroup
      #2[{#3}]%
      \ifx#1\@undefined
        \xdef#1{#3}%
      \fi
      \ifx#1\relax
        \xdef#1{#3}%
      \fi
    }%
  \fi
\expandafter\x\csname ver@luatex.sty\endcsname
\ProvidesPackage{luatex}%
  [2010/03/09 v0.4 LuaTeX basic definition package (HO)]%
%    \end{macrocode}
%
% \subsection{Catcodes}
%
%    \begin{macrocode}
\begingroup\catcode61\catcode48\catcode32=10\relax%
  \catcode13=5 % ^^M
  \endlinechar=13 %
  \catcode123=1 % {
  \catcode125=2 % }
  \catcode64=11 % @
  \def\x{\endgroup
    \expandafter\edef\csname LuT@AtEnd\endcsname{%
      \endlinechar=\the\endlinechar\relax
      \catcode13=\the\catcode13\relax
      \catcode32=\the\catcode32\relax
      \catcode35=\the\catcode35\relax
      \catcode61=\the\catcode61\relax
      \catcode64=\the\catcode64\relax
      \catcode123=\the\catcode123\relax
      \catcode125=\the\catcode125\relax
    }%
  }%
\x\catcode61\catcode48\catcode32=10\relax%
\catcode13=5 % ^^M
\endlinechar=13 %
\catcode35=6 % #
\catcode64=11 % @
\catcode123=1 % {
\catcode125=2 % }
\def\TMP@EnsureCode#1#2{%
  \edef\LuT@AtEnd{%
    \LuT@AtEnd
    \catcode#1=\the\catcode#1\relax
  }%
  \catcode#1=#2\relax
}
\TMP@EnsureCode{10}{12}% ^^J
\TMP@EnsureCode{34}{12}% "
\TMP@EnsureCode{36}{3}% $
\TMP@EnsureCode{39}{12}% '
\TMP@EnsureCode{40}{12}% (
\TMP@EnsureCode{41}{12}% )
\TMP@EnsureCode{42}{12}% *
\TMP@EnsureCode{43}{12}% +
\TMP@EnsureCode{44}{12}% ,
\TMP@EnsureCode{45}{12}% -
\TMP@EnsureCode{46}{12}% .
\TMP@EnsureCode{47}{12}% /
\TMP@EnsureCode{60}{12}% <
\TMP@EnsureCode{62}{12}% >
\TMP@EnsureCode{91}{12}% [
\TMP@EnsureCode{93}{12}% ]
\TMP@EnsureCode{95}{12}% _ (other!)
\TMP@EnsureCode{96}{12}% `
\edef\LuT@AtEnd{\LuT@AtEnd\noexpand\endinput}
%    \end{macrocode}
%
% \subsection{Check for \LuaTeX}
%
%    Without \LuaTeX\ there is no point in using this package.
%    \begin{macrocode}
\begingroup\expandafter\expandafter\expandafter\endgroup
\expandafter\ifx\csname RequirePackage\endcsname\relax
  \input infwarerr.sty\relax
  \input ifluatex.sty\relax
\else
  \RequirePackage{infwarerr}[2007/09/09]%
  \RequirePackage{ifluatex}[2009/04/10]%
\fi
%    \end{macrocode}
%    \begin{macrocode}
\ifluatex
\else
  \@PackageError{luatex}{%
    This package may only be run using LuaTeX%
  }\@ehc
  \expandafter\LuT@AtEnd
\fi%
%    \end{macrocode}
%
% \subsection{Provide Lua\TeX\ primitives}
%
%    \begin{macrocode}
\ifnum\luatexversion<36 %
  \def\LuT@MakePrimitive#1{%
    \expandafter\let\csname luatex#1\expandafter\endcsname
    \csname #1\endcsname
  }%
\else
  \def\LuT@MakeLuatexPrimitive#1{%
    \begingroup\expandafter\expandafter\expandafter\endgroup
    \expandafter\ifx\csname luatex#1\endcsname\relax
      \begingroup\expandafter\expandafter\expandafter\endgroup
      \expandafter\ifx\csname #1\endcsname\relax
      \else
        \expandafter\let
        \csname luatex#1\expandafter\endcsname
        \csname #1\endcsname
      \fi
    \fi
    \begingroup\expandafter\expandafter\expandafter\endgroup
    \expandafter\ifx\csname luatex#1\endcsname\relax
      \begingroup
        \expandafter\let\csname luatex#1\endcsname\@undefined
        \ifnum0%
          \directlua{%
            if tex.enableprimitives then %
              tex.enableprimitives('luatex',{'#1'})%
              tex.print('1')%
            end%
          }%
          \expandafter\ifx\csname luatex#1\endcsname\relax\else1\fi
        =11 %
          \global\expandafter\let
          \csname luatex#1\expandafter\endcsname
          \csname luatex#1\endcsname
        \else
          \@PackageError{luatex}{%
            tex.enableprimitives failed for `#1'%
          }\@ehc
        \fi
      \endgroup
    \fi
  }%
  \def\LuT@MakePrimitive#1{%
    \begingroup\expandafter\expandafter\expandafter\endgroup
    \expandafter\ifx\csname#1\endcsname\relax
      \begingroup
        \expandafter\let\csname#1\endcsname\@undefined
        \ifnum0%
          \directlua{%
            if tex.enableprimitives then %
              tex.enableprimitives('',{'#1'})%
              tex.print('1')%
            end%
          }%
          \expandafter\ifx\csname#1\endcsname\relax\else1\fi
        =11 %
          \global\expandafter\let
          \csname#1\expandafter\endcsname
          \csname#1\endcsname
        \else
          \@PackageError{luatex}{%
            tex.enableprimitives failed for `#1'%
          }\@ehc
        \fi
      \endgroup
    \fi
  }%
\fi
\LuT@MakeLuatexPrimitive{attribute}
\LuT@MakeLuatexPrimitive{attributedef}
\LuT@MakeLuatexPrimitive{catcodetable}
\LuT@MakeLuatexPrimitive{initcatcodetable}
\LuT@MakeLuatexPrimitive{luaescapestring}
\LuT@MakeLuatexPrimitive{savecatcodetable}
\LuT@MakePrimitive{numexpr}
%    \end{macrocode}
%
% \subsection{Inherit support for \eTeX}
%
%    Package \xpackage{etex} is not compatible for \plainTeX.
%    But it could be present if a format is used that is based
%    on \xfile{etex.src}. Therefore we only load the package
%    in case of \LaTeX\ and tests its presence independently
%    of the format by looking for \cs{et@xins}.
%    \begin{macrocode}
\begingroup\expandafter\expandafter\expandafter\endgroup
\expandafter\ifx\csname RequirePackage\endcsname\relax
\else
  \RequirePackage{etex}[1998/03/26]%
\fi
%    \end{macrocode}
%
% \subsection{Adaption of \eTeX's register allocation}
%
%    \eTeX has increased the number of \TeX\ registers
%    from $2^8$ (256) to $2^{15}$ (32768) for a register class.
%    \LuaTeX\ extends the limit further to $2^16$ (65536).
%    The allocation scheme of package \xpackage{etex} is not
%    changed. But this can be subject for discussion.
%
%    If a register class hasn't registered any local registers
%    yet, then the limit can safely be pushed to 65536.
%    \begin{macrocode}
\begingroup\expandafter\expandafter\expandafter\endgroup
\expandafter\ifx\csname et@xins\endcsname\relax
  \@PackageWarningNoLine{luatex}{%
    Support for eTeX is not loaded (etex.src)%
  }%
\else
  \def\LuT@temp#1{%
    \ifnum\count27#1=32768 %
      \count27#1=65536 %
    \fi
  }%
  \LuT@temp0%
  \LuT@temp1%
  \LuT@temp2%
  \LuT@temp3%
  \LuT@temp4%
  \LuT@temp5%
  \LuT@temp6%
%    \end{macrocode}
%
% \eTeX\ uses an array for the first 256 registers and then a tree
% structure. \LuaTeX\ stores all registers of a class in one Lua table.
% There shouldn't be large performance differences.
% This allows starting immediately in the extended
% area, leaving room for insertions.
%    \begin{macrocode}
  \let\newcount\globcount
  \let\newdimen\globdimen
  \let\newskip\globskip
  \let\newbox\globbox
\fi
%    \end{macrocode}
%
% \subsection{\plainTeX\ compatibility}
%
%    \begin{macro}{\@empty}
%    \begin{macrocode}
\expandafter\ifx\csname @empty\endcsname\relax
  \def\@empty{}%
\fi
%    \end{macrocode}
%    \end{macro}
%    \begin{macro}{\@gobble}
%    \begin{macrocode}
\expandafter\ifx\csname @gobble\endcsname\relax
  \long\def\@gobble#1{}%
\fi
%    \end{macrocode}
%    \end{macro}
%    \begin{macro}{\@firstofone}
%    \begin{macrocode}
\expandafter\ifx\csname @firstofone\endcsname\relax
  \long\def\@firstofone#1{#1}%
\fi
%    \end{macrocode}
%    \end{macro}
%    \begin{macro}{\@firstoftwo}
%    \begin{macrocode}
\expandafter\ifx\csname @firstoftwo\endcsname\relax
  \long\def\@firstoftwo#1#2{#1}%
\fi
%    \end{macrocode}
%    \end{macro}
%    \begin{macro}{\@car}
%    \begin{macrocode}
\expandafter\ifx\csname @car\endcsname\relax
  \def\@car#1#2\@nil{#1}%
\fi
%    \end{macrocode}
%    \end{macro}
%    \begin{macro}{\@cdr}
%    \begin{macrocode}
\expandafter\ifx\csname @cdr\endcsname\relax
  \def\@cdr#1#2\@nil{#2}%
\fi
%    \end{macrocode}
%    \end{macro}
%    \begin{macro}{\@ifstar}
%    \begin{macrocode}
\expandafter\ifx\csname @ifstar\endcsname\relax
  \def\@ifstar#1{%
    \@ifnextchar*{\@firstoftwo{#1}}%
  }%
%    \end{macrocode}
%    \end{macro}
%    \begin{macro}{\@ifnextchar}
%    \begin{macrocode}
  \long\def\@ifnextchar#1#2#3{%
    \let\reserved@d=#1%
    \def\reserved@a{#2}%
    \def\reserved@b{#3}%
    \futurelet\@let@token\@ifnch
  }%
%    \end{macrocode}
%    \end{macro}
%    \begin{macro}{\@ifnch}
%    \begin{macrocode}
  \def\@ifnch{%
    \ifx\@let@token\@sptoken
      \let\reserved@c\@xifnch
    \else
      \ifx\@let@token\reserved@d
        \let\reserved@c\reserved@a
      \else
        \let\reserved@c\reserved@b
      \fi
    \fi
    \reserved@c
  }%
%    \end{macrocode}
%    \end{macro}
%    \begin{macro}{\@sptoken}
%    \begin{macrocode}
  \let\LuT@temp\:%
  \def\:{\let\@sptoken= }%
  \: % explicit space
%    \end{macrocode}
%    \end{macro}
%    \begin{macro}{\@xifnch}
%    \begin{macrocode}
  \def\:{\@xifnch}%
  \expandafter\def\: {%
    \futurelet\@let@token\@ifnch
  }%
  \let\:\LuT@temp
\fi
%    \end{macrocode}
%    \end{macro}
%
%    \begin{macro}{\@tempcnta}
%    \begin{macrocode}
\expandafter\ifx\csname @tempcnta\endcsname\relax
  \csname newcount\endcsname\@tempcnta
\fi
%    \end{macrocode}
%    \end{macro}
%
%    \begin{macro}{\@tempcntb}
%    \begin{macrocode}
\expandafter\ifx\csname @tempcntb\endcsname\relax
  \csname newcount\endcsname\@tempcntb
\fi
%    \end{macrocode}
%    \end{macro}
%
%    \begin{macro}{\LuT@newcommand}
%    \begin{macrocode}
\begingroup\expandafter\expandafter\expandafter\endgroup
\expandafter\ifx\csname newcommand\endcsname\relax
  \def\LuT@newcommand#1[#2]#3{%
    \ifx#1\@undefined
      \let#1\relax
    \else
      \ifx#1\relax
      \else
        \@PackageError{luatex}{%
          \string#1 is already defined.\MessageBreak
          Redefinition is skipped%
        }\@ehc
      \fi
    \fi
    \ifx#1\relax
      \ifcase#2 %
        \def#1{#3}%
      \or
        \def#1##1{#3}%
      \or
        \def#1##1##2{#3}%
      \or
        \def#1##1##2##3{#3}%
      \or
        \@INTERNAL@ERROR
      \fi
    \fi
  }%
\else
  \def\LuT@newcommand{\newcommand*}%
\fi
%    \end{macrocode}
%    \end{macro}
%
% \subsection{Attributes}
%
% \subsubsection{Allocation}
%
%    \begin{macro}{\LuT@AllocAttribute}
%    \begin{macrocode}
\newcount\LuT@AllocAttribute
\LuT@AllocAttribute=\m@ne
%    \end{macrocode}
%    \end{macro}
%    \begin{macro}{\newattribute}
%    \begin{macrocode}
\LuT@newcommand\newattribute[1]{%
  \ifnum\LuT@AllocAttribute<65535 %
    \global\advance\LuT@AllocAttribute\@ne
    \allocationnumber\LuT@AllocAttribute
    \global\luatexattributedef#1=\allocationnumber
    \unsetattribute{#1}%
    \wlog{\string#1=\string\attribute\the\allocationnumber}%
  \else
    \errmessage{No room for a new \string\attribute}%
  \fi
}
%    \end{macrocode}
%    \end{macro}
%
% \subsubsection{Interface}
%
%    \begin{macro}{\setattribute}
%    \begin{macrocode}
\LuT@newcommand\setattribute[2]{%
  #1=\numexpr#2\relax
}
%    \end{macrocode}
%    \end{macro}
%    \begin{macro}{\unsetattribute}
%    \begin{macrocode}
\ifnum\luatexversion<37
  \LuT@newcommand\LuT@UnsetAttributeValue[0]{}%
  \let\LuT@UnsetAttributeValue\m@ne
\else
  \LuT@newcommand\LuT@UnsetAttributeValue[0]{-2147483647 }%
\fi
\LuT@newcommand\unsetattribute[1]{%
  #1=\LuT@UnsetAttributeValue
}
%    \end{macrocode}
%    \end{macro}
%
% \subsection{Catcode tables}
%
% \subsubsection{Allocation}
%
%    \begin{macro}{\LuT@AllocCatcodeTable}
%    \begin{macrocode}
\newcount\LuT@AllocCatcodeTable
\LuT@AllocCatcodeTable=\m@ne
\newcount\CatcodeTableStack
\CatcodeTableStack=\z@
%    \end{macrocode}
%    \end{macro}
%    \begin{macro}{\newcatcodetable}
%    \begin{macrocode}
\LuT@newcommand\newcatcodetable[1]{%
  \ifnum\LuT@AllocCatcodeTable<1114110 % 0x10FFFF is maximal \chardef
    % or < 268435455 % 2^28 - 1
    \global\advance\LuT@AllocCatcodeTable by\tw@
    \allocationnumber=\LuT@AllocCatcodeTable
    \global\chardef#1=\allocationnumber
    \wlog{%
      \string#1=\string\catcodetable\the\allocationnumber
    }%
  \else
    \errmessage{No room for a new \string\catcodetable}%
  \fi
}%
%    \end{macrocode}
%    \end{macro}
%
%    \begin{macro}{\IncCatcodeTableStack}
%    \begin{macrocode}
\LuT@newcommand\IncCatcodeTableStack[0]{%
  \ifnum\CatcodeTableStack<268435454 %
    \global\advance\CatcodeTableStack by\tw@
  \else
    \@PackageError{luatex}{%
      Catcode table stack overflow%
    }\@ehd
  \fi
}
%    \end{macrocode}
%    \end{macro}
%
%    \begin{macro}{\DecCatcodeTableStack}
%    \begin{macrocode}
\LuT@newcommand\DecCatcodeTableStack[0]{%
  \ifnum\CatcodeTableStack>\z@
    \global\advance\CatcodeTableStack by-2 %
  \else
    \@PackageError{luatex}{%
      Catcode table stack is empty%
    }\@ehd
  \fi
}
%    \end{macrocode}
%    \end{macro}
%
% \subsubsection{\cs{SetCatcodeRange}}
%
%    \begin{macro}{\SetCatcodeRange}
%    \begin{macrocode}
\LuT@newcommand\SetCatcodeRange[3]{%
  \edef\LuT@temp{%
    \noexpand\@tempcnta=\the\@tempcnta
    \noexpand\@tempcntb=\the\@tempcntb
    \noexpand\count@=\the\count@
    \relax
  }%
  \@tempcnta=\numexpr#1\relax
  \@tempcntb=\numexpr#2\relax
  \count@=\numexpr#3\relax
  \loop
    \unless\ifnum\@tempcnta>\@tempcntb
    \catcode\@tempcnta=\count@
    \advance\@tempcnta by \@ne
  \repeat
  \LuT@temp
}
%    \end{macrocode}
%    \end{macro}
%
% \subsubsection{Predefined catcode tables}
%
%    \begin{macrocode}
\newcatcodetable\CatcodeTableIniTeX
\newcatcodetable\CatcodeTableString
\newcatcodetable\CatcodeTableOther
\newcatcodetable\CatcodeTableLaTeX
%    \end{macrocode}
%
%    \begin{macrocode}
\luatexinitcatcodetable\CatcodeTableIniTeX
\begingroup
  \def\@makeother#1{\catcode#1=12\relax}%
  \@firstofone{%
    \luatexcatcodetable\CatcodeTableIniTeX
    \begingroup
      \SetCatcodeRange{0}{8}{15}%
      \catcode9=10 % tab
      \catcode11=15 %
      \catcode12=13 % form feed
      \SetCatcodeRange{14}{31}{15}%
      \catcode35=6 % hash
      \catcode36=3 % dollar
      \catcode38=4 % ampersand
      \catcode94=7 % circumflex
      \catcode95=8 % underscore
      \catcode123=1 % brace left
      \catcode125=2 % brace right
      \catcode126=13 % tilde
      \catcode127=15 %
      \luatexsavecatcodetable\CatcodeTableLaTeX
    \endgroup
    \@makeother{0}% nul
    \@makeother{13}% carriage return
    \@makeother{37}% percent
    \@makeother{92}% backslash
    \@makeother{127}%
    \SetCatcodeRange{65}{90}{12}% A-Z
    \SetCatcodeRange{97}{122}{12}% a-z
    \luatexsavecatcodetable\CatcodeTableString
    \@makeother{32}% space
    \luatexsavecatcodetable\CatcodeTableOther
  \endgroup
}%
%    \end{macrocode}
%
% \subsubsection{Number stack}
%
%    \begin{macro}{\LuT@NumStackEmpty}
%    A special empty stack value because of \cs{@cdr}'s brace removal.
%    \begin{macrocode}
\def\LuT@NumStackEmpty{0}
%    \end{macrocode}
%    \end{macro}
%    \begin{macro}{\LuT@NumStack}
%    \begin{macrocode}
\let\LuT@NumStack\LuT@NumStackEmpty
%    \end{macrocode}
%    \end{macro}
%    \begin{macro}{\PushCatcodeTableNumStack}
%    \begin{macrocode}
\LuT@newcommand\PushCatcodeTableNumStack[0]{%
  \xdef\LuT@NumStack{%
    {\the\luatexcatcodetable}\LuT@NumStack
  }%
}
%    \end{macrocode}
%    \end{macro}
%    \begin{macro}{\PopCatcodeTableNumStack}
%    \begin{macrocode}
\LuT@newcommand\PopCatcodeTableNumStack[0]{%
  \ifx\LuT@NumStack\LuT@NumStackEmpty
    \@PackageWarning{luatex}{Empty catcode table number stack}%
    \luatexcatcodetable\z@
  \else
    \luatexcatcodetable=\expandafter\@car\LuT@NumStack\@nil\relax
    \xdef\LuT@NumStack{%
      \expandafter\@cdr\LuT@NumStack\@nil
    }%
  \fi
}
%    \end{macrocode}
%    \end{macro}
%
% \subsubsection{Catcode regime macros}
%
%    \begin{macro}{\BeginCatcodeRegime}
%    \begin{macrocode}
\LuT@newcommand\BeginCatcodeRegime[1]{%
  \PushCatcodeTableNumStack
  \luatexcatcodetable=\numexpr#1\relax
  \IncCatcodeTableStack
  \luatexsavecatcodetable\CatcodeTableStack
  \luatexcatcodetable\CatcodeTableStack
}
%    \end{macrocode}
%    \end{macro}
%
%    \begin{macro}{\EndCatcodeRegime}
%    \begin{macrocode}
\LuT@newcommand\EndCatcodeRegime[0]{%
  \DecCatcodeTableStack
  \PopCatcodeTableNumStack
}
%    \end{macrocode}
%    \end{macro}
%
% \subsection{Lua module loader}
%
%    \begin{macrocode}
\begingroup\expandafter\expandafter\expandafter\endgroup
\expandafter\ifx\csname RequirePackage\endcsname\relax
  \input luatex-loader.sty\relax
\else
  \RequirePackage{luatex-loader}[2010/03/09]%
\fi
%    \end{macrocode}
%    \begin{macrocode}
\LuT@AtEnd%
%</package>
%    \end{macrocode}
%
%    \begin{macrocode}
%<*loader>
%    \end{macrocode}
%
%    Reload check, especially if the package is not used with \LaTeX.
%    \begin{macrocode}
\begingroup\catcode61\catcode48\catcode32=10\relax%
  \catcode13=5 % ^^M
  \endlinechar=13 %
  \catcode35=6 % #
  \catcode39=12 % '
  \catcode44=12 % ,
  \catcode45=12 % -
  \catcode46=12 % .
  \catcode58=12 % :
  \catcode64=11 % @
  \catcode123=1 % {
  \catcode125=2 % }
  \expandafter\let\expandafter\x\csname ver@luatex-loader.sty\endcsname
  \ifx\x\relax % plain-TeX, first loading
  \else
    \def\empty{}%
    \ifx\x\empty % LaTeX, first loading,
      % variable is initialized, but \ProvidesPackage not yet seen
    \else
      \expandafter\ifx\csname PackageInfo\endcsname\relax
        \def\x#1#2{%
          \immediate\write-1{Package #1 Info: #2.}%
        }%
      \else
        \def\x#1#2{\PackageInfo{#1}{#2, stopped}}%
      \fi
      \x{luatex-loader}{The package is already loaded}%
      \aftergroup\endinput
    \fi
  \fi
\endgroup%
%    \end{macrocode}
%    Package identification:
%    \begin{macrocode}
\begingroup\catcode61\catcode48\catcode32=10\relax%
  \catcode13=5 % ^^M
  \endlinechar=13 %
  \catcode35=6 % #
  \catcode39=12 % '
  \catcode40=12 % (
  \catcode41=12 % )
  \catcode44=12 % ,
  \catcode45=12 % -
  \catcode46=12 % .
  \catcode47=12 % /
  \catcode58=12 % :
  \catcode64=11 % @
  \catcode91=12 % [
  \catcode93=12 % ]
  \catcode123=1 % {
  \catcode125=2 % }
  \expandafter\ifx\csname ProvidesPackage\endcsname\relax
    \def\x#1#2#3[#4]{\endgroup
      \immediate\write-1{Package: #3 #4}%
      \xdef#1{#4}%
    }%
  \else
    \def\x#1#2[#3]{\endgroup
      #2[{#3}]%
      \ifx#1\@undefined
        \xdef#1{#3}%
      \fi
      \ifx#1\relax
        \xdef#1{#3}%
      \fi
    }%
  \fi
\expandafter\x\csname ver@luatex-loader.sty\endcsname
\ProvidesPackage{luatex-loader}%
  [2010/03/09 v0.4 Lua module loader (HO)]%
%    \end{macrocode}
%    \begin{macrocode}
\begingroup\catcode61\catcode48\catcode32=10\relax%
  \catcode13=5\endlinechar=13\relax%
  \catcode10=12 % ^^J
  \catcode34=12 % "
  \catcode39=12 % '
  \catcode40=12 % (
  \catcode41=12 % )
  \catcode44=12 % ,
  \catcode46=12 % .
  \catcode60=12 % <
  \catcode61=12 % =
  \catcode95=12 % _ (other!)
  \catcode96=12 % `
  \catcode123=1 % {
  \catcode125=2 % }
  \endlinechar=10 %
  \ifnum\luatexversion<36 %
    \directlua0%
  \else %
    \expandafter\directlua %
  \fi %
  {%
     do
      local script = "oberdiek.luatex.lua"
      local file = kpse.find_file(script, "texmfscripts")
      if file then
        texio.write_nl("(" .. file .. ")")
        dofile(file)
      else
        error("File `" .. script .. "' not found")
      end
    end
  }%
\endgroup\endinput%
%    \end{macrocode}
%
%    \begin{macrocode}
%</loader>
%    \end{macrocode}
%
% \subsection{Lua script}
%
%    Currently \LuaTeX\ does not use KPSE when searching for module
%    files. The following Lua script implements a workaround.
%    It extends |package.loader| by another search method.
%    Modules are found by the module name with extension \xext{lua}
%    similar to
%    \begin{quote}
%      \ttfamily |kpsewhich --format=texmfscripts| \meta{module}|.lua|
%    \end{quote}
%    Unhappily \textsf{kpsewhich} does not support directory components
%    in the file name. Therefore a module |a.b.c| cannot be installed
%    as \xfile{a/b/c.lua}. The script must be named \xfile{a.b.c.lua}.
%    \begin{macrocode}
%<*lua>
%    \end{macrocode}
%    \begin{macrocode}
module("oberdiek.luatex", package.seeall)
%    \end{macrocode}
%    \begin{macrocode}
function kpse_module_loader(module)
  local script = module .. ".lua"
  local file = kpse.find_file(script, "texmfscripts")
  if file then
    local loader, error = loadfile(file)
    if loader then
      texio.write_nl("(" .. file .. ")")
      return loader
    end
    return "\n\t[oberdiek.luatex.kpse_module_loader] Loading error:\n\t"
           .. error
  end
  return "\n\t[oberdiek.luatex.kpse_module_loader] Search failed"
end
table.insert(package.loaders, kpse_module_loader)
%    \end{macrocode}
%    \begin{macrocode}
%</lua>
%    \end{macrocode}
%
% \section{Test}
%
%    \begin{macrocode}
%<*test2>
\documentclass{article}
\def\LoadCommand{%
  \RequirePackage{luatex}[2010/03/09]%
}
%</test2>
%    \end{macrocode}
%    \begin{macrocode}
%<*test3>
\documentclass{article}
\def\LoadCommand{%
  \RequirePackage{luatex-loader}[2010/03/09]%
}
%</test3>
%    \end{macrocode}
% \subsection{Catcode checks for loading}
%
%    \begin{macrocode}
%<*test1>
%    \end{macrocode}
%    \begin{macrocode}
\catcode`\{=1 %
\catcode`\}=2 %
\catcode`\#=6 %
\catcode`\@=11 %
\expandafter\ifx\csname count@\endcsname\relax
  \countdef\count@=255 %
\fi
\expandafter\ifx\csname @gobble\endcsname\relax
  \long\def\@gobble#1{}%
\fi
\expandafter\ifx\csname @firstofone\endcsname\relax
  \long\def\@firstofone#1{#1}%
\fi
\expandafter\ifx\csname loop\endcsname\relax
  \expandafter\@firstofone
\else
  \expandafter\@gobble
\fi
{%
  \def\loop#1\repeat{%
    \def\body{#1}%
    \iterate
  }%
  \def\iterate{%
    \body
      \let\next\iterate
    \else
      \let\next\relax
    \fi
    \next
  }%
  \let\repeat=\fi
}%
\def\RestoreCatcodes{}
\count@=0 %
\loop
  \edef\RestoreCatcodes{%
    \RestoreCatcodes
    \catcode\the\count@=\the\catcode\count@\relax
  }%
\ifnum\count@<255 %
  \advance\count@ 1 %
\repeat

\def\RangeCatcodeInvalid#1#2{%
  \count@=#1\relax
  \loop
    \catcode\count@=15 %
  \ifnum\count@<#2\relax
    \advance\count@ 1 %
  \repeat
}
\def\RangeCatcodeCheck#1#2#3{%
  \count@=#1\relax
  \loop
    \ifnum#3=\catcode\count@
    \else
      \errmessage{%
        Character \the\count@\space
        with wrong catcode \the\catcode\count@\space
        instead of \number#3%
      }%
    \fi
  \ifnum\count@<#2\relax
    \advance\count@ 1 %
  \repeat
}
\def\space{ }
\expandafter\ifx\csname LoadCommand\endcsname\relax
  \def\LoadCommand{\input luatex.sty\relax}%
\fi
\def\Test{%
  \RangeCatcodeInvalid{0}{47}%
  \RangeCatcodeInvalid{58}{64}%
  \RangeCatcodeInvalid{91}{96}%
  \RangeCatcodeInvalid{123}{255}%
  \catcode`\@=12 %
  \catcode`\\=0 %
  \catcode`\%=14 %
  \LoadCommand
  \RangeCatcodeCheck{0}{36}{15}%
  \RangeCatcodeCheck{37}{37}{14}%
  \RangeCatcodeCheck{38}{47}{15}%
  \RangeCatcodeCheck{48}{57}{12}%
  \RangeCatcodeCheck{58}{63}{15}%
  \RangeCatcodeCheck{64}{64}{12}%
  \RangeCatcodeCheck{65}{90}{11}%
  \RangeCatcodeCheck{91}{91}{15}%
  \RangeCatcodeCheck{92}{92}{0}%
  \RangeCatcodeCheck{93}{96}{15}%
  \RangeCatcodeCheck{97}{122}{11}%
  \RangeCatcodeCheck{123}{255}{15}%
  \RestoreCatcodes
}
\Test
\csname @@end\endcsname
\end
%    \end{macrocode}
%    \begin{macrocode}
%</test1>
%    \end{macrocode}
%
% \subsection{Catcode tables}
%
% \subsubsection{Predefined catcode tables}
%
%    \begin{macrocode}
%<*test4>
\NeedsTeXFormat{LaTeX2e}
%    \end{macrocode}
%    Remember \LaTeX's initial catcodes in count registers
%    starting at \cs{TestLaTeX}.
%    \begin{macrocode}
\count0=0 %
\chardef\TestLaTeX=1000 %
\chardef\TestMax=300 %
\loop
  \count\numexpr\TestLaTeX+\count0\relax=\catcode\count0 %
\ifnum\count0<\TestMax
  \advance\count0 by 1 %
\repeat
\documentclass{minimal}
\usepackage{luatex}[2010/03/09]
\usepackage{qstest}
\IncludeTests{*}
\LogTests{log}{*}{*}
\makeatletter
\def\Check#1{%
  \Expect*{\the\count@=\the\catcode\count@}%
         *{\the\count@=#1}%
}
\newcount\scratch
\def\Test#1#2{%
  \begin{qstest}{CatcodeTable#1}{CatcodeTable#1}%
    \luatexcatcodetable\csname CatcodeTable#1\endcsname
    \count@=\z@
    \loop
      \scratch=#2\relax
      \Expect*{\the\count@=\the\catcode\count@}%
             *{\the\count@=\the\scratch}%
    \ifnum\count@<\TestMax
      \advance\count@\@ne
    \repeat
  \end{qstest}%
}
\begingroup
  % luatex-unicode-letters.tex makes some slots to letters
  \def\TestMax{169}%
  \Test{LaTeX}{\the\count\numexpr\TestLaTeX+\count@}%
\endgroup
\Test{String}{\ifnum\count@=32 10\else 12\fi}
\Test{Other}{12}
\luatexinitcatcodetable99 %
\Test{IniTeX}{%
  0\relax
  \begingroup
    \luatexcatcodetable99 %
    \global\scratch=\the\catcode\count@
  \endgroup
}
%    \end{macrocode}
%
% \subsubsection{Catcode table number stack}
%
%    \begin{macrocode}
\begin{qstest}{CatcodeTableNumStack}{CatcodeTableNumStack}
  \def\TestStack#1{%
    \Expect*{\LuT@NumStack}{#1}%
  }%
  \TestStack{0}%
  \PushCatcodeTableNumStack
  \TestStack{{0}0}%
  \@firstofone{%
    \begingroup
      \luatexinitcatcodetable12 %
      \luatexcatcodetable12 %
      \PushCatcodeTableNumStack
      \TestStack{{12}{0}0}%
      \PopCatcodeTableNumStack
      \TestStack{{0}0}%
      \PopCatcodeTableNumStack
      \TestStack{0}%
      \def\TestWarning{Missing empty stack warning}%
      \def\@PackageWarning#1#2{\def\TestWarning{empty stack}}%
      \PopCatcodeTableNumStack
      \TestStack{0}%
      \Expect*{\TestWarning}{empty stack}%
    \endgroup
  }%
\end{qstest}
%    \end{macrocode}
%
% \subsubsection{Catcode table stack}
%
%    \begin{macrocode}
\begin{qstest}{CatcodeTableStack}{CatcodeTableStack}
  \def\TestStack#1{%
    \Expect*{\the\CatcodeTableStack}{#1}%
  }%
  \TestStack{0}%
  \IncCatcodeTableStack
  \TestStack{2}%
  \IncCatcodeTableStack
  \TestStack{4}%
  \begingroup
    \IncCatcodeTableStack
    \TestStack{6}%
  \endgroup
  \TestStack{6}%
  \begingroup
    \DecCatcodeTableStack
    \TestStack{4}%
  \endgroup
  \TestStack{4}%
  \DecCatcodeTableStack
  \TestStack{2}%
  \DecCatcodeTableStack
  \TestStack{0}%
  \begingroup
    \def\TestError{Missing error}%
    \def\@PackageError#1#2#3{%
      \def\TestError{Empty stack}%
    }%
    \DecCatcodeTableStack
    \TestStack{0}%
    \Expect*{\TestError}{Empty stack}%
  \endgroup
\end{qstest}
%    \end{macrocode}
%
% \subsubsection{Catcode regime macros}
%
%    \begin{macrocode}
\begin{qstest}{CatcodeRegime}{CatcodeRegime}
  \def\TestStacks#1#2#3{%
    \Expect*{\the\luatexcatcodetable}{#1}%
    \Expect*{\the\CatcodeTableStack}{#2}%
    \Expect*{\LuT@NumStack}{#3}%
  }%
  \TestStacks{0}{0}{0}%
  \catcode`\|=7 %
  \BeginCatcodeRegime\CatcodeTableLaTeX
    \TestStacks{2}{2}{{0}0}%
    \Expect*{\the\catcode`\|}{12}%
  \EndCatcodeRegime
  \TestStacks{0}{0}{0}%
  \Expect*{\the\catcode`\|}{7}%
\end{qstest}
%    \end{macrocode}
%
% \subsection{Attribute allocation}
%
%    \begin{macrocode}
\begin{qstest}{Attributes}{Attributes}
  \newattribute\TestAttr
  \Expect*{\meaning\TestAttr}%
         *{\string\attribute\number\allocationnumber}%
  \Expect*{\the\allocationnumber}{0}%
  \begingroup
    \newattribute\TestAttr
    \Expect*{\the\allocationnumber}{1}%
  \endgroup
  \Expect*{\the\allocationnumber}{0}%
  \Expect*{\meaning\TestAttr}*{\string\attribute1}%
  \Expect*{\the\TestAttr}*{\number\LuT@UnsetAttributeValue}%
  \def\Test#1{%
    \setattribute\TestAttr{#1}%
    \Expect*{\the\TestAttr}{#1}%
  }%
  \Test{0}%
  \Test{1}%
  \Test{-1}%
  \Test{123}%
  \unsetattribute\TestAttr
  \Expect*{\the\TestAttr}*{\number\LuT@UnsetAttributeValue}%
  \begingroup
    \Expect*{\the\TestAttr}*{\number\LuT@UnsetAttributeValue}%
    \Test{1234}%
  \endgroup
  \Expect*{\the\TestAttr}*{\number\LuT@UnsetAttributeValue}%
\end{qstest}
%    \end{macrocode}
%
%    \begin{macrocode}
\@@end
%</test4>
%    \end{macrocode}
%
% \subsection{Short test for \plainTeX}
%
%    \begin{macrocode}
%<*test5>
\input luatex.sty\relax
\newattribute\TestAttr
\setattribute\TestAttr{10}
\unsetattribute\TestAttr
\newcatcodetable\TestCTa
\begingroup
  \SetCatcodeRange{`A}{`Z}{12}%
\endgroup
\BeginCatcodeRegime\CatcodeTableLaTeX
\EndCatcodeRegime
\end
%</test5>
%    \end{macrocode}
%
% \section{Installation}
%
% \subsection{Download}
%
% \paragraph{Package.} This package is available on
% CTAN\footnote{\url{ftp://ftp.ctan.org/tex-archive/}}:
% \begin{description}
% \item[\CTAN{macros/latex/contrib/oberdiek/luatex.dtx}] The source file.
% \item[\CTAN{macros/latex/contrib/oberdiek/luatex.pdf}] Documentation.
% \end{description}
%
%
% \paragraph{Bundle.} All the packages of the bundle `oberdiek'
% are also available in a TDS compliant ZIP archive. There
% the packages are already unpacked and the documentation files
% are generated. The files and directories obey the TDS standard.
% \begin{description}
% \item[\CTAN{install/macros/latex/contrib/oberdiek.tds.zip}]
% \end{description}
% \emph{TDS} refers to the standard ``A Directory Structure
% for \TeX\ Files'' (\CTAN{tds/tds.pdf}). Directories
% with \xfile{texmf} in their name are usually organized this way.
%
% \subsection{Bundle installation}
%
% \paragraph{Unpacking.} Unpack the \xfile{oberdiek.tds.zip} in the
% TDS tree (also known as \xfile{texmf} tree) of your choice.
% Example (linux):
% \begin{quote}
%   |unzip oberdiek.tds.zip -d ~/texmf|
% \end{quote}
%
% \paragraph{Script installation.}
% Check the directory \xfile{TDS:scripts/oberdiek/} for
% scripts that need further installation steps.
% Package \xpackage{attachfile2} comes with the Perl script
% \xfile{pdfatfi.pl} that should be installed in such a way
% that it can be called as \texttt{pdfatfi}.
% Example (linux):
% \begin{quote}
%   |chmod +x scripts/oberdiek/pdfatfi.pl|\\
%   |cp scripts/oberdiek/pdfatfi.pl /usr/local/bin/|
% \end{quote}
%
% \subsection{Package installation}
%
% \paragraph{Unpacking.} The \xfile{.dtx} file is a self-extracting
% \docstrip\ archive. The files are extracted by running the
% \xfile{.dtx} through \plainTeX:
% \begin{quote}
%   \verb|tex luatex.dtx|
% \end{quote}
%
% \paragraph{TDS.} Now the different files must be moved into
% the different directories in your installation TDS tree
% (also known as \xfile{texmf} tree):
% \begin{quote}
% \def\t{^^A
% \begin{tabular}{@{}>{\ttfamily}l@{ $\rightarrow$ }>{\ttfamily}l@{}}
%   luatex.sty & tex/generic/oberdiek/luatex.sty\\
%   luatex-loader.sty & tex/generic/oberdiek/luatex-loader.sty\\
%   oberdiek.luatex.lua & scripts/oberdiek/oberdiek.luatex.lua\\
%   luatex.pdf & doc/latex/oberdiek/luatex.pdf\\
%   test/luatex-test1.tex & doc/latex/oberdiek/test/luatex-test1.tex\\
%   test/luatex-test2.tex & doc/latex/oberdiek/test/luatex-test2.tex\\
%   test/luatex-test3.tex & doc/latex/oberdiek/test/luatex-test3.tex\\
%   test/luatex-test4.tex & doc/latex/oberdiek/test/luatex-test4.tex\\
%   test/luatex-test5.tex & doc/latex/oberdiek/test/luatex-test5.tex\\
%   luatex.dtx & source/latex/oberdiek/luatex.dtx\\
% \end{tabular}^^A
% }^^A
% \sbox0{\t}^^A
% \ifdim\wd0>\linewidth
%   \begingroup
%     \advance\linewidth by\leftmargin
%     \advance\linewidth by\rightmargin
%   \edef\x{\endgroup
%     \def\noexpand\lw{\the\linewidth}^^A
%   }\x
%   \def\lwbox{^^A
%     \leavevmode
%     \hbox to \linewidth{^^A
%       \kern-\leftmargin\relax
%       \hss
%       \usebox0
%       \hss
%       \kern-\rightmargin\relax
%     }^^A
%   }^^A
%   \ifdim\wd0>\lw
%     \sbox0{\small\t}^^A
%     \ifdim\wd0>\linewidth
%       \ifdim\wd0>\lw
%         \sbox0{\footnotesize\t}^^A
%         \ifdim\wd0>\linewidth
%           \ifdim\wd0>\lw
%             \sbox0{\scriptsize\t}^^A
%             \ifdim\wd0>\linewidth
%               \ifdim\wd0>\lw
%                 \sbox0{\tiny\t}^^A
%                 \ifdim\wd0>\linewidth
%                   \lwbox
%                 \else
%                   \usebox0
%                 \fi
%               \else
%                 \lwbox
%               \fi
%             \else
%               \usebox0
%             \fi
%           \else
%             \lwbox
%           \fi
%         \else
%           \usebox0
%         \fi
%       \else
%         \lwbox
%       \fi
%     \else
%       \usebox0
%     \fi
%   \else
%     \lwbox
%   \fi
% \else
%   \usebox0
% \fi
% \end{quote}
% If you have a \xfile{docstrip.cfg} that configures and enables \docstrip's
% TDS installing feature, then some files can already be in the right
% place, see the documentation of \docstrip.
%
% \subsection{Refresh file name databases}
%
% If your \TeX~distribution
% (\teTeX, \mikTeX, \dots) relies on file name databases, you must refresh
% these. For example, \teTeX\ users run \verb|texhash| or
% \verb|mktexlsr|.
%
% \subsection{Some details for the interested}
%
% \paragraph{Attached source.}
%
% The PDF documentation on CTAN also includes the
% \xfile{.dtx} source file. It can be extracted by
% AcrobatReader 6 or higher. Another option is \textsf{pdftk},
% e.g. unpack the file into the current directory:
% \begin{quote}
%   \verb|pdftk luatex.pdf unpack_files output .|
% \end{quote}
%
% \paragraph{Unpacking with \LaTeX.}
% The \xfile{.dtx} chooses its action depending on the format:
% \begin{description}
% \item[\plainTeX:] Run \docstrip\ and extract the files.
% \item[\LaTeX:] Generate the documentation.
% \end{description}
% If you insist on using \LaTeX\ for \docstrip\ (really,
% \docstrip\ does not need \LaTeX), then inform the autodetect routine
% about your intention:
% \begin{quote}
%   \verb|latex \let\install=y% \iffalse meta-comment
%
% File: luatex.dtx
% Version: 2010/03/09 v0.4
% Info: LuaTeX basic definition package
%
% Copyright (C) 2007, 2009, 2010 by
%    Heiko Oberdiek <heiko.oberdiek at googlemail.com>
%
% This work may be distributed and/or modified under the
% conditions of the LaTeX Project Public License, either
% version 1.3c of this license or (at your option) any later
% version. This version of this license is in
%    http://www.latex-project.org/lppl/lppl-1-3c.txt
% and the latest version of this license is in
%    http://www.latex-project.org/lppl.txt
% and version 1.3 or later is part of all distributions of
% LaTeX version 2005/12/01 or later.
%
% This work has the LPPL maintenance status "maintained".
%
% This Current Maintainer of this work is Heiko Oberdiek.
%
% The Base Interpreter refers to any `TeX-Format',
% because some files are installed in TDS:tex/generic//.
%
% This work consists of the main source file luatex.dtx
% and the derived files
%    luatex.sty, luatex.pdf, luatex.ins, luatex.drv, luatex-loader.sty,
%    luatex-test1.tex, luatex-test2.tex, luatex-test3.tex,
%    luatex-test4.tex, luatex-test5.tex, oberdiek.luatex.lua.
%
% Distribution:
%    CTAN:macros/latex/contrib/oberdiek/luatex.dtx
%    CTAN:macros/latex/contrib/oberdiek/luatex.pdf
%
% Unpacking:
%    (a) If luatex.ins is present:
%           tex luatex.ins
%    (b) Without luatex.ins:
%           tex luatex.dtx
%    (c) If you insist on using LaTeX
%           latex \let\install=y% \iffalse meta-comment
%
% File: luatex.dtx
% Version: 2010/03/09 v0.4
% Info: LuaTeX basic definition package
%
% Copyright (C) 2007, 2009, 2010 by
%    Heiko Oberdiek <heiko.oberdiek at googlemail.com>
%
% This work may be distributed and/or modified under the
% conditions of the LaTeX Project Public License, either
% version 1.3c of this license or (at your option) any later
% version. This version of this license is in
%    http://www.latex-project.org/lppl/lppl-1-3c.txt
% and the latest version of this license is in
%    http://www.latex-project.org/lppl.txt
% and version 1.3 or later is part of all distributions of
% LaTeX version 2005/12/01 or later.
%
% This work has the LPPL maintenance status "maintained".
%
% This Current Maintainer of this work is Heiko Oberdiek.
%
% The Base Interpreter refers to any `TeX-Format',
% because some files are installed in TDS:tex/generic//.
%
% This work consists of the main source file luatex.dtx
% and the derived files
%    luatex.sty, luatex.pdf, luatex.ins, luatex.drv, luatex-loader.sty,
%    luatex-test1.tex, luatex-test2.tex, luatex-test3.tex,
%    luatex-test4.tex, luatex-test5.tex, oberdiek.luatex.lua.
%
% Distribution:
%    CTAN:macros/latex/contrib/oberdiek/luatex.dtx
%    CTAN:macros/latex/contrib/oberdiek/luatex.pdf
%
% Unpacking:
%    (a) If luatex.ins is present:
%           tex luatex.ins
%    (b) Without luatex.ins:
%           tex luatex.dtx
%    (c) If you insist on using LaTeX
%           latex \let\install=y% \iffalse meta-comment
%
% File: luatex.dtx
% Version: 2010/03/09 v0.4
% Info: LuaTeX basic definition package
%
% Copyright (C) 2007, 2009, 2010 by
%    Heiko Oberdiek <heiko.oberdiek at googlemail.com>
%
% This work may be distributed and/or modified under the
% conditions of the LaTeX Project Public License, either
% version 1.3c of this license or (at your option) any later
% version. This version of this license is in
%    http://www.latex-project.org/lppl/lppl-1-3c.txt
% and the latest version of this license is in
%    http://www.latex-project.org/lppl.txt
% and version 1.3 or later is part of all distributions of
% LaTeX version 2005/12/01 or later.
%
% This work has the LPPL maintenance status "maintained".
%
% This Current Maintainer of this work is Heiko Oberdiek.
%
% The Base Interpreter refers to any `TeX-Format',
% because some files are installed in TDS:tex/generic//.
%
% This work consists of the main source file luatex.dtx
% and the derived files
%    luatex.sty, luatex.pdf, luatex.ins, luatex.drv, luatex-loader.sty,
%    luatex-test1.tex, luatex-test2.tex, luatex-test3.tex,
%    luatex-test4.tex, luatex-test5.tex, oberdiek.luatex.lua.
%
% Distribution:
%    CTAN:macros/latex/contrib/oberdiek/luatex.dtx
%    CTAN:macros/latex/contrib/oberdiek/luatex.pdf
%
% Unpacking:
%    (a) If luatex.ins is present:
%           tex luatex.ins
%    (b) Without luatex.ins:
%           tex luatex.dtx
%    (c) If you insist on using LaTeX
%           latex \let\install=y\input{luatex.dtx}
%        (quote the arguments according to the demands of your shell)
%
% Documentation:
%    (a) If luatex.drv is present:
%           latex luatex.drv
%    (b) Without luatex.drv:
%           latex luatex.dtx; ...
%    The class ltxdoc loads the configuration file ltxdoc.cfg
%    if available. Here you can specify further options, e.g.
%    use A4 as paper format:
%       \PassOptionsToClass{a4paper}{article}
%
%    Programm calls to get the documentation (example):
%       pdflatex luatex.dtx
%       makeindex -s gind.ist luatex.idx
%       pdflatex luatex.dtx
%       makeindex -s gind.ist luatex.idx
%       pdflatex luatex.dtx
%
% Installation:
%    TDS:tex/generic/oberdiek/luatex.sty
%    TDS:tex/generic/oberdiek/luatex-loader.sty
%    TDS:scripts/oberdiek/oberdiek.luatex.lua
%    TDS:doc/latex/oberdiek/luatex.pdf
%    TDS:doc/latex/oberdiek/test/luatex-test1.tex
%    TDS:doc/latex/oberdiek/test/luatex-test2.tex
%    TDS:doc/latex/oberdiek/test/luatex-test3.tex
%    TDS:doc/latex/oberdiek/test/luatex-test4.tex
%    TDS:doc/latex/oberdiek/test/luatex-test5.tex
%    TDS:source/latex/oberdiek/luatex.dtx
%
%<*ignore>
\begingroup
  \catcode123=1 %
  \catcode125=2 %
  \def\x{LaTeX2e}%
\expandafter\endgroup
\ifcase 0\ifx\install y1\fi\expandafter
         \ifx\csname processbatchFile\endcsname\relax\else1\fi
         \ifx\fmtname\x\else 1\fi\relax
\else\csname fi\endcsname
%</ignore>
%<*install>
\input docstrip.tex
\Msg{************************************************************************}
\Msg{* Installation}
\Msg{* Package: luatex 2010/03/09 v0.4 LuaTeX basic definition package (HO)}
\Msg{************************************************************************}

\keepsilent
\askforoverwritefalse

\let\MetaPrefix\relax
\preamble

This is a generated file.

Project: luatex
Version: 2010/03/09 v0.4

Copyright (C) 2007, 2009, 2010 by
   Heiko Oberdiek <heiko.oberdiek at googlemail.com>

This work may be distributed and/or modified under the
conditions of the LaTeX Project Public License, either
version 1.3c of this license or (at your option) any later
version. This version of this license is in
   http://www.latex-project.org/lppl/lppl-1-3c.txt
and the latest version of this license is in
   http://www.latex-project.org/lppl.txt
and version 1.3 or later is part of all distributions of
LaTeX version 2005/12/01 or later.

This work has the LPPL maintenance status "maintained".

This Current Maintainer of this work is Heiko Oberdiek.

The Base Interpreter refers to any `TeX-Format',
because some files are installed in TDS:tex/generic//.

This work consists of the main source file luatex.dtx
and the derived files
   luatex.sty, luatex.pdf, luatex.ins, luatex.drv, luatex-loader.sty,
   luatex-test1.tex, luatex-test2.tex, luatex-test3.tex,
   luatex-test4.tex, luatex-test5.tex, oberdiek.luatex.lua.

\endpreamble
\let\MetaPrefix\DoubleperCent

\generate{%
  \file{luatex.ins}{\from{luatex.dtx}{install}}%
  \file{luatex.drv}{\from{luatex.dtx}{driver}}%
  \usedir{tex/generic/oberdiek}%
  \file{luatex.sty}{\from{luatex.dtx}{package}}%
  \file{luatex-loader.sty}{\from{luatex.dtx}{loader}}%
  \usedir{doc/latex/oberdiek/test}%
  \file{luatex-test1.tex}{\from{luatex.dtx}{test1}}%
  \file{luatex-test2.tex}{\from{luatex.dtx}{test1,test2}}%
  \file{luatex-test3.tex}{\from{luatex.dtx}{test1,test3}}%
  \file{luatex-test4.tex}{\from{luatex.dtx}{test4}}%
  \file{luatex-test5.tex}{\from{luatex.dtx}{test5}}%
  \nopreamble
  \nopostamble
  \usedir{source/latex/oberdiek/catalogue}%
  \file{luatex.xml}{\from{luatex.dtx}{catalogue}}%
}
\def\MetaPrefix{-- }
\def\defaultpostamble{%
  \MetaPrefix^^J%
  \MetaPrefix\space End of File `\outFileName'.%
}
\def\currentpostamble{\defaultpostamble}%
\generate{%
  \usedir{scripts/oberdiek}%
  \file{oberdiek.luatex.lua}{\from{luatex.dtx}{lua}}%
}

\catcode32=13\relax% active space
\let =\space%
\Msg{************************************************************************}
\Msg{*}
\Msg{* To finish the installation you have to move the following}
\Msg{* files into a directory searched by TeX:}
\Msg{*}
\Msg{*     luatex.sty, luatex-loader.sty}
\Msg{*}
\Msg{* And install the following script file:}
\Msg{*}
\Msg{*     oberdiek.luatex.lua}
\Msg{*}
\Msg{* To produce the documentation run the file `luatex.drv'}
\Msg{* through LaTeX.}
\Msg{*}
\Msg{* Happy TeXing!}
\Msg{*}
\Msg{************************************************************************}

\endbatchfile
%</install>
%<*ignore>
\fi
%</ignore>
%<*driver>
\NeedsTeXFormat{LaTeX2e}
\ProvidesFile{luatex.drv}%
  [2010/03/09 v0.4 LuaTeX basic definition package (HO)]%
\documentclass{ltxdoc}
\usepackage{holtxdoc}[2011/11/22]
\usepackage{paralist}
\begin{document}
  \DocInput{luatex.dtx}%
\end{document}
%</driver>
% \fi
%
% \CheckSum{1429}
%
% \CharacterTable
%  {Upper-case    \A\B\C\D\E\F\G\H\I\J\K\L\M\N\O\P\Q\R\S\T\U\V\W\X\Y\Z
%   Lower-case    \a\b\c\d\e\f\g\h\i\j\k\l\m\n\o\p\q\r\s\t\u\v\w\x\y\z
%   Digits        \0\1\2\3\4\5\6\7\8\9
%   Exclamation   \!     Double quote  \"     Hash (number) \#
%   Dollar        \$     Percent       \%     Ampersand     \&
%   Acute accent  \'     Left paren    \(     Right paren   \)
%   Asterisk      \*     Plus          \+     Comma         \,
%   Minus         \-     Point         \.     Solidus       \/
%   Colon         \:     Semicolon     \;     Less than     \<
%   Equals        \=     Greater than  \>     Question mark \?
%   Commercial at \@     Left bracket  \[     Backslash     \\
%   Right bracket \]     Circumflex    \^     Underscore    \_
%   Grave accent  \`     Left brace    \{     Vertical bar  \|
%   Right brace   \}     Tilde         \~}
%
% \GetFileInfo{luatex.drv}
%
% \title{The \xpackage{luatex} package}
% \date{2010/03/09 v0.4}
% \author{Heiko Oberdiek\\\xemail{heiko.oberdiek at googlemail.com}}
%
% \maketitle
%
% \begin{abstract}
% This package manages the new and extended features and resources
% that \LuaTeX\ provides. Examples are attributes and catcode
% tables.
% \end{abstract}
%
% \tableofcontents
%
% \section{Documentation}
%
% \subsection{Introduction}
%
% \TeX\ provides global resources such as registers.
% But it does not provide an interface for managing these resources.
% For example, two packages want to use a counter register.
% If they take the same register
% number, then the use of both packages will conflict and they cannot
% be used together.
% Therefore formats such as \plainTeX\ or \LaTeX\ implement
% an allocation scheme for registers. A package reserves with
% \cs{newcount} an unused register number for its own exclusive use.
%
% Nowadays \TeX\ is not alone anymore: \eTeX, \pdfTeX\ and other
% compilers for \TeX\ are developed that extend and add new
% features and resources.
%
% Now \LuaTeX\ has reached beta state. It inherits most of
% \pdfTeX's features including \eTeX. Also it implements
% new concepts such as attributes or catcode tables.
%
% \subsubsection{\LaTeX}
%
% \LaTeXe\ is frozen and therefore refuses to even notice
% the new \TeX\ variants. Not even the old \eTeX\ is supported
% by its kernel. At least there is a third party package \xpackage{etex}
% that manages the new \eTeX\ resources.
%
% This package tries to do the same for \LuaTeX\ and starts to
% support at least a few of the new features.
%
% \subsubsection{\plainTeX}
%
% \LaTeX\ has inherited its resource handling from \plainTeX.
% The interface is basically the same: \cs{newcount}, \dots
% Therefore this package tries to follow this tradition by
% providing compatibility to \plainTeX. It can be loaded
% with \plainTeX\ and defines at least some of the
% features that this packages provides for \LaTeX.
%
% \subsection{Register allocation}
%
% \subsubsection{Register with 16 bit}
%
% Because \LuaTeX\ is a super set of \eTeX\ regarding registers,
% the register allocation scheme should not conflict with
% package \xpackage{etex}. Therefore this package is loaded
% to inherit its allocation scheme. The only change is currently
% that the limit is increased to 65536 registers for the following
% register classes:
% \begin{compactitem}
% \item count
% \item dimen
% \item skip
% \item muskip
% \item marks
% \item toks
% \item box
% \end{compactitem}
% This affects the number of global and local registers.
% Because it is done in a package and not in the kernel,
% it is possible that someone loads package \xpackage{etex}
% before uses the local allocation variants. This will prevent
% the extension for this register class. If more registers are
% needed, just load package \xpackage{luatex} earlier.
%
% \subsubsection{Insertions}
%
% Insertions need four registers \cs{count}, \cs{dimen}, \cs{skip},
% and \cs{box} with the same number. Usually they are allocated
% downwards from 254, 253, \dots Also \cs{newcount}, \cs{newdimen}, \dots
% fill up these register numbers from below before switching to
% higher register numbers by package \xpackage{etex}.
% When this occurs, no insertions can be allocated anymore.
%
% Therefore \cs{newcount}, \cs{newdimen}, \cs{newskip}, and
% \cs{newbox} are replaced by their global variants (\cs{globcount}, \dots)
% that use the higher numbers immediately, leaving the room for insertions.
% There should not be an efficiency penalty because \LuaTeX\ stores
% the registers of a class in the same Lua table unlike \eTeX, where
% registers below 256 are stored in an array and higher numbers
% are put in a tree structure.
%
% \subsection{Attributes}
%
% Nodes can have custom attributes in \LuaTeX. These attributes
% are organized by a new register class. As the other registers
% up to $2^{16}$ attributes are supported. An attribute value
% can be negative that means the attribute is not set. Otherwise
% \TeX's range of non-negative integers up to $2^31$ are available.
%
% \begin{declcs}{newattribute} \M{cmd}
% \end{declcs}
% Macro \cs{newattribute} defines command \meta{cmd} using \cs{attributedef}
% using an new attribute number. The new attribute is initially unset.
%
% \begin{declcs}{setattribute} \M{cmd} \M{value}
% \end{declcs}
% Macro \cs{setattribute} locally sets attribute command \meta{cmd}
% to the number \meta{value}. Valid values range from $-1$ until $2^31$
% (the upper limit is the same as for other \TeX\ integer numbers).
%
% \begin{declcs}{unsetattribute} \M{cmd}
% \end{declcs}
% Macro \cs{unsettattribute} clears the attribute command \meta{cmd}.
%
% \subsection{Catcode tables}
%
% \LuaTeX\ introduces catcode tables as new feature, see documentation.
% There is need for discussion, how to deal best:
% \begin{itemize}
% \item \cs{initcatcodetable} and \cs{setcatcodetable} act globally.
% \item \cs{catcodetable} causes an error if used with an uninitialized
%   catcode table.
% \item Large catcode table numbers should be avoided because of
%   performance breakdown.
% \item Use case \LaTeX\ package:
%   The package must not be surprised by changed catcodes and
%   must not surprise by changing catcodes accidently.
%   Catcode tables could offer a solution. At the begin
%   a catcode regime with standard catcodes is established
%   and the old one is restored afterwards.
% \item Use case: \LuaTeX's |tex.print| might be used with
%   a catcode table number, for example a table where all
%   entries have catcode ``other''.
% \item Readonly catcode tables.
% \item Is there is a need for local allocations?
%   (Package \xpackage{etex}'s \cs{loc} variants are not used in
%   \TeX\ Live 2007.)
% \end{itemize}
%
% \subsubsection{Interface proposal}
%
% The idea: \cs{newcatcodetable} allocates odd numbered catcode
% tables. Even numbered tables are managed as stack.
% Also some catcode tables are defined. These must not be
% changed.
%
% \begin{declcs}{newcatcodetable} \M{cmd}
% \end{declcs}
% Macro \cs{newcatcodetable} reserves a new catcode table and
% remembers its number in \meta{cmd}. The catcode table is
% initialized with \iniTeX's catcodes.
%
% \begin{declcs}{CatcodeTableIniTeX}\\
%   \cs{CatcodeTableString}\\
%   \cs{CatcodeTableOther}\\
%   \cs{CatcodeTableLaTeX}\\
% \end{declcs}
% These are catcode tables and must not be changed.
% \cs{CatcodeTableIniTeX} contains the catcode settings of \iniTeX.
% \cs{CatcodeTableString} follows \TeX's convention of
% \cs{string}, \cs{meaning} and friends. The space gets catcode
% 10 (space), the other characters have catcode 12 (other).
% In \cs{CatcodeTableOther} all entries have catcode 12 (other).
% \cs{CatcodeTableLaTeX} contains the setting of a pure \LaTeX\ format
% (`at' is other).
%
% \begin{declcs}{CatcodeTableStack}\\
%   \cs{IncCatcodeTableStack}\\
%   \cs{DecCatcodeTableStack}
% \end{declcs}
% \cs{CatcodeTableStack} is the stack pointer. Initially
% it is catcode table zero. \cs{IncCatcodeTableStack} and
% \cs{DecCatcodeTableStack} increments and decrements the
% stack pointer. Currently \cs{IncCatcodeTableStack} does not
% initialize a new catcode table. Both increment and decrement
% operations do not set a catcode table.
%
% \begin{declcs}{PushCatcodeTableNumStack}\\
%   \cs{PopCatcodeTableNumStack}
% \end{declcs}
% It can be handy to have a global stack for catcode table
% numbers to deal with the global assignment property of
% \cs{initcatcodetable} and \cs{savecatcodetable}.
% \cs{PushCatcodeTableNumStack} pushes the current catcode table
% on the stack.
% \cs{PopCatcodeTableNumStack} pops the topmost number off the number
% stack to set the current catcode table. Catcode table zero is
% used in case of an empty stack.
%
% \begin{declcs}{BeginCatcodeRegime} \M{catcodetable}\\
%   \cs{EndCatcodeRegime}
% \end{declcs}
% \cs{BeginCatcodeRegime} remembers the current catcode table number.
% Then it creates and uses a fresh catcode table on the stack
% that is initialized by \meta{catcodetable}:
% \begin{quote}\ttfamily
%   \cs{PushCatcodeTableNumStack}\\
%   \cs{catcodetable}\meta{catcodetable}||
%   \cs{IncCatcodeTableStack}\\
%   \cs{savecatcodetable}\cs{CatcodeTableStack}\\
%   \cs{catcodetable}\cs{CatcodeTableStack}
% \end{quote}
% \cs{EndCatcodeRegime} drops the catcode table, created by
% \cs{BeginCatcodeRegime} and sets the catcode table that was
% active before:
% \begin{quote}
%   \cs{DecCatcodeTableStack}\\
%   \cs{PopCatcodeTableNumStack}
% \end{quote}
% These macros solve the use case, described earlier for a \LaTeX\ package:
% \begin{quote}
%   |% package foobar.sty|\\
%   |\BeginCatcodeRegime\CatcodeTableLaTeX|\\
%   |\makeatletter|\\
%   |% ... package contents ...|\\
%   |\EndCatcodeRegime|\\
%   |% end of package|
% \end{quote}
% If the package wants to change catcodes after its loading,
% \cs{AtBeginDocument} or \cs{AtEndOfPackage} can be used.
%
% \begin{declcs}{SetCatcodeRange} \M{from} \M{to} \M{catcode}
% \end{declcs}
% The catcodes of characters in range from \meta{from}
% to inclusive \meta{to} are set to \meta{catcode}.
%
% \subsection{Lua module loading}
%
% Currently \LuaTeX\ (version 0.20) does not support Lua script
% files inside \xfile{TDS:scripts//}, because Lua's mechanism
% for module loading does not use the \textsf{kpathsea} library.
% Therefore this packages appends a kpse loader to the list of
% Lua's module loaders. It finds the module \meta{module} by
% \begin{quote}
%   \ttfamily |kpse.find_file("|\meta{module}|.lua", "texmfscripts")|
% \end{quote}
% Unhappily \textsf{kpathsea} does not support directory components
% in a file name. Therefore the Lua convention is not followed to
% replace dots in the module name by the directory separator.
%
% Example:
% A Lua script of a package \xpackage{foobar} wants the following
% modules:
% \begin{quote}
%   |require("foobar.hello.world")|\\
%   |require("org.somewhere.xyz")|
% \end{quote}
% Then they can be find in:
% \begin{quote}
%   |TDS:scripts/foobar/foobar.hello.world.lua|\\
%   |TDS:scripts/foobar/org.somewhere.xyz.lua|
% \end{quote}
% I would have preferred the following locations, following lua conventions,
% e.\,g.:
% \begin{quote}
%   |TDS:scripts/foobar/hello/world.lua|\\
%   |TDS:scripts/foobar/org/somewhere/xyz.lua|
% \end{quote}
% But I do not know, how to achieve this in a reliable way using
% \textsf{kpathsea}.
%
% \subsubsection{Package \xpackage{luatex-loader}}
%
% If someone do not need or want package \xpackage{luatex} but
% it's extension for module loading, then he can use
% package \xpackage{luatex-loader}. Both \plainTeX\ and \LaTeX\ are
% supported.
%
% \StopEventually{
% }
%
% \section{Implementation}
%
%    \begin{macrocode}
%<*package>
%    \end{macrocode}
%
% \subsection{Reload check and package identification}
%    Reload check, especially if the package is not used with \LaTeX.
%    \begin{macrocode}
\begingroup\catcode61\catcode48\catcode32=10\relax%
  \catcode13=5 % ^^M
  \endlinechar=13 %
  \catcode35=6 % #
  \catcode39=12 % '
  \catcode44=12 % ,
  \catcode45=12 % -
  \catcode46=12 % .
  \catcode58=12 % :
  \catcode64=11 % @
  \catcode123=1 % {
  \catcode125=2 % }
  \expandafter\let\expandafter\x\csname ver@luatex.sty\endcsname
  \ifx\x\relax % plain-TeX, first loading
  \else
    \def\empty{}%
    \ifx\x\empty % LaTeX, first loading,
      % variable is initialized, but \ProvidesPackage not yet seen
    \else
      \expandafter\ifx\csname PackageInfo\endcsname\relax
        \def\x#1#2{%
          \immediate\write-1{Package #1 Info: #2.}%
        }%
      \else
        \def\x#1#2{\PackageInfo{#1}{#2, stopped}}%
      \fi
      \x{luatex}{The package is already loaded}%
      \aftergroup\endinput
    \fi
  \fi
\endgroup%
%    \end{macrocode}
%    Package identification:
%    \begin{macrocode}
\begingroup\catcode61\catcode48\catcode32=10\relax%
  \catcode13=5 % ^^M
  \endlinechar=13 %
  \catcode35=6 % #
  \catcode39=12 % '
  \catcode40=12 % (
  \catcode41=12 % )
  \catcode44=12 % ,
  \catcode45=12 % -
  \catcode46=12 % .
  \catcode47=12 % /
  \catcode58=12 % :
  \catcode64=11 % @
  \catcode91=12 % [
  \catcode93=12 % ]
  \catcode123=1 % {
  \catcode125=2 % }
  \expandafter\ifx\csname ProvidesPackage\endcsname\relax
    \def\x#1#2#3[#4]{\endgroup
      \immediate\write-1{Package: #3 #4}%
      \xdef#1{#4}%
    }%
  \else
    \def\x#1#2[#3]{\endgroup
      #2[{#3}]%
      \ifx#1\@undefined
        \xdef#1{#3}%
      \fi
      \ifx#1\relax
        \xdef#1{#3}%
      \fi
    }%
  \fi
\expandafter\x\csname ver@luatex.sty\endcsname
\ProvidesPackage{luatex}%
  [2010/03/09 v0.4 LuaTeX basic definition package (HO)]%
%    \end{macrocode}
%
% \subsection{Catcodes}
%
%    \begin{macrocode}
\begingroup\catcode61\catcode48\catcode32=10\relax%
  \catcode13=5 % ^^M
  \endlinechar=13 %
  \catcode123=1 % {
  \catcode125=2 % }
  \catcode64=11 % @
  \def\x{\endgroup
    \expandafter\edef\csname LuT@AtEnd\endcsname{%
      \endlinechar=\the\endlinechar\relax
      \catcode13=\the\catcode13\relax
      \catcode32=\the\catcode32\relax
      \catcode35=\the\catcode35\relax
      \catcode61=\the\catcode61\relax
      \catcode64=\the\catcode64\relax
      \catcode123=\the\catcode123\relax
      \catcode125=\the\catcode125\relax
    }%
  }%
\x\catcode61\catcode48\catcode32=10\relax%
\catcode13=5 % ^^M
\endlinechar=13 %
\catcode35=6 % #
\catcode64=11 % @
\catcode123=1 % {
\catcode125=2 % }
\def\TMP@EnsureCode#1#2{%
  \edef\LuT@AtEnd{%
    \LuT@AtEnd
    \catcode#1=\the\catcode#1\relax
  }%
  \catcode#1=#2\relax
}
\TMP@EnsureCode{10}{12}% ^^J
\TMP@EnsureCode{34}{12}% "
\TMP@EnsureCode{36}{3}% $
\TMP@EnsureCode{39}{12}% '
\TMP@EnsureCode{40}{12}% (
\TMP@EnsureCode{41}{12}% )
\TMP@EnsureCode{42}{12}% *
\TMP@EnsureCode{43}{12}% +
\TMP@EnsureCode{44}{12}% ,
\TMP@EnsureCode{45}{12}% -
\TMP@EnsureCode{46}{12}% .
\TMP@EnsureCode{47}{12}% /
\TMP@EnsureCode{60}{12}% <
\TMP@EnsureCode{62}{12}% >
\TMP@EnsureCode{91}{12}% [
\TMP@EnsureCode{93}{12}% ]
\TMP@EnsureCode{95}{12}% _ (other!)
\TMP@EnsureCode{96}{12}% `
\edef\LuT@AtEnd{\LuT@AtEnd\noexpand\endinput}
%    \end{macrocode}
%
% \subsection{Check for \LuaTeX}
%
%    Without \LuaTeX\ there is no point in using this package.
%    \begin{macrocode}
\begingroup\expandafter\expandafter\expandafter\endgroup
\expandafter\ifx\csname RequirePackage\endcsname\relax
  \input infwarerr.sty\relax
  \input ifluatex.sty\relax
\else
  \RequirePackage{infwarerr}[2007/09/09]%
  \RequirePackage{ifluatex}[2009/04/10]%
\fi
%    \end{macrocode}
%    \begin{macrocode}
\ifluatex
\else
  \@PackageError{luatex}{%
    This package may only be run using LuaTeX%
  }\@ehc
  \expandafter\LuT@AtEnd
\fi%
%    \end{macrocode}
%
% \subsection{Provide Lua\TeX\ primitives}
%
%    \begin{macrocode}
\ifnum\luatexversion<36 %
  \def\LuT@MakePrimitive#1{%
    \expandafter\let\csname luatex#1\expandafter\endcsname
    \csname #1\endcsname
  }%
\else
  \def\LuT@MakeLuatexPrimitive#1{%
    \begingroup\expandafter\expandafter\expandafter\endgroup
    \expandafter\ifx\csname luatex#1\endcsname\relax
      \begingroup\expandafter\expandafter\expandafter\endgroup
      \expandafter\ifx\csname #1\endcsname\relax
      \else
        \expandafter\let
        \csname luatex#1\expandafter\endcsname
        \csname #1\endcsname
      \fi
    \fi
    \begingroup\expandafter\expandafter\expandafter\endgroup
    \expandafter\ifx\csname luatex#1\endcsname\relax
      \begingroup
        \expandafter\let\csname luatex#1\endcsname\@undefined
        \ifnum0%
          \directlua{%
            if tex.enableprimitives then %
              tex.enableprimitives('luatex',{'#1'})%
              tex.print('1')%
            end%
          }%
          \expandafter\ifx\csname luatex#1\endcsname\relax\else1\fi
        =11 %
          \global\expandafter\let
          \csname luatex#1\expandafter\endcsname
          \csname luatex#1\endcsname
        \else
          \@PackageError{luatex}{%
            tex.enableprimitives failed for `#1'%
          }\@ehc
        \fi
      \endgroup
    \fi
  }%
  \def\LuT@MakePrimitive#1{%
    \begingroup\expandafter\expandafter\expandafter\endgroup
    \expandafter\ifx\csname#1\endcsname\relax
      \begingroup
        \expandafter\let\csname#1\endcsname\@undefined
        \ifnum0%
          \directlua{%
            if tex.enableprimitives then %
              tex.enableprimitives('',{'#1'})%
              tex.print('1')%
            end%
          }%
          \expandafter\ifx\csname#1\endcsname\relax\else1\fi
        =11 %
          \global\expandafter\let
          \csname#1\expandafter\endcsname
          \csname#1\endcsname
        \else
          \@PackageError{luatex}{%
            tex.enableprimitives failed for `#1'%
          }\@ehc
        \fi
      \endgroup
    \fi
  }%
\fi
\LuT@MakeLuatexPrimitive{attribute}
\LuT@MakeLuatexPrimitive{attributedef}
\LuT@MakeLuatexPrimitive{catcodetable}
\LuT@MakeLuatexPrimitive{initcatcodetable}
\LuT@MakeLuatexPrimitive{luaescapestring}
\LuT@MakeLuatexPrimitive{savecatcodetable}
\LuT@MakePrimitive{numexpr}
%    \end{macrocode}
%
% \subsection{Inherit support for \eTeX}
%
%    Package \xpackage{etex} is not compatible for \plainTeX.
%    But it could be present if a format is used that is based
%    on \xfile{etex.src}. Therefore we only load the package
%    in case of \LaTeX\ and tests its presence independently
%    of the format by looking for \cs{et@xins}.
%    \begin{macrocode}
\begingroup\expandafter\expandafter\expandafter\endgroup
\expandafter\ifx\csname RequirePackage\endcsname\relax
\else
  \RequirePackage{etex}[1998/03/26]%
\fi
%    \end{macrocode}
%
% \subsection{Adaption of \eTeX's register allocation}
%
%    \eTeX has increased the number of \TeX\ registers
%    from $2^8$ (256) to $2^{15}$ (32768) for a register class.
%    \LuaTeX\ extends the limit further to $2^16$ (65536).
%    The allocation scheme of package \xpackage{etex} is not
%    changed. But this can be subject for discussion.
%
%    If a register class hasn't registered any local registers
%    yet, then the limit can safely be pushed to 65536.
%    \begin{macrocode}
\begingroup\expandafter\expandafter\expandafter\endgroup
\expandafter\ifx\csname et@xins\endcsname\relax
  \@PackageWarningNoLine{luatex}{%
    Support for eTeX is not loaded (etex.src)%
  }%
\else
  \def\LuT@temp#1{%
    \ifnum\count27#1=32768 %
      \count27#1=65536 %
    \fi
  }%
  \LuT@temp0%
  \LuT@temp1%
  \LuT@temp2%
  \LuT@temp3%
  \LuT@temp4%
  \LuT@temp5%
  \LuT@temp6%
%    \end{macrocode}
%
% \eTeX\ uses an array for the first 256 registers and then a tree
% structure. \LuaTeX\ stores all registers of a class in one Lua table.
% There shouldn't be large performance differences.
% This allows starting immediately in the extended
% area, leaving room for insertions.
%    \begin{macrocode}
  \let\newcount\globcount
  \let\newdimen\globdimen
  \let\newskip\globskip
  \let\newbox\globbox
\fi
%    \end{macrocode}
%
% \subsection{\plainTeX\ compatibility}
%
%    \begin{macro}{\@empty}
%    \begin{macrocode}
\expandafter\ifx\csname @empty\endcsname\relax
  \def\@empty{}%
\fi
%    \end{macrocode}
%    \end{macro}
%    \begin{macro}{\@gobble}
%    \begin{macrocode}
\expandafter\ifx\csname @gobble\endcsname\relax
  \long\def\@gobble#1{}%
\fi
%    \end{macrocode}
%    \end{macro}
%    \begin{macro}{\@firstofone}
%    \begin{macrocode}
\expandafter\ifx\csname @firstofone\endcsname\relax
  \long\def\@firstofone#1{#1}%
\fi
%    \end{macrocode}
%    \end{macro}
%    \begin{macro}{\@firstoftwo}
%    \begin{macrocode}
\expandafter\ifx\csname @firstoftwo\endcsname\relax
  \long\def\@firstoftwo#1#2{#1}%
\fi
%    \end{macrocode}
%    \end{macro}
%    \begin{macro}{\@car}
%    \begin{macrocode}
\expandafter\ifx\csname @car\endcsname\relax
  \def\@car#1#2\@nil{#1}%
\fi
%    \end{macrocode}
%    \end{macro}
%    \begin{macro}{\@cdr}
%    \begin{macrocode}
\expandafter\ifx\csname @cdr\endcsname\relax
  \def\@cdr#1#2\@nil{#2}%
\fi
%    \end{macrocode}
%    \end{macro}
%    \begin{macro}{\@ifstar}
%    \begin{macrocode}
\expandafter\ifx\csname @ifstar\endcsname\relax
  \def\@ifstar#1{%
    \@ifnextchar*{\@firstoftwo{#1}}%
  }%
%    \end{macrocode}
%    \end{macro}
%    \begin{macro}{\@ifnextchar}
%    \begin{macrocode}
  \long\def\@ifnextchar#1#2#3{%
    \let\reserved@d=#1%
    \def\reserved@a{#2}%
    \def\reserved@b{#3}%
    \futurelet\@let@token\@ifnch
  }%
%    \end{macrocode}
%    \end{macro}
%    \begin{macro}{\@ifnch}
%    \begin{macrocode}
  \def\@ifnch{%
    \ifx\@let@token\@sptoken
      \let\reserved@c\@xifnch
    \else
      \ifx\@let@token\reserved@d
        \let\reserved@c\reserved@a
      \else
        \let\reserved@c\reserved@b
      \fi
    \fi
    \reserved@c
  }%
%    \end{macrocode}
%    \end{macro}
%    \begin{macro}{\@sptoken}
%    \begin{macrocode}
  \let\LuT@temp\:%
  \def\:{\let\@sptoken= }%
  \: % explicit space
%    \end{macrocode}
%    \end{macro}
%    \begin{macro}{\@xifnch}
%    \begin{macrocode}
  \def\:{\@xifnch}%
  \expandafter\def\: {%
    \futurelet\@let@token\@ifnch
  }%
  \let\:\LuT@temp
\fi
%    \end{macrocode}
%    \end{macro}
%
%    \begin{macro}{\@tempcnta}
%    \begin{macrocode}
\expandafter\ifx\csname @tempcnta\endcsname\relax
  \csname newcount\endcsname\@tempcnta
\fi
%    \end{macrocode}
%    \end{macro}
%
%    \begin{macro}{\@tempcntb}
%    \begin{macrocode}
\expandafter\ifx\csname @tempcntb\endcsname\relax
  \csname newcount\endcsname\@tempcntb
\fi
%    \end{macrocode}
%    \end{macro}
%
%    \begin{macro}{\LuT@newcommand}
%    \begin{macrocode}
\begingroup\expandafter\expandafter\expandafter\endgroup
\expandafter\ifx\csname newcommand\endcsname\relax
  \def\LuT@newcommand#1[#2]#3{%
    \ifx#1\@undefined
      \let#1\relax
    \else
      \ifx#1\relax
      \else
        \@PackageError{luatex}{%
          \string#1 is already defined.\MessageBreak
          Redefinition is skipped%
        }\@ehc
      \fi
    \fi
    \ifx#1\relax
      \ifcase#2 %
        \def#1{#3}%
      \or
        \def#1##1{#3}%
      \or
        \def#1##1##2{#3}%
      \or
        \def#1##1##2##3{#3}%
      \or
        \@INTERNAL@ERROR
      \fi
    \fi
  }%
\else
  \def\LuT@newcommand{\newcommand*}%
\fi
%    \end{macrocode}
%    \end{macro}
%
% \subsection{Attributes}
%
% \subsubsection{Allocation}
%
%    \begin{macro}{\LuT@AllocAttribute}
%    \begin{macrocode}
\newcount\LuT@AllocAttribute
\LuT@AllocAttribute=\m@ne
%    \end{macrocode}
%    \end{macro}
%    \begin{macro}{\newattribute}
%    \begin{macrocode}
\LuT@newcommand\newattribute[1]{%
  \ifnum\LuT@AllocAttribute<65535 %
    \global\advance\LuT@AllocAttribute\@ne
    \allocationnumber\LuT@AllocAttribute
    \global\luatexattributedef#1=\allocationnumber
    \unsetattribute{#1}%
    \wlog{\string#1=\string\attribute\the\allocationnumber}%
  \else
    \errmessage{No room for a new \string\attribute}%
  \fi
}
%    \end{macrocode}
%    \end{macro}
%
% \subsubsection{Interface}
%
%    \begin{macro}{\setattribute}
%    \begin{macrocode}
\LuT@newcommand\setattribute[2]{%
  #1=\numexpr#2\relax
}
%    \end{macrocode}
%    \end{macro}
%    \begin{macro}{\unsetattribute}
%    \begin{macrocode}
\ifnum\luatexversion<37
  \LuT@newcommand\LuT@UnsetAttributeValue[0]{}%
  \let\LuT@UnsetAttributeValue\m@ne
\else
  \LuT@newcommand\LuT@UnsetAttributeValue[0]{-2147483647 }%
\fi
\LuT@newcommand\unsetattribute[1]{%
  #1=\LuT@UnsetAttributeValue
}
%    \end{macrocode}
%    \end{macro}
%
% \subsection{Catcode tables}
%
% \subsubsection{Allocation}
%
%    \begin{macro}{\LuT@AllocCatcodeTable}
%    \begin{macrocode}
\newcount\LuT@AllocCatcodeTable
\LuT@AllocCatcodeTable=\m@ne
\newcount\CatcodeTableStack
\CatcodeTableStack=\z@
%    \end{macrocode}
%    \end{macro}
%    \begin{macro}{\newcatcodetable}
%    \begin{macrocode}
\LuT@newcommand\newcatcodetable[1]{%
  \ifnum\LuT@AllocCatcodeTable<1114110 % 0x10FFFF is maximal \chardef
    % or < 268435455 % 2^28 - 1
    \global\advance\LuT@AllocCatcodeTable by\tw@
    \allocationnumber=\LuT@AllocCatcodeTable
    \global\chardef#1=\allocationnumber
    \wlog{%
      \string#1=\string\catcodetable\the\allocationnumber
    }%
  \else
    \errmessage{No room for a new \string\catcodetable}%
  \fi
}%
%    \end{macrocode}
%    \end{macro}
%
%    \begin{macro}{\IncCatcodeTableStack}
%    \begin{macrocode}
\LuT@newcommand\IncCatcodeTableStack[0]{%
  \ifnum\CatcodeTableStack<268435454 %
    \global\advance\CatcodeTableStack by\tw@
  \else
    \@PackageError{luatex}{%
      Catcode table stack overflow%
    }\@ehd
  \fi
}
%    \end{macrocode}
%    \end{macro}
%
%    \begin{macro}{\DecCatcodeTableStack}
%    \begin{macrocode}
\LuT@newcommand\DecCatcodeTableStack[0]{%
  \ifnum\CatcodeTableStack>\z@
    \global\advance\CatcodeTableStack by-2 %
  \else
    \@PackageError{luatex}{%
      Catcode table stack is empty%
    }\@ehd
  \fi
}
%    \end{macrocode}
%    \end{macro}
%
% \subsubsection{\cs{SetCatcodeRange}}
%
%    \begin{macro}{\SetCatcodeRange}
%    \begin{macrocode}
\LuT@newcommand\SetCatcodeRange[3]{%
  \edef\LuT@temp{%
    \noexpand\@tempcnta=\the\@tempcnta
    \noexpand\@tempcntb=\the\@tempcntb
    \noexpand\count@=\the\count@
    \relax
  }%
  \@tempcnta=\numexpr#1\relax
  \@tempcntb=\numexpr#2\relax
  \count@=\numexpr#3\relax
  \loop
    \unless\ifnum\@tempcnta>\@tempcntb
    \catcode\@tempcnta=\count@
    \advance\@tempcnta by \@ne
  \repeat
  \LuT@temp
}
%    \end{macrocode}
%    \end{macro}
%
% \subsubsection{Predefined catcode tables}
%
%    \begin{macrocode}
\newcatcodetable\CatcodeTableIniTeX
\newcatcodetable\CatcodeTableString
\newcatcodetable\CatcodeTableOther
\newcatcodetable\CatcodeTableLaTeX
%    \end{macrocode}
%
%    \begin{macrocode}
\luatexinitcatcodetable\CatcodeTableIniTeX
\begingroup
  \def\@makeother#1{\catcode#1=12\relax}%
  \@firstofone{%
    \luatexcatcodetable\CatcodeTableIniTeX
    \begingroup
      \SetCatcodeRange{0}{8}{15}%
      \catcode9=10 % tab
      \catcode11=15 %
      \catcode12=13 % form feed
      \SetCatcodeRange{14}{31}{15}%
      \catcode35=6 % hash
      \catcode36=3 % dollar
      \catcode38=4 % ampersand
      \catcode94=7 % circumflex
      \catcode95=8 % underscore
      \catcode123=1 % brace left
      \catcode125=2 % brace right
      \catcode126=13 % tilde
      \catcode127=15 %
      \luatexsavecatcodetable\CatcodeTableLaTeX
    \endgroup
    \@makeother{0}% nul
    \@makeother{13}% carriage return
    \@makeother{37}% percent
    \@makeother{92}% backslash
    \@makeother{127}%
    \SetCatcodeRange{65}{90}{12}% A-Z
    \SetCatcodeRange{97}{122}{12}% a-z
    \luatexsavecatcodetable\CatcodeTableString
    \@makeother{32}% space
    \luatexsavecatcodetable\CatcodeTableOther
  \endgroup
}%
%    \end{macrocode}
%
% \subsubsection{Number stack}
%
%    \begin{macro}{\LuT@NumStackEmpty}
%    A special empty stack value because of \cs{@cdr}'s brace removal.
%    \begin{macrocode}
\def\LuT@NumStackEmpty{0}
%    \end{macrocode}
%    \end{macro}
%    \begin{macro}{\LuT@NumStack}
%    \begin{macrocode}
\let\LuT@NumStack\LuT@NumStackEmpty
%    \end{macrocode}
%    \end{macro}
%    \begin{macro}{\PushCatcodeTableNumStack}
%    \begin{macrocode}
\LuT@newcommand\PushCatcodeTableNumStack[0]{%
  \xdef\LuT@NumStack{%
    {\the\luatexcatcodetable}\LuT@NumStack
  }%
}
%    \end{macrocode}
%    \end{macro}
%    \begin{macro}{\PopCatcodeTableNumStack}
%    \begin{macrocode}
\LuT@newcommand\PopCatcodeTableNumStack[0]{%
  \ifx\LuT@NumStack\LuT@NumStackEmpty
    \@PackageWarning{luatex}{Empty catcode table number stack}%
    \luatexcatcodetable\z@
  \else
    \luatexcatcodetable=\expandafter\@car\LuT@NumStack\@nil\relax
    \xdef\LuT@NumStack{%
      \expandafter\@cdr\LuT@NumStack\@nil
    }%
  \fi
}
%    \end{macrocode}
%    \end{macro}
%
% \subsubsection{Catcode regime macros}
%
%    \begin{macro}{\BeginCatcodeRegime}
%    \begin{macrocode}
\LuT@newcommand\BeginCatcodeRegime[1]{%
  \PushCatcodeTableNumStack
  \luatexcatcodetable=\numexpr#1\relax
  \IncCatcodeTableStack
  \luatexsavecatcodetable\CatcodeTableStack
  \luatexcatcodetable\CatcodeTableStack
}
%    \end{macrocode}
%    \end{macro}
%
%    \begin{macro}{\EndCatcodeRegime}
%    \begin{macrocode}
\LuT@newcommand\EndCatcodeRegime[0]{%
  \DecCatcodeTableStack
  \PopCatcodeTableNumStack
}
%    \end{macrocode}
%    \end{macro}
%
% \subsection{Lua module loader}
%
%    \begin{macrocode}
\begingroup\expandafter\expandafter\expandafter\endgroup
\expandafter\ifx\csname RequirePackage\endcsname\relax
  \input luatex-loader.sty\relax
\else
  \RequirePackage{luatex-loader}[2010/03/09]%
\fi
%    \end{macrocode}
%    \begin{macrocode}
\LuT@AtEnd%
%</package>
%    \end{macrocode}
%
%    \begin{macrocode}
%<*loader>
%    \end{macrocode}
%
%    Reload check, especially if the package is not used with \LaTeX.
%    \begin{macrocode}
\begingroup\catcode61\catcode48\catcode32=10\relax%
  \catcode13=5 % ^^M
  \endlinechar=13 %
  \catcode35=6 % #
  \catcode39=12 % '
  \catcode44=12 % ,
  \catcode45=12 % -
  \catcode46=12 % .
  \catcode58=12 % :
  \catcode64=11 % @
  \catcode123=1 % {
  \catcode125=2 % }
  \expandafter\let\expandafter\x\csname ver@luatex-loader.sty\endcsname
  \ifx\x\relax % plain-TeX, first loading
  \else
    \def\empty{}%
    \ifx\x\empty % LaTeX, first loading,
      % variable is initialized, but \ProvidesPackage not yet seen
    \else
      \expandafter\ifx\csname PackageInfo\endcsname\relax
        \def\x#1#2{%
          \immediate\write-1{Package #1 Info: #2.}%
        }%
      \else
        \def\x#1#2{\PackageInfo{#1}{#2, stopped}}%
      \fi
      \x{luatex-loader}{The package is already loaded}%
      \aftergroup\endinput
    \fi
  \fi
\endgroup%
%    \end{macrocode}
%    Package identification:
%    \begin{macrocode}
\begingroup\catcode61\catcode48\catcode32=10\relax%
  \catcode13=5 % ^^M
  \endlinechar=13 %
  \catcode35=6 % #
  \catcode39=12 % '
  \catcode40=12 % (
  \catcode41=12 % )
  \catcode44=12 % ,
  \catcode45=12 % -
  \catcode46=12 % .
  \catcode47=12 % /
  \catcode58=12 % :
  \catcode64=11 % @
  \catcode91=12 % [
  \catcode93=12 % ]
  \catcode123=1 % {
  \catcode125=2 % }
  \expandafter\ifx\csname ProvidesPackage\endcsname\relax
    \def\x#1#2#3[#4]{\endgroup
      \immediate\write-1{Package: #3 #4}%
      \xdef#1{#4}%
    }%
  \else
    \def\x#1#2[#3]{\endgroup
      #2[{#3}]%
      \ifx#1\@undefined
        \xdef#1{#3}%
      \fi
      \ifx#1\relax
        \xdef#1{#3}%
      \fi
    }%
  \fi
\expandafter\x\csname ver@luatex-loader.sty\endcsname
\ProvidesPackage{luatex-loader}%
  [2010/03/09 v0.4 Lua module loader (HO)]%
%    \end{macrocode}
%    \begin{macrocode}
\begingroup\catcode61\catcode48\catcode32=10\relax%
  \catcode13=5\endlinechar=13\relax%
  \catcode10=12 % ^^J
  \catcode34=12 % "
  \catcode39=12 % '
  \catcode40=12 % (
  \catcode41=12 % )
  \catcode44=12 % ,
  \catcode46=12 % .
  \catcode60=12 % <
  \catcode61=12 % =
  \catcode95=12 % _ (other!)
  \catcode96=12 % `
  \catcode123=1 % {
  \catcode125=2 % }
  \endlinechar=10 %
  \ifnum\luatexversion<36 %
    \directlua0%
  \else %
    \expandafter\directlua %
  \fi %
  {%
     do
      local script = "oberdiek.luatex.lua"
      local file = kpse.find_file(script, "texmfscripts")
      if file then
        texio.write_nl("(" .. file .. ")")
        dofile(file)
      else
        error("File `" .. script .. "' not found")
      end
    end
  }%
\endgroup\endinput%
%    \end{macrocode}
%
%    \begin{macrocode}
%</loader>
%    \end{macrocode}
%
% \subsection{Lua script}
%
%    Currently \LuaTeX\ does not use KPSE when searching for module
%    files. The following Lua script implements a workaround.
%    It extends |package.loader| by another search method.
%    Modules are found by the module name with extension \xext{lua}
%    similar to
%    \begin{quote}
%      \ttfamily |kpsewhich --format=texmfscripts| \meta{module}|.lua|
%    \end{quote}
%    Unhappily \textsf{kpsewhich} does not support directory components
%    in the file name. Therefore a module |a.b.c| cannot be installed
%    as \xfile{a/b/c.lua}. The script must be named \xfile{a.b.c.lua}.
%    \begin{macrocode}
%<*lua>
%    \end{macrocode}
%    \begin{macrocode}
module("oberdiek.luatex", package.seeall)
%    \end{macrocode}
%    \begin{macrocode}
function kpse_module_loader(module)
  local script = module .. ".lua"
  local file = kpse.find_file(script, "texmfscripts")
  if file then
    local loader, error = loadfile(file)
    if loader then
      texio.write_nl("(" .. file .. ")")
      return loader
    end
    return "\n\t[oberdiek.luatex.kpse_module_loader] Loading error:\n\t"
           .. error
  end
  return "\n\t[oberdiek.luatex.kpse_module_loader] Search failed"
end
table.insert(package.loaders, kpse_module_loader)
%    \end{macrocode}
%    \begin{macrocode}
%</lua>
%    \end{macrocode}
%
% \section{Test}
%
%    \begin{macrocode}
%<*test2>
\documentclass{article}
\def\LoadCommand{%
  \RequirePackage{luatex}[2010/03/09]%
}
%</test2>
%    \end{macrocode}
%    \begin{macrocode}
%<*test3>
\documentclass{article}
\def\LoadCommand{%
  \RequirePackage{luatex-loader}[2010/03/09]%
}
%</test3>
%    \end{macrocode}
% \subsection{Catcode checks for loading}
%
%    \begin{macrocode}
%<*test1>
%    \end{macrocode}
%    \begin{macrocode}
\catcode`\{=1 %
\catcode`\}=2 %
\catcode`\#=6 %
\catcode`\@=11 %
\expandafter\ifx\csname count@\endcsname\relax
  \countdef\count@=255 %
\fi
\expandafter\ifx\csname @gobble\endcsname\relax
  \long\def\@gobble#1{}%
\fi
\expandafter\ifx\csname @firstofone\endcsname\relax
  \long\def\@firstofone#1{#1}%
\fi
\expandafter\ifx\csname loop\endcsname\relax
  \expandafter\@firstofone
\else
  \expandafter\@gobble
\fi
{%
  \def\loop#1\repeat{%
    \def\body{#1}%
    \iterate
  }%
  \def\iterate{%
    \body
      \let\next\iterate
    \else
      \let\next\relax
    \fi
    \next
  }%
  \let\repeat=\fi
}%
\def\RestoreCatcodes{}
\count@=0 %
\loop
  \edef\RestoreCatcodes{%
    \RestoreCatcodes
    \catcode\the\count@=\the\catcode\count@\relax
  }%
\ifnum\count@<255 %
  \advance\count@ 1 %
\repeat

\def\RangeCatcodeInvalid#1#2{%
  \count@=#1\relax
  \loop
    \catcode\count@=15 %
  \ifnum\count@<#2\relax
    \advance\count@ 1 %
  \repeat
}
\def\RangeCatcodeCheck#1#2#3{%
  \count@=#1\relax
  \loop
    \ifnum#3=\catcode\count@
    \else
      \errmessage{%
        Character \the\count@\space
        with wrong catcode \the\catcode\count@\space
        instead of \number#3%
      }%
    \fi
  \ifnum\count@<#2\relax
    \advance\count@ 1 %
  \repeat
}
\def\space{ }
\expandafter\ifx\csname LoadCommand\endcsname\relax
  \def\LoadCommand{\input luatex.sty\relax}%
\fi
\def\Test{%
  \RangeCatcodeInvalid{0}{47}%
  \RangeCatcodeInvalid{58}{64}%
  \RangeCatcodeInvalid{91}{96}%
  \RangeCatcodeInvalid{123}{255}%
  \catcode`\@=12 %
  \catcode`\\=0 %
  \catcode`\%=14 %
  \LoadCommand
  \RangeCatcodeCheck{0}{36}{15}%
  \RangeCatcodeCheck{37}{37}{14}%
  \RangeCatcodeCheck{38}{47}{15}%
  \RangeCatcodeCheck{48}{57}{12}%
  \RangeCatcodeCheck{58}{63}{15}%
  \RangeCatcodeCheck{64}{64}{12}%
  \RangeCatcodeCheck{65}{90}{11}%
  \RangeCatcodeCheck{91}{91}{15}%
  \RangeCatcodeCheck{92}{92}{0}%
  \RangeCatcodeCheck{93}{96}{15}%
  \RangeCatcodeCheck{97}{122}{11}%
  \RangeCatcodeCheck{123}{255}{15}%
  \RestoreCatcodes
}
\Test
\csname @@end\endcsname
\end
%    \end{macrocode}
%    \begin{macrocode}
%</test1>
%    \end{macrocode}
%
% \subsection{Catcode tables}
%
% \subsubsection{Predefined catcode tables}
%
%    \begin{macrocode}
%<*test4>
\NeedsTeXFormat{LaTeX2e}
%    \end{macrocode}
%    Remember \LaTeX's initial catcodes in count registers
%    starting at \cs{TestLaTeX}.
%    \begin{macrocode}
\count0=0 %
\chardef\TestLaTeX=1000 %
\chardef\TestMax=300 %
\loop
  \count\numexpr\TestLaTeX+\count0\relax=\catcode\count0 %
\ifnum\count0<\TestMax
  \advance\count0 by 1 %
\repeat
\documentclass{minimal}
\usepackage{luatex}[2010/03/09]
\usepackage{qstest}
\IncludeTests{*}
\LogTests{log}{*}{*}
\makeatletter
\def\Check#1{%
  \Expect*{\the\count@=\the\catcode\count@}%
         *{\the\count@=#1}%
}
\newcount\scratch
\def\Test#1#2{%
  \begin{qstest}{CatcodeTable#1}{CatcodeTable#1}%
    \luatexcatcodetable\csname CatcodeTable#1\endcsname
    \count@=\z@
    \loop
      \scratch=#2\relax
      \Expect*{\the\count@=\the\catcode\count@}%
             *{\the\count@=\the\scratch}%
    \ifnum\count@<\TestMax
      \advance\count@\@ne
    \repeat
  \end{qstest}%
}
\begingroup
  % luatex-unicode-letters.tex makes some slots to letters
  \def\TestMax{169}%
  \Test{LaTeX}{\the\count\numexpr\TestLaTeX+\count@}%
\endgroup
\Test{String}{\ifnum\count@=32 10\else 12\fi}
\Test{Other}{12}
\luatexinitcatcodetable99 %
\Test{IniTeX}{%
  0\relax
  \begingroup
    \luatexcatcodetable99 %
    \global\scratch=\the\catcode\count@
  \endgroup
}
%    \end{macrocode}
%
% \subsubsection{Catcode table number stack}
%
%    \begin{macrocode}
\begin{qstest}{CatcodeTableNumStack}{CatcodeTableNumStack}
  \def\TestStack#1{%
    \Expect*{\LuT@NumStack}{#1}%
  }%
  \TestStack{0}%
  \PushCatcodeTableNumStack
  \TestStack{{0}0}%
  \@firstofone{%
    \begingroup
      \luatexinitcatcodetable12 %
      \luatexcatcodetable12 %
      \PushCatcodeTableNumStack
      \TestStack{{12}{0}0}%
      \PopCatcodeTableNumStack
      \TestStack{{0}0}%
      \PopCatcodeTableNumStack
      \TestStack{0}%
      \def\TestWarning{Missing empty stack warning}%
      \def\@PackageWarning#1#2{\def\TestWarning{empty stack}}%
      \PopCatcodeTableNumStack
      \TestStack{0}%
      \Expect*{\TestWarning}{empty stack}%
    \endgroup
  }%
\end{qstest}
%    \end{macrocode}
%
% \subsubsection{Catcode table stack}
%
%    \begin{macrocode}
\begin{qstest}{CatcodeTableStack}{CatcodeTableStack}
  \def\TestStack#1{%
    \Expect*{\the\CatcodeTableStack}{#1}%
  }%
  \TestStack{0}%
  \IncCatcodeTableStack
  \TestStack{2}%
  \IncCatcodeTableStack
  \TestStack{4}%
  \begingroup
    \IncCatcodeTableStack
    \TestStack{6}%
  \endgroup
  \TestStack{6}%
  \begingroup
    \DecCatcodeTableStack
    \TestStack{4}%
  \endgroup
  \TestStack{4}%
  \DecCatcodeTableStack
  \TestStack{2}%
  \DecCatcodeTableStack
  \TestStack{0}%
  \begingroup
    \def\TestError{Missing error}%
    \def\@PackageError#1#2#3{%
      \def\TestError{Empty stack}%
    }%
    \DecCatcodeTableStack
    \TestStack{0}%
    \Expect*{\TestError}{Empty stack}%
  \endgroup
\end{qstest}
%    \end{macrocode}
%
% \subsubsection{Catcode regime macros}
%
%    \begin{macrocode}
\begin{qstest}{CatcodeRegime}{CatcodeRegime}
  \def\TestStacks#1#2#3{%
    \Expect*{\the\luatexcatcodetable}{#1}%
    \Expect*{\the\CatcodeTableStack}{#2}%
    \Expect*{\LuT@NumStack}{#3}%
  }%
  \TestStacks{0}{0}{0}%
  \catcode`\|=7 %
  \BeginCatcodeRegime\CatcodeTableLaTeX
    \TestStacks{2}{2}{{0}0}%
    \Expect*{\the\catcode`\|}{12}%
  \EndCatcodeRegime
  \TestStacks{0}{0}{0}%
  \Expect*{\the\catcode`\|}{7}%
\end{qstest}
%    \end{macrocode}
%
% \subsection{Attribute allocation}
%
%    \begin{macrocode}
\begin{qstest}{Attributes}{Attributes}
  \newattribute\TestAttr
  \Expect*{\meaning\TestAttr}%
         *{\string\attribute\number\allocationnumber}%
  \Expect*{\the\allocationnumber}{0}%
  \begingroup
    \newattribute\TestAttr
    \Expect*{\the\allocationnumber}{1}%
  \endgroup
  \Expect*{\the\allocationnumber}{0}%
  \Expect*{\meaning\TestAttr}*{\string\attribute1}%
  \Expect*{\the\TestAttr}*{\number\LuT@UnsetAttributeValue}%
  \def\Test#1{%
    \setattribute\TestAttr{#1}%
    \Expect*{\the\TestAttr}{#1}%
  }%
  \Test{0}%
  \Test{1}%
  \Test{-1}%
  \Test{123}%
  \unsetattribute\TestAttr
  \Expect*{\the\TestAttr}*{\number\LuT@UnsetAttributeValue}%
  \begingroup
    \Expect*{\the\TestAttr}*{\number\LuT@UnsetAttributeValue}%
    \Test{1234}%
  \endgroup
  \Expect*{\the\TestAttr}*{\number\LuT@UnsetAttributeValue}%
\end{qstest}
%    \end{macrocode}
%
%    \begin{macrocode}
\@@end
%</test4>
%    \end{macrocode}
%
% \subsection{Short test for \plainTeX}
%
%    \begin{macrocode}
%<*test5>
\input luatex.sty\relax
\newattribute\TestAttr
\setattribute\TestAttr{10}
\unsetattribute\TestAttr
\newcatcodetable\TestCTa
\begingroup
  \SetCatcodeRange{`A}{`Z}{12}%
\endgroup
\BeginCatcodeRegime\CatcodeTableLaTeX
\EndCatcodeRegime
\end
%</test5>
%    \end{macrocode}
%
% \section{Installation}
%
% \subsection{Download}
%
% \paragraph{Package.} This package is available on
% CTAN\footnote{\url{ftp://ftp.ctan.org/tex-archive/}}:
% \begin{description}
% \item[\CTAN{macros/latex/contrib/oberdiek/luatex.dtx}] The source file.
% \item[\CTAN{macros/latex/contrib/oberdiek/luatex.pdf}] Documentation.
% \end{description}
%
%
% \paragraph{Bundle.} All the packages of the bundle `oberdiek'
% are also available in a TDS compliant ZIP archive. There
% the packages are already unpacked and the documentation files
% are generated. The files and directories obey the TDS standard.
% \begin{description}
% \item[\CTAN{install/macros/latex/contrib/oberdiek.tds.zip}]
% \end{description}
% \emph{TDS} refers to the standard ``A Directory Structure
% for \TeX\ Files'' (\CTAN{tds/tds.pdf}). Directories
% with \xfile{texmf} in their name are usually organized this way.
%
% \subsection{Bundle installation}
%
% \paragraph{Unpacking.} Unpack the \xfile{oberdiek.tds.zip} in the
% TDS tree (also known as \xfile{texmf} tree) of your choice.
% Example (linux):
% \begin{quote}
%   |unzip oberdiek.tds.zip -d ~/texmf|
% \end{quote}
%
% \paragraph{Script installation.}
% Check the directory \xfile{TDS:scripts/oberdiek/} for
% scripts that need further installation steps.
% Package \xpackage{attachfile2} comes with the Perl script
% \xfile{pdfatfi.pl} that should be installed in such a way
% that it can be called as \texttt{pdfatfi}.
% Example (linux):
% \begin{quote}
%   |chmod +x scripts/oberdiek/pdfatfi.pl|\\
%   |cp scripts/oberdiek/pdfatfi.pl /usr/local/bin/|
% \end{quote}
%
% \subsection{Package installation}
%
% \paragraph{Unpacking.} The \xfile{.dtx} file is a self-extracting
% \docstrip\ archive. The files are extracted by running the
% \xfile{.dtx} through \plainTeX:
% \begin{quote}
%   \verb|tex luatex.dtx|
% \end{quote}
%
% \paragraph{TDS.} Now the different files must be moved into
% the different directories in your installation TDS tree
% (also known as \xfile{texmf} tree):
% \begin{quote}
% \def\t{^^A
% \begin{tabular}{@{}>{\ttfamily}l@{ $\rightarrow$ }>{\ttfamily}l@{}}
%   luatex.sty & tex/generic/oberdiek/luatex.sty\\
%   luatex-loader.sty & tex/generic/oberdiek/luatex-loader.sty\\
%   oberdiek.luatex.lua & scripts/oberdiek/oberdiek.luatex.lua\\
%   luatex.pdf & doc/latex/oberdiek/luatex.pdf\\
%   test/luatex-test1.tex & doc/latex/oberdiek/test/luatex-test1.tex\\
%   test/luatex-test2.tex & doc/latex/oberdiek/test/luatex-test2.tex\\
%   test/luatex-test3.tex & doc/latex/oberdiek/test/luatex-test3.tex\\
%   test/luatex-test4.tex & doc/latex/oberdiek/test/luatex-test4.tex\\
%   test/luatex-test5.tex & doc/latex/oberdiek/test/luatex-test5.tex\\
%   luatex.dtx & source/latex/oberdiek/luatex.dtx\\
% \end{tabular}^^A
% }^^A
% \sbox0{\t}^^A
% \ifdim\wd0>\linewidth
%   \begingroup
%     \advance\linewidth by\leftmargin
%     \advance\linewidth by\rightmargin
%   \edef\x{\endgroup
%     \def\noexpand\lw{\the\linewidth}^^A
%   }\x
%   \def\lwbox{^^A
%     \leavevmode
%     \hbox to \linewidth{^^A
%       \kern-\leftmargin\relax
%       \hss
%       \usebox0
%       \hss
%       \kern-\rightmargin\relax
%     }^^A
%   }^^A
%   \ifdim\wd0>\lw
%     \sbox0{\small\t}^^A
%     \ifdim\wd0>\linewidth
%       \ifdim\wd0>\lw
%         \sbox0{\footnotesize\t}^^A
%         \ifdim\wd0>\linewidth
%           \ifdim\wd0>\lw
%             \sbox0{\scriptsize\t}^^A
%             \ifdim\wd0>\linewidth
%               \ifdim\wd0>\lw
%                 \sbox0{\tiny\t}^^A
%                 \ifdim\wd0>\linewidth
%                   \lwbox
%                 \else
%                   \usebox0
%                 \fi
%               \else
%                 \lwbox
%               \fi
%             \else
%               \usebox0
%             \fi
%           \else
%             \lwbox
%           \fi
%         \else
%           \usebox0
%         \fi
%       \else
%         \lwbox
%       \fi
%     \else
%       \usebox0
%     \fi
%   \else
%     \lwbox
%   \fi
% \else
%   \usebox0
% \fi
% \end{quote}
% If you have a \xfile{docstrip.cfg} that configures and enables \docstrip's
% TDS installing feature, then some files can already be in the right
% place, see the documentation of \docstrip.
%
% \subsection{Refresh file name databases}
%
% If your \TeX~distribution
% (\teTeX, \mikTeX, \dots) relies on file name databases, you must refresh
% these. For example, \teTeX\ users run \verb|texhash| or
% \verb|mktexlsr|.
%
% \subsection{Some details for the interested}
%
% \paragraph{Attached source.}
%
% The PDF documentation on CTAN also includes the
% \xfile{.dtx} source file. It can be extracted by
% AcrobatReader 6 or higher. Another option is \textsf{pdftk},
% e.g. unpack the file into the current directory:
% \begin{quote}
%   \verb|pdftk luatex.pdf unpack_files output .|
% \end{quote}
%
% \paragraph{Unpacking with \LaTeX.}
% The \xfile{.dtx} chooses its action depending on the format:
% \begin{description}
% \item[\plainTeX:] Run \docstrip\ and extract the files.
% \item[\LaTeX:] Generate the documentation.
% \end{description}
% If you insist on using \LaTeX\ for \docstrip\ (really,
% \docstrip\ does not need \LaTeX), then inform the autodetect routine
% about your intention:
% \begin{quote}
%   \verb|latex \let\install=y\input{luatex.dtx}|
% \end{quote}
% Do not forget to quote the argument according to the demands
% of your shell.
%
% \paragraph{Generating the documentation.}
% You can use both the \xfile{.dtx} or the \xfile{.drv} to generate
% the documentation. The process can be configured by the
% configuration file \xfile{ltxdoc.cfg}. For instance, put this
% line into this file, if you want to have A4 as paper format:
% \begin{quote}
%   \verb|\PassOptionsToClass{a4paper}{article}|
% \end{quote}
% An example follows how to generate the
% documentation with pdf\LaTeX:
% \begin{quote}
%\begin{verbatim}
%pdflatex luatex.dtx
%makeindex -s gind.ist luatex.idx
%pdflatex luatex.dtx
%makeindex -s gind.ist luatex.idx
%pdflatex luatex.dtx
%\end{verbatim}
% \end{quote}
%
% \section{Catalogue}
%
% The following XML file can be used as source for the
% \href{http://mirror.ctan.org/help/Catalogue/catalogue.html}{\TeX\ Catalogue}.
% The elements \texttt{caption} and \texttt{description} are imported
% from the original XML file from the Catalogue.
% The name of the XML file in the Catalogue is \xfile{luatex.xml}.
%    \begin{macrocode}
%<*catalogue>
<?xml version='1.0' encoding='us-ascii'?>
<!DOCTYPE entry SYSTEM 'catalogue.dtd'>
<entry datestamp='$Date$' modifier='$Author$' id='luatex'>
  <name>luatex</name>
  <caption>The LuaTeX engine.</caption>
  <authorref id='auth:oberdiek'/>
  <copyright owner='Heiko Oberdiek' year='2007,2009,2010'/>
  <license type='lppl1.3'/>
  <version number='0.4'/>
  <description>
    LuaTeX is an extended version of pdfTeX using Lua as an embedded
    scripting language.  The LuaTeX project&#x2019;s main objective
    is to provide an open and configurable variant of TeX while at the
    same time offering downward compatibility.
    <p/>
    LuaTeX uses Unicode (as UTF-8) as its default input encoding, and
    is able to use modern (OpenType) fonts (for both text and mathematics).
    <p/>
    It should be noted that LuaTeX is still under development; its
    specification has been declared stable, but absolute stability
    may not in practice be assumed.
    <p/>
    The package is part of the <xref refid='oberdiek'>oberdiek</xref> bundle.
  </description>
  <documentation details='Package documentation'
      href='ctan:/macros/latex/contrib/oberdiek/luatex.pdf'/>
  <ctan file='true' path='/macros/latex/contrib/oberdiek/luatex.dtx'/>
  <miktex location='oberdiek'/>
  <texlive location='oberdiek'/>
  <install path='/macros/latex/contrib/oberdiek/oberdiek.tds.zip'/>
</entry>
%</catalogue>
%    \end{macrocode}
%
% \begin{History}
%   \begin{Version}{2007/12/12 v0.1}
%   \item
%     First public version.
%   \end{Version}
%   \begin{Version}{2009/04/10 v0.2}
%   \item
%     Requires package \xpackage{ifluatex} in version 2.0 to ensure
%     \cs{luatexversion}.
%   \item
%     Updates the call of \cs{directlua}, the syntax has changed
%     in \LuaTeX\ 0.36.
%   \end{Version}
%   \begin{Version}{2009/12/02 v0.3}
%   \item
%     Unsetting of attributes updated for \LuaTeX\ 0.37.
%   \end{Version}
%   \begin{Version}{2010/03/09 v0.4}
%   \item
%     Support for lua states removed.
%   \item
%     Calling tex.enableprimitives for used primitives.
%   \end{Version}
% \end{History}
%
% \PrintIndex
%
% \Finale
\endinput

%        (quote the arguments according to the demands of your shell)
%
% Documentation:
%    (a) If luatex.drv is present:
%           latex luatex.drv
%    (b) Without luatex.drv:
%           latex luatex.dtx; ...
%    The class ltxdoc loads the configuration file ltxdoc.cfg
%    if available. Here you can specify further options, e.g.
%    use A4 as paper format:
%       \PassOptionsToClass{a4paper}{article}
%
%    Programm calls to get the documentation (example):
%       pdflatex luatex.dtx
%       makeindex -s gind.ist luatex.idx
%       pdflatex luatex.dtx
%       makeindex -s gind.ist luatex.idx
%       pdflatex luatex.dtx
%
% Installation:
%    TDS:tex/generic/oberdiek/luatex.sty
%    TDS:tex/generic/oberdiek/luatex-loader.sty
%    TDS:scripts/oberdiek/oberdiek.luatex.lua
%    TDS:doc/latex/oberdiek/luatex.pdf
%    TDS:doc/latex/oberdiek/test/luatex-test1.tex
%    TDS:doc/latex/oberdiek/test/luatex-test2.tex
%    TDS:doc/latex/oberdiek/test/luatex-test3.tex
%    TDS:doc/latex/oberdiek/test/luatex-test4.tex
%    TDS:doc/latex/oberdiek/test/luatex-test5.tex
%    TDS:source/latex/oberdiek/luatex.dtx
%
%<*ignore>
\begingroup
  \catcode123=1 %
  \catcode125=2 %
  \def\x{LaTeX2e}%
\expandafter\endgroup
\ifcase 0\ifx\install y1\fi\expandafter
         \ifx\csname processbatchFile\endcsname\relax\else1\fi
         \ifx\fmtname\x\else 1\fi\relax
\else\csname fi\endcsname
%</ignore>
%<*install>
\input docstrip.tex
\Msg{************************************************************************}
\Msg{* Installation}
\Msg{* Package: luatex 2010/03/09 v0.4 LuaTeX basic definition package (HO)}
\Msg{************************************************************************}

\keepsilent
\askforoverwritefalse

\let\MetaPrefix\relax
\preamble

This is a generated file.

Project: luatex
Version: 2010/03/09 v0.4

Copyright (C) 2007, 2009, 2010 by
   Heiko Oberdiek <heiko.oberdiek at googlemail.com>

This work may be distributed and/or modified under the
conditions of the LaTeX Project Public License, either
version 1.3c of this license or (at your option) any later
version. This version of this license is in
   http://www.latex-project.org/lppl/lppl-1-3c.txt
and the latest version of this license is in
   http://www.latex-project.org/lppl.txt
and version 1.3 or later is part of all distributions of
LaTeX version 2005/12/01 or later.

This work has the LPPL maintenance status "maintained".

This Current Maintainer of this work is Heiko Oberdiek.

The Base Interpreter refers to any `TeX-Format',
because some files are installed in TDS:tex/generic//.

This work consists of the main source file luatex.dtx
and the derived files
   luatex.sty, luatex.pdf, luatex.ins, luatex.drv, luatex-loader.sty,
   luatex-test1.tex, luatex-test2.tex, luatex-test3.tex,
   luatex-test4.tex, luatex-test5.tex, oberdiek.luatex.lua.

\endpreamble
\let\MetaPrefix\DoubleperCent

\generate{%
  \file{luatex.ins}{\from{luatex.dtx}{install}}%
  \file{luatex.drv}{\from{luatex.dtx}{driver}}%
  \usedir{tex/generic/oberdiek}%
  \file{luatex.sty}{\from{luatex.dtx}{package}}%
  \file{luatex-loader.sty}{\from{luatex.dtx}{loader}}%
  \usedir{doc/latex/oberdiek/test}%
  \file{luatex-test1.tex}{\from{luatex.dtx}{test1}}%
  \file{luatex-test2.tex}{\from{luatex.dtx}{test1,test2}}%
  \file{luatex-test3.tex}{\from{luatex.dtx}{test1,test3}}%
  \file{luatex-test4.tex}{\from{luatex.dtx}{test4}}%
  \file{luatex-test5.tex}{\from{luatex.dtx}{test5}}%
  \nopreamble
  \nopostamble
  \usedir{source/latex/oberdiek/catalogue}%
  \file{luatex.xml}{\from{luatex.dtx}{catalogue}}%
}
\def\MetaPrefix{-- }
\def\defaultpostamble{%
  \MetaPrefix^^J%
  \MetaPrefix\space End of File `\outFileName'.%
}
\def\currentpostamble{\defaultpostamble}%
\generate{%
  \usedir{scripts/oberdiek}%
  \file{oberdiek.luatex.lua}{\from{luatex.dtx}{lua}}%
}

\catcode32=13\relax% active space
\let =\space%
\Msg{************************************************************************}
\Msg{*}
\Msg{* To finish the installation you have to move the following}
\Msg{* files into a directory searched by TeX:}
\Msg{*}
\Msg{*     luatex.sty, luatex-loader.sty}
\Msg{*}
\Msg{* And install the following script file:}
\Msg{*}
\Msg{*     oberdiek.luatex.lua}
\Msg{*}
\Msg{* To produce the documentation run the file `luatex.drv'}
\Msg{* through LaTeX.}
\Msg{*}
\Msg{* Happy TeXing!}
\Msg{*}
\Msg{************************************************************************}

\endbatchfile
%</install>
%<*ignore>
\fi
%</ignore>
%<*driver>
\NeedsTeXFormat{LaTeX2e}
\ProvidesFile{luatex.drv}%
  [2010/03/09 v0.4 LuaTeX basic definition package (HO)]%
\documentclass{ltxdoc}
\usepackage{holtxdoc}[2011/11/22]
\usepackage{paralist}
\begin{document}
  \DocInput{luatex.dtx}%
\end{document}
%</driver>
% \fi
%
% \CheckSum{1429}
%
% \CharacterTable
%  {Upper-case    \A\B\C\D\E\F\G\H\I\J\K\L\M\N\O\P\Q\R\S\T\U\V\W\X\Y\Z
%   Lower-case    \a\b\c\d\e\f\g\h\i\j\k\l\m\n\o\p\q\r\s\t\u\v\w\x\y\z
%   Digits        \0\1\2\3\4\5\6\7\8\9
%   Exclamation   \!     Double quote  \"     Hash (number) \#
%   Dollar        \$     Percent       \%     Ampersand     \&
%   Acute accent  \'     Left paren    \(     Right paren   \)
%   Asterisk      \*     Plus          \+     Comma         \,
%   Minus         \-     Point         \.     Solidus       \/
%   Colon         \:     Semicolon     \;     Less than     \<
%   Equals        \=     Greater than  \>     Question mark \?
%   Commercial at \@     Left bracket  \[     Backslash     \\
%   Right bracket \]     Circumflex    \^     Underscore    \_
%   Grave accent  \`     Left brace    \{     Vertical bar  \|
%   Right brace   \}     Tilde         \~}
%
% \GetFileInfo{luatex.drv}
%
% \title{The \xpackage{luatex} package}
% \date{2010/03/09 v0.4}
% \author{Heiko Oberdiek\\\xemail{heiko.oberdiek at googlemail.com}}
%
% \maketitle
%
% \begin{abstract}
% This package manages the new and extended features and resources
% that \LuaTeX\ provides. Examples are attributes and catcode
% tables.
% \end{abstract}
%
% \tableofcontents
%
% \section{Documentation}
%
% \subsection{Introduction}
%
% \TeX\ provides global resources such as registers.
% But it does not provide an interface for managing these resources.
% For example, two packages want to use a counter register.
% If they take the same register
% number, then the use of both packages will conflict and they cannot
% be used together.
% Therefore formats such as \plainTeX\ or \LaTeX\ implement
% an allocation scheme for registers. A package reserves with
% \cs{newcount} an unused register number for its own exclusive use.
%
% Nowadays \TeX\ is not alone anymore: \eTeX, \pdfTeX\ and other
% compilers for \TeX\ are developed that extend and add new
% features and resources.
%
% Now \LuaTeX\ has reached beta state. It inherits most of
% \pdfTeX's features including \eTeX. Also it implements
% new concepts such as attributes or catcode tables.
%
% \subsubsection{\LaTeX}
%
% \LaTeXe\ is frozen and therefore refuses to even notice
% the new \TeX\ variants. Not even the old \eTeX\ is supported
% by its kernel. At least there is a third party package \xpackage{etex}
% that manages the new \eTeX\ resources.
%
% This package tries to do the same for \LuaTeX\ and starts to
% support at least a few of the new features.
%
% \subsubsection{\plainTeX}
%
% \LaTeX\ has inherited its resource handling from \plainTeX.
% The interface is basically the same: \cs{newcount}, \dots
% Therefore this package tries to follow this tradition by
% providing compatibility to \plainTeX. It can be loaded
% with \plainTeX\ and defines at least some of the
% features that this packages provides for \LaTeX.
%
% \subsection{Register allocation}
%
% \subsubsection{Register with 16 bit}
%
% Because \LuaTeX\ is a super set of \eTeX\ regarding registers,
% the register allocation scheme should not conflict with
% package \xpackage{etex}. Therefore this package is loaded
% to inherit its allocation scheme. The only change is currently
% that the limit is increased to 65536 registers for the following
% register classes:
% \begin{compactitem}
% \item count
% \item dimen
% \item skip
% \item muskip
% \item marks
% \item toks
% \item box
% \end{compactitem}
% This affects the number of global and local registers.
% Because it is done in a package and not in the kernel,
% it is possible that someone loads package \xpackage{etex}
% before uses the local allocation variants. This will prevent
% the extension for this register class. If more registers are
% needed, just load package \xpackage{luatex} earlier.
%
% \subsubsection{Insertions}
%
% Insertions need four registers \cs{count}, \cs{dimen}, \cs{skip},
% and \cs{box} with the same number. Usually they are allocated
% downwards from 254, 253, \dots Also \cs{newcount}, \cs{newdimen}, \dots
% fill up these register numbers from below before switching to
% higher register numbers by package \xpackage{etex}.
% When this occurs, no insertions can be allocated anymore.
%
% Therefore \cs{newcount}, \cs{newdimen}, \cs{newskip}, and
% \cs{newbox} are replaced by their global variants (\cs{globcount}, \dots)
% that use the higher numbers immediately, leaving the room for insertions.
% There should not be an efficiency penalty because \LuaTeX\ stores
% the registers of a class in the same Lua table unlike \eTeX, where
% registers below 256 are stored in an array and higher numbers
% are put in a tree structure.
%
% \subsection{Attributes}
%
% Nodes can have custom attributes in \LuaTeX. These attributes
% are organized by a new register class. As the other registers
% up to $2^{16}$ attributes are supported. An attribute value
% can be negative that means the attribute is not set. Otherwise
% \TeX's range of non-negative integers up to $2^31$ are available.
%
% \begin{declcs}{newattribute} \M{cmd}
% \end{declcs}
% Macro \cs{newattribute} defines command \meta{cmd} using \cs{attributedef}
% using an new attribute number. The new attribute is initially unset.
%
% \begin{declcs}{setattribute} \M{cmd} \M{value}
% \end{declcs}
% Macro \cs{setattribute} locally sets attribute command \meta{cmd}
% to the number \meta{value}. Valid values range from $-1$ until $2^31$
% (the upper limit is the same as for other \TeX\ integer numbers).
%
% \begin{declcs}{unsetattribute} \M{cmd}
% \end{declcs}
% Macro \cs{unsettattribute} clears the attribute command \meta{cmd}.
%
% \subsection{Catcode tables}
%
% \LuaTeX\ introduces catcode tables as new feature, see documentation.
% There is need for discussion, how to deal best:
% \begin{itemize}
% \item \cs{initcatcodetable} and \cs{setcatcodetable} act globally.
% \item \cs{catcodetable} causes an error if used with an uninitialized
%   catcode table.
% \item Large catcode table numbers should be avoided because of
%   performance breakdown.
% \item Use case \LaTeX\ package:
%   The package must not be surprised by changed catcodes and
%   must not surprise by changing catcodes accidently.
%   Catcode tables could offer a solution. At the begin
%   a catcode regime with standard catcodes is established
%   and the old one is restored afterwards.
% \item Use case: \LuaTeX's |tex.print| might be used with
%   a catcode table number, for example a table where all
%   entries have catcode ``other''.
% \item Readonly catcode tables.
% \item Is there is a need for local allocations?
%   (Package \xpackage{etex}'s \cs{loc} variants are not used in
%   \TeX\ Live 2007.)
% \end{itemize}
%
% \subsubsection{Interface proposal}
%
% The idea: \cs{newcatcodetable} allocates odd numbered catcode
% tables. Even numbered tables are managed as stack.
% Also some catcode tables are defined. These must not be
% changed.
%
% \begin{declcs}{newcatcodetable} \M{cmd}
% \end{declcs}
% Macro \cs{newcatcodetable} reserves a new catcode table and
% remembers its number in \meta{cmd}. The catcode table is
% initialized with \iniTeX's catcodes.
%
% \begin{declcs}{CatcodeTableIniTeX}\\
%   \cs{CatcodeTableString}\\
%   \cs{CatcodeTableOther}\\
%   \cs{CatcodeTableLaTeX}\\
% \end{declcs}
% These are catcode tables and must not be changed.
% \cs{CatcodeTableIniTeX} contains the catcode settings of \iniTeX.
% \cs{CatcodeTableString} follows \TeX's convention of
% \cs{string}, \cs{meaning} and friends. The space gets catcode
% 10 (space), the other characters have catcode 12 (other).
% In \cs{CatcodeTableOther} all entries have catcode 12 (other).
% \cs{CatcodeTableLaTeX} contains the setting of a pure \LaTeX\ format
% (`at' is other).
%
% \begin{declcs}{CatcodeTableStack}\\
%   \cs{IncCatcodeTableStack}\\
%   \cs{DecCatcodeTableStack}
% \end{declcs}
% \cs{CatcodeTableStack} is the stack pointer. Initially
% it is catcode table zero. \cs{IncCatcodeTableStack} and
% \cs{DecCatcodeTableStack} increments and decrements the
% stack pointer. Currently \cs{IncCatcodeTableStack} does not
% initialize a new catcode table. Both increment and decrement
% operations do not set a catcode table.
%
% \begin{declcs}{PushCatcodeTableNumStack}\\
%   \cs{PopCatcodeTableNumStack}
% \end{declcs}
% It can be handy to have a global stack for catcode table
% numbers to deal with the global assignment property of
% \cs{initcatcodetable} and \cs{savecatcodetable}.
% \cs{PushCatcodeTableNumStack} pushes the current catcode table
% on the stack.
% \cs{PopCatcodeTableNumStack} pops the topmost number off the number
% stack to set the current catcode table. Catcode table zero is
% used in case of an empty stack.
%
% \begin{declcs}{BeginCatcodeRegime} \M{catcodetable}\\
%   \cs{EndCatcodeRegime}
% \end{declcs}
% \cs{BeginCatcodeRegime} remembers the current catcode table number.
% Then it creates and uses a fresh catcode table on the stack
% that is initialized by \meta{catcodetable}:
% \begin{quote}\ttfamily
%   \cs{PushCatcodeTableNumStack}\\
%   \cs{catcodetable}\meta{catcodetable}||
%   \cs{IncCatcodeTableStack}\\
%   \cs{savecatcodetable}\cs{CatcodeTableStack}\\
%   \cs{catcodetable}\cs{CatcodeTableStack}
% \end{quote}
% \cs{EndCatcodeRegime} drops the catcode table, created by
% \cs{BeginCatcodeRegime} and sets the catcode table that was
% active before:
% \begin{quote}
%   \cs{DecCatcodeTableStack}\\
%   \cs{PopCatcodeTableNumStack}
% \end{quote}
% These macros solve the use case, described earlier for a \LaTeX\ package:
% \begin{quote}
%   |% package foobar.sty|\\
%   |\BeginCatcodeRegime\CatcodeTableLaTeX|\\
%   |\makeatletter|\\
%   |% ... package contents ...|\\
%   |\EndCatcodeRegime|\\
%   |% end of package|
% \end{quote}
% If the package wants to change catcodes after its loading,
% \cs{AtBeginDocument} or \cs{AtEndOfPackage} can be used.
%
% \begin{declcs}{SetCatcodeRange} \M{from} \M{to} \M{catcode}
% \end{declcs}
% The catcodes of characters in range from \meta{from}
% to inclusive \meta{to} are set to \meta{catcode}.
%
% \subsection{Lua module loading}
%
% Currently \LuaTeX\ (version 0.20) does not support Lua script
% files inside \xfile{TDS:scripts//}, because Lua's mechanism
% for module loading does not use the \textsf{kpathsea} library.
% Therefore this packages appends a kpse loader to the list of
% Lua's module loaders. It finds the module \meta{module} by
% \begin{quote}
%   \ttfamily |kpse.find_file("|\meta{module}|.lua", "texmfscripts")|
% \end{quote}
% Unhappily \textsf{kpathsea} does not support directory components
% in a file name. Therefore the Lua convention is not followed to
% replace dots in the module name by the directory separator.
%
% Example:
% A Lua script of a package \xpackage{foobar} wants the following
% modules:
% \begin{quote}
%   |require("foobar.hello.world")|\\
%   |require("org.somewhere.xyz")|
% \end{quote}
% Then they can be find in:
% \begin{quote}
%   |TDS:scripts/foobar/foobar.hello.world.lua|\\
%   |TDS:scripts/foobar/org.somewhere.xyz.lua|
% \end{quote}
% I would have preferred the following locations, following lua conventions,
% e.\,g.:
% \begin{quote}
%   |TDS:scripts/foobar/hello/world.lua|\\
%   |TDS:scripts/foobar/org/somewhere/xyz.lua|
% \end{quote}
% But I do not know, how to achieve this in a reliable way using
% \textsf{kpathsea}.
%
% \subsubsection{Package \xpackage{luatex-loader}}
%
% If someone do not need or want package \xpackage{luatex} but
% it's extension for module loading, then he can use
% package \xpackage{luatex-loader}. Both \plainTeX\ and \LaTeX\ are
% supported.
%
% \StopEventually{
% }
%
% \section{Implementation}
%
%    \begin{macrocode}
%<*package>
%    \end{macrocode}
%
% \subsection{Reload check and package identification}
%    Reload check, especially if the package is not used with \LaTeX.
%    \begin{macrocode}
\begingroup\catcode61\catcode48\catcode32=10\relax%
  \catcode13=5 % ^^M
  \endlinechar=13 %
  \catcode35=6 % #
  \catcode39=12 % '
  \catcode44=12 % ,
  \catcode45=12 % -
  \catcode46=12 % .
  \catcode58=12 % :
  \catcode64=11 % @
  \catcode123=1 % {
  \catcode125=2 % }
  \expandafter\let\expandafter\x\csname ver@luatex.sty\endcsname
  \ifx\x\relax % plain-TeX, first loading
  \else
    \def\empty{}%
    \ifx\x\empty % LaTeX, first loading,
      % variable is initialized, but \ProvidesPackage not yet seen
    \else
      \expandafter\ifx\csname PackageInfo\endcsname\relax
        \def\x#1#2{%
          \immediate\write-1{Package #1 Info: #2.}%
        }%
      \else
        \def\x#1#2{\PackageInfo{#1}{#2, stopped}}%
      \fi
      \x{luatex}{The package is already loaded}%
      \aftergroup\endinput
    \fi
  \fi
\endgroup%
%    \end{macrocode}
%    Package identification:
%    \begin{macrocode}
\begingroup\catcode61\catcode48\catcode32=10\relax%
  \catcode13=5 % ^^M
  \endlinechar=13 %
  \catcode35=6 % #
  \catcode39=12 % '
  \catcode40=12 % (
  \catcode41=12 % )
  \catcode44=12 % ,
  \catcode45=12 % -
  \catcode46=12 % .
  \catcode47=12 % /
  \catcode58=12 % :
  \catcode64=11 % @
  \catcode91=12 % [
  \catcode93=12 % ]
  \catcode123=1 % {
  \catcode125=2 % }
  \expandafter\ifx\csname ProvidesPackage\endcsname\relax
    \def\x#1#2#3[#4]{\endgroup
      \immediate\write-1{Package: #3 #4}%
      \xdef#1{#4}%
    }%
  \else
    \def\x#1#2[#3]{\endgroup
      #2[{#3}]%
      \ifx#1\@undefined
        \xdef#1{#3}%
      \fi
      \ifx#1\relax
        \xdef#1{#3}%
      \fi
    }%
  \fi
\expandafter\x\csname ver@luatex.sty\endcsname
\ProvidesPackage{luatex}%
  [2010/03/09 v0.4 LuaTeX basic definition package (HO)]%
%    \end{macrocode}
%
% \subsection{Catcodes}
%
%    \begin{macrocode}
\begingroup\catcode61\catcode48\catcode32=10\relax%
  \catcode13=5 % ^^M
  \endlinechar=13 %
  \catcode123=1 % {
  \catcode125=2 % }
  \catcode64=11 % @
  \def\x{\endgroup
    \expandafter\edef\csname LuT@AtEnd\endcsname{%
      \endlinechar=\the\endlinechar\relax
      \catcode13=\the\catcode13\relax
      \catcode32=\the\catcode32\relax
      \catcode35=\the\catcode35\relax
      \catcode61=\the\catcode61\relax
      \catcode64=\the\catcode64\relax
      \catcode123=\the\catcode123\relax
      \catcode125=\the\catcode125\relax
    }%
  }%
\x\catcode61\catcode48\catcode32=10\relax%
\catcode13=5 % ^^M
\endlinechar=13 %
\catcode35=6 % #
\catcode64=11 % @
\catcode123=1 % {
\catcode125=2 % }
\def\TMP@EnsureCode#1#2{%
  \edef\LuT@AtEnd{%
    \LuT@AtEnd
    \catcode#1=\the\catcode#1\relax
  }%
  \catcode#1=#2\relax
}
\TMP@EnsureCode{10}{12}% ^^J
\TMP@EnsureCode{34}{12}% "
\TMP@EnsureCode{36}{3}% $
\TMP@EnsureCode{39}{12}% '
\TMP@EnsureCode{40}{12}% (
\TMP@EnsureCode{41}{12}% )
\TMP@EnsureCode{42}{12}% *
\TMP@EnsureCode{43}{12}% +
\TMP@EnsureCode{44}{12}% ,
\TMP@EnsureCode{45}{12}% -
\TMP@EnsureCode{46}{12}% .
\TMP@EnsureCode{47}{12}% /
\TMP@EnsureCode{60}{12}% <
\TMP@EnsureCode{62}{12}% >
\TMP@EnsureCode{91}{12}% [
\TMP@EnsureCode{93}{12}% ]
\TMP@EnsureCode{95}{12}% _ (other!)
\TMP@EnsureCode{96}{12}% `
\edef\LuT@AtEnd{\LuT@AtEnd\noexpand\endinput}
%    \end{macrocode}
%
% \subsection{Check for \LuaTeX}
%
%    Without \LuaTeX\ there is no point in using this package.
%    \begin{macrocode}
\begingroup\expandafter\expandafter\expandafter\endgroup
\expandafter\ifx\csname RequirePackage\endcsname\relax
  \input infwarerr.sty\relax
  \input ifluatex.sty\relax
\else
  \RequirePackage{infwarerr}[2007/09/09]%
  \RequirePackage{ifluatex}[2009/04/10]%
\fi
%    \end{macrocode}
%    \begin{macrocode}
\ifluatex
\else
  \@PackageError{luatex}{%
    This package may only be run using LuaTeX%
  }\@ehc
  \expandafter\LuT@AtEnd
\fi%
%    \end{macrocode}
%
% \subsection{Provide Lua\TeX\ primitives}
%
%    \begin{macrocode}
\ifnum\luatexversion<36 %
  \def\LuT@MakePrimitive#1{%
    \expandafter\let\csname luatex#1\expandafter\endcsname
    \csname #1\endcsname
  }%
\else
  \def\LuT@MakeLuatexPrimitive#1{%
    \begingroup\expandafter\expandafter\expandafter\endgroup
    \expandafter\ifx\csname luatex#1\endcsname\relax
      \begingroup\expandafter\expandafter\expandafter\endgroup
      \expandafter\ifx\csname #1\endcsname\relax
      \else
        \expandafter\let
        \csname luatex#1\expandafter\endcsname
        \csname #1\endcsname
      \fi
    \fi
    \begingroup\expandafter\expandafter\expandafter\endgroup
    \expandafter\ifx\csname luatex#1\endcsname\relax
      \begingroup
        \expandafter\let\csname luatex#1\endcsname\@undefined
        \ifnum0%
          \directlua{%
            if tex.enableprimitives then %
              tex.enableprimitives('luatex',{'#1'})%
              tex.print('1')%
            end%
          }%
          \expandafter\ifx\csname luatex#1\endcsname\relax\else1\fi
        =11 %
          \global\expandafter\let
          \csname luatex#1\expandafter\endcsname
          \csname luatex#1\endcsname
        \else
          \@PackageError{luatex}{%
            tex.enableprimitives failed for `#1'%
          }\@ehc
        \fi
      \endgroup
    \fi
  }%
  \def\LuT@MakePrimitive#1{%
    \begingroup\expandafter\expandafter\expandafter\endgroup
    \expandafter\ifx\csname#1\endcsname\relax
      \begingroup
        \expandafter\let\csname#1\endcsname\@undefined
        \ifnum0%
          \directlua{%
            if tex.enableprimitives then %
              tex.enableprimitives('',{'#1'})%
              tex.print('1')%
            end%
          }%
          \expandafter\ifx\csname#1\endcsname\relax\else1\fi
        =11 %
          \global\expandafter\let
          \csname#1\expandafter\endcsname
          \csname#1\endcsname
        \else
          \@PackageError{luatex}{%
            tex.enableprimitives failed for `#1'%
          }\@ehc
        \fi
      \endgroup
    \fi
  }%
\fi
\LuT@MakeLuatexPrimitive{attribute}
\LuT@MakeLuatexPrimitive{attributedef}
\LuT@MakeLuatexPrimitive{catcodetable}
\LuT@MakeLuatexPrimitive{initcatcodetable}
\LuT@MakeLuatexPrimitive{luaescapestring}
\LuT@MakeLuatexPrimitive{savecatcodetable}
\LuT@MakePrimitive{numexpr}
%    \end{macrocode}
%
% \subsection{Inherit support for \eTeX}
%
%    Package \xpackage{etex} is not compatible for \plainTeX.
%    But it could be present if a format is used that is based
%    on \xfile{etex.src}. Therefore we only load the package
%    in case of \LaTeX\ and tests its presence independently
%    of the format by looking for \cs{et@xins}.
%    \begin{macrocode}
\begingroup\expandafter\expandafter\expandafter\endgroup
\expandafter\ifx\csname RequirePackage\endcsname\relax
\else
  \RequirePackage{etex}[1998/03/26]%
\fi
%    \end{macrocode}
%
% \subsection{Adaption of \eTeX's register allocation}
%
%    \eTeX has increased the number of \TeX\ registers
%    from $2^8$ (256) to $2^{15}$ (32768) for a register class.
%    \LuaTeX\ extends the limit further to $2^16$ (65536).
%    The allocation scheme of package \xpackage{etex} is not
%    changed. But this can be subject for discussion.
%
%    If a register class hasn't registered any local registers
%    yet, then the limit can safely be pushed to 65536.
%    \begin{macrocode}
\begingroup\expandafter\expandafter\expandafter\endgroup
\expandafter\ifx\csname et@xins\endcsname\relax
  \@PackageWarningNoLine{luatex}{%
    Support for eTeX is not loaded (etex.src)%
  }%
\else
  \def\LuT@temp#1{%
    \ifnum\count27#1=32768 %
      \count27#1=65536 %
    \fi
  }%
  \LuT@temp0%
  \LuT@temp1%
  \LuT@temp2%
  \LuT@temp3%
  \LuT@temp4%
  \LuT@temp5%
  \LuT@temp6%
%    \end{macrocode}
%
% \eTeX\ uses an array for the first 256 registers and then a tree
% structure. \LuaTeX\ stores all registers of a class in one Lua table.
% There shouldn't be large performance differences.
% This allows starting immediately in the extended
% area, leaving room for insertions.
%    \begin{macrocode}
  \let\newcount\globcount
  \let\newdimen\globdimen
  \let\newskip\globskip
  \let\newbox\globbox
\fi
%    \end{macrocode}
%
% \subsection{\plainTeX\ compatibility}
%
%    \begin{macro}{\@empty}
%    \begin{macrocode}
\expandafter\ifx\csname @empty\endcsname\relax
  \def\@empty{}%
\fi
%    \end{macrocode}
%    \end{macro}
%    \begin{macro}{\@gobble}
%    \begin{macrocode}
\expandafter\ifx\csname @gobble\endcsname\relax
  \long\def\@gobble#1{}%
\fi
%    \end{macrocode}
%    \end{macro}
%    \begin{macro}{\@firstofone}
%    \begin{macrocode}
\expandafter\ifx\csname @firstofone\endcsname\relax
  \long\def\@firstofone#1{#1}%
\fi
%    \end{macrocode}
%    \end{macro}
%    \begin{macro}{\@firstoftwo}
%    \begin{macrocode}
\expandafter\ifx\csname @firstoftwo\endcsname\relax
  \long\def\@firstoftwo#1#2{#1}%
\fi
%    \end{macrocode}
%    \end{macro}
%    \begin{macro}{\@car}
%    \begin{macrocode}
\expandafter\ifx\csname @car\endcsname\relax
  \def\@car#1#2\@nil{#1}%
\fi
%    \end{macrocode}
%    \end{macro}
%    \begin{macro}{\@cdr}
%    \begin{macrocode}
\expandafter\ifx\csname @cdr\endcsname\relax
  \def\@cdr#1#2\@nil{#2}%
\fi
%    \end{macrocode}
%    \end{macro}
%    \begin{macro}{\@ifstar}
%    \begin{macrocode}
\expandafter\ifx\csname @ifstar\endcsname\relax
  \def\@ifstar#1{%
    \@ifnextchar*{\@firstoftwo{#1}}%
  }%
%    \end{macrocode}
%    \end{macro}
%    \begin{macro}{\@ifnextchar}
%    \begin{macrocode}
  \long\def\@ifnextchar#1#2#3{%
    \let\reserved@d=#1%
    \def\reserved@a{#2}%
    \def\reserved@b{#3}%
    \futurelet\@let@token\@ifnch
  }%
%    \end{macrocode}
%    \end{macro}
%    \begin{macro}{\@ifnch}
%    \begin{macrocode}
  \def\@ifnch{%
    \ifx\@let@token\@sptoken
      \let\reserved@c\@xifnch
    \else
      \ifx\@let@token\reserved@d
        \let\reserved@c\reserved@a
      \else
        \let\reserved@c\reserved@b
      \fi
    \fi
    \reserved@c
  }%
%    \end{macrocode}
%    \end{macro}
%    \begin{macro}{\@sptoken}
%    \begin{macrocode}
  \let\LuT@temp\:%
  \def\:{\let\@sptoken= }%
  \: % explicit space
%    \end{macrocode}
%    \end{macro}
%    \begin{macro}{\@xifnch}
%    \begin{macrocode}
  \def\:{\@xifnch}%
  \expandafter\def\: {%
    \futurelet\@let@token\@ifnch
  }%
  \let\:\LuT@temp
\fi
%    \end{macrocode}
%    \end{macro}
%
%    \begin{macro}{\@tempcnta}
%    \begin{macrocode}
\expandafter\ifx\csname @tempcnta\endcsname\relax
  \csname newcount\endcsname\@tempcnta
\fi
%    \end{macrocode}
%    \end{macro}
%
%    \begin{macro}{\@tempcntb}
%    \begin{macrocode}
\expandafter\ifx\csname @tempcntb\endcsname\relax
  \csname newcount\endcsname\@tempcntb
\fi
%    \end{macrocode}
%    \end{macro}
%
%    \begin{macro}{\LuT@newcommand}
%    \begin{macrocode}
\begingroup\expandafter\expandafter\expandafter\endgroup
\expandafter\ifx\csname newcommand\endcsname\relax
  \def\LuT@newcommand#1[#2]#3{%
    \ifx#1\@undefined
      \let#1\relax
    \else
      \ifx#1\relax
      \else
        \@PackageError{luatex}{%
          \string#1 is already defined.\MessageBreak
          Redefinition is skipped%
        }\@ehc
      \fi
    \fi
    \ifx#1\relax
      \ifcase#2 %
        \def#1{#3}%
      \or
        \def#1##1{#3}%
      \or
        \def#1##1##2{#3}%
      \or
        \def#1##1##2##3{#3}%
      \or
        \@INTERNAL@ERROR
      \fi
    \fi
  }%
\else
  \def\LuT@newcommand{\newcommand*}%
\fi
%    \end{macrocode}
%    \end{macro}
%
% \subsection{Attributes}
%
% \subsubsection{Allocation}
%
%    \begin{macro}{\LuT@AllocAttribute}
%    \begin{macrocode}
\newcount\LuT@AllocAttribute
\LuT@AllocAttribute=\m@ne
%    \end{macrocode}
%    \end{macro}
%    \begin{macro}{\newattribute}
%    \begin{macrocode}
\LuT@newcommand\newattribute[1]{%
  \ifnum\LuT@AllocAttribute<65535 %
    \global\advance\LuT@AllocAttribute\@ne
    \allocationnumber\LuT@AllocAttribute
    \global\luatexattributedef#1=\allocationnumber
    \unsetattribute{#1}%
    \wlog{\string#1=\string\attribute\the\allocationnumber}%
  \else
    \errmessage{No room for a new \string\attribute}%
  \fi
}
%    \end{macrocode}
%    \end{macro}
%
% \subsubsection{Interface}
%
%    \begin{macro}{\setattribute}
%    \begin{macrocode}
\LuT@newcommand\setattribute[2]{%
  #1=\numexpr#2\relax
}
%    \end{macrocode}
%    \end{macro}
%    \begin{macro}{\unsetattribute}
%    \begin{macrocode}
\ifnum\luatexversion<37
  \LuT@newcommand\LuT@UnsetAttributeValue[0]{}%
  \let\LuT@UnsetAttributeValue\m@ne
\else
  \LuT@newcommand\LuT@UnsetAttributeValue[0]{-2147483647 }%
\fi
\LuT@newcommand\unsetattribute[1]{%
  #1=\LuT@UnsetAttributeValue
}
%    \end{macrocode}
%    \end{macro}
%
% \subsection{Catcode tables}
%
% \subsubsection{Allocation}
%
%    \begin{macro}{\LuT@AllocCatcodeTable}
%    \begin{macrocode}
\newcount\LuT@AllocCatcodeTable
\LuT@AllocCatcodeTable=\m@ne
\newcount\CatcodeTableStack
\CatcodeTableStack=\z@
%    \end{macrocode}
%    \end{macro}
%    \begin{macro}{\newcatcodetable}
%    \begin{macrocode}
\LuT@newcommand\newcatcodetable[1]{%
  \ifnum\LuT@AllocCatcodeTable<1114110 % 0x10FFFF is maximal \chardef
    % or < 268435455 % 2^28 - 1
    \global\advance\LuT@AllocCatcodeTable by\tw@
    \allocationnumber=\LuT@AllocCatcodeTable
    \global\chardef#1=\allocationnumber
    \wlog{%
      \string#1=\string\catcodetable\the\allocationnumber
    }%
  \else
    \errmessage{No room for a new \string\catcodetable}%
  \fi
}%
%    \end{macrocode}
%    \end{macro}
%
%    \begin{macro}{\IncCatcodeTableStack}
%    \begin{macrocode}
\LuT@newcommand\IncCatcodeTableStack[0]{%
  \ifnum\CatcodeTableStack<268435454 %
    \global\advance\CatcodeTableStack by\tw@
  \else
    \@PackageError{luatex}{%
      Catcode table stack overflow%
    }\@ehd
  \fi
}
%    \end{macrocode}
%    \end{macro}
%
%    \begin{macro}{\DecCatcodeTableStack}
%    \begin{macrocode}
\LuT@newcommand\DecCatcodeTableStack[0]{%
  \ifnum\CatcodeTableStack>\z@
    \global\advance\CatcodeTableStack by-2 %
  \else
    \@PackageError{luatex}{%
      Catcode table stack is empty%
    }\@ehd
  \fi
}
%    \end{macrocode}
%    \end{macro}
%
% \subsubsection{\cs{SetCatcodeRange}}
%
%    \begin{macro}{\SetCatcodeRange}
%    \begin{macrocode}
\LuT@newcommand\SetCatcodeRange[3]{%
  \edef\LuT@temp{%
    \noexpand\@tempcnta=\the\@tempcnta
    \noexpand\@tempcntb=\the\@tempcntb
    \noexpand\count@=\the\count@
    \relax
  }%
  \@tempcnta=\numexpr#1\relax
  \@tempcntb=\numexpr#2\relax
  \count@=\numexpr#3\relax
  \loop
    \unless\ifnum\@tempcnta>\@tempcntb
    \catcode\@tempcnta=\count@
    \advance\@tempcnta by \@ne
  \repeat
  \LuT@temp
}
%    \end{macrocode}
%    \end{macro}
%
% \subsubsection{Predefined catcode tables}
%
%    \begin{macrocode}
\newcatcodetable\CatcodeTableIniTeX
\newcatcodetable\CatcodeTableString
\newcatcodetable\CatcodeTableOther
\newcatcodetable\CatcodeTableLaTeX
%    \end{macrocode}
%
%    \begin{macrocode}
\luatexinitcatcodetable\CatcodeTableIniTeX
\begingroup
  \def\@makeother#1{\catcode#1=12\relax}%
  \@firstofone{%
    \luatexcatcodetable\CatcodeTableIniTeX
    \begingroup
      \SetCatcodeRange{0}{8}{15}%
      \catcode9=10 % tab
      \catcode11=15 %
      \catcode12=13 % form feed
      \SetCatcodeRange{14}{31}{15}%
      \catcode35=6 % hash
      \catcode36=3 % dollar
      \catcode38=4 % ampersand
      \catcode94=7 % circumflex
      \catcode95=8 % underscore
      \catcode123=1 % brace left
      \catcode125=2 % brace right
      \catcode126=13 % tilde
      \catcode127=15 %
      \luatexsavecatcodetable\CatcodeTableLaTeX
    \endgroup
    \@makeother{0}% nul
    \@makeother{13}% carriage return
    \@makeother{37}% percent
    \@makeother{92}% backslash
    \@makeother{127}%
    \SetCatcodeRange{65}{90}{12}% A-Z
    \SetCatcodeRange{97}{122}{12}% a-z
    \luatexsavecatcodetable\CatcodeTableString
    \@makeother{32}% space
    \luatexsavecatcodetable\CatcodeTableOther
  \endgroup
}%
%    \end{macrocode}
%
% \subsubsection{Number stack}
%
%    \begin{macro}{\LuT@NumStackEmpty}
%    A special empty stack value because of \cs{@cdr}'s brace removal.
%    \begin{macrocode}
\def\LuT@NumStackEmpty{0}
%    \end{macrocode}
%    \end{macro}
%    \begin{macro}{\LuT@NumStack}
%    \begin{macrocode}
\let\LuT@NumStack\LuT@NumStackEmpty
%    \end{macrocode}
%    \end{macro}
%    \begin{macro}{\PushCatcodeTableNumStack}
%    \begin{macrocode}
\LuT@newcommand\PushCatcodeTableNumStack[0]{%
  \xdef\LuT@NumStack{%
    {\the\luatexcatcodetable}\LuT@NumStack
  }%
}
%    \end{macrocode}
%    \end{macro}
%    \begin{macro}{\PopCatcodeTableNumStack}
%    \begin{macrocode}
\LuT@newcommand\PopCatcodeTableNumStack[0]{%
  \ifx\LuT@NumStack\LuT@NumStackEmpty
    \@PackageWarning{luatex}{Empty catcode table number stack}%
    \luatexcatcodetable\z@
  \else
    \luatexcatcodetable=\expandafter\@car\LuT@NumStack\@nil\relax
    \xdef\LuT@NumStack{%
      \expandafter\@cdr\LuT@NumStack\@nil
    }%
  \fi
}
%    \end{macrocode}
%    \end{macro}
%
% \subsubsection{Catcode regime macros}
%
%    \begin{macro}{\BeginCatcodeRegime}
%    \begin{macrocode}
\LuT@newcommand\BeginCatcodeRegime[1]{%
  \PushCatcodeTableNumStack
  \luatexcatcodetable=\numexpr#1\relax
  \IncCatcodeTableStack
  \luatexsavecatcodetable\CatcodeTableStack
  \luatexcatcodetable\CatcodeTableStack
}
%    \end{macrocode}
%    \end{macro}
%
%    \begin{macro}{\EndCatcodeRegime}
%    \begin{macrocode}
\LuT@newcommand\EndCatcodeRegime[0]{%
  \DecCatcodeTableStack
  \PopCatcodeTableNumStack
}
%    \end{macrocode}
%    \end{macro}
%
% \subsection{Lua module loader}
%
%    \begin{macrocode}
\begingroup\expandafter\expandafter\expandafter\endgroup
\expandafter\ifx\csname RequirePackage\endcsname\relax
  \input luatex-loader.sty\relax
\else
  \RequirePackage{luatex-loader}[2010/03/09]%
\fi
%    \end{macrocode}
%    \begin{macrocode}
\LuT@AtEnd%
%</package>
%    \end{macrocode}
%
%    \begin{macrocode}
%<*loader>
%    \end{macrocode}
%
%    Reload check, especially if the package is not used with \LaTeX.
%    \begin{macrocode}
\begingroup\catcode61\catcode48\catcode32=10\relax%
  \catcode13=5 % ^^M
  \endlinechar=13 %
  \catcode35=6 % #
  \catcode39=12 % '
  \catcode44=12 % ,
  \catcode45=12 % -
  \catcode46=12 % .
  \catcode58=12 % :
  \catcode64=11 % @
  \catcode123=1 % {
  \catcode125=2 % }
  \expandafter\let\expandafter\x\csname ver@luatex-loader.sty\endcsname
  \ifx\x\relax % plain-TeX, first loading
  \else
    \def\empty{}%
    \ifx\x\empty % LaTeX, first loading,
      % variable is initialized, but \ProvidesPackage not yet seen
    \else
      \expandafter\ifx\csname PackageInfo\endcsname\relax
        \def\x#1#2{%
          \immediate\write-1{Package #1 Info: #2.}%
        }%
      \else
        \def\x#1#2{\PackageInfo{#1}{#2, stopped}}%
      \fi
      \x{luatex-loader}{The package is already loaded}%
      \aftergroup\endinput
    \fi
  \fi
\endgroup%
%    \end{macrocode}
%    Package identification:
%    \begin{macrocode}
\begingroup\catcode61\catcode48\catcode32=10\relax%
  \catcode13=5 % ^^M
  \endlinechar=13 %
  \catcode35=6 % #
  \catcode39=12 % '
  \catcode40=12 % (
  \catcode41=12 % )
  \catcode44=12 % ,
  \catcode45=12 % -
  \catcode46=12 % .
  \catcode47=12 % /
  \catcode58=12 % :
  \catcode64=11 % @
  \catcode91=12 % [
  \catcode93=12 % ]
  \catcode123=1 % {
  \catcode125=2 % }
  \expandafter\ifx\csname ProvidesPackage\endcsname\relax
    \def\x#1#2#3[#4]{\endgroup
      \immediate\write-1{Package: #3 #4}%
      \xdef#1{#4}%
    }%
  \else
    \def\x#1#2[#3]{\endgroup
      #2[{#3}]%
      \ifx#1\@undefined
        \xdef#1{#3}%
      \fi
      \ifx#1\relax
        \xdef#1{#3}%
      \fi
    }%
  \fi
\expandafter\x\csname ver@luatex-loader.sty\endcsname
\ProvidesPackage{luatex-loader}%
  [2010/03/09 v0.4 Lua module loader (HO)]%
%    \end{macrocode}
%    \begin{macrocode}
\begingroup\catcode61\catcode48\catcode32=10\relax%
  \catcode13=5\endlinechar=13\relax%
  \catcode10=12 % ^^J
  \catcode34=12 % "
  \catcode39=12 % '
  \catcode40=12 % (
  \catcode41=12 % )
  \catcode44=12 % ,
  \catcode46=12 % .
  \catcode60=12 % <
  \catcode61=12 % =
  \catcode95=12 % _ (other!)
  \catcode96=12 % `
  \catcode123=1 % {
  \catcode125=2 % }
  \endlinechar=10 %
  \ifnum\luatexversion<36 %
    \directlua0%
  \else %
    \expandafter\directlua %
  \fi %
  {%
     do
      local script = "oberdiek.luatex.lua"
      local file = kpse.find_file(script, "texmfscripts")
      if file then
        texio.write_nl("(" .. file .. ")")
        dofile(file)
      else
        error("File `" .. script .. "' not found")
      end
    end
  }%
\endgroup\endinput%
%    \end{macrocode}
%
%    \begin{macrocode}
%</loader>
%    \end{macrocode}
%
% \subsection{Lua script}
%
%    Currently \LuaTeX\ does not use KPSE when searching for module
%    files. The following Lua script implements a workaround.
%    It extends |package.loader| by another search method.
%    Modules are found by the module name with extension \xext{lua}
%    similar to
%    \begin{quote}
%      \ttfamily |kpsewhich --format=texmfscripts| \meta{module}|.lua|
%    \end{quote}
%    Unhappily \textsf{kpsewhich} does not support directory components
%    in the file name. Therefore a module |a.b.c| cannot be installed
%    as \xfile{a/b/c.lua}. The script must be named \xfile{a.b.c.lua}.
%    \begin{macrocode}
%<*lua>
%    \end{macrocode}
%    \begin{macrocode}
module("oberdiek.luatex", package.seeall)
%    \end{macrocode}
%    \begin{macrocode}
function kpse_module_loader(module)
  local script = module .. ".lua"
  local file = kpse.find_file(script, "texmfscripts")
  if file then
    local loader, error = loadfile(file)
    if loader then
      texio.write_nl("(" .. file .. ")")
      return loader
    end
    return "\n\t[oberdiek.luatex.kpse_module_loader] Loading error:\n\t"
           .. error
  end
  return "\n\t[oberdiek.luatex.kpse_module_loader] Search failed"
end
table.insert(package.loaders, kpse_module_loader)
%    \end{macrocode}
%    \begin{macrocode}
%</lua>
%    \end{macrocode}
%
% \section{Test}
%
%    \begin{macrocode}
%<*test2>
\documentclass{article}
\def\LoadCommand{%
  \RequirePackage{luatex}[2010/03/09]%
}
%</test2>
%    \end{macrocode}
%    \begin{macrocode}
%<*test3>
\documentclass{article}
\def\LoadCommand{%
  \RequirePackage{luatex-loader}[2010/03/09]%
}
%</test3>
%    \end{macrocode}
% \subsection{Catcode checks for loading}
%
%    \begin{macrocode}
%<*test1>
%    \end{macrocode}
%    \begin{macrocode}
\catcode`\{=1 %
\catcode`\}=2 %
\catcode`\#=6 %
\catcode`\@=11 %
\expandafter\ifx\csname count@\endcsname\relax
  \countdef\count@=255 %
\fi
\expandafter\ifx\csname @gobble\endcsname\relax
  \long\def\@gobble#1{}%
\fi
\expandafter\ifx\csname @firstofone\endcsname\relax
  \long\def\@firstofone#1{#1}%
\fi
\expandafter\ifx\csname loop\endcsname\relax
  \expandafter\@firstofone
\else
  \expandafter\@gobble
\fi
{%
  \def\loop#1\repeat{%
    \def\body{#1}%
    \iterate
  }%
  \def\iterate{%
    \body
      \let\next\iterate
    \else
      \let\next\relax
    \fi
    \next
  }%
  \let\repeat=\fi
}%
\def\RestoreCatcodes{}
\count@=0 %
\loop
  \edef\RestoreCatcodes{%
    \RestoreCatcodes
    \catcode\the\count@=\the\catcode\count@\relax
  }%
\ifnum\count@<255 %
  \advance\count@ 1 %
\repeat

\def\RangeCatcodeInvalid#1#2{%
  \count@=#1\relax
  \loop
    \catcode\count@=15 %
  \ifnum\count@<#2\relax
    \advance\count@ 1 %
  \repeat
}
\def\RangeCatcodeCheck#1#2#3{%
  \count@=#1\relax
  \loop
    \ifnum#3=\catcode\count@
    \else
      \errmessage{%
        Character \the\count@\space
        with wrong catcode \the\catcode\count@\space
        instead of \number#3%
      }%
    \fi
  \ifnum\count@<#2\relax
    \advance\count@ 1 %
  \repeat
}
\def\space{ }
\expandafter\ifx\csname LoadCommand\endcsname\relax
  \def\LoadCommand{\input luatex.sty\relax}%
\fi
\def\Test{%
  \RangeCatcodeInvalid{0}{47}%
  \RangeCatcodeInvalid{58}{64}%
  \RangeCatcodeInvalid{91}{96}%
  \RangeCatcodeInvalid{123}{255}%
  \catcode`\@=12 %
  \catcode`\\=0 %
  \catcode`\%=14 %
  \LoadCommand
  \RangeCatcodeCheck{0}{36}{15}%
  \RangeCatcodeCheck{37}{37}{14}%
  \RangeCatcodeCheck{38}{47}{15}%
  \RangeCatcodeCheck{48}{57}{12}%
  \RangeCatcodeCheck{58}{63}{15}%
  \RangeCatcodeCheck{64}{64}{12}%
  \RangeCatcodeCheck{65}{90}{11}%
  \RangeCatcodeCheck{91}{91}{15}%
  \RangeCatcodeCheck{92}{92}{0}%
  \RangeCatcodeCheck{93}{96}{15}%
  \RangeCatcodeCheck{97}{122}{11}%
  \RangeCatcodeCheck{123}{255}{15}%
  \RestoreCatcodes
}
\Test
\csname @@end\endcsname
\end
%    \end{macrocode}
%    \begin{macrocode}
%</test1>
%    \end{macrocode}
%
% \subsection{Catcode tables}
%
% \subsubsection{Predefined catcode tables}
%
%    \begin{macrocode}
%<*test4>
\NeedsTeXFormat{LaTeX2e}
%    \end{macrocode}
%    Remember \LaTeX's initial catcodes in count registers
%    starting at \cs{TestLaTeX}.
%    \begin{macrocode}
\count0=0 %
\chardef\TestLaTeX=1000 %
\chardef\TestMax=300 %
\loop
  \count\numexpr\TestLaTeX+\count0\relax=\catcode\count0 %
\ifnum\count0<\TestMax
  \advance\count0 by 1 %
\repeat
\documentclass{minimal}
\usepackage{luatex}[2010/03/09]
\usepackage{qstest}
\IncludeTests{*}
\LogTests{log}{*}{*}
\makeatletter
\def\Check#1{%
  \Expect*{\the\count@=\the\catcode\count@}%
         *{\the\count@=#1}%
}
\newcount\scratch
\def\Test#1#2{%
  \begin{qstest}{CatcodeTable#1}{CatcodeTable#1}%
    \luatexcatcodetable\csname CatcodeTable#1\endcsname
    \count@=\z@
    \loop
      \scratch=#2\relax
      \Expect*{\the\count@=\the\catcode\count@}%
             *{\the\count@=\the\scratch}%
    \ifnum\count@<\TestMax
      \advance\count@\@ne
    \repeat
  \end{qstest}%
}
\begingroup
  % luatex-unicode-letters.tex makes some slots to letters
  \def\TestMax{169}%
  \Test{LaTeX}{\the\count\numexpr\TestLaTeX+\count@}%
\endgroup
\Test{String}{\ifnum\count@=32 10\else 12\fi}
\Test{Other}{12}
\luatexinitcatcodetable99 %
\Test{IniTeX}{%
  0\relax
  \begingroup
    \luatexcatcodetable99 %
    \global\scratch=\the\catcode\count@
  \endgroup
}
%    \end{macrocode}
%
% \subsubsection{Catcode table number stack}
%
%    \begin{macrocode}
\begin{qstest}{CatcodeTableNumStack}{CatcodeTableNumStack}
  \def\TestStack#1{%
    \Expect*{\LuT@NumStack}{#1}%
  }%
  \TestStack{0}%
  \PushCatcodeTableNumStack
  \TestStack{{0}0}%
  \@firstofone{%
    \begingroup
      \luatexinitcatcodetable12 %
      \luatexcatcodetable12 %
      \PushCatcodeTableNumStack
      \TestStack{{12}{0}0}%
      \PopCatcodeTableNumStack
      \TestStack{{0}0}%
      \PopCatcodeTableNumStack
      \TestStack{0}%
      \def\TestWarning{Missing empty stack warning}%
      \def\@PackageWarning#1#2{\def\TestWarning{empty stack}}%
      \PopCatcodeTableNumStack
      \TestStack{0}%
      \Expect*{\TestWarning}{empty stack}%
    \endgroup
  }%
\end{qstest}
%    \end{macrocode}
%
% \subsubsection{Catcode table stack}
%
%    \begin{macrocode}
\begin{qstest}{CatcodeTableStack}{CatcodeTableStack}
  \def\TestStack#1{%
    \Expect*{\the\CatcodeTableStack}{#1}%
  }%
  \TestStack{0}%
  \IncCatcodeTableStack
  \TestStack{2}%
  \IncCatcodeTableStack
  \TestStack{4}%
  \begingroup
    \IncCatcodeTableStack
    \TestStack{6}%
  \endgroup
  \TestStack{6}%
  \begingroup
    \DecCatcodeTableStack
    \TestStack{4}%
  \endgroup
  \TestStack{4}%
  \DecCatcodeTableStack
  \TestStack{2}%
  \DecCatcodeTableStack
  \TestStack{0}%
  \begingroup
    \def\TestError{Missing error}%
    \def\@PackageError#1#2#3{%
      \def\TestError{Empty stack}%
    }%
    \DecCatcodeTableStack
    \TestStack{0}%
    \Expect*{\TestError}{Empty stack}%
  \endgroup
\end{qstest}
%    \end{macrocode}
%
% \subsubsection{Catcode regime macros}
%
%    \begin{macrocode}
\begin{qstest}{CatcodeRegime}{CatcodeRegime}
  \def\TestStacks#1#2#3{%
    \Expect*{\the\luatexcatcodetable}{#1}%
    \Expect*{\the\CatcodeTableStack}{#2}%
    \Expect*{\LuT@NumStack}{#3}%
  }%
  \TestStacks{0}{0}{0}%
  \catcode`\|=7 %
  \BeginCatcodeRegime\CatcodeTableLaTeX
    \TestStacks{2}{2}{{0}0}%
    \Expect*{\the\catcode`\|}{12}%
  \EndCatcodeRegime
  \TestStacks{0}{0}{0}%
  \Expect*{\the\catcode`\|}{7}%
\end{qstest}
%    \end{macrocode}
%
% \subsection{Attribute allocation}
%
%    \begin{macrocode}
\begin{qstest}{Attributes}{Attributes}
  \newattribute\TestAttr
  \Expect*{\meaning\TestAttr}%
         *{\string\attribute\number\allocationnumber}%
  \Expect*{\the\allocationnumber}{0}%
  \begingroup
    \newattribute\TestAttr
    \Expect*{\the\allocationnumber}{1}%
  \endgroup
  \Expect*{\the\allocationnumber}{0}%
  \Expect*{\meaning\TestAttr}*{\string\attribute1}%
  \Expect*{\the\TestAttr}*{\number\LuT@UnsetAttributeValue}%
  \def\Test#1{%
    \setattribute\TestAttr{#1}%
    \Expect*{\the\TestAttr}{#1}%
  }%
  \Test{0}%
  \Test{1}%
  \Test{-1}%
  \Test{123}%
  \unsetattribute\TestAttr
  \Expect*{\the\TestAttr}*{\number\LuT@UnsetAttributeValue}%
  \begingroup
    \Expect*{\the\TestAttr}*{\number\LuT@UnsetAttributeValue}%
    \Test{1234}%
  \endgroup
  \Expect*{\the\TestAttr}*{\number\LuT@UnsetAttributeValue}%
\end{qstest}
%    \end{macrocode}
%
%    \begin{macrocode}
\@@end
%</test4>
%    \end{macrocode}
%
% \subsection{Short test for \plainTeX}
%
%    \begin{macrocode}
%<*test5>
\input luatex.sty\relax
\newattribute\TestAttr
\setattribute\TestAttr{10}
\unsetattribute\TestAttr
\newcatcodetable\TestCTa
\begingroup
  \SetCatcodeRange{`A}{`Z}{12}%
\endgroup
\BeginCatcodeRegime\CatcodeTableLaTeX
\EndCatcodeRegime
\end
%</test5>
%    \end{macrocode}
%
% \section{Installation}
%
% \subsection{Download}
%
% \paragraph{Package.} This package is available on
% CTAN\footnote{\url{ftp://ftp.ctan.org/tex-archive/}}:
% \begin{description}
% \item[\CTAN{macros/latex/contrib/oberdiek/luatex.dtx}] The source file.
% \item[\CTAN{macros/latex/contrib/oberdiek/luatex.pdf}] Documentation.
% \end{description}
%
%
% \paragraph{Bundle.} All the packages of the bundle `oberdiek'
% are also available in a TDS compliant ZIP archive. There
% the packages are already unpacked and the documentation files
% are generated. The files and directories obey the TDS standard.
% \begin{description}
% \item[\CTAN{install/macros/latex/contrib/oberdiek.tds.zip}]
% \end{description}
% \emph{TDS} refers to the standard ``A Directory Structure
% for \TeX\ Files'' (\CTAN{tds/tds.pdf}). Directories
% with \xfile{texmf} in their name are usually organized this way.
%
% \subsection{Bundle installation}
%
% \paragraph{Unpacking.} Unpack the \xfile{oberdiek.tds.zip} in the
% TDS tree (also known as \xfile{texmf} tree) of your choice.
% Example (linux):
% \begin{quote}
%   |unzip oberdiek.tds.zip -d ~/texmf|
% \end{quote}
%
% \paragraph{Script installation.}
% Check the directory \xfile{TDS:scripts/oberdiek/} for
% scripts that need further installation steps.
% Package \xpackage{attachfile2} comes with the Perl script
% \xfile{pdfatfi.pl} that should be installed in such a way
% that it can be called as \texttt{pdfatfi}.
% Example (linux):
% \begin{quote}
%   |chmod +x scripts/oberdiek/pdfatfi.pl|\\
%   |cp scripts/oberdiek/pdfatfi.pl /usr/local/bin/|
% \end{quote}
%
% \subsection{Package installation}
%
% \paragraph{Unpacking.} The \xfile{.dtx} file is a self-extracting
% \docstrip\ archive. The files are extracted by running the
% \xfile{.dtx} through \plainTeX:
% \begin{quote}
%   \verb|tex luatex.dtx|
% \end{quote}
%
% \paragraph{TDS.} Now the different files must be moved into
% the different directories in your installation TDS tree
% (also known as \xfile{texmf} tree):
% \begin{quote}
% \def\t{^^A
% \begin{tabular}{@{}>{\ttfamily}l@{ $\rightarrow$ }>{\ttfamily}l@{}}
%   luatex.sty & tex/generic/oberdiek/luatex.sty\\
%   luatex-loader.sty & tex/generic/oberdiek/luatex-loader.sty\\
%   oberdiek.luatex.lua & scripts/oberdiek/oberdiek.luatex.lua\\
%   luatex.pdf & doc/latex/oberdiek/luatex.pdf\\
%   test/luatex-test1.tex & doc/latex/oberdiek/test/luatex-test1.tex\\
%   test/luatex-test2.tex & doc/latex/oberdiek/test/luatex-test2.tex\\
%   test/luatex-test3.tex & doc/latex/oberdiek/test/luatex-test3.tex\\
%   test/luatex-test4.tex & doc/latex/oberdiek/test/luatex-test4.tex\\
%   test/luatex-test5.tex & doc/latex/oberdiek/test/luatex-test5.tex\\
%   luatex.dtx & source/latex/oberdiek/luatex.dtx\\
% \end{tabular}^^A
% }^^A
% \sbox0{\t}^^A
% \ifdim\wd0>\linewidth
%   \begingroup
%     \advance\linewidth by\leftmargin
%     \advance\linewidth by\rightmargin
%   \edef\x{\endgroup
%     \def\noexpand\lw{\the\linewidth}^^A
%   }\x
%   \def\lwbox{^^A
%     \leavevmode
%     \hbox to \linewidth{^^A
%       \kern-\leftmargin\relax
%       \hss
%       \usebox0
%       \hss
%       \kern-\rightmargin\relax
%     }^^A
%   }^^A
%   \ifdim\wd0>\lw
%     \sbox0{\small\t}^^A
%     \ifdim\wd0>\linewidth
%       \ifdim\wd0>\lw
%         \sbox0{\footnotesize\t}^^A
%         \ifdim\wd0>\linewidth
%           \ifdim\wd0>\lw
%             \sbox0{\scriptsize\t}^^A
%             \ifdim\wd0>\linewidth
%               \ifdim\wd0>\lw
%                 \sbox0{\tiny\t}^^A
%                 \ifdim\wd0>\linewidth
%                   \lwbox
%                 \else
%                   \usebox0
%                 \fi
%               \else
%                 \lwbox
%               \fi
%             \else
%               \usebox0
%             \fi
%           \else
%             \lwbox
%           \fi
%         \else
%           \usebox0
%         \fi
%       \else
%         \lwbox
%       \fi
%     \else
%       \usebox0
%     \fi
%   \else
%     \lwbox
%   \fi
% \else
%   \usebox0
% \fi
% \end{quote}
% If you have a \xfile{docstrip.cfg} that configures and enables \docstrip's
% TDS installing feature, then some files can already be in the right
% place, see the documentation of \docstrip.
%
% \subsection{Refresh file name databases}
%
% If your \TeX~distribution
% (\teTeX, \mikTeX, \dots) relies on file name databases, you must refresh
% these. For example, \teTeX\ users run \verb|texhash| or
% \verb|mktexlsr|.
%
% \subsection{Some details for the interested}
%
% \paragraph{Attached source.}
%
% The PDF documentation on CTAN also includes the
% \xfile{.dtx} source file. It can be extracted by
% AcrobatReader 6 or higher. Another option is \textsf{pdftk},
% e.g. unpack the file into the current directory:
% \begin{quote}
%   \verb|pdftk luatex.pdf unpack_files output .|
% \end{quote}
%
% \paragraph{Unpacking with \LaTeX.}
% The \xfile{.dtx} chooses its action depending on the format:
% \begin{description}
% \item[\plainTeX:] Run \docstrip\ and extract the files.
% \item[\LaTeX:] Generate the documentation.
% \end{description}
% If you insist on using \LaTeX\ for \docstrip\ (really,
% \docstrip\ does not need \LaTeX), then inform the autodetect routine
% about your intention:
% \begin{quote}
%   \verb|latex \let\install=y% \iffalse meta-comment
%
% File: luatex.dtx
% Version: 2010/03/09 v0.4
% Info: LuaTeX basic definition package
%
% Copyright (C) 2007, 2009, 2010 by
%    Heiko Oberdiek <heiko.oberdiek at googlemail.com>
%
% This work may be distributed and/or modified under the
% conditions of the LaTeX Project Public License, either
% version 1.3c of this license or (at your option) any later
% version. This version of this license is in
%    http://www.latex-project.org/lppl/lppl-1-3c.txt
% and the latest version of this license is in
%    http://www.latex-project.org/lppl.txt
% and version 1.3 or later is part of all distributions of
% LaTeX version 2005/12/01 or later.
%
% This work has the LPPL maintenance status "maintained".
%
% This Current Maintainer of this work is Heiko Oberdiek.
%
% The Base Interpreter refers to any `TeX-Format',
% because some files are installed in TDS:tex/generic//.
%
% This work consists of the main source file luatex.dtx
% and the derived files
%    luatex.sty, luatex.pdf, luatex.ins, luatex.drv, luatex-loader.sty,
%    luatex-test1.tex, luatex-test2.tex, luatex-test3.tex,
%    luatex-test4.tex, luatex-test5.tex, oberdiek.luatex.lua.
%
% Distribution:
%    CTAN:macros/latex/contrib/oberdiek/luatex.dtx
%    CTAN:macros/latex/contrib/oberdiek/luatex.pdf
%
% Unpacking:
%    (a) If luatex.ins is present:
%           tex luatex.ins
%    (b) Without luatex.ins:
%           tex luatex.dtx
%    (c) If you insist on using LaTeX
%           latex \let\install=y\input{luatex.dtx}
%        (quote the arguments according to the demands of your shell)
%
% Documentation:
%    (a) If luatex.drv is present:
%           latex luatex.drv
%    (b) Without luatex.drv:
%           latex luatex.dtx; ...
%    The class ltxdoc loads the configuration file ltxdoc.cfg
%    if available. Here you can specify further options, e.g.
%    use A4 as paper format:
%       \PassOptionsToClass{a4paper}{article}
%
%    Programm calls to get the documentation (example):
%       pdflatex luatex.dtx
%       makeindex -s gind.ist luatex.idx
%       pdflatex luatex.dtx
%       makeindex -s gind.ist luatex.idx
%       pdflatex luatex.dtx
%
% Installation:
%    TDS:tex/generic/oberdiek/luatex.sty
%    TDS:tex/generic/oberdiek/luatex-loader.sty
%    TDS:scripts/oberdiek/oberdiek.luatex.lua
%    TDS:doc/latex/oberdiek/luatex.pdf
%    TDS:doc/latex/oberdiek/test/luatex-test1.tex
%    TDS:doc/latex/oberdiek/test/luatex-test2.tex
%    TDS:doc/latex/oberdiek/test/luatex-test3.tex
%    TDS:doc/latex/oberdiek/test/luatex-test4.tex
%    TDS:doc/latex/oberdiek/test/luatex-test5.tex
%    TDS:source/latex/oberdiek/luatex.dtx
%
%<*ignore>
\begingroup
  \catcode123=1 %
  \catcode125=2 %
  \def\x{LaTeX2e}%
\expandafter\endgroup
\ifcase 0\ifx\install y1\fi\expandafter
         \ifx\csname processbatchFile\endcsname\relax\else1\fi
         \ifx\fmtname\x\else 1\fi\relax
\else\csname fi\endcsname
%</ignore>
%<*install>
\input docstrip.tex
\Msg{************************************************************************}
\Msg{* Installation}
\Msg{* Package: luatex 2010/03/09 v0.4 LuaTeX basic definition package (HO)}
\Msg{************************************************************************}

\keepsilent
\askforoverwritefalse

\let\MetaPrefix\relax
\preamble

This is a generated file.

Project: luatex
Version: 2010/03/09 v0.4

Copyright (C) 2007, 2009, 2010 by
   Heiko Oberdiek <heiko.oberdiek at googlemail.com>

This work may be distributed and/or modified under the
conditions of the LaTeX Project Public License, either
version 1.3c of this license or (at your option) any later
version. This version of this license is in
   http://www.latex-project.org/lppl/lppl-1-3c.txt
and the latest version of this license is in
   http://www.latex-project.org/lppl.txt
and version 1.3 or later is part of all distributions of
LaTeX version 2005/12/01 or later.

This work has the LPPL maintenance status "maintained".

This Current Maintainer of this work is Heiko Oberdiek.

The Base Interpreter refers to any `TeX-Format',
because some files are installed in TDS:tex/generic//.

This work consists of the main source file luatex.dtx
and the derived files
   luatex.sty, luatex.pdf, luatex.ins, luatex.drv, luatex-loader.sty,
   luatex-test1.tex, luatex-test2.tex, luatex-test3.tex,
   luatex-test4.tex, luatex-test5.tex, oberdiek.luatex.lua.

\endpreamble
\let\MetaPrefix\DoubleperCent

\generate{%
  \file{luatex.ins}{\from{luatex.dtx}{install}}%
  \file{luatex.drv}{\from{luatex.dtx}{driver}}%
  \usedir{tex/generic/oberdiek}%
  \file{luatex.sty}{\from{luatex.dtx}{package}}%
  \file{luatex-loader.sty}{\from{luatex.dtx}{loader}}%
  \usedir{doc/latex/oberdiek/test}%
  \file{luatex-test1.tex}{\from{luatex.dtx}{test1}}%
  \file{luatex-test2.tex}{\from{luatex.dtx}{test1,test2}}%
  \file{luatex-test3.tex}{\from{luatex.dtx}{test1,test3}}%
  \file{luatex-test4.tex}{\from{luatex.dtx}{test4}}%
  \file{luatex-test5.tex}{\from{luatex.dtx}{test5}}%
  \nopreamble
  \nopostamble
  \usedir{source/latex/oberdiek/catalogue}%
  \file{luatex.xml}{\from{luatex.dtx}{catalogue}}%
}
\def\MetaPrefix{-- }
\def\defaultpostamble{%
  \MetaPrefix^^J%
  \MetaPrefix\space End of File `\outFileName'.%
}
\def\currentpostamble{\defaultpostamble}%
\generate{%
  \usedir{scripts/oberdiek}%
  \file{oberdiek.luatex.lua}{\from{luatex.dtx}{lua}}%
}

\catcode32=13\relax% active space
\let =\space%
\Msg{************************************************************************}
\Msg{*}
\Msg{* To finish the installation you have to move the following}
\Msg{* files into a directory searched by TeX:}
\Msg{*}
\Msg{*     luatex.sty, luatex-loader.sty}
\Msg{*}
\Msg{* And install the following script file:}
\Msg{*}
\Msg{*     oberdiek.luatex.lua}
\Msg{*}
\Msg{* To produce the documentation run the file `luatex.drv'}
\Msg{* through LaTeX.}
\Msg{*}
\Msg{* Happy TeXing!}
\Msg{*}
\Msg{************************************************************************}

\endbatchfile
%</install>
%<*ignore>
\fi
%</ignore>
%<*driver>
\NeedsTeXFormat{LaTeX2e}
\ProvidesFile{luatex.drv}%
  [2010/03/09 v0.4 LuaTeX basic definition package (HO)]%
\documentclass{ltxdoc}
\usepackage{holtxdoc}[2011/11/22]
\usepackage{paralist}
\begin{document}
  \DocInput{luatex.dtx}%
\end{document}
%</driver>
% \fi
%
% \CheckSum{1429}
%
% \CharacterTable
%  {Upper-case    \A\B\C\D\E\F\G\H\I\J\K\L\M\N\O\P\Q\R\S\T\U\V\W\X\Y\Z
%   Lower-case    \a\b\c\d\e\f\g\h\i\j\k\l\m\n\o\p\q\r\s\t\u\v\w\x\y\z
%   Digits        \0\1\2\3\4\5\6\7\8\9
%   Exclamation   \!     Double quote  \"     Hash (number) \#
%   Dollar        \$     Percent       \%     Ampersand     \&
%   Acute accent  \'     Left paren    \(     Right paren   \)
%   Asterisk      \*     Plus          \+     Comma         \,
%   Minus         \-     Point         \.     Solidus       \/
%   Colon         \:     Semicolon     \;     Less than     \<
%   Equals        \=     Greater than  \>     Question mark \?
%   Commercial at \@     Left bracket  \[     Backslash     \\
%   Right bracket \]     Circumflex    \^     Underscore    \_
%   Grave accent  \`     Left brace    \{     Vertical bar  \|
%   Right brace   \}     Tilde         \~}
%
% \GetFileInfo{luatex.drv}
%
% \title{The \xpackage{luatex} package}
% \date{2010/03/09 v0.4}
% \author{Heiko Oberdiek\\\xemail{heiko.oberdiek at googlemail.com}}
%
% \maketitle
%
% \begin{abstract}
% This package manages the new and extended features and resources
% that \LuaTeX\ provides. Examples are attributes and catcode
% tables.
% \end{abstract}
%
% \tableofcontents
%
% \section{Documentation}
%
% \subsection{Introduction}
%
% \TeX\ provides global resources such as registers.
% But it does not provide an interface for managing these resources.
% For example, two packages want to use a counter register.
% If they take the same register
% number, then the use of both packages will conflict and they cannot
% be used together.
% Therefore formats such as \plainTeX\ or \LaTeX\ implement
% an allocation scheme for registers. A package reserves with
% \cs{newcount} an unused register number for its own exclusive use.
%
% Nowadays \TeX\ is not alone anymore: \eTeX, \pdfTeX\ and other
% compilers for \TeX\ are developed that extend and add new
% features and resources.
%
% Now \LuaTeX\ has reached beta state. It inherits most of
% \pdfTeX's features including \eTeX. Also it implements
% new concepts such as attributes or catcode tables.
%
% \subsubsection{\LaTeX}
%
% \LaTeXe\ is frozen and therefore refuses to even notice
% the new \TeX\ variants. Not even the old \eTeX\ is supported
% by its kernel. At least there is a third party package \xpackage{etex}
% that manages the new \eTeX\ resources.
%
% This package tries to do the same for \LuaTeX\ and starts to
% support at least a few of the new features.
%
% \subsubsection{\plainTeX}
%
% \LaTeX\ has inherited its resource handling from \plainTeX.
% The interface is basically the same: \cs{newcount}, \dots
% Therefore this package tries to follow this tradition by
% providing compatibility to \plainTeX. It can be loaded
% with \plainTeX\ and defines at least some of the
% features that this packages provides for \LaTeX.
%
% \subsection{Register allocation}
%
% \subsubsection{Register with 16 bit}
%
% Because \LuaTeX\ is a super set of \eTeX\ regarding registers,
% the register allocation scheme should not conflict with
% package \xpackage{etex}. Therefore this package is loaded
% to inherit its allocation scheme. The only change is currently
% that the limit is increased to 65536 registers for the following
% register classes:
% \begin{compactitem}
% \item count
% \item dimen
% \item skip
% \item muskip
% \item marks
% \item toks
% \item box
% \end{compactitem}
% This affects the number of global and local registers.
% Because it is done in a package and not in the kernel,
% it is possible that someone loads package \xpackage{etex}
% before uses the local allocation variants. This will prevent
% the extension for this register class. If more registers are
% needed, just load package \xpackage{luatex} earlier.
%
% \subsubsection{Insertions}
%
% Insertions need four registers \cs{count}, \cs{dimen}, \cs{skip},
% and \cs{box} with the same number. Usually they are allocated
% downwards from 254, 253, \dots Also \cs{newcount}, \cs{newdimen}, \dots
% fill up these register numbers from below before switching to
% higher register numbers by package \xpackage{etex}.
% When this occurs, no insertions can be allocated anymore.
%
% Therefore \cs{newcount}, \cs{newdimen}, \cs{newskip}, and
% \cs{newbox} are replaced by their global variants (\cs{globcount}, \dots)
% that use the higher numbers immediately, leaving the room for insertions.
% There should not be an efficiency penalty because \LuaTeX\ stores
% the registers of a class in the same Lua table unlike \eTeX, where
% registers below 256 are stored in an array and higher numbers
% are put in a tree structure.
%
% \subsection{Attributes}
%
% Nodes can have custom attributes in \LuaTeX. These attributes
% are organized by a new register class. As the other registers
% up to $2^{16}$ attributes are supported. An attribute value
% can be negative that means the attribute is not set. Otherwise
% \TeX's range of non-negative integers up to $2^31$ are available.
%
% \begin{declcs}{newattribute} \M{cmd}
% \end{declcs}
% Macro \cs{newattribute} defines command \meta{cmd} using \cs{attributedef}
% using an new attribute number. The new attribute is initially unset.
%
% \begin{declcs}{setattribute} \M{cmd} \M{value}
% \end{declcs}
% Macro \cs{setattribute} locally sets attribute command \meta{cmd}
% to the number \meta{value}. Valid values range from $-1$ until $2^31$
% (the upper limit is the same as for other \TeX\ integer numbers).
%
% \begin{declcs}{unsetattribute} \M{cmd}
% \end{declcs}
% Macro \cs{unsettattribute} clears the attribute command \meta{cmd}.
%
% \subsection{Catcode tables}
%
% \LuaTeX\ introduces catcode tables as new feature, see documentation.
% There is need for discussion, how to deal best:
% \begin{itemize}
% \item \cs{initcatcodetable} and \cs{setcatcodetable} act globally.
% \item \cs{catcodetable} causes an error if used with an uninitialized
%   catcode table.
% \item Large catcode table numbers should be avoided because of
%   performance breakdown.
% \item Use case \LaTeX\ package:
%   The package must not be surprised by changed catcodes and
%   must not surprise by changing catcodes accidently.
%   Catcode tables could offer a solution. At the begin
%   a catcode regime with standard catcodes is established
%   and the old one is restored afterwards.
% \item Use case: \LuaTeX's |tex.print| might be used with
%   a catcode table number, for example a table where all
%   entries have catcode ``other''.
% \item Readonly catcode tables.
% \item Is there is a need for local allocations?
%   (Package \xpackage{etex}'s \cs{loc} variants are not used in
%   \TeX\ Live 2007.)
% \end{itemize}
%
% \subsubsection{Interface proposal}
%
% The idea: \cs{newcatcodetable} allocates odd numbered catcode
% tables. Even numbered tables are managed as stack.
% Also some catcode tables are defined. These must not be
% changed.
%
% \begin{declcs}{newcatcodetable} \M{cmd}
% \end{declcs}
% Macro \cs{newcatcodetable} reserves a new catcode table and
% remembers its number in \meta{cmd}. The catcode table is
% initialized with \iniTeX's catcodes.
%
% \begin{declcs}{CatcodeTableIniTeX}\\
%   \cs{CatcodeTableString}\\
%   \cs{CatcodeTableOther}\\
%   \cs{CatcodeTableLaTeX}\\
% \end{declcs}
% These are catcode tables and must not be changed.
% \cs{CatcodeTableIniTeX} contains the catcode settings of \iniTeX.
% \cs{CatcodeTableString} follows \TeX's convention of
% \cs{string}, \cs{meaning} and friends. The space gets catcode
% 10 (space), the other characters have catcode 12 (other).
% In \cs{CatcodeTableOther} all entries have catcode 12 (other).
% \cs{CatcodeTableLaTeX} contains the setting of a pure \LaTeX\ format
% (`at' is other).
%
% \begin{declcs}{CatcodeTableStack}\\
%   \cs{IncCatcodeTableStack}\\
%   \cs{DecCatcodeTableStack}
% \end{declcs}
% \cs{CatcodeTableStack} is the stack pointer. Initially
% it is catcode table zero. \cs{IncCatcodeTableStack} and
% \cs{DecCatcodeTableStack} increments and decrements the
% stack pointer. Currently \cs{IncCatcodeTableStack} does not
% initialize a new catcode table. Both increment and decrement
% operations do not set a catcode table.
%
% \begin{declcs}{PushCatcodeTableNumStack}\\
%   \cs{PopCatcodeTableNumStack}
% \end{declcs}
% It can be handy to have a global stack for catcode table
% numbers to deal with the global assignment property of
% \cs{initcatcodetable} and \cs{savecatcodetable}.
% \cs{PushCatcodeTableNumStack} pushes the current catcode table
% on the stack.
% \cs{PopCatcodeTableNumStack} pops the topmost number off the number
% stack to set the current catcode table. Catcode table zero is
% used in case of an empty stack.
%
% \begin{declcs}{BeginCatcodeRegime} \M{catcodetable}\\
%   \cs{EndCatcodeRegime}
% \end{declcs}
% \cs{BeginCatcodeRegime} remembers the current catcode table number.
% Then it creates and uses a fresh catcode table on the stack
% that is initialized by \meta{catcodetable}:
% \begin{quote}\ttfamily
%   \cs{PushCatcodeTableNumStack}\\
%   \cs{catcodetable}\meta{catcodetable}||
%   \cs{IncCatcodeTableStack}\\
%   \cs{savecatcodetable}\cs{CatcodeTableStack}\\
%   \cs{catcodetable}\cs{CatcodeTableStack}
% \end{quote}
% \cs{EndCatcodeRegime} drops the catcode table, created by
% \cs{BeginCatcodeRegime} and sets the catcode table that was
% active before:
% \begin{quote}
%   \cs{DecCatcodeTableStack}\\
%   \cs{PopCatcodeTableNumStack}
% \end{quote}
% These macros solve the use case, described earlier for a \LaTeX\ package:
% \begin{quote}
%   |% package foobar.sty|\\
%   |\BeginCatcodeRegime\CatcodeTableLaTeX|\\
%   |\makeatletter|\\
%   |% ... package contents ...|\\
%   |\EndCatcodeRegime|\\
%   |% end of package|
% \end{quote}
% If the package wants to change catcodes after its loading,
% \cs{AtBeginDocument} or \cs{AtEndOfPackage} can be used.
%
% \begin{declcs}{SetCatcodeRange} \M{from} \M{to} \M{catcode}
% \end{declcs}
% The catcodes of characters in range from \meta{from}
% to inclusive \meta{to} are set to \meta{catcode}.
%
% \subsection{Lua module loading}
%
% Currently \LuaTeX\ (version 0.20) does not support Lua script
% files inside \xfile{TDS:scripts//}, because Lua's mechanism
% for module loading does not use the \textsf{kpathsea} library.
% Therefore this packages appends a kpse loader to the list of
% Lua's module loaders. It finds the module \meta{module} by
% \begin{quote}
%   \ttfamily |kpse.find_file("|\meta{module}|.lua", "texmfscripts")|
% \end{quote}
% Unhappily \textsf{kpathsea} does not support directory components
% in a file name. Therefore the Lua convention is not followed to
% replace dots in the module name by the directory separator.
%
% Example:
% A Lua script of a package \xpackage{foobar} wants the following
% modules:
% \begin{quote}
%   |require("foobar.hello.world")|\\
%   |require("org.somewhere.xyz")|
% \end{quote}
% Then they can be find in:
% \begin{quote}
%   |TDS:scripts/foobar/foobar.hello.world.lua|\\
%   |TDS:scripts/foobar/org.somewhere.xyz.lua|
% \end{quote}
% I would have preferred the following locations, following lua conventions,
% e.\,g.:
% \begin{quote}
%   |TDS:scripts/foobar/hello/world.lua|\\
%   |TDS:scripts/foobar/org/somewhere/xyz.lua|
% \end{quote}
% But I do not know, how to achieve this in a reliable way using
% \textsf{kpathsea}.
%
% \subsubsection{Package \xpackage{luatex-loader}}
%
% If someone do not need or want package \xpackage{luatex} but
% it's extension for module loading, then he can use
% package \xpackage{luatex-loader}. Both \plainTeX\ and \LaTeX\ are
% supported.
%
% \StopEventually{
% }
%
% \section{Implementation}
%
%    \begin{macrocode}
%<*package>
%    \end{macrocode}
%
% \subsection{Reload check and package identification}
%    Reload check, especially if the package is not used with \LaTeX.
%    \begin{macrocode}
\begingroup\catcode61\catcode48\catcode32=10\relax%
  \catcode13=5 % ^^M
  \endlinechar=13 %
  \catcode35=6 % #
  \catcode39=12 % '
  \catcode44=12 % ,
  \catcode45=12 % -
  \catcode46=12 % .
  \catcode58=12 % :
  \catcode64=11 % @
  \catcode123=1 % {
  \catcode125=2 % }
  \expandafter\let\expandafter\x\csname ver@luatex.sty\endcsname
  \ifx\x\relax % plain-TeX, first loading
  \else
    \def\empty{}%
    \ifx\x\empty % LaTeX, first loading,
      % variable is initialized, but \ProvidesPackage not yet seen
    \else
      \expandafter\ifx\csname PackageInfo\endcsname\relax
        \def\x#1#2{%
          \immediate\write-1{Package #1 Info: #2.}%
        }%
      \else
        \def\x#1#2{\PackageInfo{#1}{#2, stopped}}%
      \fi
      \x{luatex}{The package is already loaded}%
      \aftergroup\endinput
    \fi
  \fi
\endgroup%
%    \end{macrocode}
%    Package identification:
%    \begin{macrocode}
\begingroup\catcode61\catcode48\catcode32=10\relax%
  \catcode13=5 % ^^M
  \endlinechar=13 %
  \catcode35=6 % #
  \catcode39=12 % '
  \catcode40=12 % (
  \catcode41=12 % )
  \catcode44=12 % ,
  \catcode45=12 % -
  \catcode46=12 % .
  \catcode47=12 % /
  \catcode58=12 % :
  \catcode64=11 % @
  \catcode91=12 % [
  \catcode93=12 % ]
  \catcode123=1 % {
  \catcode125=2 % }
  \expandafter\ifx\csname ProvidesPackage\endcsname\relax
    \def\x#1#2#3[#4]{\endgroup
      \immediate\write-1{Package: #3 #4}%
      \xdef#1{#4}%
    }%
  \else
    \def\x#1#2[#3]{\endgroup
      #2[{#3}]%
      \ifx#1\@undefined
        \xdef#1{#3}%
      \fi
      \ifx#1\relax
        \xdef#1{#3}%
      \fi
    }%
  \fi
\expandafter\x\csname ver@luatex.sty\endcsname
\ProvidesPackage{luatex}%
  [2010/03/09 v0.4 LuaTeX basic definition package (HO)]%
%    \end{macrocode}
%
% \subsection{Catcodes}
%
%    \begin{macrocode}
\begingroup\catcode61\catcode48\catcode32=10\relax%
  \catcode13=5 % ^^M
  \endlinechar=13 %
  \catcode123=1 % {
  \catcode125=2 % }
  \catcode64=11 % @
  \def\x{\endgroup
    \expandafter\edef\csname LuT@AtEnd\endcsname{%
      \endlinechar=\the\endlinechar\relax
      \catcode13=\the\catcode13\relax
      \catcode32=\the\catcode32\relax
      \catcode35=\the\catcode35\relax
      \catcode61=\the\catcode61\relax
      \catcode64=\the\catcode64\relax
      \catcode123=\the\catcode123\relax
      \catcode125=\the\catcode125\relax
    }%
  }%
\x\catcode61\catcode48\catcode32=10\relax%
\catcode13=5 % ^^M
\endlinechar=13 %
\catcode35=6 % #
\catcode64=11 % @
\catcode123=1 % {
\catcode125=2 % }
\def\TMP@EnsureCode#1#2{%
  \edef\LuT@AtEnd{%
    \LuT@AtEnd
    \catcode#1=\the\catcode#1\relax
  }%
  \catcode#1=#2\relax
}
\TMP@EnsureCode{10}{12}% ^^J
\TMP@EnsureCode{34}{12}% "
\TMP@EnsureCode{36}{3}% $
\TMP@EnsureCode{39}{12}% '
\TMP@EnsureCode{40}{12}% (
\TMP@EnsureCode{41}{12}% )
\TMP@EnsureCode{42}{12}% *
\TMP@EnsureCode{43}{12}% +
\TMP@EnsureCode{44}{12}% ,
\TMP@EnsureCode{45}{12}% -
\TMP@EnsureCode{46}{12}% .
\TMP@EnsureCode{47}{12}% /
\TMP@EnsureCode{60}{12}% <
\TMP@EnsureCode{62}{12}% >
\TMP@EnsureCode{91}{12}% [
\TMP@EnsureCode{93}{12}% ]
\TMP@EnsureCode{95}{12}% _ (other!)
\TMP@EnsureCode{96}{12}% `
\edef\LuT@AtEnd{\LuT@AtEnd\noexpand\endinput}
%    \end{macrocode}
%
% \subsection{Check for \LuaTeX}
%
%    Without \LuaTeX\ there is no point in using this package.
%    \begin{macrocode}
\begingroup\expandafter\expandafter\expandafter\endgroup
\expandafter\ifx\csname RequirePackage\endcsname\relax
  \input infwarerr.sty\relax
  \input ifluatex.sty\relax
\else
  \RequirePackage{infwarerr}[2007/09/09]%
  \RequirePackage{ifluatex}[2009/04/10]%
\fi
%    \end{macrocode}
%    \begin{macrocode}
\ifluatex
\else
  \@PackageError{luatex}{%
    This package may only be run using LuaTeX%
  }\@ehc
  \expandafter\LuT@AtEnd
\fi%
%    \end{macrocode}
%
% \subsection{Provide Lua\TeX\ primitives}
%
%    \begin{macrocode}
\ifnum\luatexversion<36 %
  \def\LuT@MakePrimitive#1{%
    \expandafter\let\csname luatex#1\expandafter\endcsname
    \csname #1\endcsname
  }%
\else
  \def\LuT@MakeLuatexPrimitive#1{%
    \begingroup\expandafter\expandafter\expandafter\endgroup
    \expandafter\ifx\csname luatex#1\endcsname\relax
      \begingroup\expandafter\expandafter\expandafter\endgroup
      \expandafter\ifx\csname #1\endcsname\relax
      \else
        \expandafter\let
        \csname luatex#1\expandafter\endcsname
        \csname #1\endcsname
      \fi
    \fi
    \begingroup\expandafter\expandafter\expandafter\endgroup
    \expandafter\ifx\csname luatex#1\endcsname\relax
      \begingroup
        \expandafter\let\csname luatex#1\endcsname\@undefined
        \ifnum0%
          \directlua{%
            if tex.enableprimitives then %
              tex.enableprimitives('luatex',{'#1'})%
              tex.print('1')%
            end%
          }%
          \expandafter\ifx\csname luatex#1\endcsname\relax\else1\fi
        =11 %
          \global\expandafter\let
          \csname luatex#1\expandafter\endcsname
          \csname luatex#1\endcsname
        \else
          \@PackageError{luatex}{%
            tex.enableprimitives failed for `#1'%
          }\@ehc
        \fi
      \endgroup
    \fi
  }%
  \def\LuT@MakePrimitive#1{%
    \begingroup\expandafter\expandafter\expandafter\endgroup
    \expandafter\ifx\csname#1\endcsname\relax
      \begingroup
        \expandafter\let\csname#1\endcsname\@undefined
        \ifnum0%
          \directlua{%
            if tex.enableprimitives then %
              tex.enableprimitives('',{'#1'})%
              tex.print('1')%
            end%
          }%
          \expandafter\ifx\csname#1\endcsname\relax\else1\fi
        =11 %
          \global\expandafter\let
          \csname#1\expandafter\endcsname
          \csname#1\endcsname
        \else
          \@PackageError{luatex}{%
            tex.enableprimitives failed for `#1'%
          }\@ehc
        \fi
      \endgroup
    \fi
  }%
\fi
\LuT@MakeLuatexPrimitive{attribute}
\LuT@MakeLuatexPrimitive{attributedef}
\LuT@MakeLuatexPrimitive{catcodetable}
\LuT@MakeLuatexPrimitive{initcatcodetable}
\LuT@MakeLuatexPrimitive{luaescapestring}
\LuT@MakeLuatexPrimitive{savecatcodetable}
\LuT@MakePrimitive{numexpr}
%    \end{macrocode}
%
% \subsection{Inherit support for \eTeX}
%
%    Package \xpackage{etex} is not compatible for \plainTeX.
%    But it could be present if a format is used that is based
%    on \xfile{etex.src}. Therefore we only load the package
%    in case of \LaTeX\ and tests its presence independently
%    of the format by looking for \cs{et@xins}.
%    \begin{macrocode}
\begingroup\expandafter\expandafter\expandafter\endgroup
\expandafter\ifx\csname RequirePackage\endcsname\relax
\else
  \RequirePackage{etex}[1998/03/26]%
\fi
%    \end{macrocode}
%
% \subsection{Adaption of \eTeX's register allocation}
%
%    \eTeX has increased the number of \TeX\ registers
%    from $2^8$ (256) to $2^{15}$ (32768) for a register class.
%    \LuaTeX\ extends the limit further to $2^16$ (65536).
%    The allocation scheme of package \xpackage{etex} is not
%    changed. But this can be subject for discussion.
%
%    If a register class hasn't registered any local registers
%    yet, then the limit can safely be pushed to 65536.
%    \begin{macrocode}
\begingroup\expandafter\expandafter\expandafter\endgroup
\expandafter\ifx\csname et@xins\endcsname\relax
  \@PackageWarningNoLine{luatex}{%
    Support for eTeX is not loaded (etex.src)%
  }%
\else
  \def\LuT@temp#1{%
    \ifnum\count27#1=32768 %
      \count27#1=65536 %
    \fi
  }%
  \LuT@temp0%
  \LuT@temp1%
  \LuT@temp2%
  \LuT@temp3%
  \LuT@temp4%
  \LuT@temp5%
  \LuT@temp6%
%    \end{macrocode}
%
% \eTeX\ uses an array for the first 256 registers and then a tree
% structure. \LuaTeX\ stores all registers of a class in one Lua table.
% There shouldn't be large performance differences.
% This allows starting immediately in the extended
% area, leaving room for insertions.
%    \begin{macrocode}
  \let\newcount\globcount
  \let\newdimen\globdimen
  \let\newskip\globskip
  \let\newbox\globbox
\fi
%    \end{macrocode}
%
% \subsection{\plainTeX\ compatibility}
%
%    \begin{macro}{\@empty}
%    \begin{macrocode}
\expandafter\ifx\csname @empty\endcsname\relax
  \def\@empty{}%
\fi
%    \end{macrocode}
%    \end{macro}
%    \begin{macro}{\@gobble}
%    \begin{macrocode}
\expandafter\ifx\csname @gobble\endcsname\relax
  \long\def\@gobble#1{}%
\fi
%    \end{macrocode}
%    \end{macro}
%    \begin{macro}{\@firstofone}
%    \begin{macrocode}
\expandafter\ifx\csname @firstofone\endcsname\relax
  \long\def\@firstofone#1{#1}%
\fi
%    \end{macrocode}
%    \end{macro}
%    \begin{macro}{\@firstoftwo}
%    \begin{macrocode}
\expandafter\ifx\csname @firstoftwo\endcsname\relax
  \long\def\@firstoftwo#1#2{#1}%
\fi
%    \end{macrocode}
%    \end{macro}
%    \begin{macro}{\@car}
%    \begin{macrocode}
\expandafter\ifx\csname @car\endcsname\relax
  \def\@car#1#2\@nil{#1}%
\fi
%    \end{macrocode}
%    \end{macro}
%    \begin{macro}{\@cdr}
%    \begin{macrocode}
\expandafter\ifx\csname @cdr\endcsname\relax
  \def\@cdr#1#2\@nil{#2}%
\fi
%    \end{macrocode}
%    \end{macro}
%    \begin{macro}{\@ifstar}
%    \begin{macrocode}
\expandafter\ifx\csname @ifstar\endcsname\relax
  \def\@ifstar#1{%
    \@ifnextchar*{\@firstoftwo{#1}}%
  }%
%    \end{macrocode}
%    \end{macro}
%    \begin{macro}{\@ifnextchar}
%    \begin{macrocode}
  \long\def\@ifnextchar#1#2#3{%
    \let\reserved@d=#1%
    \def\reserved@a{#2}%
    \def\reserved@b{#3}%
    \futurelet\@let@token\@ifnch
  }%
%    \end{macrocode}
%    \end{macro}
%    \begin{macro}{\@ifnch}
%    \begin{macrocode}
  \def\@ifnch{%
    \ifx\@let@token\@sptoken
      \let\reserved@c\@xifnch
    \else
      \ifx\@let@token\reserved@d
        \let\reserved@c\reserved@a
      \else
        \let\reserved@c\reserved@b
      \fi
    \fi
    \reserved@c
  }%
%    \end{macrocode}
%    \end{macro}
%    \begin{macro}{\@sptoken}
%    \begin{macrocode}
  \let\LuT@temp\:%
  \def\:{\let\@sptoken= }%
  \: % explicit space
%    \end{macrocode}
%    \end{macro}
%    \begin{macro}{\@xifnch}
%    \begin{macrocode}
  \def\:{\@xifnch}%
  \expandafter\def\: {%
    \futurelet\@let@token\@ifnch
  }%
  \let\:\LuT@temp
\fi
%    \end{macrocode}
%    \end{macro}
%
%    \begin{macro}{\@tempcnta}
%    \begin{macrocode}
\expandafter\ifx\csname @tempcnta\endcsname\relax
  \csname newcount\endcsname\@tempcnta
\fi
%    \end{macrocode}
%    \end{macro}
%
%    \begin{macro}{\@tempcntb}
%    \begin{macrocode}
\expandafter\ifx\csname @tempcntb\endcsname\relax
  \csname newcount\endcsname\@tempcntb
\fi
%    \end{macrocode}
%    \end{macro}
%
%    \begin{macro}{\LuT@newcommand}
%    \begin{macrocode}
\begingroup\expandafter\expandafter\expandafter\endgroup
\expandafter\ifx\csname newcommand\endcsname\relax
  \def\LuT@newcommand#1[#2]#3{%
    \ifx#1\@undefined
      \let#1\relax
    \else
      \ifx#1\relax
      \else
        \@PackageError{luatex}{%
          \string#1 is already defined.\MessageBreak
          Redefinition is skipped%
        }\@ehc
      \fi
    \fi
    \ifx#1\relax
      \ifcase#2 %
        \def#1{#3}%
      \or
        \def#1##1{#3}%
      \or
        \def#1##1##2{#3}%
      \or
        \def#1##1##2##3{#3}%
      \or
        \@INTERNAL@ERROR
      \fi
    \fi
  }%
\else
  \def\LuT@newcommand{\newcommand*}%
\fi
%    \end{macrocode}
%    \end{macro}
%
% \subsection{Attributes}
%
% \subsubsection{Allocation}
%
%    \begin{macro}{\LuT@AllocAttribute}
%    \begin{macrocode}
\newcount\LuT@AllocAttribute
\LuT@AllocAttribute=\m@ne
%    \end{macrocode}
%    \end{macro}
%    \begin{macro}{\newattribute}
%    \begin{macrocode}
\LuT@newcommand\newattribute[1]{%
  \ifnum\LuT@AllocAttribute<65535 %
    \global\advance\LuT@AllocAttribute\@ne
    \allocationnumber\LuT@AllocAttribute
    \global\luatexattributedef#1=\allocationnumber
    \unsetattribute{#1}%
    \wlog{\string#1=\string\attribute\the\allocationnumber}%
  \else
    \errmessage{No room for a new \string\attribute}%
  \fi
}
%    \end{macrocode}
%    \end{macro}
%
% \subsubsection{Interface}
%
%    \begin{macro}{\setattribute}
%    \begin{macrocode}
\LuT@newcommand\setattribute[2]{%
  #1=\numexpr#2\relax
}
%    \end{macrocode}
%    \end{macro}
%    \begin{macro}{\unsetattribute}
%    \begin{macrocode}
\ifnum\luatexversion<37
  \LuT@newcommand\LuT@UnsetAttributeValue[0]{}%
  \let\LuT@UnsetAttributeValue\m@ne
\else
  \LuT@newcommand\LuT@UnsetAttributeValue[0]{-2147483647 }%
\fi
\LuT@newcommand\unsetattribute[1]{%
  #1=\LuT@UnsetAttributeValue
}
%    \end{macrocode}
%    \end{macro}
%
% \subsection{Catcode tables}
%
% \subsubsection{Allocation}
%
%    \begin{macro}{\LuT@AllocCatcodeTable}
%    \begin{macrocode}
\newcount\LuT@AllocCatcodeTable
\LuT@AllocCatcodeTable=\m@ne
\newcount\CatcodeTableStack
\CatcodeTableStack=\z@
%    \end{macrocode}
%    \end{macro}
%    \begin{macro}{\newcatcodetable}
%    \begin{macrocode}
\LuT@newcommand\newcatcodetable[1]{%
  \ifnum\LuT@AllocCatcodeTable<1114110 % 0x10FFFF is maximal \chardef
    % or < 268435455 % 2^28 - 1
    \global\advance\LuT@AllocCatcodeTable by\tw@
    \allocationnumber=\LuT@AllocCatcodeTable
    \global\chardef#1=\allocationnumber
    \wlog{%
      \string#1=\string\catcodetable\the\allocationnumber
    }%
  \else
    \errmessage{No room for a new \string\catcodetable}%
  \fi
}%
%    \end{macrocode}
%    \end{macro}
%
%    \begin{macro}{\IncCatcodeTableStack}
%    \begin{macrocode}
\LuT@newcommand\IncCatcodeTableStack[0]{%
  \ifnum\CatcodeTableStack<268435454 %
    \global\advance\CatcodeTableStack by\tw@
  \else
    \@PackageError{luatex}{%
      Catcode table stack overflow%
    }\@ehd
  \fi
}
%    \end{macrocode}
%    \end{macro}
%
%    \begin{macro}{\DecCatcodeTableStack}
%    \begin{macrocode}
\LuT@newcommand\DecCatcodeTableStack[0]{%
  \ifnum\CatcodeTableStack>\z@
    \global\advance\CatcodeTableStack by-2 %
  \else
    \@PackageError{luatex}{%
      Catcode table stack is empty%
    }\@ehd
  \fi
}
%    \end{macrocode}
%    \end{macro}
%
% \subsubsection{\cs{SetCatcodeRange}}
%
%    \begin{macro}{\SetCatcodeRange}
%    \begin{macrocode}
\LuT@newcommand\SetCatcodeRange[3]{%
  \edef\LuT@temp{%
    \noexpand\@tempcnta=\the\@tempcnta
    \noexpand\@tempcntb=\the\@tempcntb
    \noexpand\count@=\the\count@
    \relax
  }%
  \@tempcnta=\numexpr#1\relax
  \@tempcntb=\numexpr#2\relax
  \count@=\numexpr#3\relax
  \loop
    \unless\ifnum\@tempcnta>\@tempcntb
    \catcode\@tempcnta=\count@
    \advance\@tempcnta by \@ne
  \repeat
  \LuT@temp
}
%    \end{macrocode}
%    \end{macro}
%
% \subsubsection{Predefined catcode tables}
%
%    \begin{macrocode}
\newcatcodetable\CatcodeTableIniTeX
\newcatcodetable\CatcodeTableString
\newcatcodetable\CatcodeTableOther
\newcatcodetable\CatcodeTableLaTeX
%    \end{macrocode}
%
%    \begin{macrocode}
\luatexinitcatcodetable\CatcodeTableIniTeX
\begingroup
  \def\@makeother#1{\catcode#1=12\relax}%
  \@firstofone{%
    \luatexcatcodetable\CatcodeTableIniTeX
    \begingroup
      \SetCatcodeRange{0}{8}{15}%
      \catcode9=10 % tab
      \catcode11=15 %
      \catcode12=13 % form feed
      \SetCatcodeRange{14}{31}{15}%
      \catcode35=6 % hash
      \catcode36=3 % dollar
      \catcode38=4 % ampersand
      \catcode94=7 % circumflex
      \catcode95=8 % underscore
      \catcode123=1 % brace left
      \catcode125=2 % brace right
      \catcode126=13 % tilde
      \catcode127=15 %
      \luatexsavecatcodetable\CatcodeTableLaTeX
    \endgroup
    \@makeother{0}% nul
    \@makeother{13}% carriage return
    \@makeother{37}% percent
    \@makeother{92}% backslash
    \@makeother{127}%
    \SetCatcodeRange{65}{90}{12}% A-Z
    \SetCatcodeRange{97}{122}{12}% a-z
    \luatexsavecatcodetable\CatcodeTableString
    \@makeother{32}% space
    \luatexsavecatcodetable\CatcodeTableOther
  \endgroup
}%
%    \end{macrocode}
%
% \subsubsection{Number stack}
%
%    \begin{macro}{\LuT@NumStackEmpty}
%    A special empty stack value because of \cs{@cdr}'s brace removal.
%    \begin{macrocode}
\def\LuT@NumStackEmpty{0}
%    \end{macrocode}
%    \end{macro}
%    \begin{macro}{\LuT@NumStack}
%    \begin{macrocode}
\let\LuT@NumStack\LuT@NumStackEmpty
%    \end{macrocode}
%    \end{macro}
%    \begin{macro}{\PushCatcodeTableNumStack}
%    \begin{macrocode}
\LuT@newcommand\PushCatcodeTableNumStack[0]{%
  \xdef\LuT@NumStack{%
    {\the\luatexcatcodetable}\LuT@NumStack
  }%
}
%    \end{macrocode}
%    \end{macro}
%    \begin{macro}{\PopCatcodeTableNumStack}
%    \begin{macrocode}
\LuT@newcommand\PopCatcodeTableNumStack[0]{%
  \ifx\LuT@NumStack\LuT@NumStackEmpty
    \@PackageWarning{luatex}{Empty catcode table number stack}%
    \luatexcatcodetable\z@
  \else
    \luatexcatcodetable=\expandafter\@car\LuT@NumStack\@nil\relax
    \xdef\LuT@NumStack{%
      \expandafter\@cdr\LuT@NumStack\@nil
    }%
  \fi
}
%    \end{macrocode}
%    \end{macro}
%
% \subsubsection{Catcode regime macros}
%
%    \begin{macro}{\BeginCatcodeRegime}
%    \begin{macrocode}
\LuT@newcommand\BeginCatcodeRegime[1]{%
  \PushCatcodeTableNumStack
  \luatexcatcodetable=\numexpr#1\relax
  \IncCatcodeTableStack
  \luatexsavecatcodetable\CatcodeTableStack
  \luatexcatcodetable\CatcodeTableStack
}
%    \end{macrocode}
%    \end{macro}
%
%    \begin{macro}{\EndCatcodeRegime}
%    \begin{macrocode}
\LuT@newcommand\EndCatcodeRegime[0]{%
  \DecCatcodeTableStack
  \PopCatcodeTableNumStack
}
%    \end{macrocode}
%    \end{macro}
%
% \subsection{Lua module loader}
%
%    \begin{macrocode}
\begingroup\expandafter\expandafter\expandafter\endgroup
\expandafter\ifx\csname RequirePackage\endcsname\relax
  \input luatex-loader.sty\relax
\else
  \RequirePackage{luatex-loader}[2010/03/09]%
\fi
%    \end{macrocode}
%    \begin{macrocode}
\LuT@AtEnd%
%</package>
%    \end{macrocode}
%
%    \begin{macrocode}
%<*loader>
%    \end{macrocode}
%
%    Reload check, especially if the package is not used with \LaTeX.
%    \begin{macrocode}
\begingroup\catcode61\catcode48\catcode32=10\relax%
  \catcode13=5 % ^^M
  \endlinechar=13 %
  \catcode35=6 % #
  \catcode39=12 % '
  \catcode44=12 % ,
  \catcode45=12 % -
  \catcode46=12 % .
  \catcode58=12 % :
  \catcode64=11 % @
  \catcode123=1 % {
  \catcode125=2 % }
  \expandafter\let\expandafter\x\csname ver@luatex-loader.sty\endcsname
  \ifx\x\relax % plain-TeX, first loading
  \else
    \def\empty{}%
    \ifx\x\empty % LaTeX, first loading,
      % variable is initialized, but \ProvidesPackage not yet seen
    \else
      \expandafter\ifx\csname PackageInfo\endcsname\relax
        \def\x#1#2{%
          \immediate\write-1{Package #1 Info: #2.}%
        }%
      \else
        \def\x#1#2{\PackageInfo{#1}{#2, stopped}}%
      \fi
      \x{luatex-loader}{The package is already loaded}%
      \aftergroup\endinput
    \fi
  \fi
\endgroup%
%    \end{macrocode}
%    Package identification:
%    \begin{macrocode}
\begingroup\catcode61\catcode48\catcode32=10\relax%
  \catcode13=5 % ^^M
  \endlinechar=13 %
  \catcode35=6 % #
  \catcode39=12 % '
  \catcode40=12 % (
  \catcode41=12 % )
  \catcode44=12 % ,
  \catcode45=12 % -
  \catcode46=12 % .
  \catcode47=12 % /
  \catcode58=12 % :
  \catcode64=11 % @
  \catcode91=12 % [
  \catcode93=12 % ]
  \catcode123=1 % {
  \catcode125=2 % }
  \expandafter\ifx\csname ProvidesPackage\endcsname\relax
    \def\x#1#2#3[#4]{\endgroup
      \immediate\write-1{Package: #3 #4}%
      \xdef#1{#4}%
    }%
  \else
    \def\x#1#2[#3]{\endgroup
      #2[{#3}]%
      \ifx#1\@undefined
        \xdef#1{#3}%
      \fi
      \ifx#1\relax
        \xdef#1{#3}%
      \fi
    }%
  \fi
\expandafter\x\csname ver@luatex-loader.sty\endcsname
\ProvidesPackage{luatex-loader}%
  [2010/03/09 v0.4 Lua module loader (HO)]%
%    \end{macrocode}
%    \begin{macrocode}
\begingroup\catcode61\catcode48\catcode32=10\relax%
  \catcode13=5\endlinechar=13\relax%
  \catcode10=12 % ^^J
  \catcode34=12 % "
  \catcode39=12 % '
  \catcode40=12 % (
  \catcode41=12 % )
  \catcode44=12 % ,
  \catcode46=12 % .
  \catcode60=12 % <
  \catcode61=12 % =
  \catcode95=12 % _ (other!)
  \catcode96=12 % `
  \catcode123=1 % {
  \catcode125=2 % }
  \endlinechar=10 %
  \ifnum\luatexversion<36 %
    \directlua0%
  \else %
    \expandafter\directlua %
  \fi %
  {%
     do
      local script = "oberdiek.luatex.lua"
      local file = kpse.find_file(script, "texmfscripts")
      if file then
        texio.write_nl("(" .. file .. ")")
        dofile(file)
      else
        error("File `" .. script .. "' not found")
      end
    end
  }%
\endgroup\endinput%
%    \end{macrocode}
%
%    \begin{macrocode}
%</loader>
%    \end{macrocode}
%
% \subsection{Lua script}
%
%    Currently \LuaTeX\ does not use KPSE when searching for module
%    files. The following Lua script implements a workaround.
%    It extends |package.loader| by another search method.
%    Modules are found by the module name with extension \xext{lua}
%    similar to
%    \begin{quote}
%      \ttfamily |kpsewhich --format=texmfscripts| \meta{module}|.lua|
%    \end{quote}
%    Unhappily \textsf{kpsewhich} does not support directory components
%    in the file name. Therefore a module |a.b.c| cannot be installed
%    as \xfile{a/b/c.lua}. The script must be named \xfile{a.b.c.lua}.
%    \begin{macrocode}
%<*lua>
%    \end{macrocode}
%    \begin{macrocode}
module("oberdiek.luatex", package.seeall)
%    \end{macrocode}
%    \begin{macrocode}
function kpse_module_loader(module)
  local script = module .. ".lua"
  local file = kpse.find_file(script, "texmfscripts")
  if file then
    local loader, error = loadfile(file)
    if loader then
      texio.write_nl("(" .. file .. ")")
      return loader
    end
    return "\n\t[oberdiek.luatex.kpse_module_loader] Loading error:\n\t"
           .. error
  end
  return "\n\t[oberdiek.luatex.kpse_module_loader] Search failed"
end
table.insert(package.loaders, kpse_module_loader)
%    \end{macrocode}
%    \begin{macrocode}
%</lua>
%    \end{macrocode}
%
% \section{Test}
%
%    \begin{macrocode}
%<*test2>
\documentclass{article}
\def\LoadCommand{%
  \RequirePackage{luatex}[2010/03/09]%
}
%</test2>
%    \end{macrocode}
%    \begin{macrocode}
%<*test3>
\documentclass{article}
\def\LoadCommand{%
  \RequirePackage{luatex-loader}[2010/03/09]%
}
%</test3>
%    \end{macrocode}
% \subsection{Catcode checks for loading}
%
%    \begin{macrocode}
%<*test1>
%    \end{macrocode}
%    \begin{macrocode}
\catcode`\{=1 %
\catcode`\}=2 %
\catcode`\#=6 %
\catcode`\@=11 %
\expandafter\ifx\csname count@\endcsname\relax
  \countdef\count@=255 %
\fi
\expandafter\ifx\csname @gobble\endcsname\relax
  \long\def\@gobble#1{}%
\fi
\expandafter\ifx\csname @firstofone\endcsname\relax
  \long\def\@firstofone#1{#1}%
\fi
\expandafter\ifx\csname loop\endcsname\relax
  \expandafter\@firstofone
\else
  \expandafter\@gobble
\fi
{%
  \def\loop#1\repeat{%
    \def\body{#1}%
    \iterate
  }%
  \def\iterate{%
    \body
      \let\next\iterate
    \else
      \let\next\relax
    \fi
    \next
  }%
  \let\repeat=\fi
}%
\def\RestoreCatcodes{}
\count@=0 %
\loop
  \edef\RestoreCatcodes{%
    \RestoreCatcodes
    \catcode\the\count@=\the\catcode\count@\relax
  }%
\ifnum\count@<255 %
  \advance\count@ 1 %
\repeat

\def\RangeCatcodeInvalid#1#2{%
  \count@=#1\relax
  \loop
    \catcode\count@=15 %
  \ifnum\count@<#2\relax
    \advance\count@ 1 %
  \repeat
}
\def\RangeCatcodeCheck#1#2#3{%
  \count@=#1\relax
  \loop
    \ifnum#3=\catcode\count@
    \else
      \errmessage{%
        Character \the\count@\space
        with wrong catcode \the\catcode\count@\space
        instead of \number#3%
      }%
    \fi
  \ifnum\count@<#2\relax
    \advance\count@ 1 %
  \repeat
}
\def\space{ }
\expandafter\ifx\csname LoadCommand\endcsname\relax
  \def\LoadCommand{\input luatex.sty\relax}%
\fi
\def\Test{%
  \RangeCatcodeInvalid{0}{47}%
  \RangeCatcodeInvalid{58}{64}%
  \RangeCatcodeInvalid{91}{96}%
  \RangeCatcodeInvalid{123}{255}%
  \catcode`\@=12 %
  \catcode`\\=0 %
  \catcode`\%=14 %
  \LoadCommand
  \RangeCatcodeCheck{0}{36}{15}%
  \RangeCatcodeCheck{37}{37}{14}%
  \RangeCatcodeCheck{38}{47}{15}%
  \RangeCatcodeCheck{48}{57}{12}%
  \RangeCatcodeCheck{58}{63}{15}%
  \RangeCatcodeCheck{64}{64}{12}%
  \RangeCatcodeCheck{65}{90}{11}%
  \RangeCatcodeCheck{91}{91}{15}%
  \RangeCatcodeCheck{92}{92}{0}%
  \RangeCatcodeCheck{93}{96}{15}%
  \RangeCatcodeCheck{97}{122}{11}%
  \RangeCatcodeCheck{123}{255}{15}%
  \RestoreCatcodes
}
\Test
\csname @@end\endcsname
\end
%    \end{macrocode}
%    \begin{macrocode}
%</test1>
%    \end{macrocode}
%
% \subsection{Catcode tables}
%
% \subsubsection{Predefined catcode tables}
%
%    \begin{macrocode}
%<*test4>
\NeedsTeXFormat{LaTeX2e}
%    \end{macrocode}
%    Remember \LaTeX's initial catcodes in count registers
%    starting at \cs{TestLaTeX}.
%    \begin{macrocode}
\count0=0 %
\chardef\TestLaTeX=1000 %
\chardef\TestMax=300 %
\loop
  \count\numexpr\TestLaTeX+\count0\relax=\catcode\count0 %
\ifnum\count0<\TestMax
  \advance\count0 by 1 %
\repeat
\documentclass{minimal}
\usepackage{luatex}[2010/03/09]
\usepackage{qstest}
\IncludeTests{*}
\LogTests{log}{*}{*}
\makeatletter
\def\Check#1{%
  \Expect*{\the\count@=\the\catcode\count@}%
         *{\the\count@=#1}%
}
\newcount\scratch
\def\Test#1#2{%
  \begin{qstest}{CatcodeTable#1}{CatcodeTable#1}%
    \luatexcatcodetable\csname CatcodeTable#1\endcsname
    \count@=\z@
    \loop
      \scratch=#2\relax
      \Expect*{\the\count@=\the\catcode\count@}%
             *{\the\count@=\the\scratch}%
    \ifnum\count@<\TestMax
      \advance\count@\@ne
    \repeat
  \end{qstest}%
}
\begingroup
  % luatex-unicode-letters.tex makes some slots to letters
  \def\TestMax{169}%
  \Test{LaTeX}{\the\count\numexpr\TestLaTeX+\count@}%
\endgroup
\Test{String}{\ifnum\count@=32 10\else 12\fi}
\Test{Other}{12}
\luatexinitcatcodetable99 %
\Test{IniTeX}{%
  0\relax
  \begingroup
    \luatexcatcodetable99 %
    \global\scratch=\the\catcode\count@
  \endgroup
}
%    \end{macrocode}
%
% \subsubsection{Catcode table number stack}
%
%    \begin{macrocode}
\begin{qstest}{CatcodeTableNumStack}{CatcodeTableNumStack}
  \def\TestStack#1{%
    \Expect*{\LuT@NumStack}{#1}%
  }%
  \TestStack{0}%
  \PushCatcodeTableNumStack
  \TestStack{{0}0}%
  \@firstofone{%
    \begingroup
      \luatexinitcatcodetable12 %
      \luatexcatcodetable12 %
      \PushCatcodeTableNumStack
      \TestStack{{12}{0}0}%
      \PopCatcodeTableNumStack
      \TestStack{{0}0}%
      \PopCatcodeTableNumStack
      \TestStack{0}%
      \def\TestWarning{Missing empty stack warning}%
      \def\@PackageWarning#1#2{\def\TestWarning{empty stack}}%
      \PopCatcodeTableNumStack
      \TestStack{0}%
      \Expect*{\TestWarning}{empty stack}%
    \endgroup
  }%
\end{qstest}
%    \end{macrocode}
%
% \subsubsection{Catcode table stack}
%
%    \begin{macrocode}
\begin{qstest}{CatcodeTableStack}{CatcodeTableStack}
  \def\TestStack#1{%
    \Expect*{\the\CatcodeTableStack}{#1}%
  }%
  \TestStack{0}%
  \IncCatcodeTableStack
  \TestStack{2}%
  \IncCatcodeTableStack
  \TestStack{4}%
  \begingroup
    \IncCatcodeTableStack
    \TestStack{6}%
  \endgroup
  \TestStack{6}%
  \begingroup
    \DecCatcodeTableStack
    \TestStack{4}%
  \endgroup
  \TestStack{4}%
  \DecCatcodeTableStack
  \TestStack{2}%
  \DecCatcodeTableStack
  \TestStack{0}%
  \begingroup
    \def\TestError{Missing error}%
    \def\@PackageError#1#2#3{%
      \def\TestError{Empty stack}%
    }%
    \DecCatcodeTableStack
    \TestStack{0}%
    \Expect*{\TestError}{Empty stack}%
  \endgroup
\end{qstest}
%    \end{macrocode}
%
% \subsubsection{Catcode regime macros}
%
%    \begin{macrocode}
\begin{qstest}{CatcodeRegime}{CatcodeRegime}
  \def\TestStacks#1#2#3{%
    \Expect*{\the\luatexcatcodetable}{#1}%
    \Expect*{\the\CatcodeTableStack}{#2}%
    \Expect*{\LuT@NumStack}{#3}%
  }%
  \TestStacks{0}{0}{0}%
  \catcode`\|=7 %
  \BeginCatcodeRegime\CatcodeTableLaTeX
    \TestStacks{2}{2}{{0}0}%
    \Expect*{\the\catcode`\|}{12}%
  \EndCatcodeRegime
  \TestStacks{0}{0}{0}%
  \Expect*{\the\catcode`\|}{7}%
\end{qstest}
%    \end{macrocode}
%
% \subsection{Attribute allocation}
%
%    \begin{macrocode}
\begin{qstest}{Attributes}{Attributes}
  \newattribute\TestAttr
  \Expect*{\meaning\TestAttr}%
         *{\string\attribute\number\allocationnumber}%
  \Expect*{\the\allocationnumber}{0}%
  \begingroup
    \newattribute\TestAttr
    \Expect*{\the\allocationnumber}{1}%
  \endgroup
  \Expect*{\the\allocationnumber}{0}%
  \Expect*{\meaning\TestAttr}*{\string\attribute1}%
  \Expect*{\the\TestAttr}*{\number\LuT@UnsetAttributeValue}%
  \def\Test#1{%
    \setattribute\TestAttr{#1}%
    \Expect*{\the\TestAttr}{#1}%
  }%
  \Test{0}%
  \Test{1}%
  \Test{-1}%
  \Test{123}%
  \unsetattribute\TestAttr
  \Expect*{\the\TestAttr}*{\number\LuT@UnsetAttributeValue}%
  \begingroup
    \Expect*{\the\TestAttr}*{\number\LuT@UnsetAttributeValue}%
    \Test{1234}%
  \endgroup
  \Expect*{\the\TestAttr}*{\number\LuT@UnsetAttributeValue}%
\end{qstest}
%    \end{macrocode}
%
%    \begin{macrocode}
\@@end
%</test4>
%    \end{macrocode}
%
% \subsection{Short test for \plainTeX}
%
%    \begin{macrocode}
%<*test5>
\input luatex.sty\relax
\newattribute\TestAttr
\setattribute\TestAttr{10}
\unsetattribute\TestAttr
\newcatcodetable\TestCTa
\begingroup
  \SetCatcodeRange{`A}{`Z}{12}%
\endgroup
\BeginCatcodeRegime\CatcodeTableLaTeX
\EndCatcodeRegime
\end
%</test5>
%    \end{macrocode}
%
% \section{Installation}
%
% \subsection{Download}
%
% \paragraph{Package.} This package is available on
% CTAN\footnote{\url{ftp://ftp.ctan.org/tex-archive/}}:
% \begin{description}
% \item[\CTAN{macros/latex/contrib/oberdiek/luatex.dtx}] The source file.
% \item[\CTAN{macros/latex/contrib/oberdiek/luatex.pdf}] Documentation.
% \end{description}
%
%
% \paragraph{Bundle.} All the packages of the bundle `oberdiek'
% are also available in a TDS compliant ZIP archive. There
% the packages are already unpacked and the documentation files
% are generated. The files and directories obey the TDS standard.
% \begin{description}
% \item[\CTAN{install/macros/latex/contrib/oberdiek.tds.zip}]
% \end{description}
% \emph{TDS} refers to the standard ``A Directory Structure
% for \TeX\ Files'' (\CTAN{tds/tds.pdf}). Directories
% with \xfile{texmf} in their name are usually organized this way.
%
% \subsection{Bundle installation}
%
% \paragraph{Unpacking.} Unpack the \xfile{oberdiek.tds.zip} in the
% TDS tree (also known as \xfile{texmf} tree) of your choice.
% Example (linux):
% \begin{quote}
%   |unzip oberdiek.tds.zip -d ~/texmf|
% \end{quote}
%
% \paragraph{Script installation.}
% Check the directory \xfile{TDS:scripts/oberdiek/} for
% scripts that need further installation steps.
% Package \xpackage{attachfile2} comes with the Perl script
% \xfile{pdfatfi.pl} that should be installed in such a way
% that it can be called as \texttt{pdfatfi}.
% Example (linux):
% \begin{quote}
%   |chmod +x scripts/oberdiek/pdfatfi.pl|\\
%   |cp scripts/oberdiek/pdfatfi.pl /usr/local/bin/|
% \end{quote}
%
% \subsection{Package installation}
%
% \paragraph{Unpacking.} The \xfile{.dtx} file is a self-extracting
% \docstrip\ archive. The files are extracted by running the
% \xfile{.dtx} through \plainTeX:
% \begin{quote}
%   \verb|tex luatex.dtx|
% \end{quote}
%
% \paragraph{TDS.} Now the different files must be moved into
% the different directories in your installation TDS tree
% (also known as \xfile{texmf} tree):
% \begin{quote}
% \def\t{^^A
% \begin{tabular}{@{}>{\ttfamily}l@{ $\rightarrow$ }>{\ttfamily}l@{}}
%   luatex.sty & tex/generic/oberdiek/luatex.sty\\
%   luatex-loader.sty & tex/generic/oberdiek/luatex-loader.sty\\
%   oberdiek.luatex.lua & scripts/oberdiek/oberdiek.luatex.lua\\
%   luatex.pdf & doc/latex/oberdiek/luatex.pdf\\
%   test/luatex-test1.tex & doc/latex/oberdiek/test/luatex-test1.tex\\
%   test/luatex-test2.tex & doc/latex/oberdiek/test/luatex-test2.tex\\
%   test/luatex-test3.tex & doc/latex/oberdiek/test/luatex-test3.tex\\
%   test/luatex-test4.tex & doc/latex/oberdiek/test/luatex-test4.tex\\
%   test/luatex-test5.tex & doc/latex/oberdiek/test/luatex-test5.tex\\
%   luatex.dtx & source/latex/oberdiek/luatex.dtx\\
% \end{tabular}^^A
% }^^A
% \sbox0{\t}^^A
% \ifdim\wd0>\linewidth
%   \begingroup
%     \advance\linewidth by\leftmargin
%     \advance\linewidth by\rightmargin
%   \edef\x{\endgroup
%     \def\noexpand\lw{\the\linewidth}^^A
%   }\x
%   \def\lwbox{^^A
%     \leavevmode
%     \hbox to \linewidth{^^A
%       \kern-\leftmargin\relax
%       \hss
%       \usebox0
%       \hss
%       \kern-\rightmargin\relax
%     }^^A
%   }^^A
%   \ifdim\wd0>\lw
%     \sbox0{\small\t}^^A
%     \ifdim\wd0>\linewidth
%       \ifdim\wd0>\lw
%         \sbox0{\footnotesize\t}^^A
%         \ifdim\wd0>\linewidth
%           \ifdim\wd0>\lw
%             \sbox0{\scriptsize\t}^^A
%             \ifdim\wd0>\linewidth
%               \ifdim\wd0>\lw
%                 \sbox0{\tiny\t}^^A
%                 \ifdim\wd0>\linewidth
%                   \lwbox
%                 \else
%                   \usebox0
%                 \fi
%               \else
%                 \lwbox
%               \fi
%             \else
%               \usebox0
%             \fi
%           \else
%             \lwbox
%           \fi
%         \else
%           \usebox0
%         \fi
%       \else
%         \lwbox
%       \fi
%     \else
%       \usebox0
%     \fi
%   \else
%     \lwbox
%   \fi
% \else
%   \usebox0
% \fi
% \end{quote}
% If you have a \xfile{docstrip.cfg} that configures and enables \docstrip's
% TDS installing feature, then some files can already be in the right
% place, see the documentation of \docstrip.
%
% \subsection{Refresh file name databases}
%
% If your \TeX~distribution
% (\teTeX, \mikTeX, \dots) relies on file name databases, you must refresh
% these. For example, \teTeX\ users run \verb|texhash| or
% \verb|mktexlsr|.
%
% \subsection{Some details for the interested}
%
% \paragraph{Attached source.}
%
% The PDF documentation on CTAN also includes the
% \xfile{.dtx} source file. It can be extracted by
% AcrobatReader 6 or higher. Another option is \textsf{pdftk},
% e.g. unpack the file into the current directory:
% \begin{quote}
%   \verb|pdftk luatex.pdf unpack_files output .|
% \end{quote}
%
% \paragraph{Unpacking with \LaTeX.}
% The \xfile{.dtx} chooses its action depending on the format:
% \begin{description}
% \item[\plainTeX:] Run \docstrip\ and extract the files.
% \item[\LaTeX:] Generate the documentation.
% \end{description}
% If you insist on using \LaTeX\ for \docstrip\ (really,
% \docstrip\ does not need \LaTeX), then inform the autodetect routine
% about your intention:
% \begin{quote}
%   \verb|latex \let\install=y\input{luatex.dtx}|
% \end{quote}
% Do not forget to quote the argument according to the demands
% of your shell.
%
% \paragraph{Generating the documentation.}
% You can use both the \xfile{.dtx} or the \xfile{.drv} to generate
% the documentation. The process can be configured by the
% configuration file \xfile{ltxdoc.cfg}. For instance, put this
% line into this file, if you want to have A4 as paper format:
% \begin{quote}
%   \verb|\PassOptionsToClass{a4paper}{article}|
% \end{quote}
% An example follows how to generate the
% documentation with pdf\LaTeX:
% \begin{quote}
%\begin{verbatim}
%pdflatex luatex.dtx
%makeindex -s gind.ist luatex.idx
%pdflatex luatex.dtx
%makeindex -s gind.ist luatex.idx
%pdflatex luatex.dtx
%\end{verbatim}
% \end{quote}
%
% \section{Catalogue}
%
% The following XML file can be used as source for the
% \href{http://mirror.ctan.org/help/Catalogue/catalogue.html}{\TeX\ Catalogue}.
% The elements \texttt{caption} and \texttt{description} are imported
% from the original XML file from the Catalogue.
% The name of the XML file in the Catalogue is \xfile{luatex.xml}.
%    \begin{macrocode}
%<*catalogue>
<?xml version='1.0' encoding='us-ascii'?>
<!DOCTYPE entry SYSTEM 'catalogue.dtd'>
<entry datestamp='$Date$' modifier='$Author$' id='luatex'>
  <name>luatex</name>
  <caption>The LuaTeX engine.</caption>
  <authorref id='auth:oberdiek'/>
  <copyright owner='Heiko Oberdiek' year='2007,2009,2010'/>
  <license type='lppl1.3'/>
  <version number='0.4'/>
  <description>
    LuaTeX is an extended version of pdfTeX using Lua as an embedded
    scripting language.  The LuaTeX project&#x2019;s main objective
    is to provide an open and configurable variant of TeX while at the
    same time offering downward compatibility.
    <p/>
    LuaTeX uses Unicode (as UTF-8) as its default input encoding, and
    is able to use modern (OpenType) fonts (for both text and mathematics).
    <p/>
    It should be noted that LuaTeX is still under development; its
    specification has been declared stable, but absolute stability
    may not in practice be assumed.
    <p/>
    The package is part of the <xref refid='oberdiek'>oberdiek</xref> bundle.
  </description>
  <documentation details='Package documentation'
      href='ctan:/macros/latex/contrib/oberdiek/luatex.pdf'/>
  <ctan file='true' path='/macros/latex/contrib/oberdiek/luatex.dtx'/>
  <miktex location='oberdiek'/>
  <texlive location='oberdiek'/>
  <install path='/macros/latex/contrib/oberdiek/oberdiek.tds.zip'/>
</entry>
%</catalogue>
%    \end{macrocode}
%
% \begin{History}
%   \begin{Version}{2007/12/12 v0.1}
%   \item
%     First public version.
%   \end{Version}
%   \begin{Version}{2009/04/10 v0.2}
%   \item
%     Requires package \xpackage{ifluatex} in version 2.0 to ensure
%     \cs{luatexversion}.
%   \item
%     Updates the call of \cs{directlua}, the syntax has changed
%     in \LuaTeX\ 0.36.
%   \end{Version}
%   \begin{Version}{2009/12/02 v0.3}
%   \item
%     Unsetting of attributes updated for \LuaTeX\ 0.37.
%   \end{Version}
%   \begin{Version}{2010/03/09 v0.4}
%   \item
%     Support for lua states removed.
%   \item
%     Calling tex.enableprimitives for used primitives.
%   \end{Version}
% \end{History}
%
% \PrintIndex
%
% \Finale
\endinput
|
% \end{quote}
% Do not forget to quote the argument according to the demands
% of your shell.
%
% \paragraph{Generating the documentation.}
% You can use both the \xfile{.dtx} or the \xfile{.drv} to generate
% the documentation. The process can be configured by the
% configuration file \xfile{ltxdoc.cfg}. For instance, put this
% line into this file, if you want to have A4 as paper format:
% \begin{quote}
%   \verb|\PassOptionsToClass{a4paper}{article}|
% \end{quote}
% An example follows how to generate the
% documentation with pdf\LaTeX:
% \begin{quote}
%\begin{verbatim}
%pdflatex luatex.dtx
%makeindex -s gind.ist luatex.idx
%pdflatex luatex.dtx
%makeindex -s gind.ist luatex.idx
%pdflatex luatex.dtx
%\end{verbatim}
% \end{quote}
%
% \section{Catalogue}
%
% The following XML file can be used as source for the
% \href{http://mirror.ctan.org/help/Catalogue/catalogue.html}{\TeX\ Catalogue}.
% The elements \texttt{caption} and \texttt{description} are imported
% from the original XML file from the Catalogue.
% The name of the XML file in the Catalogue is \xfile{luatex.xml}.
%    \begin{macrocode}
%<*catalogue>
<?xml version='1.0' encoding='us-ascii'?>
<!DOCTYPE entry SYSTEM 'catalogue.dtd'>
<entry datestamp='$Date$' modifier='$Author$' id='luatex'>
  <name>luatex</name>
  <caption>The LuaTeX engine.</caption>
  <authorref id='auth:oberdiek'/>
  <copyright owner='Heiko Oberdiek' year='2007,2009,2010'/>
  <license type='lppl1.3'/>
  <version number='0.4'/>
  <description>
    LuaTeX is an extended version of pdfTeX using Lua as an embedded
    scripting language.  The LuaTeX project&#x2019;s main objective
    is to provide an open and configurable variant of TeX while at the
    same time offering downward compatibility.
    <p/>
    LuaTeX uses Unicode (as UTF-8) as its default input encoding, and
    is able to use modern (OpenType) fonts (for both text and mathematics).
    <p/>
    It should be noted that LuaTeX is still under development; its
    specification has been declared stable, but absolute stability
    may not in practice be assumed.
    <p/>
    The package is part of the <xref refid='oberdiek'>oberdiek</xref> bundle.
  </description>
  <documentation details='Package documentation'
      href='ctan:/macros/latex/contrib/oberdiek/luatex.pdf'/>
  <ctan file='true' path='/macros/latex/contrib/oberdiek/luatex.dtx'/>
  <miktex location='oberdiek'/>
  <texlive location='oberdiek'/>
  <install path='/macros/latex/contrib/oberdiek/oberdiek.tds.zip'/>
</entry>
%</catalogue>
%    \end{macrocode}
%
% \begin{History}
%   \begin{Version}{2007/12/12 v0.1}
%   \item
%     First public version.
%   \end{Version}
%   \begin{Version}{2009/04/10 v0.2}
%   \item
%     Requires package \xpackage{ifluatex} in version 2.0 to ensure
%     \cs{luatexversion}.
%   \item
%     Updates the call of \cs{directlua}, the syntax has changed
%     in \LuaTeX\ 0.36.
%   \end{Version}
%   \begin{Version}{2009/12/02 v0.3}
%   \item
%     Unsetting of attributes updated for \LuaTeX\ 0.37.
%   \end{Version}
%   \begin{Version}{2010/03/09 v0.4}
%   \item
%     Support for lua states removed.
%   \item
%     Calling tex.enableprimitives for used primitives.
%   \end{Version}
% \end{History}
%
% \PrintIndex
%
% \Finale
\endinput

%        (quote the arguments according to the demands of your shell)
%
% Documentation:
%    (a) If luatex.drv is present:
%           latex luatex.drv
%    (b) Without luatex.drv:
%           latex luatex.dtx; ...
%    The class ltxdoc loads the configuration file ltxdoc.cfg
%    if available. Here you can specify further options, e.g.
%    use A4 as paper format:
%       \PassOptionsToClass{a4paper}{article}
%
%    Programm calls to get the documentation (example):
%       pdflatex luatex.dtx
%       makeindex -s gind.ist luatex.idx
%       pdflatex luatex.dtx
%       makeindex -s gind.ist luatex.idx
%       pdflatex luatex.dtx
%
% Installation:
%    TDS:tex/generic/oberdiek/luatex.sty
%    TDS:tex/generic/oberdiek/luatex-loader.sty
%    TDS:scripts/oberdiek/oberdiek.luatex.lua
%    TDS:doc/latex/oberdiek/luatex.pdf
%    TDS:doc/latex/oberdiek/test/luatex-test1.tex
%    TDS:doc/latex/oberdiek/test/luatex-test2.tex
%    TDS:doc/latex/oberdiek/test/luatex-test3.tex
%    TDS:doc/latex/oberdiek/test/luatex-test4.tex
%    TDS:doc/latex/oberdiek/test/luatex-test5.tex
%    TDS:source/latex/oberdiek/luatex.dtx
%
%<*ignore>
\begingroup
  \catcode123=1 %
  \catcode125=2 %
  \def\x{LaTeX2e}%
\expandafter\endgroup
\ifcase 0\ifx\install y1\fi\expandafter
         \ifx\csname processbatchFile\endcsname\relax\else1\fi
         \ifx\fmtname\x\else 1\fi\relax
\else\csname fi\endcsname
%</ignore>
%<*install>
\input docstrip.tex
\Msg{************************************************************************}
\Msg{* Installation}
\Msg{* Package: luatex 2010/03/09 v0.4 LuaTeX basic definition package (HO)}
\Msg{************************************************************************}

\keepsilent
\askforoverwritefalse

\let\MetaPrefix\relax
\preamble

This is a generated file.

Project: luatex
Version: 2010/03/09 v0.4

Copyright (C) 2007, 2009, 2010 by
   Heiko Oberdiek <heiko.oberdiek at googlemail.com>

This work may be distributed and/or modified under the
conditions of the LaTeX Project Public License, either
version 1.3c of this license or (at your option) any later
version. This version of this license is in
   http://www.latex-project.org/lppl/lppl-1-3c.txt
and the latest version of this license is in
   http://www.latex-project.org/lppl.txt
and version 1.3 or later is part of all distributions of
LaTeX version 2005/12/01 or later.

This work has the LPPL maintenance status "maintained".

This Current Maintainer of this work is Heiko Oberdiek.

The Base Interpreter refers to any `TeX-Format',
because some files are installed in TDS:tex/generic//.

This work consists of the main source file luatex.dtx
and the derived files
   luatex.sty, luatex.pdf, luatex.ins, luatex.drv, luatex-loader.sty,
   luatex-test1.tex, luatex-test2.tex, luatex-test3.tex,
   luatex-test4.tex, luatex-test5.tex, oberdiek.luatex.lua.

\endpreamble
\let\MetaPrefix\DoubleperCent

\generate{%
  \file{luatex.ins}{\from{luatex.dtx}{install}}%
  \file{luatex.drv}{\from{luatex.dtx}{driver}}%
  \usedir{tex/generic/oberdiek}%
  \file{luatex.sty}{\from{luatex.dtx}{package}}%
  \file{luatex-loader.sty}{\from{luatex.dtx}{loader}}%
  \usedir{doc/latex/oberdiek/test}%
  \file{luatex-test1.tex}{\from{luatex.dtx}{test1}}%
  \file{luatex-test2.tex}{\from{luatex.dtx}{test1,test2}}%
  \file{luatex-test3.tex}{\from{luatex.dtx}{test1,test3}}%
  \file{luatex-test4.tex}{\from{luatex.dtx}{test4}}%
  \file{luatex-test5.tex}{\from{luatex.dtx}{test5}}%
  \nopreamble
  \nopostamble
  \usedir{source/latex/oberdiek/catalogue}%
  \file{luatex.xml}{\from{luatex.dtx}{catalogue}}%
}
\def\MetaPrefix{-- }
\def\defaultpostamble{%
  \MetaPrefix^^J%
  \MetaPrefix\space End of File `\outFileName'.%
}
\def\currentpostamble{\defaultpostamble}%
\generate{%
  \usedir{scripts/oberdiek}%
  \file{oberdiek.luatex.lua}{\from{luatex.dtx}{lua}}%
}

\catcode32=13\relax% active space
\let =\space%
\Msg{************************************************************************}
\Msg{*}
\Msg{* To finish the installation you have to move the following}
\Msg{* files into a directory searched by TeX:}
\Msg{*}
\Msg{*     luatex.sty, luatex-loader.sty}
\Msg{*}
\Msg{* And install the following script file:}
\Msg{*}
\Msg{*     oberdiek.luatex.lua}
\Msg{*}
\Msg{* To produce the documentation run the file `luatex.drv'}
\Msg{* through LaTeX.}
\Msg{*}
\Msg{* Happy TeXing!}
\Msg{*}
\Msg{************************************************************************}

\endbatchfile
%</install>
%<*ignore>
\fi
%</ignore>
%<*driver>
\NeedsTeXFormat{LaTeX2e}
\ProvidesFile{luatex.drv}%
  [2010/03/09 v0.4 LuaTeX basic definition package (HO)]%
\documentclass{ltxdoc}
\usepackage{holtxdoc}[2011/11/22]
\usepackage{paralist}
\begin{document}
  \DocInput{luatex.dtx}%
\end{document}
%</driver>
% \fi
%
% \CheckSum{1429}
%
% \CharacterTable
%  {Upper-case    \A\B\C\D\E\F\G\H\I\J\K\L\M\N\O\P\Q\R\S\T\U\V\W\X\Y\Z
%   Lower-case    \a\b\c\d\e\f\g\h\i\j\k\l\m\n\o\p\q\r\s\t\u\v\w\x\y\z
%   Digits        \0\1\2\3\4\5\6\7\8\9
%   Exclamation   \!     Double quote  \"     Hash (number) \#
%   Dollar        \$     Percent       \%     Ampersand     \&
%   Acute accent  \'     Left paren    \(     Right paren   \)
%   Asterisk      \*     Plus          \+     Comma         \,
%   Minus         \-     Point         \.     Solidus       \/
%   Colon         \:     Semicolon     \;     Less than     \<
%   Equals        \=     Greater than  \>     Question mark \?
%   Commercial at \@     Left bracket  \[     Backslash     \\
%   Right bracket \]     Circumflex    \^     Underscore    \_
%   Grave accent  \`     Left brace    \{     Vertical bar  \|
%   Right brace   \}     Tilde         \~}
%
% \GetFileInfo{luatex.drv}
%
% \title{The \xpackage{luatex} package}
% \date{2010/03/09 v0.4}
% \author{Heiko Oberdiek\\\xemail{heiko.oberdiek at googlemail.com}}
%
% \maketitle
%
% \begin{abstract}
% This package manages the new and extended features and resources
% that \LuaTeX\ provides. Examples are attributes and catcode
% tables.
% \end{abstract}
%
% \tableofcontents
%
% \section{Documentation}
%
% \subsection{Introduction}
%
% \TeX\ provides global resources such as registers.
% But it does not provide an interface for managing these resources.
% For example, two packages want to use a counter register.
% If they take the same register
% number, then the use of both packages will conflict and they cannot
% be used together.
% Therefore formats such as \plainTeX\ or \LaTeX\ implement
% an allocation scheme for registers. A package reserves with
% \cs{newcount} an unused register number for its own exclusive use.
%
% Nowadays \TeX\ is not alone anymore: \eTeX, \pdfTeX\ and other
% compilers for \TeX\ are developed that extend and add new
% features and resources.
%
% Now \LuaTeX\ has reached beta state. It inherits most of
% \pdfTeX's features including \eTeX. Also it implements
% new concepts such as attributes or catcode tables.
%
% \subsubsection{\LaTeX}
%
% \LaTeXe\ is frozen and therefore refuses to even notice
% the new \TeX\ variants. Not even the old \eTeX\ is supported
% by its kernel. At least there is a third party package \xpackage{etex}
% that manages the new \eTeX\ resources.
%
% This package tries to do the same for \LuaTeX\ and starts to
% support at least a few of the new features.
%
% \subsubsection{\plainTeX}
%
% \LaTeX\ has inherited its resource handling from \plainTeX.
% The interface is basically the same: \cs{newcount}, \dots
% Therefore this package tries to follow this tradition by
% providing compatibility to \plainTeX. It can be loaded
% with \plainTeX\ and defines at least some of the
% features that this packages provides for \LaTeX.
%
% \subsection{Register allocation}
%
% \subsubsection{Register with 16 bit}
%
% Because \LuaTeX\ is a super set of \eTeX\ regarding registers,
% the register allocation scheme should not conflict with
% package \xpackage{etex}. Therefore this package is loaded
% to inherit its allocation scheme. The only change is currently
% that the limit is increased to 65536 registers for the following
% register classes:
% \begin{compactitem}
% \item count
% \item dimen
% \item skip
% \item muskip
% \item marks
% \item toks
% \item box
% \end{compactitem}
% This affects the number of global and local registers.
% Because it is done in a package and not in the kernel,
% it is possible that someone loads package \xpackage{etex}
% before uses the local allocation variants. This will prevent
% the extension for this register class. If more registers are
% needed, just load package \xpackage{luatex} earlier.
%
% \subsubsection{Insertions}
%
% Insertions need four registers \cs{count}, \cs{dimen}, \cs{skip},
% and \cs{box} with the same number. Usually they are allocated
% downwards from 254, 253, \dots Also \cs{newcount}, \cs{newdimen}, \dots
% fill up these register numbers from below before switching to
% higher register numbers by package \xpackage{etex}.
% When this occurs, no insertions can be allocated anymore.
%
% Therefore \cs{newcount}, \cs{newdimen}, \cs{newskip}, and
% \cs{newbox} are replaced by their global variants (\cs{globcount}, \dots)
% that use the higher numbers immediately, leaving the room for insertions.
% There should not be an efficiency penalty because \LuaTeX\ stores
% the registers of a class in the same Lua table unlike \eTeX, where
% registers below 256 are stored in an array and higher numbers
% are put in a tree structure.
%
% \subsection{Attributes}
%
% Nodes can have custom attributes in \LuaTeX. These attributes
% are organized by a new register class. As the other registers
% up to $2^{16}$ attributes are supported. An attribute value
% can be negative that means the attribute is not set. Otherwise
% \TeX's range of non-negative integers up to $2^31$ are available.
%
% \begin{declcs}{newattribute} \M{cmd}
% \end{declcs}
% Macro \cs{newattribute} defines command \meta{cmd} using \cs{attributedef}
% using an new attribute number. The new attribute is initially unset.
%
% \begin{declcs}{setattribute} \M{cmd} \M{value}
% \end{declcs}
% Macro \cs{setattribute} locally sets attribute command \meta{cmd}
% to the number \meta{value}. Valid values range from $-1$ until $2^31$
% (the upper limit is the same as for other \TeX\ integer numbers).
%
% \begin{declcs}{unsetattribute} \M{cmd}
% \end{declcs}
% Macro \cs{unsettattribute} clears the attribute command \meta{cmd}.
%
% \subsection{Catcode tables}
%
% \LuaTeX\ introduces catcode tables as new feature, see documentation.
% There is need for discussion, how to deal best:
% \begin{itemize}
% \item \cs{initcatcodetable} and \cs{setcatcodetable} act globally.
% \item \cs{catcodetable} causes an error if used with an uninitialized
%   catcode table.
% \item Large catcode table numbers should be avoided because of
%   performance breakdown.
% \item Use case \LaTeX\ package:
%   The package must not be surprised by changed catcodes and
%   must not surprise by changing catcodes accidently.
%   Catcode tables could offer a solution. At the begin
%   a catcode regime with standard catcodes is established
%   and the old one is restored afterwards.
% \item Use case: \LuaTeX's |tex.print| might be used with
%   a catcode table number, for example a table where all
%   entries have catcode ``other''.
% \item Readonly catcode tables.
% \item Is there is a need for local allocations?
%   (Package \xpackage{etex}'s \cs{loc} variants are not used in
%   \TeX\ Live 2007.)
% \end{itemize}
%
% \subsubsection{Interface proposal}
%
% The idea: \cs{newcatcodetable} allocates odd numbered catcode
% tables. Even numbered tables are managed as stack.
% Also some catcode tables are defined. These must not be
% changed.
%
% \begin{declcs}{newcatcodetable} \M{cmd}
% \end{declcs}
% Macro \cs{newcatcodetable} reserves a new catcode table and
% remembers its number in \meta{cmd}. The catcode table is
% initialized with \iniTeX's catcodes.
%
% \begin{declcs}{CatcodeTableIniTeX}\\
%   \cs{CatcodeTableString}\\
%   \cs{CatcodeTableOther}\\
%   \cs{CatcodeTableLaTeX}\\
% \end{declcs}
% These are catcode tables and must not be changed.
% \cs{CatcodeTableIniTeX} contains the catcode settings of \iniTeX.
% \cs{CatcodeTableString} follows \TeX's convention of
% \cs{string}, \cs{meaning} and friends. The space gets catcode
% 10 (space), the other characters have catcode 12 (other).
% In \cs{CatcodeTableOther} all entries have catcode 12 (other).
% \cs{CatcodeTableLaTeX} contains the setting of a pure \LaTeX\ format
% (`at' is other).
%
% \begin{declcs}{CatcodeTableStack}\\
%   \cs{IncCatcodeTableStack}\\
%   \cs{DecCatcodeTableStack}
% \end{declcs}
% \cs{CatcodeTableStack} is the stack pointer. Initially
% it is catcode table zero. \cs{IncCatcodeTableStack} and
% \cs{DecCatcodeTableStack} increments and decrements the
% stack pointer. Currently \cs{IncCatcodeTableStack} does not
% initialize a new catcode table. Both increment and decrement
% operations do not set a catcode table.
%
% \begin{declcs}{PushCatcodeTableNumStack}\\
%   \cs{PopCatcodeTableNumStack}
% \end{declcs}
% It can be handy to have a global stack for catcode table
% numbers to deal with the global assignment property of
% \cs{initcatcodetable} and \cs{savecatcodetable}.
% \cs{PushCatcodeTableNumStack} pushes the current catcode table
% on the stack.
% \cs{PopCatcodeTableNumStack} pops the topmost number off the number
% stack to set the current catcode table. Catcode table zero is
% used in case of an empty stack.
%
% \begin{declcs}{BeginCatcodeRegime} \M{catcodetable}\\
%   \cs{EndCatcodeRegime}
% \end{declcs}
% \cs{BeginCatcodeRegime} remembers the current catcode table number.
% Then it creates and uses a fresh catcode table on the stack
% that is initialized by \meta{catcodetable}:
% \begin{quote}\ttfamily
%   \cs{PushCatcodeTableNumStack}\\
%   \cs{catcodetable}\meta{catcodetable}||
%   \cs{IncCatcodeTableStack}\\
%   \cs{savecatcodetable}\cs{CatcodeTableStack}\\
%   \cs{catcodetable}\cs{CatcodeTableStack}
% \end{quote}
% \cs{EndCatcodeRegime} drops the catcode table, created by
% \cs{BeginCatcodeRegime} and sets the catcode table that was
% active before:
% \begin{quote}
%   \cs{DecCatcodeTableStack}\\
%   \cs{PopCatcodeTableNumStack}
% \end{quote}
% These macros solve the use case, described earlier for a \LaTeX\ package:
% \begin{quote}
%   |% package foobar.sty|\\
%   |\BeginCatcodeRegime\CatcodeTableLaTeX|\\
%   |\makeatletter|\\
%   |% ... package contents ...|\\
%   |\EndCatcodeRegime|\\
%   |% end of package|
% \end{quote}
% If the package wants to change catcodes after its loading,
% \cs{AtBeginDocument} or \cs{AtEndOfPackage} can be used.
%
% \begin{declcs}{SetCatcodeRange} \M{from} \M{to} \M{catcode}
% \end{declcs}
% The catcodes of characters in range from \meta{from}
% to inclusive \meta{to} are set to \meta{catcode}.
%
% \subsection{Lua module loading}
%
% Currently \LuaTeX\ (version 0.20) does not support Lua script
% files inside \xfile{TDS:scripts//}, because Lua's mechanism
% for module loading does not use the \textsf{kpathsea} library.
% Therefore this packages appends a kpse loader to the list of
% Lua's module loaders. It finds the module \meta{module} by
% \begin{quote}
%   \ttfamily |kpse.find_file("|\meta{module}|.lua", "texmfscripts")|
% \end{quote}
% Unhappily \textsf{kpathsea} does not support directory components
% in a file name. Therefore the Lua convention is not followed to
% replace dots in the module name by the directory separator.
%
% Example:
% A Lua script of a package \xpackage{foobar} wants the following
% modules:
% \begin{quote}
%   |require("foobar.hello.world")|\\
%   |require("org.somewhere.xyz")|
% \end{quote}
% Then they can be find in:
% \begin{quote}
%   |TDS:scripts/foobar/foobar.hello.world.lua|\\
%   |TDS:scripts/foobar/org.somewhere.xyz.lua|
% \end{quote}
% I would have preferred the following locations, following lua conventions,
% e.\,g.:
% \begin{quote}
%   |TDS:scripts/foobar/hello/world.lua|\\
%   |TDS:scripts/foobar/org/somewhere/xyz.lua|
% \end{quote}
% But I do not know, how to achieve this in a reliable way using
% \textsf{kpathsea}.
%
% \subsubsection{Package \xpackage{luatex-loader}}
%
% If someone do not need or want package \xpackage{luatex} but
% it's extension for module loading, then he can use
% package \xpackage{luatex-loader}. Both \plainTeX\ and \LaTeX\ are
% supported.
%
% \StopEventually{
% }
%
% \section{Implementation}
%
%    \begin{macrocode}
%<*package>
%    \end{macrocode}
%
% \subsection{Reload check and package identification}
%    Reload check, especially if the package is not used with \LaTeX.
%    \begin{macrocode}
\begingroup\catcode61\catcode48\catcode32=10\relax%
  \catcode13=5 % ^^M
  \endlinechar=13 %
  \catcode35=6 % #
  \catcode39=12 % '
  \catcode44=12 % ,
  \catcode45=12 % -
  \catcode46=12 % .
  \catcode58=12 % :
  \catcode64=11 % @
  \catcode123=1 % {
  \catcode125=2 % }
  \expandafter\let\expandafter\x\csname ver@luatex.sty\endcsname
  \ifx\x\relax % plain-TeX, first loading
  \else
    \def\empty{}%
    \ifx\x\empty % LaTeX, first loading,
      % variable is initialized, but \ProvidesPackage not yet seen
    \else
      \expandafter\ifx\csname PackageInfo\endcsname\relax
        \def\x#1#2{%
          \immediate\write-1{Package #1 Info: #2.}%
        }%
      \else
        \def\x#1#2{\PackageInfo{#1}{#2, stopped}}%
      \fi
      \x{luatex}{The package is already loaded}%
      \aftergroup\endinput
    \fi
  \fi
\endgroup%
%    \end{macrocode}
%    Package identification:
%    \begin{macrocode}
\begingroup\catcode61\catcode48\catcode32=10\relax%
  \catcode13=5 % ^^M
  \endlinechar=13 %
  \catcode35=6 % #
  \catcode39=12 % '
  \catcode40=12 % (
  \catcode41=12 % )
  \catcode44=12 % ,
  \catcode45=12 % -
  \catcode46=12 % .
  \catcode47=12 % /
  \catcode58=12 % :
  \catcode64=11 % @
  \catcode91=12 % [
  \catcode93=12 % ]
  \catcode123=1 % {
  \catcode125=2 % }
  \expandafter\ifx\csname ProvidesPackage\endcsname\relax
    \def\x#1#2#3[#4]{\endgroup
      \immediate\write-1{Package: #3 #4}%
      \xdef#1{#4}%
    }%
  \else
    \def\x#1#2[#3]{\endgroup
      #2[{#3}]%
      \ifx#1\@undefined
        \xdef#1{#3}%
      \fi
      \ifx#1\relax
        \xdef#1{#3}%
      \fi
    }%
  \fi
\expandafter\x\csname ver@luatex.sty\endcsname
\ProvidesPackage{luatex}%
  [2010/03/09 v0.4 LuaTeX basic definition package (HO)]%
%    \end{macrocode}
%
% \subsection{Catcodes}
%
%    \begin{macrocode}
\begingroup\catcode61\catcode48\catcode32=10\relax%
  \catcode13=5 % ^^M
  \endlinechar=13 %
  \catcode123=1 % {
  \catcode125=2 % }
  \catcode64=11 % @
  \def\x{\endgroup
    \expandafter\edef\csname LuT@AtEnd\endcsname{%
      \endlinechar=\the\endlinechar\relax
      \catcode13=\the\catcode13\relax
      \catcode32=\the\catcode32\relax
      \catcode35=\the\catcode35\relax
      \catcode61=\the\catcode61\relax
      \catcode64=\the\catcode64\relax
      \catcode123=\the\catcode123\relax
      \catcode125=\the\catcode125\relax
    }%
  }%
\x\catcode61\catcode48\catcode32=10\relax%
\catcode13=5 % ^^M
\endlinechar=13 %
\catcode35=6 % #
\catcode64=11 % @
\catcode123=1 % {
\catcode125=2 % }
\def\TMP@EnsureCode#1#2{%
  \edef\LuT@AtEnd{%
    \LuT@AtEnd
    \catcode#1=\the\catcode#1\relax
  }%
  \catcode#1=#2\relax
}
\TMP@EnsureCode{10}{12}% ^^J
\TMP@EnsureCode{34}{12}% "
\TMP@EnsureCode{36}{3}% $
\TMP@EnsureCode{39}{12}% '
\TMP@EnsureCode{40}{12}% (
\TMP@EnsureCode{41}{12}% )
\TMP@EnsureCode{42}{12}% *
\TMP@EnsureCode{43}{12}% +
\TMP@EnsureCode{44}{12}% ,
\TMP@EnsureCode{45}{12}% -
\TMP@EnsureCode{46}{12}% .
\TMP@EnsureCode{47}{12}% /
\TMP@EnsureCode{60}{12}% <
\TMP@EnsureCode{62}{12}% >
\TMP@EnsureCode{91}{12}% [
\TMP@EnsureCode{93}{12}% ]
\TMP@EnsureCode{95}{12}% _ (other!)
\TMP@EnsureCode{96}{12}% `
\edef\LuT@AtEnd{\LuT@AtEnd\noexpand\endinput}
%    \end{macrocode}
%
% \subsection{Check for \LuaTeX}
%
%    Without \LuaTeX\ there is no point in using this package.
%    \begin{macrocode}
\begingroup\expandafter\expandafter\expandafter\endgroup
\expandafter\ifx\csname RequirePackage\endcsname\relax
  \input infwarerr.sty\relax
  \input ifluatex.sty\relax
\else
  \RequirePackage{infwarerr}[2007/09/09]%
  \RequirePackage{ifluatex}[2009/04/10]%
\fi
%    \end{macrocode}
%    \begin{macrocode}
\ifluatex
\else
  \@PackageError{luatex}{%
    This package may only be run using LuaTeX%
  }\@ehc
  \expandafter\LuT@AtEnd
\fi%
%    \end{macrocode}
%
% \subsection{Provide Lua\TeX\ primitives}
%
%    \begin{macrocode}
\ifnum\luatexversion<36 %
  \def\LuT@MakePrimitive#1{%
    \expandafter\let\csname luatex#1\expandafter\endcsname
    \csname #1\endcsname
  }%
\else
  \def\LuT@MakeLuatexPrimitive#1{%
    \begingroup\expandafter\expandafter\expandafter\endgroup
    \expandafter\ifx\csname luatex#1\endcsname\relax
      \begingroup\expandafter\expandafter\expandafter\endgroup
      \expandafter\ifx\csname #1\endcsname\relax
      \else
        \expandafter\let
        \csname luatex#1\expandafter\endcsname
        \csname #1\endcsname
      \fi
    \fi
    \begingroup\expandafter\expandafter\expandafter\endgroup
    \expandafter\ifx\csname luatex#1\endcsname\relax
      \begingroup
        \expandafter\let\csname luatex#1\endcsname\@undefined
        \ifnum0%
          \directlua{%
            if tex.enableprimitives then %
              tex.enableprimitives('luatex',{'#1'})%
              tex.print('1')%
            end%
          }%
          \expandafter\ifx\csname luatex#1\endcsname\relax\else1\fi
        =11 %
          \global\expandafter\let
          \csname luatex#1\expandafter\endcsname
          \csname luatex#1\endcsname
        \else
          \@PackageError{luatex}{%
            tex.enableprimitives failed for `#1'%
          }\@ehc
        \fi
      \endgroup
    \fi
  }%
  \def\LuT@MakePrimitive#1{%
    \begingroup\expandafter\expandafter\expandafter\endgroup
    \expandafter\ifx\csname#1\endcsname\relax
      \begingroup
        \expandafter\let\csname#1\endcsname\@undefined
        \ifnum0%
          \directlua{%
            if tex.enableprimitives then %
              tex.enableprimitives('',{'#1'})%
              tex.print('1')%
            end%
          }%
          \expandafter\ifx\csname#1\endcsname\relax\else1\fi
        =11 %
          \global\expandafter\let
          \csname#1\expandafter\endcsname
          \csname#1\endcsname
        \else
          \@PackageError{luatex}{%
            tex.enableprimitives failed for `#1'%
          }\@ehc
        \fi
      \endgroup
    \fi
  }%
\fi
\LuT@MakeLuatexPrimitive{attribute}
\LuT@MakeLuatexPrimitive{attributedef}
\LuT@MakeLuatexPrimitive{catcodetable}
\LuT@MakeLuatexPrimitive{initcatcodetable}
\LuT@MakeLuatexPrimitive{luaescapestring}
\LuT@MakeLuatexPrimitive{savecatcodetable}
\LuT@MakePrimitive{numexpr}
%    \end{macrocode}
%
% \subsection{Inherit support for \eTeX}
%
%    Package \xpackage{etex} is not compatible for \plainTeX.
%    But it could be present if a format is used that is based
%    on \xfile{etex.src}. Therefore we only load the package
%    in case of \LaTeX\ and tests its presence independently
%    of the format by looking for \cs{et@xins}.
%    \begin{macrocode}
\begingroup\expandafter\expandafter\expandafter\endgroup
\expandafter\ifx\csname RequirePackage\endcsname\relax
\else
  \RequirePackage{etex}[1998/03/26]%
\fi
%    \end{macrocode}
%
% \subsection{Adaption of \eTeX's register allocation}
%
%    \eTeX has increased the number of \TeX\ registers
%    from $2^8$ (256) to $2^{15}$ (32768) for a register class.
%    \LuaTeX\ extends the limit further to $2^16$ (65536).
%    The allocation scheme of package \xpackage{etex} is not
%    changed. But this can be subject for discussion.
%
%    If a register class hasn't registered any local registers
%    yet, then the limit can safely be pushed to 65536.
%    \begin{macrocode}
\begingroup\expandafter\expandafter\expandafter\endgroup
\expandafter\ifx\csname et@xins\endcsname\relax
  \@PackageWarningNoLine{luatex}{%
    Support for eTeX is not loaded (etex.src)%
  }%
\else
  \def\LuT@temp#1{%
    \ifnum\count27#1=32768 %
      \count27#1=65536 %
    \fi
  }%
  \LuT@temp0%
  \LuT@temp1%
  \LuT@temp2%
  \LuT@temp3%
  \LuT@temp4%
  \LuT@temp5%
  \LuT@temp6%
%    \end{macrocode}
%
% \eTeX\ uses an array for the first 256 registers and then a tree
% structure. \LuaTeX\ stores all registers of a class in one Lua table.
% There shouldn't be large performance differences.
% This allows starting immediately in the extended
% area, leaving room for insertions.
%    \begin{macrocode}
  \let\newcount\globcount
  \let\newdimen\globdimen
  \let\newskip\globskip
  \let\newbox\globbox
\fi
%    \end{macrocode}
%
% \subsection{\plainTeX\ compatibility}
%
%    \begin{macro}{\@empty}
%    \begin{macrocode}
\expandafter\ifx\csname @empty\endcsname\relax
  \def\@empty{}%
\fi
%    \end{macrocode}
%    \end{macro}
%    \begin{macro}{\@gobble}
%    \begin{macrocode}
\expandafter\ifx\csname @gobble\endcsname\relax
  \long\def\@gobble#1{}%
\fi
%    \end{macrocode}
%    \end{macro}
%    \begin{macro}{\@firstofone}
%    \begin{macrocode}
\expandafter\ifx\csname @firstofone\endcsname\relax
  \long\def\@firstofone#1{#1}%
\fi
%    \end{macrocode}
%    \end{macro}
%    \begin{macro}{\@firstoftwo}
%    \begin{macrocode}
\expandafter\ifx\csname @firstoftwo\endcsname\relax
  \long\def\@firstoftwo#1#2{#1}%
\fi
%    \end{macrocode}
%    \end{macro}
%    \begin{macro}{\@car}
%    \begin{macrocode}
\expandafter\ifx\csname @car\endcsname\relax
  \def\@car#1#2\@nil{#1}%
\fi
%    \end{macrocode}
%    \end{macro}
%    \begin{macro}{\@cdr}
%    \begin{macrocode}
\expandafter\ifx\csname @cdr\endcsname\relax
  \def\@cdr#1#2\@nil{#2}%
\fi
%    \end{macrocode}
%    \end{macro}
%    \begin{macro}{\@ifstar}
%    \begin{macrocode}
\expandafter\ifx\csname @ifstar\endcsname\relax
  \def\@ifstar#1{%
    \@ifnextchar*{\@firstoftwo{#1}}%
  }%
%    \end{macrocode}
%    \end{macro}
%    \begin{macro}{\@ifnextchar}
%    \begin{macrocode}
  \long\def\@ifnextchar#1#2#3{%
    \let\reserved@d=#1%
    \def\reserved@a{#2}%
    \def\reserved@b{#3}%
    \futurelet\@let@token\@ifnch
  }%
%    \end{macrocode}
%    \end{macro}
%    \begin{macro}{\@ifnch}
%    \begin{macrocode}
  \def\@ifnch{%
    \ifx\@let@token\@sptoken
      \let\reserved@c\@xifnch
    \else
      \ifx\@let@token\reserved@d
        \let\reserved@c\reserved@a
      \else
        \let\reserved@c\reserved@b
      \fi
    \fi
    \reserved@c
  }%
%    \end{macrocode}
%    \end{macro}
%    \begin{macro}{\@sptoken}
%    \begin{macrocode}
  \let\LuT@temp\:%
  \def\:{\let\@sptoken= }%
  \: % explicit space
%    \end{macrocode}
%    \end{macro}
%    \begin{macro}{\@xifnch}
%    \begin{macrocode}
  \def\:{\@xifnch}%
  \expandafter\def\: {%
    \futurelet\@let@token\@ifnch
  }%
  \let\:\LuT@temp
\fi
%    \end{macrocode}
%    \end{macro}
%
%    \begin{macro}{\@tempcnta}
%    \begin{macrocode}
\expandafter\ifx\csname @tempcnta\endcsname\relax
  \csname newcount\endcsname\@tempcnta
\fi
%    \end{macrocode}
%    \end{macro}
%
%    \begin{macro}{\@tempcntb}
%    \begin{macrocode}
\expandafter\ifx\csname @tempcntb\endcsname\relax
  \csname newcount\endcsname\@tempcntb
\fi
%    \end{macrocode}
%    \end{macro}
%
%    \begin{macro}{\LuT@newcommand}
%    \begin{macrocode}
\begingroup\expandafter\expandafter\expandafter\endgroup
\expandafter\ifx\csname newcommand\endcsname\relax
  \def\LuT@newcommand#1[#2]#3{%
    \ifx#1\@undefined
      \let#1\relax
    \else
      \ifx#1\relax
      \else
        \@PackageError{luatex}{%
          \string#1 is already defined.\MessageBreak
          Redefinition is skipped%
        }\@ehc
      \fi
    \fi
    \ifx#1\relax
      \ifcase#2 %
        \def#1{#3}%
      \or
        \def#1##1{#3}%
      \or
        \def#1##1##2{#3}%
      \or
        \def#1##1##2##3{#3}%
      \or
        \@INTERNAL@ERROR
      \fi
    \fi
  }%
\else
  \def\LuT@newcommand{\newcommand*}%
\fi
%    \end{macrocode}
%    \end{macro}
%
% \subsection{Attributes}
%
% \subsubsection{Allocation}
%
%    \begin{macro}{\LuT@AllocAttribute}
%    \begin{macrocode}
\newcount\LuT@AllocAttribute
\LuT@AllocAttribute=\m@ne
%    \end{macrocode}
%    \end{macro}
%    \begin{macro}{\newattribute}
%    \begin{macrocode}
\LuT@newcommand\newattribute[1]{%
  \ifnum\LuT@AllocAttribute<65535 %
    \global\advance\LuT@AllocAttribute\@ne
    \allocationnumber\LuT@AllocAttribute
    \global\luatexattributedef#1=\allocationnumber
    \unsetattribute{#1}%
    \wlog{\string#1=\string\attribute\the\allocationnumber}%
  \else
    \errmessage{No room for a new \string\attribute}%
  \fi
}
%    \end{macrocode}
%    \end{macro}
%
% \subsubsection{Interface}
%
%    \begin{macro}{\setattribute}
%    \begin{macrocode}
\LuT@newcommand\setattribute[2]{%
  #1=\numexpr#2\relax
}
%    \end{macrocode}
%    \end{macro}
%    \begin{macro}{\unsetattribute}
%    \begin{macrocode}
\ifnum\luatexversion<37
  \LuT@newcommand\LuT@UnsetAttributeValue[0]{}%
  \let\LuT@UnsetAttributeValue\m@ne
\else
  \LuT@newcommand\LuT@UnsetAttributeValue[0]{-2147483647 }%
\fi
\LuT@newcommand\unsetattribute[1]{%
  #1=\LuT@UnsetAttributeValue
}
%    \end{macrocode}
%    \end{macro}
%
% \subsection{Catcode tables}
%
% \subsubsection{Allocation}
%
%    \begin{macro}{\LuT@AllocCatcodeTable}
%    \begin{macrocode}
\newcount\LuT@AllocCatcodeTable
\LuT@AllocCatcodeTable=\m@ne
\newcount\CatcodeTableStack
\CatcodeTableStack=\z@
%    \end{macrocode}
%    \end{macro}
%    \begin{macro}{\newcatcodetable}
%    \begin{macrocode}
\LuT@newcommand\newcatcodetable[1]{%
  \ifnum\LuT@AllocCatcodeTable<1114110 % 0x10FFFF is maximal \chardef
    % or < 268435455 % 2^28 - 1
    \global\advance\LuT@AllocCatcodeTable by\tw@
    \allocationnumber=\LuT@AllocCatcodeTable
    \global\chardef#1=\allocationnumber
    \wlog{%
      \string#1=\string\catcodetable\the\allocationnumber
    }%
  \else
    \errmessage{No room for a new \string\catcodetable}%
  \fi
}%
%    \end{macrocode}
%    \end{macro}
%
%    \begin{macro}{\IncCatcodeTableStack}
%    \begin{macrocode}
\LuT@newcommand\IncCatcodeTableStack[0]{%
  \ifnum\CatcodeTableStack<268435454 %
    \global\advance\CatcodeTableStack by\tw@
  \else
    \@PackageError{luatex}{%
      Catcode table stack overflow%
    }\@ehd
  \fi
}
%    \end{macrocode}
%    \end{macro}
%
%    \begin{macro}{\DecCatcodeTableStack}
%    \begin{macrocode}
\LuT@newcommand\DecCatcodeTableStack[0]{%
  \ifnum\CatcodeTableStack>\z@
    \global\advance\CatcodeTableStack by-2 %
  \else
    \@PackageError{luatex}{%
      Catcode table stack is empty%
    }\@ehd
  \fi
}
%    \end{macrocode}
%    \end{macro}
%
% \subsubsection{\cs{SetCatcodeRange}}
%
%    \begin{macro}{\SetCatcodeRange}
%    \begin{macrocode}
\LuT@newcommand\SetCatcodeRange[3]{%
  \edef\LuT@temp{%
    \noexpand\@tempcnta=\the\@tempcnta
    \noexpand\@tempcntb=\the\@tempcntb
    \noexpand\count@=\the\count@
    \relax
  }%
  \@tempcnta=\numexpr#1\relax
  \@tempcntb=\numexpr#2\relax
  \count@=\numexpr#3\relax
  \loop
    \unless\ifnum\@tempcnta>\@tempcntb
    \catcode\@tempcnta=\count@
    \advance\@tempcnta by \@ne
  \repeat
  \LuT@temp
}
%    \end{macrocode}
%    \end{macro}
%
% \subsubsection{Predefined catcode tables}
%
%    \begin{macrocode}
\newcatcodetable\CatcodeTableIniTeX
\newcatcodetable\CatcodeTableString
\newcatcodetable\CatcodeTableOther
\newcatcodetable\CatcodeTableLaTeX
%    \end{macrocode}
%
%    \begin{macrocode}
\luatexinitcatcodetable\CatcodeTableIniTeX
\begingroup
  \def\@makeother#1{\catcode#1=12\relax}%
  \@firstofone{%
    \luatexcatcodetable\CatcodeTableIniTeX
    \begingroup
      \SetCatcodeRange{0}{8}{15}%
      \catcode9=10 % tab
      \catcode11=15 %
      \catcode12=13 % form feed
      \SetCatcodeRange{14}{31}{15}%
      \catcode35=6 % hash
      \catcode36=3 % dollar
      \catcode38=4 % ampersand
      \catcode94=7 % circumflex
      \catcode95=8 % underscore
      \catcode123=1 % brace left
      \catcode125=2 % brace right
      \catcode126=13 % tilde
      \catcode127=15 %
      \luatexsavecatcodetable\CatcodeTableLaTeX
    \endgroup
    \@makeother{0}% nul
    \@makeother{13}% carriage return
    \@makeother{37}% percent
    \@makeother{92}% backslash
    \@makeother{127}%
    \SetCatcodeRange{65}{90}{12}% A-Z
    \SetCatcodeRange{97}{122}{12}% a-z
    \luatexsavecatcodetable\CatcodeTableString
    \@makeother{32}% space
    \luatexsavecatcodetable\CatcodeTableOther
  \endgroup
}%
%    \end{macrocode}
%
% \subsubsection{Number stack}
%
%    \begin{macro}{\LuT@NumStackEmpty}
%    A special empty stack value because of \cs{@cdr}'s brace removal.
%    \begin{macrocode}
\def\LuT@NumStackEmpty{0}
%    \end{macrocode}
%    \end{macro}
%    \begin{macro}{\LuT@NumStack}
%    \begin{macrocode}
\let\LuT@NumStack\LuT@NumStackEmpty
%    \end{macrocode}
%    \end{macro}
%    \begin{macro}{\PushCatcodeTableNumStack}
%    \begin{macrocode}
\LuT@newcommand\PushCatcodeTableNumStack[0]{%
  \xdef\LuT@NumStack{%
    {\the\luatexcatcodetable}\LuT@NumStack
  }%
}
%    \end{macrocode}
%    \end{macro}
%    \begin{macro}{\PopCatcodeTableNumStack}
%    \begin{macrocode}
\LuT@newcommand\PopCatcodeTableNumStack[0]{%
  \ifx\LuT@NumStack\LuT@NumStackEmpty
    \@PackageWarning{luatex}{Empty catcode table number stack}%
    \luatexcatcodetable\z@
  \else
    \luatexcatcodetable=\expandafter\@car\LuT@NumStack\@nil\relax
    \xdef\LuT@NumStack{%
      \expandafter\@cdr\LuT@NumStack\@nil
    }%
  \fi
}
%    \end{macrocode}
%    \end{macro}
%
% \subsubsection{Catcode regime macros}
%
%    \begin{macro}{\BeginCatcodeRegime}
%    \begin{macrocode}
\LuT@newcommand\BeginCatcodeRegime[1]{%
  \PushCatcodeTableNumStack
  \luatexcatcodetable=\numexpr#1\relax
  \IncCatcodeTableStack
  \luatexsavecatcodetable\CatcodeTableStack
  \luatexcatcodetable\CatcodeTableStack
}
%    \end{macrocode}
%    \end{macro}
%
%    \begin{macro}{\EndCatcodeRegime}
%    \begin{macrocode}
\LuT@newcommand\EndCatcodeRegime[0]{%
  \DecCatcodeTableStack
  \PopCatcodeTableNumStack
}
%    \end{macrocode}
%    \end{macro}
%
% \subsection{Lua module loader}
%
%    \begin{macrocode}
\begingroup\expandafter\expandafter\expandafter\endgroup
\expandafter\ifx\csname RequirePackage\endcsname\relax
  \input luatex-loader.sty\relax
\else
  \RequirePackage{luatex-loader}[2010/03/09]%
\fi
%    \end{macrocode}
%    \begin{macrocode}
\LuT@AtEnd%
%</package>
%    \end{macrocode}
%
%    \begin{macrocode}
%<*loader>
%    \end{macrocode}
%
%    Reload check, especially if the package is not used with \LaTeX.
%    \begin{macrocode}
\begingroup\catcode61\catcode48\catcode32=10\relax%
  \catcode13=5 % ^^M
  \endlinechar=13 %
  \catcode35=6 % #
  \catcode39=12 % '
  \catcode44=12 % ,
  \catcode45=12 % -
  \catcode46=12 % .
  \catcode58=12 % :
  \catcode64=11 % @
  \catcode123=1 % {
  \catcode125=2 % }
  \expandafter\let\expandafter\x\csname ver@luatex-loader.sty\endcsname
  \ifx\x\relax % plain-TeX, first loading
  \else
    \def\empty{}%
    \ifx\x\empty % LaTeX, first loading,
      % variable is initialized, but \ProvidesPackage not yet seen
    \else
      \expandafter\ifx\csname PackageInfo\endcsname\relax
        \def\x#1#2{%
          \immediate\write-1{Package #1 Info: #2.}%
        }%
      \else
        \def\x#1#2{\PackageInfo{#1}{#2, stopped}}%
      \fi
      \x{luatex-loader}{The package is already loaded}%
      \aftergroup\endinput
    \fi
  \fi
\endgroup%
%    \end{macrocode}
%    Package identification:
%    \begin{macrocode}
\begingroup\catcode61\catcode48\catcode32=10\relax%
  \catcode13=5 % ^^M
  \endlinechar=13 %
  \catcode35=6 % #
  \catcode39=12 % '
  \catcode40=12 % (
  \catcode41=12 % )
  \catcode44=12 % ,
  \catcode45=12 % -
  \catcode46=12 % .
  \catcode47=12 % /
  \catcode58=12 % :
  \catcode64=11 % @
  \catcode91=12 % [
  \catcode93=12 % ]
  \catcode123=1 % {
  \catcode125=2 % }
  \expandafter\ifx\csname ProvidesPackage\endcsname\relax
    \def\x#1#2#3[#4]{\endgroup
      \immediate\write-1{Package: #3 #4}%
      \xdef#1{#4}%
    }%
  \else
    \def\x#1#2[#3]{\endgroup
      #2[{#3}]%
      \ifx#1\@undefined
        \xdef#1{#3}%
      \fi
      \ifx#1\relax
        \xdef#1{#3}%
      \fi
    }%
  \fi
\expandafter\x\csname ver@luatex-loader.sty\endcsname
\ProvidesPackage{luatex-loader}%
  [2010/03/09 v0.4 Lua module loader (HO)]%
%    \end{macrocode}
%    \begin{macrocode}
\begingroup\catcode61\catcode48\catcode32=10\relax%
  \catcode13=5\endlinechar=13\relax%
  \catcode10=12 % ^^J
  \catcode34=12 % "
  \catcode39=12 % '
  \catcode40=12 % (
  \catcode41=12 % )
  \catcode44=12 % ,
  \catcode46=12 % .
  \catcode60=12 % <
  \catcode61=12 % =
  \catcode95=12 % _ (other!)
  \catcode96=12 % `
  \catcode123=1 % {
  \catcode125=2 % }
  \endlinechar=10 %
  \ifnum\luatexversion<36 %
    \directlua0%
  \else %
    \expandafter\directlua %
  \fi %
  {%
     do
      local script = "oberdiek.luatex.lua"
      local file = kpse.find_file(script, "texmfscripts")
      if file then
        texio.write_nl("(" .. file .. ")")
        dofile(file)
      else
        error("File `" .. script .. "' not found")
      end
    end
  }%
\endgroup\endinput%
%    \end{macrocode}
%
%    \begin{macrocode}
%</loader>
%    \end{macrocode}
%
% \subsection{Lua script}
%
%    Currently \LuaTeX\ does not use KPSE when searching for module
%    files. The following Lua script implements a workaround.
%    It extends |package.loader| by another search method.
%    Modules are found by the module name with extension \xext{lua}
%    similar to
%    \begin{quote}
%      \ttfamily |kpsewhich --format=texmfscripts| \meta{module}|.lua|
%    \end{quote}
%    Unhappily \textsf{kpsewhich} does not support directory components
%    in the file name. Therefore a module |a.b.c| cannot be installed
%    as \xfile{a/b/c.lua}. The script must be named \xfile{a.b.c.lua}.
%    \begin{macrocode}
%<*lua>
%    \end{macrocode}
%    \begin{macrocode}
module("oberdiek.luatex", package.seeall)
%    \end{macrocode}
%    \begin{macrocode}
function kpse_module_loader(module)
  local script = module .. ".lua"
  local file = kpse.find_file(script, "texmfscripts")
  if file then
    local loader, error = loadfile(file)
    if loader then
      texio.write_nl("(" .. file .. ")")
      return loader
    end
    return "\n\t[oberdiek.luatex.kpse_module_loader] Loading error:\n\t"
           .. error
  end
  return "\n\t[oberdiek.luatex.kpse_module_loader] Search failed"
end
table.insert(package.loaders, kpse_module_loader)
%    \end{macrocode}
%    \begin{macrocode}
%</lua>
%    \end{macrocode}
%
% \section{Test}
%
%    \begin{macrocode}
%<*test2>
\documentclass{article}
\def\LoadCommand{%
  \RequirePackage{luatex}[2010/03/09]%
}
%</test2>
%    \end{macrocode}
%    \begin{macrocode}
%<*test3>
\documentclass{article}
\def\LoadCommand{%
  \RequirePackage{luatex-loader}[2010/03/09]%
}
%</test3>
%    \end{macrocode}
% \subsection{Catcode checks for loading}
%
%    \begin{macrocode}
%<*test1>
%    \end{macrocode}
%    \begin{macrocode}
\catcode`\{=1 %
\catcode`\}=2 %
\catcode`\#=6 %
\catcode`\@=11 %
\expandafter\ifx\csname count@\endcsname\relax
  \countdef\count@=255 %
\fi
\expandafter\ifx\csname @gobble\endcsname\relax
  \long\def\@gobble#1{}%
\fi
\expandafter\ifx\csname @firstofone\endcsname\relax
  \long\def\@firstofone#1{#1}%
\fi
\expandafter\ifx\csname loop\endcsname\relax
  \expandafter\@firstofone
\else
  \expandafter\@gobble
\fi
{%
  \def\loop#1\repeat{%
    \def\body{#1}%
    \iterate
  }%
  \def\iterate{%
    \body
      \let\next\iterate
    \else
      \let\next\relax
    \fi
    \next
  }%
  \let\repeat=\fi
}%
\def\RestoreCatcodes{}
\count@=0 %
\loop
  \edef\RestoreCatcodes{%
    \RestoreCatcodes
    \catcode\the\count@=\the\catcode\count@\relax
  }%
\ifnum\count@<255 %
  \advance\count@ 1 %
\repeat

\def\RangeCatcodeInvalid#1#2{%
  \count@=#1\relax
  \loop
    \catcode\count@=15 %
  \ifnum\count@<#2\relax
    \advance\count@ 1 %
  \repeat
}
\def\RangeCatcodeCheck#1#2#3{%
  \count@=#1\relax
  \loop
    \ifnum#3=\catcode\count@
    \else
      \errmessage{%
        Character \the\count@\space
        with wrong catcode \the\catcode\count@\space
        instead of \number#3%
      }%
    \fi
  \ifnum\count@<#2\relax
    \advance\count@ 1 %
  \repeat
}
\def\space{ }
\expandafter\ifx\csname LoadCommand\endcsname\relax
  \def\LoadCommand{\input luatex.sty\relax}%
\fi
\def\Test{%
  \RangeCatcodeInvalid{0}{47}%
  \RangeCatcodeInvalid{58}{64}%
  \RangeCatcodeInvalid{91}{96}%
  \RangeCatcodeInvalid{123}{255}%
  \catcode`\@=12 %
  \catcode`\\=0 %
  \catcode`\%=14 %
  \LoadCommand
  \RangeCatcodeCheck{0}{36}{15}%
  \RangeCatcodeCheck{37}{37}{14}%
  \RangeCatcodeCheck{38}{47}{15}%
  \RangeCatcodeCheck{48}{57}{12}%
  \RangeCatcodeCheck{58}{63}{15}%
  \RangeCatcodeCheck{64}{64}{12}%
  \RangeCatcodeCheck{65}{90}{11}%
  \RangeCatcodeCheck{91}{91}{15}%
  \RangeCatcodeCheck{92}{92}{0}%
  \RangeCatcodeCheck{93}{96}{15}%
  \RangeCatcodeCheck{97}{122}{11}%
  \RangeCatcodeCheck{123}{255}{15}%
  \RestoreCatcodes
}
\Test
\csname @@end\endcsname
\end
%    \end{macrocode}
%    \begin{macrocode}
%</test1>
%    \end{macrocode}
%
% \subsection{Catcode tables}
%
% \subsubsection{Predefined catcode tables}
%
%    \begin{macrocode}
%<*test4>
\NeedsTeXFormat{LaTeX2e}
%    \end{macrocode}
%    Remember \LaTeX's initial catcodes in count registers
%    starting at \cs{TestLaTeX}.
%    \begin{macrocode}
\count0=0 %
\chardef\TestLaTeX=1000 %
\chardef\TestMax=300 %
\loop
  \count\numexpr\TestLaTeX+\count0\relax=\catcode\count0 %
\ifnum\count0<\TestMax
  \advance\count0 by 1 %
\repeat
\documentclass{minimal}
\usepackage{luatex}[2010/03/09]
\usepackage{qstest}
\IncludeTests{*}
\LogTests{log}{*}{*}
\makeatletter
\def\Check#1{%
  \Expect*{\the\count@=\the\catcode\count@}%
         *{\the\count@=#1}%
}
\newcount\scratch
\def\Test#1#2{%
  \begin{qstest}{CatcodeTable#1}{CatcodeTable#1}%
    \luatexcatcodetable\csname CatcodeTable#1\endcsname
    \count@=\z@
    \loop
      \scratch=#2\relax
      \Expect*{\the\count@=\the\catcode\count@}%
             *{\the\count@=\the\scratch}%
    \ifnum\count@<\TestMax
      \advance\count@\@ne
    \repeat
  \end{qstest}%
}
\begingroup
  % luatex-unicode-letters.tex makes some slots to letters
  \def\TestMax{169}%
  \Test{LaTeX}{\the\count\numexpr\TestLaTeX+\count@}%
\endgroup
\Test{String}{\ifnum\count@=32 10\else 12\fi}
\Test{Other}{12}
\luatexinitcatcodetable99 %
\Test{IniTeX}{%
  0\relax
  \begingroup
    \luatexcatcodetable99 %
    \global\scratch=\the\catcode\count@
  \endgroup
}
%    \end{macrocode}
%
% \subsubsection{Catcode table number stack}
%
%    \begin{macrocode}
\begin{qstest}{CatcodeTableNumStack}{CatcodeTableNumStack}
  \def\TestStack#1{%
    \Expect*{\LuT@NumStack}{#1}%
  }%
  \TestStack{0}%
  \PushCatcodeTableNumStack
  \TestStack{{0}0}%
  \@firstofone{%
    \begingroup
      \luatexinitcatcodetable12 %
      \luatexcatcodetable12 %
      \PushCatcodeTableNumStack
      \TestStack{{12}{0}0}%
      \PopCatcodeTableNumStack
      \TestStack{{0}0}%
      \PopCatcodeTableNumStack
      \TestStack{0}%
      \def\TestWarning{Missing empty stack warning}%
      \def\@PackageWarning#1#2{\def\TestWarning{empty stack}}%
      \PopCatcodeTableNumStack
      \TestStack{0}%
      \Expect*{\TestWarning}{empty stack}%
    \endgroup
  }%
\end{qstest}
%    \end{macrocode}
%
% \subsubsection{Catcode table stack}
%
%    \begin{macrocode}
\begin{qstest}{CatcodeTableStack}{CatcodeTableStack}
  \def\TestStack#1{%
    \Expect*{\the\CatcodeTableStack}{#1}%
  }%
  \TestStack{0}%
  \IncCatcodeTableStack
  \TestStack{2}%
  \IncCatcodeTableStack
  \TestStack{4}%
  \begingroup
    \IncCatcodeTableStack
    \TestStack{6}%
  \endgroup
  \TestStack{6}%
  \begingroup
    \DecCatcodeTableStack
    \TestStack{4}%
  \endgroup
  \TestStack{4}%
  \DecCatcodeTableStack
  \TestStack{2}%
  \DecCatcodeTableStack
  \TestStack{0}%
  \begingroup
    \def\TestError{Missing error}%
    \def\@PackageError#1#2#3{%
      \def\TestError{Empty stack}%
    }%
    \DecCatcodeTableStack
    \TestStack{0}%
    \Expect*{\TestError}{Empty stack}%
  \endgroup
\end{qstest}
%    \end{macrocode}
%
% \subsubsection{Catcode regime macros}
%
%    \begin{macrocode}
\begin{qstest}{CatcodeRegime}{CatcodeRegime}
  \def\TestStacks#1#2#3{%
    \Expect*{\the\luatexcatcodetable}{#1}%
    \Expect*{\the\CatcodeTableStack}{#2}%
    \Expect*{\LuT@NumStack}{#3}%
  }%
  \TestStacks{0}{0}{0}%
  \catcode`\|=7 %
  \BeginCatcodeRegime\CatcodeTableLaTeX
    \TestStacks{2}{2}{{0}0}%
    \Expect*{\the\catcode`\|}{12}%
  \EndCatcodeRegime
  \TestStacks{0}{0}{0}%
  \Expect*{\the\catcode`\|}{7}%
\end{qstest}
%    \end{macrocode}
%
% \subsection{Attribute allocation}
%
%    \begin{macrocode}
\begin{qstest}{Attributes}{Attributes}
  \newattribute\TestAttr
  \Expect*{\meaning\TestAttr}%
         *{\string\attribute\number\allocationnumber}%
  \Expect*{\the\allocationnumber}{0}%
  \begingroup
    \newattribute\TestAttr
    \Expect*{\the\allocationnumber}{1}%
  \endgroup
  \Expect*{\the\allocationnumber}{0}%
  \Expect*{\meaning\TestAttr}*{\string\attribute1}%
  \Expect*{\the\TestAttr}*{\number\LuT@UnsetAttributeValue}%
  \def\Test#1{%
    \setattribute\TestAttr{#1}%
    \Expect*{\the\TestAttr}{#1}%
  }%
  \Test{0}%
  \Test{1}%
  \Test{-1}%
  \Test{123}%
  \unsetattribute\TestAttr
  \Expect*{\the\TestAttr}*{\number\LuT@UnsetAttributeValue}%
  \begingroup
    \Expect*{\the\TestAttr}*{\number\LuT@UnsetAttributeValue}%
    \Test{1234}%
  \endgroup
  \Expect*{\the\TestAttr}*{\number\LuT@UnsetAttributeValue}%
\end{qstest}
%    \end{macrocode}
%
%    \begin{macrocode}
\@@end
%</test4>
%    \end{macrocode}
%
% \subsection{Short test for \plainTeX}
%
%    \begin{macrocode}
%<*test5>
\input luatex.sty\relax
\newattribute\TestAttr
\setattribute\TestAttr{10}
\unsetattribute\TestAttr
\newcatcodetable\TestCTa
\begingroup
  \SetCatcodeRange{`A}{`Z}{12}%
\endgroup
\BeginCatcodeRegime\CatcodeTableLaTeX
\EndCatcodeRegime
\end
%</test5>
%    \end{macrocode}
%
% \section{Installation}
%
% \subsection{Download}
%
% \paragraph{Package.} This package is available on
% CTAN\footnote{\url{ftp://ftp.ctan.org/tex-archive/}}:
% \begin{description}
% \item[\CTAN{macros/latex/contrib/oberdiek/luatex.dtx}] The source file.
% \item[\CTAN{macros/latex/contrib/oberdiek/luatex.pdf}] Documentation.
% \end{description}
%
%
% \paragraph{Bundle.} All the packages of the bundle `oberdiek'
% are also available in a TDS compliant ZIP archive. There
% the packages are already unpacked and the documentation files
% are generated. The files and directories obey the TDS standard.
% \begin{description}
% \item[\CTAN{install/macros/latex/contrib/oberdiek.tds.zip}]
% \end{description}
% \emph{TDS} refers to the standard ``A Directory Structure
% for \TeX\ Files'' (\CTAN{tds/tds.pdf}). Directories
% with \xfile{texmf} in their name are usually organized this way.
%
% \subsection{Bundle installation}
%
% \paragraph{Unpacking.} Unpack the \xfile{oberdiek.tds.zip} in the
% TDS tree (also known as \xfile{texmf} tree) of your choice.
% Example (linux):
% \begin{quote}
%   |unzip oberdiek.tds.zip -d ~/texmf|
% \end{quote}
%
% \paragraph{Script installation.}
% Check the directory \xfile{TDS:scripts/oberdiek/} for
% scripts that need further installation steps.
% Package \xpackage{attachfile2} comes with the Perl script
% \xfile{pdfatfi.pl} that should be installed in such a way
% that it can be called as \texttt{pdfatfi}.
% Example (linux):
% \begin{quote}
%   |chmod +x scripts/oberdiek/pdfatfi.pl|\\
%   |cp scripts/oberdiek/pdfatfi.pl /usr/local/bin/|
% \end{quote}
%
% \subsection{Package installation}
%
% \paragraph{Unpacking.} The \xfile{.dtx} file is a self-extracting
% \docstrip\ archive. The files are extracted by running the
% \xfile{.dtx} through \plainTeX:
% \begin{quote}
%   \verb|tex luatex.dtx|
% \end{quote}
%
% \paragraph{TDS.} Now the different files must be moved into
% the different directories in your installation TDS tree
% (also known as \xfile{texmf} tree):
% \begin{quote}
% \def\t{^^A
% \begin{tabular}{@{}>{\ttfamily}l@{ $\rightarrow$ }>{\ttfamily}l@{}}
%   luatex.sty & tex/generic/oberdiek/luatex.sty\\
%   luatex-loader.sty & tex/generic/oberdiek/luatex-loader.sty\\
%   oberdiek.luatex.lua & scripts/oberdiek/oberdiek.luatex.lua\\
%   luatex.pdf & doc/latex/oberdiek/luatex.pdf\\
%   test/luatex-test1.tex & doc/latex/oberdiek/test/luatex-test1.tex\\
%   test/luatex-test2.tex & doc/latex/oberdiek/test/luatex-test2.tex\\
%   test/luatex-test3.tex & doc/latex/oberdiek/test/luatex-test3.tex\\
%   test/luatex-test4.tex & doc/latex/oberdiek/test/luatex-test4.tex\\
%   test/luatex-test5.tex & doc/latex/oberdiek/test/luatex-test5.tex\\
%   luatex.dtx & source/latex/oberdiek/luatex.dtx\\
% \end{tabular}^^A
% }^^A
% \sbox0{\t}^^A
% \ifdim\wd0>\linewidth
%   \begingroup
%     \advance\linewidth by\leftmargin
%     \advance\linewidth by\rightmargin
%   \edef\x{\endgroup
%     \def\noexpand\lw{\the\linewidth}^^A
%   }\x
%   \def\lwbox{^^A
%     \leavevmode
%     \hbox to \linewidth{^^A
%       \kern-\leftmargin\relax
%       \hss
%       \usebox0
%       \hss
%       \kern-\rightmargin\relax
%     }^^A
%   }^^A
%   \ifdim\wd0>\lw
%     \sbox0{\small\t}^^A
%     \ifdim\wd0>\linewidth
%       \ifdim\wd0>\lw
%         \sbox0{\footnotesize\t}^^A
%         \ifdim\wd0>\linewidth
%           \ifdim\wd0>\lw
%             \sbox0{\scriptsize\t}^^A
%             \ifdim\wd0>\linewidth
%               \ifdim\wd0>\lw
%                 \sbox0{\tiny\t}^^A
%                 \ifdim\wd0>\linewidth
%                   \lwbox
%                 \else
%                   \usebox0
%                 \fi
%               \else
%                 \lwbox
%               \fi
%             \else
%               \usebox0
%             \fi
%           \else
%             \lwbox
%           \fi
%         \else
%           \usebox0
%         \fi
%       \else
%         \lwbox
%       \fi
%     \else
%       \usebox0
%     \fi
%   \else
%     \lwbox
%   \fi
% \else
%   \usebox0
% \fi
% \end{quote}
% If you have a \xfile{docstrip.cfg} that configures and enables \docstrip's
% TDS installing feature, then some files can already be in the right
% place, see the documentation of \docstrip.
%
% \subsection{Refresh file name databases}
%
% If your \TeX~distribution
% (\teTeX, \mikTeX, \dots) relies on file name databases, you must refresh
% these. For example, \teTeX\ users run \verb|texhash| or
% \verb|mktexlsr|.
%
% \subsection{Some details for the interested}
%
% \paragraph{Attached source.}
%
% The PDF documentation on CTAN also includes the
% \xfile{.dtx} source file. It can be extracted by
% AcrobatReader 6 or higher. Another option is \textsf{pdftk},
% e.g. unpack the file into the current directory:
% \begin{quote}
%   \verb|pdftk luatex.pdf unpack_files output .|
% \end{quote}
%
% \paragraph{Unpacking with \LaTeX.}
% The \xfile{.dtx} chooses its action depending on the format:
% \begin{description}
% \item[\plainTeX:] Run \docstrip\ and extract the files.
% \item[\LaTeX:] Generate the documentation.
% \end{description}
% If you insist on using \LaTeX\ for \docstrip\ (really,
% \docstrip\ does not need \LaTeX), then inform the autodetect routine
% about your intention:
% \begin{quote}
%   \verb|latex \let\install=y% \iffalse meta-comment
%
% File: luatex.dtx
% Version: 2010/03/09 v0.4
% Info: LuaTeX basic definition package
%
% Copyright (C) 2007, 2009, 2010 by
%    Heiko Oberdiek <heiko.oberdiek at googlemail.com>
%
% This work may be distributed and/or modified under the
% conditions of the LaTeX Project Public License, either
% version 1.3c of this license or (at your option) any later
% version. This version of this license is in
%    http://www.latex-project.org/lppl/lppl-1-3c.txt
% and the latest version of this license is in
%    http://www.latex-project.org/lppl.txt
% and version 1.3 or later is part of all distributions of
% LaTeX version 2005/12/01 or later.
%
% This work has the LPPL maintenance status "maintained".
%
% This Current Maintainer of this work is Heiko Oberdiek.
%
% The Base Interpreter refers to any `TeX-Format',
% because some files are installed in TDS:tex/generic//.
%
% This work consists of the main source file luatex.dtx
% and the derived files
%    luatex.sty, luatex.pdf, luatex.ins, luatex.drv, luatex-loader.sty,
%    luatex-test1.tex, luatex-test2.tex, luatex-test3.tex,
%    luatex-test4.tex, luatex-test5.tex, oberdiek.luatex.lua.
%
% Distribution:
%    CTAN:macros/latex/contrib/oberdiek/luatex.dtx
%    CTAN:macros/latex/contrib/oberdiek/luatex.pdf
%
% Unpacking:
%    (a) If luatex.ins is present:
%           tex luatex.ins
%    (b) Without luatex.ins:
%           tex luatex.dtx
%    (c) If you insist on using LaTeX
%           latex \let\install=y% \iffalse meta-comment
%
% File: luatex.dtx
% Version: 2010/03/09 v0.4
% Info: LuaTeX basic definition package
%
% Copyright (C) 2007, 2009, 2010 by
%    Heiko Oberdiek <heiko.oberdiek at googlemail.com>
%
% This work may be distributed and/or modified under the
% conditions of the LaTeX Project Public License, either
% version 1.3c of this license or (at your option) any later
% version. This version of this license is in
%    http://www.latex-project.org/lppl/lppl-1-3c.txt
% and the latest version of this license is in
%    http://www.latex-project.org/lppl.txt
% and version 1.3 or later is part of all distributions of
% LaTeX version 2005/12/01 or later.
%
% This work has the LPPL maintenance status "maintained".
%
% This Current Maintainer of this work is Heiko Oberdiek.
%
% The Base Interpreter refers to any `TeX-Format',
% because some files are installed in TDS:tex/generic//.
%
% This work consists of the main source file luatex.dtx
% and the derived files
%    luatex.sty, luatex.pdf, luatex.ins, luatex.drv, luatex-loader.sty,
%    luatex-test1.tex, luatex-test2.tex, luatex-test3.tex,
%    luatex-test4.tex, luatex-test5.tex, oberdiek.luatex.lua.
%
% Distribution:
%    CTAN:macros/latex/contrib/oberdiek/luatex.dtx
%    CTAN:macros/latex/contrib/oberdiek/luatex.pdf
%
% Unpacking:
%    (a) If luatex.ins is present:
%           tex luatex.ins
%    (b) Without luatex.ins:
%           tex luatex.dtx
%    (c) If you insist on using LaTeX
%           latex \let\install=y\input{luatex.dtx}
%        (quote the arguments according to the demands of your shell)
%
% Documentation:
%    (a) If luatex.drv is present:
%           latex luatex.drv
%    (b) Without luatex.drv:
%           latex luatex.dtx; ...
%    The class ltxdoc loads the configuration file ltxdoc.cfg
%    if available. Here you can specify further options, e.g.
%    use A4 as paper format:
%       \PassOptionsToClass{a4paper}{article}
%
%    Programm calls to get the documentation (example):
%       pdflatex luatex.dtx
%       makeindex -s gind.ist luatex.idx
%       pdflatex luatex.dtx
%       makeindex -s gind.ist luatex.idx
%       pdflatex luatex.dtx
%
% Installation:
%    TDS:tex/generic/oberdiek/luatex.sty
%    TDS:tex/generic/oberdiek/luatex-loader.sty
%    TDS:scripts/oberdiek/oberdiek.luatex.lua
%    TDS:doc/latex/oberdiek/luatex.pdf
%    TDS:doc/latex/oberdiek/test/luatex-test1.tex
%    TDS:doc/latex/oberdiek/test/luatex-test2.tex
%    TDS:doc/latex/oberdiek/test/luatex-test3.tex
%    TDS:doc/latex/oberdiek/test/luatex-test4.tex
%    TDS:doc/latex/oberdiek/test/luatex-test5.tex
%    TDS:source/latex/oberdiek/luatex.dtx
%
%<*ignore>
\begingroup
  \catcode123=1 %
  \catcode125=2 %
  \def\x{LaTeX2e}%
\expandafter\endgroup
\ifcase 0\ifx\install y1\fi\expandafter
         \ifx\csname processbatchFile\endcsname\relax\else1\fi
         \ifx\fmtname\x\else 1\fi\relax
\else\csname fi\endcsname
%</ignore>
%<*install>
\input docstrip.tex
\Msg{************************************************************************}
\Msg{* Installation}
\Msg{* Package: luatex 2010/03/09 v0.4 LuaTeX basic definition package (HO)}
\Msg{************************************************************************}

\keepsilent
\askforoverwritefalse

\let\MetaPrefix\relax
\preamble

This is a generated file.

Project: luatex
Version: 2010/03/09 v0.4

Copyright (C) 2007, 2009, 2010 by
   Heiko Oberdiek <heiko.oberdiek at googlemail.com>

This work may be distributed and/or modified under the
conditions of the LaTeX Project Public License, either
version 1.3c of this license or (at your option) any later
version. This version of this license is in
   http://www.latex-project.org/lppl/lppl-1-3c.txt
and the latest version of this license is in
   http://www.latex-project.org/lppl.txt
and version 1.3 or later is part of all distributions of
LaTeX version 2005/12/01 or later.

This work has the LPPL maintenance status "maintained".

This Current Maintainer of this work is Heiko Oberdiek.

The Base Interpreter refers to any `TeX-Format',
because some files are installed in TDS:tex/generic//.

This work consists of the main source file luatex.dtx
and the derived files
   luatex.sty, luatex.pdf, luatex.ins, luatex.drv, luatex-loader.sty,
   luatex-test1.tex, luatex-test2.tex, luatex-test3.tex,
   luatex-test4.tex, luatex-test5.tex, oberdiek.luatex.lua.

\endpreamble
\let\MetaPrefix\DoubleperCent

\generate{%
  \file{luatex.ins}{\from{luatex.dtx}{install}}%
  \file{luatex.drv}{\from{luatex.dtx}{driver}}%
  \usedir{tex/generic/oberdiek}%
  \file{luatex.sty}{\from{luatex.dtx}{package}}%
  \file{luatex-loader.sty}{\from{luatex.dtx}{loader}}%
  \usedir{doc/latex/oberdiek/test}%
  \file{luatex-test1.tex}{\from{luatex.dtx}{test1}}%
  \file{luatex-test2.tex}{\from{luatex.dtx}{test1,test2}}%
  \file{luatex-test3.tex}{\from{luatex.dtx}{test1,test3}}%
  \file{luatex-test4.tex}{\from{luatex.dtx}{test4}}%
  \file{luatex-test5.tex}{\from{luatex.dtx}{test5}}%
  \nopreamble
  \nopostamble
  \usedir{source/latex/oberdiek/catalogue}%
  \file{luatex.xml}{\from{luatex.dtx}{catalogue}}%
}
\def\MetaPrefix{-- }
\def\defaultpostamble{%
  \MetaPrefix^^J%
  \MetaPrefix\space End of File `\outFileName'.%
}
\def\currentpostamble{\defaultpostamble}%
\generate{%
  \usedir{scripts/oberdiek}%
  \file{oberdiek.luatex.lua}{\from{luatex.dtx}{lua}}%
}

\catcode32=13\relax% active space
\let =\space%
\Msg{************************************************************************}
\Msg{*}
\Msg{* To finish the installation you have to move the following}
\Msg{* files into a directory searched by TeX:}
\Msg{*}
\Msg{*     luatex.sty, luatex-loader.sty}
\Msg{*}
\Msg{* And install the following script file:}
\Msg{*}
\Msg{*     oberdiek.luatex.lua}
\Msg{*}
\Msg{* To produce the documentation run the file `luatex.drv'}
\Msg{* through LaTeX.}
\Msg{*}
\Msg{* Happy TeXing!}
\Msg{*}
\Msg{************************************************************************}

\endbatchfile
%</install>
%<*ignore>
\fi
%</ignore>
%<*driver>
\NeedsTeXFormat{LaTeX2e}
\ProvidesFile{luatex.drv}%
  [2010/03/09 v0.4 LuaTeX basic definition package (HO)]%
\documentclass{ltxdoc}
\usepackage{holtxdoc}[2011/11/22]
\usepackage{paralist}
\begin{document}
  \DocInput{luatex.dtx}%
\end{document}
%</driver>
% \fi
%
% \CheckSum{1429}
%
% \CharacterTable
%  {Upper-case    \A\B\C\D\E\F\G\H\I\J\K\L\M\N\O\P\Q\R\S\T\U\V\W\X\Y\Z
%   Lower-case    \a\b\c\d\e\f\g\h\i\j\k\l\m\n\o\p\q\r\s\t\u\v\w\x\y\z
%   Digits        \0\1\2\3\4\5\6\7\8\9
%   Exclamation   \!     Double quote  \"     Hash (number) \#
%   Dollar        \$     Percent       \%     Ampersand     \&
%   Acute accent  \'     Left paren    \(     Right paren   \)
%   Asterisk      \*     Plus          \+     Comma         \,
%   Minus         \-     Point         \.     Solidus       \/
%   Colon         \:     Semicolon     \;     Less than     \<
%   Equals        \=     Greater than  \>     Question mark \?
%   Commercial at \@     Left bracket  \[     Backslash     \\
%   Right bracket \]     Circumflex    \^     Underscore    \_
%   Grave accent  \`     Left brace    \{     Vertical bar  \|
%   Right brace   \}     Tilde         \~}
%
% \GetFileInfo{luatex.drv}
%
% \title{The \xpackage{luatex} package}
% \date{2010/03/09 v0.4}
% \author{Heiko Oberdiek\\\xemail{heiko.oberdiek at googlemail.com}}
%
% \maketitle
%
% \begin{abstract}
% This package manages the new and extended features and resources
% that \LuaTeX\ provides. Examples are attributes and catcode
% tables.
% \end{abstract}
%
% \tableofcontents
%
% \section{Documentation}
%
% \subsection{Introduction}
%
% \TeX\ provides global resources such as registers.
% But it does not provide an interface for managing these resources.
% For example, two packages want to use a counter register.
% If they take the same register
% number, then the use of both packages will conflict and they cannot
% be used together.
% Therefore formats such as \plainTeX\ or \LaTeX\ implement
% an allocation scheme for registers. A package reserves with
% \cs{newcount} an unused register number for its own exclusive use.
%
% Nowadays \TeX\ is not alone anymore: \eTeX, \pdfTeX\ and other
% compilers for \TeX\ are developed that extend and add new
% features and resources.
%
% Now \LuaTeX\ has reached beta state. It inherits most of
% \pdfTeX's features including \eTeX. Also it implements
% new concepts such as attributes or catcode tables.
%
% \subsubsection{\LaTeX}
%
% \LaTeXe\ is frozen and therefore refuses to even notice
% the new \TeX\ variants. Not even the old \eTeX\ is supported
% by its kernel. At least there is a third party package \xpackage{etex}
% that manages the new \eTeX\ resources.
%
% This package tries to do the same for \LuaTeX\ and starts to
% support at least a few of the new features.
%
% \subsubsection{\plainTeX}
%
% \LaTeX\ has inherited its resource handling from \plainTeX.
% The interface is basically the same: \cs{newcount}, \dots
% Therefore this package tries to follow this tradition by
% providing compatibility to \plainTeX. It can be loaded
% with \plainTeX\ and defines at least some of the
% features that this packages provides for \LaTeX.
%
% \subsection{Register allocation}
%
% \subsubsection{Register with 16 bit}
%
% Because \LuaTeX\ is a super set of \eTeX\ regarding registers,
% the register allocation scheme should not conflict with
% package \xpackage{etex}. Therefore this package is loaded
% to inherit its allocation scheme. The only change is currently
% that the limit is increased to 65536 registers for the following
% register classes:
% \begin{compactitem}
% \item count
% \item dimen
% \item skip
% \item muskip
% \item marks
% \item toks
% \item box
% \end{compactitem}
% This affects the number of global and local registers.
% Because it is done in a package and not in the kernel,
% it is possible that someone loads package \xpackage{etex}
% before uses the local allocation variants. This will prevent
% the extension for this register class. If more registers are
% needed, just load package \xpackage{luatex} earlier.
%
% \subsubsection{Insertions}
%
% Insertions need four registers \cs{count}, \cs{dimen}, \cs{skip},
% and \cs{box} with the same number. Usually they are allocated
% downwards from 254, 253, \dots Also \cs{newcount}, \cs{newdimen}, \dots
% fill up these register numbers from below before switching to
% higher register numbers by package \xpackage{etex}.
% When this occurs, no insertions can be allocated anymore.
%
% Therefore \cs{newcount}, \cs{newdimen}, \cs{newskip}, and
% \cs{newbox} are replaced by their global variants (\cs{globcount}, \dots)
% that use the higher numbers immediately, leaving the room for insertions.
% There should not be an efficiency penalty because \LuaTeX\ stores
% the registers of a class in the same Lua table unlike \eTeX, where
% registers below 256 are stored in an array and higher numbers
% are put in a tree structure.
%
% \subsection{Attributes}
%
% Nodes can have custom attributes in \LuaTeX. These attributes
% are organized by a new register class. As the other registers
% up to $2^{16}$ attributes are supported. An attribute value
% can be negative that means the attribute is not set. Otherwise
% \TeX's range of non-negative integers up to $2^31$ are available.
%
% \begin{declcs}{newattribute} \M{cmd}
% \end{declcs}
% Macro \cs{newattribute} defines command \meta{cmd} using \cs{attributedef}
% using an new attribute number. The new attribute is initially unset.
%
% \begin{declcs}{setattribute} \M{cmd} \M{value}
% \end{declcs}
% Macro \cs{setattribute} locally sets attribute command \meta{cmd}
% to the number \meta{value}. Valid values range from $-1$ until $2^31$
% (the upper limit is the same as for other \TeX\ integer numbers).
%
% \begin{declcs}{unsetattribute} \M{cmd}
% \end{declcs}
% Macro \cs{unsettattribute} clears the attribute command \meta{cmd}.
%
% \subsection{Catcode tables}
%
% \LuaTeX\ introduces catcode tables as new feature, see documentation.
% There is need for discussion, how to deal best:
% \begin{itemize}
% \item \cs{initcatcodetable} and \cs{setcatcodetable} act globally.
% \item \cs{catcodetable} causes an error if used with an uninitialized
%   catcode table.
% \item Large catcode table numbers should be avoided because of
%   performance breakdown.
% \item Use case \LaTeX\ package:
%   The package must not be surprised by changed catcodes and
%   must not surprise by changing catcodes accidently.
%   Catcode tables could offer a solution. At the begin
%   a catcode regime with standard catcodes is established
%   and the old one is restored afterwards.
% \item Use case: \LuaTeX's |tex.print| might be used with
%   a catcode table number, for example a table where all
%   entries have catcode ``other''.
% \item Readonly catcode tables.
% \item Is there is a need for local allocations?
%   (Package \xpackage{etex}'s \cs{loc} variants are not used in
%   \TeX\ Live 2007.)
% \end{itemize}
%
% \subsubsection{Interface proposal}
%
% The idea: \cs{newcatcodetable} allocates odd numbered catcode
% tables. Even numbered tables are managed as stack.
% Also some catcode tables are defined. These must not be
% changed.
%
% \begin{declcs}{newcatcodetable} \M{cmd}
% \end{declcs}
% Macro \cs{newcatcodetable} reserves a new catcode table and
% remembers its number in \meta{cmd}. The catcode table is
% initialized with \iniTeX's catcodes.
%
% \begin{declcs}{CatcodeTableIniTeX}\\
%   \cs{CatcodeTableString}\\
%   \cs{CatcodeTableOther}\\
%   \cs{CatcodeTableLaTeX}\\
% \end{declcs}
% These are catcode tables and must not be changed.
% \cs{CatcodeTableIniTeX} contains the catcode settings of \iniTeX.
% \cs{CatcodeTableString} follows \TeX's convention of
% \cs{string}, \cs{meaning} and friends. The space gets catcode
% 10 (space), the other characters have catcode 12 (other).
% In \cs{CatcodeTableOther} all entries have catcode 12 (other).
% \cs{CatcodeTableLaTeX} contains the setting of a pure \LaTeX\ format
% (`at' is other).
%
% \begin{declcs}{CatcodeTableStack}\\
%   \cs{IncCatcodeTableStack}\\
%   \cs{DecCatcodeTableStack}
% \end{declcs}
% \cs{CatcodeTableStack} is the stack pointer. Initially
% it is catcode table zero. \cs{IncCatcodeTableStack} and
% \cs{DecCatcodeTableStack} increments and decrements the
% stack pointer. Currently \cs{IncCatcodeTableStack} does not
% initialize a new catcode table. Both increment and decrement
% operations do not set a catcode table.
%
% \begin{declcs}{PushCatcodeTableNumStack}\\
%   \cs{PopCatcodeTableNumStack}
% \end{declcs}
% It can be handy to have a global stack for catcode table
% numbers to deal with the global assignment property of
% \cs{initcatcodetable} and \cs{savecatcodetable}.
% \cs{PushCatcodeTableNumStack} pushes the current catcode table
% on the stack.
% \cs{PopCatcodeTableNumStack} pops the topmost number off the number
% stack to set the current catcode table. Catcode table zero is
% used in case of an empty stack.
%
% \begin{declcs}{BeginCatcodeRegime} \M{catcodetable}\\
%   \cs{EndCatcodeRegime}
% \end{declcs}
% \cs{BeginCatcodeRegime} remembers the current catcode table number.
% Then it creates and uses a fresh catcode table on the stack
% that is initialized by \meta{catcodetable}:
% \begin{quote}\ttfamily
%   \cs{PushCatcodeTableNumStack}\\
%   \cs{catcodetable}\meta{catcodetable}||
%   \cs{IncCatcodeTableStack}\\
%   \cs{savecatcodetable}\cs{CatcodeTableStack}\\
%   \cs{catcodetable}\cs{CatcodeTableStack}
% \end{quote}
% \cs{EndCatcodeRegime} drops the catcode table, created by
% \cs{BeginCatcodeRegime} and sets the catcode table that was
% active before:
% \begin{quote}
%   \cs{DecCatcodeTableStack}\\
%   \cs{PopCatcodeTableNumStack}
% \end{quote}
% These macros solve the use case, described earlier for a \LaTeX\ package:
% \begin{quote}
%   |% package foobar.sty|\\
%   |\BeginCatcodeRegime\CatcodeTableLaTeX|\\
%   |\makeatletter|\\
%   |% ... package contents ...|\\
%   |\EndCatcodeRegime|\\
%   |% end of package|
% \end{quote}
% If the package wants to change catcodes after its loading,
% \cs{AtBeginDocument} or \cs{AtEndOfPackage} can be used.
%
% \begin{declcs}{SetCatcodeRange} \M{from} \M{to} \M{catcode}
% \end{declcs}
% The catcodes of characters in range from \meta{from}
% to inclusive \meta{to} are set to \meta{catcode}.
%
% \subsection{Lua module loading}
%
% Currently \LuaTeX\ (version 0.20) does not support Lua script
% files inside \xfile{TDS:scripts//}, because Lua's mechanism
% for module loading does not use the \textsf{kpathsea} library.
% Therefore this packages appends a kpse loader to the list of
% Lua's module loaders. It finds the module \meta{module} by
% \begin{quote}
%   \ttfamily |kpse.find_file("|\meta{module}|.lua", "texmfscripts")|
% \end{quote}
% Unhappily \textsf{kpathsea} does not support directory components
% in a file name. Therefore the Lua convention is not followed to
% replace dots in the module name by the directory separator.
%
% Example:
% A Lua script of a package \xpackage{foobar} wants the following
% modules:
% \begin{quote}
%   |require("foobar.hello.world")|\\
%   |require("org.somewhere.xyz")|
% \end{quote}
% Then they can be find in:
% \begin{quote}
%   |TDS:scripts/foobar/foobar.hello.world.lua|\\
%   |TDS:scripts/foobar/org.somewhere.xyz.lua|
% \end{quote}
% I would have preferred the following locations, following lua conventions,
% e.\,g.:
% \begin{quote}
%   |TDS:scripts/foobar/hello/world.lua|\\
%   |TDS:scripts/foobar/org/somewhere/xyz.lua|
% \end{quote}
% But I do not know, how to achieve this in a reliable way using
% \textsf{kpathsea}.
%
% \subsubsection{Package \xpackage{luatex-loader}}
%
% If someone do not need or want package \xpackage{luatex} but
% it's extension for module loading, then he can use
% package \xpackage{luatex-loader}. Both \plainTeX\ and \LaTeX\ are
% supported.
%
% \StopEventually{
% }
%
% \section{Implementation}
%
%    \begin{macrocode}
%<*package>
%    \end{macrocode}
%
% \subsection{Reload check and package identification}
%    Reload check, especially if the package is not used with \LaTeX.
%    \begin{macrocode}
\begingroup\catcode61\catcode48\catcode32=10\relax%
  \catcode13=5 % ^^M
  \endlinechar=13 %
  \catcode35=6 % #
  \catcode39=12 % '
  \catcode44=12 % ,
  \catcode45=12 % -
  \catcode46=12 % .
  \catcode58=12 % :
  \catcode64=11 % @
  \catcode123=1 % {
  \catcode125=2 % }
  \expandafter\let\expandafter\x\csname ver@luatex.sty\endcsname
  \ifx\x\relax % plain-TeX, first loading
  \else
    \def\empty{}%
    \ifx\x\empty % LaTeX, first loading,
      % variable is initialized, but \ProvidesPackage not yet seen
    \else
      \expandafter\ifx\csname PackageInfo\endcsname\relax
        \def\x#1#2{%
          \immediate\write-1{Package #1 Info: #2.}%
        }%
      \else
        \def\x#1#2{\PackageInfo{#1}{#2, stopped}}%
      \fi
      \x{luatex}{The package is already loaded}%
      \aftergroup\endinput
    \fi
  \fi
\endgroup%
%    \end{macrocode}
%    Package identification:
%    \begin{macrocode}
\begingroup\catcode61\catcode48\catcode32=10\relax%
  \catcode13=5 % ^^M
  \endlinechar=13 %
  \catcode35=6 % #
  \catcode39=12 % '
  \catcode40=12 % (
  \catcode41=12 % )
  \catcode44=12 % ,
  \catcode45=12 % -
  \catcode46=12 % .
  \catcode47=12 % /
  \catcode58=12 % :
  \catcode64=11 % @
  \catcode91=12 % [
  \catcode93=12 % ]
  \catcode123=1 % {
  \catcode125=2 % }
  \expandafter\ifx\csname ProvidesPackage\endcsname\relax
    \def\x#1#2#3[#4]{\endgroup
      \immediate\write-1{Package: #3 #4}%
      \xdef#1{#4}%
    }%
  \else
    \def\x#1#2[#3]{\endgroup
      #2[{#3}]%
      \ifx#1\@undefined
        \xdef#1{#3}%
      \fi
      \ifx#1\relax
        \xdef#1{#3}%
      \fi
    }%
  \fi
\expandafter\x\csname ver@luatex.sty\endcsname
\ProvidesPackage{luatex}%
  [2010/03/09 v0.4 LuaTeX basic definition package (HO)]%
%    \end{macrocode}
%
% \subsection{Catcodes}
%
%    \begin{macrocode}
\begingroup\catcode61\catcode48\catcode32=10\relax%
  \catcode13=5 % ^^M
  \endlinechar=13 %
  \catcode123=1 % {
  \catcode125=2 % }
  \catcode64=11 % @
  \def\x{\endgroup
    \expandafter\edef\csname LuT@AtEnd\endcsname{%
      \endlinechar=\the\endlinechar\relax
      \catcode13=\the\catcode13\relax
      \catcode32=\the\catcode32\relax
      \catcode35=\the\catcode35\relax
      \catcode61=\the\catcode61\relax
      \catcode64=\the\catcode64\relax
      \catcode123=\the\catcode123\relax
      \catcode125=\the\catcode125\relax
    }%
  }%
\x\catcode61\catcode48\catcode32=10\relax%
\catcode13=5 % ^^M
\endlinechar=13 %
\catcode35=6 % #
\catcode64=11 % @
\catcode123=1 % {
\catcode125=2 % }
\def\TMP@EnsureCode#1#2{%
  \edef\LuT@AtEnd{%
    \LuT@AtEnd
    \catcode#1=\the\catcode#1\relax
  }%
  \catcode#1=#2\relax
}
\TMP@EnsureCode{10}{12}% ^^J
\TMP@EnsureCode{34}{12}% "
\TMP@EnsureCode{36}{3}% $
\TMP@EnsureCode{39}{12}% '
\TMP@EnsureCode{40}{12}% (
\TMP@EnsureCode{41}{12}% )
\TMP@EnsureCode{42}{12}% *
\TMP@EnsureCode{43}{12}% +
\TMP@EnsureCode{44}{12}% ,
\TMP@EnsureCode{45}{12}% -
\TMP@EnsureCode{46}{12}% .
\TMP@EnsureCode{47}{12}% /
\TMP@EnsureCode{60}{12}% <
\TMP@EnsureCode{62}{12}% >
\TMP@EnsureCode{91}{12}% [
\TMP@EnsureCode{93}{12}% ]
\TMP@EnsureCode{95}{12}% _ (other!)
\TMP@EnsureCode{96}{12}% `
\edef\LuT@AtEnd{\LuT@AtEnd\noexpand\endinput}
%    \end{macrocode}
%
% \subsection{Check for \LuaTeX}
%
%    Without \LuaTeX\ there is no point in using this package.
%    \begin{macrocode}
\begingroup\expandafter\expandafter\expandafter\endgroup
\expandafter\ifx\csname RequirePackage\endcsname\relax
  \input infwarerr.sty\relax
  \input ifluatex.sty\relax
\else
  \RequirePackage{infwarerr}[2007/09/09]%
  \RequirePackage{ifluatex}[2009/04/10]%
\fi
%    \end{macrocode}
%    \begin{macrocode}
\ifluatex
\else
  \@PackageError{luatex}{%
    This package may only be run using LuaTeX%
  }\@ehc
  \expandafter\LuT@AtEnd
\fi%
%    \end{macrocode}
%
% \subsection{Provide Lua\TeX\ primitives}
%
%    \begin{macrocode}
\ifnum\luatexversion<36 %
  \def\LuT@MakePrimitive#1{%
    \expandafter\let\csname luatex#1\expandafter\endcsname
    \csname #1\endcsname
  }%
\else
  \def\LuT@MakeLuatexPrimitive#1{%
    \begingroup\expandafter\expandafter\expandafter\endgroup
    \expandafter\ifx\csname luatex#1\endcsname\relax
      \begingroup\expandafter\expandafter\expandafter\endgroup
      \expandafter\ifx\csname #1\endcsname\relax
      \else
        \expandafter\let
        \csname luatex#1\expandafter\endcsname
        \csname #1\endcsname
      \fi
    \fi
    \begingroup\expandafter\expandafter\expandafter\endgroup
    \expandafter\ifx\csname luatex#1\endcsname\relax
      \begingroup
        \expandafter\let\csname luatex#1\endcsname\@undefined
        \ifnum0%
          \directlua{%
            if tex.enableprimitives then %
              tex.enableprimitives('luatex',{'#1'})%
              tex.print('1')%
            end%
          }%
          \expandafter\ifx\csname luatex#1\endcsname\relax\else1\fi
        =11 %
          \global\expandafter\let
          \csname luatex#1\expandafter\endcsname
          \csname luatex#1\endcsname
        \else
          \@PackageError{luatex}{%
            tex.enableprimitives failed for `#1'%
          }\@ehc
        \fi
      \endgroup
    \fi
  }%
  \def\LuT@MakePrimitive#1{%
    \begingroup\expandafter\expandafter\expandafter\endgroup
    \expandafter\ifx\csname#1\endcsname\relax
      \begingroup
        \expandafter\let\csname#1\endcsname\@undefined
        \ifnum0%
          \directlua{%
            if tex.enableprimitives then %
              tex.enableprimitives('',{'#1'})%
              tex.print('1')%
            end%
          }%
          \expandafter\ifx\csname#1\endcsname\relax\else1\fi
        =11 %
          \global\expandafter\let
          \csname#1\expandafter\endcsname
          \csname#1\endcsname
        \else
          \@PackageError{luatex}{%
            tex.enableprimitives failed for `#1'%
          }\@ehc
        \fi
      \endgroup
    \fi
  }%
\fi
\LuT@MakeLuatexPrimitive{attribute}
\LuT@MakeLuatexPrimitive{attributedef}
\LuT@MakeLuatexPrimitive{catcodetable}
\LuT@MakeLuatexPrimitive{initcatcodetable}
\LuT@MakeLuatexPrimitive{luaescapestring}
\LuT@MakeLuatexPrimitive{savecatcodetable}
\LuT@MakePrimitive{numexpr}
%    \end{macrocode}
%
% \subsection{Inherit support for \eTeX}
%
%    Package \xpackage{etex} is not compatible for \plainTeX.
%    But it could be present if a format is used that is based
%    on \xfile{etex.src}. Therefore we only load the package
%    in case of \LaTeX\ and tests its presence independently
%    of the format by looking for \cs{et@xins}.
%    \begin{macrocode}
\begingroup\expandafter\expandafter\expandafter\endgroup
\expandafter\ifx\csname RequirePackage\endcsname\relax
\else
  \RequirePackage{etex}[1998/03/26]%
\fi
%    \end{macrocode}
%
% \subsection{Adaption of \eTeX's register allocation}
%
%    \eTeX has increased the number of \TeX\ registers
%    from $2^8$ (256) to $2^{15}$ (32768) for a register class.
%    \LuaTeX\ extends the limit further to $2^16$ (65536).
%    The allocation scheme of package \xpackage{etex} is not
%    changed. But this can be subject for discussion.
%
%    If a register class hasn't registered any local registers
%    yet, then the limit can safely be pushed to 65536.
%    \begin{macrocode}
\begingroup\expandafter\expandafter\expandafter\endgroup
\expandafter\ifx\csname et@xins\endcsname\relax
  \@PackageWarningNoLine{luatex}{%
    Support for eTeX is not loaded (etex.src)%
  }%
\else
  \def\LuT@temp#1{%
    \ifnum\count27#1=32768 %
      \count27#1=65536 %
    \fi
  }%
  \LuT@temp0%
  \LuT@temp1%
  \LuT@temp2%
  \LuT@temp3%
  \LuT@temp4%
  \LuT@temp5%
  \LuT@temp6%
%    \end{macrocode}
%
% \eTeX\ uses an array for the first 256 registers and then a tree
% structure. \LuaTeX\ stores all registers of a class in one Lua table.
% There shouldn't be large performance differences.
% This allows starting immediately in the extended
% area, leaving room for insertions.
%    \begin{macrocode}
  \let\newcount\globcount
  \let\newdimen\globdimen
  \let\newskip\globskip
  \let\newbox\globbox
\fi
%    \end{macrocode}
%
% \subsection{\plainTeX\ compatibility}
%
%    \begin{macro}{\@empty}
%    \begin{macrocode}
\expandafter\ifx\csname @empty\endcsname\relax
  \def\@empty{}%
\fi
%    \end{macrocode}
%    \end{macro}
%    \begin{macro}{\@gobble}
%    \begin{macrocode}
\expandafter\ifx\csname @gobble\endcsname\relax
  \long\def\@gobble#1{}%
\fi
%    \end{macrocode}
%    \end{macro}
%    \begin{macro}{\@firstofone}
%    \begin{macrocode}
\expandafter\ifx\csname @firstofone\endcsname\relax
  \long\def\@firstofone#1{#1}%
\fi
%    \end{macrocode}
%    \end{macro}
%    \begin{macro}{\@firstoftwo}
%    \begin{macrocode}
\expandafter\ifx\csname @firstoftwo\endcsname\relax
  \long\def\@firstoftwo#1#2{#1}%
\fi
%    \end{macrocode}
%    \end{macro}
%    \begin{macro}{\@car}
%    \begin{macrocode}
\expandafter\ifx\csname @car\endcsname\relax
  \def\@car#1#2\@nil{#1}%
\fi
%    \end{macrocode}
%    \end{macro}
%    \begin{macro}{\@cdr}
%    \begin{macrocode}
\expandafter\ifx\csname @cdr\endcsname\relax
  \def\@cdr#1#2\@nil{#2}%
\fi
%    \end{macrocode}
%    \end{macro}
%    \begin{macro}{\@ifstar}
%    \begin{macrocode}
\expandafter\ifx\csname @ifstar\endcsname\relax
  \def\@ifstar#1{%
    \@ifnextchar*{\@firstoftwo{#1}}%
  }%
%    \end{macrocode}
%    \end{macro}
%    \begin{macro}{\@ifnextchar}
%    \begin{macrocode}
  \long\def\@ifnextchar#1#2#3{%
    \let\reserved@d=#1%
    \def\reserved@a{#2}%
    \def\reserved@b{#3}%
    \futurelet\@let@token\@ifnch
  }%
%    \end{macrocode}
%    \end{macro}
%    \begin{macro}{\@ifnch}
%    \begin{macrocode}
  \def\@ifnch{%
    \ifx\@let@token\@sptoken
      \let\reserved@c\@xifnch
    \else
      \ifx\@let@token\reserved@d
        \let\reserved@c\reserved@a
      \else
        \let\reserved@c\reserved@b
      \fi
    \fi
    \reserved@c
  }%
%    \end{macrocode}
%    \end{macro}
%    \begin{macro}{\@sptoken}
%    \begin{macrocode}
  \let\LuT@temp\:%
  \def\:{\let\@sptoken= }%
  \: % explicit space
%    \end{macrocode}
%    \end{macro}
%    \begin{macro}{\@xifnch}
%    \begin{macrocode}
  \def\:{\@xifnch}%
  \expandafter\def\: {%
    \futurelet\@let@token\@ifnch
  }%
  \let\:\LuT@temp
\fi
%    \end{macrocode}
%    \end{macro}
%
%    \begin{macro}{\@tempcnta}
%    \begin{macrocode}
\expandafter\ifx\csname @tempcnta\endcsname\relax
  \csname newcount\endcsname\@tempcnta
\fi
%    \end{macrocode}
%    \end{macro}
%
%    \begin{macro}{\@tempcntb}
%    \begin{macrocode}
\expandafter\ifx\csname @tempcntb\endcsname\relax
  \csname newcount\endcsname\@tempcntb
\fi
%    \end{macrocode}
%    \end{macro}
%
%    \begin{macro}{\LuT@newcommand}
%    \begin{macrocode}
\begingroup\expandafter\expandafter\expandafter\endgroup
\expandafter\ifx\csname newcommand\endcsname\relax
  \def\LuT@newcommand#1[#2]#3{%
    \ifx#1\@undefined
      \let#1\relax
    \else
      \ifx#1\relax
      \else
        \@PackageError{luatex}{%
          \string#1 is already defined.\MessageBreak
          Redefinition is skipped%
        }\@ehc
      \fi
    \fi
    \ifx#1\relax
      \ifcase#2 %
        \def#1{#3}%
      \or
        \def#1##1{#3}%
      \or
        \def#1##1##2{#3}%
      \or
        \def#1##1##2##3{#3}%
      \or
        \@INTERNAL@ERROR
      \fi
    \fi
  }%
\else
  \def\LuT@newcommand{\newcommand*}%
\fi
%    \end{macrocode}
%    \end{macro}
%
% \subsection{Attributes}
%
% \subsubsection{Allocation}
%
%    \begin{macro}{\LuT@AllocAttribute}
%    \begin{macrocode}
\newcount\LuT@AllocAttribute
\LuT@AllocAttribute=\m@ne
%    \end{macrocode}
%    \end{macro}
%    \begin{macro}{\newattribute}
%    \begin{macrocode}
\LuT@newcommand\newattribute[1]{%
  \ifnum\LuT@AllocAttribute<65535 %
    \global\advance\LuT@AllocAttribute\@ne
    \allocationnumber\LuT@AllocAttribute
    \global\luatexattributedef#1=\allocationnumber
    \unsetattribute{#1}%
    \wlog{\string#1=\string\attribute\the\allocationnumber}%
  \else
    \errmessage{No room for a new \string\attribute}%
  \fi
}
%    \end{macrocode}
%    \end{macro}
%
% \subsubsection{Interface}
%
%    \begin{macro}{\setattribute}
%    \begin{macrocode}
\LuT@newcommand\setattribute[2]{%
  #1=\numexpr#2\relax
}
%    \end{macrocode}
%    \end{macro}
%    \begin{macro}{\unsetattribute}
%    \begin{macrocode}
\ifnum\luatexversion<37
  \LuT@newcommand\LuT@UnsetAttributeValue[0]{}%
  \let\LuT@UnsetAttributeValue\m@ne
\else
  \LuT@newcommand\LuT@UnsetAttributeValue[0]{-2147483647 }%
\fi
\LuT@newcommand\unsetattribute[1]{%
  #1=\LuT@UnsetAttributeValue
}
%    \end{macrocode}
%    \end{macro}
%
% \subsection{Catcode tables}
%
% \subsubsection{Allocation}
%
%    \begin{macro}{\LuT@AllocCatcodeTable}
%    \begin{macrocode}
\newcount\LuT@AllocCatcodeTable
\LuT@AllocCatcodeTable=\m@ne
\newcount\CatcodeTableStack
\CatcodeTableStack=\z@
%    \end{macrocode}
%    \end{macro}
%    \begin{macro}{\newcatcodetable}
%    \begin{macrocode}
\LuT@newcommand\newcatcodetable[1]{%
  \ifnum\LuT@AllocCatcodeTable<1114110 % 0x10FFFF is maximal \chardef
    % or < 268435455 % 2^28 - 1
    \global\advance\LuT@AllocCatcodeTable by\tw@
    \allocationnumber=\LuT@AllocCatcodeTable
    \global\chardef#1=\allocationnumber
    \wlog{%
      \string#1=\string\catcodetable\the\allocationnumber
    }%
  \else
    \errmessage{No room for a new \string\catcodetable}%
  \fi
}%
%    \end{macrocode}
%    \end{macro}
%
%    \begin{macro}{\IncCatcodeTableStack}
%    \begin{macrocode}
\LuT@newcommand\IncCatcodeTableStack[0]{%
  \ifnum\CatcodeTableStack<268435454 %
    \global\advance\CatcodeTableStack by\tw@
  \else
    \@PackageError{luatex}{%
      Catcode table stack overflow%
    }\@ehd
  \fi
}
%    \end{macrocode}
%    \end{macro}
%
%    \begin{macro}{\DecCatcodeTableStack}
%    \begin{macrocode}
\LuT@newcommand\DecCatcodeTableStack[0]{%
  \ifnum\CatcodeTableStack>\z@
    \global\advance\CatcodeTableStack by-2 %
  \else
    \@PackageError{luatex}{%
      Catcode table stack is empty%
    }\@ehd
  \fi
}
%    \end{macrocode}
%    \end{macro}
%
% \subsubsection{\cs{SetCatcodeRange}}
%
%    \begin{macro}{\SetCatcodeRange}
%    \begin{macrocode}
\LuT@newcommand\SetCatcodeRange[3]{%
  \edef\LuT@temp{%
    \noexpand\@tempcnta=\the\@tempcnta
    \noexpand\@tempcntb=\the\@tempcntb
    \noexpand\count@=\the\count@
    \relax
  }%
  \@tempcnta=\numexpr#1\relax
  \@tempcntb=\numexpr#2\relax
  \count@=\numexpr#3\relax
  \loop
    \unless\ifnum\@tempcnta>\@tempcntb
    \catcode\@tempcnta=\count@
    \advance\@tempcnta by \@ne
  \repeat
  \LuT@temp
}
%    \end{macrocode}
%    \end{macro}
%
% \subsubsection{Predefined catcode tables}
%
%    \begin{macrocode}
\newcatcodetable\CatcodeTableIniTeX
\newcatcodetable\CatcodeTableString
\newcatcodetable\CatcodeTableOther
\newcatcodetable\CatcodeTableLaTeX
%    \end{macrocode}
%
%    \begin{macrocode}
\luatexinitcatcodetable\CatcodeTableIniTeX
\begingroup
  \def\@makeother#1{\catcode#1=12\relax}%
  \@firstofone{%
    \luatexcatcodetable\CatcodeTableIniTeX
    \begingroup
      \SetCatcodeRange{0}{8}{15}%
      \catcode9=10 % tab
      \catcode11=15 %
      \catcode12=13 % form feed
      \SetCatcodeRange{14}{31}{15}%
      \catcode35=6 % hash
      \catcode36=3 % dollar
      \catcode38=4 % ampersand
      \catcode94=7 % circumflex
      \catcode95=8 % underscore
      \catcode123=1 % brace left
      \catcode125=2 % brace right
      \catcode126=13 % tilde
      \catcode127=15 %
      \luatexsavecatcodetable\CatcodeTableLaTeX
    \endgroup
    \@makeother{0}% nul
    \@makeother{13}% carriage return
    \@makeother{37}% percent
    \@makeother{92}% backslash
    \@makeother{127}%
    \SetCatcodeRange{65}{90}{12}% A-Z
    \SetCatcodeRange{97}{122}{12}% a-z
    \luatexsavecatcodetable\CatcodeTableString
    \@makeother{32}% space
    \luatexsavecatcodetable\CatcodeTableOther
  \endgroup
}%
%    \end{macrocode}
%
% \subsubsection{Number stack}
%
%    \begin{macro}{\LuT@NumStackEmpty}
%    A special empty stack value because of \cs{@cdr}'s brace removal.
%    \begin{macrocode}
\def\LuT@NumStackEmpty{0}
%    \end{macrocode}
%    \end{macro}
%    \begin{macro}{\LuT@NumStack}
%    \begin{macrocode}
\let\LuT@NumStack\LuT@NumStackEmpty
%    \end{macrocode}
%    \end{macro}
%    \begin{macro}{\PushCatcodeTableNumStack}
%    \begin{macrocode}
\LuT@newcommand\PushCatcodeTableNumStack[0]{%
  \xdef\LuT@NumStack{%
    {\the\luatexcatcodetable}\LuT@NumStack
  }%
}
%    \end{macrocode}
%    \end{macro}
%    \begin{macro}{\PopCatcodeTableNumStack}
%    \begin{macrocode}
\LuT@newcommand\PopCatcodeTableNumStack[0]{%
  \ifx\LuT@NumStack\LuT@NumStackEmpty
    \@PackageWarning{luatex}{Empty catcode table number stack}%
    \luatexcatcodetable\z@
  \else
    \luatexcatcodetable=\expandafter\@car\LuT@NumStack\@nil\relax
    \xdef\LuT@NumStack{%
      \expandafter\@cdr\LuT@NumStack\@nil
    }%
  \fi
}
%    \end{macrocode}
%    \end{macro}
%
% \subsubsection{Catcode regime macros}
%
%    \begin{macro}{\BeginCatcodeRegime}
%    \begin{macrocode}
\LuT@newcommand\BeginCatcodeRegime[1]{%
  \PushCatcodeTableNumStack
  \luatexcatcodetable=\numexpr#1\relax
  \IncCatcodeTableStack
  \luatexsavecatcodetable\CatcodeTableStack
  \luatexcatcodetable\CatcodeTableStack
}
%    \end{macrocode}
%    \end{macro}
%
%    \begin{macro}{\EndCatcodeRegime}
%    \begin{macrocode}
\LuT@newcommand\EndCatcodeRegime[0]{%
  \DecCatcodeTableStack
  \PopCatcodeTableNumStack
}
%    \end{macrocode}
%    \end{macro}
%
% \subsection{Lua module loader}
%
%    \begin{macrocode}
\begingroup\expandafter\expandafter\expandafter\endgroup
\expandafter\ifx\csname RequirePackage\endcsname\relax
  \input luatex-loader.sty\relax
\else
  \RequirePackage{luatex-loader}[2010/03/09]%
\fi
%    \end{macrocode}
%    \begin{macrocode}
\LuT@AtEnd%
%</package>
%    \end{macrocode}
%
%    \begin{macrocode}
%<*loader>
%    \end{macrocode}
%
%    Reload check, especially if the package is not used with \LaTeX.
%    \begin{macrocode}
\begingroup\catcode61\catcode48\catcode32=10\relax%
  \catcode13=5 % ^^M
  \endlinechar=13 %
  \catcode35=6 % #
  \catcode39=12 % '
  \catcode44=12 % ,
  \catcode45=12 % -
  \catcode46=12 % .
  \catcode58=12 % :
  \catcode64=11 % @
  \catcode123=1 % {
  \catcode125=2 % }
  \expandafter\let\expandafter\x\csname ver@luatex-loader.sty\endcsname
  \ifx\x\relax % plain-TeX, first loading
  \else
    \def\empty{}%
    \ifx\x\empty % LaTeX, first loading,
      % variable is initialized, but \ProvidesPackage not yet seen
    \else
      \expandafter\ifx\csname PackageInfo\endcsname\relax
        \def\x#1#2{%
          \immediate\write-1{Package #1 Info: #2.}%
        }%
      \else
        \def\x#1#2{\PackageInfo{#1}{#2, stopped}}%
      \fi
      \x{luatex-loader}{The package is already loaded}%
      \aftergroup\endinput
    \fi
  \fi
\endgroup%
%    \end{macrocode}
%    Package identification:
%    \begin{macrocode}
\begingroup\catcode61\catcode48\catcode32=10\relax%
  \catcode13=5 % ^^M
  \endlinechar=13 %
  \catcode35=6 % #
  \catcode39=12 % '
  \catcode40=12 % (
  \catcode41=12 % )
  \catcode44=12 % ,
  \catcode45=12 % -
  \catcode46=12 % .
  \catcode47=12 % /
  \catcode58=12 % :
  \catcode64=11 % @
  \catcode91=12 % [
  \catcode93=12 % ]
  \catcode123=1 % {
  \catcode125=2 % }
  \expandafter\ifx\csname ProvidesPackage\endcsname\relax
    \def\x#1#2#3[#4]{\endgroup
      \immediate\write-1{Package: #3 #4}%
      \xdef#1{#4}%
    }%
  \else
    \def\x#1#2[#3]{\endgroup
      #2[{#3}]%
      \ifx#1\@undefined
        \xdef#1{#3}%
      \fi
      \ifx#1\relax
        \xdef#1{#3}%
      \fi
    }%
  \fi
\expandafter\x\csname ver@luatex-loader.sty\endcsname
\ProvidesPackage{luatex-loader}%
  [2010/03/09 v0.4 Lua module loader (HO)]%
%    \end{macrocode}
%    \begin{macrocode}
\begingroup\catcode61\catcode48\catcode32=10\relax%
  \catcode13=5\endlinechar=13\relax%
  \catcode10=12 % ^^J
  \catcode34=12 % "
  \catcode39=12 % '
  \catcode40=12 % (
  \catcode41=12 % )
  \catcode44=12 % ,
  \catcode46=12 % .
  \catcode60=12 % <
  \catcode61=12 % =
  \catcode95=12 % _ (other!)
  \catcode96=12 % `
  \catcode123=1 % {
  \catcode125=2 % }
  \endlinechar=10 %
  \ifnum\luatexversion<36 %
    \directlua0%
  \else %
    \expandafter\directlua %
  \fi %
  {%
     do
      local script = "oberdiek.luatex.lua"
      local file = kpse.find_file(script, "texmfscripts")
      if file then
        texio.write_nl("(" .. file .. ")")
        dofile(file)
      else
        error("File `" .. script .. "' not found")
      end
    end
  }%
\endgroup\endinput%
%    \end{macrocode}
%
%    \begin{macrocode}
%</loader>
%    \end{macrocode}
%
% \subsection{Lua script}
%
%    Currently \LuaTeX\ does not use KPSE when searching for module
%    files. The following Lua script implements a workaround.
%    It extends |package.loader| by another search method.
%    Modules are found by the module name with extension \xext{lua}
%    similar to
%    \begin{quote}
%      \ttfamily |kpsewhich --format=texmfscripts| \meta{module}|.lua|
%    \end{quote}
%    Unhappily \textsf{kpsewhich} does not support directory components
%    in the file name. Therefore a module |a.b.c| cannot be installed
%    as \xfile{a/b/c.lua}. The script must be named \xfile{a.b.c.lua}.
%    \begin{macrocode}
%<*lua>
%    \end{macrocode}
%    \begin{macrocode}
module("oberdiek.luatex", package.seeall)
%    \end{macrocode}
%    \begin{macrocode}
function kpse_module_loader(module)
  local script = module .. ".lua"
  local file = kpse.find_file(script, "texmfscripts")
  if file then
    local loader, error = loadfile(file)
    if loader then
      texio.write_nl("(" .. file .. ")")
      return loader
    end
    return "\n\t[oberdiek.luatex.kpse_module_loader] Loading error:\n\t"
           .. error
  end
  return "\n\t[oberdiek.luatex.kpse_module_loader] Search failed"
end
table.insert(package.loaders, kpse_module_loader)
%    \end{macrocode}
%    \begin{macrocode}
%</lua>
%    \end{macrocode}
%
% \section{Test}
%
%    \begin{macrocode}
%<*test2>
\documentclass{article}
\def\LoadCommand{%
  \RequirePackage{luatex}[2010/03/09]%
}
%</test2>
%    \end{macrocode}
%    \begin{macrocode}
%<*test3>
\documentclass{article}
\def\LoadCommand{%
  \RequirePackage{luatex-loader}[2010/03/09]%
}
%</test3>
%    \end{macrocode}
% \subsection{Catcode checks for loading}
%
%    \begin{macrocode}
%<*test1>
%    \end{macrocode}
%    \begin{macrocode}
\catcode`\{=1 %
\catcode`\}=2 %
\catcode`\#=6 %
\catcode`\@=11 %
\expandafter\ifx\csname count@\endcsname\relax
  \countdef\count@=255 %
\fi
\expandafter\ifx\csname @gobble\endcsname\relax
  \long\def\@gobble#1{}%
\fi
\expandafter\ifx\csname @firstofone\endcsname\relax
  \long\def\@firstofone#1{#1}%
\fi
\expandafter\ifx\csname loop\endcsname\relax
  \expandafter\@firstofone
\else
  \expandafter\@gobble
\fi
{%
  \def\loop#1\repeat{%
    \def\body{#1}%
    \iterate
  }%
  \def\iterate{%
    \body
      \let\next\iterate
    \else
      \let\next\relax
    \fi
    \next
  }%
  \let\repeat=\fi
}%
\def\RestoreCatcodes{}
\count@=0 %
\loop
  \edef\RestoreCatcodes{%
    \RestoreCatcodes
    \catcode\the\count@=\the\catcode\count@\relax
  }%
\ifnum\count@<255 %
  \advance\count@ 1 %
\repeat

\def\RangeCatcodeInvalid#1#2{%
  \count@=#1\relax
  \loop
    \catcode\count@=15 %
  \ifnum\count@<#2\relax
    \advance\count@ 1 %
  \repeat
}
\def\RangeCatcodeCheck#1#2#3{%
  \count@=#1\relax
  \loop
    \ifnum#3=\catcode\count@
    \else
      \errmessage{%
        Character \the\count@\space
        with wrong catcode \the\catcode\count@\space
        instead of \number#3%
      }%
    \fi
  \ifnum\count@<#2\relax
    \advance\count@ 1 %
  \repeat
}
\def\space{ }
\expandafter\ifx\csname LoadCommand\endcsname\relax
  \def\LoadCommand{\input luatex.sty\relax}%
\fi
\def\Test{%
  \RangeCatcodeInvalid{0}{47}%
  \RangeCatcodeInvalid{58}{64}%
  \RangeCatcodeInvalid{91}{96}%
  \RangeCatcodeInvalid{123}{255}%
  \catcode`\@=12 %
  \catcode`\\=0 %
  \catcode`\%=14 %
  \LoadCommand
  \RangeCatcodeCheck{0}{36}{15}%
  \RangeCatcodeCheck{37}{37}{14}%
  \RangeCatcodeCheck{38}{47}{15}%
  \RangeCatcodeCheck{48}{57}{12}%
  \RangeCatcodeCheck{58}{63}{15}%
  \RangeCatcodeCheck{64}{64}{12}%
  \RangeCatcodeCheck{65}{90}{11}%
  \RangeCatcodeCheck{91}{91}{15}%
  \RangeCatcodeCheck{92}{92}{0}%
  \RangeCatcodeCheck{93}{96}{15}%
  \RangeCatcodeCheck{97}{122}{11}%
  \RangeCatcodeCheck{123}{255}{15}%
  \RestoreCatcodes
}
\Test
\csname @@end\endcsname
\end
%    \end{macrocode}
%    \begin{macrocode}
%</test1>
%    \end{macrocode}
%
% \subsection{Catcode tables}
%
% \subsubsection{Predefined catcode tables}
%
%    \begin{macrocode}
%<*test4>
\NeedsTeXFormat{LaTeX2e}
%    \end{macrocode}
%    Remember \LaTeX's initial catcodes in count registers
%    starting at \cs{TestLaTeX}.
%    \begin{macrocode}
\count0=0 %
\chardef\TestLaTeX=1000 %
\chardef\TestMax=300 %
\loop
  \count\numexpr\TestLaTeX+\count0\relax=\catcode\count0 %
\ifnum\count0<\TestMax
  \advance\count0 by 1 %
\repeat
\documentclass{minimal}
\usepackage{luatex}[2010/03/09]
\usepackage{qstest}
\IncludeTests{*}
\LogTests{log}{*}{*}
\makeatletter
\def\Check#1{%
  \Expect*{\the\count@=\the\catcode\count@}%
         *{\the\count@=#1}%
}
\newcount\scratch
\def\Test#1#2{%
  \begin{qstest}{CatcodeTable#1}{CatcodeTable#1}%
    \luatexcatcodetable\csname CatcodeTable#1\endcsname
    \count@=\z@
    \loop
      \scratch=#2\relax
      \Expect*{\the\count@=\the\catcode\count@}%
             *{\the\count@=\the\scratch}%
    \ifnum\count@<\TestMax
      \advance\count@\@ne
    \repeat
  \end{qstest}%
}
\begingroup
  % luatex-unicode-letters.tex makes some slots to letters
  \def\TestMax{169}%
  \Test{LaTeX}{\the\count\numexpr\TestLaTeX+\count@}%
\endgroup
\Test{String}{\ifnum\count@=32 10\else 12\fi}
\Test{Other}{12}
\luatexinitcatcodetable99 %
\Test{IniTeX}{%
  0\relax
  \begingroup
    \luatexcatcodetable99 %
    \global\scratch=\the\catcode\count@
  \endgroup
}
%    \end{macrocode}
%
% \subsubsection{Catcode table number stack}
%
%    \begin{macrocode}
\begin{qstest}{CatcodeTableNumStack}{CatcodeTableNumStack}
  \def\TestStack#1{%
    \Expect*{\LuT@NumStack}{#1}%
  }%
  \TestStack{0}%
  \PushCatcodeTableNumStack
  \TestStack{{0}0}%
  \@firstofone{%
    \begingroup
      \luatexinitcatcodetable12 %
      \luatexcatcodetable12 %
      \PushCatcodeTableNumStack
      \TestStack{{12}{0}0}%
      \PopCatcodeTableNumStack
      \TestStack{{0}0}%
      \PopCatcodeTableNumStack
      \TestStack{0}%
      \def\TestWarning{Missing empty stack warning}%
      \def\@PackageWarning#1#2{\def\TestWarning{empty stack}}%
      \PopCatcodeTableNumStack
      \TestStack{0}%
      \Expect*{\TestWarning}{empty stack}%
    \endgroup
  }%
\end{qstest}
%    \end{macrocode}
%
% \subsubsection{Catcode table stack}
%
%    \begin{macrocode}
\begin{qstest}{CatcodeTableStack}{CatcodeTableStack}
  \def\TestStack#1{%
    \Expect*{\the\CatcodeTableStack}{#1}%
  }%
  \TestStack{0}%
  \IncCatcodeTableStack
  \TestStack{2}%
  \IncCatcodeTableStack
  \TestStack{4}%
  \begingroup
    \IncCatcodeTableStack
    \TestStack{6}%
  \endgroup
  \TestStack{6}%
  \begingroup
    \DecCatcodeTableStack
    \TestStack{4}%
  \endgroup
  \TestStack{4}%
  \DecCatcodeTableStack
  \TestStack{2}%
  \DecCatcodeTableStack
  \TestStack{0}%
  \begingroup
    \def\TestError{Missing error}%
    \def\@PackageError#1#2#3{%
      \def\TestError{Empty stack}%
    }%
    \DecCatcodeTableStack
    \TestStack{0}%
    \Expect*{\TestError}{Empty stack}%
  \endgroup
\end{qstest}
%    \end{macrocode}
%
% \subsubsection{Catcode regime macros}
%
%    \begin{macrocode}
\begin{qstest}{CatcodeRegime}{CatcodeRegime}
  \def\TestStacks#1#2#3{%
    \Expect*{\the\luatexcatcodetable}{#1}%
    \Expect*{\the\CatcodeTableStack}{#2}%
    \Expect*{\LuT@NumStack}{#3}%
  }%
  \TestStacks{0}{0}{0}%
  \catcode`\|=7 %
  \BeginCatcodeRegime\CatcodeTableLaTeX
    \TestStacks{2}{2}{{0}0}%
    \Expect*{\the\catcode`\|}{12}%
  \EndCatcodeRegime
  \TestStacks{0}{0}{0}%
  \Expect*{\the\catcode`\|}{7}%
\end{qstest}
%    \end{macrocode}
%
% \subsection{Attribute allocation}
%
%    \begin{macrocode}
\begin{qstest}{Attributes}{Attributes}
  \newattribute\TestAttr
  \Expect*{\meaning\TestAttr}%
         *{\string\attribute\number\allocationnumber}%
  \Expect*{\the\allocationnumber}{0}%
  \begingroup
    \newattribute\TestAttr
    \Expect*{\the\allocationnumber}{1}%
  \endgroup
  \Expect*{\the\allocationnumber}{0}%
  \Expect*{\meaning\TestAttr}*{\string\attribute1}%
  \Expect*{\the\TestAttr}*{\number\LuT@UnsetAttributeValue}%
  \def\Test#1{%
    \setattribute\TestAttr{#1}%
    \Expect*{\the\TestAttr}{#1}%
  }%
  \Test{0}%
  \Test{1}%
  \Test{-1}%
  \Test{123}%
  \unsetattribute\TestAttr
  \Expect*{\the\TestAttr}*{\number\LuT@UnsetAttributeValue}%
  \begingroup
    \Expect*{\the\TestAttr}*{\number\LuT@UnsetAttributeValue}%
    \Test{1234}%
  \endgroup
  \Expect*{\the\TestAttr}*{\number\LuT@UnsetAttributeValue}%
\end{qstest}
%    \end{macrocode}
%
%    \begin{macrocode}
\@@end
%</test4>
%    \end{macrocode}
%
% \subsection{Short test for \plainTeX}
%
%    \begin{macrocode}
%<*test5>
\input luatex.sty\relax
\newattribute\TestAttr
\setattribute\TestAttr{10}
\unsetattribute\TestAttr
\newcatcodetable\TestCTa
\begingroup
  \SetCatcodeRange{`A}{`Z}{12}%
\endgroup
\BeginCatcodeRegime\CatcodeTableLaTeX
\EndCatcodeRegime
\end
%</test5>
%    \end{macrocode}
%
% \section{Installation}
%
% \subsection{Download}
%
% \paragraph{Package.} This package is available on
% CTAN\footnote{\url{ftp://ftp.ctan.org/tex-archive/}}:
% \begin{description}
% \item[\CTAN{macros/latex/contrib/oberdiek/luatex.dtx}] The source file.
% \item[\CTAN{macros/latex/contrib/oberdiek/luatex.pdf}] Documentation.
% \end{description}
%
%
% \paragraph{Bundle.} All the packages of the bundle `oberdiek'
% are also available in a TDS compliant ZIP archive. There
% the packages are already unpacked and the documentation files
% are generated. The files and directories obey the TDS standard.
% \begin{description}
% \item[\CTAN{install/macros/latex/contrib/oberdiek.tds.zip}]
% \end{description}
% \emph{TDS} refers to the standard ``A Directory Structure
% for \TeX\ Files'' (\CTAN{tds/tds.pdf}). Directories
% with \xfile{texmf} in their name are usually organized this way.
%
% \subsection{Bundle installation}
%
% \paragraph{Unpacking.} Unpack the \xfile{oberdiek.tds.zip} in the
% TDS tree (also known as \xfile{texmf} tree) of your choice.
% Example (linux):
% \begin{quote}
%   |unzip oberdiek.tds.zip -d ~/texmf|
% \end{quote}
%
% \paragraph{Script installation.}
% Check the directory \xfile{TDS:scripts/oberdiek/} for
% scripts that need further installation steps.
% Package \xpackage{attachfile2} comes with the Perl script
% \xfile{pdfatfi.pl} that should be installed in such a way
% that it can be called as \texttt{pdfatfi}.
% Example (linux):
% \begin{quote}
%   |chmod +x scripts/oberdiek/pdfatfi.pl|\\
%   |cp scripts/oberdiek/pdfatfi.pl /usr/local/bin/|
% \end{quote}
%
% \subsection{Package installation}
%
% \paragraph{Unpacking.} The \xfile{.dtx} file is a self-extracting
% \docstrip\ archive. The files are extracted by running the
% \xfile{.dtx} through \plainTeX:
% \begin{quote}
%   \verb|tex luatex.dtx|
% \end{quote}
%
% \paragraph{TDS.} Now the different files must be moved into
% the different directories in your installation TDS tree
% (also known as \xfile{texmf} tree):
% \begin{quote}
% \def\t{^^A
% \begin{tabular}{@{}>{\ttfamily}l@{ $\rightarrow$ }>{\ttfamily}l@{}}
%   luatex.sty & tex/generic/oberdiek/luatex.sty\\
%   luatex-loader.sty & tex/generic/oberdiek/luatex-loader.sty\\
%   oberdiek.luatex.lua & scripts/oberdiek/oberdiek.luatex.lua\\
%   luatex.pdf & doc/latex/oberdiek/luatex.pdf\\
%   test/luatex-test1.tex & doc/latex/oberdiek/test/luatex-test1.tex\\
%   test/luatex-test2.tex & doc/latex/oberdiek/test/luatex-test2.tex\\
%   test/luatex-test3.tex & doc/latex/oberdiek/test/luatex-test3.tex\\
%   test/luatex-test4.tex & doc/latex/oberdiek/test/luatex-test4.tex\\
%   test/luatex-test5.tex & doc/latex/oberdiek/test/luatex-test5.tex\\
%   luatex.dtx & source/latex/oberdiek/luatex.dtx\\
% \end{tabular}^^A
% }^^A
% \sbox0{\t}^^A
% \ifdim\wd0>\linewidth
%   \begingroup
%     \advance\linewidth by\leftmargin
%     \advance\linewidth by\rightmargin
%   \edef\x{\endgroup
%     \def\noexpand\lw{\the\linewidth}^^A
%   }\x
%   \def\lwbox{^^A
%     \leavevmode
%     \hbox to \linewidth{^^A
%       \kern-\leftmargin\relax
%       \hss
%       \usebox0
%       \hss
%       \kern-\rightmargin\relax
%     }^^A
%   }^^A
%   \ifdim\wd0>\lw
%     \sbox0{\small\t}^^A
%     \ifdim\wd0>\linewidth
%       \ifdim\wd0>\lw
%         \sbox0{\footnotesize\t}^^A
%         \ifdim\wd0>\linewidth
%           \ifdim\wd0>\lw
%             \sbox0{\scriptsize\t}^^A
%             \ifdim\wd0>\linewidth
%               \ifdim\wd0>\lw
%                 \sbox0{\tiny\t}^^A
%                 \ifdim\wd0>\linewidth
%                   \lwbox
%                 \else
%                   \usebox0
%                 \fi
%               \else
%                 \lwbox
%               \fi
%             \else
%               \usebox0
%             \fi
%           \else
%             \lwbox
%           \fi
%         \else
%           \usebox0
%         \fi
%       \else
%         \lwbox
%       \fi
%     \else
%       \usebox0
%     \fi
%   \else
%     \lwbox
%   \fi
% \else
%   \usebox0
% \fi
% \end{quote}
% If you have a \xfile{docstrip.cfg} that configures and enables \docstrip's
% TDS installing feature, then some files can already be in the right
% place, see the documentation of \docstrip.
%
% \subsection{Refresh file name databases}
%
% If your \TeX~distribution
% (\teTeX, \mikTeX, \dots) relies on file name databases, you must refresh
% these. For example, \teTeX\ users run \verb|texhash| or
% \verb|mktexlsr|.
%
% \subsection{Some details for the interested}
%
% \paragraph{Attached source.}
%
% The PDF documentation on CTAN also includes the
% \xfile{.dtx} source file. It can be extracted by
% AcrobatReader 6 or higher. Another option is \textsf{pdftk},
% e.g. unpack the file into the current directory:
% \begin{quote}
%   \verb|pdftk luatex.pdf unpack_files output .|
% \end{quote}
%
% \paragraph{Unpacking with \LaTeX.}
% The \xfile{.dtx} chooses its action depending on the format:
% \begin{description}
% \item[\plainTeX:] Run \docstrip\ and extract the files.
% \item[\LaTeX:] Generate the documentation.
% \end{description}
% If you insist on using \LaTeX\ for \docstrip\ (really,
% \docstrip\ does not need \LaTeX), then inform the autodetect routine
% about your intention:
% \begin{quote}
%   \verb|latex \let\install=y\input{luatex.dtx}|
% \end{quote}
% Do not forget to quote the argument according to the demands
% of your shell.
%
% \paragraph{Generating the documentation.}
% You can use both the \xfile{.dtx} or the \xfile{.drv} to generate
% the documentation. The process can be configured by the
% configuration file \xfile{ltxdoc.cfg}. For instance, put this
% line into this file, if you want to have A4 as paper format:
% \begin{quote}
%   \verb|\PassOptionsToClass{a4paper}{article}|
% \end{quote}
% An example follows how to generate the
% documentation with pdf\LaTeX:
% \begin{quote}
%\begin{verbatim}
%pdflatex luatex.dtx
%makeindex -s gind.ist luatex.idx
%pdflatex luatex.dtx
%makeindex -s gind.ist luatex.idx
%pdflatex luatex.dtx
%\end{verbatim}
% \end{quote}
%
% \section{Catalogue}
%
% The following XML file can be used as source for the
% \href{http://mirror.ctan.org/help/Catalogue/catalogue.html}{\TeX\ Catalogue}.
% The elements \texttt{caption} and \texttt{description} are imported
% from the original XML file from the Catalogue.
% The name of the XML file in the Catalogue is \xfile{luatex.xml}.
%    \begin{macrocode}
%<*catalogue>
<?xml version='1.0' encoding='us-ascii'?>
<!DOCTYPE entry SYSTEM 'catalogue.dtd'>
<entry datestamp='$Date$' modifier='$Author$' id='luatex'>
  <name>luatex</name>
  <caption>The LuaTeX engine.</caption>
  <authorref id='auth:oberdiek'/>
  <copyright owner='Heiko Oberdiek' year='2007,2009,2010'/>
  <license type='lppl1.3'/>
  <version number='0.4'/>
  <description>
    LuaTeX is an extended version of pdfTeX using Lua as an embedded
    scripting language.  The LuaTeX project&#x2019;s main objective
    is to provide an open and configurable variant of TeX while at the
    same time offering downward compatibility.
    <p/>
    LuaTeX uses Unicode (as UTF-8) as its default input encoding, and
    is able to use modern (OpenType) fonts (for both text and mathematics).
    <p/>
    It should be noted that LuaTeX is still under development; its
    specification has been declared stable, but absolute stability
    may not in practice be assumed.
    <p/>
    The package is part of the <xref refid='oberdiek'>oberdiek</xref> bundle.
  </description>
  <documentation details='Package documentation'
      href='ctan:/macros/latex/contrib/oberdiek/luatex.pdf'/>
  <ctan file='true' path='/macros/latex/contrib/oberdiek/luatex.dtx'/>
  <miktex location='oberdiek'/>
  <texlive location='oberdiek'/>
  <install path='/macros/latex/contrib/oberdiek/oberdiek.tds.zip'/>
</entry>
%</catalogue>
%    \end{macrocode}
%
% \begin{History}
%   \begin{Version}{2007/12/12 v0.1}
%   \item
%     First public version.
%   \end{Version}
%   \begin{Version}{2009/04/10 v0.2}
%   \item
%     Requires package \xpackage{ifluatex} in version 2.0 to ensure
%     \cs{luatexversion}.
%   \item
%     Updates the call of \cs{directlua}, the syntax has changed
%     in \LuaTeX\ 0.36.
%   \end{Version}
%   \begin{Version}{2009/12/02 v0.3}
%   \item
%     Unsetting of attributes updated for \LuaTeX\ 0.37.
%   \end{Version}
%   \begin{Version}{2010/03/09 v0.4}
%   \item
%     Support for lua states removed.
%   \item
%     Calling tex.enableprimitives for used primitives.
%   \end{Version}
% \end{History}
%
% \PrintIndex
%
% \Finale
\endinput

%        (quote the arguments according to the demands of your shell)
%
% Documentation:
%    (a) If luatex.drv is present:
%           latex luatex.drv
%    (b) Without luatex.drv:
%           latex luatex.dtx; ...
%    The class ltxdoc loads the configuration file ltxdoc.cfg
%    if available. Here you can specify further options, e.g.
%    use A4 as paper format:
%       \PassOptionsToClass{a4paper}{article}
%
%    Programm calls to get the documentation (example):
%       pdflatex luatex.dtx
%       makeindex -s gind.ist luatex.idx
%       pdflatex luatex.dtx
%       makeindex -s gind.ist luatex.idx
%       pdflatex luatex.dtx
%
% Installation:
%    TDS:tex/generic/oberdiek/luatex.sty
%    TDS:tex/generic/oberdiek/luatex-loader.sty
%    TDS:scripts/oberdiek/oberdiek.luatex.lua
%    TDS:doc/latex/oberdiek/luatex.pdf
%    TDS:doc/latex/oberdiek/test/luatex-test1.tex
%    TDS:doc/latex/oberdiek/test/luatex-test2.tex
%    TDS:doc/latex/oberdiek/test/luatex-test3.tex
%    TDS:doc/latex/oberdiek/test/luatex-test4.tex
%    TDS:doc/latex/oberdiek/test/luatex-test5.tex
%    TDS:source/latex/oberdiek/luatex.dtx
%
%<*ignore>
\begingroup
  \catcode123=1 %
  \catcode125=2 %
  \def\x{LaTeX2e}%
\expandafter\endgroup
\ifcase 0\ifx\install y1\fi\expandafter
         \ifx\csname processbatchFile\endcsname\relax\else1\fi
         \ifx\fmtname\x\else 1\fi\relax
\else\csname fi\endcsname
%</ignore>
%<*install>
\input docstrip.tex
\Msg{************************************************************************}
\Msg{* Installation}
\Msg{* Package: luatex 2010/03/09 v0.4 LuaTeX basic definition package (HO)}
\Msg{************************************************************************}

\keepsilent
\askforoverwritefalse

\let\MetaPrefix\relax
\preamble

This is a generated file.

Project: luatex
Version: 2010/03/09 v0.4

Copyright (C) 2007, 2009, 2010 by
   Heiko Oberdiek <heiko.oberdiek at googlemail.com>

This work may be distributed and/or modified under the
conditions of the LaTeX Project Public License, either
version 1.3c of this license or (at your option) any later
version. This version of this license is in
   http://www.latex-project.org/lppl/lppl-1-3c.txt
and the latest version of this license is in
   http://www.latex-project.org/lppl.txt
and version 1.3 or later is part of all distributions of
LaTeX version 2005/12/01 or later.

This work has the LPPL maintenance status "maintained".

This Current Maintainer of this work is Heiko Oberdiek.

The Base Interpreter refers to any `TeX-Format',
because some files are installed in TDS:tex/generic//.

This work consists of the main source file luatex.dtx
and the derived files
   luatex.sty, luatex.pdf, luatex.ins, luatex.drv, luatex-loader.sty,
   luatex-test1.tex, luatex-test2.tex, luatex-test3.tex,
   luatex-test4.tex, luatex-test5.tex, oberdiek.luatex.lua.

\endpreamble
\let\MetaPrefix\DoubleperCent

\generate{%
  \file{luatex.ins}{\from{luatex.dtx}{install}}%
  \file{luatex.drv}{\from{luatex.dtx}{driver}}%
  \usedir{tex/generic/oberdiek}%
  \file{luatex.sty}{\from{luatex.dtx}{package}}%
  \file{luatex-loader.sty}{\from{luatex.dtx}{loader}}%
  \usedir{doc/latex/oberdiek/test}%
  \file{luatex-test1.tex}{\from{luatex.dtx}{test1}}%
  \file{luatex-test2.tex}{\from{luatex.dtx}{test1,test2}}%
  \file{luatex-test3.tex}{\from{luatex.dtx}{test1,test3}}%
  \file{luatex-test4.tex}{\from{luatex.dtx}{test4}}%
  \file{luatex-test5.tex}{\from{luatex.dtx}{test5}}%
  \nopreamble
  \nopostamble
  \usedir{source/latex/oberdiek/catalogue}%
  \file{luatex.xml}{\from{luatex.dtx}{catalogue}}%
}
\def\MetaPrefix{-- }
\def\defaultpostamble{%
  \MetaPrefix^^J%
  \MetaPrefix\space End of File `\outFileName'.%
}
\def\currentpostamble{\defaultpostamble}%
\generate{%
  \usedir{scripts/oberdiek}%
  \file{oberdiek.luatex.lua}{\from{luatex.dtx}{lua}}%
}

\catcode32=13\relax% active space
\let =\space%
\Msg{************************************************************************}
\Msg{*}
\Msg{* To finish the installation you have to move the following}
\Msg{* files into a directory searched by TeX:}
\Msg{*}
\Msg{*     luatex.sty, luatex-loader.sty}
\Msg{*}
\Msg{* And install the following script file:}
\Msg{*}
\Msg{*     oberdiek.luatex.lua}
\Msg{*}
\Msg{* To produce the documentation run the file `luatex.drv'}
\Msg{* through LaTeX.}
\Msg{*}
\Msg{* Happy TeXing!}
\Msg{*}
\Msg{************************************************************************}

\endbatchfile
%</install>
%<*ignore>
\fi
%</ignore>
%<*driver>
\NeedsTeXFormat{LaTeX2e}
\ProvidesFile{luatex.drv}%
  [2010/03/09 v0.4 LuaTeX basic definition package (HO)]%
\documentclass{ltxdoc}
\usepackage{holtxdoc}[2011/11/22]
\usepackage{paralist}
\begin{document}
  \DocInput{luatex.dtx}%
\end{document}
%</driver>
% \fi
%
% \CheckSum{1429}
%
% \CharacterTable
%  {Upper-case    \A\B\C\D\E\F\G\H\I\J\K\L\M\N\O\P\Q\R\S\T\U\V\W\X\Y\Z
%   Lower-case    \a\b\c\d\e\f\g\h\i\j\k\l\m\n\o\p\q\r\s\t\u\v\w\x\y\z
%   Digits        \0\1\2\3\4\5\6\7\8\9
%   Exclamation   \!     Double quote  \"     Hash (number) \#
%   Dollar        \$     Percent       \%     Ampersand     \&
%   Acute accent  \'     Left paren    \(     Right paren   \)
%   Asterisk      \*     Plus          \+     Comma         \,
%   Minus         \-     Point         \.     Solidus       \/
%   Colon         \:     Semicolon     \;     Less than     \<
%   Equals        \=     Greater than  \>     Question mark \?
%   Commercial at \@     Left bracket  \[     Backslash     \\
%   Right bracket \]     Circumflex    \^     Underscore    \_
%   Grave accent  \`     Left brace    \{     Vertical bar  \|
%   Right brace   \}     Tilde         \~}
%
% \GetFileInfo{luatex.drv}
%
% \title{The \xpackage{luatex} package}
% \date{2010/03/09 v0.4}
% \author{Heiko Oberdiek\\\xemail{heiko.oberdiek at googlemail.com}}
%
% \maketitle
%
% \begin{abstract}
% This package manages the new and extended features and resources
% that \LuaTeX\ provides. Examples are attributes and catcode
% tables.
% \end{abstract}
%
% \tableofcontents
%
% \section{Documentation}
%
% \subsection{Introduction}
%
% \TeX\ provides global resources such as registers.
% But it does not provide an interface for managing these resources.
% For example, two packages want to use a counter register.
% If they take the same register
% number, then the use of both packages will conflict and they cannot
% be used together.
% Therefore formats such as \plainTeX\ or \LaTeX\ implement
% an allocation scheme for registers. A package reserves with
% \cs{newcount} an unused register number for its own exclusive use.
%
% Nowadays \TeX\ is not alone anymore: \eTeX, \pdfTeX\ and other
% compilers for \TeX\ are developed that extend and add new
% features and resources.
%
% Now \LuaTeX\ has reached beta state. It inherits most of
% \pdfTeX's features including \eTeX. Also it implements
% new concepts such as attributes or catcode tables.
%
% \subsubsection{\LaTeX}
%
% \LaTeXe\ is frozen and therefore refuses to even notice
% the new \TeX\ variants. Not even the old \eTeX\ is supported
% by its kernel. At least there is a third party package \xpackage{etex}
% that manages the new \eTeX\ resources.
%
% This package tries to do the same for \LuaTeX\ and starts to
% support at least a few of the new features.
%
% \subsubsection{\plainTeX}
%
% \LaTeX\ has inherited its resource handling from \plainTeX.
% The interface is basically the same: \cs{newcount}, \dots
% Therefore this package tries to follow this tradition by
% providing compatibility to \plainTeX. It can be loaded
% with \plainTeX\ and defines at least some of the
% features that this packages provides for \LaTeX.
%
% \subsection{Register allocation}
%
% \subsubsection{Register with 16 bit}
%
% Because \LuaTeX\ is a super set of \eTeX\ regarding registers,
% the register allocation scheme should not conflict with
% package \xpackage{etex}. Therefore this package is loaded
% to inherit its allocation scheme. The only change is currently
% that the limit is increased to 65536 registers for the following
% register classes:
% \begin{compactitem}
% \item count
% \item dimen
% \item skip
% \item muskip
% \item marks
% \item toks
% \item box
% \end{compactitem}
% This affects the number of global and local registers.
% Because it is done in a package and not in the kernel,
% it is possible that someone loads package \xpackage{etex}
% before uses the local allocation variants. This will prevent
% the extension for this register class. If more registers are
% needed, just load package \xpackage{luatex} earlier.
%
% \subsubsection{Insertions}
%
% Insertions need four registers \cs{count}, \cs{dimen}, \cs{skip},
% and \cs{box} with the same number. Usually they are allocated
% downwards from 254, 253, \dots Also \cs{newcount}, \cs{newdimen}, \dots
% fill up these register numbers from below before switching to
% higher register numbers by package \xpackage{etex}.
% When this occurs, no insertions can be allocated anymore.
%
% Therefore \cs{newcount}, \cs{newdimen}, \cs{newskip}, and
% \cs{newbox} are replaced by their global variants (\cs{globcount}, \dots)
% that use the higher numbers immediately, leaving the room for insertions.
% There should not be an efficiency penalty because \LuaTeX\ stores
% the registers of a class in the same Lua table unlike \eTeX, where
% registers below 256 are stored in an array and higher numbers
% are put in a tree structure.
%
% \subsection{Attributes}
%
% Nodes can have custom attributes in \LuaTeX. These attributes
% are organized by a new register class. As the other registers
% up to $2^{16}$ attributes are supported. An attribute value
% can be negative that means the attribute is not set. Otherwise
% \TeX's range of non-negative integers up to $2^31$ are available.
%
% \begin{declcs}{newattribute} \M{cmd}
% \end{declcs}
% Macro \cs{newattribute} defines command \meta{cmd} using \cs{attributedef}
% using an new attribute number. The new attribute is initially unset.
%
% \begin{declcs}{setattribute} \M{cmd} \M{value}
% \end{declcs}
% Macro \cs{setattribute} locally sets attribute command \meta{cmd}
% to the number \meta{value}. Valid values range from $-1$ until $2^31$
% (the upper limit is the same as for other \TeX\ integer numbers).
%
% \begin{declcs}{unsetattribute} \M{cmd}
% \end{declcs}
% Macro \cs{unsettattribute} clears the attribute command \meta{cmd}.
%
% \subsection{Catcode tables}
%
% \LuaTeX\ introduces catcode tables as new feature, see documentation.
% There is need for discussion, how to deal best:
% \begin{itemize}
% \item \cs{initcatcodetable} and \cs{setcatcodetable} act globally.
% \item \cs{catcodetable} causes an error if used with an uninitialized
%   catcode table.
% \item Large catcode table numbers should be avoided because of
%   performance breakdown.
% \item Use case \LaTeX\ package:
%   The package must not be surprised by changed catcodes and
%   must not surprise by changing catcodes accidently.
%   Catcode tables could offer a solution. At the begin
%   a catcode regime with standard catcodes is established
%   and the old one is restored afterwards.
% \item Use case: \LuaTeX's |tex.print| might be used with
%   a catcode table number, for example a table where all
%   entries have catcode ``other''.
% \item Readonly catcode tables.
% \item Is there is a need for local allocations?
%   (Package \xpackage{etex}'s \cs{loc} variants are not used in
%   \TeX\ Live 2007.)
% \end{itemize}
%
% \subsubsection{Interface proposal}
%
% The idea: \cs{newcatcodetable} allocates odd numbered catcode
% tables. Even numbered tables are managed as stack.
% Also some catcode tables are defined. These must not be
% changed.
%
% \begin{declcs}{newcatcodetable} \M{cmd}
% \end{declcs}
% Macro \cs{newcatcodetable} reserves a new catcode table and
% remembers its number in \meta{cmd}. The catcode table is
% initialized with \iniTeX's catcodes.
%
% \begin{declcs}{CatcodeTableIniTeX}\\
%   \cs{CatcodeTableString}\\
%   \cs{CatcodeTableOther}\\
%   \cs{CatcodeTableLaTeX}\\
% \end{declcs}
% These are catcode tables and must not be changed.
% \cs{CatcodeTableIniTeX} contains the catcode settings of \iniTeX.
% \cs{CatcodeTableString} follows \TeX's convention of
% \cs{string}, \cs{meaning} and friends. The space gets catcode
% 10 (space), the other characters have catcode 12 (other).
% In \cs{CatcodeTableOther} all entries have catcode 12 (other).
% \cs{CatcodeTableLaTeX} contains the setting of a pure \LaTeX\ format
% (`at' is other).
%
% \begin{declcs}{CatcodeTableStack}\\
%   \cs{IncCatcodeTableStack}\\
%   \cs{DecCatcodeTableStack}
% \end{declcs}
% \cs{CatcodeTableStack} is the stack pointer. Initially
% it is catcode table zero. \cs{IncCatcodeTableStack} and
% \cs{DecCatcodeTableStack} increments and decrements the
% stack pointer. Currently \cs{IncCatcodeTableStack} does not
% initialize a new catcode table. Both increment and decrement
% operations do not set a catcode table.
%
% \begin{declcs}{PushCatcodeTableNumStack}\\
%   \cs{PopCatcodeTableNumStack}
% \end{declcs}
% It can be handy to have a global stack for catcode table
% numbers to deal with the global assignment property of
% \cs{initcatcodetable} and \cs{savecatcodetable}.
% \cs{PushCatcodeTableNumStack} pushes the current catcode table
% on the stack.
% \cs{PopCatcodeTableNumStack} pops the topmost number off the number
% stack to set the current catcode table. Catcode table zero is
% used in case of an empty stack.
%
% \begin{declcs}{BeginCatcodeRegime} \M{catcodetable}\\
%   \cs{EndCatcodeRegime}
% \end{declcs}
% \cs{BeginCatcodeRegime} remembers the current catcode table number.
% Then it creates and uses a fresh catcode table on the stack
% that is initialized by \meta{catcodetable}:
% \begin{quote}\ttfamily
%   \cs{PushCatcodeTableNumStack}\\
%   \cs{catcodetable}\meta{catcodetable}||
%   \cs{IncCatcodeTableStack}\\
%   \cs{savecatcodetable}\cs{CatcodeTableStack}\\
%   \cs{catcodetable}\cs{CatcodeTableStack}
% \end{quote}
% \cs{EndCatcodeRegime} drops the catcode table, created by
% \cs{BeginCatcodeRegime} and sets the catcode table that was
% active before:
% \begin{quote}
%   \cs{DecCatcodeTableStack}\\
%   \cs{PopCatcodeTableNumStack}
% \end{quote}
% These macros solve the use case, described earlier for a \LaTeX\ package:
% \begin{quote}
%   |% package foobar.sty|\\
%   |\BeginCatcodeRegime\CatcodeTableLaTeX|\\
%   |\makeatletter|\\
%   |% ... package contents ...|\\
%   |\EndCatcodeRegime|\\
%   |% end of package|
% \end{quote}
% If the package wants to change catcodes after its loading,
% \cs{AtBeginDocument} or \cs{AtEndOfPackage} can be used.
%
% \begin{declcs}{SetCatcodeRange} \M{from} \M{to} \M{catcode}
% \end{declcs}
% The catcodes of characters in range from \meta{from}
% to inclusive \meta{to} are set to \meta{catcode}.
%
% \subsection{Lua module loading}
%
% Currently \LuaTeX\ (version 0.20) does not support Lua script
% files inside \xfile{TDS:scripts//}, because Lua's mechanism
% for module loading does not use the \textsf{kpathsea} library.
% Therefore this packages appends a kpse loader to the list of
% Lua's module loaders. It finds the module \meta{module} by
% \begin{quote}
%   \ttfamily |kpse.find_file("|\meta{module}|.lua", "texmfscripts")|
% \end{quote}
% Unhappily \textsf{kpathsea} does not support directory components
% in a file name. Therefore the Lua convention is not followed to
% replace dots in the module name by the directory separator.
%
% Example:
% A Lua script of a package \xpackage{foobar} wants the following
% modules:
% \begin{quote}
%   |require("foobar.hello.world")|\\
%   |require("org.somewhere.xyz")|
% \end{quote}
% Then they can be find in:
% \begin{quote}
%   |TDS:scripts/foobar/foobar.hello.world.lua|\\
%   |TDS:scripts/foobar/org.somewhere.xyz.lua|
% \end{quote}
% I would have preferred the following locations, following lua conventions,
% e.\,g.:
% \begin{quote}
%   |TDS:scripts/foobar/hello/world.lua|\\
%   |TDS:scripts/foobar/org/somewhere/xyz.lua|
% \end{quote}
% But I do not know, how to achieve this in a reliable way using
% \textsf{kpathsea}.
%
% \subsubsection{Package \xpackage{luatex-loader}}
%
% If someone do not need or want package \xpackage{luatex} but
% it's extension for module loading, then he can use
% package \xpackage{luatex-loader}. Both \plainTeX\ and \LaTeX\ are
% supported.
%
% \StopEventually{
% }
%
% \section{Implementation}
%
%    \begin{macrocode}
%<*package>
%    \end{macrocode}
%
% \subsection{Reload check and package identification}
%    Reload check, especially if the package is not used with \LaTeX.
%    \begin{macrocode}
\begingroup\catcode61\catcode48\catcode32=10\relax%
  \catcode13=5 % ^^M
  \endlinechar=13 %
  \catcode35=6 % #
  \catcode39=12 % '
  \catcode44=12 % ,
  \catcode45=12 % -
  \catcode46=12 % .
  \catcode58=12 % :
  \catcode64=11 % @
  \catcode123=1 % {
  \catcode125=2 % }
  \expandafter\let\expandafter\x\csname ver@luatex.sty\endcsname
  \ifx\x\relax % plain-TeX, first loading
  \else
    \def\empty{}%
    \ifx\x\empty % LaTeX, first loading,
      % variable is initialized, but \ProvidesPackage not yet seen
    \else
      \expandafter\ifx\csname PackageInfo\endcsname\relax
        \def\x#1#2{%
          \immediate\write-1{Package #1 Info: #2.}%
        }%
      \else
        \def\x#1#2{\PackageInfo{#1}{#2, stopped}}%
      \fi
      \x{luatex}{The package is already loaded}%
      \aftergroup\endinput
    \fi
  \fi
\endgroup%
%    \end{macrocode}
%    Package identification:
%    \begin{macrocode}
\begingroup\catcode61\catcode48\catcode32=10\relax%
  \catcode13=5 % ^^M
  \endlinechar=13 %
  \catcode35=6 % #
  \catcode39=12 % '
  \catcode40=12 % (
  \catcode41=12 % )
  \catcode44=12 % ,
  \catcode45=12 % -
  \catcode46=12 % .
  \catcode47=12 % /
  \catcode58=12 % :
  \catcode64=11 % @
  \catcode91=12 % [
  \catcode93=12 % ]
  \catcode123=1 % {
  \catcode125=2 % }
  \expandafter\ifx\csname ProvidesPackage\endcsname\relax
    \def\x#1#2#3[#4]{\endgroup
      \immediate\write-1{Package: #3 #4}%
      \xdef#1{#4}%
    }%
  \else
    \def\x#1#2[#3]{\endgroup
      #2[{#3}]%
      \ifx#1\@undefined
        \xdef#1{#3}%
      \fi
      \ifx#1\relax
        \xdef#1{#3}%
      \fi
    }%
  \fi
\expandafter\x\csname ver@luatex.sty\endcsname
\ProvidesPackage{luatex}%
  [2010/03/09 v0.4 LuaTeX basic definition package (HO)]%
%    \end{macrocode}
%
% \subsection{Catcodes}
%
%    \begin{macrocode}
\begingroup\catcode61\catcode48\catcode32=10\relax%
  \catcode13=5 % ^^M
  \endlinechar=13 %
  \catcode123=1 % {
  \catcode125=2 % }
  \catcode64=11 % @
  \def\x{\endgroup
    \expandafter\edef\csname LuT@AtEnd\endcsname{%
      \endlinechar=\the\endlinechar\relax
      \catcode13=\the\catcode13\relax
      \catcode32=\the\catcode32\relax
      \catcode35=\the\catcode35\relax
      \catcode61=\the\catcode61\relax
      \catcode64=\the\catcode64\relax
      \catcode123=\the\catcode123\relax
      \catcode125=\the\catcode125\relax
    }%
  }%
\x\catcode61\catcode48\catcode32=10\relax%
\catcode13=5 % ^^M
\endlinechar=13 %
\catcode35=6 % #
\catcode64=11 % @
\catcode123=1 % {
\catcode125=2 % }
\def\TMP@EnsureCode#1#2{%
  \edef\LuT@AtEnd{%
    \LuT@AtEnd
    \catcode#1=\the\catcode#1\relax
  }%
  \catcode#1=#2\relax
}
\TMP@EnsureCode{10}{12}% ^^J
\TMP@EnsureCode{34}{12}% "
\TMP@EnsureCode{36}{3}% $
\TMP@EnsureCode{39}{12}% '
\TMP@EnsureCode{40}{12}% (
\TMP@EnsureCode{41}{12}% )
\TMP@EnsureCode{42}{12}% *
\TMP@EnsureCode{43}{12}% +
\TMP@EnsureCode{44}{12}% ,
\TMP@EnsureCode{45}{12}% -
\TMP@EnsureCode{46}{12}% .
\TMP@EnsureCode{47}{12}% /
\TMP@EnsureCode{60}{12}% <
\TMP@EnsureCode{62}{12}% >
\TMP@EnsureCode{91}{12}% [
\TMP@EnsureCode{93}{12}% ]
\TMP@EnsureCode{95}{12}% _ (other!)
\TMP@EnsureCode{96}{12}% `
\edef\LuT@AtEnd{\LuT@AtEnd\noexpand\endinput}
%    \end{macrocode}
%
% \subsection{Check for \LuaTeX}
%
%    Without \LuaTeX\ there is no point in using this package.
%    \begin{macrocode}
\begingroup\expandafter\expandafter\expandafter\endgroup
\expandafter\ifx\csname RequirePackage\endcsname\relax
  \input infwarerr.sty\relax
  \input ifluatex.sty\relax
\else
  \RequirePackage{infwarerr}[2007/09/09]%
  \RequirePackage{ifluatex}[2009/04/10]%
\fi
%    \end{macrocode}
%    \begin{macrocode}
\ifluatex
\else
  \@PackageError{luatex}{%
    This package may only be run using LuaTeX%
  }\@ehc
  \expandafter\LuT@AtEnd
\fi%
%    \end{macrocode}
%
% \subsection{Provide Lua\TeX\ primitives}
%
%    \begin{macrocode}
\ifnum\luatexversion<36 %
  \def\LuT@MakePrimitive#1{%
    \expandafter\let\csname luatex#1\expandafter\endcsname
    \csname #1\endcsname
  }%
\else
  \def\LuT@MakeLuatexPrimitive#1{%
    \begingroup\expandafter\expandafter\expandafter\endgroup
    \expandafter\ifx\csname luatex#1\endcsname\relax
      \begingroup\expandafter\expandafter\expandafter\endgroup
      \expandafter\ifx\csname #1\endcsname\relax
      \else
        \expandafter\let
        \csname luatex#1\expandafter\endcsname
        \csname #1\endcsname
      \fi
    \fi
    \begingroup\expandafter\expandafter\expandafter\endgroup
    \expandafter\ifx\csname luatex#1\endcsname\relax
      \begingroup
        \expandafter\let\csname luatex#1\endcsname\@undefined
        \ifnum0%
          \directlua{%
            if tex.enableprimitives then %
              tex.enableprimitives('luatex',{'#1'})%
              tex.print('1')%
            end%
          }%
          \expandafter\ifx\csname luatex#1\endcsname\relax\else1\fi
        =11 %
          \global\expandafter\let
          \csname luatex#1\expandafter\endcsname
          \csname luatex#1\endcsname
        \else
          \@PackageError{luatex}{%
            tex.enableprimitives failed for `#1'%
          }\@ehc
        \fi
      \endgroup
    \fi
  }%
  \def\LuT@MakePrimitive#1{%
    \begingroup\expandafter\expandafter\expandafter\endgroup
    \expandafter\ifx\csname#1\endcsname\relax
      \begingroup
        \expandafter\let\csname#1\endcsname\@undefined
        \ifnum0%
          \directlua{%
            if tex.enableprimitives then %
              tex.enableprimitives('',{'#1'})%
              tex.print('1')%
            end%
          }%
          \expandafter\ifx\csname#1\endcsname\relax\else1\fi
        =11 %
          \global\expandafter\let
          \csname#1\expandafter\endcsname
          \csname#1\endcsname
        \else
          \@PackageError{luatex}{%
            tex.enableprimitives failed for `#1'%
          }\@ehc
        \fi
      \endgroup
    \fi
  }%
\fi
\LuT@MakeLuatexPrimitive{attribute}
\LuT@MakeLuatexPrimitive{attributedef}
\LuT@MakeLuatexPrimitive{catcodetable}
\LuT@MakeLuatexPrimitive{initcatcodetable}
\LuT@MakeLuatexPrimitive{luaescapestring}
\LuT@MakeLuatexPrimitive{savecatcodetable}
\LuT@MakePrimitive{numexpr}
%    \end{macrocode}
%
% \subsection{Inherit support for \eTeX}
%
%    Package \xpackage{etex} is not compatible for \plainTeX.
%    But it could be present if a format is used that is based
%    on \xfile{etex.src}. Therefore we only load the package
%    in case of \LaTeX\ and tests its presence independently
%    of the format by looking for \cs{et@xins}.
%    \begin{macrocode}
\begingroup\expandafter\expandafter\expandafter\endgroup
\expandafter\ifx\csname RequirePackage\endcsname\relax
\else
  \RequirePackage{etex}[1998/03/26]%
\fi
%    \end{macrocode}
%
% \subsection{Adaption of \eTeX's register allocation}
%
%    \eTeX has increased the number of \TeX\ registers
%    from $2^8$ (256) to $2^{15}$ (32768) for a register class.
%    \LuaTeX\ extends the limit further to $2^16$ (65536).
%    The allocation scheme of package \xpackage{etex} is not
%    changed. But this can be subject for discussion.
%
%    If a register class hasn't registered any local registers
%    yet, then the limit can safely be pushed to 65536.
%    \begin{macrocode}
\begingroup\expandafter\expandafter\expandafter\endgroup
\expandafter\ifx\csname et@xins\endcsname\relax
  \@PackageWarningNoLine{luatex}{%
    Support for eTeX is not loaded (etex.src)%
  }%
\else
  \def\LuT@temp#1{%
    \ifnum\count27#1=32768 %
      \count27#1=65536 %
    \fi
  }%
  \LuT@temp0%
  \LuT@temp1%
  \LuT@temp2%
  \LuT@temp3%
  \LuT@temp4%
  \LuT@temp5%
  \LuT@temp6%
%    \end{macrocode}
%
% \eTeX\ uses an array for the first 256 registers and then a tree
% structure. \LuaTeX\ stores all registers of a class in one Lua table.
% There shouldn't be large performance differences.
% This allows starting immediately in the extended
% area, leaving room for insertions.
%    \begin{macrocode}
  \let\newcount\globcount
  \let\newdimen\globdimen
  \let\newskip\globskip
  \let\newbox\globbox
\fi
%    \end{macrocode}
%
% \subsection{\plainTeX\ compatibility}
%
%    \begin{macro}{\@empty}
%    \begin{macrocode}
\expandafter\ifx\csname @empty\endcsname\relax
  \def\@empty{}%
\fi
%    \end{macrocode}
%    \end{macro}
%    \begin{macro}{\@gobble}
%    \begin{macrocode}
\expandafter\ifx\csname @gobble\endcsname\relax
  \long\def\@gobble#1{}%
\fi
%    \end{macrocode}
%    \end{macro}
%    \begin{macro}{\@firstofone}
%    \begin{macrocode}
\expandafter\ifx\csname @firstofone\endcsname\relax
  \long\def\@firstofone#1{#1}%
\fi
%    \end{macrocode}
%    \end{macro}
%    \begin{macro}{\@firstoftwo}
%    \begin{macrocode}
\expandafter\ifx\csname @firstoftwo\endcsname\relax
  \long\def\@firstoftwo#1#2{#1}%
\fi
%    \end{macrocode}
%    \end{macro}
%    \begin{macro}{\@car}
%    \begin{macrocode}
\expandafter\ifx\csname @car\endcsname\relax
  \def\@car#1#2\@nil{#1}%
\fi
%    \end{macrocode}
%    \end{macro}
%    \begin{macro}{\@cdr}
%    \begin{macrocode}
\expandafter\ifx\csname @cdr\endcsname\relax
  \def\@cdr#1#2\@nil{#2}%
\fi
%    \end{macrocode}
%    \end{macro}
%    \begin{macro}{\@ifstar}
%    \begin{macrocode}
\expandafter\ifx\csname @ifstar\endcsname\relax
  \def\@ifstar#1{%
    \@ifnextchar*{\@firstoftwo{#1}}%
  }%
%    \end{macrocode}
%    \end{macro}
%    \begin{macro}{\@ifnextchar}
%    \begin{macrocode}
  \long\def\@ifnextchar#1#2#3{%
    \let\reserved@d=#1%
    \def\reserved@a{#2}%
    \def\reserved@b{#3}%
    \futurelet\@let@token\@ifnch
  }%
%    \end{macrocode}
%    \end{macro}
%    \begin{macro}{\@ifnch}
%    \begin{macrocode}
  \def\@ifnch{%
    \ifx\@let@token\@sptoken
      \let\reserved@c\@xifnch
    \else
      \ifx\@let@token\reserved@d
        \let\reserved@c\reserved@a
      \else
        \let\reserved@c\reserved@b
      \fi
    \fi
    \reserved@c
  }%
%    \end{macrocode}
%    \end{macro}
%    \begin{macro}{\@sptoken}
%    \begin{macrocode}
  \let\LuT@temp\:%
  \def\:{\let\@sptoken= }%
  \: % explicit space
%    \end{macrocode}
%    \end{macro}
%    \begin{macro}{\@xifnch}
%    \begin{macrocode}
  \def\:{\@xifnch}%
  \expandafter\def\: {%
    \futurelet\@let@token\@ifnch
  }%
  \let\:\LuT@temp
\fi
%    \end{macrocode}
%    \end{macro}
%
%    \begin{macro}{\@tempcnta}
%    \begin{macrocode}
\expandafter\ifx\csname @tempcnta\endcsname\relax
  \csname newcount\endcsname\@tempcnta
\fi
%    \end{macrocode}
%    \end{macro}
%
%    \begin{macro}{\@tempcntb}
%    \begin{macrocode}
\expandafter\ifx\csname @tempcntb\endcsname\relax
  \csname newcount\endcsname\@tempcntb
\fi
%    \end{macrocode}
%    \end{macro}
%
%    \begin{macro}{\LuT@newcommand}
%    \begin{macrocode}
\begingroup\expandafter\expandafter\expandafter\endgroup
\expandafter\ifx\csname newcommand\endcsname\relax
  \def\LuT@newcommand#1[#2]#3{%
    \ifx#1\@undefined
      \let#1\relax
    \else
      \ifx#1\relax
      \else
        \@PackageError{luatex}{%
          \string#1 is already defined.\MessageBreak
          Redefinition is skipped%
        }\@ehc
      \fi
    \fi
    \ifx#1\relax
      \ifcase#2 %
        \def#1{#3}%
      \or
        \def#1##1{#3}%
      \or
        \def#1##1##2{#3}%
      \or
        \def#1##1##2##3{#3}%
      \or
        \@INTERNAL@ERROR
      \fi
    \fi
  }%
\else
  \def\LuT@newcommand{\newcommand*}%
\fi
%    \end{macrocode}
%    \end{macro}
%
% \subsection{Attributes}
%
% \subsubsection{Allocation}
%
%    \begin{macro}{\LuT@AllocAttribute}
%    \begin{macrocode}
\newcount\LuT@AllocAttribute
\LuT@AllocAttribute=\m@ne
%    \end{macrocode}
%    \end{macro}
%    \begin{macro}{\newattribute}
%    \begin{macrocode}
\LuT@newcommand\newattribute[1]{%
  \ifnum\LuT@AllocAttribute<65535 %
    \global\advance\LuT@AllocAttribute\@ne
    \allocationnumber\LuT@AllocAttribute
    \global\luatexattributedef#1=\allocationnumber
    \unsetattribute{#1}%
    \wlog{\string#1=\string\attribute\the\allocationnumber}%
  \else
    \errmessage{No room for a new \string\attribute}%
  \fi
}
%    \end{macrocode}
%    \end{macro}
%
% \subsubsection{Interface}
%
%    \begin{macro}{\setattribute}
%    \begin{macrocode}
\LuT@newcommand\setattribute[2]{%
  #1=\numexpr#2\relax
}
%    \end{macrocode}
%    \end{macro}
%    \begin{macro}{\unsetattribute}
%    \begin{macrocode}
\ifnum\luatexversion<37
  \LuT@newcommand\LuT@UnsetAttributeValue[0]{}%
  \let\LuT@UnsetAttributeValue\m@ne
\else
  \LuT@newcommand\LuT@UnsetAttributeValue[0]{-2147483647 }%
\fi
\LuT@newcommand\unsetattribute[1]{%
  #1=\LuT@UnsetAttributeValue
}
%    \end{macrocode}
%    \end{macro}
%
% \subsection{Catcode tables}
%
% \subsubsection{Allocation}
%
%    \begin{macro}{\LuT@AllocCatcodeTable}
%    \begin{macrocode}
\newcount\LuT@AllocCatcodeTable
\LuT@AllocCatcodeTable=\m@ne
\newcount\CatcodeTableStack
\CatcodeTableStack=\z@
%    \end{macrocode}
%    \end{macro}
%    \begin{macro}{\newcatcodetable}
%    \begin{macrocode}
\LuT@newcommand\newcatcodetable[1]{%
  \ifnum\LuT@AllocCatcodeTable<1114110 % 0x10FFFF is maximal \chardef
    % or < 268435455 % 2^28 - 1
    \global\advance\LuT@AllocCatcodeTable by\tw@
    \allocationnumber=\LuT@AllocCatcodeTable
    \global\chardef#1=\allocationnumber
    \wlog{%
      \string#1=\string\catcodetable\the\allocationnumber
    }%
  \else
    \errmessage{No room for a new \string\catcodetable}%
  \fi
}%
%    \end{macrocode}
%    \end{macro}
%
%    \begin{macro}{\IncCatcodeTableStack}
%    \begin{macrocode}
\LuT@newcommand\IncCatcodeTableStack[0]{%
  \ifnum\CatcodeTableStack<268435454 %
    \global\advance\CatcodeTableStack by\tw@
  \else
    \@PackageError{luatex}{%
      Catcode table stack overflow%
    }\@ehd
  \fi
}
%    \end{macrocode}
%    \end{macro}
%
%    \begin{macro}{\DecCatcodeTableStack}
%    \begin{macrocode}
\LuT@newcommand\DecCatcodeTableStack[0]{%
  \ifnum\CatcodeTableStack>\z@
    \global\advance\CatcodeTableStack by-2 %
  \else
    \@PackageError{luatex}{%
      Catcode table stack is empty%
    }\@ehd
  \fi
}
%    \end{macrocode}
%    \end{macro}
%
% \subsubsection{\cs{SetCatcodeRange}}
%
%    \begin{macro}{\SetCatcodeRange}
%    \begin{macrocode}
\LuT@newcommand\SetCatcodeRange[3]{%
  \edef\LuT@temp{%
    \noexpand\@tempcnta=\the\@tempcnta
    \noexpand\@tempcntb=\the\@tempcntb
    \noexpand\count@=\the\count@
    \relax
  }%
  \@tempcnta=\numexpr#1\relax
  \@tempcntb=\numexpr#2\relax
  \count@=\numexpr#3\relax
  \loop
    \unless\ifnum\@tempcnta>\@tempcntb
    \catcode\@tempcnta=\count@
    \advance\@tempcnta by \@ne
  \repeat
  \LuT@temp
}
%    \end{macrocode}
%    \end{macro}
%
% \subsubsection{Predefined catcode tables}
%
%    \begin{macrocode}
\newcatcodetable\CatcodeTableIniTeX
\newcatcodetable\CatcodeTableString
\newcatcodetable\CatcodeTableOther
\newcatcodetable\CatcodeTableLaTeX
%    \end{macrocode}
%
%    \begin{macrocode}
\luatexinitcatcodetable\CatcodeTableIniTeX
\begingroup
  \def\@makeother#1{\catcode#1=12\relax}%
  \@firstofone{%
    \luatexcatcodetable\CatcodeTableIniTeX
    \begingroup
      \SetCatcodeRange{0}{8}{15}%
      \catcode9=10 % tab
      \catcode11=15 %
      \catcode12=13 % form feed
      \SetCatcodeRange{14}{31}{15}%
      \catcode35=6 % hash
      \catcode36=3 % dollar
      \catcode38=4 % ampersand
      \catcode94=7 % circumflex
      \catcode95=8 % underscore
      \catcode123=1 % brace left
      \catcode125=2 % brace right
      \catcode126=13 % tilde
      \catcode127=15 %
      \luatexsavecatcodetable\CatcodeTableLaTeX
    \endgroup
    \@makeother{0}% nul
    \@makeother{13}% carriage return
    \@makeother{37}% percent
    \@makeother{92}% backslash
    \@makeother{127}%
    \SetCatcodeRange{65}{90}{12}% A-Z
    \SetCatcodeRange{97}{122}{12}% a-z
    \luatexsavecatcodetable\CatcodeTableString
    \@makeother{32}% space
    \luatexsavecatcodetable\CatcodeTableOther
  \endgroup
}%
%    \end{macrocode}
%
% \subsubsection{Number stack}
%
%    \begin{macro}{\LuT@NumStackEmpty}
%    A special empty stack value because of \cs{@cdr}'s brace removal.
%    \begin{macrocode}
\def\LuT@NumStackEmpty{0}
%    \end{macrocode}
%    \end{macro}
%    \begin{macro}{\LuT@NumStack}
%    \begin{macrocode}
\let\LuT@NumStack\LuT@NumStackEmpty
%    \end{macrocode}
%    \end{macro}
%    \begin{macro}{\PushCatcodeTableNumStack}
%    \begin{macrocode}
\LuT@newcommand\PushCatcodeTableNumStack[0]{%
  \xdef\LuT@NumStack{%
    {\the\luatexcatcodetable}\LuT@NumStack
  }%
}
%    \end{macrocode}
%    \end{macro}
%    \begin{macro}{\PopCatcodeTableNumStack}
%    \begin{macrocode}
\LuT@newcommand\PopCatcodeTableNumStack[0]{%
  \ifx\LuT@NumStack\LuT@NumStackEmpty
    \@PackageWarning{luatex}{Empty catcode table number stack}%
    \luatexcatcodetable\z@
  \else
    \luatexcatcodetable=\expandafter\@car\LuT@NumStack\@nil\relax
    \xdef\LuT@NumStack{%
      \expandafter\@cdr\LuT@NumStack\@nil
    }%
  \fi
}
%    \end{macrocode}
%    \end{macro}
%
% \subsubsection{Catcode regime macros}
%
%    \begin{macro}{\BeginCatcodeRegime}
%    \begin{macrocode}
\LuT@newcommand\BeginCatcodeRegime[1]{%
  \PushCatcodeTableNumStack
  \luatexcatcodetable=\numexpr#1\relax
  \IncCatcodeTableStack
  \luatexsavecatcodetable\CatcodeTableStack
  \luatexcatcodetable\CatcodeTableStack
}
%    \end{macrocode}
%    \end{macro}
%
%    \begin{macro}{\EndCatcodeRegime}
%    \begin{macrocode}
\LuT@newcommand\EndCatcodeRegime[0]{%
  \DecCatcodeTableStack
  \PopCatcodeTableNumStack
}
%    \end{macrocode}
%    \end{macro}
%
% \subsection{Lua module loader}
%
%    \begin{macrocode}
\begingroup\expandafter\expandafter\expandafter\endgroup
\expandafter\ifx\csname RequirePackage\endcsname\relax
  \input luatex-loader.sty\relax
\else
  \RequirePackage{luatex-loader}[2010/03/09]%
\fi
%    \end{macrocode}
%    \begin{macrocode}
\LuT@AtEnd%
%</package>
%    \end{macrocode}
%
%    \begin{macrocode}
%<*loader>
%    \end{macrocode}
%
%    Reload check, especially if the package is not used with \LaTeX.
%    \begin{macrocode}
\begingroup\catcode61\catcode48\catcode32=10\relax%
  \catcode13=5 % ^^M
  \endlinechar=13 %
  \catcode35=6 % #
  \catcode39=12 % '
  \catcode44=12 % ,
  \catcode45=12 % -
  \catcode46=12 % .
  \catcode58=12 % :
  \catcode64=11 % @
  \catcode123=1 % {
  \catcode125=2 % }
  \expandafter\let\expandafter\x\csname ver@luatex-loader.sty\endcsname
  \ifx\x\relax % plain-TeX, first loading
  \else
    \def\empty{}%
    \ifx\x\empty % LaTeX, first loading,
      % variable is initialized, but \ProvidesPackage not yet seen
    \else
      \expandafter\ifx\csname PackageInfo\endcsname\relax
        \def\x#1#2{%
          \immediate\write-1{Package #1 Info: #2.}%
        }%
      \else
        \def\x#1#2{\PackageInfo{#1}{#2, stopped}}%
      \fi
      \x{luatex-loader}{The package is already loaded}%
      \aftergroup\endinput
    \fi
  \fi
\endgroup%
%    \end{macrocode}
%    Package identification:
%    \begin{macrocode}
\begingroup\catcode61\catcode48\catcode32=10\relax%
  \catcode13=5 % ^^M
  \endlinechar=13 %
  \catcode35=6 % #
  \catcode39=12 % '
  \catcode40=12 % (
  \catcode41=12 % )
  \catcode44=12 % ,
  \catcode45=12 % -
  \catcode46=12 % .
  \catcode47=12 % /
  \catcode58=12 % :
  \catcode64=11 % @
  \catcode91=12 % [
  \catcode93=12 % ]
  \catcode123=1 % {
  \catcode125=2 % }
  \expandafter\ifx\csname ProvidesPackage\endcsname\relax
    \def\x#1#2#3[#4]{\endgroup
      \immediate\write-1{Package: #3 #4}%
      \xdef#1{#4}%
    }%
  \else
    \def\x#1#2[#3]{\endgroup
      #2[{#3}]%
      \ifx#1\@undefined
        \xdef#1{#3}%
      \fi
      \ifx#1\relax
        \xdef#1{#3}%
      \fi
    }%
  \fi
\expandafter\x\csname ver@luatex-loader.sty\endcsname
\ProvidesPackage{luatex-loader}%
  [2010/03/09 v0.4 Lua module loader (HO)]%
%    \end{macrocode}
%    \begin{macrocode}
\begingroup\catcode61\catcode48\catcode32=10\relax%
  \catcode13=5\endlinechar=13\relax%
  \catcode10=12 % ^^J
  \catcode34=12 % "
  \catcode39=12 % '
  \catcode40=12 % (
  \catcode41=12 % )
  \catcode44=12 % ,
  \catcode46=12 % .
  \catcode60=12 % <
  \catcode61=12 % =
  \catcode95=12 % _ (other!)
  \catcode96=12 % `
  \catcode123=1 % {
  \catcode125=2 % }
  \endlinechar=10 %
  \ifnum\luatexversion<36 %
    \directlua0%
  \else %
    \expandafter\directlua %
  \fi %
  {%
     do
      local script = "oberdiek.luatex.lua"
      local file = kpse.find_file(script, "texmfscripts")
      if file then
        texio.write_nl("(" .. file .. ")")
        dofile(file)
      else
        error("File `" .. script .. "' not found")
      end
    end
  }%
\endgroup\endinput%
%    \end{macrocode}
%
%    \begin{macrocode}
%</loader>
%    \end{macrocode}
%
% \subsection{Lua script}
%
%    Currently \LuaTeX\ does not use KPSE when searching for module
%    files. The following Lua script implements a workaround.
%    It extends |package.loader| by another search method.
%    Modules are found by the module name with extension \xext{lua}
%    similar to
%    \begin{quote}
%      \ttfamily |kpsewhich --format=texmfscripts| \meta{module}|.lua|
%    \end{quote}
%    Unhappily \textsf{kpsewhich} does not support directory components
%    in the file name. Therefore a module |a.b.c| cannot be installed
%    as \xfile{a/b/c.lua}. The script must be named \xfile{a.b.c.lua}.
%    \begin{macrocode}
%<*lua>
%    \end{macrocode}
%    \begin{macrocode}
module("oberdiek.luatex", package.seeall)
%    \end{macrocode}
%    \begin{macrocode}
function kpse_module_loader(module)
  local script = module .. ".lua"
  local file = kpse.find_file(script, "texmfscripts")
  if file then
    local loader, error = loadfile(file)
    if loader then
      texio.write_nl("(" .. file .. ")")
      return loader
    end
    return "\n\t[oberdiek.luatex.kpse_module_loader] Loading error:\n\t"
           .. error
  end
  return "\n\t[oberdiek.luatex.kpse_module_loader] Search failed"
end
table.insert(package.loaders, kpse_module_loader)
%    \end{macrocode}
%    \begin{macrocode}
%</lua>
%    \end{macrocode}
%
% \section{Test}
%
%    \begin{macrocode}
%<*test2>
\documentclass{article}
\def\LoadCommand{%
  \RequirePackage{luatex}[2010/03/09]%
}
%</test2>
%    \end{macrocode}
%    \begin{macrocode}
%<*test3>
\documentclass{article}
\def\LoadCommand{%
  \RequirePackage{luatex-loader}[2010/03/09]%
}
%</test3>
%    \end{macrocode}
% \subsection{Catcode checks for loading}
%
%    \begin{macrocode}
%<*test1>
%    \end{macrocode}
%    \begin{macrocode}
\catcode`\{=1 %
\catcode`\}=2 %
\catcode`\#=6 %
\catcode`\@=11 %
\expandafter\ifx\csname count@\endcsname\relax
  \countdef\count@=255 %
\fi
\expandafter\ifx\csname @gobble\endcsname\relax
  \long\def\@gobble#1{}%
\fi
\expandafter\ifx\csname @firstofone\endcsname\relax
  \long\def\@firstofone#1{#1}%
\fi
\expandafter\ifx\csname loop\endcsname\relax
  \expandafter\@firstofone
\else
  \expandafter\@gobble
\fi
{%
  \def\loop#1\repeat{%
    \def\body{#1}%
    \iterate
  }%
  \def\iterate{%
    \body
      \let\next\iterate
    \else
      \let\next\relax
    \fi
    \next
  }%
  \let\repeat=\fi
}%
\def\RestoreCatcodes{}
\count@=0 %
\loop
  \edef\RestoreCatcodes{%
    \RestoreCatcodes
    \catcode\the\count@=\the\catcode\count@\relax
  }%
\ifnum\count@<255 %
  \advance\count@ 1 %
\repeat

\def\RangeCatcodeInvalid#1#2{%
  \count@=#1\relax
  \loop
    \catcode\count@=15 %
  \ifnum\count@<#2\relax
    \advance\count@ 1 %
  \repeat
}
\def\RangeCatcodeCheck#1#2#3{%
  \count@=#1\relax
  \loop
    \ifnum#3=\catcode\count@
    \else
      \errmessage{%
        Character \the\count@\space
        with wrong catcode \the\catcode\count@\space
        instead of \number#3%
      }%
    \fi
  \ifnum\count@<#2\relax
    \advance\count@ 1 %
  \repeat
}
\def\space{ }
\expandafter\ifx\csname LoadCommand\endcsname\relax
  \def\LoadCommand{\input luatex.sty\relax}%
\fi
\def\Test{%
  \RangeCatcodeInvalid{0}{47}%
  \RangeCatcodeInvalid{58}{64}%
  \RangeCatcodeInvalid{91}{96}%
  \RangeCatcodeInvalid{123}{255}%
  \catcode`\@=12 %
  \catcode`\\=0 %
  \catcode`\%=14 %
  \LoadCommand
  \RangeCatcodeCheck{0}{36}{15}%
  \RangeCatcodeCheck{37}{37}{14}%
  \RangeCatcodeCheck{38}{47}{15}%
  \RangeCatcodeCheck{48}{57}{12}%
  \RangeCatcodeCheck{58}{63}{15}%
  \RangeCatcodeCheck{64}{64}{12}%
  \RangeCatcodeCheck{65}{90}{11}%
  \RangeCatcodeCheck{91}{91}{15}%
  \RangeCatcodeCheck{92}{92}{0}%
  \RangeCatcodeCheck{93}{96}{15}%
  \RangeCatcodeCheck{97}{122}{11}%
  \RangeCatcodeCheck{123}{255}{15}%
  \RestoreCatcodes
}
\Test
\csname @@end\endcsname
\end
%    \end{macrocode}
%    \begin{macrocode}
%</test1>
%    \end{macrocode}
%
% \subsection{Catcode tables}
%
% \subsubsection{Predefined catcode tables}
%
%    \begin{macrocode}
%<*test4>
\NeedsTeXFormat{LaTeX2e}
%    \end{macrocode}
%    Remember \LaTeX's initial catcodes in count registers
%    starting at \cs{TestLaTeX}.
%    \begin{macrocode}
\count0=0 %
\chardef\TestLaTeX=1000 %
\chardef\TestMax=300 %
\loop
  \count\numexpr\TestLaTeX+\count0\relax=\catcode\count0 %
\ifnum\count0<\TestMax
  \advance\count0 by 1 %
\repeat
\documentclass{minimal}
\usepackage{luatex}[2010/03/09]
\usepackage{qstest}
\IncludeTests{*}
\LogTests{log}{*}{*}
\makeatletter
\def\Check#1{%
  \Expect*{\the\count@=\the\catcode\count@}%
         *{\the\count@=#1}%
}
\newcount\scratch
\def\Test#1#2{%
  \begin{qstest}{CatcodeTable#1}{CatcodeTable#1}%
    \luatexcatcodetable\csname CatcodeTable#1\endcsname
    \count@=\z@
    \loop
      \scratch=#2\relax
      \Expect*{\the\count@=\the\catcode\count@}%
             *{\the\count@=\the\scratch}%
    \ifnum\count@<\TestMax
      \advance\count@\@ne
    \repeat
  \end{qstest}%
}
\begingroup
  % luatex-unicode-letters.tex makes some slots to letters
  \def\TestMax{169}%
  \Test{LaTeX}{\the\count\numexpr\TestLaTeX+\count@}%
\endgroup
\Test{String}{\ifnum\count@=32 10\else 12\fi}
\Test{Other}{12}
\luatexinitcatcodetable99 %
\Test{IniTeX}{%
  0\relax
  \begingroup
    \luatexcatcodetable99 %
    \global\scratch=\the\catcode\count@
  \endgroup
}
%    \end{macrocode}
%
% \subsubsection{Catcode table number stack}
%
%    \begin{macrocode}
\begin{qstest}{CatcodeTableNumStack}{CatcodeTableNumStack}
  \def\TestStack#1{%
    \Expect*{\LuT@NumStack}{#1}%
  }%
  \TestStack{0}%
  \PushCatcodeTableNumStack
  \TestStack{{0}0}%
  \@firstofone{%
    \begingroup
      \luatexinitcatcodetable12 %
      \luatexcatcodetable12 %
      \PushCatcodeTableNumStack
      \TestStack{{12}{0}0}%
      \PopCatcodeTableNumStack
      \TestStack{{0}0}%
      \PopCatcodeTableNumStack
      \TestStack{0}%
      \def\TestWarning{Missing empty stack warning}%
      \def\@PackageWarning#1#2{\def\TestWarning{empty stack}}%
      \PopCatcodeTableNumStack
      \TestStack{0}%
      \Expect*{\TestWarning}{empty stack}%
    \endgroup
  }%
\end{qstest}
%    \end{macrocode}
%
% \subsubsection{Catcode table stack}
%
%    \begin{macrocode}
\begin{qstest}{CatcodeTableStack}{CatcodeTableStack}
  \def\TestStack#1{%
    \Expect*{\the\CatcodeTableStack}{#1}%
  }%
  \TestStack{0}%
  \IncCatcodeTableStack
  \TestStack{2}%
  \IncCatcodeTableStack
  \TestStack{4}%
  \begingroup
    \IncCatcodeTableStack
    \TestStack{6}%
  \endgroup
  \TestStack{6}%
  \begingroup
    \DecCatcodeTableStack
    \TestStack{4}%
  \endgroup
  \TestStack{4}%
  \DecCatcodeTableStack
  \TestStack{2}%
  \DecCatcodeTableStack
  \TestStack{0}%
  \begingroup
    \def\TestError{Missing error}%
    \def\@PackageError#1#2#3{%
      \def\TestError{Empty stack}%
    }%
    \DecCatcodeTableStack
    \TestStack{0}%
    \Expect*{\TestError}{Empty stack}%
  \endgroup
\end{qstest}
%    \end{macrocode}
%
% \subsubsection{Catcode regime macros}
%
%    \begin{macrocode}
\begin{qstest}{CatcodeRegime}{CatcodeRegime}
  \def\TestStacks#1#2#3{%
    \Expect*{\the\luatexcatcodetable}{#1}%
    \Expect*{\the\CatcodeTableStack}{#2}%
    \Expect*{\LuT@NumStack}{#3}%
  }%
  \TestStacks{0}{0}{0}%
  \catcode`\|=7 %
  \BeginCatcodeRegime\CatcodeTableLaTeX
    \TestStacks{2}{2}{{0}0}%
    \Expect*{\the\catcode`\|}{12}%
  \EndCatcodeRegime
  \TestStacks{0}{0}{0}%
  \Expect*{\the\catcode`\|}{7}%
\end{qstest}
%    \end{macrocode}
%
% \subsection{Attribute allocation}
%
%    \begin{macrocode}
\begin{qstest}{Attributes}{Attributes}
  \newattribute\TestAttr
  \Expect*{\meaning\TestAttr}%
         *{\string\attribute\number\allocationnumber}%
  \Expect*{\the\allocationnumber}{0}%
  \begingroup
    \newattribute\TestAttr
    \Expect*{\the\allocationnumber}{1}%
  \endgroup
  \Expect*{\the\allocationnumber}{0}%
  \Expect*{\meaning\TestAttr}*{\string\attribute1}%
  \Expect*{\the\TestAttr}*{\number\LuT@UnsetAttributeValue}%
  \def\Test#1{%
    \setattribute\TestAttr{#1}%
    \Expect*{\the\TestAttr}{#1}%
  }%
  \Test{0}%
  \Test{1}%
  \Test{-1}%
  \Test{123}%
  \unsetattribute\TestAttr
  \Expect*{\the\TestAttr}*{\number\LuT@UnsetAttributeValue}%
  \begingroup
    \Expect*{\the\TestAttr}*{\number\LuT@UnsetAttributeValue}%
    \Test{1234}%
  \endgroup
  \Expect*{\the\TestAttr}*{\number\LuT@UnsetAttributeValue}%
\end{qstest}
%    \end{macrocode}
%
%    \begin{macrocode}
\@@end
%</test4>
%    \end{macrocode}
%
% \subsection{Short test for \plainTeX}
%
%    \begin{macrocode}
%<*test5>
\input luatex.sty\relax
\newattribute\TestAttr
\setattribute\TestAttr{10}
\unsetattribute\TestAttr
\newcatcodetable\TestCTa
\begingroup
  \SetCatcodeRange{`A}{`Z}{12}%
\endgroup
\BeginCatcodeRegime\CatcodeTableLaTeX
\EndCatcodeRegime
\end
%</test5>
%    \end{macrocode}
%
% \section{Installation}
%
% \subsection{Download}
%
% \paragraph{Package.} This package is available on
% CTAN\footnote{\url{ftp://ftp.ctan.org/tex-archive/}}:
% \begin{description}
% \item[\CTAN{macros/latex/contrib/oberdiek/luatex.dtx}] The source file.
% \item[\CTAN{macros/latex/contrib/oberdiek/luatex.pdf}] Documentation.
% \end{description}
%
%
% \paragraph{Bundle.} All the packages of the bundle `oberdiek'
% are also available in a TDS compliant ZIP archive. There
% the packages are already unpacked and the documentation files
% are generated. The files and directories obey the TDS standard.
% \begin{description}
% \item[\CTAN{install/macros/latex/contrib/oberdiek.tds.zip}]
% \end{description}
% \emph{TDS} refers to the standard ``A Directory Structure
% for \TeX\ Files'' (\CTAN{tds/tds.pdf}). Directories
% with \xfile{texmf} in their name are usually organized this way.
%
% \subsection{Bundle installation}
%
% \paragraph{Unpacking.} Unpack the \xfile{oberdiek.tds.zip} in the
% TDS tree (also known as \xfile{texmf} tree) of your choice.
% Example (linux):
% \begin{quote}
%   |unzip oberdiek.tds.zip -d ~/texmf|
% \end{quote}
%
% \paragraph{Script installation.}
% Check the directory \xfile{TDS:scripts/oberdiek/} for
% scripts that need further installation steps.
% Package \xpackage{attachfile2} comes with the Perl script
% \xfile{pdfatfi.pl} that should be installed in such a way
% that it can be called as \texttt{pdfatfi}.
% Example (linux):
% \begin{quote}
%   |chmod +x scripts/oberdiek/pdfatfi.pl|\\
%   |cp scripts/oberdiek/pdfatfi.pl /usr/local/bin/|
% \end{quote}
%
% \subsection{Package installation}
%
% \paragraph{Unpacking.} The \xfile{.dtx} file is a self-extracting
% \docstrip\ archive. The files are extracted by running the
% \xfile{.dtx} through \plainTeX:
% \begin{quote}
%   \verb|tex luatex.dtx|
% \end{quote}
%
% \paragraph{TDS.} Now the different files must be moved into
% the different directories in your installation TDS tree
% (also known as \xfile{texmf} tree):
% \begin{quote}
% \def\t{^^A
% \begin{tabular}{@{}>{\ttfamily}l@{ $\rightarrow$ }>{\ttfamily}l@{}}
%   luatex.sty & tex/generic/oberdiek/luatex.sty\\
%   luatex-loader.sty & tex/generic/oberdiek/luatex-loader.sty\\
%   oberdiek.luatex.lua & scripts/oberdiek/oberdiek.luatex.lua\\
%   luatex.pdf & doc/latex/oberdiek/luatex.pdf\\
%   test/luatex-test1.tex & doc/latex/oberdiek/test/luatex-test1.tex\\
%   test/luatex-test2.tex & doc/latex/oberdiek/test/luatex-test2.tex\\
%   test/luatex-test3.tex & doc/latex/oberdiek/test/luatex-test3.tex\\
%   test/luatex-test4.tex & doc/latex/oberdiek/test/luatex-test4.tex\\
%   test/luatex-test5.tex & doc/latex/oberdiek/test/luatex-test5.tex\\
%   luatex.dtx & source/latex/oberdiek/luatex.dtx\\
% \end{tabular}^^A
% }^^A
% \sbox0{\t}^^A
% \ifdim\wd0>\linewidth
%   \begingroup
%     \advance\linewidth by\leftmargin
%     \advance\linewidth by\rightmargin
%   \edef\x{\endgroup
%     \def\noexpand\lw{\the\linewidth}^^A
%   }\x
%   \def\lwbox{^^A
%     \leavevmode
%     \hbox to \linewidth{^^A
%       \kern-\leftmargin\relax
%       \hss
%       \usebox0
%       \hss
%       \kern-\rightmargin\relax
%     }^^A
%   }^^A
%   \ifdim\wd0>\lw
%     \sbox0{\small\t}^^A
%     \ifdim\wd0>\linewidth
%       \ifdim\wd0>\lw
%         \sbox0{\footnotesize\t}^^A
%         \ifdim\wd0>\linewidth
%           \ifdim\wd0>\lw
%             \sbox0{\scriptsize\t}^^A
%             \ifdim\wd0>\linewidth
%               \ifdim\wd0>\lw
%                 \sbox0{\tiny\t}^^A
%                 \ifdim\wd0>\linewidth
%                   \lwbox
%                 \else
%                   \usebox0
%                 \fi
%               \else
%                 \lwbox
%               \fi
%             \else
%               \usebox0
%             \fi
%           \else
%             \lwbox
%           \fi
%         \else
%           \usebox0
%         \fi
%       \else
%         \lwbox
%       \fi
%     \else
%       \usebox0
%     \fi
%   \else
%     \lwbox
%   \fi
% \else
%   \usebox0
% \fi
% \end{quote}
% If you have a \xfile{docstrip.cfg} that configures and enables \docstrip's
% TDS installing feature, then some files can already be in the right
% place, see the documentation of \docstrip.
%
% \subsection{Refresh file name databases}
%
% If your \TeX~distribution
% (\teTeX, \mikTeX, \dots) relies on file name databases, you must refresh
% these. For example, \teTeX\ users run \verb|texhash| or
% \verb|mktexlsr|.
%
% \subsection{Some details for the interested}
%
% \paragraph{Attached source.}
%
% The PDF documentation on CTAN also includes the
% \xfile{.dtx} source file. It can be extracted by
% AcrobatReader 6 or higher. Another option is \textsf{pdftk},
% e.g. unpack the file into the current directory:
% \begin{quote}
%   \verb|pdftk luatex.pdf unpack_files output .|
% \end{quote}
%
% \paragraph{Unpacking with \LaTeX.}
% The \xfile{.dtx} chooses its action depending on the format:
% \begin{description}
% \item[\plainTeX:] Run \docstrip\ and extract the files.
% \item[\LaTeX:] Generate the documentation.
% \end{description}
% If you insist on using \LaTeX\ for \docstrip\ (really,
% \docstrip\ does not need \LaTeX), then inform the autodetect routine
% about your intention:
% \begin{quote}
%   \verb|latex \let\install=y% \iffalse meta-comment
%
% File: luatex.dtx
% Version: 2010/03/09 v0.4
% Info: LuaTeX basic definition package
%
% Copyright (C) 2007, 2009, 2010 by
%    Heiko Oberdiek <heiko.oberdiek at googlemail.com>
%
% This work may be distributed and/or modified under the
% conditions of the LaTeX Project Public License, either
% version 1.3c of this license or (at your option) any later
% version. This version of this license is in
%    http://www.latex-project.org/lppl/lppl-1-3c.txt
% and the latest version of this license is in
%    http://www.latex-project.org/lppl.txt
% and version 1.3 or later is part of all distributions of
% LaTeX version 2005/12/01 or later.
%
% This work has the LPPL maintenance status "maintained".
%
% This Current Maintainer of this work is Heiko Oberdiek.
%
% The Base Interpreter refers to any `TeX-Format',
% because some files are installed in TDS:tex/generic//.
%
% This work consists of the main source file luatex.dtx
% and the derived files
%    luatex.sty, luatex.pdf, luatex.ins, luatex.drv, luatex-loader.sty,
%    luatex-test1.tex, luatex-test2.tex, luatex-test3.tex,
%    luatex-test4.tex, luatex-test5.tex, oberdiek.luatex.lua.
%
% Distribution:
%    CTAN:macros/latex/contrib/oberdiek/luatex.dtx
%    CTAN:macros/latex/contrib/oberdiek/luatex.pdf
%
% Unpacking:
%    (a) If luatex.ins is present:
%           tex luatex.ins
%    (b) Without luatex.ins:
%           tex luatex.dtx
%    (c) If you insist on using LaTeX
%           latex \let\install=y\input{luatex.dtx}
%        (quote the arguments according to the demands of your shell)
%
% Documentation:
%    (a) If luatex.drv is present:
%           latex luatex.drv
%    (b) Without luatex.drv:
%           latex luatex.dtx; ...
%    The class ltxdoc loads the configuration file ltxdoc.cfg
%    if available. Here you can specify further options, e.g.
%    use A4 as paper format:
%       \PassOptionsToClass{a4paper}{article}
%
%    Programm calls to get the documentation (example):
%       pdflatex luatex.dtx
%       makeindex -s gind.ist luatex.idx
%       pdflatex luatex.dtx
%       makeindex -s gind.ist luatex.idx
%       pdflatex luatex.dtx
%
% Installation:
%    TDS:tex/generic/oberdiek/luatex.sty
%    TDS:tex/generic/oberdiek/luatex-loader.sty
%    TDS:scripts/oberdiek/oberdiek.luatex.lua
%    TDS:doc/latex/oberdiek/luatex.pdf
%    TDS:doc/latex/oberdiek/test/luatex-test1.tex
%    TDS:doc/latex/oberdiek/test/luatex-test2.tex
%    TDS:doc/latex/oberdiek/test/luatex-test3.tex
%    TDS:doc/latex/oberdiek/test/luatex-test4.tex
%    TDS:doc/latex/oberdiek/test/luatex-test5.tex
%    TDS:source/latex/oberdiek/luatex.dtx
%
%<*ignore>
\begingroup
  \catcode123=1 %
  \catcode125=2 %
  \def\x{LaTeX2e}%
\expandafter\endgroup
\ifcase 0\ifx\install y1\fi\expandafter
         \ifx\csname processbatchFile\endcsname\relax\else1\fi
         \ifx\fmtname\x\else 1\fi\relax
\else\csname fi\endcsname
%</ignore>
%<*install>
\input docstrip.tex
\Msg{************************************************************************}
\Msg{* Installation}
\Msg{* Package: luatex 2010/03/09 v0.4 LuaTeX basic definition package (HO)}
\Msg{************************************************************************}

\keepsilent
\askforoverwritefalse

\let\MetaPrefix\relax
\preamble

This is a generated file.

Project: luatex
Version: 2010/03/09 v0.4

Copyright (C) 2007, 2009, 2010 by
   Heiko Oberdiek <heiko.oberdiek at googlemail.com>

This work may be distributed and/or modified under the
conditions of the LaTeX Project Public License, either
version 1.3c of this license or (at your option) any later
version. This version of this license is in
   http://www.latex-project.org/lppl/lppl-1-3c.txt
and the latest version of this license is in
   http://www.latex-project.org/lppl.txt
and version 1.3 or later is part of all distributions of
LaTeX version 2005/12/01 or later.

This work has the LPPL maintenance status "maintained".

This Current Maintainer of this work is Heiko Oberdiek.

The Base Interpreter refers to any `TeX-Format',
because some files are installed in TDS:tex/generic//.

This work consists of the main source file luatex.dtx
and the derived files
   luatex.sty, luatex.pdf, luatex.ins, luatex.drv, luatex-loader.sty,
   luatex-test1.tex, luatex-test2.tex, luatex-test3.tex,
   luatex-test4.tex, luatex-test5.tex, oberdiek.luatex.lua.

\endpreamble
\let\MetaPrefix\DoubleperCent

\generate{%
  \file{luatex.ins}{\from{luatex.dtx}{install}}%
  \file{luatex.drv}{\from{luatex.dtx}{driver}}%
  \usedir{tex/generic/oberdiek}%
  \file{luatex.sty}{\from{luatex.dtx}{package}}%
  \file{luatex-loader.sty}{\from{luatex.dtx}{loader}}%
  \usedir{doc/latex/oberdiek/test}%
  \file{luatex-test1.tex}{\from{luatex.dtx}{test1}}%
  \file{luatex-test2.tex}{\from{luatex.dtx}{test1,test2}}%
  \file{luatex-test3.tex}{\from{luatex.dtx}{test1,test3}}%
  \file{luatex-test4.tex}{\from{luatex.dtx}{test4}}%
  \file{luatex-test5.tex}{\from{luatex.dtx}{test5}}%
  \nopreamble
  \nopostamble
  \usedir{source/latex/oberdiek/catalogue}%
  \file{luatex.xml}{\from{luatex.dtx}{catalogue}}%
}
\def\MetaPrefix{-- }
\def\defaultpostamble{%
  \MetaPrefix^^J%
  \MetaPrefix\space End of File `\outFileName'.%
}
\def\currentpostamble{\defaultpostamble}%
\generate{%
  \usedir{scripts/oberdiek}%
  \file{oberdiek.luatex.lua}{\from{luatex.dtx}{lua}}%
}

\catcode32=13\relax% active space
\let =\space%
\Msg{************************************************************************}
\Msg{*}
\Msg{* To finish the installation you have to move the following}
\Msg{* files into a directory searched by TeX:}
\Msg{*}
\Msg{*     luatex.sty, luatex-loader.sty}
\Msg{*}
\Msg{* And install the following script file:}
\Msg{*}
\Msg{*     oberdiek.luatex.lua}
\Msg{*}
\Msg{* To produce the documentation run the file `luatex.drv'}
\Msg{* through LaTeX.}
\Msg{*}
\Msg{* Happy TeXing!}
\Msg{*}
\Msg{************************************************************************}

\endbatchfile
%</install>
%<*ignore>
\fi
%</ignore>
%<*driver>
\NeedsTeXFormat{LaTeX2e}
\ProvidesFile{luatex.drv}%
  [2010/03/09 v0.4 LuaTeX basic definition package (HO)]%
\documentclass{ltxdoc}
\usepackage{holtxdoc}[2011/11/22]
\usepackage{paralist}
\begin{document}
  \DocInput{luatex.dtx}%
\end{document}
%</driver>
% \fi
%
% \CheckSum{1429}
%
% \CharacterTable
%  {Upper-case    \A\B\C\D\E\F\G\H\I\J\K\L\M\N\O\P\Q\R\S\T\U\V\W\X\Y\Z
%   Lower-case    \a\b\c\d\e\f\g\h\i\j\k\l\m\n\o\p\q\r\s\t\u\v\w\x\y\z
%   Digits        \0\1\2\3\4\5\6\7\8\9
%   Exclamation   \!     Double quote  \"     Hash (number) \#
%   Dollar        \$     Percent       \%     Ampersand     \&
%   Acute accent  \'     Left paren    \(     Right paren   \)
%   Asterisk      \*     Plus          \+     Comma         \,
%   Minus         \-     Point         \.     Solidus       \/
%   Colon         \:     Semicolon     \;     Less than     \<
%   Equals        \=     Greater than  \>     Question mark \?
%   Commercial at \@     Left bracket  \[     Backslash     \\
%   Right bracket \]     Circumflex    \^     Underscore    \_
%   Grave accent  \`     Left brace    \{     Vertical bar  \|
%   Right brace   \}     Tilde         \~}
%
% \GetFileInfo{luatex.drv}
%
% \title{The \xpackage{luatex} package}
% \date{2010/03/09 v0.4}
% \author{Heiko Oberdiek\\\xemail{heiko.oberdiek at googlemail.com}}
%
% \maketitle
%
% \begin{abstract}
% This package manages the new and extended features and resources
% that \LuaTeX\ provides. Examples are attributes and catcode
% tables.
% \end{abstract}
%
% \tableofcontents
%
% \section{Documentation}
%
% \subsection{Introduction}
%
% \TeX\ provides global resources such as registers.
% But it does not provide an interface for managing these resources.
% For example, two packages want to use a counter register.
% If they take the same register
% number, then the use of both packages will conflict and they cannot
% be used together.
% Therefore formats such as \plainTeX\ or \LaTeX\ implement
% an allocation scheme for registers. A package reserves with
% \cs{newcount} an unused register number for its own exclusive use.
%
% Nowadays \TeX\ is not alone anymore: \eTeX, \pdfTeX\ and other
% compilers for \TeX\ are developed that extend and add new
% features and resources.
%
% Now \LuaTeX\ has reached beta state. It inherits most of
% \pdfTeX's features including \eTeX. Also it implements
% new concepts such as attributes or catcode tables.
%
% \subsubsection{\LaTeX}
%
% \LaTeXe\ is frozen and therefore refuses to even notice
% the new \TeX\ variants. Not even the old \eTeX\ is supported
% by its kernel. At least there is a third party package \xpackage{etex}
% that manages the new \eTeX\ resources.
%
% This package tries to do the same for \LuaTeX\ and starts to
% support at least a few of the new features.
%
% \subsubsection{\plainTeX}
%
% \LaTeX\ has inherited its resource handling from \plainTeX.
% The interface is basically the same: \cs{newcount}, \dots
% Therefore this package tries to follow this tradition by
% providing compatibility to \plainTeX. It can be loaded
% with \plainTeX\ and defines at least some of the
% features that this packages provides for \LaTeX.
%
% \subsection{Register allocation}
%
% \subsubsection{Register with 16 bit}
%
% Because \LuaTeX\ is a super set of \eTeX\ regarding registers,
% the register allocation scheme should not conflict with
% package \xpackage{etex}. Therefore this package is loaded
% to inherit its allocation scheme. The only change is currently
% that the limit is increased to 65536 registers for the following
% register classes:
% \begin{compactitem}
% \item count
% \item dimen
% \item skip
% \item muskip
% \item marks
% \item toks
% \item box
% \end{compactitem}
% This affects the number of global and local registers.
% Because it is done in a package and not in the kernel,
% it is possible that someone loads package \xpackage{etex}
% before uses the local allocation variants. This will prevent
% the extension for this register class. If more registers are
% needed, just load package \xpackage{luatex} earlier.
%
% \subsubsection{Insertions}
%
% Insertions need four registers \cs{count}, \cs{dimen}, \cs{skip},
% and \cs{box} with the same number. Usually they are allocated
% downwards from 254, 253, \dots Also \cs{newcount}, \cs{newdimen}, \dots
% fill up these register numbers from below before switching to
% higher register numbers by package \xpackage{etex}.
% When this occurs, no insertions can be allocated anymore.
%
% Therefore \cs{newcount}, \cs{newdimen}, \cs{newskip}, and
% \cs{newbox} are replaced by their global variants (\cs{globcount}, \dots)
% that use the higher numbers immediately, leaving the room for insertions.
% There should not be an efficiency penalty because \LuaTeX\ stores
% the registers of a class in the same Lua table unlike \eTeX, where
% registers below 256 are stored in an array and higher numbers
% are put in a tree structure.
%
% \subsection{Attributes}
%
% Nodes can have custom attributes in \LuaTeX. These attributes
% are organized by a new register class. As the other registers
% up to $2^{16}$ attributes are supported. An attribute value
% can be negative that means the attribute is not set. Otherwise
% \TeX's range of non-negative integers up to $2^31$ are available.
%
% \begin{declcs}{newattribute} \M{cmd}
% \end{declcs}
% Macro \cs{newattribute} defines command \meta{cmd} using \cs{attributedef}
% using an new attribute number. The new attribute is initially unset.
%
% \begin{declcs}{setattribute} \M{cmd} \M{value}
% \end{declcs}
% Macro \cs{setattribute} locally sets attribute command \meta{cmd}
% to the number \meta{value}. Valid values range from $-1$ until $2^31$
% (the upper limit is the same as for other \TeX\ integer numbers).
%
% \begin{declcs}{unsetattribute} \M{cmd}
% \end{declcs}
% Macro \cs{unsettattribute} clears the attribute command \meta{cmd}.
%
% \subsection{Catcode tables}
%
% \LuaTeX\ introduces catcode tables as new feature, see documentation.
% There is need for discussion, how to deal best:
% \begin{itemize}
% \item \cs{initcatcodetable} and \cs{setcatcodetable} act globally.
% \item \cs{catcodetable} causes an error if used with an uninitialized
%   catcode table.
% \item Large catcode table numbers should be avoided because of
%   performance breakdown.
% \item Use case \LaTeX\ package:
%   The package must not be surprised by changed catcodes and
%   must not surprise by changing catcodes accidently.
%   Catcode tables could offer a solution. At the begin
%   a catcode regime with standard catcodes is established
%   and the old one is restored afterwards.
% \item Use case: \LuaTeX's |tex.print| might be used with
%   a catcode table number, for example a table where all
%   entries have catcode ``other''.
% \item Readonly catcode tables.
% \item Is there is a need for local allocations?
%   (Package \xpackage{etex}'s \cs{loc} variants are not used in
%   \TeX\ Live 2007.)
% \end{itemize}
%
% \subsubsection{Interface proposal}
%
% The idea: \cs{newcatcodetable} allocates odd numbered catcode
% tables. Even numbered tables are managed as stack.
% Also some catcode tables are defined. These must not be
% changed.
%
% \begin{declcs}{newcatcodetable} \M{cmd}
% \end{declcs}
% Macro \cs{newcatcodetable} reserves a new catcode table and
% remembers its number in \meta{cmd}. The catcode table is
% initialized with \iniTeX's catcodes.
%
% \begin{declcs}{CatcodeTableIniTeX}\\
%   \cs{CatcodeTableString}\\
%   \cs{CatcodeTableOther}\\
%   \cs{CatcodeTableLaTeX}\\
% \end{declcs}
% These are catcode tables and must not be changed.
% \cs{CatcodeTableIniTeX} contains the catcode settings of \iniTeX.
% \cs{CatcodeTableString} follows \TeX's convention of
% \cs{string}, \cs{meaning} and friends. The space gets catcode
% 10 (space), the other characters have catcode 12 (other).
% In \cs{CatcodeTableOther} all entries have catcode 12 (other).
% \cs{CatcodeTableLaTeX} contains the setting of a pure \LaTeX\ format
% (`at' is other).
%
% \begin{declcs}{CatcodeTableStack}\\
%   \cs{IncCatcodeTableStack}\\
%   \cs{DecCatcodeTableStack}
% \end{declcs}
% \cs{CatcodeTableStack} is the stack pointer. Initially
% it is catcode table zero. \cs{IncCatcodeTableStack} and
% \cs{DecCatcodeTableStack} increments and decrements the
% stack pointer. Currently \cs{IncCatcodeTableStack} does not
% initialize a new catcode table. Both increment and decrement
% operations do not set a catcode table.
%
% \begin{declcs}{PushCatcodeTableNumStack}\\
%   \cs{PopCatcodeTableNumStack}
% \end{declcs}
% It can be handy to have a global stack for catcode table
% numbers to deal with the global assignment property of
% \cs{initcatcodetable} and \cs{savecatcodetable}.
% \cs{PushCatcodeTableNumStack} pushes the current catcode table
% on the stack.
% \cs{PopCatcodeTableNumStack} pops the topmost number off the number
% stack to set the current catcode table. Catcode table zero is
% used in case of an empty stack.
%
% \begin{declcs}{BeginCatcodeRegime} \M{catcodetable}\\
%   \cs{EndCatcodeRegime}
% \end{declcs}
% \cs{BeginCatcodeRegime} remembers the current catcode table number.
% Then it creates and uses a fresh catcode table on the stack
% that is initialized by \meta{catcodetable}:
% \begin{quote}\ttfamily
%   \cs{PushCatcodeTableNumStack}\\
%   \cs{catcodetable}\meta{catcodetable}||
%   \cs{IncCatcodeTableStack}\\
%   \cs{savecatcodetable}\cs{CatcodeTableStack}\\
%   \cs{catcodetable}\cs{CatcodeTableStack}
% \end{quote}
% \cs{EndCatcodeRegime} drops the catcode table, created by
% \cs{BeginCatcodeRegime} and sets the catcode table that was
% active before:
% \begin{quote}
%   \cs{DecCatcodeTableStack}\\
%   \cs{PopCatcodeTableNumStack}
% \end{quote}
% These macros solve the use case, described earlier for a \LaTeX\ package:
% \begin{quote}
%   |% package foobar.sty|\\
%   |\BeginCatcodeRegime\CatcodeTableLaTeX|\\
%   |\makeatletter|\\
%   |% ... package contents ...|\\
%   |\EndCatcodeRegime|\\
%   |% end of package|
% \end{quote}
% If the package wants to change catcodes after its loading,
% \cs{AtBeginDocument} or \cs{AtEndOfPackage} can be used.
%
% \begin{declcs}{SetCatcodeRange} \M{from} \M{to} \M{catcode}
% \end{declcs}
% The catcodes of characters in range from \meta{from}
% to inclusive \meta{to} are set to \meta{catcode}.
%
% \subsection{Lua module loading}
%
% Currently \LuaTeX\ (version 0.20) does not support Lua script
% files inside \xfile{TDS:scripts//}, because Lua's mechanism
% for module loading does not use the \textsf{kpathsea} library.
% Therefore this packages appends a kpse loader to the list of
% Lua's module loaders. It finds the module \meta{module} by
% \begin{quote}
%   \ttfamily |kpse.find_file("|\meta{module}|.lua", "texmfscripts")|
% \end{quote}
% Unhappily \textsf{kpathsea} does not support directory components
% in a file name. Therefore the Lua convention is not followed to
% replace dots in the module name by the directory separator.
%
% Example:
% A Lua script of a package \xpackage{foobar} wants the following
% modules:
% \begin{quote}
%   |require("foobar.hello.world")|\\
%   |require("org.somewhere.xyz")|
% \end{quote}
% Then they can be find in:
% \begin{quote}
%   |TDS:scripts/foobar/foobar.hello.world.lua|\\
%   |TDS:scripts/foobar/org.somewhere.xyz.lua|
% \end{quote}
% I would have preferred the following locations, following lua conventions,
% e.\,g.:
% \begin{quote}
%   |TDS:scripts/foobar/hello/world.lua|\\
%   |TDS:scripts/foobar/org/somewhere/xyz.lua|
% \end{quote}
% But I do not know, how to achieve this in a reliable way using
% \textsf{kpathsea}.
%
% \subsubsection{Package \xpackage{luatex-loader}}
%
% If someone do not need or want package \xpackage{luatex} but
% it's extension for module loading, then he can use
% package \xpackage{luatex-loader}. Both \plainTeX\ and \LaTeX\ are
% supported.
%
% \StopEventually{
% }
%
% \section{Implementation}
%
%    \begin{macrocode}
%<*package>
%    \end{macrocode}
%
% \subsection{Reload check and package identification}
%    Reload check, especially if the package is not used with \LaTeX.
%    \begin{macrocode}
\begingroup\catcode61\catcode48\catcode32=10\relax%
  \catcode13=5 % ^^M
  \endlinechar=13 %
  \catcode35=6 % #
  \catcode39=12 % '
  \catcode44=12 % ,
  \catcode45=12 % -
  \catcode46=12 % .
  \catcode58=12 % :
  \catcode64=11 % @
  \catcode123=1 % {
  \catcode125=2 % }
  \expandafter\let\expandafter\x\csname ver@luatex.sty\endcsname
  \ifx\x\relax % plain-TeX, first loading
  \else
    \def\empty{}%
    \ifx\x\empty % LaTeX, first loading,
      % variable is initialized, but \ProvidesPackage not yet seen
    \else
      \expandafter\ifx\csname PackageInfo\endcsname\relax
        \def\x#1#2{%
          \immediate\write-1{Package #1 Info: #2.}%
        }%
      \else
        \def\x#1#2{\PackageInfo{#1}{#2, stopped}}%
      \fi
      \x{luatex}{The package is already loaded}%
      \aftergroup\endinput
    \fi
  \fi
\endgroup%
%    \end{macrocode}
%    Package identification:
%    \begin{macrocode}
\begingroup\catcode61\catcode48\catcode32=10\relax%
  \catcode13=5 % ^^M
  \endlinechar=13 %
  \catcode35=6 % #
  \catcode39=12 % '
  \catcode40=12 % (
  \catcode41=12 % )
  \catcode44=12 % ,
  \catcode45=12 % -
  \catcode46=12 % .
  \catcode47=12 % /
  \catcode58=12 % :
  \catcode64=11 % @
  \catcode91=12 % [
  \catcode93=12 % ]
  \catcode123=1 % {
  \catcode125=2 % }
  \expandafter\ifx\csname ProvidesPackage\endcsname\relax
    \def\x#1#2#3[#4]{\endgroup
      \immediate\write-1{Package: #3 #4}%
      \xdef#1{#4}%
    }%
  \else
    \def\x#1#2[#3]{\endgroup
      #2[{#3}]%
      \ifx#1\@undefined
        \xdef#1{#3}%
      \fi
      \ifx#1\relax
        \xdef#1{#3}%
      \fi
    }%
  \fi
\expandafter\x\csname ver@luatex.sty\endcsname
\ProvidesPackage{luatex}%
  [2010/03/09 v0.4 LuaTeX basic definition package (HO)]%
%    \end{macrocode}
%
% \subsection{Catcodes}
%
%    \begin{macrocode}
\begingroup\catcode61\catcode48\catcode32=10\relax%
  \catcode13=5 % ^^M
  \endlinechar=13 %
  \catcode123=1 % {
  \catcode125=2 % }
  \catcode64=11 % @
  \def\x{\endgroup
    \expandafter\edef\csname LuT@AtEnd\endcsname{%
      \endlinechar=\the\endlinechar\relax
      \catcode13=\the\catcode13\relax
      \catcode32=\the\catcode32\relax
      \catcode35=\the\catcode35\relax
      \catcode61=\the\catcode61\relax
      \catcode64=\the\catcode64\relax
      \catcode123=\the\catcode123\relax
      \catcode125=\the\catcode125\relax
    }%
  }%
\x\catcode61\catcode48\catcode32=10\relax%
\catcode13=5 % ^^M
\endlinechar=13 %
\catcode35=6 % #
\catcode64=11 % @
\catcode123=1 % {
\catcode125=2 % }
\def\TMP@EnsureCode#1#2{%
  \edef\LuT@AtEnd{%
    \LuT@AtEnd
    \catcode#1=\the\catcode#1\relax
  }%
  \catcode#1=#2\relax
}
\TMP@EnsureCode{10}{12}% ^^J
\TMP@EnsureCode{34}{12}% "
\TMP@EnsureCode{36}{3}% $
\TMP@EnsureCode{39}{12}% '
\TMP@EnsureCode{40}{12}% (
\TMP@EnsureCode{41}{12}% )
\TMP@EnsureCode{42}{12}% *
\TMP@EnsureCode{43}{12}% +
\TMP@EnsureCode{44}{12}% ,
\TMP@EnsureCode{45}{12}% -
\TMP@EnsureCode{46}{12}% .
\TMP@EnsureCode{47}{12}% /
\TMP@EnsureCode{60}{12}% <
\TMP@EnsureCode{62}{12}% >
\TMP@EnsureCode{91}{12}% [
\TMP@EnsureCode{93}{12}% ]
\TMP@EnsureCode{95}{12}% _ (other!)
\TMP@EnsureCode{96}{12}% `
\edef\LuT@AtEnd{\LuT@AtEnd\noexpand\endinput}
%    \end{macrocode}
%
% \subsection{Check for \LuaTeX}
%
%    Without \LuaTeX\ there is no point in using this package.
%    \begin{macrocode}
\begingroup\expandafter\expandafter\expandafter\endgroup
\expandafter\ifx\csname RequirePackage\endcsname\relax
  \input infwarerr.sty\relax
  \input ifluatex.sty\relax
\else
  \RequirePackage{infwarerr}[2007/09/09]%
  \RequirePackage{ifluatex}[2009/04/10]%
\fi
%    \end{macrocode}
%    \begin{macrocode}
\ifluatex
\else
  \@PackageError{luatex}{%
    This package may only be run using LuaTeX%
  }\@ehc
  \expandafter\LuT@AtEnd
\fi%
%    \end{macrocode}
%
% \subsection{Provide Lua\TeX\ primitives}
%
%    \begin{macrocode}
\ifnum\luatexversion<36 %
  \def\LuT@MakePrimitive#1{%
    \expandafter\let\csname luatex#1\expandafter\endcsname
    \csname #1\endcsname
  }%
\else
  \def\LuT@MakeLuatexPrimitive#1{%
    \begingroup\expandafter\expandafter\expandafter\endgroup
    \expandafter\ifx\csname luatex#1\endcsname\relax
      \begingroup\expandafter\expandafter\expandafter\endgroup
      \expandafter\ifx\csname #1\endcsname\relax
      \else
        \expandafter\let
        \csname luatex#1\expandafter\endcsname
        \csname #1\endcsname
      \fi
    \fi
    \begingroup\expandafter\expandafter\expandafter\endgroup
    \expandafter\ifx\csname luatex#1\endcsname\relax
      \begingroup
        \expandafter\let\csname luatex#1\endcsname\@undefined
        \ifnum0%
          \directlua{%
            if tex.enableprimitives then %
              tex.enableprimitives('luatex',{'#1'})%
              tex.print('1')%
            end%
          }%
          \expandafter\ifx\csname luatex#1\endcsname\relax\else1\fi
        =11 %
          \global\expandafter\let
          \csname luatex#1\expandafter\endcsname
          \csname luatex#1\endcsname
        \else
          \@PackageError{luatex}{%
            tex.enableprimitives failed for `#1'%
          }\@ehc
        \fi
      \endgroup
    \fi
  }%
  \def\LuT@MakePrimitive#1{%
    \begingroup\expandafter\expandafter\expandafter\endgroup
    \expandafter\ifx\csname#1\endcsname\relax
      \begingroup
        \expandafter\let\csname#1\endcsname\@undefined
        \ifnum0%
          \directlua{%
            if tex.enableprimitives then %
              tex.enableprimitives('',{'#1'})%
              tex.print('1')%
            end%
          }%
          \expandafter\ifx\csname#1\endcsname\relax\else1\fi
        =11 %
          \global\expandafter\let
          \csname#1\expandafter\endcsname
          \csname#1\endcsname
        \else
          \@PackageError{luatex}{%
            tex.enableprimitives failed for `#1'%
          }\@ehc
        \fi
      \endgroup
    \fi
  }%
\fi
\LuT@MakeLuatexPrimitive{attribute}
\LuT@MakeLuatexPrimitive{attributedef}
\LuT@MakeLuatexPrimitive{catcodetable}
\LuT@MakeLuatexPrimitive{initcatcodetable}
\LuT@MakeLuatexPrimitive{luaescapestring}
\LuT@MakeLuatexPrimitive{savecatcodetable}
\LuT@MakePrimitive{numexpr}
%    \end{macrocode}
%
% \subsection{Inherit support for \eTeX}
%
%    Package \xpackage{etex} is not compatible for \plainTeX.
%    But it could be present if a format is used that is based
%    on \xfile{etex.src}. Therefore we only load the package
%    in case of \LaTeX\ and tests its presence independently
%    of the format by looking for \cs{et@xins}.
%    \begin{macrocode}
\begingroup\expandafter\expandafter\expandafter\endgroup
\expandafter\ifx\csname RequirePackage\endcsname\relax
\else
  \RequirePackage{etex}[1998/03/26]%
\fi
%    \end{macrocode}
%
% \subsection{Adaption of \eTeX's register allocation}
%
%    \eTeX has increased the number of \TeX\ registers
%    from $2^8$ (256) to $2^{15}$ (32768) for a register class.
%    \LuaTeX\ extends the limit further to $2^16$ (65536).
%    The allocation scheme of package \xpackage{etex} is not
%    changed. But this can be subject for discussion.
%
%    If a register class hasn't registered any local registers
%    yet, then the limit can safely be pushed to 65536.
%    \begin{macrocode}
\begingroup\expandafter\expandafter\expandafter\endgroup
\expandafter\ifx\csname et@xins\endcsname\relax
  \@PackageWarningNoLine{luatex}{%
    Support for eTeX is not loaded (etex.src)%
  }%
\else
  \def\LuT@temp#1{%
    \ifnum\count27#1=32768 %
      \count27#1=65536 %
    \fi
  }%
  \LuT@temp0%
  \LuT@temp1%
  \LuT@temp2%
  \LuT@temp3%
  \LuT@temp4%
  \LuT@temp5%
  \LuT@temp6%
%    \end{macrocode}
%
% \eTeX\ uses an array for the first 256 registers and then a tree
% structure. \LuaTeX\ stores all registers of a class in one Lua table.
% There shouldn't be large performance differences.
% This allows starting immediately in the extended
% area, leaving room for insertions.
%    \begin{macrocode}
  \let\newcount\globcount
  \let\newdimen\globdimen
  \let\newskip\globskip
  \let\newbox\globbox
\fi
%    \end{macrocode}
%
% \subsection{\plainTeX\ compatibility}
%
%    \begin{macro}{\@empty}
%    \begin{macrocode}
\expandafter\ifx\csname @empty\endcsname\relax
  \def\@empty{}%
\fi
%    \end{macrocode}
%    \end{macro}
%    \begin{macro}{\@gobble}
%    \begin{macrocode}
\expandafter\ifx\csname @gobble\endcsname\relax
  \long\def\@gobble#1{}%
\fi
%    \end{macrocode}
%    \end{macro}
%    \begin{macro}{\@firstofone}
%    \begin{macrocode}
\expandafter\ifx\csname @firstofone\endcsname\relax
  \long\def\@firstofone#1{#1}%
\fi
%    \end{macrocode}
%    \end{macro}
%    \begin{macro}{\@firstoftwo}
%    \begin{macrocode}
\expandafter\ifx\csname @firstoftwo\endcsname\relax
  \long\def\@firstoftwo#1#2{#1}%
\fi
%    \end{macrocode}
%    \end{macro}
%    \begin{macro}{\@car}
%    \begin{macrocode}
\expandafter\ifx\csname @car\endcsname\relax
  \def\@car#1#2\@nil{#1}%
\fi
%    \end{macrocode}
%    \end{macro}
%    \begin{macro}{\@cdr}
%    \begin{macrocode}
\expandafter\ifx\csname @cdr\endcsname\relax
  \def\@cdr#1#2\@nil{#2}%
\fi
%    \end{macrocode}
%    \end{macro}
%    \begin{macro}{\@ifstar}
%    \begin{macrocode}
\expandafter\ifx\csname @ifstar\endcsname\relax
  \def\@ifstar#1{%
    \@ifnextchar*{\@firstoftwo{#1}}%
  }%
%    \end{macrocode}
%    \end{macro}
%    \begin{macro}{\@ifnextchar}
%    \begin{macrocode}
  \long\def\@ifnextchar#1#2#3{%
    \let\reserved@d=#1%
    \def\reserved@a{#2}%
    \def\reserved@b{#3}%
    \futurelet\@let@token\@ifnch
  }%
%    \end{macrocode}
%    \end{macro}
%    \begin{macro}{\@ifnch}
%    \begin{macrocode}
  \def\@ifnch{%
    \ifx\@let@token\@sptoken
      \let\reserved@c\@xifnch
    \else
      \ifx\@let@token\reserved@d
        \let\reserved@c\reserved@a
      \else
        \let\reserved@c\reserved@b
      \fi
    \fi
    \reserved@c
  }%
%    \end{macrocode}
%    \end{macro}
%    \begin{macro}{\@sptoken}
%    \begin{macrocode}
  \let\LuT@temp\:%
  \def\:{\let\@sptoken= }%
  \: % explicit space
%    \end{macrocode}
%    \end{macro}
%    \begin{macro}{\@xifnch}
%    \begin{macrocode}
  \def\:{\@xifnch}%
  \expandafter\def\: {%
    \futurelet\@let@token\@ifnch
  }%
  \let\:\LuT@temp
\fi
%    \end{macrocode}
%    \end{macro}
%
%    \begin{macro}{\@tempcnta}
%    \begin{macrocode}
\expandafter\ifx\csname @tempcnta\endcsname\relax
  \csname newcount\endcsname\@tempcnta
\fi
%    \end{macrocode}
%    \end{macro}
%
%    \begin{macro}{\@tempcntb}
%    \begin{macrocode}
\expandafter\ifx\csname @tempcntb\endcsname\relax
  \csname newcount\endcsname\@tempcntb
\fi
%    \end{macrocode}
%    \end{macro}
%
%    \begin{macro}{\LuT@newcommand}
%    \begin{macrocode}
\begingroup\expandafter\expandafter\expandafter\endgroup
\expandafter\ifx\csname newcommand\endcsname\relax
  \def\LuT@newcommand#1[#2]#3{%
    \ifx#1\@undefined
      \let#1\relax
    \else
      \ifx#1\relax
      \else
        \@PackageError{luatex}{%
          \string#1 is already defined.\MessageBreak
          Redefinition is skipped%
        }\@ehc
      \fi
    \fi
    \ifx#1\relax
      \ifcase#2 %
        \def#1{#3}%
      \or
        \def#1##1{#3}%
      \or
        \def#1##1##2{#3}%
      \or
        \def#1##1##2##3{#3}%
      \or
        \@INTERNAL@ERROR
      \fi
    \fi
  }%
\else
  \def\LuT@newcommand{\newcommand*}%
\fi
%    \end{macrocode}
%    \end{macro}
%
% \subsection{Attributes}
%
% \subsubsection{Allocation}
%
%    \begin{macro}{\LuT@AllocAttribute}
%    \begin{macrocode}
\newcount\LuT@AllocAttribute
\LuT@AllocAttribute=\m@ne
%    \end{macrocode}
%    \end{macro}
%    \begin{macro}{\newattribute}
%    \begin{macrocode}
\LuT@newcommand\newattribute[1]{%
  \ifnum\LuT@AllocAttribute<65535 %
    \global\advance\LuT@AllocAttribute\@ne
    \allocationnumber\LuT@AllocAttribute
    \global\luatexattributedef#1=\allocationnumber
    \unsetattribute{#1}%
    \wlog{\string#1=\string\attribute\the\allocationnumber}%
  \else
    \errmessage{No room for a new \string\attribute}%
  \fi
}
%    \end{macrocode}
%    \end{macro}
%
% \subsubsection{Interface}
%
%    \begin{macro}{\setattribute}
%    \begin{macrocode}
\LuT@newcommand\setattribute[2]{%
  #1=\numexpr#2\relax
}
%    \end{macrocode}
%    \end{macro}
%    \begin{macro}{\unsetattribute}
%    \begin{macrocode}
\ifnum\luatexversion<37
  \LuT@newcommand\LuT@UnsetAttributeValue[0]{}%
  \let\LuT@UnsetAttributeValue\m@ne
\else
  \LuT@newcommand\LuT@UnsetAttributeValue[0]{-2147483647 }%
\fi
\LuT@newcommand\unsetattribute[1]{%
  #1=\LuT@UnsetAttributeValue
}
%    \end{macrocode}
%    \end{macro}
%
% \subsection{Catcode tables}
%
% \subsubsection{Allocation}
%
%    \begin{macro}{\LuT@AllocCatcodeTable}
%    \begin{macrocode}
\newcount\LuT@AllocCatcodeTable
\LuT@AllocCatcodeTable=\m@ne
\newcount\CatcodeTableStack
\CatcodeTableStack=\z@
%    \end{macrocode}
%    \end{macro}
%    \begin{macro}{\newcatcodetable}
%    \begin{macrocode}
\LuT@newcommand\newcatcodetable[1]{%
  \ifnum\LuT@AllocCatcodeTable<1114110 % 0x10FFFF is maximal \chardef
    % or < 268435455 % 2^28 - 1
    \global\advance\LuT@AllocCatcodeTable by\tw@
    \allocationnumber=\LuT@AllocCatcodeTable
    \global\chardef#1=\allocationnumber
    \wlog{%
      \string#1=\string\catcodetable\the\allocationnumber
    }%
  \else
    \errmessage{No room for a new \string\catcodetable}%
  \fi
}%
%    \end{macrocode}
%    \end{macro}
%
%    \begin{macro}{\IncCatcodeTableStack}
%    \begin{macrocode}
\LuT@newcommand\IncCatcodeTableStack[0]{%
  \ifnum\CatcodeTableStack<268435454 %
    \global\advance\CatcodeTableStack by\tw@
  \else
    \@PackageError{luatex}{%
      Catcode table stack overflow%
    }\@ehd
  \fi
}
%    \end{macrocode}
%    \end{macro}
%
%    \begin{macro}{\DecCatcodeTableStack}
%    \begin{macrocode}
\LuT@newcommand\DecCatcodeTableStack[0]{%
  \ifnum\CatcodeTableStack>\z@
    \global\advance\CatcodeTableStack by-2 %
  \else
    \@PackageError{luatex}{%
      Catcode table stack is empty%
    }\@ehd
  \fi
}
%    \end{macrocode}
%    \end{macro}
%
% \subsubsection{\cs{SetCatcodeRange}}
%
%    \begin{macro}{\SetCatcodeRange}
%    \begin{macrocode}
\LuT@newcommand\SetCatcodeRange[3]{%
  \edef\LuT@temp{%
    \noexpand\@tempcnta=\the\@tempcnta
    \noexpand\@tempcntb=\the\@tempcntb
    \noexpand\count@=\the\count@
    \relax
  }%
  \@tempcnta=\numexpr#1\relax
  \@tempcntb=\numexpr#2\relax
  \count@=\numexpr#3\relax
  \loop
    \unless\ifnum\@tempcnta>\@tempcntb
    \catcode\@tempcnta=\count@
    \advance\@tempcnta by \@ne
  \repeat
  \LuT@temp
}
%    \end{macrocode}
%    \end{macro}
%
% \subsubsection{Predefined catcode tables}
%
%    \begin{macrocode}
\newcatcodetable\CatcodeTableIniTeX
\newcatcodetable\CatcodeTableString
\newcatcodetable\CatcodeTableOther
\newcatcodetable\CatcodeTableLaTeX
%    \end{macrocode}
%
%    \begin{macrocode}
\luatexinitcatcodetable\CatcodeTableIniTeX
\begingroup
  \def\@makeother#1{\catcode#1=12\relax}%
  \@firstofone{%
    \luatexcatcodetable\CatcodeTableIniTeX
    \begingroup
      \SetCatcodeRange{0}{8}{15}%
      \catcode9=10 % tab
      \catcode11=15 %
      \catcode12=13 % form feed
      \SetCatcodeRange{14}{31}{15}%
      \catcode35=6 % hash
      \catcode36=3 % dollar
      \catcode38=4 % ampersand
      \catcode94=7 % circumflex
      \catcode95=8 % underscore
      \catcode123=1 % brace left
      \catcode125=2 % brace right
      \catcode126=13 % tilde
      \catcode127=15 %
      \luatexsavecatcodetable\CatcodeTableLaTeX
    \endgroup
    \@makeother{0}% nul
    \@makeother{13}% carriage return
    \@makeother{37}% percent
    \@makeother{92}% backslash
    \@makeother{127}%
    \SetCatcodeRange{65}{90}{12}% A-Z
    \SetCatcodeRange{97}{122}{12}% a-z
    \luatexsavecatcodetable\CatcodeTableString
    \@makeother{32}% space
    \luatexsavecatcodetable\CatcodeTableOther
  \endgroup
}%
%    \end{macrocode}
%
% \subsubsection{Number stack}
%
%    \begin{macro}{\LuT@NumStackEmpty}
%    A special empty stack value because of \cs{@cdr}'s brace removal.
%    \begin{macrocode}
\def\LuT@NumStackEmpty{0}
%    \end{macrocode}
%    \end{macro}
%    \begin{macro}{\LuT@NumStack}
%    \begin{macrocode}
\let\LuT@NumStack\LuT@NumStackEmpty
%    \end{macrocode}
%    \end{macro}
%    \begin{macro}{\PushCatcodeTableNumStack}
%    \begin{macrocode}
\LuT@newcommand\PushCatcodeTableNumStack[0]{%
  \xdef\LuT@NumStack{%
    {\the\luatexcatcodetable}\LuT@NumStack
  }%
}
%    \end{macrocode}
%    \end{macro}
%    \begin{macro}{\PopCatcodeTableNumStack}
%    \begin{macrocode}
\LuT@newcommand\PopCatcodeTableNumStack[0]{%
  \ifx\LuT@NumStack\LuT@NumStackEmpty
    \@PackageWarning{luatex}{Empty catcode table number stack}%
    \luatexcatcodetable\z@
  \else
    \luatexcatcodetable=\expandafter\@car\LuT@NumStack\@nil\relax
    \xdef\LuT@NumStack{%
      \expandafter\@cdr\LuT@NumStack\@nil
    }%
  \fi
}
%    \end{macrocode}
%    \end{macro}
%
% \subsubsection{Catcode regime macros}
%
%    \begin{macro}{\BeginCatcodeRegime}
%    \begin{macrocode}
\LuT@newcommand\BeginCatcodeRegime[1]{%
  \PushCatcodeTableNumStack
  \luatexcatcodetable=\numexpr#1\relax
  \IncCatcodeTableStack
  \luatexsavecatcodetable\CatcodeTableStack
  \luatexcatcodetable\CatcodeTableStack
}
%    \end{macrocode}
%    \end{macro}
%
%    \begin{macro}{\EndCatcodeRegime}
%    \begin{macrocode}
\LuT@newcommand\EndCatcodeRegime[0]{%
  \DecCatcodeTableStack
  \PopCatcodeTableNumStack
}
%    \end{macrocode}
%    \end{macro}
%
% \subsection{Lua module loader}
%
%    \begin{macrocode}
\begingroup\expandafter\expandafter\expandafter\endgroup
\expandafter\ifx\csname RequirePackage\endcsname\relax
  \input luatex-loader.sty\relax
\else
  \RequirePackage{luatex-loader}[2010/03/09]%
\fi
%    \end{macrocode}
%    \begin{macrocode}
\LuT@AtEnd%
%</package>
%    \end{macrocode}
%
%    \begin{macrocode}
%<*loader>
%    \end{macrocode}
%
%    Reload check, especially if the package is not used with \LaTeX.
%    \begin{macrocode}
\begingroup\catcode61\catcode48\catcode32=10\relax%
  \catcode13=5 % ^^M
  \endlinechar=13 %
  \catcode35=6 % #
  \catcode39=12 % '
  \catcode44=12 % ,
  \catcode45=12 % -
  \catcode46=12 % .
  \catcode58=12 % :
  \catcode64=11 % @
  \catcode123=1 % {
  \catcode125=2 % }
  \expandafter\let\expandafter\x\csname ver@luatex-loader.sty\endcsname
  \ifx\x\relax % plain-TeX, first loading
  \else
    \def\empty{}%
    \ifx\x\empty % LaTeX, first loading,
      % variable is initialized, but \ProvidesPackage not yet seen
    \else
      \expandafter\ifx\csname PackageInfo\endcsname\relax
        \def\x#1#2{%
          \immediate\write-1{Package #1 Info: #2.}%
        }%
      \else
        \def\x#1#2{\PackageInfo{#1}{#2, stopped}}%
      \fi
      \x{luatex-loader}{The package is already loaded}%
      \aftergroup\endinput
    \fi
  \fi
\endgroup%
%    \end{macrocode}
%    Package identification:
%    \begin{macrocode}
\begingroup\catcode61\catcode48\catcode32=10\relax%
  \catcode13=5 % ^^M
  \endlinechar=13 %
  \catcode35=6 % #
  \catcode39=12 % '
  \catcode40=12 % (
  \catcode41=12 % )
  \catcode44=12 % ,
  \catcode45=12 % -
  \catcode46=12 % .
  \catcode47=12 % /
  \catcode58=12 % :
  \catcode64=11 % @
  \catcode91=12 % [
  \catcode93=12 % ]
  \catcode123=1 % {
  \catcode125=2 % }
  \expandafter\ifx\csname ProvidesPackage\endcsname\relax
    \def\x#1#2#3[#4]{\endgroup
      \immediate\write-1{Package: #3 #4}%
      \xdef#1{#4}%
    }%
  \else
    \def\x#1#2[#3]{\endgroup
      #2[{#3}]%
      \ifx#1\@undefined
        \xdef#1{#3}%
      \fi
      \ifx#1\relax
        \xdef#1{#3}%
      \fi
    }%
  \fi
\expandafter\x\csname ver@luatex-loader.sty\endcsname
\ProvidesPackage{luatex-loader}%
  [2010/03/09 v0.4 Lua module loader (HO)]%
%    \end{macrocode}
%    \begin{macrocode}
\begingroup\catcode61\catcode48\catcode32=10\relax%
  \catcode13=5\endlinechar=13\relax%
  \catcode10=12 % ^^J
  \catcode34=12 % "
  \catcode39=12 % '
  \catcode40=12 % (
  \catcode41=12 % )
  \catcode44=12 % ,
  \catcode46=12 % .
  \catcode60=12 % <
  \catcode61=12 % =
  \catcode95=12 % _ (other!)
  \catcode96=12 % `
  \catcode123=1 % {
  \catcode125=2 % }
  \endlinechar=10 %
  \ifnum\luatexversion<36 %
    \directlua0%
  \else %
    \expandafter\directlua %
  \fi %
  {%
     do
      local script = "oberdiek.luatex.lua"
      local file = kpse.find_file(script, "texmfscripts")
      if file then
        texio.write_nl("(" .. file .. ")")
        dofile(file)
      else
        error("File `" .. script .. "' not found")
      end
    end
  }%
\endgroup\endinput%
%    \end{macrocode}
%
%    \begin{macrocode}
%</loader>
%    \end{macrocode}
%
% \subsection{Lua script}
%
%    Currently \LuaTeX\ does not use KPSE when searching for module
%    files. The following Lua script implements a workaround.
%    It extends |package.loader| by another search method.
%    Modules are found by the module name with extension \xext{lua}
%    similar to
%    \begin{quote}
%      \ttfamily |kpsewhich --format=texmfscripts| \meta{module}|.lua|
%    \end{quote}
%    Unhappily \textsf{kpsewhich} does not support directory components
%    in the file name. Therefore a module |a.b.c| cannot be installed
%    as \xfile{a/b/c.lua}. The script must be named \xfile{a.b.c.lua}.
%    \begin{macrocode}
%<*lua>
%    \end{macrocode}
%    \begin{macrocode}
module("oberdiek.luatex", package.seeall)
%    \end{macrocode}
%    \begin{macrocode}
function kpse_module_loader(module)
  local script = module .. ".lua"
  local file = kpse.find_file(script, "texmfscripts")
  if file then
    local loader, error = loadfile(file)
    if loader then
      texio.write_nl("(" .. file .. ")")
      return loader
    end
    return "\n\t[oberdiek.luatex.kpse_module_loader] Loading error:\n\t"
           .. error
  end
  return "\n\t[oberdiek.luatex.kpse_module_loader] Search failed"
end
table.insert(package.loaders, kpse_module_loader)
%    \end{macrocode}
%    \begin{macrocode}
%</lua>
%    \end{macrocode}
%
% \section{Test}
%
%    \begin{macrocode}
%<*test2>
\documentclass{article}
\def\LoadCommand{%
  \RequirePackage{luatex}[2010/03/09]%
}
%</test2>
%    \end{macrocode}
%    \begin{macrocode}
%<*test3>
\documentclass{article}
\def\LoadCommand{%
  \RequirePackage{luatex-loader}[2010/03/09]%
}
%</test3>
%    \end{macrocode}
% \subsection{Catcode checks for loading}
%
%    \begin{macrocode}
%<*test1>
%    \end{macrocode}
%    \begin{macrocode}
\catcode`\{=1 %
\catcode`\}=2 %
\catcode`\#=6 %
\catcode`\@=11 %
\expandafter\ifx\csname count@\endcsname\relax
  \countdef\count@=255 %
\fi
\expandafter\ifx\csname @gobble\endcsname\relax
  \long\def\@gobble#1{}%
\fi
\expandafter\ifx\csname @firstofone\endcsname\relax
  \long\def\@firstofone#1{#1}%
\fi
\expandafter\ifx\csname loop\endcsname\relax
  \expandafter\@firstofone
\else
  \expandafter\@gobble
\fi
{%
  \def\loop#1\repeat{%
    \def\body{#1}%
    \iterate
  }%
  \def\iterate{%
    \body
      \let\next\iterate
    \else
      \let\next\relax
    \fi
    \next
  }%
  \let\repeat=\fi
}%
\def\RestoreCatcodes{}
\count@=0 %
\loop
  \edef\RestoreCatcodes{%
    \RestoreCatcodes
    \catcode\the\count@=\the\catcode\count@\relax
  }%
\ifnum\count@<255 %
  \advance\count@ 1 %
\repeat

\def\RangeCatcodeInvalid#1#2{%
  \count@=#1\relax
  \loop
    \catcode\count@=15 %
  \ifnum\count@<#2\relax
    \advance\count@ 1 %
  \repeat
}
\def\RangeCatcodeCheck#1#2#3{%
  \count@=#1\relax
  \loop
    \ifnum#3=\catcode\count@
    \else
      \errmessage{%
        Character \the\count@\space
        with wrong catcode \the\catcode\count@\space
        instead of \number#3%
      }%
    \fi
  \ifnum\count@<#2\relax
    \advance\count@ 1 %
  \repeat
}
\def\space{ }
\expandafter\ifx\csname LoadCommand\endcsname\relax
  \def\LoadCommand{\input luatex.sty\relax}%
\fi
\def\Test{%
  \RangeCatcodeInvalid{0}{47}%
  \RangeCatcodeInvalid{58}{64}%
  \RangeCatcodeInvalid{91}{96}%
  \RangeCatcodeInvalid{123}{255}%
  \catcode`\@=12 %
  \catcode`\\=0 %
  \catcode`\%=14 %
  \LoadCommand
  \RangeCatcodeCheck{0}{36}{15}%
  \RangeCatcodeCheck{37}{37}{14}%
  \RangeCatcodeCheck{38}{47}{15}%
  \RangeCatcodeCheck{48}{57}{12}%
  \RangeCatcodeCheck{58}{63}{15}%
  \RangeCatcodeCheck{64}{64}{12}%
  \RangeCatcodeCheck{65}{90}{11}%
  \RangeCatcodeCheck{91}{91}{15}%
  \RangeCatcodeCheck{92}{92}{0}%
  \RangeCatcodeCheck{93}{96}{15}%
  \RangeCatcodeCheck{97}{122}{11}%
  \RangeCatcodeCheck{123}{255}{15}%
  \RestoreCatcodes
}
\Test
\csname @@end\endcsname
\end
%    \end{macrocode}
%    \begin{macrocode}
%</test1>
%    \end{macrocode}
%
% \subsection{Catcode tables}
%
% \subsubsection{Predefined catcode tables}
%
%    \begin{macrocode}
%<*test4>
\NeedsTeXFormat{LaTeX2e}
%    \end{macrocode}
%    Remember \LaTeX's initial catcodes in count registers
%    starting at \cs{TestLaTeX}.
%    \begin{macrocode}
\count0=0 %
\chardef\TestLaTeX=1000 %
\chardef\TestMax=300 %
\loop
  \count\numexpr\TestLaTeX+\count0\relax=\catcode\count0 %
\ifnum\count0<\TestMax
  \advance\count0 by 1 %
\repeat
\documentclass{minimal}
\usepackage{luatex}[2010/03/09]
\usepackage{qstest}
\IncludeTests{*}
\LogTests{log}{*}{*}
\makeatletter
\def\Check#1{%
  \Expect*{\the\count@=\the\catcode\count@}%
         *{\the\count@=#1}%
}
\newcount\scratch
\def\Test#1#2{%
  \begin{qstest}{CatcodeTable#1}{CatcodeTable#1}%
    \luatexcatcodetable\csname CatcodeTable#1\endcsname
    \count@=\z@
    \loop
      \scratch=#2\relax
      \Expect*{\the\count@=\the\catcode\count@}%
             *{\the\count@=\the\scratch}%
    \ifnum\count@<\TestMax
      \advance\count@\@ne
    \repeat
  \end{qstest}%
}
\begingroup
  % luatex-unicode-letters.tex makes some slots to letters
  \def\TestMax{169}%
  \Test{LaTeX}{\the\count\numexpr\TestLaTeX+\count@}%
\endgroup
\Test{String}{\ifnum\count@=32 10\else 12\fi}
\Test{Other}{12}
\luatexinitcatcodetable99 %
\Test{IniTeX}{%
  0\relax
  \begingroup
    \luatexcatcodetable99 %
    \global\scratch=\the\catcode\count@
  \endgroup
}
%    \end{macrocode}
%
% \subsubsection{Catcode table number stack}
%
%    \begin{macrocode}
\begin{qstest}{CatcodeTableNumStack}{CatcodeTableNumStack}
  \def\TestStack#1{%
    \Expect*{\LuT@NumStack}{#1}%
  }%
  \TestStack{0}%
  \PushCatcodeTableNumStack
  \TestStack{{0}0}%
  \@firstofone{%
    \begingroup
      \luatexinitcatcodetable12 %
      \luatexcatcodetable12 %
      \PushCatcodeTableNumStack
      \TestStack{{12}{0}0}%
      \PopCatcodeTableNumStack
      \TestStack{{0}0}%
      \PopCatcodeTableNumStack
      \TestStack{0}%
      \def\TestWarning{Missing empty stack warning}%
      \def\@PackageWarning#1#2{\def\TestWarning{empty stack}}%
      \PopCatcodeTableNumStack
      \TestStack{0}%
      \Expect*{\TestWarning}{empty stack}%
    \endgroup
  }%
\end{qstest}
%    \end{macrocode}
%
% \subsubsection{Catcode table stack}
%
%    \begin{macrocode}
\begin{qstest}{CatcodeTableStack}{CatcodeTableStack}
  \def\TestStack#1{%
    \Expect*{\the\CatcodeTableStack}{#1}%
  }%
  \TestStack{0}%
  \IncCatcodeTableStack
  \TestStack{2}%
  \IncCatcodeTableStack
  \TestStack{4}%
  \begingroup
    \IncCatcodeTableStack
    \TestStack{6}%
  \endgroup
  \TestStack{6}%
  \begingroup
    \DecCatcodeTableStack
    \TestStack{4}%
  \endgroup
  \TestStack{4}%
  \DecCatcodeTableStack
  \TestStack{2}%
  \DecCatcodeTableStack
  \TestStack{0}%
  \begingroup
    \def\TestError{Missing error}%
    \def\@PackageError#1#2#3{%
      \def\TestError{Empty stack}%
    }%
    \DecCatcodeTableStack
    \TestStack{0}%
    \Expect*{\TestError}{Empty stack}%
  \endgroup
\end{qstest}
%    \end{macrocode}
%
% \subsubsection{Catcode regime macros}
%
%    \begin{macrocode}
\begin{qstest}{CatcodeRegime}{CatcodeRegime}
  \def\TestStacks#1#2#3{%
    \Expect*{\the\luatexcatcodetable}{#1}%
    \Expect*{\the\CatcodeTableStack}{#2}%
    \Expect*{\LuT@NumStack}{#3}%
  }%
  \TestStacks{0}{0}{0}%
  \catcode`\|=7 %
  \BeginCatcodeRegime\CatcodeTableLaTeX
    \TestStacks{2}{2}{{0}0}%
    \Expect*{\the\catcode`\|}{12}%
  \EndCatcodeRegime
  \TestStacks{0}{0}{0}%
  \Expect*{\the\catcode`\|}{7}%
\end{qstest}
%    \end{macrocode}
%
% \subsection{Attribute allocation}
%
%    \begin{macrocode}
\begin{qstest}{Attributes}{Attributes}
  \newattribute\TestAttr
  \Expect*{\meaning\TestAttr}%
         *{\string\attribute\number\allocationnumber}%
  \Expect*{\the\allocationnumber}{0}%
  \begingroup
    \newattribute\TestAttr
    \Expect*{\the\allocationnumber}{1}%
  \endgroup
  \Expect*{\the\allocationnumber}{0}%
  \Expect*{\meaning\TestAttr}*{\string\attribute1}%
  \Expect*{\the\TestAttr}*{\number\LuT@UnsetAttributeValue}%
  \def\Test#1{%
    \setattribute\TestAttr{#1}%
    \Expect*{\the\TestAttr}{#1}%
  }%
  \Test{0}%
  \Test{1}%
  \Test{-1}%
  \Test{123}%
  \unsetattribute\TestAttr
  \Expect*{\the\TestAttr}*{\number\LuT@UnsetAttributeValue}%
  \begingroup
    \Expect*{\the\TestAttr}*{\number\LuT@UnsetAttributeValue}%
    \Test{1234}%
  \endgroup
  \Expect*{\the\TestAttr}*{\number\LuT@UnsetAttributeValue}%
\end{qstest}
%    \end{macrocode}
%
%    \begin{macrocode}
\@@end
%</test4>
%    \end{macrocode}
%
% \subsection{Short test for \plainTeX}
%
%    \begin{macrocode}
%<*test5>
\input luatex.sty\relax
\newattribute\TestAttr
\setattribute\TestAttr{10}
\unsetattribute\TestAttr
\newcatcodetable\TestCTa
\begingroup
  \SetCatcodeRange{`A}{`Z}{12}%
\endgroup
\BeginCatcodeRegime\CatcodeTableLaTeX
\EndCatcodeRegime
\end
%</test5>
%    \end{macrocode}
%
% \section{Installation}
%
% \subsection{Download}
%
% \paragraph{Package.} This package is available on
% CTAN\footnote{\url{ftp://ftp.ctan.org/tex-archive/}}:
% \begin{description}
% \item[\CTAN{macros/latex/contrib/oberdiek/luatex.dtx}] The source file.
% \item[\CTAN{macros/latex/contrib/oberdiek/luatex.pdf}] Documentation.
% \end{description}
%
%
% \paragraph{Bundle.} All the packages of the bundle `oberdiek'
% are also available in a TDS compliant ZIP archive. There
% the packages are already unpacked and the documentation files
% are generated. The files and directories obey the TDS standard.
% \begin{description}
% \item[\CTAN{install/macros/latex/contrib/oberdiek.tds.zip}]
% \end{description}
% \emph{TDS} refers to the standard ``A Directory Structure
% for \TeX\ Files'' (\CTAN{tds/tds.pdf}). Directories
% with \xfile{texmf} in their name are usually organized this way.
%
% \subsection{Bundle installation}
%
% \paragraph{Unpacking.} Unpack the \xfile{oberdiek.tds.zip} in the
% TDS tree (also known as \xfile{texmf} tree) of your choice.
% Example (linux):
% \begin{quote}
%   |unzip oberdiek.tds.zip -d ~/texmf|
% \end{quote}
%
% \paragraph{Script installation.}
% Check the directory \xfile{TDS:scripts/oberdiek/} for
% scripts that need further installation steps.
% Package \xpackage{attachfile2} comes with the Perl script
% \xfile{pdfatfi.pl} that should be installed in such a way
% that it can be called as \texttt{pdfatfi}.
% Example (linux):
% \begin{quote}
%   |chmod +x scripts/oberdiek/pdfatfi.pl|\\
%   |cp scripts/oberdiek/pdfatfi.pl /usr/local/bin/|
% \end{quote}
%
% \subsection{Package installation}
%
% \paragraph{Unpacking.} The \xfile{.dtx} file is a self-extracting
% \docstrip\ archive. The files are extracted by running the
% \xfile{.dtx} through \plainTeX:
% \begin{quote}
%   \verb|tex luatex.dtx|
% \end{quote}
%
% \paragraph{TDS.} Now the different files must be moved into
% the different directories in your installation TDS tree
% (also known as \xfile{texmf} tree):
% \begin{quote}
% \def\t{^^A
% \begin{tabular}{@{}>{\ttfamily}l@{ $\rightarrow$ }>{\ttfamily}l@{}}
%   luatex.sty & tex/generic/oberdiek/luatex.sty\\
%   luatex-loader.sty & tex/generic/oberdiek/luatex-loader.sty\\
%   oberdiek.luatex.lua & scripts/oberdiek/oberdiek.luatex.lua\\
%   luatex.pdf & doc/latex/oberdiek/luatex.pdf\\
%   test/luatex-test1.tex & doc/latex/oberdiek/test/luatex-test1.tex\\
%   test/luatex-test2.tex & doc/latex/oberdiek/test/luatex-test2.tex\\
%   test/luatex-test3.tex & doc/latex/oberdiek/test/luatex-test3.tex\\
%   test/luatex-test4.tex & doc/latex/oberdiek/test/luatex-test4.tex\\
%   test/luatex-test5.tex & doc/latex/oberdiek/test/luatex-test5.tex\\
%   luatex.dtx & source/latex/oberdiek/luatex.dtx\\
% \end{tabular}^^A
% }^^A
% \sbox0{\t}^^A
% \ifdim\wd0>\linewidth
%   \begingroup
%     \advance\linewidth by\leftmargin
%     \advance\linewidth by\rightmargin
%   \edef\x{\endgroup
%     \def\noexpand\lw{\the\linewidth}^^A
%   }\x
%   \def\lwbox{^^A
%     \leavevmode
%     \hbox to \linewidth{^^A
%       \kern-\leftmargin\relax
%       \hss
%       \usebox0
%       \hss
%       \kern-\rightmargin\relax
%     }^^A
%   }^^A
%   \ifdim\wd0>\lw
%     \sbox0{\small\t}^^A
%     \ifdim\wd0>\linewidth
%       \ifdim\wd0>\lw
%         \sbox0{\footnotesize\t}^^A
%         \ifdim\wd0>\linewidth
%           \ifdim\wd0>\lw
%             \sbox0{\scriptsize\t}^^A
%             \ifdim\wd0>\linewidth
%               \ifdim\wd0>\lw
%                 \sbox0{\tiny\t}^^A
%                 \ifdim\wd0>\linewidth
%                   \lwbox
%                 \else
%                   \usebox0
%                 \fi
%               \else
%                 \lwbox
%               \fi
%             \else
%               \usebox0
%             \fi
%           \else
%             \lwbox
%           \fi
%         \else
%           \usebox0
%         \fi
%       \else
%         \lwbox
%       \fi
%     \else
%       \usebox0
%     \fi
%   \else
%     \lwbox
%   \fi
% \else
%   \usebox0
% \fi
% \end{quote}
% If you have a \xfile{docstrip.cfg} that configures and enables \docstrip's
% TDS installing feature, then some files can already be in the right
% place, see the documentation of \docstrip.
%
% \subsection{Refresh file name databases}
%
% If your \TeX~distribution
% (\teTeX, \mikTeX, \dots) relies on file name databases, you must refresh
% these. For example, \teTeX\ users run \verb|texhash| or
% \verb|mktexlsr|.
%
% \subsection{Some details for the interested}
%
% \paragraph{Attached source.}
%
% The PDF documentation on CTAN also includes the
% \xfile{.dtx} source file. It can be extracted by
% AcrobatReader 6 or higher. Another option is \textsf{pdftk},
% e.g. unpack the file into the current directory:
% \begin{quote}
%   \verb|pdftk luatex.pdf unpack_files output .|
% \end{quote}
%
% \paragraph{Unpacking with \LaTeX.}
% The \xfile{.dtx} chooses its action depending on the format:
% \begin{description}
% \item[\plainTeX:] Run \docstrip\ and extract the files.
% \item[\LaTeX:] Generate the documentation.
% \end{description}
% If you insist on using \LaTeX\ for \docstrip\ (really,
% \docstrip\ does not need \LaTeX), then inform the autodetect routine
% about your intention:
% \begin{quote}
%   \verb|latex \let\install=y\input{luatex.dtx}|
% \end{quote}
% Do not forget to quote the argument according to the demands
% of your shell.
%
% \paragraph{Generating the documentation.}
% You can use both the \xfile{.dtx} or the \xfile{.drv} to generate
% the documentation. The process can be configured by the
% configuration file \xfile{ltxdoc.cfg}. For instance, put this
% line into this file, if you want to have A4 as paper format:
% \begin{quote}
%   \verb|\PassOptionsToClass{a4paper}{article}|
% \end{quote}
% An example follows how to generate the
% documentation with pdf\LaTeX:
% \begin{quote}
%\begin{verbatim}
%pdflatex luatex.dtx
%makeindex -s gind.ist luatex.idx
%pdflatex luatex.dtx
%makeindex -s gind.ist luatex.idx
%pdflatex luatex.dtx
%\end{verbatim}
% \end{quote}
%
% \section{Catalogue}
%
% The following XML file can be used as source for the
% \href{http://mirror.ctan.org/help/Catalogue/catalogue.html}{\TeX\ Catalogue}.
% The elements \texttt{caption} and \texttt{description} are imported
% from the original XML file from the Catalogue.
% The name of the XML file in the Catalogue is \xfile{luatex.xml}.
%    \begin{macrocode}
%<*catalogue>
<?xml version='1.0' encoding='us-ascii'?>
<!DOCTYPE entry SYSTEM 'catalogue.dtd'>
<entry datestamp='$Date$' modifier='$Author$' id='luatex'>
  <name>luatex</name>
  <caption>The LuaTeX engine.</caption>
  <authorref id='auth:oberdiek'/>
  <copyright owner='Heiko Oberdiek' year='2007,2009,2010'/>
  <license type='lppl1.3'/>
  <version number='0.4'/>
  <description>
    LuaTeX is an extended version of pdfTeX using Lua as an embedded
    scripting language.  The LuaTeX project&#x2019;s main objective
    is to provide an open and configurable variant of TeX while at the
    same time offering downward compatibility.
    <p/>
    LuaTeX uses Unicode (as UTF-8) as its default input encoding, and
    is able to use modern (OpenType) fonts (for both text and mathematics).
    <p/>
    It should be noted that LuaTeX is still under development; its
    specification has been declared stable, but absolute stability
    may not in practice be assumed.
    <p/>
    The package is part of the <xref refid='oberdiek'>oberdiek</xref> bundle.
  </description>
  <documentation details='Package documentation'
      href='ctan:/macros/latex/contrib/oberdiek/luatex.pdf'/>
  <ctan file='true' path='/macros/latex/contrib/oberdiek/luatex.dtx'/>
  <miktex location='oberdiek'/>
  <texlive location='oberdiek'/>
  <install path='/macros/latex/contrib/oberdiek/oberdiek.tds.zip'/>
</entry>
%</catalogue>
%    \end{macrocode}
%
% \begin{History}
%   \begin{Version}{2007/12/12 v0.1}
%   \item
%     First public version.
%   \end{Version}
%   \begin{Version}{2009/04/10 v0.2}
%   \item
%     Requires package \xpackage{ifluatex} in version 2.0 to ensure
%     \cs{luatexversion}.
%   \item
%     Updates the call of \cs{directlua}, the syntax has changed
%     in \LuaTeX\ 0.36.
%   \end{Version}
%   \begin{Version}{2009/12/02 v0.3}
%   \item
%     Unsetting of attributes updated for \LuaTeX\ 0.37.
%   \end{Version}
%   \begin{Version}{2010/03/09 v0.4}
%   \item
%     Support for lua states removed.
%   \item
%     Calling tex.enableprimitives for used primitives.
%   \end{Version}
% \end{History}
%
% \PrintIndex
%
% \Finale
\endinput
|
% \end{quote}
% Do not forget to quote the argument according to the demands
% of your shell.
%
% \paragraph{Generating the documentation.}
% You can use both the \xfile{.dtx} or the \xfile{.drv} to generate
% the documentation. The process can be configured by the
% configuration file \xfile{ltxdoc.cfg}. For instance, put this
% line into this file, if you want to have A4 as paper format:
% \begin{quote}
%   \verb|\PassOptionsToClass{a4paper}{article}|
% \end{quote}
% An example follows how to generate the
% documentation with pdf\LaTeX:
% \begin{quote}
%\begin{verbatim}
%pdflatex luatex.dtx
%makeindex -s gind.ist luatex.idx
%pdflatex luatex.dtx
%makeindex -s gind.ist luatex.idx
%pdflatex luatex.dtx
%\end{verbatim}
% \end{quote}
%
% \section{Catalogue}
%
% The following XML file can be used as source for the
% \href{http://mirror.ctan.org/help/Catalogue/catalogue.html}{\TeX\ Catalogue}.
% The elements \texttt{caption} and \texttt{description} are imported
% from the original XML file from the Catalogue.
% The name of the XML file in the Catalogue is \xfile{luatex.xml}.
%    \begin{macrocode}
%<*catalogue>
<?xml version='1.0' encoding='us-ascii'?>
<!DOCTYPE entry SYSTEM 'catalogue.dtd'>
<entry datestamp='$Date$' modifier='$Author$' id='luatex'>
  <name>luatex</name>
  <caption>The LuaTeX engine.</caption>
  <authorref id='auth:oberdiek'/>
  <copyright owner='Heiko Oberdiek' year='2007,2009,2010'/>
  <license type='lppl1.3'/>
  <version number='0.4'/>
  <description>
    LuaTeX is an extended version of pdfTeX using Lua as an embedded
    scripting language.  The LuaTeX project&#x2019;s main objective
    is to provide an open and configurable variant of TeX while at the
    same time offering downward compatibility.
    <p/>
    LuaTeX uses Unicode (as UTF-8) as its default input encoding, and
    is able to use modern (OpenType) fonts (for both text and mathematics).
    <p/>
    It should be noted that LuaTeX is still under development; its
    specification has been declared stable, but absolute stability
    may not in practice be assumed.
    <p/>
    The package is part of the <xref refid='oberdiek'>oberdiek</xref> bundle.
  </description>
  <documentation details='Package documentation'
      href='ctan:/macros/latex/contrib/oberdiek/luatex.pdf'/>
  <ctan file='true' path='/macros/latex/contrib/oberdiek/luatex.dtx'/>
  <miktex location='oberdiek'/>
  <texlive location='oberdiek'/>
  <install path='/macros/latex/contrib/oberdiek/oberdiek.tds.zip'/>
</entry>
%</catalogue>
%    \end{macrocode}
%
% \begin{History}
%   \begin{Version}{2007/12/12 v0.1}
%   \item
%     First public version.
%   \end{Version}
%   \begin{Version}{2009/04/10 v0.2}
%   \item
%     Requires package \xpackage{ifluatex} in version 2.0 to ensure
%     \cs{luatexversion}.
%   \item
%     Updates the call of \cs{directlua}, the syntax has changed
%     in \LuaTeX\ 0.36.
%   \end{Version}
%   \begin{Version}{2009/12/02 v0.3}
%   \item
%     Unsetting of attributes updated for \LuaTeX\ 0.37.
%   \end{Version}
%   \begin{Version}{2010/03/09 v0.4}
%   \item
%     Support for lua states removed.
%   \item
%     Calling tex.enableprimitives for used primitives.
%   \end{Version}
% \end{History}
%
% \PrintIndex
%
% \Finale
\endinput
|
% \end{quote}
% Do not forget to quote the argument according to the demands
% of your shell.
%
% \paragraph{Generating the documentation.}
% You can use both the \xfile{.dtx} or the \xfile{.drv} to generate
% the documentation. The process can be configured by the
% configuration file \xfile{ltxdoc.cfg}. For instance, put this
% line into this file, if you want to have A4 as paper format:
% \begin{quote}
%   \verb|\PassOptionsToClass{a4paper}{article}|
% \end{quote}
% An example follows how to generate the
% documentation with pdf\LaTeX:
% \begin{quote}
%\begin{verbatim}
%pdflatex luatex.dtx
%makeindex -s gind.ist luatex.idx
%pdflatex luatex.dtx
%makeindex -s gind.ist luatex.idx
%pdflatex luatex.dtx
%\end{verbatim}
% \end{quote}
%
% \section{Catalogue}
%
% The following XML file can be used as source for the
% \href{http://mirror.ctan.org/help/Catalogue/catalogue.html}{\TeX\ Catalogue}.
% The elements \texttt{caption} and \texttt{description} are imported
% from the original XML file from the Catalogue.
% The name of the XML file in the Catalogue is \xfile{luatex.xml}.
%    \begin{macrocode}
%<*catalogue>
<?xml version='1.0' encoding='us-ascii'?>
<!DOCTYPE entry SYSTEM 'catalogue.dtd'>
<entry datestamp='$Date$' modifier='$Author$' id='luatex'>
  <name>luatex</name>
  <caption>The LuaTeX engine.</caption>
  <authorref id='auth:oberdiek'/>
  <copyright owner='Heiko Oberdiek' year='2007,2009,2010'/>
  <license type='lppl1.3'/>
  <version number='0.4'/>
  <description>
    LuaTeX is an extended version of pdfTeX using Lua as an embedded
    scripting language.  The LuaTeX project&#x2019;s main objective
    is to provide an open and configurable variant of TeX while at the
    same time offering downward compatibility.
    <p/>
    LuaTeX uses Unicode (as UTF-8) as its default input encoding, and
    is able to use modern (OpenType) fonts (for both text and mathematics).
    <p/>
    It should be noted that LuaTeX is still under development; its
    specification has been declared stable, but absolute stability
    may not in practice be assumed.
    <p/>
    The package is part of the <xref refid='oberdiek'>oberdiek</xref> bundle.
  </description>
  <documentation details='Package documentation'
      href='ctan:/macros/latex/contrib/oberdiek/luatex.pdf'/>
  <ctan file='true' path='/macros/latex/contrib/oberdiek/luatex.dtx'/>
  <miktex location='oberdiek'/>
  <texlive location='oberdiek'/>
  <install path='/macros/latex/contrib/oberdiek/oberdiek.tds.zip'/>
</entry>
%</catalogue>
%    \end{macrocode}
%
% \begin{History}
%   \begin{Version}{2007/12/12 v0.1}
%   \item
%     First public version.
%   \end{Version}
%   \begin{Version}{2009/04/10 v0.2}
%   \item
%     Requires package \xpackage{ifluatex} in version 2.0 to ensure
%     \cs{luatexversion}.
%   \item
%     Updates the call of \cs{directlua}, the syntax has changed
%     in \LuaTeX\ 0.36.
%   \end{Version}
%   \begin{Version}{2009/12/02 v0.3}
%   \item
%     Unsetting of attributes updated for \LuaTeX\ 0.37.
%   \end{Version}
%   \begin{Version}{2010/03/09 v0.4}
%   \item
%     Support for lua states removed.
%   \item
%     Calling tex.enableprimitives for used primitives.
%   \end{Version}
% \end{History}
%
% \PrintIndex
%
% \Finale
\endinput
|
% \end{quote}
% Do not forget to quote the argument according to the demands
% of your shell.
%
% \paragraph{Generating the documentation.}
% You can use both the \xfile{.dtx} or the \xfile{.drv} to generate
% the documentation. The process can be configured by the
% configuration file \xfile{ltxdoc.cfg}. For instance, put this
% line into this file, if you want to have A4 as paper format:
% \begin{quote}
%   \verb|\PassOptionsToClass{a4paper}{article}|
% \end{quote}
% An example follows how to generate the
% documentation with pdf\LaTeX:
% \begin{quote}
%\begin{verbatim}
%pdflatex luatex.dtx
%makeindex -s gind.ist luatex.idx
%pdflatex luatex.dtx
%makeindex -s gind.ist luatex.idx
%pdflatex luatex.dtx
%\end{verbatim}
% \end{quote}
%
% \section{Catalogue}
%
% The following XML file can be used as source for the
% \href{http://mirror.ctan.org/help/Catalogue/catalogue.html}{\TeX\ Catalogue}.
% The elements \texttt{caption} and \texttt{description} are imported
% from the original XML file from the Catalogue.
% The name of the XML file in the Catalogue is \xfile{luatex.xml}.
%    \begin{macrocode}
%<*catalogue>
<?xml version='1.0' encoding='us-ascii'?>
<!DOCTYPE entry SYSTEM 'catalogue.dtd'>
<entry datestamp='$Date$' modifier='$Author$' id='luatex'>
  <name>luatex</name>
  <caption>The LuaTeX engine.</caption>
  <authorref id='auth:oberdiek'/>
  <copyright owner='Heiko Oberdiek' year='2007,2009,2010'/>
  <license type='lppl1.3'/>
  <version number='0.4'/>
  <description>
    LuaTeX is an extended version of pdfTeX using Lua as an embedded
    scripting language.  The LuaTeX project&#x2019;s main objective
    is to provide an open and configurable variant of TeX while at the
    same time offering downward compatibility.
    <p/>
    LuaTeX uses Unicode (as UTF-8) as its default input encoding, and
    is able to use modern (OpenType) fonts (for both text and mathematics).
    <p/>
    It should be noted that LuaTeX is still under development; its
    specification has been declared stable, but absolute stability
    may not in practice be assumed.
    <p/>
    The package is part of the <xref refid='oberdiek'>oberdiek</xref> bundle.
  </description>
  <documentation details='Package documentation'
      href='ctan:/macros/latex/contrib/oberdiek/luatex.pdf'/>
  <ctan file='true' path='/macros/latex/contrib/oberdiek/luatex.dtx'/>
  <miktex location='oberdiek'/>
  <texlive location='oberdiek'/>
  <install path='/macros/latex/contrib/oberdiek/oberdiek.tds.zip'/>
</entry>
%</catalogue>
%    \end{macrocode}
%
% \begin{History}
%   \begin{Version}{2007/12/12 v0.1}
%   \item
%     First public version.
%   \end{Version}
%   \begin{Version}{2009/04/10 v0.2}
%   \item
%     Requires package \xpackage{ifluatex} in version 2.0 to ensure
%     \cs{luatexversion}.
%   \item
%     Updates the call of \cs{directlua}, the syntax has changed
%     in \LuaTeX\ 0.36.
%   \end{Version}
%   \begin{Version}{2009/12/02 v0.3}
%   \item
%     Unsetting of attributes updated for \LuaTeX\ 0.37.
%   \end{Version}
%   \begin{Version}{2010/03/09 v0.4}
%   \item
%     Support for lua states removed.
%   \item
%     Calling tex.enableprimitives for used primitives.
%   \end{Version}
% \end{History}
%
% \PrintIndex
%
% \Finale
\endinput
