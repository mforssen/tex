% \iffalse meta-comment
%
% File: hypcap.dtx
% Version: 2011/02/16 v1.11
% Info: Adjusting the anchors of captions
%
% Copyright (C) 1999-2001, 2006-2008, 2011 by
%    Heiko Oberdiek <heiko.oberdiek at googlemail.com>
%
% This work may be distributed and/or modified under the
% conditions of the LaTeX Project Public License, either
% version 1.3c of this license or (at your option) any later
% version. This version of this license is in
%    http://www.latex-project.org/lppl/lppl-1-3c.txt
% and the latest version of this license is in
%    http://www.latex-project.org/lppl.txt
% and version 1.3 or later is part of all distributions of
% LaTeX version 2005/12/01 or later.
%
% This work has the LPPL maintenance status "maintained".
%
% This Current Maintainer of this work is Heiko Oberdiek.
%
% This work consists of the main source file hypcap.dtx
% and the derived files
%    hypcap.sty, hypcap.pdf, hypcap.ins, hypcap.drv.
%
% Distribution:
%    CTAN:macros/latex/contrib/oberdiek/hypcap.dtx
%    CTAN:macros/latex/contrib/oberdiek/hypcap.pdf
%
% Unpacking:
%    (a) If hypcap.ins is present:
%           tex hypcap.ins
%    (b) Without hypcap.ins:
%           tex hypcap.dtx
%    (c) If you insist on using LaTeX
%           latex \let\install=y% \iffalse meta-comment
%
% File: hypcap.dtx
% Version: 2011/02/16 v1.11
% Info: Adjusting the anchors of captions
%
% Copyright (C) 1999-2001, 2006-2008, 2011 by
%    Heiko Oberdiek <heiko.oberdiek at googlemail.com>
%
% This work may be distributed and/or modified under the
% conditions of the LaTeX Project Public License, either
% version 1.3c of this license or (at your option) any later
% version. This version of this license is in
%    http://www.latex-project.org/lppl/lppl-1-3c.txt
% and the latest version of this license is in
%    http://www.latex-project.org/lppl.txt
% and version 1.3 or later is part of all distributions of
% LaTeX version 2005/12/01 or later.
%
% This work has the LPPL maintenance status "maintained".
%
% This Current Maintainer of this work is Heiko Oberdiek.
%
% This work consists of the main source file hypcap.dtx
% and the derived files
%    hypcap.sty, hypcap.pdf, hypcap.ins, hypcap.drv.
%
% Distribution:
%    CTAN:macros/latex/contrib/oberdiek/hypcap.dtx
%    CTAN:macros/latex/contrib/oberdiek/hypcap.pdf
%
% Unpacking:
%    (a) If hypcap.ins is present:
%           tex hypcap.ins
%    (b) Without hypcap.ins:
%           tex hypcap.dtx
%    (c) If you insist on using LaTeX
%           latex \let\install=y% \iffalse meta-comment
%
% File: hypcap.dtx
% Version: 2011/02/16 v1.11
% Info: Adjusting the anchors of captions
%
% Copyright (C) 1999-2001, 2006-2008, 2011 by
%    Heiko Oberdiek <heiko.oberdiek at googlemail.com>
%
% This work may be distributed and/or modified under the
% conditions of the LaTeX Project Public License, either
% version 1.3c of this license or (at your option) any later
% version. This version of this license is in
%    http://www.latex-project.org/lppl/lppl-1-3c.txt
% and the latest version of this license is in
%    http://www.latex-project.org/lppl.txt
% and version 1.3 or later is part of all distributions of
% LaTeX version 2005/12/01 or later.
%
% This work has the LPPL maintenance status "maintained".
%
% This Current Maintainer of this work is Heiko Oberdiek.
%
% This work consists of the main source file hypcap.dtx
% and the derived files
%    hypcap.sty, hypcap.pdf, hypcap.ins, hypcap.drv.
%
% Distribution:
%    CTAN:macros/latex/contrib/oberdiek/hypcap.dtx
%    CTAN:macros/latex/contrib/oberdiek/hypcap.pdf
%
% Unpacking:
%    (a) If hypcap.ins is present:
%           tex hypcap.ins
%    (b) Without hypcap.ins:
%           tex hypcap.dtx
%    (c) If you insist on using LaTeX
%           latex \let\install=y% \iffalse meta-comment
%
% File: hypcap.dtx
% Version: 2011/02/16 v1.11
% Info: Adjusting the anchors of captions
%
% Copyright (C) 1999-2001, 2006-2008, 2011 by
%    Heiko Oberdiek <heiko.oberdiek at googlemail.com>
%
% This work may be distributed and/or modified under the
% conditions of the LaTeX Project Public License, either
% version 1.3c of this license or (at your option) any later
% version. This version of this license is in
%    http://www.latex-project.org/lppl/lppl-1-3c.txt
% and the latest version of this license is in
%    http://www.latex-project.org/lppl.txt
% and version 1.3 or later is part of all distributions of
% LaTeX version 2005/12/01 or later.
%
% This work has the LPPL maintenance status "maintained".
%
% This Current Maintainer of this work is Heiko Oberdiek.
%
% This work consists of the main source file hypcap.dtx
% and the derived files
%    hypcap.sty, hypcap.pdf, hypcap.ins, hypcap.drv.
%
% Distribution:
%    CTAN:macros/latex/contrib/oberdiek/hypcap.dtx
%    CTAN:macros/latex/contrib/oberdiek/hypcap.pdf
%
% Unpacking:
%    (a) If hypcap.ins is present:
%           tex hypcap.ins
%    (b) Without hypcap.ins:
%           tex hypcap.dtx
%    (c) If you insist on using LaTeX
%           latex \let\install=y\input{hypcap.dtx}
%        (quote the arguments according to the demands of your shell)
%
% Documentation:
%    (a) If hypcap.drv is present:
%           latex hypcap.drv
%    (b) Without hypcap.drv:
%           latex hypcap.dtx; ...
%    The class ltxdoc loads the configuration file ltxdoc.cfg
%    if available. Here you can specify further options, e.g.
%    use A4 as paper format:
%       \PassOptionsToClass{a4paper}{article}
%
%    Programm calls to get the documentation (example):
%       pdflatex hypcap.dtx
%       makeindex -s gind.ist hypcap.idx
%       pdflatex hypcap.dtx
%       makeindex -s gind.ist hypcap.idx
%       pdflatex hypcap.dtx
%
% Installation:
%    TDS:tex/latex/oberdiek/hypcap.sty
%    TDS:doc/latex/oberdiek/hypcap.pdf
%    TDS:source/latex/oberdiek/hypcap.dtx
%
%<*ignore>
\begingroup
  \catcode123=1 %
  \catcode125=2 %
  \def\x{LaTeX2e}%
\expandafter\endgroup
\ifcase 0\ifx\install y1\fi\expandafter
         \ifx\csname processbatchFile\endcsname\relax\else1\fi
         \ifx\fmtname\x\else 1\fi\relax
\else\csname fi\endcsname
%</ignore>
%<*install>
\input docstrip.tex
\Msg{************************************************************************}
\Msg{* Installation}
\Msg{* Package: hypcap 2011/02/16 v1.11 Adjusting the anchors of captions (HO)}
\Msg{************************************************************************}

\keepsilent
\askforoverwritefalse

\let\MetaPrefix\relax
\preamble

This is a generated file.

Project: hypcap
Version: 2011/02/16 v1.11

Copyright (C) 1999-2001, 2006-2008, 2011 by
   Heiko Oberdiek <heiko.oberdiek at googlemail.com>

This work may be distributed and/or modified under the
conditions of the LaTeX Project Public License, either
version 1.3c of this license or (at your option) any later
version. This version of this license is in
   http://www.latex-project.org/lppl/lppl-1-3c.txt
and the latest version of this license is in
   http://www.latex-project.org/lppl.txt
and version 1.3 or later is part of all distributions of
LaTeX version 2005/12/01 or later.

This work has the LPPL maintenance status "maintained".

This Current Maintainer of this work is Heiko Oberdiek.

This work consists of the main source file hypcap.dtx
and the derived files
   hypcap.sty, hypcap.pdf, hypcap.ins, hypcap.drv.

\endpreamble
\let\MetaPrefix\DoubleperCent

\generate{%
  \file{hypcap.ins}{\from{hypcap.dtx}{install}}%
  \file{hypcap.drv}{\from{hypcap.dtx}{driver}}%
  \usedir{tex/latex/oberdiek}%
  \file{hypcap.sty}{\from{hypcap.dtx}{package}}%
  \nopreamble
  \nopostamble
  \usedir{source/latex/oberdiek/catalogue}%
  \file{hypcap.xml}{\from{hypcap.dtx}{catalogue}}%
}

\catcode32=13\relax% active space
\let =\space%
\Msg{************************************************************************}
\Msg{*}
\Msg{* To finish the installation you have to move the following}
\Msg{* file into a directory searched by TeX:}
\Msg{*}
\Msg{*     hypcap.sty}
\Msg{*}
\Msg{* To produce the documentation run the file `hypcap.drv'}
\Msg{* through LaTeX.}
\Msg{*}
\Msg{* Happy TeXing!}
\Msg{*}
\Msg{************************************************************************}

\endbatchfile
%</install>
%<*ignore>
\fi
%</ignore>
%<*driver>
\NeedsTeXFormat{LaTeX2e}
\ProvidesFile{hypcap.drv}%
  [2011/02/16 v1.11 Adjusting the anchors of captions (HO)]%
\documentclass{ltxdoc}
\usepackage{holtxdoc}[2011/11/22]
\begin{document}
  \DocInput{hypcap.dtx}%
\end{document}
%</driver>
% \fi
%
% \CheckSum{174}
%
% \CharacterTable
%  {Upper-case    \A\B\C\D\E\F\G\H\I\J\K\L\M\N\O\P\Q\R\S\T\U\V\W\X\Y\Z
%   Lower-case    \a\b\c\d\e\f\g\h\i\j\k\l\m\n\o\p\q\r\s\t\u\v\w\x\y\z
%   Digits        \0\1\2\3\4\5\6\7\8\9
%   Exclamation   \!     Double quote  \"     Hash (number) \#
%   Dollar        \$     Percent       \%     Ampersand     \&
%   Acute accent  \'     Left paren    \(     Right paren   \)
%   Asterisk      \*     Plus          \+     Comma         \,
%   Minus         \-     Point         \.     Solidus       \/
%   Colon         \:     Semicolon     \;     Less than     \<
%   Equals        \=     Greater than  \>     Question mark \?
%   Commercial at \@     Left bracket  \[     Backslash     \\
%   Right bracket \]     Circumflex    \^     Underscore    \_
%   Grave accent  \`     Left brace    \{     Vertical bar  \|
%   Right brace   \}     Tilde         \~}
%
% \GetFileInfo{hypcap.drv}
%
% \title{The \xpackage{hypcap} package}
% \date{2011/02/16 v1.11}
% \author{Heiko Oberdiek\\\xemail{heiko.oberdiek at googlemail.com}}
%
% \maketitle
%
% \begin{abstract}
% This package tries a solution of the problem with
% hyperref, that links to floats points below the
% caption and not at the beginning of the float.
% Therefore this package divides the task into two
% part, the link setting with \cs{capstart} or
% automatically at the beginning of a float and
% the rest in the \cs{caption} command.
% \end{abstract}
%
% \tableofcontents
%
% \newpage
% \section{Usage}
%    The package \Package{hypcap} requires that \Package{hyperref}
%    is loaded first:
%    \begin{quote}
%      |\usepackage[...]{hyperref}|\\
%      |\usepackage[...]{hypcap}|
%    \end{quote}
%
% \subsection{Package options}
%    The names of the four float environments |figure|, |figure*|,
%    |table|, or |table*| can be used as option. Then the package
%    redefines the environment in order to insert \cmd{\capstart}
%    (see below) in the beginning of the environment automatically.
%
%    Option |all| enables the redefinitions of all four
%    float environments. For other environments see
%    the user command \cmd{\hypcapredef}.
%
% \subsection{User commands}
%    \begin{description}
%    \item[\cmd{\capstart}:]\DescribeMacro{\capstart}
%      First this command increments the counter (\cmd{\@captype}).
%      Then it makes an anchor for package \Package{hyperref}.
%      At last \cmd{\caption} is redefined to remove the
%      anchor setting part from \Package{hyperref}'s \cmd{\caption}.
%
%      The package expects the following structure of a float
%      environment:
%      \begin{quote}
%        |\begin{|\textit{|float|}|}...|\\
%        |\capstart|\\
%        |...|\\
%        |\caption{...}|\\
%        |...|\\
%        |\end{|\textit{|float|}|}|
%      \end{quote}
%      There can be several \cmd{\caption} commands. For these
%      you need \cmd{\capstart} again:
%      \begin{quote}
%        |\capstart ... \caption... \capstart ... \caption...|
%      \end{quote}
%      And the \cmd{\caption} command itself can be put in a group.
%
%      With the options, described above, the extra
%      writing of \cmd{\capstart} can be avoided.
%      Consequently, there must be a \cmd{\caption} in every
%      environment of this type, specified by the option.
%      If you want to use more than one \cmd{\caption} in this
%      environment, you have to state \cmd{\capstart} again.
%    \item[\cmd{\hypcapspace}:]\DescribeMacro{\hypcapspace}
%      Because it looks poor, if the link points exactly at top of
%      the figure, there is additional space: \cmd{\hypcapspace},
%      the default is |0.5\baselineskip|, examples:
%      \begin{quote}
%        |\renewcommand{\hypcapspace}{0pt}| removes the space\\
%        |\renewcommand{\hypcapspace}{1pt}| sets a fix value
%      \end{quote}
%    \item[\cmd{\hypcapredef}:]\DescribeMacro{\hypcapredef}
%      If there are other float environments, that
%      should automatically execute \cmd{\capstart}, then
%      a redefinition with \cmd{\hypcapredef} can be tried:
%      \begin{quote}
%        |\hypcapredef{myfloat}|
%      \end{quote}
%      Only environments with one optional parameter are
%      supported.
%    \item[\cs{capstartfalse}, \cs{capstarttrue}:]^^A
%      \DescribeMacro{\capstartfalse}^^A
%      \DescribeMacro{\capstarttrue}^^A
%      Since 2008/09/08 v1.10.\\
%      They disable and enable \cs{capstart}. They can be used to
%      cancel the effect of a redefined float environment.
%      Example:
%\begin{quote}
%\catcode`Y=\active
%\def Y#1Z{\textit{\cs{#1}}}
%\begin{verbatim}
%\documentclass{article}
%\usepackage{hyperref}
%\usepackage[figure]{hypcap}[2008/09/08]
%
%\begin{document}
%  \section{Hello World}
%  \begin{figure}
%    \caption{Figure with caption A}
%  \end{figure}
%  YcapstartfalseZ
%  \begin{figure}
%    Figure without caption
%  \end{figure}
%  YcapstarttrueZ
%  \begin{figure}
%    \caption{Figure with caption B}
%  \end{figure}
%\end{document}
%\end{verbatim}
%\end{quote}
%    \end{description}
%
% \subsection{Limitations}
%    \begin{itemize}
%    \item Packages that redefine \cmd{\caption} or
%          \cmd{\@caption}.
%    \end{itemize}
%
% \StopEventually{
% }
%
% \section{Implementation}
%    \begin{macrocode}
%<*package>
%    \end{macrocode}
%    Package identification.
%    \begin{macrocode}
\NeedsTeXFormat{LaTeX2e}
\ProvidesPackage{hypcap}%
  [2011/02/16 v1.11 Adjusting the anchors of captions (HO)]
%    \end{macrocode}
%
%    For unique command names this package uses |hc@| as prefix
%    for internal command names.
%
%    First we check, if package \Package{hyperref} is loaded:
%    \begin{macrocode}
\@ifundefined{hyper@@anchor}{%
  \PackageError{hypcap}{You have to load 'hyperref' first}\@ehc
  \endinput
}{}
%    \end{macrocode}
%
%    \begin{macrocode}
\RequirePackage{letltxmacro}[2008/06/24]
%    \end{macrocode}
%
%    \begin{macro}{\hc@org@caption}
%    Save the original meaning of \cmd{\caption}:
%    \begin{macrocode}
\newcommand*\hc@org@caption{}
\let\hc@org@caption\caption
%    \end{macrocode}
%    \end{macro}
%
%    \newcommand*{\BeginMacro}[1]{\expandafter\DoBeginMacro\csname#1\endcsname}
%    \newcommand*{\DoBeginMacro}[1]{\begin{macro}{#1}}
%    \newcommand*{\Cmd}[1]{\expandafter\cmd\csname#1\endcsname}
%    \BeginMacro{if@capstart}
%     The switch \Cmd{if@capstart} helps to detect
%     \cmd{\capstart} commands with missing \cmd{\caption} macros.
%     Because \cmd{\caption} can occur inside a group, assignments
%     to the switch have to be made global.
%    \begin{macrocode}
\newif\if@capstart
%    \end{macrocode}
%    \end{macro}
%
%    \begin{macro}{\hypcapspace}
%    The anchor is raised.by \cmd{\hypcapspace}.
%    \begin{macrocode}
\newcommand*\hypcapspace{.5\baselineskip}
%    \end{macrocode}
%    \end{macro}
%
%    \begin{macro}{\ifcapstart}
%    \begin{macrocode}
\newif\ifcapstart
\capstarttrue
%    \end{macrocode}
%    \end{macro}
%
%    \begin{macro}{\capstart}
%     The macro \cmd{\capstart} contains the first part of
%     the \cmd{\caption} command: Incrementing the counter
%     and setting the anchor.
%    \begin{macrocode}
\newcommand*\capstart{%
  \ifcapstart
    \H@refstepcounter\@captype % first part of caption
    \hyper@makecurrent\@captype
    \global\let\hc@currentHref\@currentHref
    \vspace*{-\hypcapspace}%
    \begingroup
      \let\leavevmode\relax
      \hyper@@anchor\@currentHref\relax
    \endgroup
    \vspace*{\hypcapspace}%
    \hc@hyperref{\let\caption\hc@caption}%
    \global\@capstarttrue
    \global\advance\csname c@\@captype\endcsname\m@ne
  \fi
}
%    \end{macrocode}
%    \end{macro}
%
%    \begin{macrocode}
\@ifpackagelater{hyperref}{2007/04/09}{%
  \let\hc@hyperref\@gobble
}{%
  \let\hc@hyperref\@firstofone
}
%    \end{macrocode}
%
%    \begin{macro}{\hc@caption}
%    The new \cmd{\caption} command without the first part
%    is defined in the macro \cmd{\hc@caption}.
%    \begin{macrocode}
\def\hc@caption{%
  \global\advance\csname c@\@captype\endcsname\@ne
  \@dblarg{\hc@@caption\@captype}%
}
%    \end{macrocode}
%    \end{macro}
%
%    \begin{macro}{\hc@@caption}
%    This is a copy of package \Package{hyperref}'s \cmd{\@caption}
%    macro without making the anchor, because this is already done
%    in \cmd{\capstart}.
%    \begin{macrocode}
\long\def\hc@@caption#1[#2]#3{%
  \let\caption\hc@org@caption
  \global\@capstartfalse
  \ifHy@hypertexnames
    \hyper@makecurrent\@captype
  \else
    \global\let\@currentHref\hc@currentHref
  \fi
  \par\addcontentsline{%
    \csname ext@#1\endcsname}{#1}{%
    \protect\numberline{%
      \csname the#1\endcsname
    }{\ignorespaces #2}%
  }%
  \begingroup
    \@parboxrestore
    \normalsize
    \@makecaption{\csname fnum@#1\endcsname}{%
      \ignorespaces#3%
    }%
    \par
  \endgroup
}
%    \end{macrocode}
%    \end{macro}
%
%    \begin{macro}{\hypcapredef}
%    The macro \cmd{\hypcapredef} prepares the call of
%    \cmd{\hc@redef} that will redefine the environment
%    that is given in the argument.
%    \begin{macrocode}
\def\hypcapredef#1{%
  \expandafter\hc@redef\csname hc@org#1\expandafter\endcsname
                       \csname hc@orgend#1\expandafter\endcsname
                       \expandafter{#1}%
}
%    \end{macrocode}
%    \end{macro}
%    \begin{macro}{\hc@redef}
%    The old meaning of the environment is saved. Then
%    \cmd{\capstart} is appended in the begin part.
%    The end part contains a check
%    that produces an error message in case of \cmd{\capstart}
%    without \cmd{\capstart} (\cmd{\capstart} has incremented
%    the counter).
%    \begin{macrocode}
\def\hc@redef#1#2#3{%
  \newcommand#1{}%
  \expandafter\LetLtxMacro\expandafter#1\csname#3\endcsname
  \expandafter\LetLtxMacro\expandafter#2\csname end#3\endcsname
  \renewenvironment*{#3}[1][]{%
    \ifx\\##1\\%
      #1\relax
    \else
      #1[##1]% hash-ok (compatibility for float)
    \fi
    \capstart
  }{%
    \if@capstart
      \PackageError{hypcap}{You have forgotten to use \string\caption}%
      \global\@capstartfalse
    \else
    \fi
    #2%
  }%
}
%    \end{macrocode}
%    \end{macro}
%
%     At last the options are defined and processed.
%    \begin{macrocode}
\DeclareOption{figure}{\hypcapredef{\CurrentOption}}
\DeclareOption{figure*}{\hypcapredef{\CurrentOption}}
\DeclareOption{table}{\hypcapredef{\CurrentOption}}
\DeclareOption{table*}{\hypcapredef{\CurrentOption}}
\DeclareOption{all}{%
  \hypcapredef{figure}%
  \hypcapredef{figure*}%
  \hypcapredef{table}%
  \hypcapredef{table*}%
}
\ProcessOptions\relax
%    \end{macrocode}
%
%    \begin{macrocode}
%</package>
%    \end{macrocode}
%
% \section{Installation}
%
% \subsection{Download}
%
% \paragraph{Package.} This package is available on
% CTAN\footnote{\url{ftp://ftp.ctan.org/tex-archive/}}:
% \begin{description}
% \item[\CTAN{macros/latex/contrib/oberdiek/hypcap.dtx}] The source file.
% \item[\CTAN{macros/latex/contrib/oberdiek/hypcap.pdf}] Documentation.
% \end{description}
%
%
% \paragraph{Bundle.} All the packages of the bundle `oberdiek'
% are also available in a TDS compliant ZIP archive. There
% the packages are already unpacked and the documentation files
% are generated. The files and directories obey the TDS standard.
% \begin{description}
% \item[\CTAN{install/macros/latex/contrib/oberdiek.tds.zip}]
% \end{description}
% \emph{TDS} refers to the standard ``A Directory Structure
% for \TeX\ Files'' (\CTAN{tds/tds.pdf}). Directories
% with \xfile{texmf} in their name are usually organized this way.
%
% \subsection{Bundle installation}
%
% \paragraph{Unpacking.} Unpack the \xfile{oberdiek.tds.zip} in the
% TDS tree (also known as \xfile{texmf} tree) of your choice.
% Example (linux):
% \begin{quote}
%   |unzip oberdiek.tds.zip -d ~/texmf|
% \end{quote}
%
% \paragraph{Script installation.}
% Check the directory \xfile{TDS:scripts/oberdiek/} for
% scripts that need further installation steps.
% Package \xpackage{attachfile2} comes with the Perl script
% \xfile{pdfatfi.pl} that should be installed in such a way
% that it can be called as \texttt{pdfatfi}.
% Example (linux):
% \begin{quote}
%   |chmod +x scripts/oberdiek/pdfatfi.pl|\\
%   |cp scripts/oberdiek/pdfatfi.pl /usr/local/bin/|
% \end{quote}
%
% \subsection{Package installation}
%
% \paragraph{Unpacking.} The \xfile{.dtx} file is a self-extracting
% \docstrip\ archive. The files are extracted by running the
% \xfile{.dtx} through \plainTeX:
% \begin{quote}
%   \verb|tex hypcap.dtx|
% \end{quote}
%
% \paragraph{TDS.} Now the different files must be moved into
% the different directories in your installation TDS tree
% (also known as \xfile{texmf} tree):
% \begin{quote}
% \def\t{^^A
% \begin{tabular}{@{}>{\ttfamily}l@{ $\rightarrow$ }>{\ttfamily}l@{}}
%   hypcap.sty & tex/latex/oberdiek/hypcap.sty\\
%   hypcap.pdf & doc/latex/oberdiek/hypcap.pdf\\
%   hypcap.dtx & source/latex/oberdiek/hypcap.dtx\\
% \end{tabular}^^A
% }^^A
% \sbox0{\t}^^A
% \ifdim\wd0>\linewidth
%   \begingroup
%     \advance\linewidth by\leftmargin
%     \advance\linewidth by\rightmargin
%   \edef\x{\endgroup
%     \def\noexpand\lw{\the\linewidth}^^A
%   }\x
%   \def\lwbox{^^A
%     \leavevmode
%     \hbox to \linewidth{^^A
%       \kern-\leftmargin\relax
%       \hss
%       \usebox0
%       \hss
%       \kern-\rightmargin\relax
%     }^^A
%   }^^A
%   \ifdim\wd0>\lw
%     \sbox0{\small\t}^^A
%     \ifdim\wd0>\linewidth
%       \ifdim\wd0>\lw
%         \sbox0{\footnotesize\t}^^A
%         \ifdim\wd0>\linewidth
%           \ifdim\wd0>\lw
%             \sbox0{\scriptsize\t}^^A
%             \ifdim\wd0>\linewidth
%               \ifdim\wd0>\lw
%                 \sbox0{\tiny\t}^^A
%                 \ifdim\wd0>\linewidth
%                   \lwbox
%                 \else
%                   \usebox0
%                 \fi
%               \else
%                 \lwbox
%               \fi
%             \else
%               \usebox0
%             \fi
%           \else
%             \lwbox
%           \fi
%         \else
%           \usebox0
%         \fi
%       \else
%         \lwbox
%       \fi
%     \else
%       \usebox0
%     \fi
%   \else
%     \lwbox
%   \fi
% \else
%   \usebox0
% \fi
% \end{quote}
% If you have a \xfile{docstrip.cfg} that configures and enables \docstrip's
% TDS installing feature, then some files can already be in the right
% place, see the documentation of \docstrip.
%
% \subsection{Refresh file name databases}
%
% If your \TeX~distribution
% (\teTeX, \mikTeX, \dots) relies on file name databases, you must refresh
% these. For example, \teTeX\ users run \verb|texhash| or
% \verb|mktexlsr|.
%
% \subsection{Some details for the interested}
%
% \paragraph{Attached source.}
%
% The PDF documentation on CTAN also includes the
% \xfile{.dtx} source file. It can be extracted by
% AcrobatReader 6 or higher. Another option is \textsf{pdftk},
% e.g. unpack the file into the current directory:
% \begin{quote}
%   \verb|pdftk hypcap.pdf unpack_files output .|
% \end{quote}
%
% \paragraph{Unpacking with \LaTeX.}
% The \xfile{.dtx} chooses its action depending on the format:
% \begin{description}
% \item[\plainTeX:] Run \docstrip\ and extract the files.
% \item[\LaTeX:] Generate the documentation.
% \end{description}
% If you insist on using \LaTeX\ for \docstrip\ (really,
% \docstrip\ does not need \LaTeX), then inform the autodetect routine
% about your intention:
% \begin{quote}
%   \verb|latex \let\install=y\input{hypcap.dtx}|
% \end{quote}
% Do not forget to quote the argument according to the demands
% of your shell.
%
% \paragraph{Generating the documentation.}
% You can use both the \xfile{.dtx} or the \xfile{.drv} to generate
% the documentation. The process can be configured by the
% configuration file \xfile{ltxdoc.cfg}. For instance, put this
% line into this file, if you want to have A4 as paper format:
% \begin{quote}
%   \verb|\PassOptionsToClass{a4paper}{article}|
% \end{quote}
% An example follows how to generate the
% documentation with pdf\LaTeX:
% \begin{quote}
%\begin{verbatim}
%pdflatex hypcap.dtx
%makeindex -s gind.ist hypcap.idx
%pdflatex hypcap.dtx
%makeindex -s gind.ist hypcap.idx
%pdflatex hypcap.dtx
%\end{verbatim}
% \end{quote}
%
% \section{Catalogue}
%
% The following XML file can be used as source for the
% \href{http://mirror.ctan.org/help/Catalogue/catalogue.html}{\TeX\ Catalogue}.
% The elements \texttt{caption} and \texttt{description} are imported
% from the original XML file from the Catalogue.
% The name of the XML file in the Catalogue is \xfile{hypcap.xml}.
%    \begin{macrocode}
%<*catalogue>
<?xml version='1.0' encoding='us-ascii'?>
<!DOCTYPE entry SYSTEM 'catalogue.dtd'>
<entry datestamp='$Date$' modifier='$Author$' id='hypcap'>
  <name>hypcap</name>
  <caption>Adjusting the anchors of captions.</caption>
  <authorref id='auth:oberdiek'/>
  <copyright owner='Heiko Oberdiek' year='1999-2001,2006-2008,2011'/>
  <license type='lppl1.3'/>
  <version number='1.11'/>
  <description>
    The package offers a solution to the problem that when you link to
    a float using <xref refid='hyperref'>hyperref</xref>, the link
    anchors to below the float's caption, rather than the beginning of
    the float.
    <p/>
    Hypcap defines a separate \capstart command, which you put where
    you want links to end; you should have a \capstart command for each
    \caption command.  Package options can be used to auto-insert a
    \capstart at the start of a float environment.
    <p/>
    The package is part of the <xref refid='oberdiek'>oberdiek</xref> bundle.
  </description>
  <documentation details='Package documentation'
      href='ctan:/macros/latex/contrib/oberdiek/hypcap.pdf'/>
  <ctan file='true' path='/macros/latex/contrib/oberdiek/hypcap.dtx'/>
  <miktex location='oberdiek'/>
  <texlive location='oberdiek'/>
  <install path='/macros/latex/contrib/oberdiek/oberdiek.tds.zip'/>
</entry>
%</catalogue>
%    \end{macrocode}
%
% \begin{History}
%   \begin{Version}{1999/02/13 v1.0}
%   \item
%     A beginning version, published in newsgroup \xnewsgroup{comp.text.tex}:\\
%     \URL{``\link{Re: \xpackage{hyperref} and figures}''}^^A
%     {http://groups.google.com/group/comp.text.tex/msg/5c9b47b001a9379c}
%   \end{Version}
%   \begin{Version}{2000/08/14 v1.1}
%   \item
%     Global assignments of \cs{if@capstart}
%     in order to allow \cs{caption} in groups.
%   \item
%     Option \texttt{all} added.
%   \end{Version}
%   \begin{Version}{2000/09/07 v1.2}
%   \item
%     Package in dtx format.
%   \end{Version}
%   \begin{Version}{2001/08/27 v1.3}
%   \item
%     Bug fix with hyperref's pdfmark driver\\
%     (\cs{leavevmode} in
%     \cs{hyper@@anchor}/\cs{pdf@rect}).
%   \end{Version}
%   \begin{Version}{2001/09/06 v1.4}
%   \item
%     Small fixes in the dtx file.
%   \end{Version}
%   \begin{Version}{2006/02/20 v1.5}
%   \item
%     Code is not changed.
%   \item
%     New DTX framework.
%   \end{Version}
%   \begin{Version}{2007/02/19 v1.6}
%   \item
%     Fix for hypertexnames=false.
%   \end{Version}
%   \begin{Version}{2007/04/09 v1.7}
%   \item
%     Stuff in \cs{caption} moved to \xpackage{hyperref}.
%     This avoids redefinitions of \cs{caption} and \cs{@caption}
%     (idea of Axel Sommerfeldt).
%   \item
%     Fix for subfigure (Marco Kuhlmann, Amilcar do Carmo Lucas).
%   \end{Version}
%   \begin{Version}{2008/04/14 v1.8}
%   \item
%     \cs{hc@redef} fixed to get package \xpackage{float} work
%     (Axel Sommerfeldt).
%   \end{Version}
%   \begin{Version}{2008/08/11 v1.9}
%   \item
%     Code is not changed.
%   \item
%     URLs updated.
%   \end{Version}
%   \begin{Version}{2008/09/08 v1.10}
%   \item
%     \cs{capstartfalse} and \cs{capstarttrue} added.
%   \end{Version}
%   \begin{Version}{2011/02/16 v1.11}
%   \item
%     \cs{hc@redef} fixed by using package \xpackage{letltxmacro}.
%   \end{Version}
% \end{History}
%
% \PrintIndex
%
% \Finale
\endinput

%        (quote the arguments according to the demands of your shell)
%
% Documentation:
%    (a) If hypcap.drv is present:
%           latex hypcap.drv
%    (b) Without hypcap.drv:
%           latex hypcap.dtx; ...
%    The class ltxdoc loads the configuration file ltxdoc.cfg
%    if available. Here you can specify further options, e.g.
%    use A4 as paper format:
%       \PassOptionsToClass{a4paper}{article}
%
%    Programm calls to get the documentation (example):
%       pdflatex hypcap.dtx
%       makeindex -s gind.ist hypcap.idx
%       pdflatex hypcap.dtx
%       makeindex -s gind.ist hypcap.idx
%       pdflatex hypcap.dtx
%
% Installation:
%    TDS:tex/latex/oberdiek/hypcap.sty
%    TDS:doc/latex/oberdiek/hypcap.pdf
%    TDS:source/latex/oberdiek/hypcap.dtx
%
%<*ignore>
\begingroup
  \catcode123=1 %
  \catcode125=2 %
  \def\x{LaTeX2e}%
\expandafter\endgroup
\ifcase 0\ifx\install y1\fi\expandafter
         \ifx\csname processbatchFile\endcsname\relax\else1\fi
         \ifx\fmtname\x\else 1\fi\relax
\else\csname fi\endcsname
%</ignore>
%<*install>
\input docstrip.tex
\Msg{************************************************************************}
\Msg{* Installation}
\Msg{* Package: hypcap 2011/02/16 v1.11 Adjusting the anchors of captions (HO)}
\Msg{************************************************************************}

\keepsilent
\askforoverwritefalse

\let\MetaPrefix\relax
\preamble

This is a generated file.

Project: hypcap
Version: 2011/02/16 v1.11

Copyright (C) 1999-2001, 2006-2008, 2011 by
   Heiko Oberdiek <heiko.oberdiek at googlemail.com>

This work may be distributed and/or modified under the
conditions of the LaTeX Project Public License, either
version 1.3c of this license or (at your option) any later
version. This version of this license is in
   http://www.latex-project.org/lppl/lppl-1-3c.txt
and the latest version of this license is in
   http://www.latex-project.org/lppl.txt
and version 1.3 or later is part of all distributions of
LaTeX version 2005/12/01 or later.

This work has the LPPL maintenance status "maintained".

This Current Maintainer of this work is Heiko Oberdiek.

This work consists of the main source file hypcap.dtx
and the derived files
   hypcap.sty, hypcap.pdf, hypcap.ins, hypcap.drv.

\endpreamble
\let\MetaPrefix\DoubleperCent

\generate{%
  \file{hypcap.ins}{\from{hypcap.dtx}{install}}%
  \file{hypcap.drv}{\from{hypcap.dtx}{driver}}%
  \usedir{tex/latex/oberdiek}%
  \file{hypcap.sty}{\from{hypcap.dtx}{package}}%
  \nopreamble
  \nopostamble
  \usedir{source/latex/oberdiek/catalogue}%
  \file{hypcap.xml}{\from{hypcap.dtx}{catalogue}}%
}

\catcode32=13\relax% active space
\let =\space%
\Msg{************************************************************************}
\Msg{*}
\Msg{* To finish the installation you have to move the following}
\Msg{* file into a directory searched by TeX:}
\Msg{*}
\Msg{*     hypcap.sty}
\Msg{*}
\Msg{* To produce the documentation run the file `hypcap.drv'}
\Msg{* through LaTeX.}
\Msg{*}
\Msg{* Happy TeXing!}
\Msg{*}
\Msg{************************************************************************}

\endbatchfile
%</install>
%<*ignore>
\fi
%</ignore>
%<*driver>
\NeedsTeXFormat{LaTeX2e}
\ProvidesFile{hypcap.drv}%
  [2011/02/16 v1.11 Adjusting the anchors of captions (HO)]%
\documentclass{ltxdoc}
\usepackage{holtxdoc}[2011/11/22]
\begin{document}
  \DocInput{hypcap.dtx}%
\end{document}
%</driver>
% \fi
%
% \CheckSum{174}
%
% \CharacterTable
%  {Upper-case    \A\B\C\D\E\F\G\H\I\J\K\L\M\N\O\P\Q\R\S\T\U\V\W\X\Y\Z
%   Lower-case    \a\b\c\d\e\f\g\h\i\j\k\l\m\n\o\p\q\r\s\t\u\v\w\x\y\z
%   Digits        \0\1\2\3\4\5\6\7\8\9
%   Exclamation   \!     Double quote  \"     Hash (number) \#
%   Dollar        \$     Percent       \%     Ampersand     \&
%   Acute accent  \'     Left paren    \(     Right paren   \)
%   Asterisk      \*     Plus          \+     Comma         \,
%   Minus         \-     Point         \.     Solidus       \/
%   Colon         \:     Semicolon     \;     Less than     \<
%   Equals        \=     Greater than  \>     Question mark \?
%   Commercial at \@     Left bracket  \[     Backslash     \\
%   Right bracket \]     Circumflex    \^     Underscore    \_
%   Grave accent  \`     Left brace    \{     Vertical bar  \|
%   Right brace   \}     Tilde         \~}
%
% \GetFileInfo{hypcap.drv}
%
% \title{The \xpackage{hypcap} package}
% \date{2011/02/16 v1.11}
% \author{Heiko Oberdiek\\\xemail{heiko.oberdiek at googlemail.com}}
%
% \maketitle
%
% \begin{abstract}
% This package tries a solution of the problem with
% hyperref, that links to floats points below the
% caption and not at the beginning of the float.
% Therefore this package divides the task into two
% part, the link setting with \cs{capstart} or
% automatically at the beginning of a float and
% the rest in the \cs{caption} command.
% \end{abstract}
%
% \tableofcontents
%
% \newpage
% \section{Usage}
%    The package \Package{hypcap} requires that \Package{hyperref}
%    is loaded first:
%    \begin{quote}
%      |\usepackage[...]{hyperref}|\\
%      |\usepackage[...]{hypcap}|
%    \end{quote}
%
% \subsection{Package options}
%    The names of the four float environments |figure|, |figure*|,
%    |table|, or |table*| can be used as option. Then the package
%    redefines the environment in order to insert \cmd{\capstart}
%    (see below) in the beginning of the environment automatically.
%
%    Option |all| enables the redefinitions of all four
%    float environments. For other environments see
%    the user command \cmd{\hypcapredef}.
%
% \subsection{User commands}
%    \begin{description}
%    \item[\cmd{\capstart}:]\DescribeMacro{\capstart}
%      First this command increments the counter (\cmd{\@captype}).
%      Then it makes an anchor for package \Package{hyperref}.
%      At last \cmd{\caption} is redefined to remove the
%      anchor setting part from \Package{hyperref}'s \cmd{\caption}.
%
%      The package expects the following structure of a float
%      environment:
%      \begin{quote}
%        |\begin{|\textit{|float|}|}...|\\
%        |\capstart|\\
%        |...|\\
%        |\caption{...}|\\
%        |...|\\
%        |\end{|\textit{|float|}|}|
%      \end{quote}
%      There can be several \cmd{\caption} commands. For these
%      you need \cmd{\capstart} again:
%      \begin{quote}
%        |\capstart ... \caption... \capstart ... \caption...|
%      \end{quote}
%      And the \cmd{\caption} command itself can be put in a group.
%
%      With the options, described above, the extra
%      writing of \cmd{\capstart} can be avoided.
%      Consequently, there must be a \cmd{\caption} in every
%      environment of this type, specified by the option.
%      If you want to use more than one \cmd{\caption} in this
%      environment, you have to state \cmd{\capstart} again.
%    \item[\cmd{\hypcapspace}:]\DescribeMacro{\hypcapspace}
%      Because it looks poor, if the link points exactly at top of
%      the figure, there is additional space: \cmd{\hypcapspace},
%      the default is |0.5\baselineskip|, examples:
%      \begin{quote}
%        |\renewcommand{\hypcapspace}{0pt}| removes the space\\
%        |\renewcommand{\hypcapspace}{1pt}| sets a fix value
%      \end{quote}
%    \item[\cmd{\hypcapredef}:]\DescribeMacro{\hypcapredef}
%      If there are other float environments, that
%      should automatically execute \cmd{\capstart}, then
%      a redefinition with \cmd{\hypcapredef} can be tried:
%      \begin{quote}
%        |\hypcapredef{myfloat}|
%      \end{quote}
%      Only environments with one optional parameter are
%      supported.
%    \item[\cs{capstartfalse}, \cs{capstarttrue}:]^^A
%      \DescribeMacro{\capstartfalse}^^A
%      \DescribeMacro{\capstarttrue}^^A
%      Since 2008/09/08 v1.10.\\
%      They disable and enable \cs{capstart}. They can be used to
%      cancel the effect of a redefined float environment.
%      Example:
%\begin{quote}
%\catcode`Y=\active
%\def Y#1Z{\textit{\cs{#1}}}
%\begin{verbatim}
%\documentclass{article}
%\usepackage{hyperref}
%\usepackage[figure]{hypcap}[2008/09/08]
%
%\begin{document}
%  \section{Hello World}
%  \begin{figure}
%    \caption{Figure with caption A}
%  \end{figure}
%  YcapstartfalseZ
%  \begin{figure}
%    Figure without caption
%  \end{figure}
%  YcapstarttrueZ
%  \begin{figure}
%    \caption{Figure with caption B}
%  \end{figure}
%\end{document}
%\end{verbatim}
%\end{quote}
%    \end{description}
%
% \subsection{Limitations}
%    \begin{itemize}
%    \item Packages that redefine \cmd{\caption} or
%          \cmd{\@caption}.
%    \end{itemize}
%
% \StopEventually{
% }
%
% \section{Implementation}
%    \begin{macrocode}
%<*package>
%    \end{macrocode}
%    Package identification.
%    \begin{macrocode}
\NeedsTeXFormat{LaTeX2e}
\ProvidesPackage{hypcap}%
  [2011/02/16 v1.11 Adjusting the anchors of captions (HO)]
%    \end{macrocode}
%
%    For unique command names this package uses |hc@| as prefix
%    for internal command names.
%
%    First we check, if package \Package{hyperref} is loaded:
%    \begin{macrocode}
\@ifundefined{hyper@@anchor}{%
  \PackageError{hypcap}{You have to load 'hyperref' first}\@ehc
  \endinput
}{}
%    \end{macrocode}
%
%    \begin{macrocode}
\RequirePackage{letltxmacro}[2008/06/24]
%    \end{macrocode}
%
%    \begin{macro}{\hc@org@caption}
%    Save the original meaning of \cmd{\caption}:
%    \begin{macrocode}
\newcommand*\hc@org@caption{}
\let\hc@org@caption\caption
%    \end{macrocode}
%    \end{macro}
%
%    \newcommand*{\BeginMacro}[1]{\expandafter\DoBeginMacro\csname#1\endcsname}
%    \newcommand*{\DoBeginMacro}[1]{\begin{macro}{#1}}
%    \newcommand*{\Cmd}[1]{\expandafter\cmd\csname#1\endcsname}
%    \BeginMacro{if@capstart}
%     The switch \Cmd{if@capstart} helps to detect
%     \cmd{\capstart} commands with missing \cmd{\caption} macros.
%     Because \cmd{\caption} can occur inside a group, assignments
%     to the switch have to be made global.
%    \begin{macrocode}
\newif\if@capstart
%    \end{macrocode}
%    \end{macro}
%
%    \begin{macro}{\hypcapspace}
%    The anchor is raised.by \cmd{\hypcapspace}.
%    \begin{macrocode}
\newcommand*\hypcapspace{.5\baselineskip}
%    \end{macrocode}
%    \end{macro}
%
%    \begin{macro}{\ifcapstart}
%    \begin{macrocode}
\newif\ifcapstart
\capstarttrue
%    \end{macrocode}
%    \end{macro}
%
%    \begin{macro}{\capstart}
%     The macro \cmd{\capstart} contains the first part of
%     the \cmd{\caption} command: Incrementing the counter
%     and setting the anchor.
%    \begin{macrocode}
\newcommand*\capstart{%
  \ifcapstart
    \H@refstepcounter\@captype % first part of caption
    \hyper@makecurrent\@captype
    \global\let\hc@currentHref\@currentHref
    \vspace*{-\hypcapspace}%
    \begingroup
      \let\leavevmode\relax
      \hyper@@anchor\@currentHref\relax
    \endgroup
    \vspace*{\hypcapspace}%
    \hc@hyperref{\let\caption\hc@caption}%
    \global\@capstarttrue
    \global\advance\csname c@\@captype\endcsname\m@ne
  \fi
}
%    \end{macrocode}
%    \end{macro}
%
%    \begin{macrocode}
\@ifpackagelater{hyperref}{2007/04/09}{%
  \let\hc@hyperref\@gobble
}{%
  \let\hc@hyperref\@firstofone
}
%    \end{macrocode}
%
%    \begin{macro}{\hc@caption}
%    The new \cmd{\caption} command without the first part
%    is defined in the macro \cmd{\hc@caption}.
%    \begin{macrocode}
\def\hc@caption{%
  \global\advance\csname c@\@captype\endcsname\@ne
  \@dblarg{\hc@@caption\@captype}%
}
%    \end{macrocode}
%    \end{macro}
%
%    \begin{macro}{\hc@@caption}
%    This is a copy of package \Package{hyperref}'s \cmd{\@caption}
%    macro without making the anchor, because this is already done
%    in \cmd{\capstart}.
%    \begin{macrocode}
\long\def\hc@@caption#1[#2]#3{%
  \let\caption\hc@org@caption
  \global\@capstartfalse
  \ifHy@hypertexnames
    \hyper@makecurrent\@captype
  \else
    \global\let\@currentHref\hc@currentHref
  \fi
  \par\addcontentsline{%
    \csname ext@#1\endcsname}{#1}{%
    \protect\numberline{%
      \csname the#1\endcsname
    }{\ignorespaces #2}%
  }%
  \begingroup
    \@parboxrestore
    \normalsize
    \@makecaption{\csname fnum@#1\endcsname}{%
      \ignorespaces#3%
    }%
    \par
  \endgroup
}
%    \end{macrocode}
%    \end{macro}
%
%    \begin{macro}{\hypcapredef}
%    The macro \cmd{\hypcapredef} prepares the call of
%    \cmd{\hc@redef} that will redefine the environment
%    that is given in the argument.
%    \begin{macrocode}
\def\hypcapredef#1{%
  \expandafter\hc@redef\csname hc@org#1\expandafter\endcsname
                       \csname hc@orgend#1\expandafter\endcsname
                       \expandafter{#1}%
}
%    \end{macrocode}
%    \end{macro}
%    \begin{macro}{\hc@redef}
%    The old meaning of the environment is saved. Then
%    \cmd{\capstart} is appended in the begin part.
%    The end part contains a check
%    that produces an error message in case of \cmd{\capstart}
%    without \cmd{\capstart} (\cmd{\capstart} has incremented
%    the counter).
%    \begin{macrocode}
\def\hc@redef#1#2#3{%
  \newcommand#1{}%
  \expandafter\LetLtxMacro\expandafter#1\csname#3\endcsname
  \expandafter\LetLtxMacro\expandafter#2\csname end#3\endcsname
  \renewenvironment*{#3}[1][]{%
    \ifx\\##1\\%
      #1\relax
    \else
      #1[##1]% hash-ok (compatibility for float)
    \fi
    \capstart
  }{%
    \if@capstart
      \PackageError{hypcap}{You have forgotten to use \string\caption}%
      \global\@capstartfalse
    \else
    \fi
    #2%
  }%
}
%    \end{macrocode}
%    \end{macro}
%
%     At last the options are defined and processed.
%    \begin{macrocode}
\DeclareOption{figure}{\hypcapredef{\CurrentOption}}
\DeclareOption{figure*}{\hypcapredef{\CurrentOption}}
\DeclareOption{table}{\hypcapredef{\CurrentOption}}
\DeclareOption{table*}{\hypcapredef{\CurrentOption}}
\DeclareOption{all}{%
  \hypcapredef{figure}%
  \hypcapredef{figure*}%
  \hypcapredef{table}%
  \hypcapredef{table*}%
}
\ProcessOptions\relax
%    \end{macrocode}
%
%    \begin{macrocode}
%</package>
%    \end{macrocode}
%
% \section{Installation}
%
% \subsection{Download}
%
% \paragraph{Package.} This package is available on
% CTAN\footnote{\url{ftp://ftp.ctan.org/tex-archive/}}:
% \begin{description}
% \item[\CTAN{macros/latex/contrib/oberdiek/hypcap.dtx}] The source file.
% \item[\CTAN{macros/latex/contrib/oberdiek/hypcap.pdf}] Documentation.
% \end{description}
%
%
% \paragraph{Bundle.} All the packages of the bundle `oberdiek'
% are also available in a TDS compliant ZIP archive. There
% the packages are already unpacked and the documentation files
% are generated. The files and directories obey the TDS standard.
% \begin{description}
% \item[\CTAN{install/macros/latex/contrib/oberdiek.tds.zip}]
% \end{description}
% \emph{TDS} refers to the standard ``A Directory Structure
% for \TeX\ Files'' (\CTAN{tds/tds.pdf}). Directories
% with \xfile{texmf} in their name are usually organized this way.
%
% \subsection{Bundle installation}
%
% \paragraph{Unpacking.} Unpack the \xfile{oberdiek.tds.zip} in the
% TDS tree (also known as \xfile{texmf} tree) of your choice.
% Example (linux):
% \begin{quote}
%   |unzip oberdiek.tds.zip -d ~/texmf|
% \end{quote}
%
% \paragraph{Script installation.}
% Check the directory \xfile{TDS:scripts/oberdiek/} for
% scripts that need further installation steps.
% Package \xpackage{attachfile2} comes with the Perl script
% \xfile{pdfatfi.pl} that should be installed in such a way
% that it can be called as \texttt{pdfatfi}.
% Example (linux):
% \begin{quote}
%   |chmod +x scripts/oberdiek/pdfatfi.pl|\\
%   |cp scripts/oberdiek/pdfatfi.pl /usr/local/bin/|
% \end{quote}
%
% \subsection{Package installation}
%
% \paragraph{Unpacking.} The \xfile{.dtx} file is a self-extracting
% \docstrip\ archive. The files are extracted by running the
% \xfile{.dtx} through \plainTeX:
% \begin{quote}
%   \verb|tex hypcap.dtx|
% \end{quote}
%
% \paragraph{TDS.} Now the different files must be moved into
% the different directories in your installation TDS tree
% (also known as \xfile{texmf} tree):
% \begin{quote}
% \def\t{^^A
% \begin{tabular}{@{}>{\ttfamily}l@{ $\rightarrow$ }>{\ttfamily}l@{}}
%   hypcap.sty & tex/latex/oberdiek/hypcap.sty\\
%   hypcap.pdf & doc/latex/oberdiek/hypcap.pdf\\
%   hypcap.dtx & source/latex/oberdiek/hypcap.dtx\\
% \end{tabular}^^A
% }^^A
% \sbox0{\t}^^A
% \ifdim\wd0>\linewidth
%   \begingroup
%     \advance\linewidth by\leftmargin
%     \advance\linewidth by\rightmargin
%   \edef\x{\endgroup
%     \def\noexpand\lw{\the\linewidth}^^A
%   }\x
%   \def\lwbox{^^A
%     \leavevmode
%     \hbox to \linewidth{^^A
%       \kern-\leftmargin\relax
%       \hss
%       \usebox0
%       \hss
%       \kern-\rightmargin\relax
%     }^^A
%   }^^A
%   \ifdim\wd0>\lw
%     \sbox0{\small\t}^^A
%     \ifdim\wd0>\linewidth
%       \ifdim\wd0>\lw
%         \sbox0{\footnotesize\t}^^A
%         \ifdim\wd0>\linewidth
%           \ifdim\wd0>\lw
%             \sbox0{\scriptsize\t}^^A
%             \ifdim\wd0>\linewidth
%               \ifdim\wd0>\lw
%                 \sbox0{\tiny\t}^^A
%                 \ifdim\wd0>\linewidth
%                   \lwbox
%                 \else
%                   \usebox0
%                 \fi
%               \else
%                 \lwbox
%               \fi
%             \else
%               \usebox0
%             \fi
%           \else
%             \lwbox
%           \fi
%         \else
%           \usebox0
%         \fi
%       \else
%         \lwbox
%       \fi
%     \else
%       \usebox0
%     \fi
%   \else
%     \lwbox
%   \fi
% \else
%   \usebox0
% \fi
% \end{quote}
% If you have a \xfile{docstrip.cfg} that configures and enables \docstrip's
% TDS installing feature, then some files can already be in the right
% place, see the documentation of \docstrip.
%
% \subsection{Refresh file name databases}
%
% If your \TeX~distribution
% (\teTeX, \mikTeX, \dots) relies on file name databases, you must refresh
% these. For example, \teTeX\ users run \verb|texhash| or
% \verb|mktexlsr|.
%
% \subsection{Some details for the interested}
%
% \paragraph{Attached source.}
%
% The PDF documentation on CTAN also includes the
% \xfile{.dtx} source file. It can be extracted by
% AcrobatReader 6 or higher. Another option is \textsf{pdftk},
% e.g. unpack the file into the current directory:
% \begin{quote}
%   \verb|pdftk hypcap.pdf unpack_files output .|
% \end{quote}
%
% \paragraph{Unpacking with \LaTeX.}
% The \xfile{.dtx} chooses its action depending on the format:
% \begin{description}
% \item[\plainTeX:] Run \docstrip\ and extract the files.
% \item[\LaTeX:] Generate the documentation.
% \end{description}
% If you insist on using \LaTeX\ for \docstrip\ (really,
% \docstrip\ does not need \LaTeX), then inform the autodetect routine
% about your intention:
% \begin{quote}
%   \verb|latex \let\install=y% \iffalse meta-comment
%
% File: hypcap.dtx
% Version: 2011/02/16 v1.11
% Info: Adjusting the anchors of captions
%
% Copyright (C) 1999-2001, 2006-2008, 2011 by
%    Heiko Oberdiek <heiko.oberdiek at googlemail.com>
%
% This work may be distributed and/or modified under the
% conditions of the LaTeX Project Public License, either
% version 1.3c of this license or (at your option) any later
% version. This version of this license is in
%    http://www.latex-project.org/lppl/lppl-1-3c.txt
% and the latest version of this license is in
%    http://www.latex-project.org/lppl.txt
% and version 1.3 or later is part of all distributions of
% LaTeX version 2005/12/01 or later.
%
% This work has the LPPL maintenance status "maintained".
%
% This Current Maintainer of this work is Heiko Oberdiek.
%
% This work consists of the main source file hypcap.dtx
% and the derived files
%    hypcap.sty, hypcap.pdf, hypcap.ins, hypcap.drv.
%
% Distribution:
%    CTAN:macros/latex/contrib/oberdiek/hypcap.dtx
%    CTAN:macros/latex/contrib/oberdiek/hypcap.pdf
%
% Unpacking:
%    (a) If hypcap.ins is present:
%           tex hypcap.ins
%    (b) Without hypcap.ins:
%           tex hypcap.dtx
%    (c) If you insist on using LaTeX
%           latex \let\install=y\input{hypcap.dtx}
%        (quote the arguments according to the demands of your shell)
%
% Documentation:
%    (a) If hypcap.drv is present:
%           latex hypcap.drv
%    (b) Without hypcap.drv:
%           latex hypcap.dtx; ...
%    The class ltxdoc loads the configuration file ltxdoc.cfg
%    if available. Here you can specify further options, e.g.
%    use A4 as paper format:
%       \PassOptionsToClass{a4paper}{article}
%
%    Programm calls to get the documentation (example):
%       pdflatex hypcap.dtx
%       makeindex -s gind.ist hypcap.idx
%       pdflatex hypcap.dtx
%       makeindex -s gind.ist hypcap.idx
%       pdflatex hypcap.dtx
%
% Installation:
%    TDS:tex/latex/oberdiek/hypcap.sty
%    TDS:doc/latex/oberdiek/hypcap.pdf
%    TDS:source/latex/oberdiek/hypcap.dtx
%
%<*ignore>
\begingroup
  \catcode123=1 %
  \catcode125=2 %
  \def\x{LaTeX2e}%
\expandafter\endgroup
\ifcase 0\ifx\install y1\fi\expandafter
         \ifx\csname processbatchFile\endcsname\relax\else1\fi
         \ifx\fmtname\x\else 1\fi\relax
\else\csname fi\endcsname
%</ignore>
%<*install>
\input docstrip.tex
\Msg{************************************************************************}
\Msg{* Installation}
\Msg{* Package: hypcap 2011/02/16 v1.11 Adjusting the anchors of captions (HO)}
\Msg{************************************************************************}

\keepsilent
\askforoverwritefalse

\let\MetaPrefix\relax
\preamble

This is a generated file.

Project: hypcap
Version: 2011/02/16 v1.11

Copyright (C) 1999-2001, 2006-2008, 2011 by
   Heiko Oberdiek <heiko.oberdiek at googlemail.com>

This work may be distributed and/or modified under the
conditions of the LaTeX Project Public License, either
version 1.3c of this license or (at your option) any later
version. This version of this license is in
   http://www.latex-project.org/lppl/lppl-1-3c.txt
and the latest version of this license is in
   http://www.latex-project.org/lppl.txt
and version 1.3 or later is part of all distributions of
LaTeX version 2005/12/01 or later.

This work has the LPPL maintenance status "maintained".

This Current Maintainer of this work is Heiko Oberdiek.

This work consists of the main source file hypcap.dtx
and the derived files
   hypcap.sty, hypcap.pdf, hypcap.ins, hypcap.drv.

\endpreamble
\let\MetaPrefix\DoubleperCent

\generate{%
  \file{hypcap.ins}{\from{hypcap.dtx}{install}}%
  \file{hypcap.drv}{\from{hypcap.dtx}{driver}}%
  \usedir{tex/latex/oberdiek}%
  \file{hypcap.sty}{\from{hypcap.dtx}{package}}%
  \nopreamble
  \nopostamble
  \usedir{source/latex/oberdiek/catalogue}%
  \file{hypcap.xml}{\from{hypcap.dtx}{catalogue}}%
}

\catcode32=13\relax% active space
\let =\space%
\Msg{************************************************************************}
\Msg{*}
\Msg{* To finish the installation you have to move the following}
\Msg{* file into a directory searched by TeX:}
\Msg{*}
\Msg{*     hypcap.sty}
\Msg{*}
\Msg{* To produce the documentation run the file `hypcap.drv'}
\Msg{* through LaTeX.}
\Msg{*}
\Msg{* Happy TeXing!}
\Msg{*}
\Msg{************************************************************************}

\endbatchfile
%</install>
%<*ignore>
\fi
%</ignore>
%<*driver>
\NeedsTeXFormat{LaTeX2e}
\ProvidesFile{hypcap.drv}%
  [2011/02/16 v1.11 Adjusting the anchors of captions (HO)]%
\documentclass{ltxdoc}
\usepackage{holtxdoc}[2011/11/22]
\begin{document}
  \DocInput{hypcap.dtx}%
\end{document}
%</driver>
% \fi
%
% \CheckSum{174}
%
% \CharacterTable
%  {Upper-case    \A\B\C\D\E\F\G\H\I\J\K\L\M\N\O\P\Q\R\S\T\U\V\W\X\Y\Z
%   Lower-case    \a\b\c\d\e\f\g\h\i\j\k\l\m\n\o\p\q\r\s\t\u\v\w\x\y\z
%   Digits        \0\1\2\3\4\5\6\7\8\9
%   Exclamation   \!     Double quote  \"     Hash (number) \#
%   Dollar        \$     Percent       \%     Ampersand     \&
%   Acute accent  \'     Left paren    \(     Right paren   \)
%   Asterisk      \*     Plus          \+     Comma         \,
%   Minus         \-     Point         \.     Solidus       \/
%   Colon         \:     Semicolon     \;     Less than     \<
%   Equals        \=     Greater than  \>     Question mark \?
%   Commercial at \@     Left bracket  \[     Backslash     \\
%   Right bracket \]     Circumflex    \^     Underscore    \_
%   Grave accent  \`     Left brace    \{     Vertical bar  \|
%   Right brace   \}     Tilde         \~}
%
% \GetFileInfo{hypcap.drv}
%
% \title{The \xpackage{hypcap} package}
% \date{2011/02/16 v1.11}
% \author{Heiko Oberdiek\\\xemail{heiko.oberdiek at googlemail.com}}
%
% \maketitle
%
% \begin{abstract}
% This package tries a solution of the problem with
% hyperref, that links to floats points below the
% caption and not at the beginning of the float.
% Therefore this package divides the task into two
% part, the link setting with \cs{capstart} or
% automatically at the beginning of a float and
% the rest in the \cs{caption} command.
% \end{abstract}
%
% \tableofcontents
%
% \newpage
% \section{Usage}
%    The package \Package{hypcap} requires that \Package{hyperref}
%    is loaded first:
%    \begin{quote}
%      |\usepackage[...]{hyperref}|\\
%      |\usepackage[...]{hypcap}|
%    \end{quote}
%
% \subsection{Package options}
%    The names of the four float environments |figure|, |figure*|,
%    |table|, or |table*| can be used as option. Then the package
%    redefines the environment in order to insert \cmd{\capstart}
%    (see below) in the beginning of the environment automatically.
%
%    Option |all| enables the redefinitions of all four
%    float environments. For other environments see
%    the user command \cmd{\hypcapredef}.
%
% \subsection{User commands}
%    \begin{description}
%    \item[\cmd{\capstart}:]\DescribeMacro{\capstart}
%      First this command increments the counter (\cmd{\@captype}).
%      Then it makes an anchor for package \Package{hyperref}.
%      At last \cmd{\caption} is redefined to remove the
%      anchor setting part from \Package{hyperref}'s \cmd{\caption}.
%
%      The package expects the following structure of a float
%      environment:
%      \begin{quote}
%        |\begin{|\textit{|float|}|}...|\\
%        |\capstart|\\
%        |...|\\
%        |\caption{...}|\\
%        |...|\\
%        |\end{|\textit{|float|}|}|
%      \end{quote}
%      There can be several \cmd{\caption} commands. For these
%      you need \cmd{\capstart} again:
%      \begin{quote}
%        |\capstart ... \caption... \capstart ... \caption...|
%      \end{quote}
%      And the \cmd{\caption} command itself can be put in a group.
%
%      With the options, described above, the extra
%      writing of \cmd{\capstart} can be avoided.
%      Consequently, there must be a \cmd{\caption} in every
%      environment of this type, specified by the option.
%      If you want to use more than one \cmd{\caption} in this
%      environment, you have to state \cmd{\capstart} again.
%    \item[\cmd{\hypcapspace}:]\DescribeMacro{\hypcapspace}
%      Because it looks poor, if the link points exactly at top of
%      the figure, there is additional space: \cmd{\hypcapspace},
%      the default is |0.5\baselineskip|, examples:
%      \begin{quote}
%        |\renewcommand{\hypcapspace}{0pt}| removes the space\\
%        |\renewcommand{\hypcapspace}{1pt}| sets a fix value
%      \end{quote}
%    \item[\cmd{\hypcapredef}:]\DescribeMacro{\hypcapredef}
%      If there are other float environments, that
%      should automatically execute \cmd{\capstart}, then
%      a redefinition with \cmd{\hypcapredef} can be tried:
%      \begin{quote}
%        |\hypcapredef{myfloat}|
%      \end{quote}
%      Only environments with one optional parameter are
%      supported.
%    \item[\cs{capstartfalse}, \cs{capstarttrue}:]^^A
%      \DescribeMacro{\capstartfalse}^^A
%      \DescribeMacro{\capstarttrue}^^A
%      Since 2008/09/08 v1.10.\\
%      They disable and enable \cs{capstart}. They can be used to
%      cancel the effect of a redefined float environment.
%      Example:
%\begin{quote}
%\catcode`Y=\active
%\def Y#1Z{\textit{\cs{#1}}}
%\begin{verbatim}
%\documentclass{article}
%\usepackage{hyperref}
%\usepackage[figure]{hypcap}[2008/09/08]
%
%\begin{document}
%  \section{Hello World}
%  \begin{figure}
%    \caption{Figure with caption A}
%  \end{figure}
%  YcapstartfalseZ
%  \begin{figure}
%    Figure without caption
%  \end{figure}
%  YcapstarttrueZ
%  \begin{figure}
%    \caption{Figure with caption B}
%  \end{figure}
%\end{document}
%\end{verbatim}
%\end{quote}
%    \end{description}
%
% \subsection{Limitations}
%    \begin{itemize}
%    \item Packages that redefine \cmd{\caption} or
%          \cmd{\@caption}.
%    \end{itemize}
%
% \StopEventually{
% }
%
% \section{Implementation}
%    \begin{macrocode}
%<*package>
%    \end{macrocode}
%    Package identification.
%    \begin{macrocode}
\NeedsTeXFormat{LaTeX2e}
\ProvidesPackage{hypcap}%
  [2011/02/16 v1.11 Adjusting the anchors of captions (HO)]
%    \end{macrocode}
%
%    For unique command names this package uses |hc@| as prefix
%    for internal command names.
%
%    First we check, if package \Package{hyperref} is loaded:
%    \begin{macrocode}
\@ifundefined{hyper@@anchor}{%
  \PackageError{hypcap}{You have to load 'hyperref' first}\@ehc
  \endinput
}{}
%    \end{macrocode}
%
%    \begin{macrocode}
\RequirePackage{letltxmacro}[2008/06/24]
%    \end{macrocode}
%
%    \begin{macro}{\hc@org@caption}
%    Save the original meaning of \cmd{\caption}:
%    \begin{macrocode}
\newcommand*\hc@org@caption{}
\let\hc@org@caption\caption
%    \end{macrocode}
%    \end{macro}
%
%    \newcommand*{\BeginMacro}[1]{\expandafter\DoBeginMacro\csname#1\endcsname}
%    \newcommand*{\DoBeginMacro}[1]{\begin{macro}{#1}}
%    \newcommand*{\Cmd}[1]{\expandafter\cmd\csname#1\endcsname}
%    \BeginMacro{if@capstart}
%     The switch \Cmd{if@capstart} helps to detect
%     \cmd{\capstart} commands with missing \cmd{\caption} macros.
%     Because \cmd{\caption} can occur inside a group, assignments
%     to the switch have to be made global.
%    \begin{macrocode}
\newif\if@capstart
%    \end{macrocode}
%    \end{macro}
%
%    \begin{macro}{\hypcapspace}
%    The anchor is raised.by \cmd{\hypcapspace}.
%    \begin{macrocode}
\newcommand*\hypcapspace{.5\baselineskip}
%    \end{macrocode}
%    \end{macro}
%
%    \begin{macro}{\ifcapstart}
%    \begin{macrocode}
\newif\ifcapstart
\capstarttrue
%    \end{macrocode}
%    \end{macro}
%
%    \begin{macro}{\capstart}
%     The macro \cmd{\capstart} contains the first part of
%     the \cmd{\caption} command: Incrementing the counter
%     and setting the anchor.
%    \begin{macrocode}
\newcommand*\capstart{%
  \ifcapstart
    \H@refstepcounter\@captype % first part of caption
    \hyper@makecurrent\@captype
    \global\let\hc@currentHref\@currentHref
    \vspace*{-\hypcapspace}%
    \begingroup
      \let\leavevmode\relax
      \hyper@@anchor\@currentHref\relax
    \endgroup
    \vspace*{\hypcapspace}%
    \hc@hyperref{\let\caption\hc@caption}%
    \global\@capstarttrue
    \global\advance\csname c@\@captype\endcsname\m@ne
  \fi
}
%    \end{macrocode}
%    \end{macro}
%
%    \begin{macrocode}
\@ifpackagelater{hyperref}{2007/04/09}{%
  \let\hc@hyperref\@gobble
}{%
  \let\hc@hyperref\@firstofone
}
%    \end{macrocode}
%
%    \begin{macro}{\hc@caption}
%    The new \cmd{\caption} command without the first part
%    is defined in the macro \cmd{\hc@caption}.
%    \begin{macrocode}
\def\hc@caption{%
  \global\advance\csname c@\@captype\endcsname\@ne
  \@dblarg{\hc@@caption\@captype}%
}
%    \end{macrocode}
%    \end{macro}
%
%    \begin{macro}{\hc@@caption}
%    This is a copy of package \Package{hyperref}'s \cmd{\@caption}
%    macro without making the anchor, because this is already done
%    in \cmd{\capstart}.
%    \begin{macrocode}
\long\def\hc@@caption#1[#2]#3{%
  \let\caption\hc@org@caption
  \global\@capstartfalse
  \ifHy@hypertexnames
    \hyper@makecurrent\@captype
  \else
    \global\let\@currentHref\hc@currentHref
  \fi
  \par\addcontentsline{%
    \csname ext@#1\endcsname}{#1}{%
    \protect\numberline{%
      \csname the#1\endcsname
    }{\ignorespaces #2}%
  }%
  \begingroup
    \@parboxrestore
    \normalsize
    \@makecaption{\csname fnum@#1\endcsname}{%
      \ignorespaces#3%
    }%
    \par
  \endgroup
}
%    \end{macrocode}
%    \end{macro}
%
%    \begin{macro}{\hypcapredef}
%    The macro \cmd{\hypcapredef} prepares the call of
%    \cmd{\hc@redef} that will redefine the environment
%    that is given in the argument.
%    \begin{macrocode}
\def\hypcapredef#1{%
  \expandafter\hc@redef\csname hc@org#1\expandafter\endcsname
                       \csname hc@orgend#1\expandafter\endcsname
                       \expandafter{#1}%
}
%    \end{macrocode}
%    \end{macro}
%    \begin{macro}{\hc@redef}
%    The old meaning of the environment is saved. Then
%    \cmd{\capstart} is appended in the begin part.
%    The end part contains a check
%    that produces an error message in case of \cmd{\capstart}
%    without \cmd{\capstart} (\cmd{\capstart} has incremented
%    the counter).
%    \begin{macrocode}
\def\hc@redef#1#2#3{%
  \newcommand#1{}%
  \expandafter\LetLtxMacro\expandafter#1\csname#3\endcsname
  \expandafter\LetLtxMacro\expandafter#2\csname end#3\endcsname
  \renewenvironment*{#3}[1][]{%
    \ifx\\##1\\%
      #1\relax
    \else
      #1[##1]% hash-ok (compatibility for float)
    \fi
    \capstart
  }{%
    \if@capstart
      \PackageError{hypcap}{You have forgotten to use \string\caption}%
      \global\@capstartfalse
    \else
    \fi
    #2%
  }%
}
%    \end{macrocode}
%    \end{macro}
%
%     At last the options are defined and processed.
%    \begin{macrocode}
\DeclareOption{figure}{\hypcapredef{\CurrentOption}}
\DeclareOption{figure*}{\hypcapredef{\CurrentOption}}
\DeclareOption{table}{\hypcapredef{\CurrentOption}}
\DeclareOption{table*}{\hypcapredef{\CurrentOption}}
\DeclareOption{all}{%
  \hypcapredef{figure}%
  \hypcapredef{figure*}%
  \hypcapredef{table}%
  \hypcapredef{table*}%
}
\ProcessOptions\relax
%    \end{macrocode}
%
%    \begin{macrocode}
%</package>
%    \end{macrocode}
%
% \section{Installation}
%
% \subsection{Download}
%
% \paragraph{Package.} This package is available on
% CTAN\footnote{\url{ftp://ftp.ctan.org/tex-archive/}}:
% \begin{description}
% \item[\CTAN{macros/latex/contrib/oberdiek/hypcap.dtx}] The source file.
% \item[\CTAN{macros/latex/contrib/oberdiek/hypcap.pdf}] Documentation.
% \end{description}
%
%
% \paragraph{Bundle.} All the packages of the bundle `oberdiek'
% are also available in a TDS compliant ZIP archive. There
% the packages are already unpacked and the documentation files
% are generated. The files and directories obey the TDS standard.
% \begin{description}
% \item[\CTAN{install/macros/latex/contrib/oberdiek.tds.zip}]
% \end{description}
% \emph{TDS} refers to the standard ``A Directory Structure
% for \TeX\ Files'' (\CTAN{tds/tds.pdf}). Directories
% with \xfile{texmf} in their name are usually organized this way.
%
% \subsection{Bundle installation}
%
% \paragraph{Unpacking.} Unpack the \xfile{oberdiek.tds.zip} in the
% TDS tree (also known as \xfile{texmf} tree) of your choice.
% Example (linux):
% \begin{quote}
%   |unzip oberdiek.tds.zip -d ~/texmf|
% \end{quote}
%
% \paragraph{Script installation.}
% Check the directory \xfile{TDS:scripts/oberdiek/} for
% scripts that need further installation steps.
% Package \xpackage{attachfile2} comes with the Perl script
% \xfile{pdfatfi.pl} that should be installed in such a way
% that it can be called as \texttt{pdfatfi}.
% Example (linux):
% \begin{quote}
%   |chmod +x scripts/oberdiek/pdfatfi.pl|\\
%   |cp scripts/oberdiek/pdfatfi.pl /usr/local/bin/|
% \end{quote}
%
% \subsection{Package installation}
%
% \paragraph{Unpacking.} The \xfile{.dtx} file is a self-extracting
% \docstrip\ archive. The files are extracted by running the
% \xfile{.dtx} through \plainTeX:
% \begin{quote}
%   \verb|tex hypcap.dtx|
% \end{quote}
%
% \paragraph{TDS.} Now the different files must be moved into
% the different directories in your installation TDS tree
% (also known as \xfile{texmf} tree):
% \begin{quote}
% \def\t{^^A
% \begin{tabular}{@{}>{\ttfamily}l@{ $\rightarrow$ }>{\ttfamily}l@{}}
%   hypcap.sty & tex/latex/oberdiek/hypcap.sty\\
%   hypcap.pdf & doc/latex/oberdiek/hypcap.pdf\\
%   hypcap.dtx & source/latex/oberdiek/hypcap.dtx\\
% \end{tabular}^^A
% }^^A
% \sbox0{\t}^^A
% \ifdim\wd0>\linewidth
%   \begingroup
%     \advance\linewidth by\leftmargin
%     \advance\linewidth by\rightmargin
%   \edef\x{\endgroup
%     \def\noexpand\lw{\the\linewidth}^^A
%   }\x
%   \def\lwbox{^^A
%     \leavevmode
%     \hbox to \linewidth{^^A
%       \kern-\leftmargin\relax
%       \hss
%       \usebox0
%       \hss
%       \kern-\rightmargin\relax
%     }^^A
%   }^^A
%   \ifdim\wd0>\lw
%     \sbox0{\small\t}^^A
%     \ifdim\wd0>\linewidth
%       \ifdim\wd0>\lw
%         \sbox0{\footnotesize\t}^^A
%         \ifdim\wd0>\linewidth
%           \ifdim\wd0>\lw
%             \sbox0{\scriptsize\t}^^A
%             \ifdim\wd0>\linewidth
%               \ifdim\wd0>\lw
%                 \sbox0{\tiny\t}^^A
%                 \ifdim\wd0>\linewidth
%                   \lwbox
%                 \else
%                   \usebox0
%                 \fi
%               \else
%                 \lwbox
%               \fi
%             \else
%               \usebox0
%             \fi
%           \else
%             \lwbox
%           \fi
%         \else
%           \usebox0
%         \fi
%       \else
%         \lwbox
%       \fi
%     \else
%       \usebox0
%     \fi
%   \else
%     \lwbox
%   \fi
% \else
%   \usebox0
% \fi
% \end{quote}
% If you have a \xfile{docstrip.cfg} that configures and enables \docstrip's
% TDS installing feature, then some files can already be in the right
% place, see the documentation of \docstrip.
%
% \subsection{Refresh file name databases}
%
% If your \TeX~distribution
% (\teTeX, \mikTeX, \dots) relies on file name databases, you must refresh
% these. For example, \teTeX\ users run \verb|texhash| or
% \verb|mktexlsr|.
%
% \subsection{Some details for the interested}
%
% \paragraph{Attached source.}
%
% The PDF documentation on CTAN also includes the
% \xfile{.dtx} source file. It can be extracted by
% AcrobatReader 6 or higher. Another option is \textsf{pdftk},
% e.g. unpack the file into the current directory:
% \begin{quote}
%   \verb|pdftk hypcap.pdf unpack_files output .|
% \end{quote}
%
% \paragraph{Unpacking with \LaTeX.}
% The \xfile{.dtx} chooses its action depending on the format:
% \begin{description}
% \item[\plainTeX:] Run \docstrip\ and extract the files.
% \item[\LaTeX:] Generate the documentation.
% \end{description}
% If you insist on using \LaTeX\ for \docstrip\ (really,
% \docstrip\ does not need \LaTeX), then inform the autodetect routine
% about your intention:
% \begin{quote}
%   \verb|latex \let\install=y\input{hypcap.dtx}|
% \end{quote}
% Do not forget to quote the argument according to the demands
% of your shell.
%
% \paragraph{Generating the documentation.}
% You can use both the \xfile{.dtx} or the \xfile{.drv} to generate
% the documentation. The process can be configured by the
% configuration file \xfile{ltxdoc.cfg}. For instance, put this
% line into this file, if you want to have A4 as paper format:
% \begin{quote}
%   \verb|\PassOptionsToClass{a4paper}{article}|
% \end{quote}
% An example follows how to generate the
% documentation with pdf\LaTeX:
% \begin{quote}
%\begin{verbatim}
%pdflatex hypcap.dtx
%makeindex -s gind.ist hypcap.idx
%pdflatex hypcap.dtx
%makeindex -s gind.ist hypcap.idx
%pdflatex hypcap.dtx
%\end{verbatim}
% \end{quote}
%
% \section{Catalogue}
%
% The following XML file can be used as source for the
% \href{http://mirror.ctan.org/help/Catalogue/catalogue.html}{\TeX\ Catalogue}.
% The elements \texttt{caption} and \texttt{description} are imported
% from the original XML file from the Catalogue.
% The name of the XML file in the Catalogue is \xfile{hypcap.xml}.
%    \begin{macrocode}
%<*catalogue>
<?xml version='1.0' encoding='us-ascii'?>
<!DOCTYPE entry SYSTEM 'catalogue.dtd'>
<entry datestamp='$Date$' modifier='$Author$' id='hypcap'>
  <name>hypcap</name>
  <caption>Adjusting the anchors of captions.</caption>
  <authorref id='auth:oberdiek'/>
  <copyright owner='Heiko Oberdiek' year='1999-2001,2006-2008,2011'/>
  <license type='lppl1.3'/>
  <version number='1.11'/>
  <description>
    The package offers a solution to the problem that when you link to
    a float using <xref refid='hyperref'>hyperref</xref>, the link
    anchors to below the float's caption, rather than the beginning of
    the float.
    <p/>
    Hypcap defines a separate \capstart command, which you put where
    you want links to end; you should have a \capstart command for each
    \caption command.  Package options can be used to auto-insert a
    \capstart at the start of a float environment.
    <p/>
    The package is part of the <xref refid='oberdiek'>oberdiek</xref> bundle.
  </description>
  <documentation details='Package documentation'
      href='ctan:/macros/latex/contrib/oberdiek/hypcap.pdf'/>
  <ctan file='true' path='/macros/latex/contrib/oberdiek/hypcap.dtx'/>
  <miktex location='oberdiek'/>
  <texlive location='oberdiek'/>
  <install path='/macros/latex/contrib/oberdiek/oberdiek.tds.zip'/>
</entry>
%</catalogue>
%    \end{macrocode}
%
% \begin{History}
%   \begin{Version}{1999/02/13 v1.0}
%   \item
%     A beginning version, published in newsgroup \xnewsgroup{comp.text.tex}:\\
%     \URL{``\link{Re: \xpackage{hyperref} and figures}''}^^A
%     {http://groups.google.com/group/comp.text.tex/msg/5c9b47b001a9379c}
%   \end{Version}
%   \begin{Version}{2000/08/14 v1.1}
%   \item
%     Global assignments of \cs{if@capstart}
%     in order to allow \cs{caption} in groups.
%   \item
%     Option \texttt{all} added.
%   \end{Version}
%   \begin{Version}{2000/09/07 v1.2}
%   \item
%     Package in dtx format.
%   \end{Version}
%   \begin{Version}{2001/08/27 v1.3}
%   \item
%     Bug fix with hyperref's pdfmark driver\\
%     (\cs{leavevmode} in
%     \cs{hyper@@anchor}/\cs{pdf@rect}).
%   \end{Version}
%   \begin{Version}{2001/09/06 v1.4}
%   \item
%     Small fixes in the dtx file.
%   \end{Version}
%   \begin{Version}{2006/02/20 v1.5}
%   \item
%     Code is not changed.
%   \item
%     New DTX framework.
%   \end{Version}
%   \begin{Version}{2007/02/19 v1.6}
%   \item
%     Fix for hypertexnames=false.
%   \end{Version}
%   \begin{Version}{2007/04/09 v1.7}
%   \item
%     Stuff in \cs{caption} moved to \xpackage{hyperref}.
%     This avoids redefinitions of \cs{caption} and \cs{@caption}
%     (idea of Axel Sommerfeldt).
%   \item
%     Fix for subfigure (Marco Kuhlmann, Amilcar do Carmo Lucas).
%   \end{Version}
%   \begin{Version}{2008/04/14 v1.8}
%   \item
%     \cs{hc@redef} fixed to get package \xpackage{float} work
%     (Axel Sommerfeldt).
%   \end{Version}
%   \begin{Version}{2008/08/11 v1.9}
%   \item
%     Code is not changed.
%   \item
%     URLs updated.
%   \end{Version}
%   \begin{Version}{2008/09/08 v1.10}
%   \item
%     \cs{capstartfalse} and \cs{capstarttrue} added.
%   \end{Version}
%   \begin{Version}{2011/02/16 v1.11}
%   \item
%     \cs{hc@redef} fixed by using package \xpackage{letltxmacro}.
%   \end{Version}
% \end{History}
%
% \PrintIndex
%
% \Finale
\endinput
|
% \end{quote}
% Do not forget to quote the argument according to the demands
% of your shell.
%
% \paragraph{Generating the documentation.}
% You can use both the \xfile{.dtx} or the \xfile{.drv} to generate
% the documentation. The process can be configured by the
% configuration file \xfile{ltxdoc.cfg}. For instance, put this
% line into this file, if you want to have A4 as paper format:
% \begin{quote}
%   \verb|\PassOptionsToClass{a4paper}{article}|
% \end{quote}
% An example follows how to generate the
% documentation with pdf\LaTeX:
% \begin{quote}
%\begin{verbatim}
%pdflatex hypcap.dtx
%makeindex -s gind.ist hypcap.idx
%pdflatex hypcap.dtx
%makeindex -s gind.ist hypcap.idx
%pdflatex hypcap.dtx
%\end{verbatim}
% \end{quote}
%
% \section{Catalogue}
%
% The following XML file can be used as source for the
% \href{http://mirror.ctan.org/help/Catalogue/catalogue.html}{\TeX\ Catalogue}.
% The elements \texttt{caption} and \texttt{description} are imported
% from the original XML file from the Catalogue.
% The name of the XML file in the Catalogue is \xfile{hypcap.xml}.
%    \begin{macrocode}
%<*catalogue>
<?xml version='1.0' encoding='us-ascii'?>
<!DOCTYPE entry SYSTEM 'catalogue.dtd'>
<entry datestamp='$Date$' modifier='$Author$' id='hypcap'>
  <name>hypcap</name>
  <caption>Adjusting the anchors of captions.</caption>
  <authorref id='auth:oberdiek'/>
  <copyright owner='Heiko Oberdiek' year='1999-2001,2006-2008,2011'/>
  <license type='lppl1.3'/>
  <version number='1.11'/>
  <description>
    The package offers a solution to the problem that when you link to
    a float using <xref refid='hyperref'>hyperref</xref>, the link
    anchors to below the float's caption, rather than the beginning of
    the float.
    <p/>
    Hypcap defines a separate \capstart command, which you put where
    you want links to end; you should have a \capstart command for each
    \caption command.  Package options can be used to auto-insert a
    \capstart at the start of a float environment.
    <p/>
    The package is part of the <xref refid='oberdiek'>oberdiek</xref> bundle.
  </description>
  <documentation details='Package documentation'
      href='ctan:/macros/latex/contrib/oberdiek/hypcap.pdf'/>
  <ctan file='true' path='/macros/latex/contrib/oberdiek/hypcap.dtx'/>
  <miktex location='oberdiek'/>
  <texlive location='oberdiek'/>
  <install path='/macros/latex/contrib/oberdiek/oberdiek.tds.zip'/>
</entry>
%</catalogue>
%    \end{macrocode}
%
% \begin{History}
%   \begin{Version}{1999/02/13 v1.0}
%   \item
%     A beginning version, published in newsgroup \xnewsgroup{comp.text.tex}:\\
%     \URL{``\link{Re: \xpackage{hyperref} and figures}''}^^A
%     {http://groups.google.com/group/comp.text.tex/msg/5c9b47b001a9379c}
%   \end{Version}
%   \begin{Version}{2000/08/14 v1.1}
%   \item
%     Global assignments of \cs{if@capstart}
%     in order to allow \cs{caption} in groups.
%   \item
%     Option \texttt{all} added.
%   \end{Version}
%   \begin{Version}{2000/09/07 v1.2}
%   \item
%     Package in dtx format.
%   \end{Version}
%   \begin{Version}{2001/08/27 v1.3}
%   \item
%     Bug fix with hyperref's pdfmark driver\\
%     (\cs{leavevmode} in
%     \cs{hyper@@anchor}/\cs{pdf@rect}).
%   \end{Version}
%   \begin{Version}{2001/09/06 v1.4}
%   \item
%     Small fixes in the dtx file.
%   \end{Version}
%   \begin{Version}{2006/02/20 v1.5}
%   \item
%     Code is not changed.
%   \item
%     New DTX framework.
%   \end{Version}
%   \begin{Version}{2007/02/19 v1.6}
%   \item
%     Fix for hypertexnames=false.
%   \end{Version}
%   \begin{Version}{2007/04/09 v1.7}
%   \item
%     Stuff in \cs{caption} moved to \xpackage{hyperref}.
%     This avoids redefinitions of \cs{caption} and \cs{@caption}
%     (idea of Axel Sommerfeldt).
%   \item
%     Fix for subfigure (Marco Kuhlmann, Amilcar do Carmo Lucas).
%   \end{Version}
%   \begin{Version}{2008/04/14 v1.8}
%   \item
%     \cs{hc@redef} fixed to get package \xpackage{float} work
%     (Axel Sommerfeldt).
%   \end{Version}
%   \begin{Version}{2008/08/11 v1.9}
%   \item
%     Code is not changed.
%   \item
%     URLs updated.
%   \end{Version}
%   \begin{Version}{2008/09/08 v1.10}
%   \item
%     \cs{capstartfalse} and \cs{capstarttrue} added.
%   \end{Version}
%   \begin{Version}{2011/02/16 v1.11}
%   \item
%     \cs{hc@redef} fixed by using package \xpackage{letltxmacro}.
%   \end{Version}
% \end{History}
%
% \PrintIndex
%
% \Finale
\endinput

%        (quote the arguments according to the demands of your shell)
%
% Documentation:
%    (a) If hypcap.drv is present:
%           latex hypcap.drv
%    (b) Without hypcap.drv:
%           latex hypcap.dtx; ...
%    The class ltxdoc loads the configuration file ltxdoc.cfg
%    if available. Here you can specify further options, e.g.
%    use A4 as paper format:
%       \PassOptionsToClass{a4paper}{article}
%
%    Programm calls to get the documentation (example):
%       pdflatex hypcap.dtx
%       makeindex -s gind.ist hypcap.idx
%       pdflatex hypcap.dtx
%       makeindex -s gind.ist hypcap.idx
%       pdflatex hypcap.dtx
%
% Installation:
%    TDS:tex/latex/oberdiek/hypcap.sty
%    TDS:doc/latex/oberdiek/hypcap.pdf
%    TDS:source/latex/oberdiek/hypcap.dtx
%
%<*ignore>
\begingroup
  \catcode123=1 %
  \catcode125=2 %
  \def\x{LaTeX2e}%
\expandafter\endgroup
\ifcase 0\ifx\install y1\fi\expandafter
         \ifx\csname processbatchFile\endcsname\relax\else1\fi
         \ifx\fmtname\x\else 1\fi\relax
\else\csname fi\endcsname
%</ignore>
%<*install>
\input docstrip.tex
\Msg{************************************************************************}
\Msg{* Installation}
\Msg{* Package: hypcap 2011/02/16 v1.11 Adjusting the anchors of captions (HO)}
\Msg{************************************************************************}

\keepsilent
\askforoverwritefalse

\let\MetaPrefix\relax
\preamble

This is a generated file.

Project: hypcap
Version: 2011/02/16 v1.11

Copyright (C) 1999-2001, 2006-2008, 2011 by
   Heiko Oberdiek <heiko.oberdiek at googlemail.com>

This work may be distributed and/or modified under the
conditions of the LaTeX Project Public License, either
version 1.3c of this license or (at your option) any later
version. This version of this license is in
   http://www.latex-project.org/lppl/lppl-1-3c.txt
and the latest version of this license is in
   http://www.latex-project.org/lppl.txt
and version 1.3 or later is part of all distributions of
LaTeX version 2005/12/01 or later.

This work has the LPPL maintenance status "maintained".

This Current Maintainer of this work is Heiko Oberdiek.

This work consists of the main source file hypcap.dtx
and the derived files
   hypcap.sty, hypcap.pdf, hypcap.ins, hypcap.drv.

\endpreamble
\let\MetaPrefix\DoubleperCent

\generate{%
  \file{hypcap.ins}{\from{hypcap.dtx}{install}}%
  \file{hypcap.drv}{\from{hypcap.dtx}{driver}}%
  \usedir{tex/latex/oberdiek}%
  \file{hypcap.sty}{\from{hypcap.dtx}{package}}%
  \nopreamble
  \nopostamble
  \usedir{source/latex/oberdiek/catalogue}%
  \file{hypcap.xml}{\from{hypcap.dtx}{catalogue}}%
}

\catcode32=13\relax% active space
\let =\space%
\Msg{************************************************************************}
\Msg{*}
\Msg{* To finish the installation you have to move the following}
\Msg{* file into a directory searched by TeX:}
\Msg{*}
\Msg{*     hypcap.sty}
\Msg{*}
\Msg{* To produce the documentation run the file `hypcap.drv'}
\Msg{* through LaTeX.}
\Msg{*}
\Msg{* Happy TeXing!}
\Msg{*}
\Msg{************************************************************************}

\endbatchfile
%</install>
%<*ignore>
\fi
%</ignore>
%<*driver>
\NeedsTeXFormat{LaTeX2e}
\ProvidesFile{hypcap.drv}%
  [2011/02/16 v1.11 Adjusting the anchors of captions (HO)]%
\documentclass{ltxdoc}
\usepackage{holtxdoc}[2011/11/22]
\begin{document}
  \DocInput{hypcap.dtx}%
\end{document}
%</driver>
% \fi
%
% \CheckSum{174}
%
% \CharacterTable
%  {Upper-case    \A\B\C\D\E\F\G\H\I\J\K\L\M\N\O\P\Q\R\S\T\U\V\W\X\Y\Z
%   Lower-case    \a\b\c\d\e\f\g\h\i\j\k\l\m\n\o\p\q\r\s\t\u\v\w\x\y\z
%   Digits        \0\1\2\3\4\5\6\7\8\9
%   Exclamation   \!     Double quote  \"     Hash (number) \#
%   Dollar        \$     Percent       \%     Ampersand     \&
%   Acute accent  \'     Left paren    \(     Right paren   \)
%   Asterisk      \*     Plus          \+     Comma         \,
%   Minus         \-     Point         \.     Solidus       \/
%   Colon         \:     Semicolon     \;     Less than     \<
%   Equals        \=     Greater than  \>     Question mark \?
%   Commercial at \@     Left bracket  \[     Backslash     \\
%   Right bracket \]     Circumflex    \^     Underscore    \_
%   Grave accent  \`     Left brace    \{     Vertical bar  \|
%   Right brace   \}     Tilde         \~}
%
% \GetFileInfo{hypcap.drv}
%
% \title{The \xpackage{hypcap} package}
% \date{2011/02/16 v1.11}
% \author{Heiko Oberdiek\\\xemail{heiko.oberdiek at googlemail.com}}
%
% \maketitle
%
% \begin{abstract}
% This package tries a solution of the problem with
% hyperref, that links to floats points below the
% caption and not at the beginning of the float.
% Therefore this package divides the task into two
% part, the link setting with \cs{capstart} or
% automatically at the beginning of a float and
% the rest in the \cs{caption} command.
% \end{abstract}
%
% \tableofcontents
%
% \newpage
% \section{Usage}
%    The package \Package{hypcap} requires that \Package{hyperref}
%    is loaded first:
%    \begin{quote}
%      |\usepackage[...]{hyperref}|\\
%      |\usepackage[...]{hypcap}|
%    \end{quote}
%
% \subsection{Package options}
%    The names of the four float environments |figure|, |figure*|,
%    |table|, or |table*| can be used as option. Then the package
%    redefines the environment in order to insert \cmd{\capstart}
%    (see below) in the beginning of the environment automatically.
%
%    Option |all| enables the redefinitions of all four
%    float environments. For other environments see
%    the user command \cmd{\hypcapredef}.
%
% \subsection{User commands}
%    \begin{description}
%    \item[\cmd{\capstart}:]\DescribeMacro{\capstart}
%      First this command increments the counter (\cmd{\@captype}).
%      Then it makes an anchor for package \Package{hyperref}.
%      At last \cmd{\caption} is redefined to remove the
%      anchor setting part from \Package{hyperref}'s \cmd{\caption}.
%
%      The package expects the following structure of a float
%      environment:
%      \begin{quote}
%        |\begin{|\textit{|float|}|}...|\\
%        |\capstart|\\
%        |...|\\
%        |\caption{...}|\\
%        |...|\\
%        |\end{|\textit{|float|}|}|
%      \end{quote}
%      There can be several \cmd{\caption} commands. For these
%      you need \cmd{\capstart} again:
%      \begin{quote}
%        |\capstart ... \caption... \capstart ... \caption...|
%      \end{quote}
%      And the \cmd{\caption} command itself can be put in a group.
%
%      With the options, described above, the extra
%      writing of \cmd{\capstart} can be avoided.
%      Consequently, there must be a \cmd{\caption} in every
%      environment of this type, specified by the option.
%      If you want to use more than one \cmd{\caption} in this
%      environment, you have to state \cmd{\capstart} again.
%    \item[\cmd{\hypcapspace}:]\DescribeMacro{\hypcapspace}
%      Because it looks poor, if the link points exactly at top of
%      the figure, there is additional space: \cmd{\hypcapspace},
%      the default is |0.5\baselineskip|, examples:
%      \begin{quote}
%        |\renewcommand{\hypcapspace}{0pt}| removes the space\\
%        |\renewcommand{\hypcapspace}{1pt}| sets a fix value
%      \end{quote}
%    \item[\cmd{\hypcapredef}:]\DescribeMacro{\hypcapredef}
%      If there are other float environments, that
%      should automatically execute \cmd{\capstart}, then
%      a redefinition with \cmd{\hypcapredef} can be tried:
%      \begin{quote}
%        |\hypcapredef{myfloat}|
%      \end{quote}
%      Only environments with one optional parameter are
%      supported.
%    \item[\cs{capstartfalse}, \cs{capstarttrue}:]^^A
%      \DescribeMacro{\capstartfalse}^^A
%      \DescribeMacro{\capstarttrue}^^A
%      Since 2008/09/08 v1.10.\\
%      They disable and enable \cs{capstart}. They can be used to
%      cancel the effect of a redefined float environment.
%      Example:
%\begin{quote}
%\catcode`Y=\active
%\def Y#1Z{\textit{\cs{#1}}}
%\begin{verbatim}
%\documentclass{article}
%\usepackage{hyperref}
%\usepackage[figure]{hypcap}[2008/09/08]
%
%\begin{document}
%  \section{Hello World}
%  \begin{figure}
%    \caption{Figure with caption A}
%  \end{figure}
%  YcapstartfalseZ
%  \begin{figure}
%    Figure without caption
%  \end{figure}
%  YcapstarttrueZ
%  \begin{figure}
%    \caption{Figure with caption B}
%  \end{figure}
%\end{document}
%\end{verbatim}
%\end{quote}
%    \end{description}
%
% \subsection{Limitations}
%    \begin{itemize}
%    \item Packages that redefine \cmd{\caption} or
%          \cmd{\@caption}.
%    \end{itemize}
%
% \StopEventually{
% }
%
% \section{Implementation}
%    \begin{macrocode}
%<*package>
%    \end{macrocode}
%    Package identification.
%    \begin{macrocode}
\NeedsTeXFormat{LaTeX2e}
\ProvidesPackage{hypcap}%
  [2011/02/16 v1.11 Adjusting the anchors of captions (HO)]
%    \end{macrocode}
%
%    For unique command names this package uses |hc@| as prefix
%    for internal command names.
%
%    First we check, if package \Package{hyperref} is loaded:
%    \begin{macrocode}
\@ifundefined{hyper@@anchor}{%
  \PackageError{hypcap}{You have to load 'hyperref' first}\@ehc
  \endinput
}{}
%    \end{macrocode}
%
%    \begin{macrocode}
\RequirePackage{letltxmacro}[2008/06/24]
%    \end{macrocode}
%
%    \begin{macro}{\hc@org@caption}
%    Save the original meaning of \cmd{\caption}:
%    \begin{macrocode}
\newcommand*\hc@org@caption{}
\let\hc@org@caption\caption
%    \end{macrocode}
%    \end{macro}
%
%    \newcommand*{\BeginMacro}[1]{\expandafter\DoBeginMacro\csname#1\endcsname}
%    \newcommand*{\DoBeginMacro}[1]{\begin{macro}{#1}}
%    \newcommand*{\Cmd}[1]{\expandafter\cmd\csname#1\endcsname}
%    \BeginMacro{if@capstart}
%     The switch \Cmd{if@capstart} helps to detect
%     \cmd{\capstart} commands with missing \cmd{\caption} macros.
%     Because \cmd{\caption} can occur inside a group, assignments
%     to the switch have to be made global.
%    \begin{macrocode}
\newif\if@capstart
%    \end{macrocode}
%    \end{macro}
%
%    \begin{macro}{\hypcapspace}
%    The anchor is raised.by \cmd{\hypcapspace}.
%    \begin{macrocode}
\newcommand*\hypcapspace{.5\baselineskip}
%    \end{macrocode}
%    \end{macro}
%
%    \begin{macro}{\ifcapstart}
%    \begin{macrocode}
\newif\ifcapstart
\capstarttrue
%    \end{macrocode}
%    \end{macro}
%
%    \begin{macro}{\capstart}
%     The macro \cmd{\capstart} contains the first part of
%     the \cmd{\caption} command: Incrementing the counter
%     and setting the anchor.
%    \begin{macrocode}
\newcommand*\capstart{%
  \ifcapstart
    \H@refstepcounter\@captype % first part of caption
    \hyper@makecurrent\@captype
    \global\let\hc@currentHref\@currentHref
    \vspace*{-\hypcapspace}%
    \begingroup
      \let\leavevmode\relax
      \hyper@@anchor\@currentHref\relax
    \endgroup
    \vspace*{\hypcapspace}%
    \hc@hyperref{\let\caption\hc@caption}%
    \global\@capstarttrue
    \global\advance\csname c@\@captype\endcsname\m@ne
  \fi
}
%    \end{macrocode}
%    \end{macro}
%
%    \begin{macrocode}
\@ifpackagelater{hyperref}{2007/04/09}{%
  \let\hc@hyperref\@gobble
}{%
  \let\hc@hyperref\@firstofone
}
%    \end{macrocode}
%
%    \begin{macro}{\hc@caption}
%    The new \cmd{\caption} command without the first part
%    is defined in the macro \cmd{\hc@caption}.
%    \begin{macrocode}
\def\hc@caption{%
  \global\advance\csname c@\@captype\endcsname\@ne
  \@dblarg{\hc@@caption\@captype}%
}
%    \end{macrocode}
%    \end{macro}
%
%    \begin{macro}{\hc@@caption}
%    This is a copy of package \Package{hyperref}'s \cmd{\@caption}
%    macro without making the anchor, because this is already done
%    in \cmd{\capstart}.
%    \begin{macrocode}
\long\def\hc@@caption#1[#2]#3{%
  \let\caption\hc@org@caption
  \global\@capstartfalse
  \ifHy@hypertexnames
    \hyper@makecurrent\@captype
  \else
    \global\let\@currentHref\hc@currentHref
  \fi
  \par\addcontentsline{%
    \csname ext@#1\endcsname}{#1}{%
    \protect\numberline{%
      \csname the#1\endcsname
    }{\ignorespaces #2}%
  }%
  \begingroup
    \@parboxrestore
    \normalsize
    \@makecaption{\csname fnum@#1\endcsname}{%
      \ignorespaces#3%
    }%
    \par
  \endgroup
}
%    \end{macrocode}
%    \end{macro}
%
%    \begin{macro}{\hypcapredef}
%    The macro \cmd{\hypcapredef} prepares the call of
%    \cmd{\hc@redef} that will redefine the environment
%    that is given in the argument.
%    \begin{macrocode}
\def\hypcapredef#1{%
  \expandafter\hc@redef\csname hc@org#1\expandafter\endcsname
                       \csname hc@orgend#1\expandafter\endcsname
                       \expandafter{#1}%
}
%    \end{macrocode}
%    \end{macro}
%    \begin{macro}{\hc@redef}
%    The old meaning of the environment is saved. Then
%    \cmd{\capstart} is appended in the begin part.
%    The end part contains a check
%    that produces an error message in case of \cmd{\capstart}
%    without \cmd{\capstart} (\cmd{\capstart} has incremented
%    the counter).
%    \begin{macrocode}
\def\hc@redef#1#2#3{%
  \newcommand#1{}%
  \expandafter\LetLtxMacro\expandafter#1\csname#3\endcsname
  \expandafter\LetLtxMacro\expandafter#2\csname end#3\endcsname
  \renewenvironment*{#3}[1][]{%
    \ifx\\##1\\%
      #1\relax
    \else
      #1[##1]% hash-ok (compatibility for float)
    \fi
    \capstart
  }{%
    \if@capstart
      \PackageError{hypcap}{You have forgotten to use \string\caption}%
      \global\@capstartfalse
    \else
    \fi
    #2%
  }%
}
%    \end{macrocode}
%    \end{macro}
%
%     At last the options are defined and processed.
%    \begin{macrocode}
\DeclareOption{figure}{\hypcapredef{\CurrentOption}}
\DeclareOption{figure*}{\hypcapredef{\CurrentOption}}
\DeclareOption{table}{\hypcapredef{\CurrentOption}}
\DeclareOption{table*}{\hypcapredef{\CurrentOption}}
\DeclareOption{all}{%
  \hypcapredef{figure}%
  \hypcapredef{figure*}%
  \hypcapredef{table}%
  \hypcapredef{table*}%
}
\ProcessOptions\relax
%    \end{macrocode}
%
%    \begin{macrocode}
%</package>
%    \end{macrocode}
%
% \section{Installation}
%
% \subsection{Download}
%
% \paragraph{Package.} This package is available on
% CTAN\footnote{\url{ftp://ftp.ctan.org/tex-archive/}}:
% \begin{description}
% \item[\CTAN{macros/latex/contrib/oberdiek/hypcap.dtx}] The source file.
% \item[\CTAN{macros/latex/contrib/oberdiek/hypcap.pdf}] Documentation.
% \end{description}
%
%
% \paragraph{Bundle.} All the packages of the bundle `oberdiek'
% are also available in a TDS compliant ZIP archive. There
% the packages are already unpacked and the documentation files
% are generated. The files and directories obey the TDS standard.
% \begin{description}
% \item[\CTAN{install/macros/latex/contrib/oberdiek.tds.zip}]
% \end{description}
% \emph{TDS} refers to the standard ``A Directory Structure
% for \TeX\ Files'' (\CTAN{tds/tds.pdf}). Directories
% with \xfile{texmf} in their name are usually organized this way.
%
% \subsection{Bundle installation}
%
% \paragraph{Unpacking.} Unpack the \xfile{oberdiek.tds.zip} in the
% TDS tree (also known as \xfile{texmf} tree) of your choice.
% Example (linux):
% \begin{quote}
%   |unzip oberdiek.tds.zip -d ~/texmf|
% \end{quote}
%
% \paragraph{Script installation.}
% Check the directory \xfile{TDS:scripts/oberdiek/} for
% scripts that need further installation steps.
% Package \xpackage{attachfile2} comes with the Perl script
% \xfile{pdfatfi.pl} that should be installed in such a way
% that it can be called as \texttt{pdfatfi}.
% Example (linux):
% \begin{quote}
%   |chmod +x scripts/oberdiek/pdfatfi.pl|\\
%   |cp scripts/oberdiek/pdfatfi.pl /usr/local/bin/|
% \end{quote}
%
% \subsection{Package installation}
%
% \paragraph{Unpacking.} The \xfile{.dtx} file is a self-extracting
% \docstrip\ archive. The files are extracted by running the
% \xfile{.dtx} through \plainTeX:
% \begin{quote}
%   \verb|tex hypcap.dtx|
% \end{quote}
%
% \paragraph{TDS.} Now the different files must be moved into
% the different directories in your installation TDS tree
% (also known as \xfile{texmf} tree):
% \begin{quote}
% \def\t{^^A
% \begin{tabular}{@{}>{\ttfamily}l@{ $\rightarrow$ }>{\ttfamily}l@{}}
%   hypcap.sty & tex/latex/oberdiek/hypcap.sty\\
%   hypcap.pdf & doc/latex/oberdiek/hypcap.pdf\\
%   hypcap.dtx & source/latex/oberdiek/hypcap.dtx\\
% \end{tabular}^^A
% }^^A
% \sbox0{\t}^^A
% \ifdim\wd0>\linewidth
%   \begingroup
%     \advance\linewidth by\leftmargin
%     \advance\linewidth by\rightmargin
%   \edef\x{\endgroup
%     \def\noexpand\lw{\the\linewidth}^^A
%   }\x
%   \def\lwbox{^^A
%     \leavevmode
%     \hbox to \linewidth{^^A
%       \kern-\leftmargin\relax
%       \hss
%       \usebox0
%       \hss
%       \kern-\rightmargin\relax
%     }^^A
%   }^^A
%   \ifdim\wd0>\lw
%     \sbox0{\small\t}^^A
%     \ifdim\wd0>\linewidth
%       \ifdim\wd0>\lw
%         \sbox0{\footnotesize\t}^^A
%         \ifdim\wd0>\linewidth
%           \ifdim\wd0>\lw
%             \sbox0{\scriptsize\t}^^A
%             \ifdim\wd0>\linewidth
%               \ifdim\wd0>\lw
%                 \sbox0{\tiny\t}^^A
%                 \ifdim\wd0>\linewidth
%                   \lwbox
%                 \else
%                   \usebox0
%                 \fi
%               \else
%                 \lwbox
%               \fi
%             \else
%               \usebox0
%             \fi
%           \else
%             \lwbox
%           \fi
%         \else
%           \usebox0
%         \fi
%       \else
%         \lwbox
%       \fi
%     \else
%       \usebox0
%     \fi
%   \else
%     \lwbox
%   \fi
% \else
%   \usebox0
% \fi
% \end{quote}
% If you have a \xfile{docstrip.cfg} that configures and enables \docstrip's
% TDS installing feature, then some files can already be in the right
% place, see the documentation of \docstrip.
%
% \subsection{Refresh file name databases}
%
% If your \TeX~distribution
% (\teTeX, \mikTeX, \dots) relies on file name databases, you must refresh
% these. For example, \teTeX\ users run \verb|texhash| or
% \verb|mktexlsr|.
%
% \subsection{Some details for the interested}
%
% \paragraph{Attached source.}
%
% The PDF documentation on CTAN also includes the
% \xfile{.dtx} source file. It can be extracted by
% AcrobatReader 6 or higher. Another option is \textsf{pdftk},
% e.g. unpack the file into the current directory:
% \begin{quote}
%   \verb|pdftk hypcap.pdf unpack_files output .|
% \end{quote}
%
% \paragraph{Unpacking with \LaTeX.}
% The \xfile{.dtx} chooses its action depending on the format:
% \begin{description}
% \item[\plainTeX:] Run \docstrip\ and extract the files.
% \item[\LaTeX:] Generate the documentation.
% \end{description}
% If you insist on using \LaTeX\ for \docstrip\ (really,
% \docstrip\ does not need \LaTeX), then inform the autodetect routine
% about your intention:
% \begin{quote}
%   \verb|latex \let\install=y% \iffalse meta-comment
%
% File: hypcap.dtx
% Version: 2011/02/16 v1.11
% Info: Adjusting the anchors of captions
%
% Copyright (C) 1999-2001, 2006-2008, 2011 by
%    Heiko Oberdiek <heiko.oberdiek at googlemail.com>
%
% This work may be distributed and/or modified under the
% conditions of the LaTeX Project Public License, either
% version 1.3c of this license or (at your option) any later
% version. This version of this license is in
%    http://www.latex-project.org/lppl/lppl-1-3c.txt
% and the latest version of this license is in
%    http://www.latex-project.org/lppl.txt
% and version 1.3 or later is part of all distributions of
% LaTeX version 2005/12/01 or later.
%
% This work has the LPPL maintenance status "maintained".
%
% This Current Maintainer of this work is Heiko Oberdiek.
%
% This work consists of the main source file hypcap.dtx
% and the derived files
%    hypcap.sty, hypcap.pdf, hypcap.ins, hypcap.drv.
%
% Distribution:
%    CTAN:macros/latex/contrib/oberdiek/hypcap.dtx
%    CTAN:macros/latex/contrib/oberdiek/hypcap.pdf
%
% Unpacking:
%    (a) If hypcap.ins is present:
%           tex hypcap.ins
%    (b) Without hypcap.ins:
%           tex hypcap.dtx
%    (c) If you insist on using LaTeX
%           latex \let\install=y% \iffalse meta-comment
%
% File: hypcap.dtx
% Version: 2011/02/16 v1.11
% Info: Adjusting the anchors of captions
%
% Copyright (C) 1999-2001, 2006-2008, 2011 by
%    Heiko Oberdiek <heiko.oberdiek at googlemail.com>
%
% This work may be distributed and/or modified under the
% conditions of the LaTeX Project Public License, either
% version 1.3c of this license or (at your option) any later
% version. This version of this license is in
%    http://www.latex-project.org/lppl/lppl-1-3c.txt
% and the latest version of this license is in
%    http://www.latex-project.org/lppl.txt
% and version 1.3 or later is part of all distributions of
% LaTeX version 2005/12/01 or later.
%
% This work has the LPPL maintenance status "maintained".
%
% This Current Maintainer of this work is Heiko Oberdiek.
%
% This work consists of the main source file hypcap.dtx
% and the derived files
%    hypcap.sty, hypcap.pdf, hypcap.ins, hypcap.drv.
%
% Distribution:
%    CTAN:macros/latex/contrib/oberdiek/hypcap.dtx
%    CTAN:macros/latex/contrib/oberdiek/hypcap.pdf
%
% Unpacking:
%    (a) If hypcap.ins is present:
%           tex hypcap.ins
%    (b) Without hypcap.ins:
%           tex hypcap.dtx
%    (c) If you insist on using LaTeX
%           latex \let\install=y\input{hypcap.dtx}
%        (quote the arguments according to the demands of your shell)
%
% Documentation:
%    (a) If hypcap.drv is present:
%           latex hypcap.drv
%    (b) Without hypcap.drv:
%           latex hypcap.dtx; ...
%    The class ltxdoc loads the configuration file ltxdoc.cfg
%    if available. Here you can specify further options, e.g.
%    use A4 as paper format:
%       \PassOptionsToClass{a4paper}{article}
%
%    Programm calls to get the documentation (example):
%       pdflatex hypcap.dtx
%       makeindex -s gind.ist hypcap.idx
%       pdflatex hypcap.dtx
%       makeindex -s gind.ist hypcap.idx
%       pdflatex hypcap.dtx
%
% Installation:
%    TDS:tex/latex/oberdiek/hypcap.sty
%    TDS:doc/latex/oberdiek/hypcap.pdf
%    TDS:source/latex/oberdiek/hypcap.dtx
%
%<*ignore>
\begingroup
  \catcode123=1 %
  \catcode125=2 %
  \def\x{LaTeX2e}%
\expandafter\endgroup
\ifcase 0\ifx\install y1\fi\expandafter
         \ifx\csname processbatchFile\endcsname\relax\else1\fi
         \ifx\fmtname\x\else 1\fi\relax
\else\csname fi\endcsname
%</ignore>
%<*install>
\input docstrip.tex
\Msg{************************************************************************}
\Msg{* Installation}
\Msg{* Package: hypcap 2011/02/16 v1.11 Adjusting the anchors of captions (HO)}
\Msg{************************************************************************}

\keepsilent
\askforoverwritefalse

\let\MetaPrefix\relax
\preamble

This is a generated file.

Project: hypcap
Version: 2011/02/16 v1.11

Copyright (C) 1999-2001, 2006-2008, 2011 by
   Heiko Oberdiek <heiko.oberdiek at googlemail.com>

This work may be distributed and/or modified under the
conditions of the LaTeX Project Public License, either
version 1.3c of this license or (at your option) any later
version. This version of this license is in
   http://www.latex-project.org/lppl/lppl-1-3c.txt
and the latest version of this license is in
   http://www.latex-project.org/lppl.txt
and version 1.3 or later is part of all distributions of
LaTeX version 2005/12/01 or later.

This work has the LPPL maintenance status "maintained".

This Current Maintainer of this work is Heiko Oberdiek.

This work consists of the main source file hypcap.dtx
and the derived files
   hypcap.sty, hypcap.pdf, hypcap.ins, hypcap.drv.

\endpreamble
\let\MetaPrefix\DoubleperCent

\generate{%
  \file{hypcap.ins}{\from{hypcap.dtx}{install}}%
  \file{hypcap.drv}{\from{hypcap.dtx}{driver}}%
  \usedir{tex/latex/oberdiek}%
  \file{hypcap.sty}{\from{hypcap.dtx}{package}}%
  \nopreamble
  \nopostamble
  \usedir{source/latex/oberdiek/catalogue}%
  \file{hypcap.xml}{\from{hypcap.dtx}{catalogue}}%
}

\catcode32=13\relax% active space
\let =\space%
\Msg{************************************************************************}
\Msg{*}
\Msg{* To finish the installation you have to move the following}
\Msg{* file into a directory searched by TeX:}
\Msg{*}
\Msg{*     hypcap.sty}
\Msg{*}
\Msg{* To produce the documentation run the file `hypcap.drv'}
\Msg{* through LaTeX.}
\Msg{*}
\Msg{* Happy TeXing!}
\Msg{*}
\Msg{************************************************************************}

\endbatchfile
%</install>
%<*ignore>
\fi
%</ignore>
%<*driver>
\NeedsTeXFormat{LaTeX2e}
\ProvidesFile{hypcap.drv}%
  [2011/02/16 v1.11 Adjusting the anchors of captions (HO)]%
\documentclass{ltxdoc}
\usepackage{holtxdoc}[2011/11/22]
\begin{document}
  \DocInput{hypcap.dtx}%
\end{document}
%</driver>
% \fi
%
% \CheckSum{174}
%
% \CharacterTable
%  {Upper-case    \A\B\C\D\E\F\G\H\I\J\K\L\M\N\O\P\Q\R\S\T\U\V\W\X\Y\Z
%   Lower-case    \a\b\c\d\e\f\g\h\i\j\k\l\m\n\o\p\q\r\s\t\u\v\w\x\y\z
%   Digits        \0\1\2\3\4\5\6\7\8\9
%   Exclamation   \!     Double quote  \"     Hash (number) \#
%   Dollar        \$     Percent       \%     Ampersand     \&
%   Acute accent  \'     Left paren    \(     Right paren   \)
%   Asterisk      \*     Plus          \+     Comma         \,
%   Minus         \-     Point         \.     Solidus       \/
%   Colon         \:     Semicolon     \;     Less than     \<
%   Equals        \=     Greater than  \>     Question mark \?
%   Commercial at \@     Left bracket  \[     Backslash     \\
%   Right bracket \]     Circumflex    \^     Underscore    \_
%   Grave accent  \`     Left brace    \{     Vertical bar  \|
%   Right brace   \}     Tilde         \~}
%
% \GetFileInfo{hypcap.drv}
%
% \title{The \xpackage{hypcap} package}
% \date{2011/02/16 v1.11}
% \author{Heiko Oberdiek\\\xemail{heiko.oberdiek at googlemail.com}}
%
% \maketitle
%
% \begin{abstract}
% This package tries a solution of the problem with
% hyperref, that links to floats points below the
% caption and not at the beginning of the float.
% Therefore this package divides the task into two
% part, the link setting with \cs{capstart} or
% automatically at the beginning of a float and
% the rest in the \cs{caption} command.
% \end{abstract}
%
% \tableofcontents
%
% \newpage
% \section{Usage}
%    The package \Package{hypcap} requires that \Package{hyperref}
%    is loaded first:
%    \begin{quote}
%      |\usepackage[...]{hyperref}|\\
%      |\usepackage[...]{hypcap}|
%    \end{quote}
%
% \subsection{Package options}
%    The names of the four float environments |figure|, |figure*|,
%    |table|, or |table*| can be used as option. Then the package
%    redefines the environment in order to insert \cmd{\capstart}
%    (see below) in the beginning of the environment automatically.
%
%    Option |all| enables the redefinitions of all four
%    float environments. For other environments see
%    the user command \cmd{\hypcapredef}.
%
% \subsection{User commands}
%    \begin{description}
%    \item[\cmd{\capstart}:]\DescribeMacro{\capstart}
%      First this command increments the counter (\cmd{\@captype}).
%      Then it makes an anchor for package \Package{hyperref}.
%      At last \cmd{\caption} is redefined to remove the
%      anchor setting part from \Package{hyperref}'s \cmd{\caption}.
%
%      The package expects the following structure of a float
%      environment:
%      \begin{quote}
%        |\begin{|\textit{|float|}|}...|\\
%        |\capstart|\\
%        |...|\\
%        |\caption{...}|\\
%        |...|\\
%        |\end{|\textit{|float|}|}|
%      \end{quote}
%      There can be several \cmd{\caption} commands. For these
%      you need \cmd{\capstart} again:
%      \begin{quote}
%        |\capstart ... \caption... \capstart ... \caption...|
%      \end{quote}
%      And the \cmd{\caption} command itself can be put in a group.
%
%      With the options, described above, the extra
%      writing of \cmd{\capstart} can be avoided.
%      Consequently, there must be a \cmd{\caption} in every
%      environment of this type, specified by the option.
%      If you want to use more than one \cmd{\caption} in this
%      environment, you have to state \cmd{\capstart} again.
%    \item[\cmd{\hypcapspace}:]\DescribeMacro{\hypcapspace}
%      Because it looks poor, if the link points exactly at top of
%      the figure, there is additional space: \cmd{\hypcapspace},
%      the default is |0.5\baselineskip|, examples:
%      \begin{quote}
%        |\renewcommand{\hypcapspace}{0pt}| removes the space\\
%        |\renewcommand{\hypcapspace}{1pt}| sets a fix value
%      \end{quote}
%    \item[\cmd{\hypcapredef}:]\DescribeMacro{\hypcapredef}
%      If there are other float environments, that
%      should automatically execute \cmd{\capstart}, then
%      a redefinition with \cmd{\hypcapredef} can be tried:
%      \begin{quote}
%        |\hypcapredef{myfloat}|
%      \end{quote}
%      Only environments with one optional parameter are
%      supported.
%    \item[\cs{capstartfalse}, \cs{capstarttrue}:]^^A
%      \DescribeMacro{\capstartfalse}^^A
%      \DescribeMacro{\capstarttrue}^^A
%      Since 2008/09/08 v1.10.\\
%      They disable and enable \cs{capstart}. They can be used to
%      cancel the effect of a redefined float environment.
%      Example:
%\begin{quote}
%\catcode`Y=\active
%\def Y#1Z{\textit{\cs{#1}}}
%\begin{verbatim}
%\documentclass{article}
%\usepackage{hyperref}
%\usepackage[figure]{hypcap}[2008/09/08]
%
%\begin{document}
%  \section{Hello World}
%  \begin{figure}
%    \caption{Figure with caption A}
%  \end{figure}
%  YcapstartfalseZ
%  \begin{figure}
%    Figure without caption
%  \end{figure}
%  YcapstarttrueZ
%  \begin{figure}
%    \caption{Figure with caption B}
%  \end{figure}
%\end{document}
%\end{verbatim}
%\end{quote}
%    \end{description}
%
% \subsection{Limitations}
%    \begin{itemize}
%    \item Packages that redefine \cmd{\caption} or
%          \cmd{\@caption}.
%    \end{itemize}
%
% \StopEventually{
% }
%
% \section{Implementation}
%    \begin{macrocode}
%<*package>
%    \end{macrocode}
%    Package identification.
%    \begin{macrocode}
\NeedsTeXFormat{LaTeX2e}
\ProvidesPackage{hypcap}%
  [2011/02/16 v1.11 Adjusting the anchors of captions (HO)]
%    \end{macrocode}
%
%    For unique command names this package uses |hc@| as prefix
%    for internal command names.
%
%    First we check, if package \Package{hyperref} is loaded:
%    \begin{macrocode}
\@ifundefined{hyper@@anchor}{%
  \PackageError{hypcap}{You have to load 'hyperref' first}\@ehc
  \endinput
}{}
%    \end{macrocode}
%
%    \begin{macrocode}
\RequirePackage{letltxmacro}[2008/06/24]
%    \end{macrocode}
%
%    \begin{macro}{\hc@org@caption}
%    Save the original meaning of \cmd{\caption}:
%    \begin{macrocode}
\newcommand*\hc@org@caption{}
\let\hc@org@caption\caption
%    \end{macrocode}
%    \end{macro}
%
%    \newcommand*{\BeginMacro}[1]{\expandafter\DoBeginMacro\csname#1\endcsname}
%    \newcommand*{\DoBeginMacro}[1]{\begin{macro}{#1}}
%    \newcommand*{\Cmd}[1]{\expandafter\cmd\csname#1\endcsname}
%    \BeginMacro{if@capstart}
%     The switch \Cmd{if@capstart} helps to detect
%     \cmd{\capstart} commands with missing \cmd{\caption} macros.
%     Because \cmd{\caption} can occur inside a group, assignments
%     to the switch have to be made global.
%    \begin{macrocode}
\newif\if@capstart
%    \end{macrocode}
%    \end{macro}
%
%    \begin{macro}{\hypcapspace}
%    The anchor is raised.by \cmd{\hypcapspace}.
%    \begin{macrocode}
\newcommand*\hypcapspace{.5\baselineskip}
%    \end{macrocode}
%    \end{macro}
%
%    \begin{macro}{\ifcapstart}
%    \begin{macrocode}
\newif\ifcapstart
\capstarttrue
%    \end{macrocode}
%    \end{macro}
%
%    \begin{macro}{\capstart}
%     The macro \cmd{\capstart} contains the first part of
%     the \cmd{\caption} command: Incrementing the counter
%     and setting the anchor.
%    \begin{macrocode}
\newcommand*\capstart{%
  \ifcapstart
    \H@refstepcounter\@captype % first part of caption
    \hyper@makecurrent\@captype
    \global\let\hc@currentHref\@currentHref
    \vspace*{-\hypcapspace}%
    \begingroup
      \let\leavevmode\relax
      \hyper@@anchor\@currentHref\relax
    \endgroup
    \vspace*{\hypcapspace}%
    \hc@hyperref{\let\caption\hc@caption}%
    \global\@capstarttrue
    \global\advance\csname c@\@captype\endcsname\m@ne
  \fi
}
%    \end{macrocode}
%    \end{macro}
%
%    \begin{macrocode}
\@ifpackagelater{hyperref}{2007/04/09}{%
  \let\hc@hyperref\@gobble
}{%
  \let\hc@hyperref\@firstofone
}
%    \end{macrocode}
%
%    \begin{macro}{\hc@caption}
%    The new \cmd{\caption} command without the first part
%    is defined in the macro \cmd{\hc@caption}.
%    \begin{macrocode}
\def\hc@caption{%
  \global\advance\csname c@\@captype\endcsname\@ne
  \@dblarg{\hc@@caption\@captype}%
}
%    \end{macrocode}
%    \end{macro}
%
%    \begin{macro}{\hc@@caption}
%    This is a copy of package \Package{hyperref}'s \cmd{\@caption}
%    macro without making the anchor, because this is already done
%    in \cmd{\capstart}.
%    \begin{macrocode}
\long\def\hc@@caption#1[#2]#3{%
  \let\caption\hc@org@caption
  \global\@capstartfalse
  \ifHy@hypertexnames
    \hyper@makecurrent\@captype
  \else
    \global\let\@currentHref\hc@currentHref
  \fi
  \par\addcontentsline{%
    \csname ext@#1\endcsname}{#1}{%
    \protect\numberline{%
      \csname the#1\endcsname
    }{\ignorespaces #2}%
  }%
  \begingroup
    \@parboxrestore
    \normalsize
    \@makecaption{\csname fnum@#1\endcsname}{%
      \ignorespaces#3%
    }%
    \par
  \endgroup
}
%    \end{macrocode}
%    \end{macro}
%
%    \begin{macro}{\hypcapredef}
%    The macro \cmd{\hypcapredef} prepares the call of
%    \cmd{\hc@redef} that will redefine the environment
%    that is given in the argument.
%    \begin{macrocode}
\def\hypcapredef#1{%
  \expandafter\hc@redef\csname hc@org#1\expandafter\endcsname
                       \csname hc@orgend#1\expandafter\endcsname
                       \expandafter{#1}%
}
%    \end{macrocode}
%    \end{macro}
%    \begin{macro}{\hc@redef}
%    The old meaning of the environment is saved. Then
%    \cmd{\capstart} is appended in the begin part.
%    The end part contains a check
%    that produces an error message in case of \cmd{\capstart}
%    without \cmd{\capstart} (\cmd{\capstart} has incremented
%    the counter).
%    \begin{macrocode}
\def\hc@redef#1#2#3{%
  \newcommand#1{}%
  \expandafter\LetLtxMacro\expandafter#1\csname#3\endcsname
  \expandafter\LetLtxMacro\expandafter#2\csname end#3\endcsname
  \renewenvironment*{#3}[1][]{%
    \ifx\\##1\\%
      #1\relax
    \else
      #1[##1]% hash-ok (compatibility for float)
    \fi
    \capstart
  }{%
    \if@capstart
      \PackageError{hypcap}{You have forgotten to use \string\caption}%
      \global\@capstartfalse
    \else
    \fi
    #2%
  }%
}
%    \end{macrocode}
%    \end{macro}
%
%     At last the options are defined and processed.
%    \begin{macrocode}
\DeclareOption{figure}{\hypcapredef{\CurrentOption}}
\DeclareOption{figure*}{\hypcapredef{\CurrentOption}}
\DeclareOption{table}{\hypcapredef{\CurrentOption}}
\DeclareOption{table*}{\hypcapredef{\CurrentOption}}
\DeclareOption{all}{%
  \hypcapredef{figure}%
  \hypcapredef{figure*}%
  \hypcapredef{table}%
  \hypcapredef{table*}%
}
\ProcessOptions\relax
%    \end{macrocode}
%
%    \begin{macrocode}
%</package>
%    \end{macrocode}
%
% \section{Installation}
%
% \subsection{Download}
%
% \paragraph{Package.} This package is available on
% CTAN\footnote{\url{ftp://ftp.ctan.org/tex-archive/}}:
% \begin{description}
% \item[\CTAN{macros/latex/contrib/oberdiek/hypcap.dtx}] The source file.
% \item[\CTAN{macros/latex/contrib/oberdiek/hypcap.pdf}] Documentation.
% \end{description}
%
%
% \paragraph{Bundle.} All the packages of the bundle `oberdiek'
% are also available in a TDS compliant ZIP archive. There
% the packages are already unpacked and the documentation files
% are generated. The files and directories obey the TDS standard.
% \begin{description}
% \item[\CTAN{install/macros/latex/contrib/oberdiek.tds.zip}]
% \end{description}
% \emph{TDS} refers to the standard ``A Directory Structure
% for \TeX\ Files'' (\CTAN{tds/tds.pdf}). Directories
% with \xfile{texmf} in their name are usually organized this way.
%
% \subsection{Bundle installation}
%
% \paragraph{Unpacking.} Unpack the \xfile{oberdiek.tds.zip} in the
% TDS tree (also known as \xfile{texmf} tree) of your choice.
% Example (linux):
% \begin{quote}
%   |unzip oberdiek.tds.zip -d ~/texmf|
% \end{quote}
%
% \paragraph{Script installation.}
% Check the directory \xfile{TDS:scripts/oberdiek/} for
% scripts that need further installation steps.
% Package \xpackage{attachfile2} comes with the Perl script
% \xfile{pdfatfi.pl} that should be installed in such a way
% that it can be called as \texttt{pdfatfi}.
% Example (linux):
% \begin{quote}
%   |chmod +x scripts/oberdiek/pdfatfi.pl|\\
%   |cp scripts/oberdiek/pdfatfi.pl /usr/local/bin/|
% \end{quote}
%
% \subsection{Package installation}
%
% \paragraph{Unpacking.} The \xfile{.dtx} file is a self-extracting
% \docstrip\ archive. The files are extracted by running the
% \xfile{.dtx} through \plainTeX:
% \begin{quote}
%   \verb|tex hypcap.dtx|
% \end{quote}
%
% \paragraph{TDS.} Now the different files must be moved into
% the different directories in your installation TDS tree
% (also known as \xfile{texmf} tree):
% \begin{quote}
% \def\t{^^A
% \begin{tabular}{@{}>{\ttfamily}l@{ $\rightarrow$ }>{\ttfamily}l@{}}
%   hypcap.sty & tex/latex/oberdiek/hypcap.sty\\
%   hypcap.pdf & doc/latex/oberdiek/hypcap.pdf\\
%   hypcap.dtx & source/latex/oberdiek/hypcap.dtx\\
% \end{tabular}^^A
% }^^A
% \sbox0{\t}^^A
% \ifdim\wd0>\linewidth
%   \begingroup
%     \advance\linewidth by\leftmargin
%     \advance\linewidth by\rightmargin
%   \edef\x{\endgroup
%     \def\noexpand\lw{\the\linewidth}^^A
%   }\x
%   \def\lwbox{^^A
%     \leavevmode
%     \hbox to \linewidth{^^A
%       \kern-\leftmargin\relax
%       \hss
%       \usebox0
%       \hss
%       \kern-\rightmargin\relax
%     }^^A
%   }^^A
%   \ifdim\wd0>\lw
%     \sbox0{\small\t}^^A
%     \ifdim\wd0>\linewidth
%       \ifdim\wd0>\lw
%         \sbox0{\footnotesize\t}^^A
%         \ifdim\wd0>\linewidth
%           \ifdim\wd0>\lw
%             \sbox0{\scriptsize\t}^^A
%             \ifdim\wd0>\linewidth
%               \ifdim\wd0>\lw
%                 \sbox0{\tiny\t}^^A
%                 \ifdim\wd0>\linewidth
%                   \lwbox
%                 \else
%                   \usebox0
%                 \fi
%               \else
%                 \lwbox
%               \fi
%             \else
%               \usebox0
%             \fi
%           \else
%             \lwbox
%           \fi
%         \else
%           \usebox0
%         \fi
%       \else
%         \lwbox
%       \fi
%     \else
%       \usebox0
%     \fi
%   \else
%     \lwbox
%   \fi
% \else
%   \usebox0
% \fi
% \end{quote}
% If you have a \xfile{docstrip.cfg} that configures and enables \docstrip's
% TDS installing feature, then some files can already be in the right
% place, see the documentation of \docstrip.
%
% \subsection{Refresh file name databases}
%
% If your \TeX~distribution
% (\teTeX, \mikTeX, \dots) relies on file name databases, you must refresh
% these. For example, \teTeX\ users run \verb|texhash| or
% \verb|mktexlsr|.
%
% \subsection{Some details for the interested}
%
% \paragraph{Attached source.}
%
% The PDF documentation on CTAN also includes the
% \xfile{.dtx} source file. It can be extracted by
% AcrobatReader 6 or higher. Another option is \textsf{pdftk},
% e.g. unpack the file into the current directory:
% \begin{quote}
%   \verb|pdftk hypcap.pdf unpack_files output .|
% \end{quote}
%
% \paragraph{Unpacking with \LaTeX.}
% The \xfile{.dtx} chooses its action depending on the format:
% \begin{description}
% \item[\plainTeX:] Run \docstrip\ and extract the files.
% \item[\LaTeX:] Generate the documentation.
% \end{description}
% If you insist on using \LaTeX\ for \docstrip\ (really,
% \docstrip\ does not need \LaTeX), then inform the autodetect routine
% about your intention:
% \begin{quote}
%   \verb|latex \let\install=y\input{hypcap.dtx}|
% \end{quote}
% Do not forget to quote the argument according to the demands
% of your shell.
%
% \paragraph{Generating the documentation.}
% You can use both the \xfile{.dtx} or the \xfile{.drv} to generate
% the documentation. The process can be configured by the
% configuration file \xfile{ltxdoc.cfg}. For instance, put this
% line into this file, if you want to have A4 as paper format:
% \begin{quote}
%   \verb|\PassOptionsToClass{a4paper}{article}|
% \end{quote}
% An example follows how to generate the
% documentation with pdf\LaTeX:
% \begin{quote}
%\begin{verbatim}
%pdflatex hypcap.dtx
%makeindex -s gind.ist hypcap.idx
%pdflatex hypcap.dtx
%makeindex -s gind.ist hypcap.idx
%pdflatex hypcap.dtx
%\end{verbatim}
% \end{quote}
%
% \section{Catalogue}
%
% The following XML file can be used as source for the
% \href{http://mirror.ctan.org/help/Catalogue/catalogue.html}{\TeX\ Catalogue}.
% The elements \texttt{caption} and \texttt{description} are imported
% from the original XML file from the Catalogue.
% The name of the XML file in the Catalogue is \xfile{hypcap.xml}.
%    \begin{macrocode}
%<*catalogue>
<?xml version='1.0' encoding='us-ascii'?>
<!DOCTYPE entry SYSTEM 'catalogue.dtd'>
<entry datestamp='$Date$' modifier='$Author$' id='hypcap'>
  <name>hypcap</name>
  <caption>Adjusting the anchors of captions.</caption>
  <authorref id='auth:oberdiek'/>
  <copyright owner='Heiko Oberdiek' year='1999-2001,2006-2008,2011'/>
  <license type='lppl1.3'/>
  <version number='1.11'/>
  <description>
    The package offers a solution to the problem that when you link to
    a float using <xref refid='hyperref'>hyperref</xref>, the link
    anchors to below the float's caption, rather than the beginning of
    the float.
    <p/>
    Hypcap defines a separate \capstart command, which you put where
    you want links to end; you should have a \capstart command for each
    \caption command.  Package options can be used to auto-insert a
    \capstart at the start of a float environment.
    <p/>
    The package is part of the <xref refid='oberdiek'>oberdiek</xref> bundle.
  </description>
  <documentation details='Package documentation'
      href='ctan:/macros/latex/contrib/oberdiek/hypcap.pdf'/>
  <ctan file='true' path='/macros/latex/contrib/oberdiek/hypcap.dtx'/>
  <miktex location='oberdiek'/>
  <texlive location='oberdiek'/>
  <install path='/macros/latex/contrib/oberdiek/oberdiek.tds.zip'/>
</entry>
%</catalogue>
%    \end{macrocode}
%
% \begin{History}
%   \begin{Version}{1999/02/13 v1.0}
%   \item
%     A beginning version, published in newsgroup \xnewsgroup{comp.text.tex}:\\
%     \URL{``\link{Re: \xpackage{hyperref} and figures}''}^^A
%     {http://groups.google.com/group/comp.text.tex/msg/5c9b47b001a9379c}
%   \end{Version}
%   \begin{Version}{2000/08/14 v1.1}
%   \item
%     Global assignments of \cs{if@capstart}
%     in order to allow \cs{caption} in groups.
%   \item
%     Option \texttt{all} added.
%   \end{Version}
%   \begin{Version}{2000/09/07 v1.2}
%   \item
%     Package in dtx format.
%   \end{Version}
%   \begin{Version}{2001/08/27 v1.3}
%   \item
%     Bug fix with hyperref's pdfmark driver\\
%     (\cs{leavevmode} in
%     \cs{hyper@@anchor}/\cs{pdf@rect}).
%   \end{Version}
%   \begin{Version}{2001/09/06 v1.4}
%   \item
%     Small fixes in the dtx file.
%   \end{Version}
%   \begin{Version}{2006/02/20 v1.5}
%   \item
%     Code is not changed.
%   \item
%     New DTX framework.
%   \end{Version}
%   \begin{Version}{2007/02/19 v1.6}
%   \item
%     Fix for hypertexnames=false.
%   \end{Version}
%   \begin{Version}{2007/04/09 v1.7}
%   \item
%     Stuff in \cs{caption} moved to \xpackage{hyperref}.
%     This avoids redefinitions of \cs{caption} and \cs{@caption}
%     (idea of Axel Sommerfeldt).
%   \item
%     Fix for subfigure (Marco Kuhlmann, Amilcar do Carmo Lucas).
%   \end{Version}
%   \begin{Version}{2008/04/14 v1.8}
%   \item
%     \cs{hc@redef} fixed to get package \xpackage{float} work
%     (Axel Sommerfeldt).
%   \end{Version}
%   \begin{Version}{2008/08/11 v1.9}
%   \item
%     Code is not changed.
%   \item
%     URLs updated.
%   \end{Version}
%   \begin{Version}{2008/09/08 v1.10}
%   \item
%     \cs{capstartfalse} and \cs{capstarttrue} added.
%   \end{Version}
%   \begin{Version}{2011/02/16 v1.11}
%   \item
%     \cs{hc@redef} fixed by using package \xpackage{letltxmacro}.
%   \end{Version}
% \end{History}
%
% \PrintIndex
%
% \Finale
\endinput

%        (quote the arguments according to the demands of your shell)
%
% Documentation:
%    (a) If hypcap.drv is present:
%           latex hypcap.drv
%    (b) Without hypcap.drv:
%           latex hypcap.dtx; ...
%    The class ltxdoc loads the configuration file ltxdoc.cfg
%    if available. Here you can specify further options, e.g.
%    use A4 as paper format:
%       \PassOptionsToClass{a4paper}{article}
%
%    Programm calls to get the documentation (example):
%       pdflatex hypcap.dtx
%       makeindex -s gind.ist hypcap.idx
%       pdflatex hypcap.dtx
%       makeindex -s gind.ist hypcap.idx
%       pdflatex hypcap.dtx
%
% Installation:
%    TDS:tex/latex/oberdiek/hypcap.sty
%    TDS:doc/latex/oberdiek/hypcap.pdf
%    TDS:source/latex/oberdiek/hypcap.dtx
%
%<*ignore>
\begingroup
  \catcode123=1 %
  \catcode125=2 %
  \def\x{LaTeX2e}%
\expandafter\endgroup
\ifcase 0\ifx\install y1\fi\expandafter
         \ifx\csname processbatchFile\endcsname\relax\else1\fi
         \ifx\fmtname\x\else 1\fi\relax
\else\csname fi\endcsname
%</ignore>
%<*install>
\input docstrip.tex
\Msg{************************************************************************}
\Msg{* Installation}
\Msg{* Package: hypcap 2011/02/16 v1.11 Adjusting the anchors of captions (HO)}
\Msg{************************************************************************}

\keepsilent
\askforoverwritefalse

\let\MetaPrefix\relax
\preamble

This is a generated file.

Project: hypcap
Version: 2011/02/16 v1.11

Copyright (C) 1999-2001, 2006-2008, 2011 by
   Heiko Oberdiek <heiko.oberdiek at googlemail.com>

This work may be distributed and/or modified under the
conditions of the LaTeX Project Public License, either
version 1.3c of this license or (at your option) any later
version. This version of this license is in
   http://www.latex-project.org/lppl/lppl-1-3c.txt
and the latest version of this license is in
   http://www.latex-project.org/lppl.txt
and version 1.3 or later is part of all distributions of
LaTeX version 2005/12/01 or later.

This work has the LPPL maintenance status "maintained".

This Current Maintainer of this work is Heiko Oberdiek.

This work consists of the main source file hypcap.dtx
and the derived files
   hypcap.sty, hypcap.pdf, hypcap.ins, hypcap.drv.

\endpreamble
\let\MetaPrefix\DoubleperCent

\generate{%
  \file{hypcap.ins}{\from{hypcap.dtx}{install}}%
  \file{hypcap.drv}{\from{hypcap.dtx}{driver}}%
  \usedir{tex/latex/oberdiek}%
  \file{hypcap.sty}{\from{hypcap.dtx}{package}}%
  \nopreamble
  \nopostamble
  \usedir{source/latex/oberdiek/catalogue}%
  \file{hypcap.xml}{\from{hypcap.dtx}{catalogue}}%
}

\catcode32=13\relax% active space
\let =\space%
\Msg{************************************************************************}
\Msg{*}
\Msg{* To finish the installation you have to move the following}
\Msg{* file into a directory searched by TeX:}
\Msg{*}
\Msg{*     hypcap.sty}
\Msg{*}
\Msg{* To produce the documentation run the file `hypcap.drv'}
\Msg{* through LaTeX.}
\Msg{*}
\Msg{* Happy TeXing!}
\Msg{*}
\Msg{************************************************************************}

\endbatchfile
%</install>
%<*ignore>
\fi
%</ignore>
%<*driver>
\NeedsTeXFormat{LaTeX2e}
\ProvidesFile{hypcap.drv}%
  [2011/02/16 v1.11 Adjusting the anchors of captions (HO)]%
\documentclass{ltxdoc}
\usepackage{holtxdoc}[2011/11/22]
\begin{document}
  \DocInput{hypcap.dtx}%
\end{document}
%</driver>
% \fi
%
% \CheckSum{174}
%
% \CharacterTable
%  {Upper-case    \A\B\C\D\E\F\G\H\I\J\K\L\M\N\O\P\Q\R\S\T\U\V\W\X\Y\Z
%   Lower-case    \a\b\c\d\e\f\g\h\i\j\k\l\m\n\o\p\q\r\s\t\u\v\w\x\y\z
%   Digits        \0\1\2\3\4\5\6\7\8\9
%   Exclamation   \!     Double quote  \"     Hash (number) \#
%   Dollar        \$     Percent       \%     Ampersand     \&
%   Acute accent  \'     Left paren    \(     Right paren   \)
%   Asterisk      \*     Plus          \+     Comma         \,
%   Minus         \-     Point         \.     Solidus       \/
%   Colon         \:     Semicolon     \;     Less than     \<
%   Equals        \=     Greater than  \>     Question mark \?
%   Commercial at \@     Left bracket  \[     Backslash     \\
%   Right bracket \]     Circumflex    \^     Underscore    \_
%   Grave accent  \`     Left brace    \{     Vertical bar  \|
%   Right brace   \}     Tilde         \~}
%
% \GetFileInfo{hypcap.drv}
%
% \title{The \xpackage{hypcap} package}
% \date{2011/02/16 v1.11}
% \author{Heiko Oberdiek\\\xemail{heiko.oberdiek at googlemail.com}}
%
% \maketitle
%
% \begin{abstract}
% This package tries a solution of the problem with
% hyperref, that links to floats points below the
% caption and not at the beginning of the float.
% Therefore this package divides the task into two
% part, the link setting with \cs{capstart} or
% automatically at the beginning of a float and
% the rest in the \cs{caption} command.
% \end{abstract}
%
% \tableofcontents
%
% \newpage
% \section{Usage}
%    The package \Package{hypcap} requires that \Package{hyperref}
%    is loaded first:
%    \begin{quote}
%      |\usepackage[...]{hyperref}|\\
%      |\usepackage[...]{hypcap}|
%    \end{quote}
%
% \subsection{Package options}
%    The names of the four float environments |figure|, |figure*|,
%    |table|, or |table*| can be used as option. Then the package
%    redefines the environment in order to insert \cmd{\capstart}
%    (see below) in the beginning of the environment automatically.
%
%    Option |all| enables the redefinitions of all four
%    float environments. For other environments see
%    the user command \cmd{\hypcapredef}.
%
% \subsection{User commands}
%    \begin{description}
%    \item[\cmd{\capstart}:]\DescribeMacro{\capstart}
%      First this command increments the counter (\cmd{\@captype}).
%      Then it makes an anchor for package \Package{hyperref}.
%      At last \cmd{\caption} is redefined to remove the
%      anchor setting part from \Package{hyperref}'s \cmd{\caption}.
%
%      The package expects the following structure of a float
%      environment:
%      \begin{quote}
%        |\begin{|\textit{|float|}|}...|\\
%        |\capstart|\\
%        |...|\\
%        |\caption{...}|\\
%        |...|\\
%        |\end{|\textit{|float|}|}|
%      \end{quote}
%      There can be several \cmd{\caption} commands. For these
%      you need \cmd{\capstart} again:
%      \begin{quote}
%        |\capstart ... \caption... \capstart ... \caption...|
%      \end{quote}
%      And the \cmd{\caption} command itself can be put in a group.
%
%      With the options, described above, the extra
%      writing of \cmd{\capstart} can be avoided.
%      Consequently, there must be a \cmd{\caption} in every
%      environment of this type, specified by the option.
%      If you want to use more than one \cmd{\caption} in this
%      environment, you have to state \cmd{\capstart} again.
%    \item[\cmd{\hypcapspace}:]\DescribeMacro{\hypcapspace}
%      Because it looks poor, if the link points exactly at top of
%      the figure, there is additional space: \cmd{\hypcapspace},
%      the default is |0.5\baselineskip|, examples:
%      \begin{quote}
%        |\renewcommand{\hypcapspace}{0pt}| removes the space\\
%        |\renewcommand{\hypcapspace}{1pt}| sets a fix value
%      \end{quote}
%    \item[\cmd{\hypcapredef}:]\DescribeMacro{\hypcapredef}
%      If there are other float environments, that
%      should automatically execute \cmd{\capstart}, then
%      a redefinition with \cmd{\hypcapredef} can be tried:
%      \begin{quote}
%        |\hypcapredef{myfloat}|
%      \end{quote}
%      Only environments with one optional parameter are
%      supported.
%    \item[\cs{capstartfalse}, \cs{capstarttrue}:]^^A
%      \DescribeMacro{\capstartfalse}^^A
%      \DescribeMacro{\capstarttrue}^^A
%      Since 2008/09/08 v1.10.\\
%      They disable and enable \cs{capstart}. They can be used to
%      cancel the effect of a redefined float environment.
%      Example:
%\begin{quote}
%\catcode`Y=\active
%\def Y#1Z{\textit{\cs{#1}}}
%\begin{verbatim}
%\documentclass{article}
%\usepackage{hyperref}
%\usepackage[figure]{hypcap}[2008/09/08]
%
%\begin{document}
%  \section{Hello World}
%  \begin{figure}
%    \caption{Figure with caption A}
%  \end{figure}
%  YcapstartfalseZ
%  \begin{figure}
%    Figure without caption
%  \end{figure}
%  YcapstarttrueZ
%  \begin{figure}
%    \caption{Figure with caption B}
%  \end{figure}
%\end{document}
%\end{verbatim}
%\end{quote}
%    \end{description}
%
% \subsection{Limitations}
%    \begin{itemize}
%    \item Packages that redefine \cmd{\caption} or
%          \cmd{\@caption}.
%    \end{itemize}
%
% \StopEventually{
% }
%
% \section{Implementation}
%    \begin{macrocode}
%<*package>
%    \end{macrocode}
%    Package identification.
%    \begin{macrocode}
\NeedsTeXFormat{LaTeX2e}
\ProvidesPackage{hypcap}%
  [2011/02/16 v1.11 Adjusting the anchors of captions (HO)]
%    \end{macrocode}
%
%    For unique command names this package uses |hc@| as prefix
%    for internal command names.
%
%    First we check, if package \Package{hyperref} is loaded:
%    \begin{macrocode}
\@ifundefined{hyper@@anchor}{%
  \PackageError{hypcap}{You have to load 'hyperref' first}\@ehc
  \endinput
}{}
%    \end{macrocode}
%
%    \begin{macrocode}
\RequirePackage{letltxmacro}[2008/06/24]
%    \end{macrocode}
%
%    \begin{macro}{\hc@org@caption}
%    Save the original meaning of \cmd{\caption}:
%    \begin{macrocode}
\newcommand*\hc@org@caption{}
\let\hc@org@caption\caption
%    \end{macrocode}
%    \end{macro}
%
%    \newcommand*{\BeginMacro}[1]{\expandafter\DoBeginMacro\csname#1\endcsname}
%    \newcommand*{\DoBeginMacro}[1]{\begin{macro}{#1}}
%    \newcommand*{\Cmd}[1]{\expandafter\cmd\csname#1\endcsname}
%    \BeginMacro{if@capstart}
%     The switch \Cmd{if@capstart} helps to detect
%     \cmd{\capstart} commands with missing \cmd{\caption} macros.
%     Because \cmd{\caption} can occur inside a group, assignments
%     to the switch have to be made global.
%    \begin{macrocode}
\newif\if@capstart
%    \end{macrocode}
%    \end{macro}
%
%    \begin{macro}{\hypcapspace}
%    The anchor is raised.by \cmd{\hypcapspace}.
%    \begin{macrocode}
\newcommand*\hypcapspace{.5\baselineskip}
%    \end{macrocode}
%    \end{macro}
%
%    \begin{macro}{\ifcapstart}
%    \begin{macrocode}
\newif\ifcapstart
\capstarttrue
%    \end{macrocode}
%    \end{macro}
%
%    \begin{macro}{\capstart}
%     The macro \cmd{\capstart} contains the first part of
%     the \cmd{\caption} command: Incrementing the counter
%     and setting the anchor.
%    \begin{macrocode}
\newcommand*\capstart{%
  \ifcapstart
    \H@refstepcounter\@captype % first part of caption
    \hyper@makecurrent\@captype
    \global\let\hc@currentHref\@currentHref
    \vspace*{-\hypcapspace}%
    \begingroup
      \let\leavevmode\relax
      \hyper@@anchor\@currentHref\relax
    \endgroup
    \vspace*{\hypcapspace}%
    \hc@hyperref{\let\caption\hc@caption}%
    \global\@capstarttrue
    \global\advance\csname c@\@captype\endcsname\m@ne
  \fi
}
%    \end{macrocode}
%    \end{macro}
%
%    \begin{macrocode}
\@ifpackagelater{hyperref}{2007/04/09}{%
  \let\hc@hyperref\@gobble
}{%
  \let\hc@hyperref\@firstofone
}
%    \end{macrocode}
%
%    \begin{macro}{\hc@caption}
%    The new \cmd{\caption} command without the first part
%    is defined in the macro \cmd{\hc@caption}.
%    \begin{macrocode}
\def\hc@caption{%
  \global\advance\csname c@\@captype\endcsname\@ne
  \@dblarg{\hc@@caption\@captype}%
}
%    \end{macrocode}
%    \end{macro}
%
%    \begin{macro}{\hc@@caption}
%    This is a copy of package \Package{hyperref}'s \cmd{\@caption}
%    macro without making the anchor, because this is already done
%    in \cmd{\capstart}.
%    \begin{macrocode}
\long\def\hc@@caption#1[#2]#3{%
  \let\caption\hc@org@caption
  \global\@capstartfalse
  \ifHy@hypertexnames
    \hyper@makecurrent\@captype
  \else
    \global\let\@currentHref\hc@currentHref
  \fi
  \par\addcontentsline{%
    \csname ext@#1\endcsname}{#1}{%
    \protect\numberline{%
      \csname the#1\endcsname
    }{\ignorespaces #2}%
  }%
  \begingroup
    \@parboxrestore
    \normalsize
    \@makecaption{\csname fnum@#1\endcsname}{%
      \ignorespaces#3%
    }%
    \par
  \endgroup
}
%    \end{macrocode}
%    \end{macro}
%
%    \begin{macro}{\hypcapredef}
%    The macro \cmd{\hypcapredef} prepares the call of
%    \cmd{\hc@redef} that will redefine the environment
%    that is given in the argument.
%    \begin{macrocode}
\def\hypcapredef#1{%
  \expandafter\hc@redef\csname hc@org#1\expandafter\endcsname
                       \csname hc@orgend#1\expandafter\endcsname
                       \expandafter{#1}%
}
%    \end{macrocode}
%    \end{macro}
%    \begin{macro}{\hc@redef}
%    The old meaning of the environment is saved. Then
%    \cmd{\capstart} is appended in the begin part.
%    The end part contains a check
%    that produces an error message in case of \cmd{\capstart}
%    without \cmd{\capstart} (\cmd{\capstart} has incremented
%    the counter).
%    \begin{macrocode}
\def\hc@redef#1#2#3{%
  \newcommand#1{}%
  \expandafter\LetLtxMacro\expandafter#1\csname#3\endcsname
  \expandafter\LetLtxMacro\expandafter#2\csname end#3\endcsname
  \renewenvironment*{#3}[1][]{%
    \ifx\\##1\\%
      #1\relax
    \else
      #1[##1]% hash-ok (compatibility for float)
    \fi
    \capstart
  }{%
    \if@capstart
      \PackageError{hypcap}{You have forgotten to use \string\caption}%
      \global\@capstartfalse
    \else
    \fi
    #2%
  }%
}
%    \end{macrocode}
%    \end{macro}
%
%     At last the options are defined and processed.
%    \begin{macrocode}
\DeclareOption{figure}{\hypcapredef{\CurrentOption}}
\DeclareOption{figure*}{\hypcapredef{\CurrentOption}}
\DeclareOption{table}{\hypcapredef{\CurrentOption}}
\DeclareOption{table*}{\hypcapredef{\CurrentOption}}
\DeclareOption{all}{%
  \hypcapredef{figure}%
  \hypcapredef{figure*}%
  \hypcapredef{table}%
  \hypcapredef{table*}%
}
\ProcessOptions\relax
%    \end{macrocode}
%
%    \begin{macrocode}
%</package>
%    \end{macrocode}
%
% \section{Installation}
%
% \subsection{Download}
%
% \paragraph{Package.} This package is available on
% CTAN\footnote{\url{ftp://ftp.ctan.org/tex-archive/}}:
% \begin{description}
% \item[\CTAN{macros/latex/contrib/oberdiek/hypcap.dtx}] The source file.
% \item[\CTAN{macros/latex/contrib/oberdiek/hypcap.pdf}] Documentation.
% \end{description}
%
%
% \paragraph{Bundle.} All the packages of the bundle `oberdiek'
% are also available in a TDS compliant ZIP archive. There
% the packages are already unpacked and the documentation files
% are generated. The files and directories obey the TDS standard.
% \begin{description}
% \item[\CTAN{install/macros/latex/contrib/oberdiek.tds.zip}]
% \end{description}
% \emph{TDS} refers to the standard ``A Directory Structure
% for \TeX\ Files'' (\CTAN{tds/tds.pdf}). Directories
% with \xfile{texmf} in their name are usually organized this way.
%
% \subsection{Bundle installation}
%
% \paragraph{Unpacking.} Unpack the \xfile{oberdiek.tds.zip} in the
% TDS tree (also known as \xfile{texmf} tree) of your choice.
% Example (linux):
% \begin{quote}
%   |unzip oberdiek.tds.zip -d ~/texmf|
% \end{quote}
%
% \paragraph{Script installation.}
% Check the directory \xfile{TDS:scripts/oberdiek/} for
% scripts that need further installation steps.
% Package \xpackage{attachfile2} comes with the Perl script
% \xfile{pdfatfi.pl} that should be installed in such a way
% that it can be called as \texttt{pdfatfi}.
% Example (linux):
% \begin{quote}
%   |chmod +x scripts/oberdiek/pdfatfi.pl|\\
%   |cp scripts/oberdiek/pdfatfi.pl /usr/local/bin/|
% \end{quote}
%
% \subsection{Package installation}
%
% \paragraph{Unpacking.} The \xfile{.dtx} file is a self-extracting
% \docstrip\ archive. The files are extracted by running the
% \xfile{.dtx} through \plainTeX:
% \begin{quote}
%   \verb|tex hypcap.dtx|
% \end{quote}
%
% \paragraph{TDS.} Now the different files must be moved into
% the different directories in your installation TDS tree
% (also known as \xfile{texmf} tree):
% \begin{quote}
% \def\t{^^A
% \begin{tabular}{@{}>{\ttfamily}l@{ $\rightarrow$ }>{\ttfamily}l@{}}
%   hypcap.sty & tex/latex/oberdiek/hypcap.sty\\
%   hypcap.pdf & doc/latex/oberdiek/hypcap.pdf\\
%   hypcap.dtx & source/latex/oberdiek/hypcap.dtx\\
% \end{tabular}^^A
% }^^A
% \sbox0{\t}^^A
% \ifdim\wd0>\linewidth
%   \begingroup
%     \advance\linewidth by\leftmargin
%     \advance\linewidth by\rightmargin
%   \edef\x{\endgroup
%     \def\noexpand\lw{\the\linewidth}^^A
%   }\x
%   \def\lwbox{^^A
%     \leavevmode
%     \hbox to \linewidth{^^A
%       \kern-\leftmargin\relax
%       \hss
%       \usebox0
%       \hss
%       \kern-\rightmargin\relax
%     }^^A
%   }^^A
%   \ifdim\wd0>\lw
%     \sbox0{\small\t}^^A
%     \ifdim\wd0>\linewidth
%       \ifdim\wd0>\lw
%         \sbox0{\footnotesize\t}^^A
%         \ifdim\wd0>\linewidth
%           \ifdim\wd0>\lw
%             \sbox0{\scriptsize\t}^^A
%             \ifdim\wd0>\linewidth
%               \ifdim\wd0>\lw
%                 \sbox0{\tiny\t}^^A
%                 \ifdim\wd0>\linewidth
%                   \lwbox
%                 \else
%                   \usebox0
%                 \fi
%               \else
%                 \lwbox
%               \fi
%             \else
%               \usebox0
%             \fi
%           \else
%             \lwbox
%           \fi
%         \else
%           \usebox0
%         \fi
%       \else
%         \lwbox
%       \fi
%     \else
%       \usebox0
%     \fi
%   \else
%     \lwbox
%   \fi
% \else
%   \usebox0
% \fi
% \end{quote}
% If you have a \xfile{docstrip.cfg} that configures and enables \docstrip's
% TDS installing feature, then some files can already be in the right
% place, see the documentation of \docstrip.
%
% \subsection{Refresh file name databases}
%
% If your \TeX~distribution
% (\teTeX, \mikTeX, \dots) relies on file name databases, you must refresh
% these. For example, \teTeX\ users run \verb|texhash| or
% \verb|mktexlsr|.
%
% \subsection{Some details for the interested}
%
% \paragraph{Attached source.}
%
% The PDF documentation on CTAN also includes the
% \xfile{.dtx} source file. It can be extracted by
% AcrobatReader 6 or higher. Another option is \textsf{pdftk},
% e.g. unpack the file into the current directory:
% \begin{quote}
%   \verb|pdftk hypcap.pdf unpack_files output .|
% \end{quote}
%
% \paragraph{Unpacking with \LaTeX.}
% The \xfile{.dtx} chooses its action depending on the format:
% \begin{description}
% \item[\plainTeX:] Run \docstrip\ and extract the files.
% \item[\LaTeX:] Generate the documentation.
% \end{description}
% If you insist on using \LaTeX\ for \docstrip\ (really,
% \docstrip\ does not need \LaTeX), then inform the autodetect routine
% about your intention:
% \begin{quote}
%   \verb|latex \let\install=y% \iffalse meta-comment
%
% File: hypcap.dtx
% Version: 2011/02/16 v1.11
% Info: Adjusting the anchors of captions
%
% Copyright (C) 1999-2001, 2006-2008, 2011 by
%    Heiko Oberdiek <heiko.oberdiek at googlemail.com>
%
% This work may be distributed and/or modified under the
% conditions of the LaTeX Project Public License, either
% version 1.3c of this license or (at your option) any later
% version. This version of this license is in
%    http://www.latex-project.org/lppl/lppl-1-3c.txt
% and the latest version of this license is in
%    http://www.latex-project.org/lppl.txt
% and version 1.3 or later is part of all distributions of
% LaTeX version 2005/12/01 or later.
%
% This work has the LPPL maintenance status "maintained".
%
% This Current Maintainer of this work is Heiko Oberdiek.
%
% This work consists of the main source file hypcap.dtx
% and the derived files
%    hypcap.sty, hypcap.pdf, hypcap.ins, hypcap.drv.
%
% Distribution:
%    CTAN:macros/latex/contrib/oberdiek/hypcap.dtx
%    CTAN:macros/latex/contrib/oberdiek/hypcap.pdf
%
% Unpacking:
%    (a) If hypcap.ins is present:
%           tex hypcap.ins
%    (b) Without hypcap.ins:
%           tex hypcap.dtx
%    (c) If you insist on using LaTeX
%           latex \let\install=y\input{hypcap.dtx}
%        (quote the arguments according to the demands of your shell)
%
% Documentation:
%    (a) If hypcap.drv is present:
%           latex hypcap.drv
%    (b) Without hypcap.drv:
%           latex hypcap.dtx; ...
%    The class ltxdoc loads the configuration file ltxdoc.cfg
%    if available. Here you can specify further options, e.g.
%    use A4 as paper format:
%       \PassOptionsToClass{a4paper}{article}
%
%    Programm calls to get the documentation (example):
%       pdflatex hypcap.dtx
%       makeindex -s gind.ist hypcap.idx
%       pdflatex hypcap.dtx
%       makeindex -s gind.ist hypcap.idx
%       pdflatex hypcap.dtx
%
% Installation:
%    TDS:tex/latex/oberdiek/hypcap.sty
%    TDS:doc/latex/oberdiek/hypcap.pdf
%    TDS:source/latex/oberdiek/hypcap.dtx
%
%<*ignore>
\begingroup
  \catcode123=1 %
  \catcode125=2 %
  \def\x{LaTeX2e}%
\expandafter\endgroup
\ifcase 0\ifx\install y1\fi\expandafter
         \ifx\csname processbatchFile\endcsname\relax\else1\fi
         \ifx\fmtname\x\else 1\fi\relax
\else\csname fi\endcsname
%</ignore>
%<*install>
\input docstrip.tex
\Msg{************************************************************************}
\Msg{* Installation}
\Msg{* Package: hypcap 2011/02/16 v1.11 Adjusting the anchors of captions (HO)}
\Msg{************************************************************************}

\keepsilent
\askforoverwritefalse

\let\MetaPrefix\relax
\preamble

This is a generated file.

Project: hypcap
Version: 2011/02/16 v1.11

Copyright (C) 1999-2001, 2006-2008, 2011 by
   Heiko Oberdiek <heiko.oberdiek at googlemail.com>

This work may be distributed and/or modified under the
conditions of the LaTeX Project Public License, either
version 1.3c of this license or (at your option) any later
version. This version of this license is in
   http://www.latex-project.org/lppl/lppl-1-3c.txt
and the latest version of this license is in
   http://www.latex-project.org/lppl.txt
and version 1.3 or later is part of all distributions of
LaTeX version 2005/12/01 or later.

This work has the LPPL maintenance status "maintained".

This Current Maintainer of this work is Heiko Oberdiek.

This work consists of the main source file hypcap.dtx
and the derived files
   hypcap.sty, hypcap.pdf, hypcap.ins, hypcap.drv.

\endpreamble
\let\MetaPrefix\DoubleperCent

\generate{%
  \file{hypcap.ins}{\from{hypcap.dtx}{install}}%
  \file{hypcap.drv}{\from{hypcap.dtx}{driver}}%
  \usedir{tex/latex/oberdiek}%
  \file{hypcap.sty}{\from{hypcap.dtx}{package}}%
  \nopreamble
  \nopostamble
  \usedir{source/latex/oberdiek/catalogue}%
  \file{hypcap.xml}{\from{hypcap.dtx}{catalogue}}%
}

\catcode32=13\relax% active space
\let =\space%
\Msg{************************************************************************}
\Msg{*}
\Msg{* To finish the installation you have to move the following}
\Msg{* file into a directory searched by TeX:}
\Msg{*}
\Msg{*     hypcap.sty}
\Msg{*}
\Msg{* To produce the documentation run the file `hypcap.drv'}
\Msg{* through LaTeX.}
\Msg{*}
\Msg{* Happy TeXing!}
\Msg{*}
\Msg{************************************************************************}

\endbatchfile
%</install>
%<*ignore>
\fi
%</ignore>
%<*driver>
\NeedsTeXFormat{LaTeX2e}
\ProvidesFile{hypcap.drv}%
  [2011/02/16 v1.11 Adjusting the anchors of captions (HO)]%
\documentclass{ltxdoc}
\usepackage{holtxdoc}[2011/11/22]
\begin{document}
  \DocInput{hypcap.dtx}%
\end{document}
%</driver>
% \fi
%
% \CheckSum{174}
%
% \CharacterTable
%  {Upper-case    \A\B\C\D\E\F\G\H\I\J\K\L\M\N\O\P\Q\R\S\T\U\V\W\X\Y\Z
%   Lower-case    \a\b\c\d\e\f\g\h\i\j\k\l\m\n\o\p\q\r\s\t\u\v\w\x\y\z
%   Digits        \0\1\2\3\4\5\6\7\8\9
%   Exclamation   \!     Double quote  \"     Hash (number) \#
%   Dollar        \$     Percent       \%     Ampersand     \&
%   Acute accent  \'     Left paren    \(     Right paren   \)
%   Asterisk      \*     Plus          \+     Comma         \,
%   Minus         \-     Point         \.     Solidus       \/
%   Colon         \:     Semicolon     \;     Less than     \<
%   Equals        \=     Greater than  \>     Question mark \?
%   Commercial at \@     Left bracket  \[     Backslash     \\
%   Right bracket \]     Circumflex    \^     Underscore    \_
%   Grave accent  \`     Left brace    \{     Vertical bar  \|
%   Right brace   \}     Tilde         \~}
%
% \GetFileInfo{hypcap.drv}
%
% \title{The \xpackage{hypcap} package}
% \date{2011/02/16 v1.11}
% \author{Heiko Oberdiek\\\xemail{heiko.oberdiek at googlemail.com}}
%
% \maketitle
%
% \begin{abstract}
% This package tries a solution of the problem with
% hyperref, that links to floats points below the
% caption and not at the beginning of the float.
% Therefore this package divides the task into two
% part, the link setting with \cs{capstart} or
% automatically at the beginning of a float and
% the rest in the \cs{caption} command.
% \end{abstract}
%
% \tableofcontents
%
% \newpage
% \section{Usage}
%    The package \Package{hypcap} requires that \Package{hyperref}
%    is loaded first:
%    \begin{quote}
%      |\usepackage[...]{hyperref}|\\
%      |\usepackage[...]{hypcap}|
%    \end{quote}
%
% \subsection{Package options}
%    The names of the four float environments |figure|, |figure*|,
%    |table|, or |table*| can be used as option. Then the package
%    redefines the environment in order to insert \cmd{\capstart}
%    (see below) in the beginning of the environment automatically.
%
%    Option |all| enables the redefinitions of all four
%    float environments. For other environments see
%    the user command \cmd{\hypcapredef}.
%
% \subsection{User commands}
%    \begin{description}
%    \item[\cmd{\capstart}:]\DescribeMacro{\capstart}
%      First this command increments the counter (\cmd{\@captype}).
%      Then it makes an anchor for package \Package{hyperref}.
%      At last \cmd{\caption} is redefined to remove the
%      anchor setting part from \Package{hyperref}'s \cmd{\caption}.
%
%      The package expects the following structure of a float
%      environment:
%      \begin{quote}
%        |\begin{|\textit{|float|}|}...|\\
%        |\capstart|\\
%        |...|\\
%        |\caption{...}|\\
%        |...|\\
%        |\end{|\textit{|float|}|}|
%      \end{quote}
%      There can be several \cmd{\caption} commands. For these
%      you need \cmd{\capstart} again:
%      \begin{quote}
%        |\capstart ... \caption... \capstart ... \caption...|
%      \end{quote}
%      And the \cmd{\caption} command itself can be put in a group.
%
%      With the options, described above, the extra
%      writing of \cmd{\capstart} can be avoided.
%      Consequently, there must be a \cmd{\caption} in every
%      environment of this type, specified by the option.
%      If you want to use more than one \cmd{\caption} in this
%      environment, you have to state \cmd{\capstart} again.
%    \item[\cmd{\hypcapspace}:]\DescribeMacro{\hypcapspace}
%      Because it looks poor, if the link points exactly at top of
%      the figure, there is additional space: \cmd{\hypcapspace},
%      the default is |0.5\baselineskip|, examples:
%      \begin{quote}
%        |\renewcommand{\hypcapspace}{0pt}| removes the space\\
%        |\renewcommand{\hypcapspace}{1pt}| sets a fix value
%      \end{quote}
%    \item[\cmd{\hypcapredef}:]\DescribeMacro{\hypcapredef}
%      If there are other float environments, that
%      should automatically execute \cmd{\capstart}, then
%      a redefinition with \cmd{\hypcapredef} can be tried:
%      \begin{quote}
%        |\hypcapredef{myfloat}|
%      \end{quote}
%      Only environments with one optional parameter are
%      supported.
%    \item[\cs{capstartfalse}, \cs{capstarttrue}:]^^A
%      \DescribeMacro{\capstartfalse}^^A
%      \DescribeMacro{\capstarttrue}^^A
%      Since 2008/09/08 v1.10.\\
%      They disable and enable \cs{capstart}. They can be used to
%      cancel the effect of a redefined float environment.
%      Example:
%\begin{quote}
%\catcode`Y=\active
%\def Y#1Z{\textit{\cs{#1}}}
%\begin{verbatim}
%\documentclass{article}
%\usepackage{hyperref}
%\usepackage[figure]{hypcap}[2008/09/08]
%
%\begin{document}
%  \section{Hello World}
%  \begin{figure}
%    \caption{Figure with caption A}
%  \end{figure}
%  YcapstartfalseZ
%  \begin{figure}
%    Figure without caption
%  \end{figure}
%  YcapstarttrueZ
%  \begin{figure}
%    \caption{Figure with caption B}
%  \end{figure}
%\end{document}
%\end{verbatim}
%\end{quote}
%    \end{description}
%
% \subsection{Limitations}
%    \begin{itemize}
%    \item Packages that redefine \cmd{\caption} or
%          \cmd{\@caption}.
%    \end{itemize}
%
% \StopEventually{
% }
%
% \section{Implementation}
%    \begin{macrocode}
%<*package>
%    \end{macrocode}
%    Package identification.
%    \begin{macrocode}
\NeedsTeXFormat{LaTeX2e}
\ProvidesPackage{hypcap}%
  [2011/02/16 v1.11 Adjusting the anchors of captions (HO)]
%    \end{macrocode}
%
%    For unique command names this package uses |hc@| as prefix
%    for internal command names.
%
%    First we check, if package \Package{hyperref} is loaded:
%    \begin{macrocode}
\@ifundefined{hyper@@anchor}{%
  \PackageError{hypcap}{You have to load 'hyperref' first}\@ehc
  \endinput
}{}
%    \end{macrocode}
%
%    \begin{macrocode}
\RequirePackage{letltxmacro}[2008/06/24]
%    \end{macrocode}
%
%    \begin{macro}{\hc@org@caption}
%    Save the original meaning of \cmd{\caption}:
%    \begin{macrocode}
\newcommand*\hc@org@caption{}
\let\hc@org@caption\caption
%    \end{macrocode}
%    \end{macro}
%
%    \newcommand*{\BeginMacro}[1]{\expandafter\DoBeginMacro\csname#1\endcsname}
%    \newcommand*{\DoBeginMacro}[1]{\begin{macro}{#1}}
%    \newcommand*{\Cmd}[1]{\expandafter\cmd\csname#1\endcsname}
%    \BeginMacro{if@capstart}
%     The switch \Cmd{if@capstart} helps to detect
%     \cmd{\capstart} commands with missing \cmd{\caption} macros.
%     Because \cmd{\caption} can occur inside a group, assignments
%     to the switch have to be made global.
%    \begin{macrocode}
\newif\if@capstart
%    \end{macrocode}
%    \end{macro}
%
%    \begin{macro}{\hypcapspace}
%    The anchor is raised.by \cmd{\hypcapspace}.
%    \begin{macrocode}
\newcommand*\hypcapspace{.5\baselineskip}
%    \end{macrocode}
%    \end{macro}
%
%    \begin{macro}{\ifcapstart}
%    \begin{macrocode}
\newif\ifcapstart
\capstarttrue
%    \end{macrocode}
%    \end{macro}
%
%    \begin{macro}{\capstart}
%     The macro \cmd{\capstart} contains the first part of
%     the \cmd{\caption} command: Incrementing the counter
%     and setting the anchor.
%    \begin{macrocode}
\newcommand*\capstart{%
  \ifcapstart
    \H@refstepcounter\@captype % first part of caption
    \hyper@makecurrent\@captype
    \global\let\hc@currentHref\@currentHref
    \vspace*{-\hypcapspace}%
    \begingroup
      \let\leavevmode\relax
      \hyper@@anchor\@currentHref\relax
    \endgroup
    \vspace*{\hypcapspace}%
    \hc@hyperref{\let\caption\hc@caption}%
    \global\@capstarttrue
    \global\advance\csname c@\@captype\endcsname\m@ne
  \fi
}
%    \end{macrocode}
%    \end{macro}
%
%    \begin{macrocode}
\@ifpackagelater{hyperref}{2007/04/09}{%
  \let\hc@hyperref\@gobble
}{%
  \let\hc@hyperref\@firstofone
}
%    \end{macrocode}
%
%    \begin{macro}{\hc@caption}
%    The new \cmd{\caption} command without the first part
%    is defined in the macro \cmd{\hc@caption}.
%    \begin{macrocode}
\def\hc@caption{%
  \global\advance\csname c@\@captype\endcsname\@ne
  \@dblarg{\hc@@caption\@captype}%
}
%    \end{macrocode}
%    \end{macro}
%
%    \begin{macro}{\hc@@caption}
%    This is a copy of package \Package{hyperref}'s \cmd{\@caption}
%    macro without making the anchor, because this is already done
%    in \cmd{\capstart}.
%    \begin{macrocode}
\long\def\hc@@caption#1[#2]#3{%
  \let\caption\hc@org@caption
  \global\@capstartfalse
  \ifHy@hypertexnames
    \hyper@makecurrent\@captype
  \else
    \global\let\@currentHref\hc@currentHref
  \fi
  \par\addcontentsline{%
    \csname ext@#1\endcsname}{#1}{%
    \protect\numberline{%
      \csname the#1\endcsname
    }{\ignorespaces #2}%
  }%
  \begingroup
    \@parboxrestore
    \normalsize
    \@makecaption{\csname fnum@#1\endcsname}{%
      \ignorespaces#3%
    }%
    \par
  \endgroup
}
%    \end{macrocode}
%    \end{macro}
%
%    \begin{macro}{\hypcapredef}
%    The macro \cmd{\hypcapredef} prepares the call of
%    \cmd{\hc@redef} that will redefine the environment
%    that is given in the argument.
%    \begin{macrocode}
\def\hypcapredef#1{%
  \expandafter\hc@redef\csname hc@org#1\expandafter\endcsname
                       \csname hc@orgend#1\expandafter\endcsname
                       \expandafter{#1}%
}
%    \end{macrocode}
%    \end{macro}
%    \begin{macro}{\hc@redef}
%    The old meaning of the environment is saved. Then
%    \cmd{\capstart} is appended in the begin part.
%    The end part contains a check
%    that produces an error message in case of \cmd{\capstart}
%    without \cmd{\capstart} (\cmd{\capstart} has incremented
%    the counter).
%    \begin{macrocode}
\def\hc@redef#1#2#3{%
  \newcommand#1{}%
  \expandafter\LetLtxMacro\expandafter#1\csname#3\endcsname
  \expandafter\LetLtxMacro\expandafter#2\csname end#3\endcsname
  \renewenvironment*{#3}[1][]{%
    \ifx\\##1\\%
      #1\relax
    \else
      #1[##1]% hash-ok (compatibility for float)
    \fi
    \capstart
  }{%
    \if@capstart
      \PackageError{hypcap}{You have forgotten to use \string\caption}%
      \global\@capstartfalse
    \else
    \fi
    #2%
  }%
}
%    \end{macrocode}
%    \end{macro}
%
%     At last the options are defined and processed.
%    \begin{macrocode}
\DeclareOption{figure}{\hypcapredef{\CurrentOption}}
\DeclareOption{figure*}{\hypcapredef{\CurrentOption}}
\DeclareOption{table}{\hypcapredef{\CurrentOption}}
\DeclareOption{table*}{\hypcapredef{\CurrentOption}}
\DeclareOption{all}{%
  \hypcapredef{figure}%
  \hypcapredef{figure*}%
  \hypcapredef{table}%
  \hypcapredef{table*}%
}
\ProcessOptions\relax
%    \end{macrocode}
%
%    \begin{macrocode}
%</package>
%    \end{macrocode}
%
% \section{Installation}
%
% \subsection{Download}
%
% \paragraph{Package.} This package is available on
% CTAN\footnote{\url{ftp://ftp.ctan.org/tex-archive/}}:
% \begin{description}
% \item[\CTAN{macros/latex/contrib/oberdiek/hypcap.dtx}] The source file.
% \item[\CTAN{macros/latex/contrib/oberdiek/hypcap.pdf}] Documentation.
% \end{description}
%
%
% \paragraph{Bundle.} All the packages of the bundle `oberdiek'
% are also available in a TDS compliant ZIP archive. There
% the packages are already unpacked and the documentation files
% are generated. The files and directories obey the TDS standard.
% \begin{description}
% \item[\CTAN{install/macros/latex/contrib/oberdiek.tds.zip}]
% \end{description}
% \emph{TDS} refers to the standard ``A Directory Structure
% for \TeX\ Files'' (\CTAN{tds/tds.pdf}). Directories
% with \xfile{texmf} in their name are usually organized this way.
%
% \subsection{Bundle installation}
%
% \paragraph{Unpacking.} Unpack the \xfile{oberdiek.tds.zip} in the
% TDS tree (also known as \xfile{texmf} tree) of your choice.
% Example (linux):
% \begin{quote}
%   |unzip oberdiek.tds.zip -d ~/texmf|
% \end{quote}
%
% \paragraph{Script installation.}
% Check the directory \xfile{TDS:scripts/oberdiek/} for
% scripts that need further installation steps.
% Package \xpackage{attachfile2} comes with the Perl script
% \xfile{pdfatfi.pl} that should be installed in such a way
% that it can be called as \texttt{pdfatfi}.
% Example (linux):
% \begin{quote}
%   |chmod +x scripts/oberdiek/pdfatfi.pl|\\
%   |cp scripts/oberdiek/pdfatfi.pl /usr/local/bin/|
% \end{quote}
%
% \subsection{Package installation}
%
% \paragraph{Unpacking.} The \xfile{.dtx} file is a self-extracting
% \docstrip\ archive. The files are extracted by running the
% \xfile{.dtx} through \plainTeX:
% \begin{quote}
%   \verb|tex hypcap.dtx|
% \end{quote}
%
% \paragraph{TDS.} Now the different files must be moved into
% the different directories in your installation TDS tree
% (also known as \xfile{texmf} tree):
% \begin{quote}
% \def\t{^^A
% \begin{tabular}{@{}>{\ttfamily}l@{ $\rightarrow$ }>{\ttfamily}l@{}}
%   hypcap.sty & tex/latex/oberdiek/hypcap.sty\\
%   hypcap.pdf & doc/latex/oberdiek/hypcap.pdf\\
%   hypcap.dtx & source/latex/oberdiek/hypcap.dtx\\
% \end{tabular}^^A
% }^^A
% \sbox0{\t}^^A
% \ifdim\wd0>\linewidth
%   \begingroup
%     \advance\linewidth by\leftmargin
%     \advance\linewidth by\rightmargin
%   \edef\x{\endgroup
%     \def\noexpand\lw{\the\linewidth}^^A
%   }\x
%   \def\lwbox{^^A
%     \leavevmode
%     \hbox to \linewidth{^^A
%       \kern-\leftmargin\relax
%       \hss
%       \usebox0
%       \hss
%       \kern-\rightmargin\relax
%     }^^A
%   }^^A
%   \ifdim\wd0>\lw
%     \sbox0{\small\t}^^A
%     \ifdim\wd0>\linewidth
%       \ifdim\wd0>\lw
%         \sbox0{\footnotesize\t}^^A
%         \ifdim\wd0>\linewidth
%           \ifdim\wd0>\lw
%             \sbox0{\scriptsize\t}^^A
%             \ifdim\wd0>\linewidth
%               \ifdim\wd0>\lw
%                 \sbox0{\tiny\t}^^A
%                 \ifdim\wd0>\linewidth
%                   \lwbox
%                 \else
%                   \usebox0
%                 \fi
%               \else
%                 \lwbox
%               \fi
%             \else
%               \usebox0
%             \fi
%           \else
%             \lwbox
%           \fi
%         \else
%           \usebox0
%         \fi
%       \else
%         \lwbox
%       \fi
%     \else
%       \usebox0
%     \fi
%   \else
%     \lwbox
%   \fi
% \else
%   \usebox0
% \fi
% \end{quote}
% If you have a \xfile{docstrip.cfg} that configures and enables \docstrip's
% TDS installing feature, then some files can already be in the right
% place, see the documentation of \docstrip.
%
% \subsection{Refresh file name databases}
%
% If your \TeX~distribution
% (\teTeX, \mikTeX, \dots) relies on file name databases, you must refresh
% these. For example, \teTeX\ users run \verb|texhash| or
% \verb|mktexlsr|.
%
% \subsection{Some details for the interested}
%
% \paragraph{Attached source.}
%
% The PDF documentation on CTAN also includes the
% \xfile{.dtx} source file. It can be extracted by
% AcrobatReader 6 or higher. Another option is \textsf{pdftk},
% e.g. unpack the file into the current directory:
% \begin{quote}
%   \verb|pdftk hypcap.pdf unpack_files output .|
% \end{quote}
%
% \paragraph{Unpacking with \LaTeX.}
% The \xfile{.dtx} chooses its action depending on the format:
% \begin{description}
% \item[\plainTeX:] Run \docstrip\ and extract the files.
% \item[\LaTeX:] Generate the documentation.
% \end{description}
% If you insist on using \LaTeX\ for \docstrip\ (really,
% \docstrip\ does not need \LaTeX), then inform the autodetect routine
% about your intention:
% \begin{quote}
%   \verb|latex \let\install=y\input{hypcap.dtx}|
% \end{quote}
% Do not forget to quote the argument according to the demands
% of your shell.
%
% \paragraph{Generating the documentation.}
% You can use both the \xfile{.dtx} or the \xfile{.drv} to generate
% the documentation. The process can be configured by the
% configuration file \xfile{ltxdoc.cfg}. For instance, put this
% line into this file, if you want to have A4 as paper format:
% \begin{quote}
%   \verb|\PassOptionsToClass{a4paper}{article}|
% \end{quote}
% An example follows how to generate the
% documentation with pdf\LaTeX:
% \begin{quote}
%\begin{verbatim}
%pdflatex hypcap.dtx
%makeindex -s gind.ist hypcap.idx
%pdflatex hypcap.dtx
%makeindex -s gind.ist hypcap.idx
%pdflatex hypcap.dtx
%\end{verbatim}
% \end{quote}
%
% \section{Catalogue}
%
% The following XML file can be used as source for the
% \href{http://mirror.ctan.org/help/Catalogue/catalogue.html}{\TeX\ Catalogue}.
% The elements \texttt{caption} and \texttt{description} are imported
% from the original XML file from the Catalogue.
% The name of the XML file in the Catalogue is \xfile{hypcap.xml}.
%    \begin{macrocode}
%<*catalogue>
<?xml version='1.0' encoding='us-ascii'?>
<!DOCTYPE entry SYSTEM 'catalogue.dtd'>
<entry datestamp='$Date$' modifier='$Author$' id='hypcap'>
  <name>hypcap</name>
  <caption>Adjusting the anchors of captions.</caption>
  <authorref id='auth:oberdiek'/>
  <copyright owner='Heiko Oberdiek' year='1999-2001,2006-2008,2011'/>
  <license type='lppl1.3'/>
  <version number='1.11'/>
  <description>
    The package offers a solution to the problem that when you link to
    a float using <xref refid='hyperref'>hyperref</xref>, the link
    anchors to below the float's caption, rather than the beginning of
    the float.
    <p/>
    Hypcap defines a separate \capstart command, which you put where
    you want links to end; you should have a \capstart command for each
    \caption command.  Package options can be used to auto-insert a
    \capstart at the start of a float environment.
    <p/>
    The package is part of the <xref refid='oberdiek'>oberdiek</xref> bundle.
  </description>
  <documentation details='Package documentation'
      href='ctan:/macros/latex/contrib/oberdiek/hypcap.pdf'/>
  <ctan file='true' path='/macros/latex/contrib/oberdiek/hypcap.dtx'/>
  <miktex location='oberdiek'/>
  <texlive location='oberdiek'/>
  <install path='/macros/latex/contrib/oberdiek/oberdiek.tds.zip'/>
</entry>
%</catalogue>
%    \end{macrocode}
%
% \begin{History}
%   \begin{Version}{1999/02/13 v1.0}
%   \item
%     A beginning version, published in newsgroup \xnewsgroup{comp.text.tex}:\\
%     \URL{``\link{Re: \xpackage{hyperref} and figures}''}^^A
%     {http://groups.google.com/group/comp.text.tex/msg/5c9b47b001a9379c}
%   \end{Version}
%   \begin{Version}{2000/08/14 v1.1}
%   \item
%     Global assignments of \cs{if@capstart}
%     in order to allow \cs{caption} in groups.
%   \item
%     Option \texttt{all} added.
%   \end{Version}
%   \begin{Version}{2000/09/07 v1.2}
%   \item
%     Package in dtx format.
%   \end{Version}
%   \begin{Version}{2001/08/27 v1.3}
%   \item
%     Bug fix with hyperref's pdfmark driver\\
%     (\cs{leavevmode} in
%     \cs{hyper@@anchor}/\cs{pdf@rect}).
%   \end{Version}
%   \begin{Version}{2001/09/06 v1.4}
%   \item
%     Small fixes in the dtx file.
%   \end{Version}
%   \begin{Version}{2006/02/20 v1.5}
%   \item
%     Code is not changed.
%   \item
%     New DTX framework.
%   \end{Version}
%   \begin{Version}{2007/02/19 v1.6}
%   \item
%     Fix for hypertexnames=false.
%   \end{Version}
%   \begin{Version}{2007/04/09 v1.7}
%   \item
%     Stuff in \cs{caption} moved to \xpackage{hyperref}.
%     This avoids redefinitions of \cs{caption} and \cs{@caption}
%     (idea of Axel Sommerfeldt).
%   \item
%     Fix for subfigure (Marco Kuhlmann, Amilcar do Carmo Lucas).
%   \end{Version}
%   \begin{Version}{2008/04/14 v1.8}
%   \item
%     \cs{hc@redef} fixed to get package \xpackage{float} work
%     (Axel Sommerfeldt).
%   \end{Version}
%   \begin{Version}{2008/08/11 v1.9}
%   \item
%     Code is not changed.
%   \item
%     URLs updated.
%   \end{Version}
%   \begin{Version}{2008/09/08 v1.10}
%   \item
%     \cs{capstartfalse} and \cs{capstarttrue} added.
%   \end{Version}
%   \begin{Version}{2011/02/16 v1.11}
%   \item
%     \cs{hc@redef} fixed by using package \xpackage{letltxmacro}.
%   \end{Version}
% \end{History}
%
% \PrintIndex
%
% \Finale
\endinput
|
% \end{quote}
% Do not forget to quote the argument according to the demands
% of your shell.
%
% \paragraph{Generating the documentation.}
% You can use both the \xfile{.dtx} or the \xfile{.drv} to generate
% the documentation. The process can be configured by the
% configuration file \xfile{ltxdoc.cfg}. For instance, put this
% line into this file, if you want to have A4 as paper format:
% \begin{quote}
%   \verb|\PassOptionsToClass{a4paper}{article}|
% \end{quote}
% An example follows how to generate the
% documentation with pdf\LaTeX:
% \begin{quote}
%\begin{verbatim}
%pdflatex hypcap.dtx
%makeindex -s gind.ist hypcap.idx
%pdflatex hypcap.dtx
%makeindex -s gind.ist hypcap.idx
%pdflatex hypcap.dtx
%\end{verbatim}
% \end{quote}
%
% \section{Catalogue}
%
% The following XML file can be used as source for the
% \href{http://mirror.ctan.org/help/Catalogue/catalogue.html}{\TeX\ Catalogue}.
% The elements \texttt{caption} and \texttt{description} are imported
% from the original XML file from the Catalogue.
% The name of the XML file in the Catalogue is \xfile{hypcap.xml}.
%    \begin{macrocode}
%<*catalogue>
<?xml version='1.0' encoding='us-ascii'?>
<!DOCTYPE entry SYSTEM 'catalogue.dtd'>
<entry datestamp='$Date$' modifier='$Author$' id='hypcap'>
  <name>hypcap</name>
  <caption>Adjusting the anchors of captions.</caption>
  <authorref id='auth:oberdiek'/>
  <copyright owner='Heiko Oberdiek' year='1999-2001,2006-2008,2011'/>
  <license type='lppl1.3'/>
  <version number='1.11'/>
  <description>
    The package offers a solution to the problem that when you link to
    a float using <xref refid='hyperref'>hyperref</xref>, the link
    anchors to below the float's caption, rather than the beginning of
    the float.
    <p/>
    Hypcap defines a separate \capstart command, which you put where
    you want links to end; you should have a \capstart command for each
    \caption command.  Package options can be used to auto-insert a
    \capstart at the start of a float environment.
    <p/>
    The package is part of the <xref refid='oberdiek'>oberdiek</xref> bundle.
  </description>
  <documentation details='Package documentation'
      href='ctan:/macros/latex/contrib/oberdiek/hypcap.pdf'/>
  <ctan file='true' path='/macros/latex/contrib/oberdiek/hypcap.dtx'/>
  <miktex location='oberdiek'/>
  <texlive location='oberdiek'/>
  <install path='/macros/latex/contrib/oberdiek/oberdiek.tds.zip'/>
</entry>
%</catalogue>
%    \end{macrocode}
%
% \begin{History}
%   \begin{Version}{1999/02/13 v1.0}
%   \item
%     A beginning version, published in newsgroup \xnewsgroup{comp.text.tex}:\\
%     \URL{``\link{Re: \xpackage{hyperref} and figures}''}^^A
%     {http://groups.google.com/group/comp.text.tex/msg/5c9b47b001a9379c}
%   \end{Version}
%   \begin{Version}{2000/08/14 v1.1}
%   \item
%     Global assignments of \cs{if@capstart}
%     in order to allow \cs{caption} in groups.
%   \item
%     Option \texttt{all} added.
%   \end{Version}
%   \begin{Version}{2000/09/07 v1.2}
%   \item
%     Package in dtx format.
%   \end{Version}
%   \begin{Version}{2001/08/27 v1.3}
%   \item
%     Bug fix with hyperref's pdfmark driver\\
%     (\cs{leavevmode} in
%     \cs{hyper@@anchor}/\cs{pdf@rect}).
%   \end{Version}
%   \begin{Version}{2001/09/06 v1.4}
%   \item
%     Small fixes in the dtx file.
%   \end{Version}
%   \begin{Version}{2006/02/20 v1.5}
%   \item
%     Code is not changed.
%   \item
%     New DTX framework.
%   \end{Version}
%   \begin{Version}{2007/02/19 v1.6}
%   \item
%     Fix for hypertexnames=false.
%   \end{Version}
%   \begin{Version}{2007/04/09 v1.7}
%   \item
%     Stuff in \cs{caption} moved to \xpackage{hyperref}.
%     This avoids redefinitions of \cs{caption} and \cs{@caption}
%     (idea of Axel Sommerfeldt).
%   \item
%     Fix for subfigure (Marco Kuhlmann, Amilcar do Carmo Lucas).
%   \end{Version}
%   \begin{Version}{2008/04/14 v1.8}
%   \item
%     \cs{hc@redef} fixed to get package \xpackage{float} work
%     (Axel Sommerfeldt).
%   \end{Version}
%   \begin{Version}{2008/08/11 v1.9}
%   \item
%     Code is not changed.
%   \item
%     URLs updated.
%   \end{Version}
%   \begin{Version}{2008/09/08 v1.10}
%   \item
%     \cs{capstartfalse} and \cs{capstarttrue} added.
%   \end{Version}
%   \begin{Version}{2011/02/16 v1.11}
%   \item
%     \cs{hc@redef} fixed by using package \xpackage{letltxmacro}.
%   \end{Version}
% \end{History}
%
% \PrintIndex
%
% \Finale
\endinput
|
% \end{quote}
% Do not forget to quote the argument according to the demands
% of your shell.
%
% \paragraph{Generating the documentation.}
% You can use both the \xfile{.dtx} or the \xfile{.drv} to generate
% the documentation. The process can be configured by the
% configuration file \xfile{ltxdoc.cfg}. For instance, put this
% line into this file, if you want to have A4 as paper format:
% \begin{quote}
%   \verb|\PassOptionsToClass{a4paper}{article}|
% \end{quote}
% An example follows how to generate the
% documentation with pdf\LaTeX:
% \begin{quote}
%\begin{verbatim}
%pdflatex hypcap.dtx
%makeindex -s gind.ist hypcap.idx
%pdflatex hypcap.dtx
%makeindex -s gind.ist hypcap.idx
%pdflatex hypcap.dtx
%\end{verbatim}
% \end{quote}
%
% \section{Catalogue}
%
% The following XML file can be used as source for the
% \href{http://mirror.ctan.org/help/Catalogue/catalogue.html}{\TeX\ Catalogue}.
% The elements \texttt{caption} and \texttt{description} are imported
% from the original XML file from the Catalogue.
% The name of the XML file in the Catalogue is \xfile{hypcap.xml}.
%    \begin{macrocode}
%<*catalogue>
<?xml version='1.0' encoding='us-ascii'?>
<!DOCTYPE entry SYSTEM 'catalogue.dtd'>
<entry datestamp='$Date$' modifier='$Author$' id='hypcap'>
  <name>hypcap</name>
  <caption>Adjusting the anchors of captions.</caption>
  <authorref id='auth:oberdiek'/>
  <copyright owner='Heiko Oberdiek' year='1999-2001,2006-2008,2011'/>
  <license type='lppl1.3'/>
  <version number='1.11'/>
  <description>
    The package offers a solution to the problem that when you link to
    a float using <xref refid='hyperref'>hyperref</xref>, the link
    anchors to below the float's caption, rather than the beginning of
    the float.
    <p/>
    Hypcap defines a separate \capstart command, which you put where
    you want links to end; you should have a \capstart command for each
    \caption command.  Package options can be used to auto-insert a
    \capstart at the start of a float environment.
    <p/>
    The package is part of the <xref refid='oberdiek'>oberdiek</xref> bundle.
  </description>
  <documentation details='Package documentation'
      href='ctan:/macros/latex/contrib/oberdiek/hypcap.pdf'/>
  <ctan file='true' path='/macros/latex/contrib/oberdiek/hypcap.dtx'/>
  <miktex location='oberdiek'/>
  <texlive location='oberdiek'/>
  <install path='/macros/latex/contrib/oberdiek/oberdiek.tds.zip'/>
</entry>
%</catalogue>
%    \end{macrocode}
%
% \begin{History}
%   \begin{Version}{1999/02/13 v1.0}
%   \item
%     A beginning version, published in newsgroup \xnewsgroup{comp.text.tex}:\\
%     \URL{``\link{Re: \xpackage{hyperref} and figures}''}^^A
%     {http://groups.google.com/group/comp.text.tex/msg/5c9b47b001a9379c}
%   \end{Version}
%   \begin{Version}{2000/08/14 v1.1}
%   \item
%     Global assignments of \cs{if@capstart}
%     in order to allow \cs{caption} in groups.
%   \item
%     Option \texttt{all} added.
%   \end{Version}
%   \begin{Version}{2000/09/07 v1.2}
%   \item
%     Package in dtx format.
%   \end{Version}
%   \begin{Version}{2001/08/27 v1.3}
%   \item
%     Bug fix with hyperref's pdfmark driver\\
%     (\cs{leavevmode} in
%     \cs{hyper@@anchor}/\cs{pdf@rect}).
%   \end{Version}
%   \begin{Version}{2001/09/06 v1.4}
%   \item
%     Small fixes in the dtx file.
%   \end{Version}
%   \begin{Version}{2006/02/20 v1.5}
%   \item
%     Code is not changed.
%   \item
%     New DTX framework.
%   \end{Version}
%   \begin{Version}{2007/02/19 v1.6}
%   \item
%     Fix for hypertexnames=false.
%   \end{Version}
%   \begin{Version}{2007/04/09 v1.7}
%   \item
%     Stuff in \cs{caption} moved to \xpackage{hyperref}.
%     This avoids redefinitions of \cs{caption} and \cs{@caption}
%     (idea of Axel Sommerfeldt).
%   \item
%     Fix for subfigure (Marco Kuhlmann, Amilcar do Carmo Lucas).
%   \end{Version}
%   \begin{Version}{2008/04/14 v1.8}
%   \item
%     \cs{hc@redef} fixed to get package \xpackage{float} work
%     (Axel Sommerfeldt).
%   \end{Version}
%   \begin{Version}{2008/08/11 v1.9}
%   \item
%     Code is not changed.
%   \item
%     URLs updated.
%   \end{Version}
%   \begin{Version}{2008/09/08 v1.10}
%   \item
%     \cs{capstartfalse} and \cs{capstarttrue} added.
%   \end{Version}
%   \begin{Version}{2011/02/16 v1.11}
%   \item
%     \cs{hc@redef} fixed by using package \xpackage{letltxmacro}.
%   \end{Version}
% \end{History}
%
% \PrintIndex
%
% \Finale
\endinput

%        (quote the arguments according to the demands of your shell)
%
% Documentation:
%    (a) If hypcap.drv is present:
%           latex hypcap.drv
%    (b) Without hypcap.drv:
%           latex hypcap.dtx; ...
%    The class ltxdoc loads the configuration file ltxdoc.cfg
%    if available. Here you can specify further options, e.g.
%    use A4 as paper format:
%       \PassOptionsToClass{a4paper}{article}
%
%    Programm calls to get the documentation (example):
%       pdflatex hypcap.dtx
%       makeindex -s gind.ist hypcap.idx
%       pdflatex hypcap.dtx
%       makeindex -s gind.ist hypcap.idx
%       pdflatex hypcap.dtx
%
% Installation:
%    TDS:tex/latex/oberdiek/hypcap.sty
%    TDS:doc/latex/oberdiek/hypcap.pdf
%    TDS:source/latex/oberdiek/hypcap.dtx
%
%<*ignore>
\begingroup
  \catcode123=1 %
  \catcode125=2 %
  \def\x{LaTeX2e}%
\expandafter\endgroup
\ifcase 0\ifx\install y1\fi\expandafter
         \ifx\csname processbatchFile\endcsname\relax\else1\fi
         \ifx\fmtname\x\else 1\fi\relax
\else\csname fi\endcsname
%</ignore>
%<*install>
\input docstrip.tex
\Msg{************************************************************************}
\Msg{* Installation}
\Msg{* Package: hypcap 2011/02/16 v1.11 Adjusting the anchors of captions (HO)}
\Msg{************************************************************************}

\keepsilent
\askforoverwritefalse

\let\MetaPrefix\relax
\preamble

This is a generated file.

Project: hypcap
Version: 2011/02/16 v1.11

Copyright (C) 1999-2001, 2006-2008, 2011 by
   Heiko Oberdiek <heiko.oberdiek at googlemail.com>

This work may be distributed and/or modified under the
conditions of the LaTeX Project Public License, either
version 1.3c of this license or (at your option) any later
version. This version of this license is in
   http://www.latex-project.org/lppl/lppl-1-3c.txt
and the latest version of this license is in
   http://www.latex-project.org/lppl.txt
and version 1.3 or later is part of all distributions of
LaTeX version 2005/12/01 or later.

This work has the LPPL maintenance status "maintained".

This Current Maintainer of this work is Heiko Oberdiek.

This work consists of the main source file hypcap.dtx
and the derived files
   hypcap.sty, hypcap.pdf, hypcap.ins, hypcap.drv.

\endpreamble
\let\MetaPrefix\DoubleperCent

\generate{%
  \file{hypcap.ins}{\from{hypcap.dtx}{install}}%
  \file{hypcap.drv}{\from{hypcap.dtx}{driver}}%
  \usedir{tex/latex/oberdiek}%
  \file{hypcap.sty}{\from{hypcap.dtx}{package}}%
  \nopreamble
  \nopostamble
  \usedir{source/latex/oberdiek/catalogue}%
  \file{hypcap.xml}{\from{hypcap.dtx}{catalogue}}%
}

\catcode32=13\relax% active space
\let =\space%
\Msg{************************************************************************}
\Msg{*}
\Msg{* To finish the installation you have to move the following}
\Msg{* file into a directory searched by TeX:}
\Msg{*}
\Msg{*     hypcap.sty}
\Msg{*}
\Msg{* To produce the documentation run the file `hypcap.drv'}
\Msg{* through LaTeX.}
\Msg{*}
\Msg{* Happy TeXing!}
\Msg{*}
\Msg{************************************************************************}

\endbatchfile
%</install>
%<*ignore>
\fi
%</ignore>
%<*driver>
\NeedsTeXFormat{LaTeX2e}
\ProvidesFile{hypcap.drv}%
  [2011/02/16 v1.11 Adjusting the anchors of captions (HO)]%
\documentclass{ltxdoc}
\usepackage{holtxdoc}[2011/11/22]
\begin{document}
  \DocInput{hypcap.dtx}%
\end{document}
%</driver>
% \fi
%
% \CheckSum{174}
%
% \CharacterTable
%  {Upper-case    \A\B\C\D\E\F\G\H\I\J\K\L\M\N\O\P\Q\R\S\T\U\V\W\X\Y\Z
%   Lower-case    \a\b\c\d\e\f\g\h\i\j\k\l\m\n\o\p\q\r\s\t\u\v\w\x\y\z
%   Digits        \0\1\2\3\4\5\6\7\8\9
%   Exclamation   \!     Double quote  \"     Hash (number) \#
%   Dollar        \$     Percent       \%     Ampersand     \&
%   Acute accent  \'     Left paren    \(     Right paren   \)
%   Asterisk      \*     Plus          \+     Comma         \,
%   Minus         \-     Point         \.     Solidus       \/
%   Colon         \:     Semicolon     \;     Less than     \<
%   Equals        \=     Greater than  \>     Question mark \?
%   Commercial at \@     Left bracket  \[     Backslash     \\
%   Right bracket \]     Circumflex    \^     Underscore    \_
%   Grave accent  \`     Left brace    \{     Vertical bar  \|
%   Right brace   \}     Tilde         \~}
%
% \GetFileInfo{hypcap.drv}
%
% \title{The \xpackage{hypcap} package}
% \date{2011/02/16 v1.11}
% \author{Heiko Oberdiek\\\xemail{heiko.oberdiek at googlemail.com}}
%
% \maketitle
%
% \begin{abstract}
% This package tries a solution of the problem with
% hyperref, that links to floats points below the
% caption and not at the beginning of the float.
% Therefore this package divides the task into two
% part, the link setting with \cs{capstart} or
% automatically at the beginning of a float and
% the rest in the \cs{caption} command.
% \end{abstract}
%
% \tableofcontents
%
% \newpage
% \section{Usage}
%    The package \Package{hypcap} requires that \Package{hyperref}
%    is loaded first:
%    \begin{quote}
%      |\usepackage[...]{hyperref}|\\
%      |\usepackage[...]{hypcap}|
%    \end{quote}
%
% \subsection{Package options}
%    The names of the four float environments |figure|, |figure*|,
%    |table|, or |table*| can be used as option. Then the package
%    redefines the environment in order to insert \cmd{\capstart}
%    (see below) in the beginning of the environment automatically.
%
%    Option |all| enables the redefinitions of all four
%    float environments. For other environments see
%    the user command \cmd{\hypcapredef}.
%
% \subsection{User commands}
%    \begin{description}
%    \item[\cmd{\capstart}:]\DescribeMacro{\capstart}
%      First this command increments the counter (\cmd{\@captype}).
%      Then it makes an anchor for package \Package{hyperref}.
%      At last \cmd{\caption} is redefined to remove the
%      anchor setting part from \Package{hyperref}'s \cmd{\caption}.
%
%      The package expects the following structure of a float
%      environment:
%      \begin{quote}
%        |\begin{|\textit{|float|}|}...|\\
%        |\capstart|\\
%        |...|\\
%        |\caption{...}|\\
%        |...|\\
%        |\end{|\textit{|float|}|}|
%      \end{quote}
%      There can be several \cmd{\caption} commands. For these
%      you need \cmd{\capstart} again:
%      \begin{quote}
%        |\capstart ... \caption... \capstart ... \caption...|
%      \end{quote}
%      And the \cmd{\caption} command itself can be put in a group.
%
%      With the options, described above, the extra
%      writing of \cmd{\capstart} can be avoided.
%      Consequently, there must be a \cmd{\caption} in every
%      environment of this type, specified by the option.
%      If you want to use more than one \cmd{\caption} in this
%      environment, you have to state \cmd{\capstart} again.
%    \item[\cmd{\hypcapspace}:]\DescribeMacro{\hypcapspace}
%      Because it looks poor, if the link points exactly at top of
%      the figure, there is additional space: \cmd{\hypcapspace},
%      the default is |0.5\baselineskip|, examples:
%      \begin{quote}
%        |\renewcommand{\hypcapspace}{0pt}| removes the space\\
%        |\renewcommand{\hypcapspace}{1pt}| sets a fix value
%      \end{quote}
%    \item[\cmd{\hypcapredef}:]\DescribeMacro{\hypcapredef}
%      If there are other float environments, that
%      should automatically execute \cmd{\capstart}, then
%      a redefinition with \cmd{\hypcapredef} can be tried:
%      \begin{quote}
%        |\hypcapredef{myfloat}|
%      \end{quote}
%      Only environments with one optional parameter are
%      supported.
%    \item[\cs{capstartfalse}, \cs{capstarttrue}:]^^A
%      \DescribeMacro{\capstartfalse}^^A
%      \DescribeMacro{\capstarttrue}^^A
%      Since 2008/09/08 v1.10.\\
%      They disable and enable \cs{capstart}. They can be used to
%      cancel the effect of a redefined float environment.
%      Example:
%\begin{quote}
%\catcode`Y=\active
%\def Y#1Z{\textit{\cs{#1}}}
%\begin{verbatim}
%\documentclass{article}
%\usepackage{hyperref}
%\usepackage[figure]{hypcap}[2008/09/08]
%
%\begin{document}
%  \section{Hello World}
%  \begin{figure}
%    \caption{Figure with caption A}
%  \end{figure}
%  YcapstartfalseZ
%  \begin{figure}
%    Figure without caption
%  \end{figure}
%  YcapstarttrueZ
%  \begin{figure}
%    \caption{Figure with caption B}
%  \end{figure}
%\end{document}
%\end{verbatim}
%\end{quote}
%    \end{description}
%
% \subsection{Limitations}
%    \begin{itemize}
%    \item Packages that redefine \cmd{\caption} or
%          \cmd{\@caption}.
%    \end{itemize}
%
% \StopEventually{
% }
%
% \section{Implementation}
%    \begin{macrocode}
%<*package>
%    \end{macrocode}
%    Package identification.
%    \begin{macrocode}
\NeedsTeXFormat{LaTeX2e}
\ProvidesPackage{hypcap}%
  [2011/02/16 v1.11 Adjusting the anchors of captions (HO)]
%    \end{macrocode}
%
%    For unique command names this package uses |hc@| as prefix
%    for internal command names.
%
%    First we check, if package \Package{hyperref} is loaded:
%    \begin{macrocode}
\@ifundefined{hyper@@anchor}{%
  \PackageError{hypcap}{You have to load 'hyperref' first}\@ehc
  \endinput
}{}
%    \end{macrocode}
%
%    \begin{macrocode}
\RequirePackage{letltxmacro}[2008/06/24]
%    \end{macrocode}
%
%    \begin{macro}{\hc@org@caption}
%    Save the original meaning of \cmd{\caption}:
%    \begin{macrocode}
\newcommand*\hc@org@caption{}
\let\hc@org@caption\caption
%    \end{macrocode}
%    \end{macro}
%
%    \newcommand*{\BeginMacro}[1]{\expandafter\DoBeginMacro\csname#1\endcsname}
%    \newcommand*{\DoBeginMacro}[1]{\begin{macro}{#1}}
%    \newcommand*{\Cmd}[1]{\expandafter\cmd\csname#1\endcsname}
%    \BeginMacro{if@capstart}
%     The switch \Cmd{if@capstart} helps to detect
%     \cmd{\capstart} commands with missing \cmd{\caption} macros.
%     Because \cmd{\caption} can occur inside a group, assignments
%     to the switch have to be made global.
%    \begin{macrocode}
\newif\if@capstart
%    \end{macrocode}
%    \end{macro}
%
%    \begin{macro}{\hypcapspace}
%    The anchor is raised.by \cmd{\hypcapspace}.
%    \begin{macrocode}
\newcommand*\hypcapspace{.5\baselineskip}
%    \end{macrocode}
%    \end{macro}
%
%    \begin{macro}{\ifcapstart}
%    \begin{macrocode}
\newif\ifcapstart
\capstarttrue
%    \end{macrocode}
%    \end{macro}
%
%    \begin{macro}{\capstart}
%     The macro \cmd{\capstart} contains the first part of
%     the \cmd{\caption} command: Incrementing the counter
%     and setting the anchor.
%    \begin{macrocode}
\newcommand*\capstart{%
  \ifcapstart
    \H@refstepcounter\@captype % first part of caption
    \hyper@makecurrent\@captype
    \global\let\hc@currentHref\@currentHref
    \vspace*{-\hypcapspace}%
    \begingroup
      \let\leavevmode\relax
      \hyper@@anchor\@currentHref\relax
    \endgroup
    \vspace*{\hypcapspace}%
    \hc@hyperref{\let\caption\hc@caption}%
    \global\@capstarttrue
    \global\advance\csname c@\@captype\endcsname\m@ne
  \fi
}
%    \end{macrocode}
%    \end{macro}
%
%    \begin{macrocode}
\@ifpackagelater{hyperref}{2007/04/09}{%
  \let\hc@hyperref\@gobble
}{%
  \let\hc@hyperref\@firstofone
}
%    \end{macrocode}
%
%    \begin{macro}{\hc@caption}
%    The new \cmd{\caption} command without the first part
%    is defined in the macro \cmd{\hc@caption}.
%    \begin{macrocode}
\def\hc@caption{%
  \global\advance\csname c@\@captype\endcsname\@ne
  \@dblarg{\hc@@caption\@captype}%
}
%    \end{macrocode}
%    \end{macro}
%
%    \begin{macro}{\hc@@caption}
%    This is a copy of package \Package{hyperref}'s \cmd{\@caption}
%    macro without making the anchor, because this is already done
%    in \cmd{\capstart}.
%    \begin{macrocode}
\long\def\hc@@caption#1[#2]#3{%
  \let\caption\hc@org@caption
  \global\@capstartfalse
  \ifHy@hypertexnames
    \hyper@makecurrent\@captype
  \else
    \global\let\@currentHref\hc@currentHref
  \fi
  \par\addcontentsline{%
    \csname ext@#1\endcsname}{#1}{%
    \protect\numberline{%
      \csname the#1\endcsname
    }{\ignorespaces #2}%
  }%
  \begingroup
    \@parboxrestore
    \normalsize
    \@makecaption{\csname fnum@#1\endcsname}{%
      \ignorespaces#3%
    }%
    \par
  \endgroup
}
%    \end{macrocode}
%    \end{macro}
%
%    \begin{macro}{\hypcapredef}
%    The macro \cmd{\hypcapredef} prepares the call of
%    \cmd{\hc@redef} that will redefine the environment
%    that is given in the argument.
%    \begin{macrocode}
\def\hypcapredef#1{%
  \expandafter\hc@redef\csname hc@org#1\expandafter\endcsname
                       \csname hc@orgend#1\expandafter\endcsname
                       \expandafter{#1}%
}
%    \end{macrocode}
%    \end{macro}
%    \begin{macro}{\hc@redef}
%    The old meaning of the environment is saved. Then
%    \cmd{\capstart} is appended in the begin part.
%    The end part contains a check
%    that produces an error message in case of \cmd{\capstart}
%    without \cmd{\capstart} (\cmd{\capstart} has incremented
%    the counter).
%    \begin{macrocode}
\def\hc@redef#1#2#3{%
  \newcommand#1{}%
  \expandafter\LetLtxMacro\expandafter#1\csname#3\endcsname
  \expandafter\LetLtxMacro\expandafter#2\csname end#3\endcsname
  \renewenvironment*{#3}[1][]{%
    \ifx\\##1\\%
      #1\relax
    \else
      #1[##1]% hash-ok (compatibility for float)
    \fi
    \capstart
  }{%
    \if@capstart
      \PackageError{hypcap}{You have forgotten to use \string\caption}%
      \global\@capstartfalse
    \else
    \fi
    #2%
  }%
}
%    \end{macrocode}
%    \end{macro}
%
%     At last the options are defined and processed.
%    \begin{macrocode}
\DeclareOption{figure}{\hypcapredef{\CurrentOption}}
\DeclareOption{figure*}{\hypcapredef{\CurrentOption}}
\DeclareOption{table}{\hypcapredef{\CurrentOption}}
\DeclareOption{table*}{\hypcapredef{\CurrentOption}}
\DeclareOption{all}{%
  \hypcapredef{figure}%
  \hypcapredef{figure*}%
  \hypcapredef{table}%
  \hypcapredef{table*}%
}
\ProcessOptions\relax
%    \end{macrocode}
%
%    \begin{macrocode}
%</package>
%    \end{macrocode}
%
% \section{Installation}
%
% \subsection{Download}
%
% \paragraph{Package.} This package is available on
% CTAN\footnote{\url{ftp://ftp.ctan.org/tex-archive/}}:
% \begin{description}
% \item[\CTAN{macros/latex/contrib/oberdiek/hypcap.dtx}] The source file.
% \item[\CTAN{macros/latex/contrib/oberdiek/hypcap.pdf}] Documentation.
% \end{description}
%
%
% \paragraph{Bundle.} All the packages of the bundle `oberdiek'
% are also available in a TDS compliant ZIP archive. There
% the packages are already unpacked and the documentation files
% are generated. The files and directories obey the TDS standard.
% \begin{description}
% \item[\CTAN{install/macros/latex/contrib/oberdiek.tds.zip}]
% \end{description}
% \emph{TDS} refers to the standard ``A Directory Structure
% for \TeX\ Files'' (\CTAN{tds/tds.pdf}). Directories
% with \xfile{texmf} in their name are usually organized this way.
%
% \subsection{Bundle installation}
%
% \paragraph{Unpacking.} Unpack the \xfile{oberdiek.tds.zip} in the
% TDS tree (also known as \xfile{texmf} tree) of your choice.
% Example (linux):
% \begin{quote}
%   |unzip oberdiek.tds.zip -d ~/texmf|
% \end{quote}
%
% \paragraph{Script installation.}
% Check the directory \xfile{TDS:scripts/oberdiek/} for
% scripts that need further installation steps.
% Package \xpackage{attachfile2} comes with the Perl script
% \xfile{pdfatfi.pl} that should be installed in such a way
% that it can be called as \texttt{pdfatfi}.
% Example (linux):
% \begin{quote}
%   |chmod +x scripts/oberdiek/pdfatfi.pl|\\
%   |cp scripts/oberdiek/pdfatfi.pl /usr/local/bin/|
% \end{quote}
%
% \subsection{Package installation}
%
% \paragraph{Unpacking.} The \xfile{.dtx} file is a self-extracting
% \docstrip\ archive. The files are extracted by running the
% \xfile{.dtx} through \plainTeX:
% \begin{quote}
%   \verb|tex hypcap.dtx|
% \end{quote}
%
% \paragraph{TDS.} Now the different files must be moved into
% the different directories in your installation TDS tree
% (also known as \xfile{texmf} tree):
% \begin{quote}
% \def\t{^^A
% \begin{tabular}{@{}>{\ttfamily}l@{ $\rightarrow$ }>{\ttfamily}l@{}}
%   hypcap.sty & tex/latex/oberdiek/hypcap.sty\\
%   hypcap.pdf & doc/latex/oberdiek/hypcap.pdf\\
%   hypcap.dtx & source/latex/oberdiek/hypcap.dtx\\
% \end{tabular}^^A
% }^^A
% \sbox0{\t}^^A
% \ifdim\wd0>\linewidth
%   \begingroup
%     \advance\linewidth by\leftmargin
%     \advance\linewidth by\rightmargin
%   \edef\x{\endgroup
%     \def\noexpand\lw{\the\linewidth}^^A
%   }\x
%   \def\lwbox{^^A
%     \leavevmode
%     \hbox to \linewidth{^^A
%       \kern-\leftmargin\relax
%       \hss
%       \usebox0
%       \hss
%       \kern-\rightmargin\relax
%     }^^A
%   }^^A
%   \ifdim\wd0>\lw
%     \sbox0{\small\t}^^A
%     \ifdim\wd0>\linewidth
%       \ifdim\wd0>\lw
%         \sbox0{\footnotesize\t}^^A
%         \ifdim\wd0>\linewidth
%           \ifdim\wd0>\lw
%             \sbox0{\scriptsize\t}^^A
%             \ifdim\wd0>\linewidth
%               \ifdim\wd0>\lw
%                 \sbox0{\tiny\t}^^A
%                 \ifdim\wd0>\linewidth
%                   \lwbox
%                 \else
%                   \usebox0
%                 \fi
%               \else
%                 \lwbox
%               \fi
%             \else
%               \usebox0
%             \fi
%           \else
%             \lwbox
%           \fi
%         \else
%           \usebox0
%         \fi
%       \else
%         \lwbox
%       \fi
%     \else
%       \usebox0
%     \fi
%   \else
%     \lwbox
%   \fi
% \else
%   \usebox0
% \fi
% \end{quote}
% If you have a \xfile{docstrip.cfg} that configures and enables \docstrip's
% TDS installing feature, then some files can already be in the right
% place, see the documentation of \docstrip.
%
% \subsection{Refresh file name databases}
%
% If your \TeX~distribution
% (\teTeX, \mikTeX, \dots) relies on file name databases, you must refresh
% these. For example, \teTeX\ users run \verb|texhash| or
% \verb|mktexlsr|.
%
% \subsection{Some details for the interested}
%
% \paragraph{Attached source.}
%
% The PDF documentation on CTAN also includes the
% \xfile{.dtx} source file. It can be extracted by
% AcrobatReader 6 or higher. Another option is \textsf{pdftk},
% e.g. unpack the file into the current directory:
% \begin{quote}
%   \verb|pdftk hypcap.pdf unpack_files output .|
% \end{quote}
%
% \paragraph{Unpacking with \LaTeX.}
% The \xfile{.dtx} chooses its action depending on the format:
% \begin{description}
% \item[\plainTeX:] Run \docstrip\ and extract the files.
% \item[\LaTeX:] Generate the documentation.
% \end{description}
% If you insist on using \LaTeX\ for \docstrip\ (really,
% \docstrip\ does not need \LaTeX), then inform the autodetect routine
% about your intention:
% \begin{quote}
%   \verb|latex \let\install=y% \iffalse meta-comment
%
% File: hypcap.dtx
% Version: 2011/02/16 v1.11
% Info: Adjusting the anchors of captions
%
% Copyright (C) 1999-2001, 2006-2008, 2011 by
%    Heiko Oberdiek <heiko.oberdiek at googlemail.com>
%
% This work may be distributed and/or modified under the
% conditions of the LaTeX Project Public License, either
% version 1.3c of this license or (at your option) any later
% version. This version of this license is in
%    http://www.latex-project.org/lppl/lppl-1-3c.txt
% and the latest version of this license is in
%    http://www.latex-project.org/lppl.txt
% and version 1.3 or later is part of all distributions of
% LaTeX version 2005/12/01 or later.
%
% This work has the LPPL maintenance status "maintained".
%
% This Current Maintainer of this work is Heiko Oberdiek.
%
% This work consists of the main source file hypcap.dtx
% and the derived files
%    hypcap.sty, hypcap.pdf, hypcap.ins, hypcap.drv.
%
% Distribution:
%    CTAN:macros/latex/contrib/oberdiek/hypcap.dtx
%    CTAN:macros/latex/contrib/oberdiek/hypcap.pdf
%
% Unpacking:
%    (a) If hypcap.ins is present:
%           tex hypcap.ins
%    (b) Without hypcap.ins:
%           tex hypcap.dtx
%    (c) If you insist on using LaTeX
%           latex \let\install=y% \iffalse meta-comment
%
% File: hypcap.dtx
% Version: 2011/02/16 v1.11
% Info: Adjusting the anchors of captions
%
% Copyright (C) 1999-2001, 2006-2008, 2011 by
%    Heiko Oberdiek <heiko.oberdiek at googlemail.com>
%
% This work may be distributed and/or modified under the
% conditions of the LaTeX Project Public License, either
% version 1.3c of this license or (at your option) any later
% version. This version of this license is in
%    http://www.latex-project.org/lppl/lppl-1-3c.txt
% and the latest version of this license is in
%    http://www.latex-project.org/lppl.txt
% and version 1.3 or later is part of all distributions of
% LaTeX version 2005/12/01 or later.
%
% This work has the LPPL maintenance status "maintained".
%
% This Current Maintainer of this work is Heiko Oberdiek.
%
% This work consists of the main source file hypcap.dtx
% and the derived files
%    hypcap.sty, hypcap.pdf, hypcap.ins, hypcap.drv.
%
% Distribution:
%    CTAN:macros/latex/contrib/oberdiek/hypcap.dtx
%    CTAN:macros/latex/contrib/oberdiek/hypcap.pdf
%
% Unpacking:
%    (a) If hypcap.ins is present:
%           tex hypcap.ins
%    (b) Without hypcap.ins:
%           tex hypcap.dtx
%    (c) If you insist on using LaTeX
%           latex \let\install=y% \iffalse meta-comment
%
% File: hypcap.dtx
% Version: 2011/02/16 v1.11
% Info: Adjusting the anchors of captions
%
% Copyright (C) 1999-2001, 2006-2008, 2011 by
%    Heiko Oberdiek <heiko.oberdiek at googlemail.com>
%
% This work may be distributed and/or modified under the
% conditions of the LaTeX Project Public License, either
% version 1.3c of this license or (at your option) any later
% version. This version of this license is in
%    http://www.latex-project.org/lppl/lppl-1-3c.txt
% and the latest version of this license is in
%    http://www.latex-project.org/lppl.txt
% and version 1.3 or later is part of all distributions of
% LaTeX version 2005/12/01 or later.
%
% This work has the LPPL maintenance status "maintained".
%
% This Current Maintainer of this work is Heiko Oberdiek.
%
% This work consists of the main source file hypcap.dtx
% and the derived files
%    hypcap.sty, hypcap.pdf, hypcap.ins, hypcap.drv.
%
% Distribution:
%    CTAN:macros/latex/contrib/oberdiek/hypcap.dtx
%    CTAN:macros/latex/contrib/oberdiek/hypcap.pdf
%
% Unpacking:
%    (a) If hypcap.ins is present:
%           tex hypcap.ins
%    (b) Without hypcap.ins:
%           tex hypcap.dtx
%    (c) If you insist on using LaTeX
%           latex \let\install=y\input{hypcap.dtx}
%        (quote the arguments according to the demands of your shell)
%
% Documentation:
%    (a) If hypcap.drv is present:
%           latex hypcap.drv
%    (b) Without hypcap.drv:
%           latex hypcap.dtx; ...
%    The class ltxdoc loads the configuration file ltxdoc.cfg
%    if available. Here you can specify further options, e.g.
%    use A4 as paper format:
%       \PassOptionsToClass{a4paper}{article}
%
%    Programm calls to get the documentation (example):
%       pdflatex hypcap.dtx
%       makeindex -s gind.ist hypcap.idx
%       pdflatex hypcap.dtx
%       makeindex -s gind.ist hypcap.idx
%       pdflatex hypcap.dtx
%
% Installation:
%    TDS:tex/latex/oberdiek/hypcap.sty
%    TDS:doc/latex/oberdiek/hypcap.pdf
%    TDS:source/latex/oberdiek/hypcap.dtx
%
%<*ignore>
\begingroup
  \catcode123=1 %
  \catcode125=2 %
  \def\x{LaTeX2e}%
\expandafter\endgroup
\ifcase 0\ifx\install y1\fi\expandafter
         \ifx\csname processbatchFile\endcsname\relax\else1\fi
         \ifx\fmtname\x\else 1\fi\relax
\else\csname fi\endcsname
%</ignore>
%<*install>
\input docstrip.tex
\Msg{************************************************************************}
\Msg{* Installation}
\Msg{* Package: hypcap 2011/02/16 v1.11 Adjusting the anchors of captions (HO)}
\Msg{************************************************************************}

\keepsilent
\askforoverwritefalse

\let\MetaPrefix\relax
\preamble

This is a generated file.

Project: hypcap
Version: 2011/02/16 v1.11

Copyright (C) 1999-2001, 2006-2008, 2011 by
   Heiko Oberdiek <heiko.oberdiek at googlemail.com>

This work may be distributed and/or modified under the
conditions of the LaTeX Project Public License, either
version 1.3c of this license or (at your option) any later
version. This version of this license is in
   http://www.latex-project.org/lppl/lppl-1-3c.txt
and the latest version of this license is in
   http://www.latex-project.org/lppl.txt
and version 1.3 or later is part of all distributions of
LaTeX version 2005/12/01 or later.

This work has the LPPL maintenance status "maintained".

This Current Maintainer of this work is Heiko Oberdiek.

This work consists of the main source file hypcap.dtx
and the derived files
   hypcap.sty, hypcap.pdf, hypcap.ins, hypcap.drv.

\endpreamble
\let\MetaPrefix\DoubleperCent

\generate{%
  \file{hypcap.ins}{\from{hypcap.dtx}{install}}%
  \file{hypcap.drv}{\from{hypcap.dtx}{driver}}%
  \usedir{tex/latex/oberdiek}%
  \file{hypcap.sty}{\from{hypcap.dtx}{package}}%
  \nopreamble
  \nopostamble
  \usedir{source/latex/oberdiek/catalogue}%
  \file{hypcap.xml}{\from{hypcap.dtx}{catalogue}}%
}

\catcode32=13\relax% active space
\let =\space%
\Msg{************************************************************************}
\Msg{*}
\Msg{* To finish the installation you have to move the following}
\Msg{* file into a directory searched by TeX:}
\Msg{*}
\Msg{*     hypcap.sty}
\Msg{*}
\Msg{* To produce the documentation run the file `hypcap.drv'}
\Msg{* through LaTeX.}
\Msg{*}
\Msg{* Happy TeXing!}
\Msg{*}
\Msg{************************************************************************}

\endbatchfile
%</install>
%<*ignore>
\fi
%</ignore>
%<*driver>
\NeedsTeXFormat{LaTeX2e}
\ProvidesFile{hypcap.drv}%
  [2011/02/16 v1.11 Adjusting the anchors of captions (HO)]%
\documentclass{ltxdoc}
\usepackage{holtxdoc}[2011/11/22]
\begin{document}
  \DocInput{hypcap.dtx}%
\end{document}
%</driver>
% \fi
%
% \CheckSum{174}
%
% \CharacterTable
%  {Upper-case    \A\B\C\D\E\F\G\H\I\J\K\L\M\N\O\P\Q\R\S\T\U\V\W\X\Y\Z
%   Lower-case    \a\b\c\d\e\f\g\h\i\j\k\l\m\n\o\p\q\r\s\t\u\v\w\x\y\z
%   Digits        \0\1\2\3\4\5\6\7\8\9
%   Exclamation   \!     Double quote  \"     Hash (number) \#
%   Dollar        \$     Percent       \%     Ampersand     \&
%   Acute accent  \'     Left paren    \(     Right paren   \)
%   Asterisk      \*     Plus          \+     Comma         \,
%   Minus         \-     Point         \.     Solidus       \/
%   Colon         \:     Semicolon     \;     Less than     \<
%   Equals        \=     Greater than  \>     Question mark \?
%   Commercial at \@     Left bracket  \[     Backslash     \\
%   Right bracket \]     Circumflex    \^     Underscore    \_
%   Grave accent  \`     Left brace    \{     Vertical bar  \|
%   Right brace   \}     Tilde         \~}
%
% \GetFileInfo{hypcap.drv}
%
% \title{The \xpackage{hypcap} package}
% \date{2011/02/16 v1.11}
% \author{Heiko Oberdiek\\\xemail{heiko.oberdiek at googlemail.com}}
%
% \maketitle
%
% \begin{abstract}
% This package tries a solution of the problem with
% hyperref, that links to floats points below the
% caption and not at the beginning of the float.
% Therefore this package divides the task into two
% part, the link setting with \cs{capstart} or
% automatically at the beginning of a float and
% the rest in the \cs{caption} command.
% \end{abstract}
%
% \tableofcontents
%
% \newpage
% \section{Usage}
%    The package \Package{hypcap} requires that \Package{hyperref}
%    is loaded first:
%    \begin{quote}
%      |\usepackage[...]{hyperref}|\\
%      |\usepackage[...]{hypcap}|
%    \end{quote}
%
% \subsection{Package options}
%    The names of the four float environments |figure|, |figure*|,
%    |table|, or |table*| can be used as option. Then the package
%    redefines the environment in order to insert \cmd{\capstart}
%    (see below) in the beginning of the environment automatically.
%
%    Option |all| enables the redefinitions of all four
%    float environments. For other environments see
%    the user command \cmd{\hypcapredef}.
%
% \subsection{User commands}
%    \begin{description}
%    \item[\cmd{\capstart}:]\DescribeMacro{\capstart}
%      First this command increments the counter (\cmd{\@captype}).
%      Then it makes an anchor for package \Package{hyperref}.
%      At last \cmd{\caption} is redefined to remove the
%      anchor setting part from \Package{hyperref}'s \cmd{\caption}.
%
%      The package expects the following structure of a float
%      environment:
%      \begin{quote}
%        |\begin{|\textit{|float|}|}...|\\
%        |\capstart|\\
%        |...|\\
%        |\caption{...}|\\
%        |...|\\
%        |\end{|\textit{|float|}|}|
%      \end{quote}
%      There can be several \cmd{\caption} commands. For these
%      you need \cmd{\capstart} again:
%      \begin{quote}
%        |\capstart ... \caption... \capstart ... \caption...|
%      \end{quote}
%      And the \cmd{\caption} command itself can be put in a group.
%
%      With the options, described above, the extra
%      writing of \cmd{\capstart} can be avoided.
%      Consequently, there must be a \cmd{\caption} in every
%      environment of this type, specified by the option.
%      If you want to use more than one \cmd{\caption} in this
%      environment, you have to state \cmd{\capstart} again.
%    \item[\cmd{\hypcapspace}:]\DescribeMacro{\hypcapspace}
%      Because it looks poor, if the link points exactly at top of
%      the figure, there is additional space: \cmd{\hypcapspace},
%      the default is |0.5\baselineskip|, examples:
%      \begin{quote}
%        |\renewcommand{\hypcapspace}{0pt}| removes the space\\
%        |\renewcommand{\hypcapspace}{1pt}| sets a fix value
%      \end{quote}
%    \item[\cmd{\hypcapredef}:]\DescribeMacro{\hypcapredef}
%      If there are other float environments, that
%      should automatically execute \cmd{\capstart}, then
%      a redefinition with \cmd{\hypcapredef} can be tried:
%      \begin{quote}
%        |\hypcapredef{myfloat}|
%      \end{quote}
%      Only environments with one optional parameter are
%      supported.
%    \item[\cs{capstartfalse}, \cs{capstarttrue}:]^^A
%      \DescribeMacro{\capstartfalse}^^A
%      \DescribeMacro{\capstarttrue}^^A
%      Since 2008/09/08 v1.10.\\
%      They disable and enable \cs{capstart}. They can be used to
%      cancel the effect of a redefined float environment.
%      Example:
%\begin{quote}
%\catcode`Y=\active
%\def Y#1Z{\textit{\cs{#1}}}
%\begin{verbatim}
%\documentclass{article}
%\usepackage{hyperref}
%\usepackage[figure]{hypcap}[2008/09/08]
%
%\begin{document}
%  \section{Hello World}
%  \begin{figure}
%    \caption{Figure with caption A}
%  \end{figure}
%  YcapstartfalseZ
%  \begin{figure}
%    Figure without caption
%  \end{figure}
%  YcapstarttrueZ
%  \begin{figure}
%    \caption{Figure with caption B}
%  \end{figure}
%\end{document}
%\end{verbatim}
%\end{quote}
%    \end{description}
%
% \subsection{Limitations}
%    \begin{itemize}
%    \item Packages that redefine \cmd{\caption} or
%          \cmd{\@caption}.
%    \end{itemize}
%
% \StopEventually{
% }
%
% \section{Implementation}
%    \begin{macrocode}
%<*package>
%    \end{macrocode}
%    Package identification.
%    \begin{macrocode}
\NeedsTeXFormat{LaTeX2e}
\ProvidesPackage{hypcap}%
  [2011/02/16 v1.11 Adjusting the anchors of captions (HO)]
%    \end{macrocode}
%
%    For unique command names this package uses |hc@| as prefix
%    for internal command names.
%
%    First we check, if package \Package{hyperref} is loaded:
%    \begin{macrocode}
\@ifundefined{hyper@@anchor}{%
  \PackageError{hypcap}{You have to load 'hyperref' first}\@ehc
  \endinput
}{}
%    \end{macrocode}
%
%    \begin{macrocode}
\RequirePackage{letltxmacro}[2008/06/24]
%    \end{macrocode}
%
%    \begin{macro}{\hc@org@caption}
%    Save the original meaning of \cmd{\caption}:
%    \begin{macrocode}
\newcommand*\hc@org@caption{}
\let\hc@org@caption\caption
%    \end{macrocode}
%    \end{macro}
%
%    \newcommand*{\BeginMacro}[1]{\expandafter\DoBeginMacro\csname#1\endcsname}
%    \newcommand*{\DoBeginMacro}[1]{\begin{macro}{#1}}
%    \newcommand*{\Cmd}[1]{\expandafter\cmd\csname#1\endcsname}
%    \BeginMacro{if@capstart}
%     The switch \Cmd{if@capstart} helps to detect
%     \cmd{\capstart} commands with missing \cmd{\caption} macros.
%     Because \cmd{\caption} can occur inside a group, assignments
%     to the switch have to be made global.
%    \begin{macrocode}
\newif\if@capstart
%    \end{macrocode}
%    \end{macro}
%
%    \begin{macro}{\hypcapspace}
%    The anchor is raised.by \cmd{\hypcapspace}.
%    \begin{macrocode}
\newcommand*\hypcapspace{.5\baselineskip}
%    \end{macrocode}
%    \end{macro}
%
%    \begin{macro}{\ifcapstart}
%    \begin{macrocode}
\newif\ifcapstart
\capstarttrue
%    \end{macrocode}
%    \end{macro}
%
%    \begin{macro}{\capstart}
%     The macro \cmd{\capstart} contains the first part of
%     the \cmd{\caption} command: Incrementing the counter
%     and setting the anchor.
%    \begin{macrocode}
\newcommand*\capstart{%
  \ifcapstart
    \H@refstepcounter\@captype % first part of caption
    \hyper@makecurrent\@captype
    \global\let\hc@currentHref\@currentHref
    \vspace*{-\hypcapspace}%
    \begingroup
      \let\leavevmode\relax
      \hyper@@anchor\@currentHref\relax
    \endgroup
    \vspace*{\hypcapspace}%
    \hc@hyperref{\let\caption\hc@caption}%
    \global\@capstarttrue
    \global\advance\csname c@\@captype\endcsname\m@ne
  \fi
}
%    \end{macrocode}
%    \end{macro}
%
%    \begin{macrocode}
\@ifpackagelater{hyperref}{2007/04/09}{%
  \let\hc@hyperref\@gobble
}{%
  \let\hc@hyperref\@firstofone
}
%    \end{macrocode}
%
%    \begin{macro}{\hc@caption}
%    The new \cmd{\caption} command without the first part
%    is defined in the macro \cmd{\hc@caption}.
%    \begin{macrocode}
\def\hc@caption{%
  \global\advance\csname c@\@captype\endcsname\@ne
  \@dblarg{\hc@@caption\@captype}%
}
%    \end{macrocode}
%    \end{macro}
%
%    \begin{macro}{\hc@@caption}
%    This is a copy of package \Package{hyperref}'s \cmd{\@caption}
%    macro without making the anchor, because this is already done
%    in \cmd{\capstart}.
%    \begin{macrocode}
\long\def\hc@@caption#1[#2]#3{%
  \let\caption\hc@org@caption
  \global\@capstartfalse
  \ifHy@hypertexnames
    \hyper@makecurrent\@captype
  \else
    \global\let\@currentHref\hc@currentHref
  \fi
  \par\addcontentsline{%
    \csname ext@#1\endcsname}{#1}{%
    \protect\numberline{%
      \csname the#1\endcsname
    }{\ignorespaces #2}%
  }%
  \begingroup
    \@parboxrestore
    \normalsize
    \@makecaption{\csname fnum@#1\endcsname}{%
      \ignorespaces#3%
    }%
    \par
  \endgroup
}
%    \end{macrocode}
%    \end{macro}
%
%    \begin{macro}{\hypcapredef}
%    The macro \cmd{\hypcapredef} prepares the call of
%    \cmd{\hc@redef} that will redefine the environment
%    that is given in the argument.
%    \begin{macrocode}
\def\hypcapredef#1{%
  \expandafter\hc@redef\csname hc@org#1\expandafter\endcsname
                       \csname hc@orgend#1\expandafter\endcsname
                       \expandafter{#1}%
}
%    \end{macrocode}
%    \end{macro}
%    \begin{macro}{\hc@redef}
%    The old meaning of the environment is saved. Then
%    \cmd{\capstart} is appended in the begin part.
%    The end part contains a check
%    that produces an error message in case of \cmd{\capstart}
%    without \cmd{\capstart} (\cmd{\capstart} has incremented
%    the counter).
%    \begin{macrocode}
\def\hc@redef#1#2#3{%
  \newcommand#1{}%
  \expandafter\LetLtxMacro\expandafter#1\csname#3\endcsname
  \expandafter\LetLtxMacro\expandafter#2\csname end#3\endcsname
  \renewenvironment*{#3}[1][]{%
    \ifx\\##1\\%
      #1\relax
    \else
      #1[##1]% hash-ok (compatibility for float)
    \fi
    \capstart
  }{%
    \if@capstart
      \PackageError{hypcap}{You have forgotten to use \string\caption}%
      \global\@capstartfalse
    \else
    \fi
    #2%
  }%
}
%    \end{macrocode}
%    \end{macro}
%
%     At last the options are defined and processed.
%    \begin{macrocode}
\DeclareOption{figure}{\hypcapredef{\CurrentOption}}
\DeclareOption{figure*}{\hypcapredef{\CurrentOption}}
\DeclareOption{table}{\hypcapredef{\CurrentOption}}
\DeclareOption{table*}{\hypcapredef{\CurrentOption}}
\DeclareOption{all}{%
  \hypcapredef{figure}%
  \hypcapredef{figure*}%
  \hypcapredef{table}%
  \hypcapredef{table*}%
}
\ProcessOptions\relax
%    \end{macrocode}
%
%    \begin{macrocode}
%</package>
%    \end{macrocode}
%
% \section{Installation}
%
% \subsection{Download}
%
% \paragraph{Package.} This package is available on
% CTAN\footnote{\url{ftp://ftp.ctan.org/tex-archive/}}:
% \begin{description}
% \item[\CTAN{macros/latex/contrib/oberdiek/hypcap.dtx}] The source file.
% \item[\CTAN{macros/latex/contrib/oberdiek/hypcap.pdf}] Documentation.
% \end{description}
%
%
% \paragraph{Bundle.} All the packages of the bundle `oberdiek'
% are also available in a TDS compliant ZIP archive. There
% the packages are already unpacked and the documentation files
% are generated. The files and directories obey the TDS standard.
% \begin{description}
% \item[\CTAN{install/macros/latex/contrib/oberdiek.tds.zip}]
% \end{description}
% \emph{TDS} refers to the standard ``A Directory Structure
% for \TeX\ Files'' (\CTAN{tds/tds.pdf}). Directories
% with \xfile{texmf} in their name are usually organized this way.
%
% \subsection{Bundle installation}
%
% \paragraph{Unpacking.} Unpack the \xfile{oberdiek.tds.zip} in the
% TDS tree (also known as \xfile{texmf} tree) of your choice.
% Example (linux):
% \begin{quote}
%   |unzip oberdiek.tds.zip -d ~/texmf|
% \end{quote}
%
% \paragraph{Script installation.}
% Check the directory \xfile{TDS:scripts/oberdiek/} for
% scripts that need further installation steps.
% Package \xpackage{attachfile2} comes with the Perl script
% \xfile{pdfatfi.pl} that should be installed in such a way
% that it can be called as \texttt{pdfatfi}.
% Example (linux):
% \begin{quote}
%   |chmod +x scripts/oberdiek/pdfatfi.pl|\\
%   |cp scripts/oberdiek/pdfatfi.pl /usr/local/bin/|
% \end{quote}
%
% \subsection{Package installation}
%
% \paragraph{Unpacking.} The \xfile{.dtx} file is a self-extracting
% \docstrip\ archive. The files are extracted by running the
% \xfile{.dtx} through \plainTeX:
% \begin{quote}
%   \verb|tex hypcap.dtx|
% \end{quote}
%
% \paragraph{TDS.} Now the different files must be moved into
% the different directories in your installation TDS tree
% (also known as \xfile{texmf} tree):
% \begin{quote}
% \def\t{^^A
% \begin{tabular}{@{}>{\ttfamily}l@{ $\rightarrow$ }>{\ttfamily}l@{}}
%   hypcap.sty & tex/latex/oberdiek/hypcap.sty\\
%   hypcap.pdf & doc/latex/oberdiek/hypcap.pdf\\
%   hypcap.dtx & source/latex/oberdiek/hypcap.dtx\\
% \end{tabular}^^A
% }^^A
% \sbox0{\t}^^A
% \ifdim\wd0>\linewidth
%   \begingroup
%     \advance\linewidth by\leftmargin
%     \advance\linewidth by\rightmargin
%   \edef\x{\endgroup
%     \def\noexpand\lw{\the\linewidth}^^A
%   }\x
%   \def\lwbox{^^A
%     \leavevmode
%     \hbox to \linewidth{^^A
%       \kern-\leftmargin\relax
%       \hss
%       \usebox0
%       \hss
%       \kern-\rightmargin\relax
%     }^^A
%   }^^A
%   \ifdim\wd0>\lw
%     \sbox0{\small\t}^^A
%     \ifdim\wd0>\linewidth
%       \ifdim\wd0>\lw
%         \sbox0{\footnotesize\t}^^A
%         \ifdim\wd0>\linewidth
%           \ifdim\wd0>\lw
%             \sbox0{\scriptsize\t}^^A
%             \ifdim\wd0>\linewidth
%               \ifdim\wd0>\lw
%                 \sbox0{\tiny\t}^^A
%                 \ifdim\wd0>\linewidth
%                   \lwbox
%                 \else
%                   \usebox0
%                 \fi
%               \else
%                 \lwbox
%               \fi
%             \else
%               \usebox0
%             \fi
%           \else
%             \lwbox
%           \fi
%         \else
%           \usebox0
%         \fi
%       \else
%         \lwbox
%       \fi
%     \else
%       \usebox0
%     \fi
%   \else
%     \lwbox
%   \fi
% \else
%   \usebox0
% \fi
% \end{quote}
% If you have a \xfile{docstrip.cfg} that configures and enables \docstrip's
% TDS installing feature, then some files can already be in the right
% place, see the documentation of \docstrip.
%
% \subsection{Refresh file name databases}
%
% If your \TeX~distribution
% (\teTeX, \mikTeX, \dots) relies on file name databases, you must refresh
% these. For example, \teTeX\ users run \verb|texhash| or
% \verb|mktexlsr|.
%
% \subsection{Some details for the interested}
%
% \paragraph{Attached source.}
%
% The PDF documentation on CTAN also includes the
% \xfile{.dtx} source file. It can be extracted by
% AcrobatReader 6 or higher. Another option is \textsf{pdftk},
% e.g. unpack the file into the current directory:
% \begin{quote}
%   \verb|pdftk hypcap.pdf unpack_files output .|
% \end{quote}
%
% \paragraph{Unpacking with \LaTeX.}
% The \xfile{.dtx} chooses its action depending on the format:
% \begin{description}
% \item[\plainTeX:] Run \docstrip\ and extract the files.
% \item[\LaTeX:] Generate the documentation.
% \end{description}
% If you insist on using \LaTeX\ for \docstrip\ (really,
% \docstrip\ does not need \LaTeX), then inform the autodetect routine
% about your intention:
% \begin{quote}
%   \verb|latex \let\install=y\input{hypcap.dtx}|
% \end{quote}
% Do not forget to quote the argument according to the demands
% of your shell.
%
% \paragraph{Generating the documentation.}
% You can use both the \xfile{.dtx} or the \xfile{.drv} to generate
% the documentation. The process can be configured by the
% configuration file \xfile{ltxdoc.cfg}. For instance, put this
% line into this file, if you want to have A4 as paper format:
% \begin{quote}
%   \verb|\PassOptionsToClass{a4paper}{article}|
% \end{quote}
% An example follows how to generate the
% documentation with pdf\LaTeX:
% \begin{quote}
%\begin{verbatim}
%pdflatex hypcap.dtx
%makeindex -s gind.ist hypcap.idx
%pdflatex hypcap.dtx
%makeindex -s gind.ist hypcap.idx
%pdflatex hypcap.dtx
%\end{verbatim}
% \end{quote}
%
% \section{Catalogue}
%
% The following XML file can be used as source for the
% \href{http://mirror.ctan.org/help/Catalogue/catalogue.html}{\TeX\ Catalogue}.
% The elements \texttt{caption} and \texttt{description} are imported
% from the original XML file from the Catalogue.
% The name of the XML file in the Catalogue is \xfile{hypcap.xml}.
%    \begin{macrocode}
%<*catalogue>
<?xml version='1.0' encoding='us-ascii'?>
<!DOCTYPE entry SYSTEM 'catalogue.dtd'>
<entry datestamp='$Date$' modifier='$Author$' id='hypcap'>
  <name>hypcap</name>
  <caption>Adjusting the anchors of captions.</caption>
  <authorref id='auth:oberdiek'/>
  <copyright owner='Heiko Oberdiek' year='1999-2001,2006-2008,2011'/>
  <license type='lppl1.3'/>
  <version number='1.11'/>
  <description>
    The package offers a solution to the problem that when you link to
    a float using <xref refid='hyperref'>hyperref</xref>, the link
    anchors to below the float's caption, rather than the beginning of
    the float.
    <p/>
    Hypcap defines a separate \capstart command, which you put where
    you want links to end; you should have a \capstart command for each
    \caption command.  Package options can be used to auto-insert a
    \capstart at the start of a float environment.
    <p/>
    The package is part of the <xref refid='oberdiek'>oberdiek</xref> bundle.
  </description>
  <documentation details='Package documentation'
      href='ctan:/macros/latex/contrib/oberdiek/hypcap.pdf'/>
  <ctan file='true' path='/macros/latex/contrib/oberdiek/hypcap.dtx'/>
  <miktex location='oberdiek'/>
  <texlive location='oberdiek'/>
  <install path='/macros/latex/contrib/oberdiek/oberdiek.tds.zip'/>
</entry>
%</catalogue>
%    \end{macrocode}
%
% \begin{History}
%   \begin{Version}{1999/02/13 v1.0}
%   \item
%     A beginning version, published in newsgroup \xnewsgroup{comp.text.tex}:\\
%     \URL{``\link{Re: \xpackage{hyperref} and figures}''}^^A
%     {http://groups.google.com/group/comp.text.tex/msg/5c9b47b001a9379c}
%   \end{Version}
%   \begin{Version}{2000/08/14 v1.1}
%   \item
%     Global assignments of \cs{if@capstart}
%     in order to allow \cs{caption} in groups.
%   \item
%     Option \texttt{all} added.
%   \end{Version}
%   \begin{Version}{2000/09/07 v1.2}
%   \item
%     Package in dtx format.
%   \end{Version}
%   \begin{Version}{2001/08/27 v1.3}
%   \item
%     Bug fix with hyperref's pdfmark driver\\
%     (\cs{leavevmode} in
%     \cs{hyper@@anchor}/\cs{pdf@rect}).
%   \end{Version}
%   \begin{Version}{2001/09/06 v1.4}
%   \item
%     Small fixes in the dtx file.
%   \end{Version}
%   \begin{Version}{2006/02/20 v1.5}
%   \item
%     Code is not changed.
%   \item
%     New DTX framework.
%   \end{Version}
%   \begin{Version}{2007/02/19 v1.6}
%   \item
%     Fix for hypertexnames=false.
%   \end{Version}
%   \begin{Version}{2007/04/09 v1.7}
%   \item
%     Stuff in \cs{caption} moved to \xpackage{hyperref}.
%     This avoids redefinitions of \cs{caption} and \cs{@caption}
%     (idea of Axel Sommerfeldt).
%   \item
%     Fix for subfigure (Marco Kuhlmann, Amilcar do Carmo Lucas).
%   \end{Version}
%   \begin{Version}{2008/04/14 v1.8}
%   \item
%     \cs{hc@redef} fixed to get package \xpackage{float} work
%     (Axel Sommerfeldt).
%   \end{Version}
%   \begin{Version}{2008/08/11 v1.9}
%   \item
%     Code is not changed.
%   \item
%     URLs updated.
%   \end{Version}
%   \begin{Version}{2008/09/08 v1.10}
%   \item
%     \cs{capstartfalse} and \cs{capstarttrue} added.
%   \end{Version}
%   \begin{Version}{2011/02/16 v1.11}
%   \item
%     \cs{hc@redef} fixed by using package \xpackage{letltxmacro}.
%   \end{Version}
% \end{History}
%
% \PrintIndex
%
% \Finale
\endinput

%        (quote the arguments according to the demands of your shell)
%
% Documentation:
%    (a) If hypcap.drv is present:
%           latex hypcap.drv
%    (b) Without hypcap.drv:
%           latex hypcap.dtx; ...
%    The class ltxdoc loads the configuration file ltxdoc.cfg
%    if available. Here you can specify further options, e.g.
%    use A4 as paper format:
%       \PassOptionsToClass{a4paper}{article}
%
%    Programm calls to get the documentation (example):
%       pdflatex hypcap.dtx
%       makeindex -s gind.ist hypcap.idx
%       pdflatex hypcap.dtx
%       makeindex -s gind.ist hypcap.idx
%       pdflatex hypcap.dtx
%
% Installation:
%    TDS:tex/latex/oberdiek/hypcap.sty
%    TDS:doc/latex/oberdiek/hypcap.pdf
%    TDS:source/latex/oberdiek/hypcap.dtx
%
%<*ignore>
\begingroup
  \catcode123=1 %
  \catcode125=2 %
  \def\x{LaTeX2e}%
\expandafter\endgroup
\ifcase 0\ifx\install y1\fi\expandafter
         \ifx\csname processbatchFile\endcsname\relax\else1\fi
         \ifx\fmtname\x\else 1\fi\relax
\else\csname fi\endcsname
%</ignore>
%<*install>
\input docstrip.tex
\Msg{************************************************************************}
\Msg{* Installation}
\Msg{* Package: hypcap 2011/02/16 v1.11 Adjusting the anchors of captions (HO)}
\Msg{************************************************************************}

\keepsilent
\askforoverwritefalse

\let\MetaPrefix\relax
\preamble

This is a generated file.

Project: hypcap
Version: 2011/02/16 v1.11

Copyright (C) 1999-2001, 2006-2008, 2011 by
   Heiko Oberdiek <heiko.oberdiek at googlemail.com>

This work may be distributed and/or modified under the
conditions of the LaTeX Project Public License, either
version 1.3c of this license or (at your option) any later
version. This version of this license is in
   http://www.latex-project.org/lppl/lppl-1-3c.txt
and the latest version of this license is in
   http://www.latex-project.org/lppl.txt
and version 1.3 or later is part of all distributions of
LaTeX version 2005/12/01 or later.

This work has the LPPL maintenance status "maintained".

This Current Maintainer of this work is Heiko Oberdiek.

This work consists of the main source file hypcap.dtx
and the derived files
   hypcap.sty, hypcap.pdf, hypcap.ins, hypcap.drv.

\endpreamble
\let\MetaPrefix\DoubleperCent

\generate{%
  \file{hypcap.ins}{\from{hypcap.dtx}{install}}%
  \file{hypcap.drv}{\from{hypcap.dtx}{driver}}%
  \usedir{tex/latex/oberdiek}%
  \file{hypcap.sty}{\from{hypcap.dtx}{package}}%
  \nopreamble
  \nopostamble
  \usedir{source/latex/oberdiek/catalogue}%
  \file{hypcap.xml}{\from{hypcap.dtx}{catalogue}}%
}

\catcode32=13\relax% active space
\let =\space%
\Msg{************************************************************************}
\Msg{*}
\Msg{* To finish the installation you have to move the following}
\Msg{* file into a directory searched by TeX:}
\Msg{*}
\Msg{*     hypcap.sty}
\Msg{*}
\Msg{* To produce the documentation run the file `hypcap.drv'}
\Msg{* through LaTeX.}
\Msg{*}
\Msg{* Happy TeXing!}
\Msg{*}
\Msg{************************************************************************}

\endbatchfile
%</install>
%<*ignore>
\fi
%</ignore>
%<*driver>
\NeedsTeXFormat{LaTeX2e}
\ProvidesFile{hypcap.drv}%
  [2011/02/16 v1.11 Adjusting the anchors of captions (HO)]%
\documentclass{ltxdoc}
\usepackage{holtxdoc}[2011/11/22]
\begin{document}
  \DocInput{hypcap.dtx}%
\end{document}
%</driver>
% \fi
%
% \CheckSum{174}
%
% \CharacterTable
%  {Upper-case    \A\B\C\D\E\F\G\H\I\J\K\L\M\N\O\P\Q\R\S\T\U\V\W\X\Y\Z
%   Lower-case    \a\b\c\d\e\f\g\h\i\j\k\l\m\n\o\p\q\r\s\t\u\v\w\x\y\z
%   Digits        \0\1\2\3\4\5\6\7\8\9
%   Exclamation   \!     Double quote  \"     Hash (number) \#
%   Dollar        \$     Percent       \%     Ampersand     \&
%   Acute accent  \'     Left paren    \(     Right paren   \)
%   Asterisk      \*     Plus          \+     Comma         \,
%   Minus         \-     Point         \.     Solidus       \/
%   Colon         \:     Semicolon     \;     Less than     \<
%   Equals        \=     Greater than  \>     Question mark \?
%   Commercial at \@     Left bracket  \[     Backslash     \\
%   Right bracket \]     Circumflex    \^     Underscore    \_
%   Grave accent  \`     Left brace    \{     Vertical bar  \|
%   Right brace   \}     Tilde         \~}
%
% \GetFileInfo{hypcap.drv}
%
% \title{The \xpackage{hypcap} package}
% \date{2011/02/16 v1.11}
% \author{Heiko Oberdiek\\\xemail{heiko.oberdiek at googlemail.com}}
%
% \maketitle
%
% \begin{abstract}
% This package tries a solution of the problem with
% hyperref, that links to floats points below the
% caption and not at the beginning of the float.
% Therefore this package divides the task into two
% part, the link setting with \cs{capstart} or
% automatically at the beginning of a float and
% the rest in the \cs{caption} command.
% \end{abstract}
%
% \tableofcontents
%
% \newpage
% \section{Usage}
%    The package \Package{hypcap} requires that \Package{hyperref}
%    is loaded first:
%    \begin{quote}
%      |\usepackage[...]{hyperref}|\\
%      |\usepackage[...]{hypcap}|
%    \end{quote}
%
% \subsection{Package options}
%    The names of the four float environments |figure|, |figure*|,
%    |table|, or |table*| can be used as option. Then the package
%    redefines the environment in order to insert \cmd{\capstart}
%    (see below) in the beginning of the environment automatically.
%
%    Option |all| enables the redefinitions of all four
%    float environments. For other environments see
%    the user command \cmd{\hypcapredef}.
%
% \subsection{User commands}
%    \begin{description}
%    \item[\cmd{\capstart}:]\DescribeMacro{\capstart}
%      First this command increments the counter (\cmd{\@captype}).
%      Then it makes an anchor for package \Package{hyperref}.
%      At last \cmd{\caption} is redefined to remove the
%      anchor setting part from \Package{hyperref}'s \cmd{\caption}.
%
%      The package expects the following structure of a float
%      environment:
%      \begin{quote}
%        |\begin{|\textit{|float|}|}...|\\
%        |\capstart|\\
%        |...|\\
%        |\caption{...}|\\
%        |...|\\
%        |\end{|\textit{|float|}|}|
%      \end{quote}
%      There can be several \cmd{\caption} commands. For these
%      you need \cmd{\capstart} again:
%      \begin{quote}
%        |\capstart ... \caption... \capstart ... \caption...|
%      \end{quote}
%      And the \cmd{\caption} command itself can be put in a group.
%
%      With the options, described above, the extra
%      writing of \cmd{\capstart} can be avoided.
%      Consequently, there must be a \cmd{\caption} in every
%      environment of this type, specified by the option.
%      If you want to use more than one \cmd{\caption} in this
%      environment, you have to state \cmd{\capstart} again.
%    \item[\cmd{\hypcapspace}:]\DescribeMacro{\hypcapspace}
%      Because it looks poor, if the link points exactly at top of
%      the figure, there is additional space: \cmd{\hypcapspace},
%      the default is |0.5\baselineskip|, examples:
%      \begin{quote}
%        |\renewcommand{\hypcapspace}{0pt}| removes the space\\
%        |\renewcommand{\hypcapspace}{1pt}| sets a fix value
%      \end{quote}
%    \item[\cmd{\hypcapredef}:]\DescribeMacro{\hypcapredef}
%      If there are other float environments, that
%      should automatically execute \cmd{\capstart}, then
%      a redefinition with \cmd{\hypcapredef} can be tried:
%      \begin{quote}
%        |\hypcapredef{myfloat}|
%      \end{quote}
%      Only environments with one optional parameter are
%      supported.
%    \item[\cs{capstartfalse}, \cs{capstarttrue}:]^^A
%      \DescribeMacro{\capstartfalse}^^A
%      \DescribeMacro{\capstarttrue}^^A
%      Since 2008/09/08 v1.10.\\
%      They disable and enable \cs{capstart}. They can be used to
%      cancel the effect of a redefined float environment.
%      Example:
%\begin{quote}
%\catcode`Y=\active
%\def Y#1Z{\textit{\cs{#1}}}
%\begin{verbatim}
%\documentclass{article}
%\usepackage{hyperref}
%\usepackage[figure]{hypcap}[2008/09/08]
%
%\begin{document}
%  \section{Hello World}
%  \begin{figure}
%    \caption{Figure with caption A}
%  \end{figure}
%  YcapstartfalseZ
%  \begin{figure}
%    Figure without caption
%  \end{figure}
%  YcapstarttrueZ
%  \begin{figure}
%    \caption{Figure with caption B}
%  \end{figure}
%\end{document}
%\end{verbatim}
%\end{quote}
%    \end{description}
%
% \subsection{Limitations}
%    \begin{itemize}
%    \item Packages that redefine \cmd{\caption} or
%          \cmd{\@caption}.
%    \end{itemize}
%
% \StopEventually{
% }
%
% \section{Implementation}
%    \begin{macrocode}
%<*package>
%    \end{macrocode}
%    Package identification.
%    \begin{macrocode}
\NeedsTeXFormat{LaTeX2e}
\ProvidesPackage{hypcap}%
  [2011/02/16 v1.11 Adjusting the anchors of captions (HO)]
%    \end{macrocode}
%
%    For unique command names this package uses |hc@| as prefix
%    for internal command names.
%
%    First we check, if package \Package{hyperref} is loaded:
%    \begin{macrocode}
\@ifundefined{hyper@@anchor}{%
  \PackageError{hypcap}{You have to load 'hyperref' first}\@ehc
  \endinput
}{}
%    \end{macrocode}
%
%    \begin{macrocode}
\RequirePackage{letltxmacro}[2008/06/24]
%    \end{macrocode}
%
%    \begin{macro}{\hc@org@caption}
%    Save the original meaning of \cmd{\caption}:
%    \begin{macrocode}
\newcommand*\hc@org@caption{}
\let\hc@org@caption\caption
%    \end{macrocode}
%    \end{macro}
%
%    \newcommand*{\BeginMacro}[1]{\expandafter\DoBeginMacro\csname#1\endcsname}
%    \newcommand*{\DoBeginMacro}[1]{\begin{macro}{#1}}
%    \newcommand*{\Cmd}[1]{\expandafter\cmd\csname#1\endcsname}
%    \BeginMacro{if@capstart}
%     The switch \Cmd{if@capstart} helps to detect
%     \cmd{\capstart} commands with missing \cmd{\caption} macros.
%     Because \cmd{\caption} can occur inside a group, assignments
%     to the switch have to be made global.
%    \begin{macrocode}
\newif\if@capstart
%    \end{macrocode}
%    \end{macro}
%
%    \begin{macro}{\hypcapspace}
%    The anchor is raised.by \cmd{\hypcapspace}.
%    \begin{macrocode}
\newcommand*\hypcapspace{.5\baselineskip}
%    \end{macrocode}
%    \end{macro}
%
%    \begin{macro}{\ifcapstart}
%    \begin{macrocode}
\newif\ifcapstart
\capstarttrue
%    \end{macrocode}
%    \end{macro}
%
%    \begin{macro}{\capstart}
%     The macro \cmd{\capstart} contains the first part of
%     the \cmd{\caption} command: Incrementing the counter
%     and setting the anchor.
%    \begin{macrocode}
\newcommand*\capstart{%
  \ifcapstart
    \H@refstepcounter\@captype % first part of caption
    \hyper@makecurrent\@captype
    \global\let\hc@currentHref\@currentHref
    \vspace*{-\hypcapspace}%
    \begingroup
      \let\leavevmode\relax
      \hyper@@anchor\@currentHref\relax
    \endgroup
    \vspace*{\hypcapspace}%
    \hc@hyperref{\let\caption\hc@caption}%
    \global\@capstarttrue
    \global\advance\csname c@\@captype\endcsname\m@ne
  \fi
}
%    \end{macrocode}
%    \end{macro}
%
%    \begin{macrocode}
\@ifpackagelater{hyperref}{2007/04/09}{%
  \let\hc@hyperref\@gobble
}{%
  \let\hc@hyperref\@firstofone
}
%    \end{macrocode}
%
%    \begin{macro}{\hc@caption}
%    The new \cmd{\caption} command without the first part
%    is defined in the macro \cmd{\hc@caption}.
%    \begin{macrocode}
\def\hc@caption{%
  \global\advance\csname c@\@captype\endcsname\@ne
  \@dblarg{\hc@@caption\@captype}%
}
%    \end{macrocode}
%    \end{macro}
%
%    \begin{macro}{\hc@@caption}
%    This is a copy of package \Package{hyperref}'s \cmd{\@caption}
%    macro without making the anchor, because this is already done
%    in \cmd{\capstart}.
%    \begin{macrocode}
\long\def\hc@@caption#1[#2]#3{%
  \let\caption\hc@org@caption
  \global\@capstartfalse
  \ifHy@hypertexnames
    \hyper@makecurrent\@captype
  \else
    \global\let\@currentHref\hc@currentHref
  \fi
  \par\addcontentsline{%
    \csname ext@#1\endcsname}{#1}{%
    \protect\numberline{%
      \csname the#1\endcsname
    }{\ignorespaces #2}%
  }%
  \begingroup
    \@parboxrestore
    \normalsize
    \@makecaption{\csname fnum@#1\endcsname}{%
      \ignorespaces#3%
    }%
    \par
  \endgroup
}
%    \end{macrocode}
%    \end{macro}
%
%    \begin{macro}{\hypcapredef}
%    The macro \cmd{\hypcapredef} prepares the call of
%    \cmd{\hc@redef} that will redefine the environment
%    that is given in the argument.
%    \begin{macrocode}
\def\hypcapredef#1{%
  \expandafter\hc@redef\csname hc@org#1\expandafter\endcsname
                       \csname hc@orgend#1\expandafter\endcsname
                       \expandafter{#1}%
}
%    \end{macrocode}
%    \end{macro}
%    \begin{macro}{\hc@redef}
%    The old meaning of the environment is saved. Then
%    \cmd{\capstart} is appended in the begin part.
%    The end part contains a check
%    that produces an error message in case of \cmd{\capstart}
%    without \cmd{\capstart} (\cmd{\capstart} has incremented
%    the counter).
%    \begin{macrocode}
\def\hc@redef#1#2#3{%
  \newcommand#1{}%
  \expandafter\LetLtxMacro\expandafter#1\csname#3\endcsname
  \expandafter\LetLtxMacro\expandafter#2\csname end#3\endcsname
  \renewenvironment*{#3}[1][]{%
    \ifx\\##1\\%
      #1\relax
    \else
      #1[##1]% hash-ok (compatibility for float)
    \fi
    \capstart
  }{%
    \if@capstart
      \PackageError{hypcap}{You have forgotten to use \string\caption}%
      \global\@capstartfalse
    \else
    \fi
    #2%
  }%
}
%    \end{macrocode}
%    \end{macro}
%
%     At last the options are defined and processed.
%    \begin{macrocode}
\DeclareOption{figure}{\hypcapredef{\CurrentOption}}
\DeclareOption{figure*}{\hypcapredef{\CurrentOption}}
\DeclareOption{table}{\hypcapredef{\CurrentOption}}
\DeclareOption{table*}{\hypcapredef{\CurrentOption}}
\DeclareOption{all}{%
  \hypcapredef{figure}%
  \hypcapredef{figure*}%
  \hypcapredef{table}%
  \hypcapredef{table*}%
}
\ProcessOptions\relax
%    \end{macrocode}
%
%    \begin{macrocode}
%</package>
%    \end{macrocode}
%
% \section{Installation}
%
% \subsection{Download}
%
% \paragraph{Package.} This package is available on
% CTAN\footnote{\url{ftp://ftp.ctan.org/tex-archive/}}:
% \begin{description}
% \item[\CTAN{macros/latex/contrib/oberdiek/hypcap.dtx}] The source file.
% \item[\CTAN{macros/latex/contrib/oberdiek/hypcap.pdf}] Documentation.
% \end{description}
%
%
% \paragraph{Bundle.} All the packages of the bundle `oberdiek'
% are also available in a TDS compliant ZIP archive. There
% the packages are already unpacked and the documentation files
% are generated. The files and directories obey the TDS standard.
% \begin{description}
% \item[\CTAN{install/macros/latex/contrib/oberdiek.tds.zip}]
% \end{description}
% \emph{TDS} refers to the standard ``A Directory Structure
% for \TeX\ Files'' (\CTAN{tds/tds.pdf}). Directories
% with \xfile{texmf} in their name are usually organized this way.
%
% \subsection{Bundle installation}
%
% \paragraph{Unpacking.} Unpack the \xfile{oberdiek.tds.zip} in the
% TDS tree (also known as \xfile{texmf} tree) of your choice.
% Example (linux):
% \begin{quote}
%   |unzip oberdiek.tds.zip -d ~/texmf|
% \end{quote}
%
% \paragraph{Script installation.}
% Check the directory \xfile{TDS:scripts/oberdiek/} for
% scripts that need further installation steps.
% Package \xpackage{attachfile2} comes with the Perl script
% \xfile{pdfatfi.pl} that should be installed in such a way
% that it can be called as \texttt{pdfatfi}.
% Example (linux):
% \begin{quote}
%   |chmod +x scripts/oberdiek/pdfatfi.pl|\\
%   |cp scripts/oberdiek/pdfatfi.pl /usr/local/bin/|
% \end{quote}
%
% \subsection{Package installation}
%
% \paragraph{Unpacking.} The \xfile{.dtx} file is a self-extracting
% \docstrip\ archive. The files are extracted by running the
% \xfile{.dtx} through \plainTeX:
% \begin{quote}
%   \verb|tex hypcap.dtx|
% \end{quote}
%
% \paragraph{TDS.} Now the different files must be moved into
% the different directories in your installation TDS tree
% (also known as \xfile{texmf} tree):
% \begin{quote}
% \def\t{^^A
% \begin{tabular}{@{}>{\ttfamily}l@{ $\rightarrow$ }>{\ttfamily}l@{}}
%   hypcap.sty & tex/latex/oberdiek/hypcap.sty\\
%   hypcap.pdf & doc/latex/oberdiek/hypcap.pdf\\
%   hypcap.dtx & source/latex/oberdiek/hypcap.dtx\\
% \end{tabular}^^A
% }^^A
% \sbox0{\t}^^A
% \ifdim\wd0>\linewidth
%   \begingroup
%     \advance\linewidth by\leftmargin
%     \advance\linewidth by\rightmargin
%   \edef\x{\endgroup
%     \def\noexpand\lw{\the\linewidth}^^A
%   }\x
%   \def\lwbox{^^A
%     \leavevmode
%     \hbox to \linewidth{^^A
%       \kern-\leftmargin\relax
%       \hss
%       \usebox0
%       \hss
%       \kern-\rightmargin\relax
%     }^^A
%   }^^A
%   \ifdim\wd0>\lw
%     \sbox0{\small\t}^^A
%     \ifdim\wd0>\linewidth
%       \ifdim\wd0>\lw
%         \sbox0{\footnotesize\t}^^A
%         \ifdim\wd0>\linewidth
%           \ifdim\wd0>\lw
%             \sbox0{\scriptsize\t}^^A
%             \ifdim\wd0>\linewidth
%               \ifdim\wd0>\lw
%                 \sbox0{\tiny\t}^^A
%                 \ifdim\wd0>\linewidth
%                   \lwbox
%                 \else
%                   \usebox0
%                 \fi
%               \else
%                 \lwbox
%               \fi
%             \else
%               \usebox0
%             \fi
%           \else
%             \lwbox
%           \fi
%         \else
%           \usebox0
%         \fi
%       \else
%         \lwbox
%       \fi
%     \else
%       \usebox0
%     \fi
%   \else
%     \lwbox
%   \fi
% \else
%   \usebox0
% \fi
% \end{quote}
% If you have a \xfile{docstrip.cfg} that configures and enables \docstrip's
% TDS installing feature, then some files can already be in the right
% place, see the documentation of \docstrip.
%
% \subsection{Refresh file name databases}
%
% If your \TeX~distribution
% (\teTeX, \mikTeX, \dots) relies on file name databases, you must refresh
% these. For example, \teTeX\ users run \verb|texhash| or
% \verb|mktexlsr|.
%
% \subsection{Some details for the interested}
%
% \paragraph{Attached source.}
%
% The PDF documentation on CTAN also includes the
% \xfile{.dtx} source file. It can be extracted by
% AcrobatReader 6 or higher. Another option is \textsf{pdftk},
% e.g. unpack the file into the current directory:
% \begin{quote}
%   \verb|pdftk hypcap.pdf unpack_files output .|
% \end{quote}
%
% \paragraph{Unpacking with \LaTeX.}
% The \xfile{.dtx} chooses its action depending on the format:
% \begin{description}
% \item[\plainTeX:] Run \docstrip\ and extract the files.
% \item[\LaTeX:] Generate the documentation.
% \end{description}
% If you insist on using \LaTeX\ for \docstrip\ (really,
% \docstrip\ does not need \LaTeX), then inform the autodetect routine
% about your intention:
% \begin{quote}
%   \verb|latex \let\install=y% \iffalse meta-comment
%
% File: hypcap.dtx
% Version: 2011/02/16 v1.11
% Info: Adjusting the anchors of captions
%
% Copyright (C) 1999-2001, 2006-2008, 2011 by
%    Heiko Oberdiek <heiko.oberdiek at googlemail.com>
%
% This work may be distributed and/or modified under the
% conditions of the LaTeX Project Public License, either
% version 1.3c of this license or (at your option) any later
% version. This version of this license is in
%    http://www.latex-project.org/lppl/lppl-1-3c.txt
% and the latest version of this license is in
%    http://www.latex-project.org/lppl.txt
% and version 1.3 or later is part of all distributions of
% LaTeX version 2005/12/01 or later.
%
% This work has the LPPL maintenance status "maintained".
%
% This Current Maintainer of this work is Heiko Oberdiek.
%
% This work consists of the main source file hypcap.dtx
% and the derived files
%    hypcap.sty, hypcap.pdf, hypcap.ins, hypcap.drv.
%
% Distribution:
%    CTAN:macros/latex/contrib/oberdiek/hypcap.dtx
%    CTAN:macros/latex/contrib/oberdiek/hypcap.pdf
%
% Unpacking:
%    (a) If hypcap.ins is present:
%           tex hypcap.ins
%    (b) Without hypcap.ins:
%           tex hypcap.dtx
%    (c) If you insist on using LaTeX
%           latex \let\install=y\input{hypcap.dtx}
%        (quote the arguments according to the demands of your shell)
%
% Documentation:
%    (a) If hypcap.drv is present:
%           latex hypcap.drv
%    (b) Without hypcap.drv:
%           latex hypcap.dtx; ...
%    The class ltxdoc loads the configuration file ltxdoc.cfg
%    if available. Here you can specify further options, e.g.
%    use A4 as paper format:
%       \PassOptionsToClass{a4paper}{article}
%
%    Programm calls to get the documentation (example):
%       pdflatex hypcap.dtx
%       makeindex -s gind.ist hypcap.idx
%       pdflatex hypcap.dtx
%       makeindex -s gind.ist hypcap.idx
%       pdflatex hypcap.dtx
%
% Installation:
%    TDS:tex/latex/oberdiek/hypcap.sty
%    TDS:doc/latex/oberdiek/hypcap.pdf
%    TDS:source/latex/oberdiek/hypcap.dtx
%
%<*ignore>
\begingroup
  \catcode123=1 %
  \catcode125=2 %
  \def\x{LaTeX2e}%
\expandafter\endgroup
\ifcase 0\ifx\install y1\fi\expandafter
         \ifx\csname processbatchFile\endcsname\relax\else1\fi
         \ifx\fmtname\x\else 1\fi\relax
\else\csname fi\endcsname
%</ignore>
%<*install>
\input docstrip.tex
\Msg{************************************************************************}
\Msg{* Installation}
\Msg{* Package: hypcap 2011/02/16 v1.11 Adjusting the anchors of captions (HO)}
\Msg{************************************************************************}

\keepsilent
\askforoverwritefalse

\let\MetaPrefix\relax
\preamble

This is a generated file.

Project: hypcap
Version: 2011/02/16 v1.11

Copyright (C) 1999-2001, 2006-2008, 2011 by
   Heiko Oberdiek <heiko.oberdiek at googlemail.com>

This work may be distributed and/or modified under the
conditions of the LaTeX Project Public License, either
version 1.3c of this license or (at your option) any later
version. This version of this license is in
   http://www.latex-project.org/lppl/lppl-1-3c.txt
and the latest version of this license is in
   http://www.latex-project.org/lppl.txt
and version 1.3 or later is part of all distributions of
LaTeX version 2005/12/01 or later.

This work has the LPPL maintenance status "maintained".

This Current Maintainer of this work is Heiko Oberdiek.

This work consists of the main source file hypcap.dtx
and the derived files
   hypcap.sty, hypcap.pdf, hypcap.ins, hypcap.drv.

\endpreamble
\let\MetaPrefix\DoubleperCent

\generate{%
  \file{hypcap.ins}{\from{hypcap.dtx}{install}}%
  \file{hypcap.drv}{\from{hypcap.dtx}{driver}}%
  \usedir{tex/latex/oberdiek}%
  \file{hypcap.sty}{\from{hypcap.dtx}{package}}%
  \nopreamble
  \nopostamble
  \usedir{source/latex/oberdiek/catalogue}%
  \file{hypcap.xml}{\from{hypcap.dtx}{catalogue}}%
}

\catcode32=13\relax% active space
\let =\space%
\Msg{************************************************************************}
\Msg{*}
\Msg{* To finish the installation you have to move the following}
\Msg{* file into a directory searched by TeX:}
\Msg{*}
\Msg{*     hypcap.sty}
\Msg{*}
\Msg{* To produce the documentation run the file `hypcap.drv'}
\Msg{* through LaTeX.}
\Msg{*}
\Msg{* Happy TeXing!}
\Msg{*}
\Msg{************************************************************************}

\endbatchfile
%</install>
%<*ignore>
\fi
%</ignore>
%<*driver>
\NeedsTeXFormat{LaTeX2e}
\ProvidesFile{hypcap.drv}%
  [2011/02/16 v1.11 Adjusting the anchors of captions (HO)]%
\documentclass{ltxdoc}
\usepackage{holtxdoc}[2011/11/22]
\begin{document}
  \DocInput{hypcap.dtx}%
\end{document}
%</driver>
% \fi
%
% \CheckSum{174}
%
% \CharacterTable
%  {Upper-case    \A\B\C\D\E\F\G\H\I\J\K\L\M\N\O\P\Q\R\S\T\U\V\W\X\Y\Z
%   Lower-case    \a\b\c\d\e\f\g\h\i\j\k\l\m\n\o\p\q\r\s\t\u\v\w\x\y\z
%   Digits        \0\1\2\3\4\5\6\7\8\9
%   Exclamation   \!     Double quote  \"     Hash (number) \#
%   Dollar        \$     Percent       \%     Ampersand     \&
%   Acute accent  \'     Left paren    \(     Right paren   \)
%   Asterisk      \*     Plus          \+     Comma         \,
%   Minus         \-     Point         \.     Solidus       \/
%   Colon         \:     Semicolon     \;     Less than     \<
%   Equals        \=     Greater than  \>     Question mark \?
%   Commercial at \@     Left bracket  \[     Backslash     \\
%   Right bracket \]     Circumflex    \^     Underscore    \_
%   Grave accent  \`     Left brace    \{     Vertical bar  \|
%   Right brace   \}     Tilde         \~}
%
% \GetFileInfo{hypcap.drv}
%
% \title{The \xpackage{hypcap} package}
% \date{2011/02/16 v1.11}
% \author{Heiko Oberdiek\\\xemail{heiko.oberdiek at googlemail.com}}
%
% \maketitle
%
% \begin{abstract}
% This package tries a solution of the problem with
% hyperref, that links to floats points below the
% caption and not at the beginning of the float.
% Therefore this package divides the task into two
% part, the link setting with \cs{capstart} or
% automatically at the beginning of a float and
% the rest in the \cs{caption} command.
% \end{abstract}
%
% \tableofcontents
%
% \newpage
% \section{Usage}
%    The package \Package{hypcap} requires that \Package{hyperref}
%    is loaded first:
%    \begin{quote}
%      |\usepackage[...]{hyperref}|\\
%      |\usepackage[...]{hypcap}|
%    \end{quote}
%
% \subsection{Package options}
%    The names of the four float environments |figure|, |figure*|,
%    |table|, or |table*| can be used as option. Then the package
%    redefines the environment in order to insert \cmd{\capstart}
%    (see below) in the beginning of the environment automatically.
%
%    Option |all| enables the redefinitions of all four
%    float environments. For other environments see
%    the user command \cmd{\hypcapredef}.
%
% \subsection{User commands}
%    \begin{description}
%    \item[\cmd{\capstart}:]\DescribeMacro{\capstart}
%      First this command increments the counter (\cmd{\@captype}).
%      Then it makes an anchor for package \Package{hyperref}.
%      At last \cmd{\caption} is redefined to remove the
%      anchor setting part from \Package{hyperref}'s \cmd{\caption}.
%
%      The package expects the following structure of a float
%      environment:
%      \begin{quote}
%        |\begin{|\textit{|float|}|}...|\\
%        |\capstart|\\
%        |...|\\
%        |\caption{...}|\\
%        |...|\\
%        |\end{|\textit{|float|}|}|
%      \end{quote}
%      There can be several \cmd{\caption} commands. For these
%      you need \cmd{\capstart} again:
%      \begin{quote}
%        |\capstart ... \caption... \capstart ... \caption...|
%      \end{quote}
%      And the \cmd{\caption} command itself can be put in a group.
%
%      With the options, described above, the extra
%      writing of \cmd{\capstart} can be avoided.
%      Consequently, there must be a \cmd{\caption} in every
%      environment of this type, specified by the option.
%      If you want to use more than one \cmd{\caption} in this
%      environment, you have to state \cmd{\capstart} again.
%    \item[\cmd{\hypcapspace}:]\DescribeMacro{\hypcapspace}
%      Because it looks poor, if the link points exactly at top of
%      the figure, there is additional space: \cmd{\hypcapspace},
%      the default is |0.5\baselineskip|, examples:
%      \begin{quote}
%        |\renewcommand{\hypcapspace}{0pt}| removes the space\\
%        |\renewcommand{\hypcapspace}{1pt}| sets a fix value
%      \end{quote}
%    \item[\cmd{\hypcapredef}:]\DescribeMacro{\hypcapredef}
%      If there are other float environments, that
%      should automatically execute \cmd{\capstart}, then
%      a redefinition with \cmd{\hypcapredef} can be tried:
%      \begin{quote}
%        |\hypcapredef{myfloat}|
%      \end{quote}
%      Only environments with one optional parameter are
%      supported.
%    \item[\cs{capstartfalse}, \cs{capstarttrue}:]^^A
%      \DescribeMacro{\capstartfalse}^^A
%      \DescribeMacro{\capstarttrue}^^A
%      Since 2008/09/08 v1.10.\\
%      They disable and enable \cs{capstart}. They can be used to
%      cancel the effect of a redefined float environment.
%      Example:
%\begin{quote}
%\catcode`Y=\active
%\def Y#1Z{\textit{\cs{#1}}}
%\begin{verbatim}
%\documentclass{article}
%\usepackage{hyperref}
%\usepackage[figure]{hypcap}[2008/09/08]
%
%\begin{document}
%  \section{Hello World}
%  \begin{figure}
%    \caption{Figure with caption A}
%  \end{figure}
%  YcapstartfalseZ
%  \begin{figure}
%    Figure without caption
%  \end{figure}
%  YcapstarttrueZ
%  \begin{figure}
%    \caption{Figure with caption B}
%  \end{figure}
%\end{document}
%\end{verbatim}
%\end{quote}
%    \end{description}
%
% \subsection{Limitations}
%    \begin{itemize}
%    \item Packages that redefine \cmd{\caption} or
%          \cmd{\@caption}.
%    \end{itemize}
%
% \StopEventually{
% }
%
% \section{Implementation}
%    \begin{macrocode}
%<*package>
%    \end{macrocode}
%    Package identification.
%    \begin{macrocode}
\NeedsTeXFormat{LaTeX2e}
\ProvidesPackage{hypcap}%
  [2011/02/16 v1.11 Adjusting the anchors of captions (HO)]
%    \end{macrocode}
%
%    For unique command names this package uses |hc@| as prefix
%    for internal command names.
%
%    First we check, if package \Package{hyperref} is loaded:
%    \begin{macrocode}
\@ifundefined{hyper@@anchor}{%
  \PackageError{hypcap}{You have to load 'hyperref' first}\@ehc
  \endinput
}{}
%    \end{macrocode}
%
%    \begin{macrocode}
\RequirePackage{letltxmacro}[2008/06/24]
%    \end{macrocode}
%
%    \begin{macro}{\hc@org@caption}
%    Save the original meaning of \cmd{\caption}:
%    \begin{macrocode}
\newcommand*\hc@org@caption{}
\let\hc@org@caption\caption
%    \end{macrocode}
%    \end{macro}
%
%    \newcommand*{\BeginMacro}[1]{\expandafter\DoBeginMacro\csname#1\endcsname}
%    \newcommand*{\DoBeginMacro}[1]{\begin{macro}{#1}}
%    \newcommand*{\Cmd}[1]{\expandafter\cmd\csname#1\endcsname}
%    \BeginMacro{if@capstart}
%     The switch \Cmd{if@capstart} helps to detect
%     \cmd{\capstart} commands with missing \cmd{\caption} macros.
%     Because \cmd{\caption} can occur inside a group, assignments
%     to the switch have to be made global.
%    \begin{macrocode}
\newif\if@capstart
%    \end{macrocode}
%    \end{macro}
%
%    \begin{macro}{\hypcapspace}
%    The anchor is raised.by \cmd{\hypcapspace}.
%    \begin{macrocode}
\newcommand*\hypcapspace{.5\baselineskip}
%    \end{macrocode}
%    \end{macro}
%
%    \begin{macro}{\ifcapstart}
%    \begin{macrocode}
\newif\ifcapstart
\capstarttrue
%    \end{macrocode}
%    \end{macro}
%
%    \begin{macro}{\capstart}
%     The macro \cmd{\capstart} contains the first part of
%     the \cmd{\caption} command: Incrementing the counter
%     and setting the anchor.
%    \begin{macrocode}
\newcommand*\capstart{%
  \ifcapstart
    \H@refstepcounter\@captype % first part of caption
    \hyper@makecurrent\@captype
    \global\let\hc@currentHref\@currentHref
    \vspace*{-\hypcapspace}%
    \begingroup
      \let\leavevmode\relax
      \hyper@@anchor\@currentHref\relax
    \endgroup
    \vspace*{\hypcapspace}%
    \hc@hyperref{\let\caption\hc@caption}%
    \global\@capstarttrue
    \global\advance\csname c@\@captype\endcsname\m@ne
  \fi
}
%    \end{macrocode}
%    \end{macro}
%
%    \begin{macrocode}
\@ifpackagelater{hyperref}{2007/04/09}{%
  \let\hc@hyperref\@gobble
}{%
  \let\hc@hyperref\@firstofone
}
%    \end{macrocode}
%
%    \begin{macro}{\hc@caption}
%    The new \cmd{\caption} command without the first part
%    is defined in the macro \cmd{\hc@caption}.
%    \begin{macrocode}
\def\hc@caption{%
  \global\advance\csname c@\@captype\endcsname\@ne
  \@dblarg{\hc@@caption\@captype}%
}
%    \end{macrocode}
%    \end{macro}
%
%    \begin{macro}{\hc@@caption}
%    This is a copy of package \Package{hyperref}'s \cmd{\@caption}
%    macro without making the anchor, because this is already done
%    in \cmd{\capstart}.
%    \begin{macrocode}
\long\def\hc@@caption#1[#2]#3{%
  \let\caption\hc@org@caption
  \global\@capstartfalse
  \ifHy@hypertexnames
    \hyper@makecurrent\@captype
  \else
    \global\let\@currentHref\hc@currentHref
  \fi
  \par\addcontentsline{%
    \csname ext@#1\endcsname}{#1}{%
    \protect\numberline{%
      \csname the#1\endcsname
    }{\ignorespaces #2}%
  }%
  \begingroup
    \@parboxrestore
    \normalsize
    \@makecaption{\csname fnum@#1\endcsname}{%
      \ignorespaces#3%
    }%
    \par
  \endgroup
}
%    \end{macrocode}
%    \end{macro}
%
%    \begin{macro}{\hypcapredef}
%    The macro \cmd{\hypcapredef} prepares the call of
%    \cmd{\hc@redef} that will redefine the environment
%    that is given in the argument.
%    \begin{macrocode}
\def\hypcapredef#1{%
  \expandafter\hc@redef\csname hc@org#1\expandafter\endcsname
                       \csname hc@orgend#1\expandafter\endcsname
                       \expandafter{#1}%
}
%    \end{macrocode}
%    \end{macro}
%    \begin{macro}{\hc@redef}
%    The old meaning of the environment is saved. Then
%    \cmd{\capstart} is appended in the begin part.
%    The end part contains a check
%    that produces an error message in case of \cmd{\capstart}
%    without \cmd{\capstart} (\cmd{\capstart} has incremented
%    the counter).
%    \begin{macrocode}
\def\hc@redef#1#2#3{%
  \newcommand#1{}%
  \expandafter\LetLtxMacro\expandafter#1\csname#3\endcsname
  \expandafter\LetLtxMacro\expandafter#2\csname end#3\endcsname
  \renewenvironment*{#3}[1][]{%
    \ifx\\##1\\%
      #1\relax
    \else
      #1[##1]% hash-ok (compatibility for float)
    \fi
    \capstart
  }{%
    \if@capstart
      \PackageError{hypcap}{You have forgotten to use \string\caption}%
      \global\@capstartfalse
    \else
    \fi
    #2%
  }%
}
%    \end{macrocode}
%    \end{macro}
%
%     At last the options are defined and processed.
%    \begin{macrocode}
\DeclareOption{figure}{\hypcapredef{\CurrentOption}}
\DeclareOption{figure*}{\hypcapredef{\CurrentOption}}
\DeclareOption{table}{\hypcapredef{\CurrentOption}}
\DeclareOption{table*}{\hypcapredef{\CurrentOption}}
\DeclareOption{all}{%
  \hypcapredef{figure}%
  \hypcapredef{figure*}%
  \hypcapredef{table}%
  \hypcapredef{table*}%
}
\ProcessOptions\relax
%    \end{macrocode}
%
%    \begin{macrocode}
%</package>
%    \end{macrocode}
%
% \section{Installation}
%
% \subsection{Download}
%
% \paragraph{Package.} This package is available on
% CTAN\footnote{\url{ftp://ftp.ctan.org/tex-archive/}}:
% \begin{description}
% \item[\CTAN{macros/latex/contrib/oberdiek/hypcap.dtx}] The source file.
% \item[\CTAN{macros/latex/contrib/oberdiek/hypcap.pdf}] Documentation.
% \end{description}
%
%
% \paragraph{Bundle.} All the packages of the bundle `oberdiek'
% are also available in a TDS compliant ZIP archive. There
% the packages are already unpacked and the documentation files
% are generated. The files and directories obey the TDS standard.
% \begin{description}
% \item[\CTAN{install/macros/latex/contrib/oberdiek.tds.zip}]
% \end{description}
% \emph{TDS} refers to the standard ``A Directory Structure
% for \TeX\ Files'' (\CTAN{tds/tds.pdf}). Directories
% with \xfile{texmf} in their name are usually organized this way.
%
% \subsection{Bundle installation}
%
% \paragraph{Unpacking.} Unpack the \xfile{oberdiek.tds.zip} in the
% TDS tree (also known as \xfile{texmf} tree) of your choice.
% Example (linux):
% \begin{quote}
%   |unzip oberdiek.tds.zip -d ~/texmf|
% \end{quote}
%
% \paragraph{Script installation.}
% Check the directory \xfile{TDS:scripts/oberdiek/} for
% scripts that need further installation steps.
% Package \xpackage{attachfile2} comes with the Perl script
% \xfile{pdfatfi.pl} that should be installed in such a way
% that it can be called as \texttt{pdfatfi}.
% Example (linux):
% \begin{quote}
%   |chmod +x scripts/oberdiek/pdfatfi.pl|\\
%   |cp scripts/oberdiek/pdfatfi.pl /usr/local/bin/|
% \end{quote}
%
% \subsection{Package installation}
%
% \paragraph{Unpacking.} The \xfile{.dtx} file is a self-extracting
% \docstrip\ archive. The files are extracted by running the
% \xfile{.dtx} through \plainTeX:
% \begin{quote}
%   \verb|tex hypcap.dtx|
% \end{quote}
%
% \paragraph{TDS.} Now the different files must be moved into
% the different directories in your installation TDS tree
% (also known as \xfile{texmf} tree):
% \begin{quote}
% \def\t{^^A
% \begin{tabular}{@{}>{\ttfamily}l@{ $\rightarrow$ }>{\ttfamily}l@{}}
%   hypcap.sty & tex/latex/oberdiek/hypcap.sty\\
%   hypcap.pdf & doc/latex/oberdiek/hypcap.pdf\\
%   hypcap.dtx & source/latex/oberdiek/hypcap.dtx\\
% \end{tabular}^^A
% }^^A
% \sbox0{\t}^^A
% \ifdim\wd0>\linewidth
%   \begingroup
%     \advance\linewidth by\leftmargin
%     \advance\linewidth by\rightmargin
%   \edef\x{\endgroup
%     \def\noexpand\lw{\the\linewidth}^^A
%   }\x
%   \def\lwbox{^^A
%     \leavevmode
%     \hbox to \linewidth{^^A
%       \kern-\leftmargin\relax
%       \hss
%       \usebox0
%       \hss
%       \kern-\rightmargin\relax
%     }^^A
%   }^^A
%   \ifdim\wd0>\lw
%     \sbox0{\small\t}^^A
%     \ifdim\wd0>\linewidth
%       \ifdim\wd0>\lw
%         \sbox0{\footnotesize\t}^^A
%         \ifdim\wd0>\linewidth
%           \ifdim\wd0>\lw
%             \sbox0{\scriptsize\t}^^A
%             \ifdim\wd0>\linewidth
%               \ifdim\wd0>\lw
%                 \sbox0{\tiny\t}^^A
%                 \ifdim\wd0>\linewidth
%                   \lwbox
%                 \else
%                   \usebox0
%                 \fi
%               \else
%                 \lwbox
%               \fi
%             \else
%               \usebox0
%             \fi
%           \else
%             \lwbox
%           \fi
%         \else
%           \usebox0
%         \fi
%       \else
%         \lwbox
%       \fi
%     \else
%       \usebox0
%     \fi
%   \else
%     \lwbox
%   \fi
% \else
%   \usebox0
% \fi
% \end{quote}
% If you have a \xfile{docstrip.cfg} that configures and enables \docstrip's
% TDS installing feature, then some files can already be in the right
% place, see the documentation of \docstrip.
%
% \subsection{Refresh file name databases}
%
% If your \TeX~distribution
% (\teTeX, \mikTeX, \dots) relies on file name databases, you must refresh
% these. For example, \teTeX\ users run \verb|texhash| or
% \verb|mktexlsr|.
%
% \subsection{Some details for the interested}
%
% \paragraph{Attached source.}
%
% The PDF documentation on CTAN also includes the
% \xfile{.dtx} source file. It can be extracted by
% AcrobatReader 6 or higher. Another option is \textsf{pdftk},
% e.g. unpack the file into the current directory:
% \begin{quote}
%   \verb|pdftk hypcap.pdf unpack_files output .|
% \end{quote}
%
% \paragraph{Unpacking with \LaTeX.}
% The \xfile{.dtx} chooses its action depending on the format:
% \begin{description}
% \item[\plainTeX:] Run \docstrip\ and extract the files.
% \item[\LaTeX:] Generate the documentation.
% \end{description}
% If you insist on using \LaTeX\ for \docstrip\ (really,
% \docstrip\ does not need \LaTeX), then inform the autodetect routine
% about your intention:
% \begin{quote}
%   \verb|latex \let\install=y\input{hypcap.dtx}|
% \end{quote}
% Do not forget to quote the argument according to the demands
% of your shell.
%
% \paragraph{Generating the documentation.}
% You can use both the \xfile{.dtx} or the \xfile{.drv} to generate
% the documentation. The process can be configured by the
% configuration file \xfile{ltxdoc.cfg}. For instance, put this
% line into this file, if you want to have A4 as paper format:
% \begin{quote}
%   \verb|\PassOptionsToClass{a4paper}{article}|
% \end{quote}
% An example follows how to generate the
% documentation with pdf\LaTeX:
% \begin{quote}
%\begin{verbatim}
%pdflatex hypcap.dtx
%makeindex -s gind.ist hypcap.idx
%pdflatex hypcap.dtx
%makeindex -s gind.ist hypcap.idx
%pdflatex hypcap.dtx
%\end{verbatim}
% \end{quote}
%
% \section{Catalogue}
%
% The following XML file can be used as source for the
% \href{http://mirror.ctan.org/help/Catalogue/catalogue.html}{\TeX\ Catalogue}.
% The elements \texttt{caption} and \texttt{description} are imported
% from the original XML file from the Catalogue.
% The name of the XML file in the Catalogue is \xfile{hypcap.xml}.
%    \begin{macrocode}
%<*catalogue>
<?xml version='1.0' encoding='us-ascii'?>
<!DOCTYPE entry SYSTEM 'catalogue.dtd'>
<entry datestamp='$Date$' modifier='$Author$' id='hypcap'>
  <name>hypcap</name>
  <caption>Adjusting the anchors of captions.</caption>
  <authorref id='auth:oberdiek'/>
  <copyright owner='Heiko Oberdiek' year='1999-2001,2006-2008,2011'/>
  <license type='lppl1.3'/>
  <version number='1.11'/>
  <description>
    The package offers a solution to the problem that when you link to
    a float using <xref refid='hyperref'>hyperref</xref>, the link
    anchors to below the float's caption, rather than the beginning of
    the float.
    <p/>
    Hypcap defines a separate \capstart command, which you put where
    you want links to end; you should have a \capstart command for each
    \caption command.  Package options can be used to auto-insert a
    \capstart at the start of a float environment.
    <p/>
    The package is part of the <xref refid='oberdiek'>oberdiek</xref> bundle.
  </description>
  <documentation details='Package documentation'
      href='ctan:/macros/latex/contrib/oberdiek/hypcap.pdf'/>
  <ctan file='true' path='/macros/latex/contrib/oberdiek/hypcap.dtx'/>
  <miktex location='oberdiek'/>
  <texlive location='oberdiek'/>
  <install path='/macros/latex/contrib/oberdiek/oberdiek.tds.zip'/>
</entry>
%</catalogue>
%    \end{macrocode}
%
% \begin{History}
%   \begin{Version}{1999/02/13 v1.0}
%   \item
%     A beginning version, published in newsgroup \xnewsgroup{comp.text.tex}:\\
%     \URL{``\link{Re: \xpackage{hyperref} and figures}''}^^A
%     {http://groups.google.com/group/comp.text.tex/msg/5c9b47b001a9379c}
%   \end{Version}
%   \begin{Version}{2000/08/14 v1.1}
%   \item
%     Global assignments of \cs{if@capstart}
%     in order to allow \cs{caption} in groups.
%   \item
%     Option \texttt{all} added.
%   \end{Version}
%   \begin{Version}{2000/09/07 v1.2}
%   \item
%     Package in dtx format.
%   \end{Version}
%   \begin{Version}{2001/08/27 v1.3}
%   \item
%     Bug fix with hyperref's pdfmark driver\\
%     (\cs{leavevmode} in
%     \cs{hyper@@anchor}/\cs{pdf@rect}).
%   \end{Version}
%   \begin{Version}{2001/09/06 v1.4}
%   \item
%     Small fixes in the dtx file.
%   \end{Version}
%   \begin{Version}{2006/02/20 v1.5}
%   \item
%     Code is not changed.
%   \item
%     New DTX framework.
%   \end{Version}
%   \begin{Version}{2007/02/19 v1.6}
%   \item
%     Fix for hypertexnames=false.
%   \end{Version}
%   \begin{Version}{2007/04/09 v1.7}
%   \item
%     Stuff in \cs{caption} moved to \xpackage{hyperref}.
%     This avoids redefinitions of \cs{caption} and \cs{@caption}
%     (idea of Axel Sommerfeldt).
%   \item
%     Fix for subfigure (Marco Kuhlmann, Amilcar do Carmo Lucas).
%   \end{Version}
%   \begin{Version}{2008/04/14 v1.8}
%   \item
%     \cs{hc@redef} fixed to get package \xpackage{float} work
%     (Axel Sommerfeldt).
%   \end{Version}
%   \begin{Version}{2008/08/11 v1.9}
%   \item
%     Code is not changed.
%   \item
%     URLs updated.
%   \end{Version}
%   \begin{Version}{2008/09/08 v1.10}
%   \item
%     \cs{capstartfalse} and \cs{capstarttrue} added.
%   \end{Version}
%   \begin{Version}{2011/02/16 v1.11}
%   \item
%     \cs{hc@redef} fixed by using package \xpackage{letltxmacro}.
%   \end{Version}
% \end{History}
%
% \PrintIndex
%
% \Finale
\endinput
|
% \end{quote}
% Do not forget to quote the argument according to the demands
% of your shell.
%
% \paragraph{Generating the documentation.}
% You can use both the \xfile{.dtx} or the \xfile{.drv} to generate
% the documentation. The process can be configured by the
% configuration file \xfile{ltxdoc.cfg}. For instance, put this
% line into this file, if you want to have A4 as paper format:
% \begin{quote}
%   \verb|\PassOptionsToClass{a4paper}{article}|
% \end{quote}
% An example follows how to generate the
% documentation with pdf\LaTeX:
% \begin{quote}
%\begin{verbatim}
%pdflatex hypcap.dtx
%makeindex -s gind.ist hypcap.idx
%pdflatex hypcap.dtx
%makeindex -s gind.ist hypcap.idx
%pdflatex hypcap.dtx
%\end{verbatim}
% \end{quote}
%
% \section{Catalogue}
%
% The following XML file can be used as source for the
% \href{http://mirror.ctan.org/help/Catalogue/catalogue.html}{\TeX\ Catalogue}.
% The elements \texttt{caption} and \texttt{description} are imported
% from the original XML file from the Catalogue.
% The name of the XML file in the Catalogue is \xfile{hypcap.xml}.
%    \begin{macrocode}
%<*catalogue>
<?xml version='1.0' encoding='us-ascii'?>
<!DOCTYPE entry SYSTEM 'catalogue.dtd'>
<entry datestamp='$Date$' modifier='$Author$' id='hypcap'>
  <name>hypcap</name>
  <caption>Adjusting the anchors of captions.</caption>
  <authorref id='auth:oberdiek'/>
  <copyright owner='Heiko Oberdiek' year='1999-2001,2006-2008,2011'/>
  <license type='lppl1.3'/>
  <version number='1.11'/>
  <description>
    The package offers a solution to the problem that when you link to
    a float using <xref refid='hyperref'>hyperref</xref>, the link
    anchors to below the float's caption, rather than the beginning of
    the float.
    <p/>
    Hypcap defines a separate \capstart command, which you put where
    you want links to end; you should have a \capstart command for each
    \caption command.  Package options can be used to auto-insert a
    \capstart at the start of a float environment.
    <p/>
    The package is part of the <xref refid='oberdiek'>oberdiek</xref> bundle.
  </description>
  <documentation details='Package documentation'
      href='ctan:/macros/latex/contrib/oberdiek/hypcap.pdf'/>
  <ctan file='true' path='/macros/latex/contrib/oberdiek/hypcap.dtx'/>
  <miktex location='oberdiek'/>
  <texlive location='oberdiek'/>
  <install path='/macros/latex/contrib/oberdiek/oberdiek.tds.zip'/>
</entry>
%</catalogue>
%    \end{macrocode}
%
% \begin{History}
%   \begin{Version}{1999/02/13 v1.0}
%   \item
%     A beginning version, published in newsgroup \xnewsgroup{comp.text.tex}:\\
%     \URL{``\link{Re: \xpackage{hyperref} and figures}''}^^A
%     {http://groups.google.com/group/comp.text.tex/msg/5c9b47b001a9379c}
%   \end{Version}
%   \begin{Version}{2000/08/14 v1.1}
%   \item
%     Global assignments of \cs{if@capstart}
%     in order to allow \cs{caption} in groups.
%   \item
%     Option \texttt{all} added.
%   \end{Version}
%   \begin{Version}{2000/09/07 v1.2}
%   \item
%     Package in dtx format.
%   \end{Version}
%   \begin{Version}{2001/08/27 v1.3}
%   \item
%     Bug fix with hyperref's pdfmark driver\\
%     (\cs{leavevmode} in
%     \cs{hyper@@anchor}/\cs{pdf@rect}).
%   \end{Version}
%   \begin{Version}{2001/09/06 v1.4}
%   \item
%     Small fixes in the dtx file.
%   \end{Version}
%   \begin{Version}{2006/02/20 v1.5}
%   \item
%     Code is not changed.
%   \item
%     New DTX framework.
%   \end{Version}
%   \begin{Version}{2007/02/19 v1.6}
%   \item
%     Fix for hypertexnames=false.
%   \end{Version}
%   \begin{Version}{2007/04/09 v1.7}
%   \item
%     Stuff in \cs{caption} moved to \xpackage{hyperref}.
%     This avoids redefinitions of \cs{caption} and \cs{@caption}
%     (idea of Axel Sommerfeldt).
%   \item
%     Fix for subfigure (Marco Kuhlmann, Amilcar do Carmo Lucas).
%   \end{Version}
%   \begin{Version}{2008/04/14 v1.8}
%   \item
%     \cs{hc@redef} fixed to get package \xpackage{float} work
%     (Axel Sommerfeldt).
%   \end{Version}
%   \begin{Version}{2008/08/11 v1.9}
%   \item
%     Code is not changed.
%   \item
%     URLs updated.
%   \end{Version}
%   \begin{Version}{2008/09/08 v1.10}
%   \item
%     \cs{capstartfalse} and \cs{capstarttrue} added.
%   \end{Version}
%   \begin{Version}{2011/02/16 v1.11}
%   \item
%     \cs{hc@redef} fixed by using package \xpackage{letltxmacro}.
%   \end{Version}
% \end{History}
%
% \PrintIndex
%
% \Finale
\endinput

%        (quote the arguments according to the demands of your shell)
%
% Documentation:
%    (a) If hypcap.drv is present:
%           latex hypcap.drv
%    (b) Without hypcap.drv:
%           latex hypcap.dtx; ...
%    The class ltxdoc loads the configuration file ltxdoc.cfg
%    if available. Here you can specify further options, e.g.
%    use A4 as paper format:
%       \PassOptionsToClass{a4paper}{article}
%
%    Programm calls to get the documentation (example):
%       pdflatex hypcap.dtx
%       makeindex -s gind.ist hypcap.idx
%       pdflatex hypcap.dtx
%       makeindex -s gind.ist hypcap.idx
%       pdflatex hypcap.dtx
%
% Installation:
%    TDS:tex/latex/oberdiek/hypcap.sty
%    TDS:doc/latex/oberdiek/hypcap.pdf
%    TDS:source/latex/oberdiek/hypcap.dtx
%
%<*ignore>
\begingroup
  \catcode123=1 %
  \catcode125=2 %
  \def\x{LaTeX2e}%
\expandafter\endgroup
\ifcase 0\ifx\install y1\fi\expandafter
         \ifx\csname processbatchFile\endcsname\relax\else1\fi
         \ifx\fmtname\x\else 1\fi\relax
\else\csname fi\endcsname
%</ignore>
%<*install>
\input docstrip.tex
\Msg{************************************************************************}
\Msg{* Installation}
\Msg{* Package: hypcap 2011/02/16 v1.11 Adjusting the anchors of captions (HO)}
\Msg{************************************************************************}

\keepsilent
\askforoverwritefalse

\let\MetaPrefix\relax
\preamble

This is a generated file.

Project: hypcap
Version: 2011/02/16 v1.11

Copyright (C) 1999-2001, 2006-2008, 2011 by
   Heiko Oberdiek <heiko.oberdiek at googlemail.com>

This work may be distributed and/or modified under the
conditions of the LaTeX Project Public License, either
version 1.3c of this license or (at your option) any later
version. This version of this license is in
   http://www.latex-project.org/lppl/lppl-1-3c.txt
and the latest version of this license is in
   http://www.latex-project.org/lppl.txt
and version 1.3 or later is part of all distributions of
LaTeX version 2005/12/01 or later.

This work has the LPPL maintenance status "maintained".

This Current Maintainer of this work is Heiko Oberdiek.

This work consists of the main source file hypcap.dtx
and the derived files
   hypcap.sty, hypcap.pdf, hypcap.ins, hypcap.drv.

\endpreamble
\let\MetaPrefix\DoubleperCent

\generate{%
  \file{hypcap.ins}{\from{hypcap.dtx}{install}}%
  \file{hypcap.drv}{\from{hypcap.dtx}{driver}}%
  \usedir{tex/latex/oberdiek}%
  \file{hypcap.sty}{\from{hypcap.dtx}{package}}%
  \nopreamble
  \nopostamble
  \usedir{source/latex/oberdiek/catalogue}%
  \file{hypcap.xml}{\from{hypcap.dtx}{catalogue}}%
}

\catcode32=13\relax% active space
\let =\space%
\Msg{************************************************************************}
\Msg{*}
\Msg{* To finish the installation you have to move the following}
\Msg{* file into a directory searched by TeX:}
\Msg{*}
\Msg{*     hypcap.sty}
\Msg{*}
\Msg{* To produce the documentation run the file `hypcap.drv'}
\Msg{* through LaTeX.}
\Msg{*}
\Msg{* Happy TeXing!}
\Msg{*}
\Msg{************************************************************************}

\endbatchfile
%</install>
%<*ignore>
\fi
%</ignore>
%<*driver>
\NeedsTeXFormat{LaTeX2e}
\ProvidesFile{hypcap.drv}%
  [2011/02/16 v1.11 Adjusting the anchors of captions (HO)]%
\documentclass{ltxdoc}
\usepackage{holtxdoc}[2011/11/22]
\begin{document}
  \DocInput{hypcap.dtx}%
\end{document}
%</driver>
% \fi
%
% \CheckSum{174}
%
% \CharacterTable
%  {Upper-case    \A\B\C\D\E\F\G\H\I\J\K\L\M\N\O\P\Q\R\S\T\U\V\W\X\Y\Z
%   Lower-case    \a\b\c\d\e\f\g\h\i\j\k\l\m\n\o\p\q\r\s\t\u\v\w\x\y\z
%   Digits        \0\1\2\3\4\5\6\7\8\9
%   Exclamation   \!     Double quote  \"     Hash (number) \#
%   Dollar        \$     Percent       \%     Ampersand     \&
%   Acute accent  \'     Left paren    \(     Right paren   \)
%   Asterisk      \*     Plus          \+     Comma         \,
%   Minus         \-     Point         \.     Solidus       \/
%   Colon         \:     Semicolon     \;     Less than     \<
%   Equals        \=     Greater than  \>     Question mark \?
%   Commercial at \@     Left bracket  \[     Backslash     \\
%   Right bracket \]     Circumflex    \^     Underscore    \_
%   Grave accent  \`     Left brace    \{     Vertical bar  \|
%   Right brace   \}     Tilde         \~}
%
% \GetFileInfo{hypcap.drv}
%
% \title{The \xpackage{hypcap} package}
% \date{2011/02/16 v1.11}
% \author{Heiko Oberdiek\\\xemail{heiko.oberdiek at googlemail.com}}
%
% \maketitle
%
% \begin{abstract}
% This package tries a solution of the problem with
% hyperref, that links to floats points below the
% caption and not at the beginning of the float.
% Therefore this package divides the task into two
% part, the link setting with \cs{capstart} or
% automatically at the beginning of a float and
% the rest in the \cs{caption} command.
% \end{abstract}
%
% \tableofcontents
%
% \newpage
% \section{Usage}
%    The package \Package{hypcap} requires that \Package{hyperref}
%    is loaded first:
%    \begin{quote}
%      |\usepackage[...]{hyperref}|\\
%      |\usepackage[...]{hypcap}|
%    \end{quote}
%
% \subsection{Package options}
%    The names of the four float environments |figure|, |figure*|,
%    |table|, or |table*| can be used as option. Then the package
%    redefines the environment in order to insert \cmd{\capstart}
%    (see below) in the beginning of the environment automatically.
%
%    Option |all| enables the redefinitions of all four
%    float environments. For other environments see
%    the user command \cmd{\hypcapredef}.
%
% \subsection{User commands}
%    \begin{description}
%    \item[\cmd{\capstart}:]\DescribeMacro{\capstart}
%      First this command increments the counter (\cmd{\@captype}).
%      Then it makes an anchor for package \Package{hyperref}.
%      At last \cmd{\caption} is redefined to remove the
%      anchor setting part from \Package{hyperref}'s \cmd{\caption}.
%
%      The package expects the following structure of a float
%      environment:
%      \begin{quote}
%        |\begin{|\textit{|float|}|}...|\\
%        |\capstart|\\
%        |...|\\
%        |\caption{...}|\\
%        |...|\\
%        |\end{|\textit{|float|}|}|
%      \end{quote}
%      There can be several \cmd{\caption} commands. For these
%      you need \cmd{\capstart} again:
%      \begin{quote}
%        |\capstart ... \caption... \capstart ... \caption...|
%      \end{quote}
%      And the \cmd{\caption} command itself can be put in a group.
%
%      With the options, described above, the extra
%      writing of \cmd{\capstart} can be avoided.
%      Consequently, there must be a \cmd{\caption} in every
%      environment of this type, specified by the option.
%      If you want to use more than one \cmd{\caption} in this
%      environment, you have to state \cmd{\capstart} again.
%    \item[\cmd{\hypcapspace}:]\DescribeMacro{\hypcapspace}
%      Because it looks poor, if the link points exactly at top of
%      the figure, there is additional space: \cmd{\hypcapspace},
%      the default is |0.5\baselineskip|, examples:
%      \begin{quote}
%        |\renewcommand{\hypcapspace}{0pt}| removes the space\\
%        |\renewcommand{\hypcapspace}{1pt}| sets a fix value
%      \end{quote}
%    \item[\cmd{\hypcapredef}:]\DescribeMacro{\hypcapredef}
%      If there are other float environments, that
%      should automatically execute \cmd{\capstart}, then
%      a redefinition with \cmd{\hypcapredef} can be tried:
%      \begin{quote}
%        |\hypcapredef{myfloat}|
%      \end{quote}
%      Only environments with one optional parameter are
%      supported.
%    \item[\cs{capstartfalse}, \cs{capstarttrue}:]^^A
%      \DescribeMacro{\capstartfalse}^^A
%      \DescribeMacro{\capstarttrue}^^A
%      Since 2008/09/08 v1.10.\\
%      They disable and enable \cs{capstart}. They can be used to
%      cancel the effect of a redefined float environment.
%      Example:
%\begin{quote}
%\catcode`Y=\active
%\def Y#1Z{\textit{\cs{#1}}}
%\begin{verbatim}
%\documentclass{article}
%\usepackage{hyperref}
%\usepackage[figure]{hypcap}[2008/09/08]
%
%\begin{document}
%  \section{Hello World}
%  \begin{figure}
%    \caption{Figure with caption A}
%  \end{figure}
%  YcapstartfalseZ
%  \begin{figure}
%    Figure without caption
%  \end{figure}
%  YcapstarttrueZ
%  \begin{figure}
%    \caption{Figure with caption B}
%  \end{figure}
%\end{document}
%\end{verbatim}
%\end{quote}
%    \end{description}
%
% \subsection{Limitations}
%    \begin{itemize}
%    \item Packages that redefine \cmd{\caption} or
%          \cmd{\@caption}.
%    \end{itemize}
%
% \StopEventually{
% }
%
% \section{Implementation}
%    \begin{macrocode}
%<*package>
%    \end{macrocode}
%    Package identification.
%    \begin{macrocode}
\NeedsTeXFormat{LaTeX2e}
\ProvidesPackage{hypcap}%
  [2011/02/16 v1.11 Adjusting the anchors of captions (HO)]
%    \end{macrocode}
%
%    For unique command names this package uses |hc@| as prefix
%    for internal command names.
%
%    First we check, if package \Package{hyperref} is loaded:
%    \begin{macrocode}
\@ifundefined{hyper@@anchor}{%
  \PackageError{hypcap}{You have to load 'hyperref' first}\@ehc
  \endinput
}{}
%    \end{macrocode}
%
%    \begin{macrocode}
\RequirePackage{letltxmacro}[2008/06/24]
%    \end{macrocode}
%
%    \begin{macro}{\hc@org@caption}
%    Save the original meaning of \cmd{\caption}:
%    \begin{macrocode}
\newcommand*\hc@org@caption{}
\let\hc@org@caption\caption
%    \end{macrocode}
%    \end{macro}
%
%    \newcommand*{\BeginMacro}[1]{\expandafter\DoBeginMacro\csname#1\endcsname}
%    \newcommand*{\DoBeginMacro}[1]{\begin{macro}{#1}}
%    \newcommand*{\Cmd}[1]{\expandafter\cmd\csname#1\endcsname}
%    \BeginMacro{if@capstart}
%     The switch \Cmd{if@capstart} helps to detect
%     \cmd{\capstart} commands with missing \cmd{\caption} macros.
%     Because \cmd{\caption} can occur inside a group, assignments
%     to the switch have to be made global.
%    \begin{macrocode}
\newif\if@capstart
%    \end{macrocode}
%    \end{macro}
%
%    \begin{macro}{\hypcapspace}
%    The anchor is raised.by \cmd{\hypcapspace}.
%    \begin{macrocode}
\newcommand*\hypcapspace{.5\baselineskip}
%    \end{macrocode}
%    \end{macro}
%
%    \begin{macro}{\ifcapstart}
%    \begin{macrocode}
\newif\ifcapstart
\capstarttrue
%    \end{macrocode}
%    \end{macro}
%
%    \begin{macro}{\capstart}
%     The macro \cmd{\capstart} contains the first part of
%     the \cmd{\caption} command: Incrementing the counter
%     and setting the anchor.
%    \begin{macrocode}
\newcommand*\capstart{%
  \ifcapstart
    \H@refstepcounter\@captype % first part of caption
    \hyper@makecurrent\@captype
    \global\let\hc@currentHref\@currentHref
    \vspace*{-\hypcapspace}%
    \begingroup
      \let\leavevmode\relax
      \hyper@@anchor\@currentHref\relax
    \endgroup
    \vspace*{\hypcapspace}%
    \hc@hyperref{\let\caption\hc@caption}%
    \global\@capstarttrue
    \global\advance\csname c@\@captype\endcsname\m@ne
  \fi
}
%    \end{macrocode}
%    \end{macro}
%
%    \begin{macrocode}
\@ifpackagelater{hyperref}{2007/04/09}{%
  \let\hc@hyperref\@gobble
}{%
  \let\hc@hyperref\@firstofone
}
%    \end{macrocode}
%
%    \begin{macro}{\hc@caption}
%    The new \cmd{\caption} command without the first part
%    is defined in the macro \cmd{\hc@caption}.
%    \begin{macrocode}
\def\hc@caption{%
  \global\advance\csname c@\@captype\endcsname\@ne
  \@dblarg{\hc@@caption\@captype}%
}
%    \end{macrocode}
%    \end{macro}
%
%    \begin{macro}{\hc@@caption}
%    This is a copy of package \Package{hyperref}'s \cmd{\@caption}
%    macro without making the anchor, because this is already done
%    in \cmd{\capstart}.
%    \begin{macrocode}
\long\def\hc@@caption#1[#2]#3{%
  \let\caption\hc@org@caption
  \global\@capstartfalse
  \ifHy@hypertexnames
    \hyper@makecurrent\@captype
  \else
    \global\let\@currentHref\hc@currentHref
  \fi
  \par\addcontentsline{%
    \csname ext@#1\endcsname}{#1}{%
    \protect\numberline{%
      \csname the#1\endcsname
    }{\ignorespaces #2}%
  }%
  \begingroup
    \@parboxrestore
    \normalsize
    \@makecaption{\csname fnum@#1\endcsname}{%
      \ignorespaces#3%
    }%
    \par
  \endgroup
}
%    \end{macrocode}
%    \end{macro}
%
%    \begin{macro}{\hypcapredef}
%    The macro \cmd{\hypcapredef} prepares the call of
%    \cmd{\hc@redef} that will redefine the environment
%    that is given in the argument.
%    \begin{macrocode}
\def\hypcapredef#1{%
  \expandafter\hc@redef\csname hc@org#1\expandafter\endcsname
                       \csname hc@orgend#1\expandafter\endcsname
                       \expandafter{#1}%
}
%    \end{macrocode}
%    \end{macro}
%    \begin{macro}{\hc@redef}
%    The old meaning of the environment is saved. Then
%    \cmd{\capstart} is appended in the begin part.
%    The end part contains a check
%    that produces an error message in case of \cmd{\capstart}
%    without \cmd{\capstart} (\cmd{\capstart} has incremented
%    the counter).
%    \begin{macrocode}
\def\hc@redef#1#2#3{%
  \newcommand#1{}%
  \expandafter\LetLtxMacro\expandafter#1\csname#3\endcsname
  \expandafter\LetLtxMacro\expandafter#2\csname end#3\endcsname
  \renewenvironment*{#3}[1][]{%
    \ifx\\##1\\%
      #1\relax
    \else
      #1[##1]% hash-ok (compatibility for float)
    \fi
    \capstart
  }{%
    \if@capstart
      \PackageError{hypcap}{You have forgotten to use \string\caption}%
      \global\@capstartfalse
    \else
    \fi
    #2%
  }%
}
%    \end{macrocode}
%    \end{macro}
%
%     At last the options are defined and processed.
%    \begin{macrocode}
\DeclareOption{figure}{\hypcapredef{\CurrentOption}}
\DeclareOption{figure*}{\hypcapredef{\CurrentOption}}
\DeclareOption{table}{\hypcapredef{\CurrentOption}}
\DeclareOption{table*}{\hypcapredef{\CurrentOption}}
\DeclareOption{all}{%
  \hypcapredef{figure}%
  \hypcapredef{figure*}%
  \hypcapredef{table}%
  \hypcapredef{table*}%
}
\ProcessOptions\relax
%    \end{macrocode}
%
%    \begin{macrocode}
%</package>
%    \end{macrocode}
%
% \section{Installation}
%
% \subsection{Download}
%
% \paragraph{Package.} This package is available on
% CTAN\footnote{\url{ftp://ftp.ctan.org/tex-archive/}}:
% \begin{description}
% \item[\CTAN{macros/latex/contrib/oberdiek/hypcap.dtx}] The source file.
% \item[\CTAN{macros/latex/contrib/oberdiek/hypcap.pdf}] Documentation.
% \end{description}
%
%
% \paragraph{Bundle.} All the packages of the bundle `oberdiek'
% are also available in a TDS compliant ZIP archive. There
% the packages are already unpacked and the documentation files
% are generated. The files and directories obey the TDS standard.
% \begin{description}
% \item[\CTAN{install/macros/latex/contrib/oberdiek.tds.zip}]
% \end{description}
% \emph{TDS} refers to the standard ``A Directory Structure
% for \TeX\ Files'' (\CTAN{tds/tds.pdf}). Directories
% with \xfile{texmf} in their name are usually organized this way.
%
% \subsection{Bundle installation}
%
% \paragraph{Unpacking.} Unpack the \xfile{oberdiek.tds.zip} in the
% TDS tree (also known as \xfile{texmf} tree) of your choice.
% Example (linux):
% \begin{quote}
%   |unzip oberdiek.tds.zip -d ~/texmf|
% \end{quote}
%
% \paragraph{Script installation.}
% Check the directory \xfile{TDS:scripts/oberdiek/} for
% scripts that need further installation steps.
% Package \xpackage{attachfile2} comes with the Perl script
% \xfile{pdfatfi.pl} that should be installed in such a way
% that it can be called as \texttt{pdfatfi}.
% Example (linux):
% \begin{quote}
%   |chmod +x scripts/oberdiek/pdfatfi.pl|\\
%   |cp scripts/oberdiek/pdfatfi.pl /usr/local/bin/|
% \end{quote}
%
% \subsection{Package installation}
%
% \paragraph{Unpacking.} The \xfile{.dtx} file is a self-extracting
% \docstrip\ archive. The files are extracted by running the
% \xfile{.dtx} through \plainTeX:
% \begin{quote}
%   \verb|tex hypcap.dtx|
% \end{quote}
%
% \paragraph{TDS.} Now the different files must be moved into
% the different directories in your installation TDS tree
% (also known as \xfile{texmf} tree):
% \begin{quote}
% \def\t{^^A
% \begin{tabular}{@{}>{\ttfamily}l@{ $\rightarrow$ }>{\ttfamily}l@{}}
%   hypcap.sty & tex/latex/oberdiek/hypcap.sty\\
%   hypcap.pdf & doc/latex/oberdiek/hypcap.pdf\\
%   hypcap.dtx & source/latex/oberdiek/hypcap.dtx\\
% \end{tabular}^^A
% }^^A
% \sbox0{\t}^^A
% \ifdim\wd0>\linewidth
%   \begingroup
%     \advance\linewidth by\leftmargin
%     \advance\linewidth by\rightmargin
%   \edef\x{\endgroup
%     \def\noexpand\lw{\the\linewidth}^^A
%   }\x
%   \def\lwbox{^^A
%     \leavevmode
%     \hbox to \linewidth{^^A
%       \kern-\leftmargin\relax
%       \hss
%       \usebox0
%       \hss
%       \kern-\rightmargin\relax
%     }^^A
%   }^^A
%   \ifdim\wd0>\lw
%     \sbox0{\small\t}^^A
%     \ifdim\wd0>\linewidth
%       \ifdim\wd0>\lw
%         \sbox0{\footnotesize\t}^^A
%         \ifdim\wd0>\linewidth
%           \ifdim\wd0>\lw
%             \sbox0{\scriptsize\t}^^A
%             \ifdim\wd0>\linewidth
%               \ifdim\wd0>\lw
%                 \sbox0{\tiny\t}^^A
%                 \ifdim\wd0>\linewidth
%                   \lwbox
%                 \else
%                   \usebox0
%                 \fi
%               \else
%                 \lwbox
%               \fi
%             \else
%               \usebox0
%             \fi
%           \else
%             \lwbox
%           \fi
%         \else
%           \usebox0
%         \fi
%       \else
%         \lwbox
%       \fi
%     \else
%       \usebox0
%     \fi
%   \else
%     \lwbox
%   \fi
% \else
%   \usebox0
% \fi
% \end{quote}
% If you have a \xfile{docstrip.cfg} that configures and enables \docstrip's
% TDS installing feature, then some files can already be in the right
% place, see the documentation of \docstrip.
%
% \subsection{Refresh file name databases}
%
% If your \TeX~distribution
% (\teTeX, \mikTeX, \dots) relies on file name databases, you must refresh
% these. For example, \teTeX\ users run \verb|texhash| or
% \verb|mktexlsr|.
%
% \subsection{Some details for the interested}
%
% \paragraph{Attached source.}
%
% The PDF documentation on CTAN also includes the
% \xfile{.dtx} source file. It can be extracted by
% AcrobatReader 6 or higher. Another option is \textsf{pdftk},
% e.g. unpack the file into the current directory:
% \begin{quote}
%   \verb|pdftk hypcap.pdf unpack_files output .|
% \end{quote}
%
% \paragraph{Unpacking with \LaTeX.}
% The \xfile{.dtx} chooses its action depending on the format:
% \begin{description}
% \item[\plainTeX:] Run \docstrip\ and extract the files.
% \item[\LaTeX:] Generate the documentation.
% \end{description}
% If you insist on using \LaTeX\ for \docstrip\ (really,
% \docstrip\ does not need \LaTeX), then inform the autodetect routine
% about your intention:
% \begin{quote}
%   \verb|latex \let\install=y% \iffalse meta-comment
%
% File: hypcap.dtx
% Version: 2011/02/16 v1.11
% Info: Adjusting the anchors of captions
%
% Copyright (C) 1999-2001, 2006-2008, 2011 by
%    Heiko Oberdiek <heiko.oberdiek at googlemail.com>
%
% This work may be distributed and/or modified under the
% conditions of the LaTeX Project Public License, either
% version 1.3c of this license or (at your option) any later
% version. This version of this license is in
%    http://www.latex-project.org/lppl/lppl-1-3c.txt
% and the latest version of this license is in
%    http://www.latex-project.org/lppl.txt
% and version 1.3 or later is part of all distributions of
% LaTeX version 2005/12/01 or later.
%
% This work has the LPPL maintenance status "maintained".
%
% This Current Maintainer of this work is Heiko Oberdiek.
%
% This work consists of the main source file hypcap.dtx
% and the derived files
%    hypcap.sty, hypcap.pdf, hypcap.ins, hypcap.drv.
%
% Distribution:
%    CTAN:macros/latex/contrib/oberdiek/hypcap.dtx
%    CTAN:macros/latex/contrib/oberdiek/hypcap.pdf
%
% Unpacking:
%    (a) If hypcap.ins is present:
%           tex hypcap.ins
%    (b) Without hypcap.ins:
%           tex hypcap.dtx
%    (c) If you insist on using LaTeX
%           latex \let\install=y% \iffalse meta-comment
%
% File: hypcap.dtx
% Version: 2011/02/16 v1.11
% Info: Adjusting the anchors of captions
%
% Copyright (C) 1999-2001, 2006-2008, 2011 by
%    Heiko Oberdiek <heiko.oberdiek at googlemail.com>
%
% This work may be distributed and/or modified under the
% conditions of the LaTeX Project Public License, either
% version 1.3c of this license or (at your option) any later
% version. This version of this license is in
%    http://www.latex-project.org/lppl/lppl-1-3c.txt
% and the latest version of this license is in
%    http://www.latex-project.org/lppl.txt
% and version 1.3 or later is part of all distributions of
% LaTeX version 2005/12/01 or later.
%
% This work has the LPPL maintenance status "maintained".
%
% This Current Maintainer of this work is Heiko Oberdiek.
%
% This work consists of the main source file hypcap.dtx
% and the derived files
%    hypcap.sty, hypcap.pdf, hypcap.ins, hypcap.drv.
%
% Distribution:
%    CTAN:macros/latex/contrib/oberdiek/hypcap.dtx
%    CTAN:macros/latex/contrib/oberdiek/hypcap.pdf
%
% Unpacking:
%    (a) If hypcap.ins is present:
%           tex hypcap.ins
%    (b) Without hypcap.ins:
%           tex hypcap.dtx
%    (c) If you insist on using LaTeX
%           latex \let\install=y\input{hypcap.dtx}
%        (quote the arguments according to the demands of your shell)
%
% Documentation:
%    (a) If hypcap.drv is present:
%           latex hypcap.drv
%    (b) Without hypcap.drv:
%           latex hypcap.dtx; ...
%    The class ltxdoc loads the configuration file ltxdoc.cfg
%    if available. Here you can specify further options, e.g.
%    use A4 as paper format:
%       \PassOptionsToClass{a4paper}{article}
%
%    Programm calls to get the documentation (example):
%       pdflatex hypcap.dtx
%       makeindex -s gind.ist hypcap.idx
%       pdflatex hypcap.dtx
%       makeindex -s gind.ist hypcap.idx
%       pdflatex hypcap.dtx
%
% Installation:
%    TDS:tex/latex/oberdiek/hypcap.sty
%    TDS:doc/latex/oberdiek/hypcap.pdf
%    TDS:source/latex/oberdiek/hypcap.dtx
%
%<*ignore>
\begingroup
  \catcode123=1 %
  \catcode125=2 %
  \def\x{LaTeX2e}%
\expandafter\endgroup
\ifcase 0\ifx\install y1\fi\expandafter
         \ifx\csname processbatchFile\endcsname\relax\else1\fi
         \ifx\fmtname\x\else 1\fi\relax
\else\csname fi\endcsname
%</ignore>
%<*install>
\input docstrip.tex
\Msg{************************************************************************}
\Msg{* Installation}
\Msg{* Package: hypcap 2011/02/16 v1.11 Adjusting the anchors of captions (HO)}
\Msg{************************************************************************}

\keepsilent
\askforoverwritefalse

\let\MetaPrefix\relax
\preamble

This is a generated file.

Project: hypcap
Version: 2011/02/16 v1.11

Copyright (C) 1999-2001, 2006-2008, 2011 by
   Heiko Oberdiek <heiko.oberdiek at googlemail.com>

This work may be distributed and/or modified under the
conditions of the LaTeX Project Public License, either
version 1.3c of this license or (at your option) any later
version. This version of this license is in
   http://www.latex-project.org/lppl/lppl-1-3c.txt
and the latest version of this license is in
   http://www.latex-project.org/lppl.txt
and version 1.3 or later is part of all distributions of
LaTeX version 2005/12/01 or later.

This work has the LPPL maintenance status "maintained".

This Current Maintainer of this work is Heiko Oberdiek.

This work consists of the main source file hypcap.dtx
and the derived files
   hypcap.sty, hypcap.pdf, hypcap.ins, hypcap.drv.

\endpreamble
\let\MetaPrefix\DoubleperCent

\generate{%
  \file{hypcap.ins}{\from{hypcap.dtx}{install}}%
  \file{hypcap.drv}{\from{hypcap.dtx}{driver}}%
  \usedir{tex/latex/oberdiek}%
  \file{hypcap.sty}{\from{hypcap.dtx}{package}}%
  \nopreamble
  \nopostamble
  \usedir{source/latex/oberdiek/catalogue}%
  \file{hypcap.xml}{\from{hypcap.dtx}{catalogue}}%
}

\catcode32=13\relax% active space
\let =\space%
\Msg{************************************************************************}
\Msg{*}
\Msg{* To finish the installation you have to move the following}
\Msg{* file into a directory searched by TeX:}
\Msg{*}
\Msg{*     hypcap.sty}
\Msg{*}
\Msg{* To produce the documentation run the file `hypcap.drv'}
\Msg{* through LaTeX.}
\Msg{*}
\Msg{* Happy TeXing!}
\Msg{*}
\Msg{************************************************************************}

\endbatchfile
%</install>
%<*ignore>
\fi
%</ignore>
%<*driver>
\NeedsTeXFormat{LaTeX2e}
\ProvidesFile{hypcap.drv}%
  [2011/02/16 v1.11 Adjusting the anchors of captions (HO)]%
\documentclass{ltxdoc}
\usepackage{holtxdoc}[2011/11/22]
\begin{document}
  \DocInput{hypcap.dtx}%
\end{document}
%</driver>
% \fi
%
% \CheckSum{174}
%
% \CharacterTable
%  {Upper-case    \A\B\C\D\E\F\G\H\I\J\K\L\M\N\O\P\Q\R\S\T\U\V\W\X\Y\Z
%   Lower-case    \a\b\c\d\e\f\g\h\i\j\k\l\m\n\o\p\q\r\s\t\u\v\w\x\y\z
%   Digits        \0\1\2\3\4\5\6\7\8\9
%   Exclamation   \!     Double quote  \"     Hash (number) \#
%   Dollar        \$     Percent       \%     Ampersand     \&
%   Acute accent  \'     Left paren    \(     Right paren   \)
%   Asterisk      \*     Plus          \+     Comma         \,
%   Minus         \-     Point         \.     Solidus       \/
%   Colon         \:     Semicolon     \;     Less than     \<
%   Equals        \=     Greater than  \>     Question mark \?
%   Commercial at \@     Left bracket  \[     Backslash     \\
%   Right bracket \]     Circumflex    \^     Underscore    \_
%   Grave accent  \`     Left brace    \{     Vertical bar  \|
%   Right brace   \}     Tilde         \~}
%
% \GetFileInfo{hypcap.drv}
%
% \title{The \xpackage{hypcap} package}
% \date{2011/02/16 v1.11}
% \author{Heiko Oberdiek\\\xemail{heiko.oberdiek at googlemail.com}}
%
% \maketitle
%
% \begin{abstract}
% This package tries a solution of the problem with
% hyperref, that links to floats points below the
% caption and not at the beginning of the float.
% Therefore this package divides the task into two
% part, the link setting with \cs{capstart} or
% automatically at the beginning of a float and
% the rest in the \cs{caption} command.
% \end{abstract}
%
% \tableofcontents
%
% \newpage
% \section{Usage}
%    The package \Package{hypcap} requires that \Package{hyperref}
%    is loaded first:
%    \begin{quote}
%      |\usepackage[...]{hyperref}|\\
%      |\usepackage[...]{hypcap}|
%    \end{quote}
%
% \subsection{Package options}
%    The names of the four float environments |figure|, |figure*|,
%    |table|, or |table*| can be used as option. Then the package
%    redefines the environment in order to insert \cmd{\capstart}
%    (see below) in the beginning of the environment automatically.
%
%    Option |all| enables the redefinitions of all four
%    float environments. For other environments see
%    the user command \cmd{\hypcapredef}.
%
% \subsection{User commands}
%    \begin{description}
%    \item[\cmd{\capstart}:]\DescribeMacro{\capstart}
%      First this command increments the counter (\cmd{\@captype}).
%      Then it makes an anchor for package \Package{hyperref}.
%      At last \cmd{\caption} is redefined to remove the
%      anchor setting part from \Package{hyperref}'s \cmd{\caption}.
%
%      The package expects the following structure of a float
%      environment:
%      \begin{quote}
%        |\begin{|\textit{|float|}|}...|\\
%        |\capstart|\\
%        |...|\\
%        |\caption{...}|\\
%        |...|\\
%        |\end{|\textit{|float|}|}|
%      \end{quote}
%      There can be several \cmd{\caption} commands. For these
%      you need \cmd{\capstart} again:
%      \begin{quote}
%        |\capstart ... \caption... \capstart ... \caption...|
%      \end{quote}
%      And the \cmd{\caption} command itself can be put in a group.
%
%      With the options, described above, the extra
%      writing of \cmd{\capstart} can be avoided.
%      Consequently, there must be a \cmd{\caption} in every
%      environment of this type, specified by the option.
%      If you want to use more than one \cmd{\caption} in this
%      environment, you have to state \cmd{\capstart} again.
%    \item[\cmd{\hypcapspace}:]\DescribeMacro{\hypcapspace}
%      Because it looks poor, if the link points exactly at top of
%      the figure, there is additional space: \cmd{\hypcapspace},
%      the default is |0.5\baselineskip|, examples:
%      \begin{quote}
%        |\renewcommand{\hypcapspace}{0pt}| removes the space\\
%        |\renewcommand{\hypcapspace}{1pt}| sets a fix value
%      \end{quote}
%    \item[\cmd{\hypcapredef}:]\DescribeMacro{\hypcapredef}
%      If there are other float environments, that
%      should automatically execute \cmd{\capstart}, then
%      a redefinition with \cmd{\hypcapredef} can be tried:
%      \begin{quote}
%        |\hypcapredef{myfloat}|
%      \end{quote}
%      Only environments with one optional parameter are
%      supported.
%    \item[\cs{capstartfalse}, \cs{capstarttrue}:]^^A
%      \DescribeMacro{\capstartfalse}^^A
%      \DescribeMacro{\capstarttrue}^^A
%      Since 2008/09/08 v1.10.\\
%      They disable and enable \cs{capstart}. They can be used to
%      cancel the effect of a redefined float environment.
%      Example:
%\begin{quote}
%\catcode`Y=\active
%\def Y#1Z{\textit{\cs{#1}}}
%\begin{verbatim}
%\documentclass{article}
%\usepackage{hyperref}
%\usepackage[figure]{hypcap}[2008/09/08]
%
%\begin{document}
%  \section{Hello World}
%  \begin{figure}
%    \caption{Figure with caption A}
%  \end{figure}
%  YcapstartfalseZ
%  \begin{figure}
%    Figure without caption
%  \end{figure}
%  YcapstarttrueZ
%  \begin{figure}
%    \caption{Figure with caption B}
%  \end{figure}
%\end{document}
%\end{verbatim}
%\end{quote}
%    \end{description}
%
% \subsection{Limitations}
%    \begin{itemize}
%    \item Packages that redefine \cmd{\caption} or
%          \cmd{\@caption}.
%    \end{itemize}
%
% \StopEventually{
% }
%
% \section{Implementation}
%    \begin{macrocode}
%<*package>
%    \end{macrocode}
%    Package identification.
%    \begin{macrocode}
\NeedsTeXFormat{LaTeX2e}
\ProvidesPackage{hypcap}%
  [2011/02/16 v1.11 Adjusting the anchors of captions (HO)]
%    \end{macrocode}
%
%    For unique command names this package uses |hc@| as prefix
%    for internal command names.
%
%    First we check, if package \Package{hyperref} is loaded:
%    \begin{macrocode}
\@ifundefined{hyper@@anchor}{%
  \PackageError{hypcap}{You have to load 'hyperref' first}\@ehc
  \endinput
}{}
%    \end{macrocode}
%
%    \begin{macrocode}
\RequirePackage{letltxmacro}[2008/06/24]
%    \end{macrocode}
%
%    \begin{macro}{\hc@org@caption}
%    Save the original meaning of \cmd{\caption}:
%    \begin{macrocode}
\newcommand*\hc@org@caption{}
\let\hc@org@caption\caption
%    \end{macrocode}
%    \end{macro}
%
%    \newcommand*{\BeginMacro}[1]{\expandafter\DoBeginMacro\csname#1\endcsname}
%    \newcommand*{\DoBeginMacro}[1]{\begin{macro}{#1}}
%    \newcommand*{\Cmd}[1]{\expandafter\cmd\csname#1\endcsname}
%    \BeginMacro{if@capstart}
%     The switch \Cmd{if@capstart} helps to detect
%     \cmd{\capstart} commands with missing \cmd{\caption} macros.
%     Because \cmd{\caption} can occur inside a group, assignments
%     to the switch have to be made global.
%    \begin{macrocode}
\newif\if@capstart
%    \end{macrocode}
%    \end{macro}
%
%    \begin{macro}{\hypcapspace}
%    The anchor is raised.by \cmd{\hypcapspace}.
%    \begin{macrocode}
\newcommand*\hypcapspace{.5\baselineskip}
%    \end{macrocode}
%    \end{macro}
%
%    \begin{macro}{\ifcapstart}
%    \begin{macrocode}
\newif\ifcapstart
\capstarttrue
%    \end{macrocode}
%    \end{macro}
%
%    \begin{macro}{\capstart}
%     The macro \cmd{\capstart} contains the first part of
%     the \cmd{\caption} command: Incrementing the counter
%     and setting the anchor.
%    \begin{macrocode}
\newcommand*\capstart{%
  \ifcapstart
    \H@refstepcounter\@captype % first part of caption
    \hyper@makecurrent\@captype
    \global\let\hc@currentHref\@currentHref
    \vspace*{-\hypcapspace}%
    \begingroup
      \let\leavevmode\relax
      \hyper@@anchor\@currentHref\relax
    \endgroup
    \vspace*{\hypcapspace}%
    \hc@hyperref{\let\caption\hc@caption}%
    \global\@capstarttrue
    \global\advance\csname c@\@captype\endcsname\m@ne
  \fi
}
%    \end{macrocode}
%    \end{macro}
%
%    \begin{macrocode}
\@ifpackagelater{hyperref}{2007/04/09}{%
  \let\hc@hyperref\@gobble
}{%
  \let\hc@hyperref\@firstofone
}
%    \end{macrocode}
%
%    \begin{macro}{\hc@caption}
%    The new \cmd{\caption} command without the first part
%    is defined in the macro \cmd{\hc@caption}.
%    \begin{macrocode}
\def\hc@caption{%
  \global\advance\csname c@\@captype\endcsname\@ne
  \@dblarg{\hc@@caption\@captype}%
}
%    \end{macrocode}
%    \end{macro}
%
%    \begin{macro}{\hc@@caption}
%    This is a copy of package \Package{hyperref}'s \cmd{\@caption}
%    macro without making the anchor, because this is already done
%    in \cmd{\capstart}.
%    \begin{macrocode}
\long\def\hc@@caption#1[#2]#3{%
  \let\caption\hc@org@caption
  \global\@capstartfalse
  \ifHy@hypertexnames
    \hyper@makecurrent\@captype
  \else
    \global\let\@currentHref\hc@currentHref
  \fi
  \par\addcontentsline{%
    \csname ext@#1\endcsname}{#1}{%
    \protect\numberline{%
      \csname the#1\endcsname
    }{\ignorespaces #2}%
  }%
  \begingroup
    \@parboxrestore
    \normalsize
    \@makecaption{\csname fnum@#1\endcsname}{%
      \ignorespaces#3%
    }%
    \par
  \endgroup
}
%    \end{macrocode}
%    \end{macro}
%
%    \begin{macro}{\hypcapredef}
%    The macro \cmd{\hypcapredef} prepares the call of
%    \cmd{\hc@redef} that will redefine the environment
%    that is given in the argument.
%    \begin{macrocode}
\def\hypcapredef#1{%
  \expandafter\hc@redef\csname hc@org#1\expandafter\endcsname
                       \csname hc@orgend#1\expandafter\endcsname
                       \expandafter{#1}%
}
%    \end{macrocode}
%    \end{macro}
%    \begin{macro}{\hc@redef}
%    The old meaning of the environment is saved. Then
%    \cmd{\capstart} is appended in the begin part.
%    The end part contains a check
%    that produces an error message in case of \cmd{\capstart}
%    without \cmd{\capstart} (\cmd{\capstart} has incremented
%    the counter).
%    \begin{macrocode}
\def\hc@redef#1#2#3{%
  \newcommand#1{}%
  \expandafter\LetLtxMacro\expandafter#1\csname#3\endcsname
  \expandafter\LetLtxMacro\expandafter#2\csname end#3\endcsname
  \renewenvironment*{#3}[1][]{%
    \ifx\\##1\\%
      #1\relax
    \else
      #1[##1]% hash-ok (compatibility for float)
    \fi
    \capstart
  }{%
    \if@capstart
      \PackageError{hypcap}{You have forgotten to use \string\caption}%
      \global\@capstartfalse
    \else
    \fi
    #2%
  }%
}
%    \end{macrocode}
%    \end{macro}
%
%     At last the options are defined and processed.
%    \begin{macrocode}
\DeclareOption{figure}{\hypcapredef{\CurrentOption}}
\DeclareOption{figure*}{\hypcapredef{\CurrentOption}}
\DeclareOption{table}{\hypcapredef{\CurrentOption}}
\DeclareOption{table*}{\hypcapredef{\CurrentOption}}
\DeclareOption{all}{%
  \hypcapredef{figure}%
  \hypcapredef{figure*}%
  \hypcapredef{table}%
  \hypcapredef{table*}%
}
\ProcessOptions\relax
%    \end{macrocode}
%
%    \begin{macrocode}
%</package>
%    \end{macrocode}
%
% \section{Installation}
%
% \subsection{Download}
%
% \paragraph{Package.} This package is available on
% CTAN\footnote{\url{ftp://ftp.ctan.org/tex-archive/}}:
% \begin{description}
% \item[\CTAN{macros/latex/contrib/oberdiek/hypcap.dtx}] The source file.
% \item[\CTAN{macros/latex/contrib/oberdiek/hypcap.pdf}] Documentation.
% \end{description}
%
%
% \paragraph{Bundle.} All the packages of the bundle `oberdiek'
% are also available in a TDS compliant ZIP archive. There
% the packages are already unpacked and the documentation files
% are generated. The files and directories obey the TDS standard.
% \begin{description}
% \item[\CTAN{install/macros/latex/contrib/oberdiek.tds.zip}]
% \end{description}
% \emph{TDS} refers to the standard ``A Directory Structure
% for \TeX\ Files'' (\CTAN{tds/tds.pdf}). Directories
% with \xfile{texmf} in their name are usually organized this way.
%
% \subsection{Bundle installation}
%
% \paragraph{Unpacking.} Unpack the \xfile{oberdiek.tds.zip} in the
% TDS tree (also known as \xfile{texmf} tree) of your choice.
% Example (linux):
% \begin{quote}
%   |unzip oberdiek.tds.zip -d ~/texmf|
% \end{quote}
%
% \paragraph{Script installation.}
% Check the directory \xfile{TDS:scripts/oberdiek/} for
% scripts that need further installation steps.
% Package \xpackage{attachfile2} comes with the Perl script
% \xfile{pdfatfi.pl} that should be installed in such a way
% that it can be called as \texttt{pdfatfi}.
% Example (linux):
% \begin{quote}
%   |chmod +x scripts/oberdiek/pdfatfi.pl|\\
%   |cp scripts/oberdiek/pdfatfi.pl /usr/local/bin/|
% \end{quote}
%
% \subsection{Package installation}
%
% \paragraph{Unpacking.} The \xfile{.dtx} file is a self-extracting
% \docstrip\ archive. The files are extracted by running the
% \xfile{.dtx} through \plainTeX:
% \begin{quote}
%   \verb|tex hypcap.dtx|
% \end{quote}
%
% \paragraph{TDS.} Now the different files must be moved into
% the different directories in your installation TDS tree
% (also known as \xfile{texmf} tree):
% \begin{quote}
% \def\t{^^A
% \begin{tabular}{@{}>{\ttfamily}l@{ $\rightarrow$ }>{\ttfamily}l@{}}
%   hypcap.sty & tex/latex/oberdiek/hypcap.sty\\
%   hypcap.pdf & doc/latex/oberdiek/hypcap.pdf\\
%   hypcap.dtx & source/latex/oberdiek/hypcap.dtx\\
% \end{tabular}^^A
% }^^A
% \sbox0{\t}^^A
% \ifdim\wd0>\linewidth
%   \begingroup
%     \advance\linewidth by\leftmargin
%     \advance\linewidth by\rightmargin
%   \edef\x{\endgroup
%     \def\noexpand\lw{\the\linewidth}^^A
%   }\x
%   \def\lwbox{^^A
%     \leavevmode
%     \hbox to \linewidth{^^A
%       \kern-\leftmargin\relax
%       \hss
%       \usebox0
%       \hss
%       \kern-\rightmargin\relax
%     }^^A
%   }^^A
%   \ifdim\wd0>\lw
%     \sbox0{\small\t}^^A
%     \ifdim\wd0>\linewidth
%       \ifdim\wd0>\lw
%         \sbox0{\footnotesize\t}^^A
%         \ifdim\wd0>\linewidth
%           \ifdim\wd0>\lw
%             \sbox0{\scriptsize\t}^^A
%             \ifdim\wd0>\linewidth
%               \ifdim\wd0>\lw
%                 \sbox0{\tiny\t}^^A
%                 \ifdim\wd0>\linewidth
%                   \lwbox
%                 \else
%                   \usebox0
%                 \fi
%               \else
%                 \lwbox
%               \fi
%             \else
%               \usebox0
%             \fi
%           \else
%             \lwbox
%           \fi
%         \else
%           \usebox0
%         \fi
%       \else
%         \lwbox
%       \fi
%     \else
%       \usebox0
%     \fi
%   \else
%     \lwbox
%   \fi
% \else
%   \usebox0
% \fi
% \end{quote}
% If you have a \xfile{docstrip.cfg} that configures and enables \docstrip's
% TDS installing feature, then some files can already be in the right
% place, see the documentation of \docstrip.
%
% \subsection{Refresh file name databases}
%
% If your \TeX~distribution
% (\teTeX, \mikTeX, \dots) relies on file name databases, you must refresh
% these. For example, \teTeX\ users run \verb|texhash| or
% \verb|mktexlsr|.
%
% \subsection{Some details for the interested}
%
% \paragraph{Attached source.}
%
% The PDF documentation on CTAN also includes the
% \xfile{.dtx} source file. It can be extracted by
% AcrobatReader 6 or higher. Another option is \textsf{pdftk},
% e.g. unpack the file into the current directory:
% \begin{quote}
%   \verb|pdftk hypcap.pdf unpack_files output .|
% \end{quote}
%
% \paragraph{Unpacking with \LaTeX.}
% The \xfile{.dtx} chooses its action depending on the format:
% \begin{description}
% \item[\plainTeX:] Run \docstrip\ and extract the files.
% \item[\LaTeX:] Generate the documentation.
% \end{description}
% If you insist on using \LaTeX\ for \docstrip\ (really,
% \docstrip\ does not need \LaTeX), then inform the autodetect routine
% about your intention:
% \begin{quote}
%   \verb|latex \let\install=y\input{hypcap.dtx}|
% \end{quote}
% Do not forget to quote the argument according to the demands
% of your shell.
%
% \paragraph{Generating the documentation.}
% You can use both the \xfile{.dtx} or the \xfile{.drv} to generate
% the documentation. The process can be configured by the
% configuration file \xfile{ltxdoc.cfg}. For instance, put this
% line into this file, if you want to have A4 as paper format:
% \begin{quote}
%   \verb|\PassOptionsToClass{a4paper}{article}|
% \end{quote}
% An example follows how to generate the
% documentation with pdf\LaTeX:
% \begin{quote}
%\begin{verbatim}
%pdflatex hypcap.dtx
%makeindex -s gind.ist hypcap.idx
%pdflatex hypcap.dtx
%makeindex -s gind.ist hypcap.idx
%pdflatex hypcap.dtx
%\end{verbatim}
% \end{quote}
%
% \section{Catalogue}
%
% The following XML file can be used as source for the
% \href{http://mirror.ctan.org/help/Catalogue/catalogue.html}{\TeX\ Catalogue}.
% The elements \texttt{caption} and \texttt{description} are imported
% from the original XML file from the Catalogue.
% The name of the XML file in the Catalogue is \xfile{hypcap.xml}.
%    \begin{macrocode}
%<*catalogue>
<?xml version='1.0' encoding='us-ascii'?>
<!DOCTYPE entry SYSTEM 'catalogue.dtd'>
<entry datestamp='$Date$' modifier='$Author$' id='hypcap'>
  <name>hypcap</name>
  <caption>Adjusting the anchors of captions.</caption>
  <authorref id='auth:oberdiek'/>
  <copyright owner='Heiko Oberdiek' year='1999-2001,2006-2008,2011'/>
  <license type='lppl1.3'/>
  <version number='1.11'/>
  <description>
    The package offers a solution to the problem that when you link to
    a float using <xref refid='hyperref'>hyperref</xref>, the link
    anchors to below the float's caption, rather than the beginning of
    the float.
    <p/>
    Hypcap defines a separate \capstart command, which you put where
    you want links to end; you should have a \capstart command for each
    \caption command.  Package options can be used to auto-insert a
    \capstart at the start of a float environment.
    <p/>
    The package is part of the <xref refid='oberdiek'>oberdiek</xref> bundle.
  </description>
  <documentation details='Package documentation'
      href='ctan:/macros/latex/contrib/oberdiek/hypcap.pdf'/>
  <ctan file='true' path='/macros/latex/contrib/oberdiek/hypcap.dtx'/>
  <miktex location='oberdiek'/>
  <texlive location='oberdiek'/>
  <install path='/macros/latex/contrib/oberdiek/oberdiek.tds.zip'/>
</entry>
%</catalogue>
%    \end{macrocode}
%
% \begin{History}
%   \begin{Version}{1999/02/13 v1.0}
%   \item
%     A beginning version, published in newsgroup \xnewsgroup{comp.text.tex}:\\
%     \URL{``\link{Re: \xpackage{hyperref} and figures}''}^^A
%     {http://groups.google.com/group/comp.text.tex/msg/5c9b47b001a9379c}
%   \end{Version}
%   \begin{Version}{2000/08/14 v1.1}
%   \item
%     Global assignments of \cs{if@capstart}
%     in order to allow \cs{caption} in groups.
%   \item
%     Option \texttt{all} added.
%   \end{Version}
%   \begin{Version}{2000/09/07 v1.2}
%   \item
%     Package in dtx format.
%   \end{Version}
%   \begin{Version}{2001/08/27 v1.3}
%   \item
%     Bug fix with hyperref's pdfmark driver\\
%     (\cs{leavevmode} in
%     \cs{hyper@@anchor}/\cs{pdf@rect}).
%   \end{Version}
%   \begin{Version}{2001/09/06 v1.4}
%   \item
%     Small fixes in the dtx file.
%   \end{Version}
%   \begin{Version}{2006/02/20 v1.5}
%   \item
%     Code is not changed.
%   \item
%     New DTX framework.
%   \end{Version}
%   \begin{Version}{2007/02/19 v1.6}
%   \item
%     Fix for hypertexnames=false.
%   \end{Version}
%   \begin{Version}{2007/04/09 v1.7}
%   \item
%     Stuff in \cs{caption} moved to \xpackage{hyperref}.
%     This avoids redefinitions of \cs{caption} and \cs{@caption}
%     (idea of Axel Sommerfeldt).
%   \item
%     Fix for subfigure (Marco Kuhlmann, Amilcar do Carmo Lucas).
%   \end{Version}
%   \begin{Version}{2008/04/14 v1.8}
%   \item
%     \cs{hc@redef} fixed to get package \xpackage{float} work
%     (Axel Sommerfeldt).
%   \end{Version}
%   \begin{Version}{2008/08/11 v1.9}
%   \item
%     Code is not changed.
%   \item
%     URLs updated.
%   \end{Version}
%   \begin{Version}{2008/09/08 v1.10}
%   \item
%     \cs{capstartfalse} and \cs{capstarttrue} added.
%   \end{Version}
%   \begin{Version}{2011/02/16 v1.11}
%   \item
%     \cs{hc@redef} fixed by using package \xpackage{letltxmacro}.
%   \end{Version}
% \end{History}
%
% \PrintIndex
%
% \Finale
\endinput

%        (quote the arguments according to the demands of your shell)
%
% Documentation:
%    (a) If hypcap.drv is present:
%           latex hypcap.drv
%    (b) Without hypcap.drv:
%           latex hypcap.dtx; ...
%    The class ltxdoc loads the configuration file ltxdoc.cfg
%    if available. Here you can specify further options, e.g.
%    use A4 as paper format:
%       \PassOptionsToClass{a4paper}{article}
%
%    Programm calls to get the documentation (example):
%       pdflatex hypcap.dtx
%       makeindex -s gind.ist hypcap.idx
%       pdflatex hypcap.dtx
%       makeindex -s gind.ist hypcap.idx
%       pdflatex hypcap.dtx
%
% Installation:
%    TDS:tex/latex/oberdiek/hypcap.sty
%    TDS:doc/latex/oberdiek/hypcap.pdf
%    TDS:source/latex/oberdiek/hypcap.dtx
%
%<*ignore>
\begingroup
  \catcode123=1 %
  \catcode125=2 %
  \def\x{LaTeX2e}%
\expandafter\endgroup
\ifcase 0\ifx\install y1\fi\expandafter
         \ifx\csname processbatchFile\endcsname\relax\else1\fi
         \ifx\fmtname\x\else 1\fi\relax
\else\csname fi\endcsname
%</ignore>
%<*install>
\input docstrip.tex
\Msg{************************************************************************}
\Msg{* Installation}
\Msg{* Package: hypcap 2011/02/16 v1.11 Adjusting the anchors of captions (HO)}
\Msg{************************************************************************}

\keepsilent
\askforoverwritefalse

\let\MetaPrefix\relax
\preamble

This is a generated file.

Project: hypcap
Version: 2011/02/16 v1.11

Copyright (C) 1999-2001, 2006-2008, 2011 by
   Heiko Oberdiek <heiko.oberdiek at googlemail.com>

This work may be distributed and/or modified under the
conditions of the LaTeX Project Public License, either
version 1.3c of this license or (at your option) any later
version. This version of this license is in
   http://www.latex-project.org/lppl/lppl-1-3c.txt
and the latest version of this license is in
   http://www.latex-project.org/lppl.txt
and version 1.3 or later is part of all distributions of
LaTeX version 2005/12/01 or later.

This work has the LPPL maintenance status "maintained".

This Current Maintainer of this work is Heiko Oberdiek.

This work consists of the main source file hypcap.dtx
and the derived files
   hypcap.sty, hypcap.pdf, hypcap.ins, hypcap.drv.

\endpreamble
\let\MetaPrefix\DoubleperCent

\generate{%
  \file{hypcap.ins}{\from{hypcap.dtx}{install}}%
  \file{hypcap.drv}{\from{hypcap.dtx}{driver}}%
  \usedir{tex/latex/oberdiek}%
  \file{hypcap.sty}{\from{hypcap.dtx}{package}}%
  \nopreamble
  \nopostamble
  \usedir{source/latex/oberdiek/catalogue}%
  \file{hypcap.xml}{\from{hypcap.dtx}{catalogue}}%
}

\catcode32=13\relax% active space
\let =\space%
\Msg{************************************************************************}
\Msg{*}
\Msg{* To finish the installation you have to move the following}
\Msg{* file into a directory searched by TeX:}
\Msg{*}
\Msg{*     hypcap.sty}
\Msg{*}
\Msg{* To produce the documentation run the file `hypcap.drv'}
\Msg{* through LaTeX.}
\Msg{*}
\Msg{* Happy TeXing!}
\Msg{*}
\Msg{************************************************************************}

\endbatchfile
%</install>
%<*ignore>
\fi
%</ignore>
%<*driver>
\NeedsTeXFormat{LaTeX2e}
\ProvidesFile{hypcap.drv}%
  [2011/02/16 v1.11 Adjusting the anchors of captions (HO)]%
\documentclass{ltxdoc}
\usepackage{holtxdoc}[2011/11/22]
\begin{document}
  \DocInput{hypcap.dtx}%
\end{document}
%</driver>
% \fi
%
% \CheckSum{174}
%
% \CharacterTable
%  {Upper-case    \A\B\C\D\E\F\G\H\I\J\K\L\M\N\O\P\Q\R\S\T\U\V\W\X\Y\Z
%   Lower-case    \a\b\c\d\e\f\g\h\i\j\k\l\m\n\o\p\q\r\s\t\u\v\w\x\y\z
%   Digits        \0\1\2\3\4\5\6\7\8\9
%   Exclamation   \!     Double quote  \"     Hash (number) \#
%   Dollar        \$     Percent       \%     Ampersand     \&
%   Acute accent  \'     Left paren    \(     Right paren   \)
%   Asterisk      \*     Plus          \+     Comma         \,
%   Minus         \-     Point         \.     Solidus       \/
%   Colon         \:     Semicolon     \;     Less than     \<
%   Equals        \=     Greater than  \>     Question mark \?
%   Commercial at \@     Left bracket  \[     Backslash     \\
%   Right bracket \]     Circumflex    \^     Underscore    \_
%   Grave accent  \`     Left brace    \{     Vertical bar  \|
%   Right brace   \}     Tilde         \~}
%
% \GetFileInfo{hypcap.drv}
%
% \title{The \xpackage{hypcap} package}
% \date{2011/02/16 v1.11}
% \author{Heiko Oberdiek\\\xemail{heiko.oberdiek at googlemail.com}}
%
% \maketitle
%
% \begin{abstract}
% This package tries a solution of the problem with
% hyperref, that links to floats points below the
% caption and not at the beginning of the float.
% Therefore this package divides the task into two
% part, the link setting with \cs{capstart} or
% automatically at the beginning of a float and
% the rest in the \cs{caption} command.
% \end{abstract}
%
% \tableofcontents
%
% \newpage
% \section{Usage}
%    The package \Package{hypcap} requires that \Package{hyperref}
%    is loaded first:
%    \begin{quote}
%      |\usepackage[...]{hyperref}|\\
%      |\usepackage[...]{hypcap}|
%    \end{quote}
%
% \subsection{Package options}
%    The names of the four float environments |figure|, |figure*|,
%    |table|, or |table*| can be used as option. Then the package
%    redefines the environment in order to insert \cmd{\capstart}
%    (see below) in the beginning of the environment automatically.
%
%    Option |all| enables the redefinitions of all four
%    float environments. For other environments see
%    the user command \cmd{\hypcapredef}.
%
% \subsection{User commands}
%    \begin{description}
%    \item[\cmd{\capstart}:]\DescribeMacro{\capstart}
%      First this command increments the counter (\cmd{\@captype}).
%      Then it makes an anchor for package \Package{hyperref}.
%      At last \cmd{\caption} is redefined to remove the
%      anchor setting part from \Package{hyperref}'s \cmd{\caption}.
%
%      The package expects the following structure of a float
%      environment:
%      \begin{quote}
%        |\begin{|\textit{|float|}|}...|\\
%        |\capstart|\\
%        |...|\\
%        |\caption{...}|\\
%        |...|\\
%        |\end{|\textit{|float|}|}|
%      \end{quote}
%      There can be several \cmd{\caption} commands. For these
%      you need \cmd{\capstart} again:
%      \begin{quote}
%        |\capstart ... \caption... \capstart ... \caption...|
%      \end{quote}
%      And the \cmd{\caption} command itself can be put in a group.
%
%      With the options, described above, the extra
%      writing of \cmd{\capstart} can be avoided.
%      Consequently, there must be a \cmd{\caption} in every
%      environment of this type, specified by the option.
%      If you want to use more than one \cmd{\caption} in this
%      environment, you have to state \cmd{\capstart} again.
%    \item[\cmd{\hypcapspace}:]\DescribeMacro{\hypcapspace}
%      Because it looks poor, if the link points exactly at top of
%      the figure, there is additional space: \cmd{\hypcapspace},
%      the default is |0.5\baselineskip|, examples:
%      \begin{quote}
%        |\renewcommand{\hypcapspace}{0pt}| removes the space\\
%        |\renewcommand{\hypcapspace}{1pt}| sets a fix value
%      \end{quote}
%    \item[\cmd{\hypcapredef}:]\DescribeMacro{\hypcapredef}
%      If there are other float environments, that
%      should automatically execute \cmd{\capstart}, then
%      a redefinition with \cmd{\hypcapredef} can be tried:
%      \begin{quote}
%        |\hypcapredef{myfloat}|
%      \end{quote}
%      Only environments with one optional parameter are
%      supported.
%    \item[\cs{capstartfalse}, \cs{capstarttrue}:]^^A
%      \DescribeMacro{\capstartfalse}^^A
%      \DescribeMacro{\capstarttrue}^^A
%      Since 2008/09/08 v1.10.\\
%      They disable and enable \cs{capstart}. They can be used to
%      cancel the effect of a redefined float environment.
%      Example:
%\begin{quote}
%\catcode`Y=\active
%\def Y#1Z{\textit{\cs{#1}}}
%\begin{verbatim}
%\documentclass{article}
%\usepackage{hyperref}
%\usepackage[figure]{hypcap}[2008/09/08]
%
%\begin{document}
%  \section{Hello World}
%  \begin{figure}
%    \caption{Figure with caption A}
%  \end{figure}
%  YcapstartfalseZ
%  \begin{figure}
%    Figure without caption
%  \end{figure}
%  YcapstarttrueZ
%  \begin{figure}
%    \caption{Figure with caption B}
%  \end{figure}
%\end{document}
%\end{verbatim}
%\end{quote}
%    \end{description}
%
% \subsection{Limitations}
%    \begin{itemize}
%    \item Packages that redefine \cmd{\caption} or
%          \cmd{\@caption}.
%    \end{itemize}
%
% \StopEventually{
% }
%
% \section{Implementation}
%    \begin{macrocode}
%<*package>
%    \end{macrocode}
%    Package identification.
%    \begin{macrocode}
\NeedsTeXFormat{LaTeX2e}
\ProvidesPackage{hypcap}%
  [2011/02/16 v1.11 Adjusting the anchors of captions (HO)]
%    \end{macrocode}
%
%    For unique command names this package uses |hc@| as prefix
%    for internal command names.
%
%    First we check, if package \Package{hyperref} is loaded:
%    \begin{macrocode}
\@ifundefined{hyper@@anchor}{%
  \PackageError{hypcap}{You have to load 'hyperref' first}\@ehc
  \endinput
}{}
%    \end{macrocode}
%
%    \begin{macrocode}
\RequirePackage{letltxmacro}[2008/06/24]
%    \end{macrocode}
%
%    \begin{macro}{\hc@org@caption}
%    Save the original meaning of \cmd{\caption}:
%    \begin{macrocode}
\newcommand*\hc@org@caption{}
\let\hc@org@caption\caption
%    \end{macrocode}
%    \end{macro}
%
%    \newcommand*{\BeginMacro}[1]{\expandafter\DoBeginMacro\csname#1\endcsname}
%    \newcommand*{\DoBeginMacro}[1]{\begin{macro}{#1}}
%    \newcommand*{\Cmd}[1]{\expandafter\cmd\csname#1\endcsname}
%    \BeginMacro{if@capstart}
%     The switch \Cmd{if@capstart} helps to detect
%     \cmd{\capstart} commands with missing \cmd{\caption} macros.
%     Because \cmd{\caption} can occur inside a group, assignments
%     to the switch have to be made global.
%    \begin{macrocode}
\newif\if@capstart
%    \end{macrocode}
%    \end{macro}
%
%    \begin{macro}{\hypcapspace}
%    The anchor is raised.by \cmd{\hypcapspace}.
%    \begin{macrocode}
\newcommand*\hypcapspace{.5\baselineskip}
%    \end{macrocode}
%    \end{macro}
%
%    \begin{macro}{\ifcapstart}
%    \begin{macrocode}
\newif\ifcapstart
\capstarttrue
%    \end{macrocode}
%    \end{macro}
%
%    \begin{macro}{\capstart}
%     The macro \cmd{\capstart} contains the first part of
%     the \cmd{\caption} command: Incrementing the counter
%     and setting the anchor.
%    \begin{macrocode}
\newcommand*\capstart{%
  \ifcapstart
    \H@refstepcounter\@captype % first part of caption
    \hyper@makecurrent\@captype
    \global\let\hc@currentHref\@currentHref
    \vspace*{-\hypcapspace}%
    \begingroup
      \let\leavevmode\relax
      \hyper@@anchor\@currentHref\relax
    \endgroup
    \vspace*{\hypcapspace}%
    \hc@hyperref{\let\caption\hc@caption}%
    \global\@capstarttrue
    \global\advance\csname c@\@captype\endcsname\m@ne
  \fi
}
%    \end{macrocode}
%    \end{macro}
%
%    \begin{macrocode}
\@ifpackagelater{hyperref}{2007/04/09}{%
  \let\hc@hyperref\@gobble
}{%
  \let\hc@hyperref\@firstofone
}
%    \end{macrocode}
%
%    \begin{macro}{\hc@caption}
%    The new \cmd{\caption} command without the first part
%    is defined in the macro \cmd{\hc@caption}.
%    \begin{macrocode}
\def\hc@caption{%
  \global\advance\csname c@\@captype\endcsname\@ne
  \@dblarg{\hc@@caption\@captype}%
}
%    \end{macrocode}
%    \end{macro}
%
%    \begin{macro}{\hc@@caption}
%    This is a copy of package \Package{hyperref}'s \cmd{\@caption}
%    macro without making the anchor, because this is already done
%    in \cmd{\capstart}.
%    \begin{macrocode}
\long\def\hc@@caption#1[#2]#3{%
  \let\caption\hc@org@caption
  \global\@capstartfalse
  \ifHy@hypertexnames
    \hyper@makecurrent\@captype
  \else
    \global\let\@currentHref\hc@currentHref
  \fi
  \par\addcontentsline{%
    \csname ext@#1\endcsname}{#1}{%
    \protect\numberline{%
      \csname the#1\endcsname
    }{\ignorespaces #2}%
  }%
  \begingroup
    \@parboxrestore
    \normalsize
    \@makecaption{\csname fnum@#1\endcsname}{%
      \ignorespaces#3%
    }%
    \par
  \endgroup
}
%    \end{macrocode}
%    \end{macro}
%
%    \begin{macro}{\hypcapredef}
%    The macro \cmd{\hypcapredef} prepares the call of
%    \cmd{\hc@redef} that will redefine the environment
%    that is given in the argument.
%    \begin{macrocode}
\def\hypcapredef#1{%
  \expandafter\hc@redef\csname hc@org#1\expandafter\endcsname
                       \csname hc@orgend#1\expandafter\endcsname
                       \expandafter{#1}%
}
%    \end{macrocode}
%    \end{macro}
%    \begin{macro}{\hc@redef}
%    The old meaning of the environment is saved. Then
%    \cmd{\capstart} is appended in the begin part.
%    The end part contains a check
%    that produces an error message in case of \cmd{\capstart}
%    without \cmd{\capstart} (\cmd{\capstart} has incremented
%    the counter).
%    \begin{macrocode}
\def\hc@redef#1#2#3{%
  \newcommand#1{}%
  \expandafter\LetLtxMacro\expandafter#1\csname#3\endcsname
  \expandafter\LetLtxMacro\expandafter#2\csname end#3\endcsname
  \renewenvironment*{#3}[1][]{%
    \ifx\\##1\\%
      #1\relax
    \else
      #1[##1]% hash-ok (compatibility for float)
    \fi
    \capstart
  }{%
    \if@capstart
      \PackageError{hypcap}{You have forgotten to use \string\caption}%
      \global\@capstartfalse
    \else
    \fi
    #2%
  }%
}
%    \end{macrocode}
%    \end{macro}
%
%     At last the options are defined and processed.
%    \begin{macrocode}
\DeclareOption{figure}{\hypcapredef{\CurrentOption}}
\DeclareOption{figure*}{\hypcapredef{\CurrentOption}}
\DeclareOption{table}{\hypcapredef{\CurrentOption}}
\DeclareOption{table*}{\hypcapredef{\CurrentOption}}
\DeclareOption{all}{%
  \hypcapredef{figure}%
  \hypcapredef{figure*}%
  \hypcapredef{table}%
  \hypcapredef{table*}%
}
\ProcessOptions\relax
%    \end{macrocode}
%
%    \begin{macrocode}
%</package>
%    \end{macrocode}
%
% \section{Installation}
%
% \subsection{Download}
%
% \paragraph{Package.} This package is available on
% CTAN\footnote{\url{ftp://ftp.ctan.org/tex-archive/}}:
% \begin{description}
% \item[\CTAN{macros/latex/contrib/oberdiek/hypcap.dtx}] The source file.
% \item[\CTAN{macros/latex/contrib/oberdiek/hypcap.pdf}] Documentation.
% \end{description}
%
%
% \paragraph{Bundle.} All the packages of the bundle `oberdiek'
% are also available in a TDS compliant ZIP archive. There
% the packages are already unpacked and the documentation files
% are generated. The files and directories obey the TDS standard.
% \begin{description}
% \item[\CTAN{install/macros/latex/contrib/oberdiek.tds.zip}]
% \end{description}
% \emph{TDS} refers to the standard ``A Directory Structure
% for \TeX\ Files'' (\CTAN{tds/tds.pdf}). Directories
% with \xfile{texmf} in their name are usually organized this way.
%
% \subsection{Bundle installation}
%
% \paragraph{Unpacking.} Unpack the \xfile{oberdiek.tds.zip} in the
% TDS tree (also known as \xfile{texmf} tree) of your choice.
% Example (linux):
% \begin{quote}
%   |unzip oberdiek.tds.zip -d ~/texmf|
% \end{quote}
%
% \paragraph{Script installation.}
% Check the directory \xfile{TDS:scripts/oberdiek/} for
% scripts that need further installation steps.
% Package \xpackage{attachfile2} comes with the Perl script
% \xfile{pdfatfi.pl} that should be installed in such a way
% that it can be called as \texttt{pdfatfi}.
% Example (linux):
% \begin{quote}
%   |chmod +x scripts/oberdiek/pdfatfi.pl|\\
%   |cp scripts/oberdiek/pdfatfi.pl /usr/local/bin/|
% \end{quote}
%
% \subsection{Package installation}
%
% \paragraph{Unpacking.} The \xfile{.dtx} file is a self-extracting
% \docstrip\ archive. The files are extracted by running the
% \xfile{.dtx} through \plainTeX:
% \begin{quote}
%   \verb|tex hypcap.dtx|
% \end{quote}
%
% \paragraph{TDS.} Now the different files must be moved into
% the different directories in your installation TDS tree
% (also known as \xfile{texmf} tree):
% \begin{quote}
% \def\t{^^A
% \begin{tabular}{@{}>{\ttfamily}l@{ $\rightarrow$ }>{\ttfamily}l@{}}
%   hypcap.sty & tex/latex/oberdiek/hypcap.sty\\
%   hypcap.pdf & doc/latex/oberdiek/hypcap.pdf\\
%   hypcap.dtx & source/latex/oberdiek/hypcap.dtx\\
% \end{tabular}^^A
% }^^A
% \sbox0{\t}^^A
% \ifdim\wd0>\linewidth
%   \begingroup
%     \advance\linewidth by\leftmargin
%     \advance\linewidth by\rightmargin
%   \edef\x{\endgroup
%     \def\noexpand\lw{\the\linewidth}^^A
%   }\x
%   \def\lwbox{^^A
%     \leavevmode
%     \hbox to \linewidth{^^A
%       \kern-\leftmargin\relax
%       \hss
%       \usebox0
%       \hss
%       \kern-\rightmargin\relax
%     }^^A
%   }^^A
%   \ifdim\wd0>\lw
%     \sbox0{\small\t}^^A
%     \ifdim\wd0>\linewidth
%       \ifdim\wd0>\lw
%         \sbox0{\footnotesize\t}^^A
%         \ifdim\wd0>\linewidth
%           \ifdim\wd0>\lw
%             \sbox0{\scriptsize\t}^^A
%             \ifdim\wd0>\linewidth
%               \ifdim\wd0>\lw
%                 \sbox0{\tiny\t}^^A
%                 \ifdim\wd0>\linewidth
%                   \lwbox
%                 \else
%                   \usebox0
%                 \fi
%               \else
%                 \lwbox
%               \fi
%             \else
%               \usebox0
%             \fi
%           \else
%             \lwbox
%           \fi
%         \else
%           \usebox0
%         \fi
%       \else
%         \lwbox
%       \fi
%     \else
%       \usebox0
%     \fi
%   \else
%     \lwbox
%   \fi
% \else
%   \usebox0
% \fi
% \end{quote}
% If you have a \xfile{docstrip.cfg} that configures and enables \docstrip's
% TDS installing feature, then some files can already be in the right
% place, see the documentation of \docstrip.
%
% \subsection{Refresh file name databases}
%
% If your \TeX~distribution
% (\teTeX, \mikTeX, \dots) relies on file name databases, you must refresh
% these. For example, \teTeX\ users run \verb|texhash| or
% \verb|mktexlsr|.
%
% \subsection{Some details for the interested}
%
% \paragraph{Attached source.}
%
% The PDF documentation on CTAN also includes the
% \xfile{.dtx} source file. It can be extracted by
% AcrobatReader 6 or higher. Another option is \textsf{pdftk},
% e.g. unpack the file into the current directory:
% \begin{quote}
%   \verb|pdftk hypcap.pdf unpack_files output .|
% \end{quote}
%
% \paragraph{Unpacking with \LaTeX.}
% The \xfile{.dtx} chooses its action depending on the format:
% \begin{description}
% \item[\plainTeX:] Run \docstrip\ and extract the files.
% \item[\LaTeX:] Generate the documentation.
% \end{description}
% If you insist on using \LaTeX\ for \docstrip\ (really,
% \docstrip\ does not need \LaTeX), then inform the autodetect routine
% about your intention:
% \begin{quote}
%   \verb|latex \let\install=y% \iffalse meta-comment
%
% File: hypcap.dtx
% Version: 2011/02/16 v1.11
% Info: Adjusting the anchors of captions
%
% Copyright (C) 1999-2001, 2006-2008, 2011 by
%    Heiko Oberdiek <heiko.oberdiek at googlemail.com>
%
% This work may be distributed and/or modified under the
% conditions of the LaTeX Project Public License, either
% version 1.3c of this license or (at your option) any later
% version. This version of this license is in
%    http://www.latex-project.org/lppl/lppl-1-3c.txt
% and the latest version of this license is in
%    http://www.latex-project.org/lppl.txt
% and version 1.3 or later is part of all distributions of
% LaTeX version 2005/12/01 or later.
%
% This work has the LPPL maintenance status "maintained".
%
% This Current Maintainer of this work is Heiko Oberdiek.
%
% This work consists of the main source file hypcap.dtx
% and the derived files
%    hypcap.sty, hypcap.pdf, hypcap.ins, hypcap.drv.
%
% Distribution:
%    CTAN:macros/latex/contrib/oberdiek/hypcap.dtx
%    CTAN:macros/latex/contrib/oberdiek/hypcap.pdf
%
% Unpacking:
%    (a) If hypcap.ins is present:
%           tex hypcap.ins
%    (b) Without hypcap.ins:
%           tex hypcap.dtx
%    (c) If you insist on using LaTeX
%           latex \let\install=y\input{hypcap.dtx}
%        (quote the arguments according to the demands of your shell)
%
% Documentation:
%    (a) If hypcap.drv is present:
%           latex hypcap.drv
%    (b) Without hypcap.drv:
%           latex hypcap.dtx; ...
%    The class ltxdoc loads the configuration file ltxdoc.cfg
%    if available. Here you can specify further options, e.g.
%    use A4 as paper format:
%       \PassOptionsToClass{a4paper}{article}
%
%    Programm calls to get the documentation (example):
%       pdflatex hypcap.dtx
%       makeindex -s gind.ist hypcap.idx
%       pdflatex hypcap.dtx
%       makeindex -s gind.ist hypcap.idx
%       pdflatex hypcap.dtx
%
% Installation:
%    TDS:tex/latex/oberdiek/hypcap.sty
%    TDS:doc/latex/oberdiek/hypcap.pdf
%    TDS:source/latex/oberdiek/hypcap.dtx
%
%<*ignore>
\begingroup
  \catcode123=1 %
  \catcode125=2 %
  \def\x{LaTeX2e}%
\expandafter\endgroup
\ifcase 0\ifx\install y1\fi\expandafter
         \ifx\csname processbatchFile\endcsname\relax\else1\fi
         \ifx\fmtname\x\else 1\fi\relax
\else\csname fi\endcsname
%</ignore>
%<*install>
\input docstrip.tex
\Msg{************************************************************************}
\Msg{* Installation}
\Msg{* Package: hypcap 2011/02/16 v1.11 Adjusting the anchors of captions (HO)}
\Msg{************************************************************************}

\keepsilent
\askforoverwritefalse

\let\MetaPrefix\relax
\preamble

This is a generated file.

Project: hypcap
Version: 2011/02/16 v1.11

Copyright (C) 1999-2001, 2006-2008, 2011 by
   Heiko Oberdiek <heiko.oberdiek at googlemail.com>

This work may be distributed and/or modified under the
conditions of the LaTeX Project Public License, either
version 1.3c of this license or (at your option) any later
version. This version of this license is in
   http://www.latex-project.org/lppl/lppl-1-3c.txt
and the latest version of this license is in
   http://www.latex-project.org/lppl.txt
and version 1.3 or later is part of all distributions of
LaTeX version 2005/12/01 or later.

This work has the LPPL maintenance status "maintained".

This Current Maintainer of this work is Heiko Oberdiek.

This work consists of the main source file hypcap.dtx
and the derived files
   hypcap.sty, hypcap.pdf, hypcap.ins, hypcap.drv.

\endpreamble
\let\MetaPrefix\DoubleperCent

\generate{%
  \file{hypcap.ins}{\from{hypcap.dtx}{install}}%
  \file{hypcap.drv}{\from{hypcap.dtx}{driver}}%
  \usedir{tex/latex/oberdiek}%
  \file{hypcap.sty}{\from{hypcap.dtx}{package}}%
  \nopreamble
  \nopostamble
  \usedir{source/latex/oberdiek/catalogue}%
  \file{hypcap.xml}{\from{hypcap.dtx}{catalogue}}%
}

\catcode32=13\relax% active space
\let =\space%
\Msg{************************************************************************}
\Msg{*}
\Msg{* To finish the installation you have to move the following}
\Msg{* file into a directory searched by TeX:}
\Msg{*}
\Msg{*     hypcap.sty}
\Msg{*}
\Msg{* To produce the documentation run the file `hypcap.drv'}
\Msg{* through LaTeX.}
\Msg{*}
\Msg{* Happy TeXing!}
\Msg{*}
\Msg{************************************************************************}

\endbatchfile
%</install>
%<*ignore>
\fi
%</ignore>
%<*driver>
\NeedsTeXFormat{LaTeX2e}
\ProvidesFile{hypcap.drv}%
  [2011/02/16 v1.11 Adjusting the anchors of captions (HO)]%
\documentclass{ltxdoc}
\usepackage{holtxdoc}[2011/11/22]
\begin{document}
  \DocInput{hypcap.dtx}%
\end{document}
%</driver>
% \fi
%
% \CheckSum{174}
%
% \CharacterTable
%  {Upper-case    \A\B\C\D\E\F\G\H\I\J\K\L\M\N\O\P\Q\R\S\T\U\V\W\X\Y\Z
%   Lower-case    \a\b\c\d\e\f\g\h\i\j\k\l\m\n\o\p\q\r\s\t\u\v\w\x\y\z
%   Digits        \0\1\2\3\4\5\6\7\8\9
%   Exclamation   \!     Double quote  \"     Hash (number) \#
%   Dollar        \$     Percent       \%     Ampersand     \&
%   Acute accent  \'     Left paren    \(     Right paren   \)
%   Asterisk      \*     Plus          \+     Comma         \,
%   Minus         \-     Point         \.     Solidus       \/
%   Colon         \:     Semicolon     \;     Less than     \<
%   Equals        \=     Greater than  \>     Question mark \?
%   Commercial at \@     Left bracket  \[     Backslash     \\
%   Right bracket \]     Circumflex    \^     Underscore    \_
%   Grave accent  \`     Left brace    \{     Vertical bar  \|
%   Right brace   \}     Tilde         \~}
%
% \GetFileInfo{hypcap.drv}
%
% \title{The \xpackage{hypcap} package}
% \date{2011/02/16 v1.11}
% \author{Heiko Oberdiek\\\xemail{heiko.oberdiek at googlemail.com}}
%
% \maketitle
%
% \begin{abstract}
% This package tries a solution of the problem with
% hyperref, that links to floats points below the
% caption and not at the beginning of the float.
% Therefore this package divides the task into two
% part, the link setting with \cs{capstart} or
% automatically at the beginning of a float and
% the rest in the \cs{caption} command.
% \end{abstract}
%
% \tableofcontents
%
% \newpage
% \section{Usage}
%    The package \Package{hypcap} requires that \Package{hyperref}
%    is loaded first:
%    \begin{quote}
%      |\usepackage[...]{hyperref}|\\
%      |\usepackage[...]{hypcap}|
%    \end{quote}
%
% \subsection{Package options}
%    The names of the four float environments |figure|, |figure*|,
%    |table|, or |table*| can be used as option. Then the package
%    redefines the environment in order to insert \cmd{\capstart}
%    (see below) in the beginning of the environment automatically.
%
%    Option |all| enables the redefinitions of all four
%    float environments. For other environments see
%    the user command \cmd{\hypcapredef}.
%
% \subsection{User commands}
%    \begin{description}
%    \item[\cmd{\capstart}:]\DescribeMacro{\capstart}
%      First this command increments the counter (\cmd{\@captype}).
%      Then it makes an anchor for package \Package{hyperref}.
%      At last \cmd{\caption} is redefined to remove the
%      anchor setting part from \Package{hyperref}'s \cmd{\caption}.
%
%      The package expects the following structure of a float
%      environment:
%      \begin{quote}
%        |\begin{|\textit{|float|}|}...|\\
%        |\capstart|\\
%        |...|\\
%        |\caption{...}|\\
%        |...|\\
%        |\end{|\textit{|float|}|}|
%      \end{quote}
%      There can be several \cmd{\caption} commands. For these
%      you need \cmd{\capstart} again:
%      \begin{quote}
%        |\capstart ... \caption... \capstart ... \caption...|
%      \end{quote}
%      And the \cmd{\caption} command itself can be put in a group.
%
%      With the options, described above, the extra
%      writing of \cmd{\capstart} can be avoided.
%      Consequently, there must be a \cmd{\caption} in every
%      environment of this type, specified by the option.
%      If you want to use more than one \cmd{\caption} in this
%      environment, you have to state \cmd{\capstart} again.
%    \item[\cmd{\hypcapspace}:]\DescribeMacro{\hypcapspace}
%      Because it looks poor, if the link points exactly at top of
%      the figure, there is additional space: \cmd{\hypcapspace},
%      the default is |0.5\baselineskip|, examples:
%      \begin{quote}
%        |\renewcommand{\hypcapspace}{0pt}| removes the space\\
%        |\renewcommand{\hypcapspace}{1pt}| sets a fix value
%      \end{quote}
%    \item[\cmd{\hypcapredef}:]\DescribeMacro{\hypcapredef}
%      If there are other float environments, that
%      should automatically execute \cmd{\capstart}, then
%      a redefinition with \cmd{\hypcapredef} can be tried:
%      \begin{quote}
%        |\hypcapredef{myfloat}|
%      \end{quote}
%      Only environments with one optional parameter are
%      supported.
%    \item[\cs{capstartfalse}, \cs{capstarttrue}:]^^A
%      \DescribeMacro{\capstartfalse}^^A
%      \DescribeMacro{\capstarttrue}^^A
%      Since 2008/09/08 v1.10.\\
%      They disable and enable \cs{capstart}. They can be used to
%      cancel the effect of a redefined float environment.
%      Example:
%\begin{quote}
%\catcode`Y=\active
%\def Y#1Z{\textit{\cs{#1}}}
%\begin{verbatim}
%\documentclass{article}
%\usepackage{hyperref}
%\usepackage[figure]{hypcap}[2008/09/08]
%
%\begin{document}
%  \section{Hello World}
%  \begin{figure}
%    \caption{Figure with caption A}
%  \end{figure}
%  YcapstartfalseZ
%  \begin{figure}
%    Figure without caption
%  \end{figure}
%  YcapstarttrueZ
%  \begin{figure}
%    \caption{Figure with caption B}
%  \end{figure}
%\end{document}
%\end{verbatim}
%\end{quote}
%    \end{description}
%
% \subsection{Limitations}
%    \begin{itemize}
%    \item Packages that redefine \cmd{\caption} or
%          \cmd{\@caption}.
%    \end{itemize}
%
% \StopEventually{
% }
%
% \section{Implementation}
%    \begin{macrocode}
%<*package>
%    \end{macrocode}
%    Package identification.
%    \begin{macrocode}
\NeedsTeXFormat{LaTeX2e}
\ProvidesPackage{hypcap}%
  [2011/02/16 v1.11 Adjusting the anchors of captions (HO)]
%    \end{macrocode}
%
%    For unique command names this package uses |hc@| as prefix
%    for internal command names.
%
%    First we check, if package \Package{hyperref} is loaded:
%    \begin{macrocode}
\@ifundefined{hyper@@anchor}{%
  \PackageError{hypcap}{You have to load 'hyperref' first}\@ehc
  \endinput
}{}
%    \end{macrocode}
%
%    \begin{macrocode}
\RequirePackage{letltxmacro}[2008/06/24]
%    \end{macrocode}
%
%    \begin{macro}{\hc@org@caption}
%    Save the original meaning of \cmd{\caption}:
%    \begin{macrocode}
\newcommand*\hc@org@caption{}
\let\hc@org@caption\caption
%    \end{macrocode}
%    \end{macro}
%
%    \newcommand*{\BeginMacro}[1]{\expandafter\DoBeginMacro\csname#1\endcsname}
%    \newcommand*{\DoBeginMacro}[1]{\begin{macro}{#1}}
%    \newcommand*{\Cmd}[1]{\expandafter\cmd\csname#1\endcsname}
%    \BeginMacro{if@capstart}
%     The switch \Cmd{if@capstart} helps to detect
%     \cmd{\capstart} commands with missing \cmd{\caption} macros.
%     Because \cmd{\caption} can occur inside a group, assignments
%     to the switch have to be made global.
%    \begin{macrocode}
\newif\if@capstart
%    \end{macrocode}
%    \end{macro}
%
%    \begin{macro}{\hypcapspace}
%    The anchor is raised.by \cmd{\hypcapspace}.
%    \begin{macrocode}
\newcommand*\hypcapspace{.5\baselineskip}
%    \end{macrocode}
%    \end{macro}
%
%    \begin{macro}{\ifcapstart}
%    \begin{macrocode}
\newif\ifcapstart
\capstarttrue
%    \end{macrocode}
%    \end{macro}
%
%    \begin{macro}{\capstart}
%     The macro \cmd{\capstart} contains the first part of
%     the \cmd{\caption} command: Incrementing the counter
%     and setting the anchor.
%    \begin{macrocode}
\newcommand*\capstart{%
  \ifcapstart
    \H@refstepcounter\@captype % first part of caption
    \hyper@makecurrent\@captype
    \global\let\hc@currentHref\@currentHref
    \vspace*{-\hypcapspace}%
    \begingroup
      \let\leavevmode\relax
      \hyper@@anchor\@currentHref\relax
    \endgroup
    \vspace*{\hypcapspace}%
    \hc@hyperref{\let\caption\hc@caption}%
    \global\@capstarttrue
    \global\advance\csname c@\@captype\endcsname\m@ne
  \fi
}
%    \end{macrocode}
%    \end{macro}
%
%    \begin{macrocode}
\@ifpackagelater{hyperref}{2007/04/09}{%
  \let\hc@hyperref\@gobble
}{%
  \let\hc@hyperref\@firstofone
}
%    \end{macrocode}
%
%    \begin{macro}{\hc@caption}
%    The new \cmd{\caption} command without the first part
%    is defined in the macro \cmd{\hc@caption}.
%    \begin{macrocode}
\def\hc@caption{%
  \global\advance\csname c@\@captype\endcsname\@ne
  \@dblarg{\hc@@caption\@captype}%
}
%    \end{macrocode}
%    \end{macro}
%
%    \begin{macro}{\hc@@caption}
%    This is a copy of package \Package{hyperref}'s \cmd{\@caption}
%    macro without making the anchor, because this is already done
%    in \cmd{\capstart}.
%    \begin{macrocode}
\long\def\hc@@caption#1[#2]#3{%
  \let\caption\hc@org@caption
  \global\@capstartfalse
  \ifHy@hypertexnames
    \hyper@makecurrent\@captype
  \else
    \global\let\@currentHref\hc@currentHref
  \fi
  \par\addcontentsline{%
    \csname ext@#1\endcsname}{#1}{%
    \protect\numberline{%
      \csname the#1\endcsname
    }{\ignorespaces #2}%
  }%
  \begingroup
    \@parboxrestore
    \normalsize
    \@makecaption{\csname fnum@#1\endcsname}{%
      \ignorespaces#3%
    }%
    \par
  \endgroup
}
%    \end{macrocode}
%    \end{macro}
%
%    \begin{macro}{\hypcapredef}
%    The macro \cmd{\hypcapredef} prepares the call of
%    \cmd{\hc@redef} that will redefine the environment
%    that is given in the argument.
%    \begin{macrocode}
\def\hypcapredef#1{%
  \expandafter\hc@redef\csname hc@org#1\expandafter\endcsname
                       \csname hc@orgend#1\expandafter\endcsname
                       \expandafter{#1}%
}
%    \end{macrocode}
%    \end{macro}
%    \begin{macro}{\hc@redef}
%    The old meaning of the environment is saved. Then
%    \cmd{\capstart} is appended in the begin part.
%    The end part contains a check
%    that produces an error message in case of \cmd{\capstart}
%    without \cmd{\capstart} (\cmd{\capstart} has incremented
%    the counter).
%    \begin{macrocode}
\def\hc@redef#1#2#3{%
  \newcommand#1{}%
  \expandafter\LetLtxMacro\expandafter#1\csname#3\endcsname
  \expandafter\LetLtxMacro\expandafter#2\csname end#3\endcsname
  \renewenvironment*{#3}[1][]{%
    \ifx\\##1\\%
      #1\relax
    \else
      #1[##1]% hash-ok (compatibility for float)
    \fi
    \capstart
  }{%
    \if@capstart
      \PackageError{hypcap}{You have forgotten to use \string\caption}%
      \global\@capstartfalse
    \else
    \fi
    #2%
  }%
}
%    \end{macrocode}
%    \end{macro}
%
%     At last the options are defined and processed.
%    \begin{macrocode}
\DeclareOption{figure}{\hypcapredef{\CurrentOption}}
\DeclareOption{figure*}{\hypcapredef{\CurrentOption}}
\DeclareOption{table}{\hypcapredef{\CurrentOption}}
\DeclareOption{table*}{\hypcapredef{\CurrentOption}}
\DeclareOption{all}{%
  \hypcapredef{figure}%
  \hypcapredef{figure*}%
  \hypcapredef{table}%
  \hypcapredef{table*}%
}
\ProcessOptions\relax
%    \end{macrocode}
%
%    \begin{macrocode}
%</package>
%    \end{macrocode}
%
% \section{Installation}
%
% \subsection{Download}
%
% \paragraph{Package.} This package is available on
% CTAN\footnote{\url{ftp://ftp.ctan.org/tex-archive/}}:
% \begin{description}
% \item[\CTAN{macros/latex/contrib/oberdiek/hypcap.dtx}] The source file.
% \item[\CTAN{macros/latex/contrib/oberdiek/hypcap.pdf}] Documentation.
% \end{description}
%
%
% \paragraph{Bundle.} All the packages of the bundle `oberdiek'
% are also available in a TDS compliant ZIP archive. There
% the packages are already unpacked and the documentation files
% are generated. The files and directories obey the TDS standard.
% \begin{description}
% \item[\CTAN{install/macros/latex/contrib/oberdiek.tds.zip}]
% \end{description}
% \emph{TDS} refers to the standard ``A Directory Structure
% for \TeX\ Files'' (\CTAN{tds/tds.pdf}). Directories
% with \xfile{texmf} in their name are usually organized this way.
%
% \subsection{Bundle installation}
%
% \paragraph{Unpacking.} Unpack the \xfile{oberdiek.tds.zip} in the
% TDS tree (also known as \xfile{texmf} tree) of your choice.
% Example (linux):
% \begin{quote}
%   |unzip oberdiek.tds.zip -d ~/texmf|
% \end{quote}
%
% \paragraph{Script installation.}
% Check the directory \xfile{TDS:scripts/oberdiek/} for
% scripts that need further installation steps.
% Package \xpackage{attachfile2} comes with the Perl script
% \xfile{pdfatfi.pl} that should be installed in such a way
% that it can be called as \texttt{pdfatfi}.
% Example (linux):
% \begin{quote}
%   |chmod +x scripts/oberdiek/pdfatfi.pl|\\
%   |cp scripts/oberdiek/pdfatfi.pl /usr/local/bin/|
% \end{quote}
%
% \subsection{Package installation}
%
% \paragraph{Unpacking.} The \xfile{.dtx} file is a self-extracting
% \docstrip\ archive. The files are extracted by running the
% \xfile{.dtx} through \plainTeX:
% \begin{quote}
%   \verb|tex hypcap.dtx|
% \end{quote}
%
% \paragraph{TDS.} Now the different files must be moved into
% the different directories in your installation TDS tree
% (also known as \xfile{texmf} tree):
% \begin{quote}
% \def\t{^^A
% \begin{tabular}{@{}>{\ttfamily}l@{ $\rightarrow$ }>{\ttfamily}l@{}}
%   hypcap.sty & tex/latex/oberdiek/hypcap.sty\\
%   hypcap.pdf & doc/latex/oberdiek/hypcap.pdf\\
%   hypcap.dtx & source/latex/oberdiek/hypcap.dtx\\
% \end{tabular}^^A
% }^^A
% \sbox0{\t}^^A
% \ifdim\wd0>\linewidth
%   \begingroup
%     \advance\linewidth by\leftmargin
%     \advance\linewidth by\rightmargin
%   \edef\x{\endgroup
%     \def\noexpand\lw{\the\linewidth}^^A
%   }\x
%   \def\lwbox{^^A
%     \leavevmode
%     \hbox to \linewidth{^^A
%       \kern-\leftmargin\relax
%       \hss
%       \usebox0
%       \hss
%       \kern-\rightmargin\relax
%     }^^A
%   }^^A
%   \ifdim\wd0>\lw
%     \sbox0{\small\t}^^A
%     \ifdim\wd0>\linewidth
%       \ifdim\wd0>\lw
%         \sbox0{\footnotesize\t}^^A
%         \ifdim\wd0>\linewidth
%           \ifdim\wd0>\lw
%             \sbox0{\scriptsize\t}^^A
%             \ifdim\wd0>\linewidth
%               \ifdim\wd0>\lw
%                 \sbox0{\tiny\t}^^A
%                 \ifdim\wd0>\linewidth
%                   \lwbox
%                 \else
%                   \usebox0
%                 \fi
%               \else
%                 \lwbox
%               \fi
%             \else
%               \usebox0
%             \fi
%           \else
%             \lwbox
%           \fi
%         \else
%           \usebox0
%         \fi
%       \else
%         \lwbox
%       \fi
%     \else
%       \usebox0
%     \fi
%   \else
%     \lwbox
%   \fi
% \else
%   \usebox0
% \fi
% \end{quote}
% If you have a \xfile{docstrip.cfg} that configures and enables \docstrip's
% TDS installing feature, then some files can already be in the right
% place, see the documentation of \docstrip.
%
% \subsection{Refresh file name databases}
%
% If your \TeX~distribution
% (\teTeX, \mikTeX, \dots) relies on file name databases, you must refresh
% these. For example, \teTeX\ users run \verb|texhash| or
% \verb|mktexlsr|.
%
% \subsection{Some details for the interested}
%
% \paragraph{Attached source.}
%
% The PDF documentation on CTAN also includes the
% \xfile{.dtx} source file. It can be extracted by
% AcrobatReader 6 or higher. Another option is \textsf{pdftk},
% e.g. unpack the file into the current directory:
% \begin{quote}
%   \verb|pdftk hypcap.pdf unpack_files output .|
% \end{quote}
%
% \paragraph{Unpacking with \LaTeX.}
% The \xfile{.dtx} chooses its action depending on the format:
% \begin{description}
% \item[\plainTeX:] Run \docstrip\ and extract the files.
% \item[\LaTeX:] Generate the documentation.
% \end{description}
% If you insist on using \LaTeX\ for \docstrip\ (really,
% \docstrip\ does not need \LaTeX), then inform the autodetect routine
% about your intention:
% \begin{quote}
%   \verb|latex \let\install=y\input{hypcap.dtx}|
% \end{quote}
% Do not forget to quote the argument according to the demands
% of your shell.
%
% \paragraph{Generating the documentation.}
% You can use both the \xfile{.dtx} or the \xfile{.drv} to generate
% the documentation. The process can be configured by the
% configuration file \xfile{ltxdoc.cfg}. For instance, put this
% line into this file, if you want to have A4 as paper format:
% \begin{quote}
%   \verb|\PassOptionsToClass{a4paper}{article}|
% \end{quote}
% An example follows how to generate the
% documentation with pdf\LaTeX:
% \begin{quote}
%\begin{verbatim}
%pdflatex hypcap.dtx
%makeindex -s gind.ist hypcap.idx
%pdflatex hypcap.dtx
%makeindex -s gind.ist hypcap.idx
%pdflatex hypcap.dtx
%\end{verbatim}
% \end{quote}
%
% \section{Catalogue}
%
% The following XML file can be used as source for the
% \href{http://mirror.ctan.org/help/Catalogue/catalogue.html}{\TeX\ Catalogue}.
% The elements \texttt{caption} and \texttt{description} are imported
% from the original XML file from the Catalogue.
% The name of the XML file in the Catalogue is \xfile{hypcap.xml}.
%    \begin{macrocode}
%<*catalogue>
<?xml version='1.0' encoding='us-ascii'?>
<!DOCTYPE entry SYSTEM 'catalogue.dtd'>
<entry datestamp='$Date$' modifier='$Author$' id='hypcap'>
  <name>hypcap</name>
  <caption>Adjusting the anchors of captions.</caption>
  <authorref id='auth:oberdiek'/>
  <copyright owner='Heiko Oberdiek' year='1999-2001,2006-2008,2011'/>
  <license type='lppl1.3'/>
  <version number='1.11'/>
  <description>
    The package offers a solution to the problem that when you link to
    a float using <xref refid='hyperref'>hyperref</xref>, the link
    anchors to below the float's caption, rather than the beginning of
    the float.
    <p/>
    Hypcap defines a separate \capstart command, which you put where
    you want links to end; you should have a \capstart command for each
    \caption command.  Package options can be used to auto-insert a
    \capstart at the start of a float environment.
    <p/>
    The package is part of the <xref refid='oberdiek'>oberdiek</xref> bundle.
  </description>
  <documentation details='Package documentation'
      href='ctan:/macros/latex/contrib/oberdiek/hypcap.pdf'/>
  <ctan file='true' path='/macros/latex/contrib/oberdiek/hypcap.dtx'/>
  <miktex location='oberdiek'/>
  <texlive location='oberdiek'/>
  <install path='/macros/latex/contrib/oberdiek/oberdiek.tds.zip'/>
</entry>
%</catalogue>
%    \end{macrocode}
%
% \begin{History}
%   \begin{Version}{1999/02/13 v1.0}
%   \item
%     A beginning version, published in newsgroup \xnewsgroup{comp.text.tex}:\\
%     \URL{``\link{Re: \xpackage{hyperref} and figures}''}^^A
%     {http://groups.google.com/group/comp.text.tex/msg/5c9b47b001a9379c}
%   \end{Version}
%   \begin{Version}{2000/08/14 v1.1}
%   \item
%     Global assignments of \cs{if@capstart}
%     in order to allow \cs{caption} in groups.
%   \item
%     Option \texttt{all} added.
%   \end{Version}
%   \begin{Version}{2000/09/07 v1.2}
%   \item
%     Package in dtx format.
%   \end{Version}
%   \begin{Version}{2001/08/27 v1.3}
%   \item
%     Bug fix with hyperref's pdfmark driver\\
%     (\cs{leavevmode} in
%     \cs{hyper@@anchor}/\cs{pdf@rect}).
%   \end{Version}
%   \begin{Version}{2001/09/06 v1.4}
%   \item
%     Small fixes in the dtx file.
%   \end{Version}
%   \begin{Version}{2006/02/20 v1.5}
%   \item
%     Code is not changed.
%   \item
%     New DTX framework.
%   \end{Version}
%   \begin{Version}{2007/02/19 v1.6}
%   \item
%     Fix for hypertexnames=false.
%   \end{Version}
%   \begin{Version}{2007/04/09 v1.7}
%   \item
%     Stuff in \cs{caption} moved to \xpackage{hyperref}.
%     This avoids redefinitions of \cs{caption} and \cs{@caption}
%     (idea of Axel Sommerfeldt).
%   \item
%     Fix for subfigure (Marco Kuhlmann, Amilcar do Carmo Lucas).
%   \end{Version}
%   \begin{Version}{2008/04/14 v1.8}
%   \item
%     \cs{hc@redef} fixed to get package \xpackage{float} work
%     (Axel Sommerfeldt).
%   \end{Version}
%   \begin{Version}{2008/08/11 v1.9}
%   \item
%     Code is not changed.
%   \item
%     URLs updated.
%   \end{Version}
%   \begin{Version}{2008/09/08 v1.10}
%   \item
%     \cs{capstartfalse} and \cs{capstarttrue} added.
%   \end{Version}
%   \begin{Version}{2011/02/16 v1.11}
%   \item
%     \cs{hc@redef} fixed by using package \xpackage{letltxmacro}.
%   \end{Version}
% \end{History}
%
% \PrintIndex
%
% \Finale
\endinput
|
% \end{quote}
% Do not forget to quote the argument according to the demands
% of your shell.
%
% \paragraph{Generating the documentation.}
% You can use both the \xfile{.dtx} or the \xfile{.drv} to generate
% the documentation. The process can be configured by the
% configuration file \xfile{ltxdoc.cfg}. For instance, put this
% line into this file, if you want to have A4 as paper format:
% \begin{quote}
%   \verb|\PassOptionsToClass{a4paper}{article}|
% \end{quote}
% An example follows how to generate the
% documentation with pdf\LaTeX:
% \begin{quote}
%\begin{verbatim}
%pdflatex hypcap.dtx
%makeindex -s gind.ist hypcap.idx
%pdflatex hypcap.dtx
%makeindex -s gind.ist hypcap.idx
%pdflatex hypcap.dtx
%\end{verbatim}
% \end{quote}
%
% \section{Catalogue}
%
% The following XML file can be used as source for the
% \href{http://mirror.ctan.org/help/Catalogue/catalogue.html}{\TeX\ Catalogue}.
% The elements \texttt{caption} and \texttt{description} are imported
% from the original XML file from the Catalogue.
% The name of the XML file in the Catalogue is \xfile{hypcap.xml}.
%    \begin{macrocode}
%<*catalogue>
<?xml version='1.0' encoding='us-ascii'?>
<!DOCTYPE entry SYSTEM 'catalogue.dtd'>
<entry datestamp='$Date$' modifier='$Author$' id='hypcap'>
  <name>hypcap</name>
  <caption>Adjusting the anchors of captions.</caption>
  <authorref id='auth:oberdiek'/>
  <copyright owner='Heiko Oberdiek' year='1999-2001,2006-2008,2011'/>
  <license type='lppl1.3'/>
  <version number='1.11'/>
  <description>
    The package offers a solution to the problem that when you link to
    a float using <xref refid='hyperref'>hyperref</xref>, the link
    anchors to below the float's caption, rather than the beginning of
    the float.
    <p/>
    Hypcap defines a separate \capstart command, which you put where
    you want links to end; you should have a \capstart command for each
    \caption command.  Package options can be used to auto-insert a
    \capstart at the start of a float environment.
    <p/>
    The package is part of the <xref refid='oberdiek'>oberdiek</xref> bundle.
  </description>
  <documentation details='Package documentation'
      href='ctan:/macros/latex/contrib/oberdiek/hypcap.pdf'/>
  <ctan file='true' path='/macros/latex/contrib/oberdiek/hypcap.dtx'/>
  <miktex location='oberdiek'/>
  <texlive location='oberdiek'/>
  <install path='/macros/latex/contrib/oberdiek/oberdiek.tds.zip'/>
</entry>
%</catalogue>
%    \end{macrocode}
%
% \begin{History}
%   \begin{Version}{1999/02/13 v1.0}
%   \item
%     A beginning version, published in newsgroup \xnewsgroup{comp.text.tex}:\\
%     \URL{``\link{Re: \xpackage{hyperref} and figures}''}^^A
%     {http://groups.google.com/group/comp.text.tex/msg/5c9b47b001a9379c}
%   \end{Version}
%   \begin{Version}{2000/08/14 v1.1}
%   \item
%     Global assignments of \cs{if@capstart}
%     in order to allow \cs{caption} in groups.
%   \item
%     Option \texttt{all} added.
%   \end{Version}
%   \begin{Version}{2000/09/07 v1.2}
%   \item
%     Package in dtx format.
%   \end{Version}
%   \begin{Version}{2001/08/27 v1.3}
%   \item
%     Bug fix with hyperref's pdfmark driver\\
%     (\cs{leavevmode} in
%     \cs{hyper@@anchor}/\cs{pdf@rect}).
%   \end{Version}
%   \begin{Version}{2001/09/06 v1.4}
%   \item
%     Small fixes in the dtx file.
%   \end{Version}
%   \begin{Version}{2006/02/20 v1.5}
%   \item
%     Code is not changed.
%   \item
%     New DTX framework.
%   \end{Version}
%   \begin{Version}{2007/02/19 v1.6}
%   \item
%     Fix for hypertexnames=false.
%   \end{Version}
%   \begin{Version}{2007/04/09 v1.7}
%   \item
%     Stuff in \cs{caption} moved to \xpackage{hyperref}.
%     This avoids redefinitions of \cs{caption} and \cs{@caption}
%     (idea of Axel Sommerfeldt).
%   \item
%     Fix for subfigure (Marco Kuhlmann, Amilcar do Carmo Lucas).
%   \end{Version}
%   \begin{Version}{2008/04/14 v1.8}
%   \item
%     \cs{hc@redef} fixed to get package \xpackage{float} work
%     (Axel Sommerfeldt).
%   \end{Version}
%   \begin{Version}{2008/08/11 v1.9}
%   \item
%     Code is not changed.
%   \item
%     URLs updated.
%   \end{Version}
%   \begin{Version}{2008/09/08 v1.10}
%   \item
%     \cs{capstartfalse} and \cs{capstarttrue} added.
%   \end{Version}
%   \begin{Version}{2011/02/16 v1.11}
%   \item
%     \cs{hc@redef} fixed by using package \xpackage{letltxmacro}.
%   \end{Version}
% \end{History}
%
% \PrintIndex
%
% \Finale
\endinput
|
% \end{quote}
% Do not forget to quote the argument according to the demands
% of your shell.
%
% \paragraph{Generating the documentation.}
% You can use both the \xfile{.dtx} or the \xfile{.drv} to generate
% the documentation. The process can be configured by the
% configuration file \xfile{ltxdoc.cfg}. For instance, put this
% line into this file, if you want to have A4 as paper format:
% \begin{quote}
%   \verb|\PassOptionsToClass{a4paper}{article}|
% \end{quote}
% An example follows how to generate the
% documentation with pdf\LaTeX:
% \begin{quote}
%\begin{verbatim}
%pdflatex hypcap.dtx
%makeindex -s gind.ist hypcap.idx
%pdflatex hypcap.dtx
%makeindex -s gind.ist hypcap.idx
%pdflatex hypcap.dtx
%\end{verbatim}
% \end{quote}
%
% \section{Catalogue}
%
% The following XML file can be used as source for the
% \href{http://mirror.ctan.org/help/Catalogue/catalogue.html}{\TeX\ Catalogue}.
% The elements \texttt{caption} and \texttt{description} are imported
% from the original XML file from the Catalogue.
% The name of the XML file in the Catalogue is \xfile{hypcap.xml}.
%    \begin{macrocode}
%<*catalogue>
<?xml version='1.0' encoding='us-ascii'?>
<!DOCTYPE entry SYSTEM 'catalogue.dtd'>
<entry datestamp='$Date$' modifier='$Author$' id='hypcap'>
  <name>hypcap</name>
  <caption>Adjusting the anchors of captions.</caption>
  <authorref id='auth:oberdiek'/>
  <copyright owner='Heiko Oberdiek' year='1999-2001,2006-2008,2011'/>
  <license type='lppl1.3'/>
  <version number='1.11'/>
  <description>
    The package offers a solution to the problem that when you link to
    a float using <xref refid='hyperref'>hyperref</xref>, the link
    anchors to below the float's caption, rather than the beginning of
    the float.
    <p/>
    Hypcap defines a separate \capstart command, which you put where
    you want links to end; you should have a \capstart command for each
    \caption command.  Package options can be used to auto-insert a
    \capstart at the start of a float environment.
    <p/>
    The package is part of the <xref refid='oberdiek'>oberdiek</xref> bundle.
  </description>
  <documentation details='Package documentation'
      href='ctan:/macros/latex/contrib/oberdiek/hypcap.pdf'/>
  <ctan file='true' path='/macros/latex/contrib/oberdiek/hypcap.dtx'/>
  <miktex location='oberdiek'/>
  <texlive location='oberdiek'/>
  <install path='/macros/latex/contrib/oberdiek/oberdiek.tds.zip'/>
</entry>
%</catalogue>
%    \end{macrocode}
%
% \begin{History}
%   \begin{Version}{1999/02/13 v1.0}
%   \item
%     A beginning version, published in newsgroup \xnewsgroup{comp.text.tex}:\\
%     \URL{``\link{Re: \xpackage{hyperref} and figures}''}^^A
%     {http://groups.google.com/group/comp.text.tex/msg/5c9b47b001a9379c}
%   \end{Version}
%   \begin{Version}{2000/08/14 v1.1}
%   \item
%     Global assignments of \cs{if@capstart}
%     in order to allow \cs{caption} in groups.
%   \item
%     Option \texttt{all} added.
%   \end{Version}
%   \begin{Version}{2000/09/07 v1.2}
%   \item
%     Package in dtx format.
%   \end{Version}
%   \begin{Version}{2001/08/27 v1.3}
%   \item
%     Bug fix with hyperref's pdfmark driver\\
%     (\cs{leavevmode} in
%     \cs{hyper@@anchor}/\cs{pdf@rect}).
%   \end{Version}
%   \begin{Version}{2001/09/06 v1.4}
%   \item
%     Small fixes in the dtx file.
%   \end{Version}
%   \begin{Version}{2006/02/20 v1.5}
%   \item
%     Code is not changed.
%   \item
%     New DTX framework.
%   \end{Version}
%   \begin{Version}{2007/02/19 v1.6}
%   \item
%     Fix for hypertexnames=false.
%   \end{Version}
%   \begin{Version}{2007/04/09 v1.7}
%   \item
%     Stuff in \cs{caption} moved to \xpackage{hyperref}.
%     This avoids redefinitions of \cs{caption} and \cs{@caption}
%     (idea of Axel Sommerfeldt).
%   \item
%     Fix for subfigure (Marco Kuhlmann, Amilcar do Carmo Lucas).
%   \end{Version}
%   \begin{Version}{2008/04/14 v1.8}
%   \item
%     \cs{hc@redef} fixed to get package \xpackage{float} work
%     (Axel Sommerfeldt).
%   \end{Version}
%   \begin{Version}{2008/08/11 v1.9}
%   \item
%     Code is not changed.
%   \item
%     URLs updated.
%   \end{Version}
%   \begin{Version}{2008/09/08 v1.10}
%   \item
%     \cs{capstartfalse} and \cs{capstarttrue} added.
%   \end{Version}
%   \begin{Version}{2011/02/16 v1.11}
%   \item
%     \cs{hc@redef} fixed by using package \xpackage{letltxmacro}.
%   \end{Version}
% \end{History}
%
% \PrintIndex
%
% \Finale
\endinput
|
% \end{quote}
% Do not forget to quote the argument according to the demands
% of your shell.
%
% \paragraph{Generating the documentation.}
% You can use both the \xfile{.dtx} or the \xfile{.drv} to generate
% the documentation. The process can be configured by the
% configuration file \xfile{ltxdoc.cfg}. For instance, put this
% line into this file, if you want to have A4 as paper format:
% \begin{quote}
%   \verb|\PassOptionsToClass{a4paper}{article}|
% \end{quote}
% An example follows how to generate the
% documentation with pdf\LaTeX:
% \begin{quote}
%\begin{verbatim}
%pdflatex hypcap.dtx
%makeindex -s gind.ist hypcap.idx
%pdflatex hypcap.dtx
%makeindex -s gind.ist hypcap.idx
%pdflatex hypcap.dtx
%\end{verbatim}
% \end{quote}
%
% \section{Catalogue}
%
% The following XML file can be used as source for the
% \href{http://mirror.ctan.org/help/Catalogue/catalogue.html}{\TeX\ Catalogue}.
% The elements \texttt{caption} and \texttt{description} are imported
% from the original XML file from the Catalogue.
% The name of the XML file in the Catalogue is \xfile{hypcap.xml}.
%    \begin{macrocode}
%<*catalogue>
<?xml version='1.0' encoding='us-ascii'?>
<!DOCTYPE entry SYSTEM 'catalogue.dtd'>
<entry datestamp='$Date$' modifier='$Author$' id='hypcap'>
  <name>hypcap</name>
  <caption>Adjusting the anchors of captions.</caption>
  <authorref id='auth:oberdiek'/>
  <copyright owner='Heiko Oberdiek' year='1999-2001,2006-2008,2011'/>
  <license type='lppl1.3'/>
  <version number='1.11'/>
  <description>
    The package offers a solution to the problem that when you link to
    a float using <xref refid='hyperref'>hyperref</xref>, the link
    anchors to below the float's caption, rather than the beginning of
    the float.
    <p/>
    Hypcap defines a separate \capstart command, which you put where
    you want links to end; you should have a \capstart command for each
    \caption command.  Package options can be used to auto-insert a
    \capstart at the start of a float environment.
    <p/>
    The package is part of the <xref refid='oberdiek'>oberdiek</xref> bundle.
  </description>
  <documentation details='Package documentation'
      href='ctan:/macros/latex/contrib/oberdiek/hypcap.pdf'/>
  <ctan file='true' path='/macros/latex/contrib/oberdiek/hypcap.dtx'/>
  <miktex location='oberdiek'/>
  <texlive location='oberdiek'/>
  <install path='/macros/latex/contrib/oberdiek/oberdiek.tds.zip'/>
</entry>
%</catalogue>
%    \end{macrocode}
%
% \begin{History}
%   \begin{Version}{1999/02/13 v1.0}
%   \item
%     A beginning version, published in newsgroup \xnewsgroup{comp.text.tex}:\\
%     \URL{``\link{Re: \xpackage{hyperref} and figures}''}^^A
%     {http://groups.google.com/group/comp.text.tex/msg/5c9b47b001a9379c}
%   \end{Version}
%   \begin{Version}{2000/08/14 v1.1}
%   \item
%     Global assignments of \cs{if@capstart}
%     in order to allow \cs{caption} in groups.
%   \item
%     Option \texttt{all} added.
%   \end{Version}
%   \begin{Version}{2000/09/07 v1.2}
%   \item
%     Package in dtx format.
%   \end{Version}
%   \begin{Version}{2001/08/27 v1.3}
%   \item
%     Bug fix with hyperref's pdfmark driver\\
%     (\cs{leavevmode} in
%     \cs{hyper@@anchor}/\cs{pdf@rect}).
%   \end{Version}
%   \begin{Version}{2001/09/06 v1.4}
%   \item
%     Small fixes in the dtx file.
%   \end{Version}
%   \begin{Version}{2006/02/20 v1.5}
%   \item
%     Code is not changed.
%   \item
%     New DTX framework.
%   \end{Version}
%   \begin{Version}{2007/02/19 v1.6}
%   \item
%     Fix for hypertexnames=false.
%   \end{Version}
%   \begin{Version}{2007/04/09 v1.7}
%   \item
%     Stuff in \cs{caption} moved to \xpackage{hyperref}.
%     This avoids redefinitions of \cs{caption} and \cs{@caption}
%     (idea of Axel Sommerfeldt).
%   \item
%     Fix for subfigure (Marco Kuhlmann, Amilcar do Carmo Lucas).
%   \end{Version}
%   \begin{Version}{2008/04/14 v1.8}
%   \item
%     \cs{hc@redef} fixed to get package \xpackage{float} work
%     (Axel Sommerfeldt).
%   \end{Version}
%   \begin{Version}{2008/08/11 v1.9}
%   \item
%     Code is not changed.
%   \item
%     URLs updated.
%   \end{Version}
%   \begin{Version}{2008/09/08 v1.10}
%   \item
%     \cs{capstartfalse} and \cs{capstarttrue} added.
%   \end{Version}
%   \begin{Version}{2011/02/16 v1.11}
%   \item
%     \cs{hc@redef} fixed by using package \xpackage{letltxmacro}.
%   \end{Version}
% \end{History}
%
% \PrintIndex
%
% \Finale
\endinput
