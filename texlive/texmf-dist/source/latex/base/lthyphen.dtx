% \iffalse meta-comment
%
% Copyright 1993 1994 1995 1996 1997 1998 1999 2000 2001 2002 2003 2004 2005 2006 2007 2008 2009
% The LaTeX3 Project and any individual authors listed elsewhere
% in this file. 
% 
% This file is part of the LaTeX base system.
% -------------------------------------------
% 
% It may be distributed and/or modified under the
% conditions of the LaTeX Project Public License, either version 1.3c
% of this license or (at your option) any later version.
% The latest version of this license is in
%    http://www.latex-project.org/lppl.txt
% and version 1.3c or later is part of all distributions of LaTeX 
% version 2005/12/01 or later.
% 
% This file has the LPPL maintenance status "maintained".
% 
% The list of all files belonging to the LaTeX base distribution is
% given in the file `manifest.txt'. See also `legal.txt' for additional
% information.
% 
% The list of derived (unpacked) files belonging to the distribution 
% and covered by LPPL is defined by the unpacking scripts (with 
% extension .ins) which are part of the distribution.
% 
% \fi
% \iffalse
%<*driver>
\ProvidesFile{lthyphen.dtx}
%</driver>
%<default>\ProvidesFile{hyphen.ltx}
% \fi
%         \ProvidesFile{lthyphen.dtx}
          [1994/12/04 v1.0h LaTeX Kernel (hyphenation interface)]
%
%
% \CheckSum{13}
%% \CharacterTable
%%  {Upper-case    \A\B\C\D\E\F\G\H\I\J\K\L\M\N\O\P\Q\R\S\T\U\V\W\X\Y\Z
%%   Lower-case    \a\b\c\d\e\f\g\h\i\j\k\l\m\n\o\p\q\r\s\t\u\v\w\x\y\z
%%   Digits        \0\1\2\3\4\5\6\7\8\9
%%   Exclamation   \!     Double quote  \"     Hash (number) \#
%%   Dollar        \$     Percent       \%     Ampersand     \&
%%   Acute accent  \'     Left paren    \(     Right paren   \)
%%   Asterisk      \*     Plus          \+     Comma         \,
%%   Minus         \-     Point         \.     Solidus       \/
%%   Colon         \:     Semicolon     \;     Less than     \<
%%   Equals        \=     Greater than  \>     Question mark \?
%%   Commercial at \@     Left bracket  \[     Backslash     \\
%%   Right bracket \]     Circumflex    \^     Underscore    \_
%%   Grave accent  \`     Left brace    \{     Vertical bar  \|
%%   Right brace   \}     Tilde         \~}
%
%\iffalse        This is a META comment
%
% File `lthyphen.dtx'.
% Copyright (C) 1994-94 LaTeX3 project, Frank Mittelbach and
% Rainer Sch\"opf, all rights reserved.
%
%\fi
% \GetFileInfo{lthyphen.dtx}
% \title{The \texttt{lthyphen.dtx} file\thanks
%     {This file has version number \fileversion, dated \filedate.}\\
%       for use with \LaTeXe}
% \author{Frank Mittelbach, Chris Rowley \and Rainer Sch\"opf}
%
% \def\dst{{\normalfont\scshape docstrip}}
% \setcounter{StandardModuleDepth}{1}
%
%
% \maketitle
%
% This file contains the code for loading hyphenation patterns into
% \LaTeX. Most of this will end up in a file called
% \texttt{hyphen.ltx}. If you wish to customize your \LaTeX{} system
% in respect of hyphenation patterns, write a file
% \texttt{hyphen.cfg}. If this file exists, it will be loaded instead
% of \texttt{hyphen.ltx}.  See the comments below for additional
% information.
% 
% \StopEventually{}
%
% To produce the printed version of this file the following code
% is used. It can be extracted with the \dst{} program, or one can run
% this file directly through \LaTeXe{}.
%    \begin{macrocode}
%<*driver>
\documentclass{ltxdoc}
\begin{document}
\DocInput{lthyphen.dtx}
\end{document}
%</driver>
%    \end{macrocode}
%
% \changes{v0.1c}{1994/03/07}{move the 2ekernel code to ltfinal.dtx}
% \changes{v1.0g}{1994/12/01}{Rename lthyphen.ltx/cfg to hyphen.ltx/cfg}
% \changes{v1.0h}{1994/12/04}{Documentation edits for /1989}
%
% The default file |hyphen.ltx| loads hyphenation patterns for US
% english.  If you want to load additional or other hyphenation
% patterns, you should create a file |hyphen.cfg|. This is best done
% by starting from |hyphen.ltx|.
%
% For backward compatibility, the default file, |hyphen.ltx|,
% first tries to load
% the file |hyphen.tex|. If this file exists, an information
% message is issued and the appropriate defaults for \TeX's internal
% parameters are set: |\language| is initialized to $0$, and
% |\lefthyphenmin| and |\righthyphenmin| to $2$ and $3$, respectively,
% to disallow x- or -xx breaks.
%    \begin{macrocode}
%<*default>
\InputIfFileExists{hyphen.tex}%
   {\message{Loading hyphenation patterns for US english.}%
    \language=0
    \lefthyphenmin=2 \righthyphenmin=3 }%
%    \end{macrocode}
% Otherwise, since we cannot do anything without any hyphenation
% patterns, an error message is printed and the Ini\TeX{} run
% is terminated by invoking |\@@end| (which is the \LaTeXe{} name
% for \TeX's |\end| primitive).
%    \begin{macrocode}
   {\errhelp{The configuration for hyphenation is incorrectly
             installed.^^J%
             If you don't understand this error message you need
             to seek^^Jexpert advice.}%
    \errmessage{OOPS! I can't find any hyphenation patterns for
                US english.^^J \space Think of getting some or the
                latex2e setup will never succeed}\@@end}
%</default>
%    \end{macrocode}
% The following example describes the possible contents of a file
% |hyphen.cfg| that will load both US English and German hyphenation
% patterns, making the former the default. 
% It sets |\language| to $0$ for the US patterns and to $1$ for the
% German patterns.
% Then |\language| is set to $0$ to make this the default and the
% default values of |\lefthyphenmin| and |\righthyphenmin| are set.
% \begin{verbatim}
%\language=0
%\input hyphen % (or \input ushyphen1 if the file has been renamed)
%\language=1
%\input ghyph31
%\language=0
%\lefthyphenmin=2
%\righthyphenmin=3
%\endinput
%\end{verbatim}
%
% Another possibility is to use the package |babel|, by Johannes Braams.
% That package is distributed with a suitable |hyphen.cfg| file.
%
%
% \Finale
%
\endinput
