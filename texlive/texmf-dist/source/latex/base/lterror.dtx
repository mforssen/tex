% \iffalse meta-comment
%
% Copyright 1993 1994 1995 1996 1997 1998 1999 2000 2001 2002 2003 2004 2005 2006 2007 2008 2009
% The LaTeX3 Project and any individual authors listed elsewhere
% in this file. 
% 
% This file is part of the LaTeX base system.
% -------------------------------------------
% 
% It may be distributed and/or modified under the
% conditions of the LaTeX Project Public License, either version 1.3c
% of this license or (at your option) any later version.
% The latest version of this license is in
%    http://www.latex-project.org/lppl.txt
% and version 1.3c or later is part of all distributions of LaTeX 
% version 2005/12/01 or later.
% 
% This file has the LPPL maintenance status "maintained".
% 
% The list of all files belonging to the LaTeX base distribution is
% given in the file `manifest.txt'. See also `legal.txt' for additional
% information.
% 
% The list of derived (unpacked) files belonging to the distribution 
% and covered by LPPL is defined by the unpacking scripts (with 
% extension .ins) which are part of the distribution.
% 
% \fi
%
% \iffalse
%%% From File: lterror.dtx
%<def>\ProvidesFile{autoerr.sty}
%<def>       [1995/11/28 v1.2l LaTeX error autoload file]
%<*driver>
% \fi
\ProvidesFile{lterror.dtx}
             [1998/05/28 v1.2n LaTeX Kernel (errors)]
% \iffalse
\documentclass{ltxdoc}
\GetFileInfo{lterror.dtx}
\title{\filename}
\date{\filedate}
 \author{%
  Johannes Braams\and
  David Carlisle\and
  Alan Jeffrey\and
  Leslie Lamport\and
  Frank Mittelbach\and
  Chris Rowley\and
  Rainer Sch\"opf}
\begin{document}
\maketitle
 \DocInput{\filename}
\end{document}
%</driver>
% \fi
%
% \CheckSum{442}
%
% \changes{v1.1a}{1994/05/16}{(ASAJ) Split from ltinit.dtx.}
% \changes{v1.1a}{1994/05/16}{(ASAJ) Completely new error interface.}
% \changes{v1.1b}{1994/05/17}{(ASAJ) Moved error stuff from
%    ltdefns.dtx.}
% \changes{v1.1c}{1994/05/20}{(ASAJ) Added \cs{@latex@info@no@line}.}
% \changes{v1.1c}{1994/05/20}{(ASAJ) Added missing full stops.}
% \changes{v1.1c}{1994/05/20}{(ASAJ) Fixed a bug with
%    \cs{@inmatherr}.}
% \changes{v1.1d}{1994/05/21}{(ASAJ) Made the error commands robust.}
% \changes{v1.1e}{1994/05/22}{(ASAJ) Replaced bgroup by begingroup in
%    error messages, to stop extra mathords creeping into math mode.}
% \changes{v1.2a}{1994/05/22}{(ASAJ) Replaced \cs{@generic@message} and 
%    \cs{@generic@error} by \cs{GenericError}, \cs{GenericWarning} and 
%    \cs{GenericInfo}.}
% \changes{v1.2a}{1994/05/22}{(ASAJ) Made \cs{GenericError}, 
%    \cs{GenericWarning} and \cs{GenericInfo} robust.}
% \changes{v1.2a}{1994/05/22}{(ASAJ) Replaced \cmd\\ and tilde by 
%    \cs{MessageBreak} and \cs{space}.}
% \changes{v1.2a}{1994/05/22}{(ASAJ) Replaces \cs{string} by 
%    \cs{protect} in some messages.}
% \changes{v1.2f}{1994/05/24}{(DPC) wrap long lines}
% \changes{v1.2h}{1994/11/17}
%         {\cs{@tempa} to \cs{reserved@a}}
% \changes{v1.2i}{1995/09/01}
%         {Add autoload support}
% \changes{v1.2j}{1995/10/09}
%         {Modify autoload support}
%
% \section{Error handling}
%
% This section defines \LaTeX's error commands.
%
% \StopEventually{}
%
%
% The `2ekernel' code ensures that a |\usepackage{autoerr}| is
% essentially ignored if a `full' format is being used that has
% the error messages already in the format.
%    \begin{macrocode}
%<2ekernel>\expandafter\let\csname ver@autoerr.sty\endcsname\fmtversion
%    \end{macrocode}
%
%    \begin{macrocode}
%<*2ekernel|autoload>
%    \end{macrocode}
%
% \subsection{General commands}
%
% \begin{macro}{\MessageBreak}
%    This command prints a new-line inside a message, followed by a 
%    continuation line begun with |\@msg@continuation|.  Normally it is 
%    defined to be |\relax|, but inside messages, it is let to 
%    |\@message@break|.
%    \begin{macrocode}
\let\MessageBreak\relax
%    \end{macrocode}
% \end{macro}
%
% \changes{v1.2g}{1994/11/04}{Added \cs{set@display@protect} to
%    \cs{Generic*} commands.  ASAJ.}
%
% \begin{macro}{\GenericInfo}
%    This takes two arguments: a continuation and a message, and sends 
%    the result to the log file.
%    \begin{macrocode}
\DeclareRobustCommand{\GenericInfo}[2]{%
   \begingroup
      \def\MessageBreak{^^J#1}%
      \set@display@protect
      \immediate\write\m@ne{#2\on@line.}%
   \endgroup
}
%    \end{macrocode}
% \end{macro}
%
% \begin{macro}{\GenericWarning}
%    This takes two arguments: a continuation and a message, and sends 
%    the result to the screen.
%    \begin{macrocode}
\DeclareRobustCommand{\GenericWarning}[2]{%
   \begingroup
      \def\MessageBreak{^^J#1}%
      \set@display@protect
      \immediate\write\@unused{^^J#2\on@line.^^J}%
   \endgroup
}
%</2ekernel|autoload>
%    \end{macrocode}
% \end{macro}
%
% \begin{macro}{\GenericError}
% This macro takes four arguments: a continuation,
% an error message, where to go for further information, and the help
% information.  It displays the error message, and sets the error help
% (the result of typing |h| to the prompt), and does a horrible hack
% to turn the last context line (which by default is the only context
% line) into just three dots.  This could be made more efficient.
% \changes{v1.2d}{1994/05/22}
%     {(DPC) New version using long command name.}
% \changes{v1.2d}{1994/05/22}
%     {(DPC) Alternative version added for old TeXs}
%    \begin{macrocode}
%<autoload>\def\GenericError{\@autoerr\GenericError}
%<*2ekernel|def>
\bgroup
\lccode`\@=`\ %
\lccode`\~=`\ %
\lccode`\}=`\ %
\lccode`\{=`\ %
\lccode`\T=`\T%
\lccode`\H=`\H%
\catcode`\ =11\relax%
\lowercase{%
\egroup%
%    \end{macrocode}
%
% Unfortunately \TeX\ versions older than 3.141 have a bug which means
% that |^^J| does not force a linebreak in |\message| and |\errmessage|
% commands. So for these old \TeX's we use |\typeout| to produce the
% message, and then have an empty |\errmessage| command. This causes an
% extra line of the form
%\begin{verbatim}
%! .
%\end{verbatim}
% To appear on the terminal, but if you do not like it, you can always
% upgrade your \TeX! In order for your format to use this version, you
% must define the macro |\@TeXversion| to be the version number, e.g.,
% 3.14 of the underlying \TeX. See the comments in
% \texttt{ltdircheck.dtx}.
%    \begin{macrocode}
\dimen@\ifx\@TeXversion\@undefined4\else\@TeXversion\fi\p@%
\ifdim\dimen@>3.14\p@%
%    \end{macrocode}
%
% First the `standard case'.
%    \begin{macrocode}
\DeclareRobustCommand{\GenericError}[4]{%
\begingroup%
\immediate\write\@unused{}%
\def\MessageBreak{^^J}%
\set@display@protect%
\edef%
%    %<-------------------do not delete this space!------------------->%
\@err@                                                                 %
{{#4}}%
\errhelp
%    %<-------------------do not delete this space!------------------->%
\@err@                                                                 %
\let
%    %<-------------------do not delete this space!------------------->%
\@err@                                                                 %
\@empty
\def\MessageBreak{^^J#1}%
\def~{\errmessage{%
#2.^^J^^J%
#3^^J%
Type  H <return>  for immediate help%
%    %<-------------------do not delete this space!------------------->%
\@err@                                                                 %
}}%
~%
\endgroup}%
%    \end{macrocode}
%
%    \begin{macrocode}
\else%
%    \end{macrocode}
%
% Secondly the version for old \TeX's.
%    \begin{macrocode}
\DeclareRobustCommand{\GenericError}[4]{%
\begingroup%
\immediate\write\@unused{}%
\def\MessageBreak{^^J}%
\set@display@protect%
\edef%
%    %<-------------------do not delete this space!------------------->%
\@err@                                                                 %
{{#4}}%
\errhelp
%    %<-------------------do not delete this space!------------------->%
\@err@                                                                 %
\let
%    %<-------------------do not delete this space!------------------->%
\@err@                                                                 %
\errmessage
\def\MessageBreak{^^J#1}%
\def~{\typeout{! %
#2.^^J^^J%
#3^^J%
Type  H <return>  for immediate help.}%
%    %<-------------------do not delete this space!------------------->%
\@err@                                                                 %
{}}%
~%
\endgroup}%
%    \end{macrocode}
%
%    \begin{macrocode}
\fi}%
%</2ekernel|def>
%    \end{macrocode}
% \end{macro}
%
% \begin{macro}{\PackageError}
% \begin{macro}{\PackageWarning}
% \begin{macro}{\PackageWarningNoLine}
% \begin{macro}{\PackageInfo}
% \begin{macro}{\ClassError}
% \begin{macro}{\ClassWarning}
% \begin{macro}{\ClassWarningNoLine}
% \begin{macro}{\ClassInfo}
%  These commands are intended for use by package and class writers, to
%  give information to authors.  The syntax is:
%    \begin{quote}
%       |\PackageError{|\meta{package}|}{|^^A
%          \meta{error}|}{|\meta{help}|}| \\
%       |\PackageWarning{|\meta{package}|}{|\meta{warning}|}| \\
%       |\PackageWarningNoLine{|\meta{package}|}{|\meta{warning}|}| \\
%       |\PackageInfo{|\meta{package}|}{|\meta{info}|}|
%    \end{quote}
%    and similarly for classes.  The |Error| commands print the
%    \meta{error} message, and present the interactive prompt; if the
%    author types |h|, then the \meta{help} information is displayed.
%    The |Warning| commands produce a warning but do not present the
%    interactive prompt.  The |WarningNoLine| commands do the same,
%    but don't print the input line number.  The |Info| commands write
%    the message to the
%    |log| file.  Within the messages, the command 
%    |\MessageBreak| can be used to
%    break a line, |\protect| can be used to protect command names,
%    and |\space| is a space, for example:
% \begin{verbatim}
%    \newcommand{\foo}{FOO}
%    \PackageWarning{ethel}{%
%       Your hovercraft is full of eels,\MessageBreak
%       and \protect\foo\space is \foo}
% \end{verbatim}
%    produces:
% \begin{verbatim}
%    Package ethel warning: Your hovercraft is full of eels,
%    (ethel)                and \foo is FOO on input line 54.
% \end{verbatim}
%
%    \begin{macrocode}
%<autoload>\def\PackageError{\@autoerr\PackageError}
%<*2ekernel|def>
\gdef\PackageError#1#2#3{%
   \GenericError{%
      (#1)\@spaces\@spaces\@spaces\@spaces
   }{%
      Package #1 Error: #2%
   }{%
      See the #1 package documentation for explanation.%
   }{#3}%
}
%</2ekernel|def>
%    \end{macrocode}
%
%    \begin{macrocode}
%<*2ekernel|autoload>
\def\PackageWarning#1#2{%
   \GenericWarning{%
      (#1)\@spaces\@spaces\@spaces\@spaces
   }{%
      Package #1 Warning: #2%
   }%
}
\def\PackageWarningNoLine#1#2{%
   \PackageWarning{#1}{#2\@gobble}%
}
\def\PackageInfo#1#2{%
   \GenericInfo{%
      (#1) \@spaces\@spaces\@spaces
   }{%
      Package #1 Info: #2%
   }%
}
%</2ekernel|autoload>
%    \end{macrocode}
%
% \changes{v1.2l}{1995/11/28}
%     {Typo in autoload code /1985}
%    \begin{macrocode}
%<autoload>\def\ClassError{\@autoerr\ClassError}
%<*2ekernel|def>
\gdef\ClassError#1#2#3{%
   \GenericError{%
      (#1) \space\@spaces\@spaces\@spaces
   }{%
      Class #1 Error: #2%
   }{%
      See the #1 class documentation for explanation.%
   }{#3}%
}
%</2ekernel|def>
%    \end{macrocode}
%
%    \begin{macrocode}
%<*2ekernel|autoload>
\def\ClassWarning#1#2{%
   \GenericWarning{%
      (#1) \space\@spaces\@spaces\@spaces
   }{%
      Class #1 Warning: #2%
   }%
}
\def\ClassWarningNoLine#1#2{%
   \ClassWarning{#1}{#2\@gobble}%
}
\def\ClassInfo#1#2{%
   \GenericInfo{%
      (#1) \space\space\@spaces\@spaces
   }{%
      Class #1 Info: #2%
   }%
}
%</2ekernel|autoload>
%    \end{macrocode}
% \end{macro}
% \end{macro}
% \end{macro}
% \end{macro}
% \end{macro}
% \end{macro}
% \end{macro}
% \end{macro}
%
% \begin{macro}{\@latex@error}
% \begin{macro}{\@latex@warning}
% \begin{macro}{\@latex@warning@no@line}
% \begin{macro}{\@latex@info}
% \begin{macro}{\@latex@info@no@line}
% \changes{v1.2e}{1994/05/24}{Macro added}
%    Errors and other info, for use in the \LaTeX{} core.
%    \begin{macrocode}
%<autoload>\def\@latex@error{\@autoerr\@latex@error}
%<*2ekernel|def>
\gdef\@latex@error#1#2{%
   \GenericError{%
      \space\space\space\@spaces\@spaces\@spaces
   }{%
      LaTeX Error: #1%
   }{%
      See the LaTeX manual or LaTeX Companion for explanation.%
   }{#2}%
}
%</2ekernel|def>
%    \end{macrocode}
%
%    \begin{macrocode}
%<*2ekernel|autoload>
\def\@latex@warning#1{%
   \GenericWarning{%
      \space\space\space\@spaces\@spaces\@spaces
   }{%
      LaTeX Warning: #1%
   }%
}
%    \end{macrocode}
%
%    \begin{macrocode}
\def\@latex@warning@no@line#1{%
   \@latex@warning{#1\@gobble}}
%    \end{macrocode}
%
%    \begin{macrocode}
\def\@latex@info#1{%
   \GenericInfo{%
      \@spaces\@spaces\@spaces
   }{%
      LaTeX Info: #1%
   }%
}
%    \end{macrocode}
%
%    \begin{macrocode}
\def\@latex@info@no@line#1{%
  \@latex@info{#1\@gobble}}
%    \end{macrocode}
%
%    |\@font@warning| and |\@font@info| are defined later since they
%    have to be redefined by the \texttt{tracefnt} package.
%\begin{verbatim}
%\def\@font@warning#1{%
%   \GenericWarning{%
%      {(font)\@spaces\@spaces}%
%      {Font Warning: #1}%
% }
%\def\@font@info#1{%
%   \GenericInfo{%
%      (font)\space\@spaces
%   }{%
%      Font Info: #1%
%   }%
% }
%\end{verbatim}
% \end{macro}
% \end{macro}
% \end{macro}
% \end{macro}
% \end{macro}
%
% \begin{macro}{\c@errorcontextlines}
% \changes{LaTeX2e}{1993/11/22}{Macro added}
%  |\errorcontextlines| as a \LaTeX\ counter, so that it may be be
%  manipulated with |\setcounter| (once it is defined :-)
%    \begin{macrocode}
\let\c@errorcontextlines\errorcontextlines
\c@errorcontextlines=-1
%    \end{macrocode}
% \end{macro}
%
% \changes{v1.0d}{1994/03/28}
%     {Remove test for \cs{inputlineno} undefined.}
% \begin{macro}{\on@line}
%    The message ` on input line~$n$', if possible.
%    \begin{macrocode}
\ifnum\inputlineno=\m@ne
  \let\on@line\@empty
\else
  \def\on@line{ on input line \the\inputlineno}
\fi
%    \end{macrocode}
% \end{macro}
%
%  \begin{macro}{\@warning}
%  \begin{macro}{\@@warning}
%  \begin{macro}{\@latexerr}
%     Older \LaTeX{} messages.  For the moment, these
%     |\let| to the new message commands.  They may be changed later,
%     once only obsolete packages and classes contain them.
% \changes{v1.0b}{1993/12/03}{Set \cs{c@errorcontextlines} to -1}
% \changes{v1.0e}{1993/04/09}{Mention The Companion}
% \changes{v1.0f}{1993/04/11}{Remove setting of errorcontextlines}
% \changes{v1.0k}{1994/05/01}{(CAR) Added draft \cs{@latexinfo}.}
% \changes{v1.0n}{1994/05/10}{(ASAJ) Added extra blank lines to
%           \cs{@latexerr}.}
% \changes{v1.0o}{1994/05/11}
%     {(ASAJ) Removed one of the extra blank lines to \cs{@latexerr}.}
%    \begin{macrocode}
\let\@warning\@latex@warning
\let\@@warning\@latex@warning@no@line
%</2ekernel|autoload>
\global\let\@latexerr\@latex@error
%    \end{macrocode}
%  \end{macro}
%  \end{macro}
%  \end{macro}
%
% \begin{macro}{\@spaces}
%    Four spaces.
%    \begin{macrocode}
%<*2ekernel|autoload>
\def\@spaces{\space\space\space\space}
%</2ekernel|autoload>
%    \end{macrocode}
% \end{macro}
%
% \subsection{Specific errors}
%
% \begin{macro}{\@eha}
% \begin{macro}{\@ehb}
% \begin{macro}{\@ehc}
% \begin{macro}{\@ehd}
%    The more common error help messages.
%    \begin{macrocode}
%<*2ekernel|def>
\gdef\@eha{%
  Your command was ignored.\MessageBreak
  Type \space I <command> <return> \space to replace it %
  with another command,\MessageBreak
  or \space <return> \space to continue without it.}
\gdef\@ehb{%
  You've lost some text. \space \@ehc}
\gdef\@ehc{%
  Try typing \space <return> %
  \space to proceed.\MessageBreak
  If that doesn't work, type \space X <return> \space to quit.}
\gdef\@ehd{%
  You're in trouble here.  \space\@ehc}
%</2ekernel|def>
%    \end{macrocode}
% As |\latex@error| triggers the autoload, these definitions
% should not be needed in the autoload format, but just to be safe\ldots
%    \begin{macrocode}
%<*autoload>
\let\@eha\@empty\let\@ehb\@empty\let\@ehc\@empty\let\@ehd\@empty
%</autoload>
%    \end{macrocode}
% \end{macro}
% \end{macro}
% \end{macro}
% \end{macro}
%
% Here are most of the error message-generating commands of \LaTeX.
% \begin{macro}{\@autoerr}
% Make this autoload command robust, as it may be read in at
% unpredicatble times.
%    \begin{macrocode}
%<autoload>\def\@autoerr{\protect\@autoload{err}\protect}
%    \end{macrocode}
% \end{macro}
%
% \begin{macro}{\@notdefinable}
%    Error message generated in |\@ifdefinable| from calls
%    to one of the commands |\newcommand|, |\newlength| or |\newtheorem|
%    specifying an already-defined command name or one that begins
%    |\end...|.
% \changes{v1.2n}{1998/05/28}{Added message re `end...' pr/1555}
%    \begin{macrocode}
\gdef\@notdefinable{%
%<!autoload> \@latex@error{%
%<!autoload>   Command \@backslashchar\reserved@a\space
%<!autoload>   already defined.\MessageBreak
%<!autoload>   Or name \@backslashchar\@qend... illegal,
%<!autoload>   see p.192 of the manual}\@eha}
%<autoload>  \@autoerr\@notdefinable}
%    \end{macrocode}
% \end{macro}
%
% \begin{macro}{\@nolnerr}
% Generated by |\newline| and |\\| when called in vertical mode.
%    \begin{macrocode}
\gdef\@nolnerr{%
%<!autoload>  \@latex@error{There's no line here to end}\@eha}
%<autoload>  \@autoerr\@nolnerr}
%    \end{macrocode}
% \end{macro}
%
% \begin{macro}{\@nocounterr}
% \changes{v1.0h}{1994/04/17}{New name for error message,
%       old error message (without arg) kept}
%  Generated by |\setcounter|, |\addtocounter| or
%  |\newcounter| if applied to an undefined counter \meta{cnt}.
%
% \begin{macro}{\@nocnterr}
% Obsolete error message generated in \LaTeX2.09 by
% |\setcounter|, |\addtocounter| or |\newcounter|
% for undefined counter.
% DO NOT use for \LaTeXe\ it MIGHT vanish!
% Use |\@nocounterr|\marg{cnt} instead.
%    \begin{macrocode}
\gdef\@nocounterr#1{%
%<!autoload>  \@latex@error{No counter '#1' defined}\@eha}
%<autoload>  \@autoerr\@nocounterr}
\gdef\@nocnterr{\@nocounterr?}
%    \end{macrocode}
% \end{macro}
% \end{macro}
%
% \begin{macro}{\@ctrerr}
% Called when trying to print the value of a counter
% numbered by letters that's greater than 26.
%    \begin{macrocode}
\gdef\@ctrerr{%
%<!autoload>  \@latex@error{Counter too large}\@ehb}
%<autoload>  \@autoerr\@ctrerr}
%    \end{macrocode}
% \end{macro}
%
% \begin{macro}{\@nodocument}
% Error produced if paragraphs are typeset in the preamble.
% \changes{v1.2m}{1996/11/04}{Always define \cs{@nodocument}
%           in kernel, so that it can be cleared by \cs{document}.}
%    \begin{macrocode}
%<!def>\gdef\@nodocument{%
%<!def>  \@latex@error{Missing \protect\begin{document}}\@ehd}
%    \end{macrocode}
% \end{macro}
%
% \begin{macro}{\@badend}
% Called by |\end| that doesn't match its |\begin|.
% RmS 1992/08/24: added code to |\@badend| to display position of
%               non-matching |\begin|.
% FMi 1993/01/14: missing space added.
%    \begin{macrocode}
\gdef\@badend#1{%
%<!autoload>  \@latex@error{\protect\begin{\@currenvir}\@currenvline
%<!autoload>                     \space ended by \protect\end{#1}}\@eha}
%<autoload>  \@autoerr\@badend}
%    \end{macrocode}
% \end{macro}
%
% \begin{macro}{\@badmath}
% Called by |\[|, |\]|, |\(| or |\)| when used in wrong mode.
%    \begin{macrocode}
\gdef\@badmath{%
%<!autoload>  \@latex@error{Bad math environment delimiter}\@eha}
%<autoload>  \@autoerr\@badmath}
%    \end{macrocode}
% \end{macro}
%
% \begin{macro}{\@toodeep}
% Called by a list environment nested more than six levels
% deep, or an enumerate or itemize nested more than four levels.
%    \begin{macrocode}
\gdef\@toodeep{%
%<!autoload>  \@latex@error{Too deeply nested}\@ehd}
%<autoload>  \@autoerr\@toodeep}
%    \end{macrocode}
% \end{macro}
%
% \begin{macro}{\@badpoptabs}
% Called by |\endtabbing| when not enough |\poptabs| have
% occurred, or by |\poptabs| when too many have occurred.
%    \begin{macrocode}
\gdef\@badpoptabs{%
%<!autoload>  \@latex@error{\protect\pushtabs\space and \protect\poptabs
%<!autoload>      \space don't match}\@ehd}
%<autoload>  \@autoerr\@badpoptabs}
%    \end{macrocode}
% \end{macro}
%
% \begin{macro}{\@badtab}
% Called by |\>|, |\+| , |\-| or |\<| when stepping to an undefined tab.
%    \begin{macrocode}
\gdef\@badtab{%
%<!autoload> \@latex@error{Undefined tab position}\@ehd}
%<autoload>  \@autoerr\@badtab}
%    \end{macrocode}
% \end{macro}
%
% \begin{macro}{\@preamerr}
% \changes{v1.2k}{1995/10/24}
%         {Modify autoload support}
%    This error is special: it appears in places where we normally have
%    to |\protect| expansions. However, to prevent a protection of
%    the error message itself (which would result in the message
%    getting printed not issued on the terminal) we need to locally
%    reset |\protect| to |\relax|.
%    \begin{macrocode}
\gdef\@preamerr#1{%
  \begingroup
    \let\protect\relax
%<*!autoload>
    \@latex@error{\ifcase #1 Illegal character\or
     Missing @-exp\or Missing p-arg\fi\space
     in array arg}\@ehd
%</!autoload>
%<autoload>  \@autoerr\@preamerr{#1}%
  \endgroup}
%    \end{macrocode}
% \end{macro}
%
% \begin{macro}{\@badlinearg}
% Occurs in |\line| and |\vector| command when a bad slope
% argument is encountered.
%    \begin{macrocode}
\gdef\@badlinearg{%
%<!autoload>  \@latex@error{%
%<!autoload>       Bad \protect\line\space or \protect\vector 
%<!autoload>       \space argument}\@ehb}
%<autoload>  \@autoerr\@badlinearg}
%    \end{macrocode}
% \end{macro}
%
% \begin{macro}{\@parmoderr}
% Occurs in a float environment or a |\marginpar| when
% encountered in inner vertical mode.
%    \begin{macrocode}
\gdef\@parmoderr{%
%<!autoload>  \@latex@error{Not in outer par mode}\@ehb}
%<autoload>  \@autoerr\@parmoderr}
%    \end{macrocode}
% \end{macro}
%
% \begin{macro}{\@fltovf}
% Occurs in float environment or |\marginpar| when there
% are no more free boxes for storing floats.
%    \begin{macrocode}
\gdef\@fltovf{%
%<!autoload>  \@latex@error{Too many unprocessed floats}\@ehb}
%<autoload>  \@autoerr\@fltovf}
%    \end{macrocode}
% \end{macro}
%
% \begin{macro}{\@latexbug}
% Occurs in output routine.  This is bad news.
%    \begin{macrocode}
\gdef\@latexbug{%
%<!autoload>  \@latex@error{This may be a LaTeX bug}{Call for help}}
%<autoload>  \@autoerr\@latexbug}
%    \end{macrocode}
% \end{macro}
%
% \begin{macro}{\@badcrerr}
%    This error was removed and replaced by |\@nolnerr|.
% \changes{v1.0m}{1994/05/04}{Error message removed}
%    \begin{macrocode}
%\def\@badcrerr {\@latex@error{Bad use of \protect\\}\@ehc}
%    \end{macrocode}
% \end{macro}
%
% \begin{macro}{\@noitemerr}
% |\addvspace| or |\addpenalty| was called when not in
%  vmode. Probably caused by a missing |\item|.
%    \begin{macrocode}
\gdef\@noitemerr{%
%<!autoload>  \@latex@error{Something's wrong--perhaps a missing %
%<!autoload>      \protect\item}\@ehc}
%<autoload>  \@autoerr\@noitemerr}
%    \end{macrocode}
% \end{macro}
%
% \begin{macro}{\@notprerr}
% A command that can be used only in the preamble
% appears after the command |\begin{document}|.
%    \begin{macrocode}
\gdef\@notprerr{%
%<!autoload>  \@latex@error{Can be used only in preamble}\@eha}
%<autoload>  \@autoerr\@notprerr}
%    \end{macrocode}
% \end{macro}
%
%  \begin{macro}{\@inmatherr}
% \changes{v1.0j}{1994/04/28}{Macro added}
% \changes{v1.1c}{1994/04/28}{Replaced \cs{noexpand} with \cs{protect}.}
%    Issued by commands that don't work correctly within math (like
%    |\item|). There is no real error recovery happening, e.g., the
%    user might get additional errors afterwards.
%    \begin{macrocode}
\gdef\@inmatherr#1{%
   \relax
   \ifmmode
%<!autoload> \@latex@error{Command \protect#1 invalid in math mode}\@ehc
%<autoload>  \@autoerr\@inmatherr#1%
   \fi}
%    \end{macrocode}
%  \end{macro}
%
% \begin{macro}{\@invalidchar}
% \changes{LaTeX2.09}{1993/09/19}
%     {(RmS) Error message for invalid input characters.}
% \changes{v1.0d}{1994/03/28}
%     {(DPC) Comment out (use catcode15 instead)}
%    An error for use with invalid characters.  This is commented
%    out, since we decided to use chatcode 15 instead.
%    \begin{macrocode}
%\def\@invalidchar{\@latex@error{Invalid character in input}\@ehc}
%    \end{macrocode}
% \end{macro}
% 
% As well as the above error commands some error messages are directly
% coded to save space. The Messages alrerady present in \LaTeX2.09
% included: 
%
% |Environment --- undefined|\\
%  Issued by |\begin| for undefined environment.
%
% |tab overflow|\\
%  Occurs in |\= when| maximum number of tabs exceeded.
%
% |\< in mid line|\\
% Occurs in |\<| when it appears in middle of line.
%
% |Float(s) lost|\\
% In output routine, caused by a float environment or
% |\marginpar| occurring in inner vertical mode.
%
%\Finale
\endinput
